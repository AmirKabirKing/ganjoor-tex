\begin{center}
\section*{غزل شماره ۶۲۷: نشستی در دل من چونت جویم}
\label{sec:627}
\addcontentsline{toc}{section}{\nameref{sec:627}}
\begin{longtable}{l p{0.5cm} r}
نشستی در دل من چونت جویم
&&
دلم خون شد مگر در خونت جویم
\\
تو با من در درون جان نشسته
&&
من از هر دو جهان بیرونت جویم
\\
چو فردا گم نخواهی بود جاوید
&&
پس آن بهتر بود کاکنونت جویم
\\
مرا گویی چو گم گردی مرا جوی
&&
چو بی چونی تو آخر چونت جویم
\\
چو راهت را نه سر پیداست نه پای
&&
نه سر نه پای چون گردونت جویم
\\
یقین دانم که در دستم کم آیی
&&
اگرچه هر زمان افزونت جویم
\\
چو در دستم نمی‌آیی ز یک وجه
&&
از آن هر روز دیگرگونت جویم
\\
چو هر دم می‌کنی صد رنگ ظاهر
&&
سزد گر همچو بوقلمونت جویم
\\
نیایی ذره‌ای در دست هرگز
&&
اگر هر دم به صد افسونت جویم
\\
نمیرم تا ابد گر درد خود را
&&
مفرح از لب میگونت جویم
\\
چو دریا گشت چشم من ز شوقت
&&
چگونه لؤلؤ مکنونت جویم
\\
شکر ریز فریدم می نباید
&&
شکر از خندهٔ موزونت جویم
\\
\end{longtable}
\end{center}
