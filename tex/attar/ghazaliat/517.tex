\begin{center}
\section*{غزل شماره ۵۱۷: دل رفت وز جان خبر ندارم}
\label{sec:517}
\addcontentsline{toc}{section}{\nameref{sec:517}}
\begin{longtable}{l p{0.5cm} r}
دل رفت وز جان خبر ندارم
&&
این بود سخن دگر ندارم
\\
گرچه شده‌ام چو موی بی او
&&
یک موی ازو خبر ندارم
\\
همچون گویم که در ره او
&&
دارم سر او و سر ندارم
\\
هم بی خبرم ز کار هر دم
&&
هم یک دم کارگر ندارم
\\
راه است بدو ز ذره ذره
&&
من دیدهٔ راهبر ندارم
\\
خورشید همه جهان گرفته است
&&
من سوخته دل نظر ندارم
\\
چندان که روم به نیستی در
&&
از هستی او گذر ندارم
\\
فریاد که زیر پرده مردم
&&
افسوس که پرده در ندارم
\\
گرچه همه چیزها بدیدم
&&
جز نام ز نامور ندارم
\\
زان چیز که اصل چیزها اوست
&&
مویی خبر و اثر ندارم
\\
دردا که شدم به خاک و در دست
&&
جز باد ز خشک و تر ندارم
\\
فی‌الجمله نصیبه‌ای که بایست
&&
گر دارم ازو وگر ندارم
\\
افسانهٔ عشق او شدم من
&&
وافسانه جزین ز بر ندارم
\\
با این همه ناامیدی عشق
&&
دل از غم عشق بر ندارم
\\
سیمرغ جهانم و چو عطار
&&
یک مرغ به زیر پر ندارم
\\
\end{longtable}
\end{center}
