\begin{center}
\section*{غزل شماره ۳۶۲: آن روی به جز قمر که آراید}
\label{sec:362}
\addcontentsline{toc}{section}{\nameref{sec:362}}
\begin{longtable}{l p{0.5cm} r}
آن روی به جز قمر که آراید
&&
وان لعل به جز شکر که فرساید
\\
بس جان که ز پرده در جهان افتد
&&
چون روی ز زیر پرده بنماید
\\
در زیبایی و عالم افروزی
&&
رویی دارد چنان که می‌باید
\\
خورشید چو روی او همی بیند
&&
می‌گردد و پشت دست می‌خاید
\\
امروز قیامتی است از خطش
&&
خطی که هزار فتنه می‌زاید
\\
گویی ز بنفشه گلستانش را
&&
مشاطهٔ حسن می‌بیاراید
\\
آورد خطی و دل ببرد از من
&&
جان منتظر است تا چه فرماید
\\
زین بیع و شری که خط او دارد
&&
جز خون جگر مرا چه بگشاید
\\
الحق ز معاملان خط او
&&
دیری است که بوی مشک می‌آید
\\
زین گونه که خط او درآبم زد
&&
شک نیست که دوستی بیفزاید
\\
عطار اگر چنین کند سودا
&&
چه سود چو جان او نیاساید
\\
\end{longtable}
\end{center}
