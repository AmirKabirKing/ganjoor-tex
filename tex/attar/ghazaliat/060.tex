\begin{center}
\section*{غزل شماره ۶۰: عشق را گوهر ز کانی دیگر است}
\label{sec:060}
\addcontentsline{toc}{section}{\nameref{sec:060}}
\begin{longtable}{l p{0.5cm} r}
عشق را گوهر ز کانی دیگر است
&&
مرغ عشق از آشیانی دیگر است
\\
هرکه با جان عشق بازد این خطاست
&&
عشق بازیدن ز جانی دیگر است
\\
عاشقی بس خوش جهانی است ای پسر
&&
وان جهان را آسمانی دیگر است
\\
کی کند عاشق نگاهی در جهان
&&
زانکه عاشق را جهانی دیگر است
\\
در نیابد کس زبان عاشقان
&&
زانکه عاشق را زبانی دیگر است
\\
کس نداند مرد عاشق را ولیک
&&
هر گروهی را گمانی دیگر است
\\
نیست عاشق را به یک موضع قرار
&&
هر زمانی در مکانی دیگر است
\\
نی خطا گفتم برون است از مکان
&&
لامکان او را نشانی دیگر است
\\
گرچه عاشق خود در اینجا در میان است
&&
جای دیگر در میانی دیگر است
\\
جوهر عطار در سودای عشق
&&
گویی از بحری و کانی دیگر است
\\
\end{longtable}
\end{center}
