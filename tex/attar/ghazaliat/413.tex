\begin{center}
\section*{غزل شماره ۴۱۳: عمر رفت و تو منی داری هنوز}
\label{sec:413}
\addcontentsline{toc}{section}{\nameref{sec:413}}
\begin{longtable}{l p{0.5cm} r}
عمر رفت و تو منی داری هنوز
&&
راه بر ناایمنی داری هنوز
\\
زخم کاید بر منی آید همه
&&
تا تو می‌رنجی منی داری هنوز
\\
صد منی می‌زاید از تو هر نفس
&&
وی عجب آبستنی داری هنوز
\\
پیر گشتی و بسی کردی سلوک
&&
طبع رند گلخنی داری هنوز
\\
همرهان رفتند و یاران گم شدند
&&
همچنان تو ساکنی داری هنوز
\\
روز و شب در پرده با چندین ملک
&&
عادت اهریمنی داری هنوز
\\
روی گردانیده‌ای از تیرگی
&&
پشت سوی روشنی داری هنوز
\\
دلبرت در دوستی کی ره دهد
&&
چون دلی پر دشمنی داری هنوز
\\
می‌زنی دم از پی معنی ولیک
&&
تو کجا آن چاشنی داری هنوز
\\
در گریبان کش سر و بنشین خموش
&&
چون بسی تر دامنی داری هنوز
\\
خویشتن را می‌کش و می‌کش بلا
&&
زانکه نفس کشتنی داری هنوز
\\
رهبری چون آید از تو ای فرید
&&
چون تو عزم رهزنی داری هنوز
\\
\end{longtable}
\end{center}
