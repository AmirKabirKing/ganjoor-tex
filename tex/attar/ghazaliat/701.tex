\begin{center}
\section*{غزل شماره ۷۰۱: ذره‌ای نادیده گنج روی تو}
\label{sec:701}
\addcontentsline{toc}{section}{\nameref{sec:701}}
\begin{longtable}{l p{0.5cm} r}
ذره‌ای نادیده گنج روی تو
&&
ره بزد بر ما طلسم موی تو
\\
گشت رویم چون نگارستان ز اشک
&&
ای نگارستان جانم روی تو
\\
هست خورشید رخت زیر نقاب
&&
جملهٔ ذرات چشماروی تو
\\
در درون چون نافهٔ آهوی حسن
&&
خون جان‌ها مشک شد بر بوی تو
\\
شیر گردون جامه می‌پوشد کبود
&&
از سواد چشم چون آهوی تو
\\
آسمان را چون زمین در حقه کرد
&&
آرزوی حقهٔ للی تو
\\
هندویم هندوی زلفت را به جان
&&
گر توان شد هندوی هندوی تو
\\
چون ز چشمت تیرباران در رسید
&&
طاق افتادیم از ابروی تو
\\
نی که بنمودیم صد سحر حلال
&&
در صفات نرگس جادوی تو
\\
خاک خواهم گشت تا بادی مرا
&&
بو که برساند به خاک کوی تو
\\
نی ز چون من خاک گردی از درت
&&
گر مرا بادی رساند سوی تو
\\
چون کند از توکسی پهلو تهی
&&
چون همی هستند در پهلوی تو
\\
از کمان عشق بگریز ای فرید
&&
کین کمانی نیست بر بازوی تو
\\
\end{longtable}
\end{center}
