\begin{center}
\section*{غزل شماره ۴۳۱: ای پیر مناجاتی رختت به قلندر کش}
\label{sec:431}
\addcontentsline{toc}{section}{\nameref{sec:431}}
\begin{longtable}{l p{0.5cm} r}
ای پیر مناجاتی رختت به قلندر کش
&&
دل از دو جهان برکن دردی ببر اندر کش
\\
یا چون زن کم‌دان شو یا محرم مردان شو
&&
یا در صف رندان شو یا خرقه ز سر برکش
\\
چون فتنهٔ آن ماهی چون رهرو این راهی
&&
بار غم اگر خواهی از کون فزون تر کش
\\
خمار و قلندر شو مست می دلبر شو
&&
ور گفت که کافر شو هان تا نشوی سرکش
\\
چون کافر اوباشی هرچند ز اوباشی
&&
با دوست به قلاشی هم دست کنی درکش
\\
گفتی که به عشق اندر گر کشته شوی بهتر
&&
اینک من و اینک سر فرمان بر و خنجر کش
\\
ای دلبر سیمین‌بر گفتی که نداری زر
&&
بی زر نبود دلبر از جان بگذر زر کش
\\
عطار که سیم آرد بر روی چو زر بازد
&&
چون صفوت دین دارد گو درد قلندر کش
\\
\end{longtable}
\end{center}
