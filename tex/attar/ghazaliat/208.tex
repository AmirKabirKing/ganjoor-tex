\begin{center}
\section*{غزل شماره ۲۰۸: عزم خرابات بی‌قنا نتوان کرد}
\label{sec:208}
\addcontentsline{toc}{section}{\nameref{sec:208}}
\begin{longtable}{l p{0.5cm} r}
عزم خرابات بی‌قنا نتوان کرد
&&
دست به یک درد بی صفا نتوان کرد
\\
چون نه وجود است نه عدم به خرابات
&&
لاجرم این یک از آن جدا نتوان کرد
\\
شاه مباش و گدا مباش که آنجا
&&
هیچ نشان شه و گدا نتوان کرد
\\
گم شدن و بیخودی است راه خرابات
&&
توشهٔ این راه جز فنا نتوان کرد
\\
هر که ز خود محو گشت در بن این دیر
&&
وعدهٔ اثبات او وفا نتوان کرد
\\
سایه که در قرص آفتاب فرو شد
&&
تا به ابد چارهٔ بقا نتوان کرد
\\
لا شو اگر عزم می‌کنی تو به بالا
&&
زانکه چنین عزم جز به لا نتوان کرد
\\
گر قدری عمر بی‌حضور کنی فوت
&&
تا به ابد آن قدر قضا نتوان کرد
\\
خود قدری نیست این قدر که جهان است
&&
ترک جهانی به یک خطا نتوان کرد
\\
گر ز خرابات درد قسم تو آید
&&
تا ابد الابدش دوا نتوان کرد
\\
چون به خرابات حاجت تو حضور است
&&
حاجت تو بی میی روا نتوان کرد
\\
یار عزیز است خاصه یار خرابات
&&
در حق یاری چنین ریا نتوان کرد
\\
هم نفسی دردکش اگر به کف آری
&&
دامن او یک نفس رها نتوان کرد
\\
تا که نگردد فرید درد کش دیر
&&
قصه دردی کشان ادا نتوان کرد
\\
\end{longtable}
\end{center}
