\begin{center}
\section*{غزل شماره ۶۰۶: چه مقصود ار چه بسیاری دویدیم}
\label{sec:606}
\addcontentsline{toc}{section}{\nameref{sec:606}}
\begin{longtable}{l p{0.5cm} r}
چه مقصود ار چه بسیاری دویدیم
&&
که از مقصود خود بویی ندیدیم
\\
بسی زاری و دلتنگی نمودیم
&&
بسی خواری و بی برگی کشیدیم
\\
بسی در گفتگوی دوست بودیم
&&
بسی در جستجویش ره بریدیم
\\
گهی سجاده و محراب جستیم
&&
گهی رندی و قلاشی گزیدیم
\\
به هر ره کان کسی گیرد گرفتیم
&&
به هر پر کان کسی پرد پریدیم
\\
چو عشق او جهان بفروخت بر ما
&&
به جان و دل غم عشقش خریدیم
\\
مگر معشوق ما با ماست زیرا
&&
ز نور حضرت او ناپدیدیم
\\
به دست ما به جز باد هوا نیست
&&
که چون بادی به عالم بر وزیدیم
\\
درین حیرت همی بودیم عمری
&&
درین محنت به خون بر می‌تپیدیم
\\
کنون رفتیم و عمر ما به سر شد
&&
کنون این ره به پایان آوریدیم
\\
دریغا کز سگ کویش نشانی
&&
ندیدیم ار چه بسیاری دویدیم
\\
بسی بر بوی او بودیم و بویی
&&
به ما نرسید و ما از غم رسیدیم
\\
چو مقصودی نبود از هرچه گفتیم
&&
میان خاک تاریک آرمیدیم
\\
کنون عطار را بدرود کردیم
&&
کنون امید ازین عالم بریدیم
\\
\end{longtable}
\end{center}
