\begin{center}
\section*{غزل شماره ۳۴۸: زلف را چون به قصد تاب دهد}
\label{sec:348}
\addcontentsline{toc}{section}{\nameref{sec:348}}
\begin{longtable}{l p{0.5cm} r}
زلف را چون به قصد تاب دهد
&&
کفر را سر به مهر آب دهد
\\
باز چون درکشد نقاب از روی
&&
همه کفار را جواب دهد
\\
چون درآید به جلوه ماه رخش
&&
تاب در جان آفتاب دهد
\\
تیر چشمش که کم خطا کرده است
&&
مالش عاشقان صواب دهد
\\
همه خامان بی حقیقت را
&&
سر زلفش هزار تاب دهد
\\
تشنگان را که خار هجر نهاد
&&
لب گلرنگ او شراب دهد
\\
غم او زان چنین قوی افتاد
&&
که دلم دایمش کباب دهد
\\
گاه شعرم بدو شکر ریزد
&&
گاه چشمم بدو گلاب دهد
\\
گر دلم می‌دهد غمش را جای
&&
گنج را جایگه خراب دهد
\\
دل به جان باز می‌نهد غم او
&&
تا درین دردش انقلاب دهد
\\
دل عطار چون ز دست بشد
&&
چکند تن در اضطراب دهد
\\
\end{longtable}
\end{center}
