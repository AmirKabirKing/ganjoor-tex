\begin{center}
\section*{غزل شماره ۳۶۴: دو جهان بی‌توام نمی‌باید}
\label{sec:364}
\addcontentsline{toc}{section}{\nameref{sec:364}}
\begin{longtable}{l p{0.5cm} r}
دو جهان بی‌توام نمی‌باید
&&
نه یکی بس دو ام نمی‌باید
\\
هرچه خواهم ز تو تو به زانی
&&
از توام جز توام نمی‌باید
\\
قبله‌ام چون جمال روی تو بس
&&
رویی از هر سوم نمی‌باید
\\
جان من چون بشنید قول الست
&&
این تن بدخوم نمی‌باید
\\
بسم از هر دو کون قول قدیم
&&
اجتهادی نوم نمی‌باید
\\
گرچه مویی شدم ز شوق رخت
&&
قوت بازوم نمی‌باید
\\
ضعف من چون ز اشتیاق تو خاست
&&
ذره‌ای نیروم نمی‌باید
\\
چون چنین در ره توام عاجز
&&
هیچ بیرون شوم نمی‌باید
\\
گرچه دردم گذشت از اندازه
&&
زحمت داروم نمی‌باید
\\
صد گره هست از تو بر کارم
&&
گره ابروم نمی‌باید
\\
پیچ پیچی برون بر از کارم
&&
که دل صد توم نمی‌باید
\\
ور نخواهی گشاد در بر من
&&
هیچ هم زانوم نمی‌باید
\\
چون به تو راه نیست محوم کن
&&
که بدو و نیکوم نمی‌باید
\\
شیر مردی اگر به من نرسید
&&
سگ در پهلوم نمی‌باید
\\
نفس جادوم کوه کرد بسی
&&
توبهٔ جادوم نمی‌باید
\\
ای فرید از بهانه دست بدار
&&
ترکی هندوم نمی‌باید
\\
\end{longtable}
\end{center}
