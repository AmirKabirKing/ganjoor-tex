\begin{center}
\section*{غزل شماره ۳۱۰: گر فلک دیده بر آن چهرهٔ زیبا فکند}
\label{sec:310}
\addcontentsline{toc}{section}{\nameref{sec:310}}
\begin{longtable}{l p{0.5cm} r}
گر فلک دیده بر آن چهرهٔ زیبا فکند
&&
ماه را موی کشان کرده به صحرا فکند
\\
هر شبی زان بگشاید فلک این چندین چشم
&&
بو که یک چشم بر آن طلعت زیبا فکند
\\
همچو پروانه به نظارهٔ او شمع سپهر
&&
پر زنان خویش برین گلشن خضرا فکند
\\
خاک او زان شده‌ام تا چو میی نوش کند
&&
جرعه‌ای بوی لبش یافته بر ما فکند
\\
چون دل سوخته اندر سر زلفش بستم
&&
هر دم از دست بیندازد و در پا فکند
\\
زلف در پای چرا می‌فکند زانکه کمند
&&
شرط آن است که از زیر به بالا فکند
\\
غمش از صومعه عطار جگر سوخته را
&&
هر نفس نعره‌زنان بر سر غوغا فکند
\\
\end{longtable}
\end{center}
