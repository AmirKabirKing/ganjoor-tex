\begin{center}
\section*{غزل شماره ۳۵۸: گر نه از خاک درت باد صبا می‌آید}
\label{sec:358}
\addcontentsline{toc}{section}{\nameref{sec:358}}
\begin{longtable}{l p{0.5cm} r}
گر نه از خاک درت باد صبا می‌آید
&&
صبحدم مشک‌فشان پس ز کجا می‌آید
\\
ای جگرسوختگان عهد کهن تازه کنید
&&
که گل تازه به دلداری ما می‌آید
\\
گل تر را ز دم صبح به شام اندازد
&&
این چنین گرم که گلگون صبا می‌آید
\\
به هواداری گل ذره صفت در رقص آی
&&
کم ز ذره نه‌ای او هم ز هوا می‌آید
\\
تا گذر کرد نسیم سحری بر در دوست
&&
نوش‌دارو ز دم زهرگیا می‌آید
\\
عمر و عیش از سر صد ناز و طرب می‌گذرد
&&
بلبل و گل ز سر برگ و نوا می‌آید
\\
بوی بر مشک ختا از دم عطار هوا
&&
زانکه ناکست کزو بوی خطا می‌آید
\\
بلبل شیفته را بی گل تر عمر عزیز
&&
قدری فوت شد از بهر قضا می‌آید
\\
بلبل سوخته را در جگر آب است که نیست
&&
گل سیراب چنین تشنه چرا می‌آید
\\
گل که غنچه به بر از خون دلش پرورده است
&&
از کله‌داری او بسته قبا می‌آید
\\
از بنفشه به عجب مانده‌ام کز چه سبب
&&
روز طفلی به چمن پشت دوتا می‌آید
\\
نسترن کوتهی عمر مگر می‌داند
&&
زان چنین بی سر و بن بر سر پا می‌آید
\\
بر شکر خندهٔ گل درد دل کس نگذاشت
&&
دم عطار کزو بوی دوا می‌آید
\\
\end{longtable}
\end{center}
