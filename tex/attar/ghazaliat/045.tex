\begin{center}
\section*{غزل شماره ۴۵: لعل گلرنگت شکربار آمدست}
\label{sec:045}
\addcontentsline{toc}{section}{\nameref{sec:045}}
\begin{longtable}{l p{0.5cm} r}
لعل گلرنگت شکربار آمدست
&&
قسم من زان گل همه خار آمدست
\\
گو لبت بر من جهان بفروش ازانک
&&
صد جهان جانش خریدار آمدست
\\
پاره دل زانم که در دل دوختن
&&
نرگس تو پاره‌ای کار آمدست
\\
دل نمی‌بینم مگر چون هر دلی
&&
در خم زلفت گرفتار آمدست
\\
پستهٔ شورت نمک دارد بسی
&&
زین سبب گویی جگر خوار آمدست
\\
نی خطا گفتم ز شیرینی که هست
&&
پستهٔ شورت شکربار آمدست
\\
چشمهٔ نور است روی او ولیک
&&
آن دو لب یک دانه نار آمدست
\\
زان شکر لب شور در عالم فتاد
&&
کان شکر لب تلخ گفتار آمدست
\\
چشمه نوشش که چشم سوز نیست
&&
درج لعل در شهوار آمدست
\\
عاشقا روی چو ماه او نگر
&&
کافتابش عاشق زار آمدست
\\
دست بر سر پیش رویش آفتاب
&&
پای کوبان ذره کردار آمدست
\\
بر همه عالم ستم کردست او
&&
با چنان رویی به بازار آمدست
\\
آری آری روشن است این همچو روز
&&
کان سیه گر چون ستمکار آمدست
\\
خون جان ماست آن خون نی شفق
&&
گر سوی مغرب پدیدار آمدست
\\
آنچه در صد سال قسم خلق نیست
&&
بی رخ او قسم عطار آمدست
\\
\end{longtable}
\end{center}
