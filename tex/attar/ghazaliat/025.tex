\begin{center}
\section*{غزل شماره ۲۵: دلبرم در حسن طاق افتاده است}
\label{sec:025}
\addcontentsline{toc}{section}{\nameref{sec:025}}
\begin{longtable}{l p{0.5cm} r}
دلبرم در حسن طاق افتاده است
&&
قسم من زو اشتیاق افتاده است
\\
بر سر پایم چو کرسی ز انتظار
&&
کو چو عرش سیم ساق افتاده است
\\
گر رسد یک شب خیال وصل او
&&
برق در زیرش براق افتاده است
\\
لیک اندر تیه هجرش گرد من
&&
سد اسکندر یتاق افتاده است
\\
کی فتد در دوزخ این آتش کزو
&&
در خراسان و عراق افتاده است
\\
بر هم افتاده چو زلفش هر نفس
&&
کشته تو در فراق افتاده است
\\
می‌ندانم تا به عمدا می‌کشد
&&
یا چنین خود اتفاق افتاده است
\\
تا که روی همچو ماهش دیده‌ام
&&
ماه بختم در محاق افتاده است
\\
ابروی او جز کمان چرخ نیست
&&
زانکه همچون چرخ طاق افتاده است
\\
چون ندارد ترک سیمینم میان
&&
پس چرا زرین نطاق افتاده است
\\
این همه باریک بینی فرید
&&
از میان آن وشاق افتاده است
\\
\end{longtable}
\end{center}
