\begin{center}
\section*{غزل شماره ۴۴: ندای غیب به جان تو می‌رسد پیوست}
\label{sec:044}
\addcontentsline{toc}{section}{\nameref{sec:044}}
\begin{longtable}{l p{0.5cm} r}
ندای غیب به جان تو می‌رسد پیوست
&&
که پای در نه و کوتاه کن ز دنیی دست
\\
هزار بادیه در پیش بیش داری تو
&&
تو این چنین ز شراب غرور ماندی مست
\\
جهان پلی است بدان سوی جه که هر ساعت
&&
پدید آید ازین پل هزار جای شکست
\\
به پل برون نشود با چنین پلی کارت
&&
برو بجه ز چنین پل که نیست جای نشست
\\
چو سیل پل‌شکن از کوه سر فرود آرد
&&
بیوفتد پل و در زیر پل بمانی پست
\\
تو غافلی و به هفتاد پشت شد چو کمان
&&
تو خوش بخفته‌ای و تیر عمر رفت از شست
\\
اگر تو زار بگریی به صد هزاران چشم
&&
ز کار بیهدهٔ خویش جای آنت هست
\\
فرشته‌ای تو و دیوی سرشته در تو به هم
&&
گهی فرشته طلب، گه بمانده دیو پرست
\\
هزار بار به نامرده طوطی جانت
&&
چگونه زین قفس آهنین تواند جست
\\
تو گرچه زنده‌ای امروز لیک در گوری
&&
چون تن به گور فرو رفت جان ز گور برست
\\
چون جان بمرد ازین زندگانی ناخوش
&&
ز خود برید و میان خوشی به حق پیوست
\\
میان جشن بقا کرد نوش نوشش باد
&&
ز دست ساقی جان ساغر شراب الست
\\
دل آن دل است که چون از نهاد خویش گسست
&&
ز کبریای حق اندیشه می‌کند پیوست
\\
به حکم بند قبای فلک ز هم بگشاد
&&
دلی که از کمر معرفت میان در بست
\\
به زیر خاک بسی خواب داری ای عطار
&&
مخسب خیز چو عمر آمدت به نیمهٔ شصت
\\
\end{longtable}
\end{center}
