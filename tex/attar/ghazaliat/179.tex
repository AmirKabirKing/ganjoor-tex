\begin{center}
\section*{غزل شماره ۱۷۹: خطی کان سرو بالا می‌درآرد}
\label{sec:179}
\addcontentsline{toc}{section}{\nameref{sec:179}}
\begin{longtable}{l p{0.5cm} r}
خطی کان سرو بالا می‌درآرد
&&
برای کشتن ما می‌درآرد
\\
به زیبایی گل سرخش به انصاف
&&
خطی سرسبز زیبا می‌درآرد
\\
بگرد روی همچون ماه گویی
&&
هلالی عنبرآسا می‌درآرد
\\
پری رویا کنون منشور حسنت
&&
ز خط سبز طغرا می‌درآرد
\\
ازین پس با تو رنگم در نگیرد
&&
که لعلت رنگ مینا می‌درآرد
\\
هر آن رنگی که پنهان می‌سرشتی
&&
کنون روی تو پیدا می‌درآرد
\\
هر آن کشتی که من بر خشک راندم
&&
کنون چشمم به دریا می‌درآرد
\\
به ترکی هندوی زلف تو هر دم
&&
دلی دیگر ز یغما می‌درآرد
\\
سر زلفت که جان ها دخل دارد
&&
چنین دخلی به تنها می‌درآرد
\\
ولی بر پشتی روی چو ماهت
&&
بسا کس را که از پا می‌درآرد
\\
فرید از دست زلفت کی برد سر
&&
که زلفت سر به غوغا می‌درآرد
\\
\end{longtable}
\end{center}
