\begin{center}
\section*{غزل شماره ۱۵۱: چون نظر بر روی جانان اوفتاد}
\label{sec:151}
\addcontentsline{toc}{section}{\nameref{sec:151}}
\begin{longtable}{l p{0.5cm} r}
چون نظر بر روی جانان اوفتاد
&&
آتشی در خرمن جان اوفتاد
\\
روی جان دیگر نبیند تا ابد
&&
هر که او در بند جانان اوفتاد
\\
ذره‌ای خورشید رویش شد پدید
&&
ولوله در جن و انسان اوفتاد
\\
جان انس از شوق او آتش گرفت
&&
پس از آنجا در دل جان اوفتاد
\\
کرد تاوان بی‌رخ او آفتاب
&&
لاجرم در قید تاوان اوفتاد
\\
هر که مویی سرکشید از عشق او
&&
بی سر آنجا چون گریبان اوفتاد
\\
هر کجا نقش نگاری پای بست
&&
تا ابد در دست رضوان اوفتاد
\\
وانکه را رنگی و بویی راه زد
&&
در حجاب سخت خذلان اوفتاد
\\
چون وصالش دانه‌ای بر دام بست
&&
مرغ دل در دام هجران اوفتاد
\\
بی سر و بن دید عاشق راه او
&&
بی سر و بن در بیابان اوفتاد
\\
راز عشقش عالمی بی منتهاست
&&
ظن مبر کین کار آسان اوفتاد
\\
تا به کلی بر نخیزی از دو کون
&&
محرم این راز نتوان اوفتاد
\\
چون رهی بس دور و بس دشوار بود
&&
لاجرم عطار حیران اوفتاد
\\
\end{longtable}
\end{center}
