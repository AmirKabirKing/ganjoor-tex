\begin{center}
\section*{غزل شماره ۱۵۹: دل ز هوای تو یک زمان نشکیبد}
\label{sec:159}
\addcontentsline{toc}{section}{\nameref{sec:159}}
\begin{longtable}{l p{0.5cm} r}
دل ز هوای تو یک زمان نشکیبد
&&
دل چه بود عقل و وهم جان نشکیبد
\\
هر که دلی دارد و نشان تو یابد
&&
از طلب چون تو دلستان نشکیبد
\\
گرچه جهان را بسی کس است شکیبا
&&
هیچ کسی از تو در جهان نشکیبد
\\
ذرهٔ سودای تو که سود جهان است
&&
سود دل آن است کز زیان نشکیبد
\\
گرچه زبان را مجال یاد تو نبود
&&
یک نفس از یاد تو زبان نشکیبد
\\
چون نشکیبد ز آب ماهی بی آب
&&
دیده ز ماه تو همچنان نشکیبد
\\
مردم آبی چشم از آتش عشقت
&&
بی رخت از آب یک زمان نشکیبد
\\
گرچه بنالم ولی نه آن ز تو نالم
&&
ناله کنم زانکه ناتوان نشکیبد
\\
چون نرسد دست من به جز به فغانی
&&
نیست عجب گر ز دل فغان نشکیبد
\\
می‌نشکیبد دمی ز کوی تو عطار
&&
بلبل گویا ز بوستان نشکیبد
\\
\end{longtable}
\end{center}
