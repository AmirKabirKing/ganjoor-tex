\begin{center}
\section*{غزل شماره ۱۷۴: دلی کز عشق او دیوانه گردد}
\label{sec:174}
\addcontentsline{toc}{section}{\nameref{sec:174}}
\begin{longtable}{l p{0.5cm} r}
دلی کز عشق او دیوانه گردد
&&
وجودش با عدم همخانه گردد
\\
رخش شمع است و عقل ار عقل دارد
&&
ز عشق شمع او دیوانه گردد
\\
کسی باید که از آتش نترسد
&&
به گرد شمع چون پروانه گردد
\\
به شکر آنکه زان آتش بسوزد
&&
همه در عالم شکرانه گردد
\\
کسی کو بر وجود خویش لرزد
&&
همان بهتر که در کاشانه گردد
\\
اگر بر جان خود لرزد پیاده
&&
به فرزینی کجا فرزانه گردد
\\
بخیلی کو به یک جو زر بمیرد
&&
چرا گرد مقامرخانه گردد
\\
چو ماهی آشنا جوید درین بحر
&&
بکل از خاکیان بیگانه گردد
\\
چو در دریا فتاد آن خشک نانه
&&
مکن تعجیل تا ترنانه گردد
\\
اگر تو دم زنی از سر این بحر
&&
دل خونابه را پیمانه گردد
\\
بسی افسون کند غواص دریا
&&
که در دم داشتن مردانه گردد
\\
اگر در قعر دریا دم برآرد
&&
همه افسون او افسانه گردد
\\
درین دریا دل پر درد عطار
&&
ندانم مرد گردد یا نگردد
\\
\end{longtable}
\end{center}
