\begin{center}
\section*{غزل شماره ۴۲۴: بیچاره دلم که نرگس مستش}
\label{sec:424}
\addcontentsline{toc}{section}{\nameref{sec:424}}
\begin{longtable}{l p{0.5cm} r}
بیچاره دلم که نرگس مستش
&&
صد توبه به یک کرشمه بشکستش
\\
از شوق رخش چو مست شد چشمش
&&
از من چه عجب اگر شوم مستش
\\
دست‌آویزی شگرف می‌بینم
&&
هفتاد و دو فرقه را خم شستش
\\
خورشید که دست برد در خوبی
&&
نتواند ریخت آب بر دستش
\\
چون ماه که رخش حسن می‌تازد
&&
صد غاشیه‌کش به دلبری هسش
\\
صد جان باید به هر دمم تا من
&&
بر فرق کنم نثار پیوستش
\\
جانا دل من که مرغ دام توست
&&
از دام تو دست کی دهد جستش
\\
عقلی که گره‌گشای خلق آمد
&&
سودای رخ تو رخت بربستش
\\
عطار به تحفه گر فرستد جان
&&
فریاد همی کند که مفرستش
\\
\end{longtable}
\end{center}
