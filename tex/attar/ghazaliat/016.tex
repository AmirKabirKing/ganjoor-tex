\begin{center}
\section*{غزل شماره ۱۶: تا درین زندان فانی زندگانی باشدت}
\label{sec:016}
\addcontentsline{toc}{section}{\nameref{sec:016}}
\begin{longtable}{l p{0.5cm} r}
تا درین زندان فانی زندگانی باشدت
&&
کنج عزلت گیر تا گنج معانی باشدت
\\
این جهان را ترک کن تا چون گذشتی زین جهان
&&
این جهانت گر نباشد آن جهانی باشدت
\\
کام و ناکام این زمان در کام خود درهم شکن
&&
تا به کام خویش فردا کامرانی باشدت
\\
روزکی چندی چو مردان صبر کن در رنج و غم
&&
تا که بعداز رنج گنج شایگانی باشدت
\\
روی خود را زعفرانی کن به بیداری شب
&&
تا به روز حشر روی ارغوانی باشدت
\\
گر به ترک عالم فانی بگویی مردوار
&&
عالم باقی و ذوق جاودانی باشدت
\\
صبحدم درهای دولتخانه‌ها بگشاده‌اند
&&
عرضه کن گر آن زمان راز نهانی باشدت
\\
تا کی از بی حاصلی ای پیرمرد بچه طبع
&&
در هوای نفس مستی و گرانی باشدت
\\
از تن تو کی شود این نفس سگ سیرت برون
&&
تا به صورت خانهٔ تن استخوانی باشدت
\\
گر توانی کشت این سگ را به شمشیر ادب
&&
زان پس ار تو دولتی جویی نشانی باشدت
\\
گر بمیری در میان زندگی عطاروار
&&
چون درآید مرگ عین زندگانی باشدت
\\
\end{longtable}
\end{center}
