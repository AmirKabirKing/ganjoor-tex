\begin{center}
\section*{غزل شماره ۶۶۱: ترسا بچه‌ای ناگه چون دید عیان من}
\label{sec:661}
\addcontentsline{toc}{section}{\nameref{sec:661}}
\begin{longtable}{l p{0.5cm} r}
ترسا بچه‌ای ناگه چون دید عیان من
&&
صد چشمه ز چشم من بارید روان من
\\
دی زاهد دین بودم سجاده نشین بودم
&&
امروز چنان دیدم زنار میان من
\\
سجاده به می داده وز خرقه تبرایی
&&
نه کفر و نه ایمانی درمانده ز جان من
\\
نه بنده نه آزادم نه مدت خود دانم
&&
این است کنون حاصل در بتکده جان من
\\
با دل گفتم ای دل زنهار مشو ترسا
&&
در حال دل خسته بشکست امان من
\\
گفتم که منم ای جان در پرده مسیحایی
&&
صد قوم دگر دیدم سرگشته بسان من
\\
گویند عطاری را چونی تو ز ترسایی
&&
حقا که درون خود کفر است نهان من
\\
\end{longtable}
\end{center}
