\begin{center}
\section*{غزل شماره ۷۴: در دلم تا برق عشق او بجست}
\label{sec:074}
\addcontentsline{toc}{section}{\nameref{sec:074}}
\begin{longtable}{l p{0.5cm} r}
در دلم تا برق عشق او بجست
&&
رونق بازار زهد من شکست
\\
چون مرا می‌دید دل برخاسته
&&
دل ز من بربود و درجانم نشست
\\
خنجر خون‌ریز او خونم بریخت
&&
ناوک سر تیز او جانم بخست
\\
آتش عشقش ز غیرت بر دلم
&&
تاختن آورد همچون شیر مست
\\
بانگ بر من زد که ای ناحق شناس
&&
دل به ما ده چند باشی بت‌پرست
\\
گر سر هستی ما داری تمام
&&
در ره ما نیست گردان هرچه هست
\\
هر که او در هستی ما نیست شد
&&
دایم از ننگ وجود خویش رست
\\
می‌ندانی کز چه ماندی در حجاب
&&
پردهٔ هستی تو ره بر تو بست
\\
مرغ دل چون واقف اسرار گشت
&&
می‌طپید از شوق چون ماهی بشست
\\
بر امید این گهر در بحر عشق
&&
غرقه شد وان گوهرش نامد به دست
\\
آخر این نومیدی ای عطار چیست
&&
تو نه ای مردانه همتای تو هست
\\
\end{longtable}
\end{center}
