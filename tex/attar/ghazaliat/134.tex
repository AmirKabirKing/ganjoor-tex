\begin{center}
\section*{غزل شماره ۱۳۴: بس که دل تشنه سوخت وز لبت آبی نیافت}
\label{sec:134}
\addcontentsline{toc}{section}{\nameref{sec:134}}
\begin{longtable}{l p{0.5cm} r}
بس که دل تشنه سوخت وز لبت آبی نیافت
&&
مست می عشق شد و از تو شرابی نیافت
\\
داشتم امید آنک بو که در آیی به خواب
&&
عمر شد و دل ز هجر خون شد و خوابی نیافت
\\
تشنهٔ وصل تو دل چون به درت کرد روی
&&
ماند به در حلقه‌وار وز درت آبی نیافت
\\
دل ز تو بیهوش شد دیده برو زد گلاب
&&
زانکه به از آب چشم دیده گلابی نیافت
\\
چند زند بر نمک یار دلم گوییا
&&
به ز دل عاشقان هیچ کبابی نیافت
\\
دل چو ز نومیدیت زود فرو شد به خود
&&
خود ز میان برگرفت هیچ نقابی نیافت
\\
گفتمش آخر چه شد کین دل من روز و شب
&&
سوی تو آواز داد وز تو خطابی نیافت
\\
گفت مرا خوانده‌ای لیک نه از جان و دل
&&
هر که ز جانم نخواند هیچ جوابی نیافت
\\
در ره ما هر که را سایهٔ او پیش اوست
&&
از تف خورشید عشق تابش و تابی نیافت
\\
گر تو خرابی ز عشق جان تو آباد شد
&&
زانکه کسی گنج عشق جز به خرابی نیافت
\\
تا دل عطار دید هستی خود را حجاب
&&
رهزن خود شد مقیم تا که حجابی نیافت
\\
\end{longtable}
\end{center}
