\begin{center}
\section*{غزل شماره ۲۰۴: پشت بر روی جهان خواهیم کرد}
\label{sec:204}
\addcontentsline{toc}{section}{\nameref{sec:204}}
\begin{longtable}{l p{0.5cm} r}
پشت بر روی جهان خواهیم کرد
&&
قبله روی دلستان خواهیم کرد
\\
سود ما سودایی عشقت بس است
&&
گرچه دین و دل زیان خواهیم کرد
\\
خاصه عشقش را که سلطان دل است
&&
مرکبی از خون روان خواهیم کرد
\\
دل اگر خون شد ز عشقش باک نیست
&&
کین چنین کاری به جان خواهیم کرد
\\
گر در اول روز خون کردیم دل
&&
روز آخر جان فشان خواهیم کرد
\\
ذره ذره در ره سودای تو
&&
پایهای نردبان خواهیم کرد
\\
چون به یک یک پایه بر خواهیم رفت
&&
پایه‌ای زین دو جهان خواهیم کرد
\\
تا کسی چشمی زند بر هم به حکم
&&
ما دو عالم در میان خواهیم کرد
\\
آن روش کز هرچه گویم برتر است
&&
برتر از هفت آسمان خواهیم کرد
\\
وآن سفر کافلاک هرگز آن نکرد
&&
ما کنون در یک زمان خواهیم کرد
\\
گر کند چرخ فلک صد قرن سیر
&&
ما به یک دم بیش از آن خواهیم کرد
\\
پس به یک ذره و یک یک وجود
&&
خویشتن را امتحان خواهیم کرد
\\
سر ز یک یک ذره بر خواهیم تافت
&&
وز همه عالم کران خواهیم کرد
\\
شبنمی بی‌پا و سر خواهیم شد
&&
قصد بحر جاودان خواهیم کرد
\\
تا ابد چندان که ره خواهیم رفت
&&
منزل اول نشان خواهیم کرد
\\
نیست از پیشان ره کس را خبر
&&
پس خبر از کاروان خواهیم کرد
\\
کس جواب ما نخواهد داد باز
&&
گرچه بسیاری فغان خواهیم کرد
\\
گر بسی معشوق را خواهیم جست
&&
هم وجود خود عیان خواهیم کرد
\\
ور شود مویی ز معشوق آشکار
&&
ما همه خود را نهان خواهیم کرد
\\
چون فرید اینجا دو عالم محو شد
&&
پس چگونه ره بیان خواهیم کرد
\\
\end{longtable}
\end{center}
