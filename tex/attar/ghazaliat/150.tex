\begin{center}
\section*{غزل شماره ۱۵۰: گر هندوی زلفت ز درازی به ره افتاد}
\label{sec:150}
\addcontentsline{toc}{section}{\nameref{sec:150}}
\begin{longtable}{l p{0.5cm} r}
گر هندوی زلفت ز درازی به ره افتاد
&&
زنگی بچهٔ خال تو بر جایگه افتاد
\\
در آرزوی زلف چو زنجیر تو عقلم
&&
دیوانگی آورد و به یک ره ز ره افتاد
\\
چون باد بسی داشت سر زلف تو در سر
&&
از فرق همه تخت‌نشینان کله افتاد
\\
سرسبزی گلگون رخت را که بدیدم
&&
چون طرهٔ شبرنگ تو روزم سیه افتاد
\\
که کرد ز عشق رخ تو توبه زمانی
&&
کز شومی آن توبه نه در صد گنه افتاد
\\
حقا که اگر تا که جهان بود به خوبیت
&&
بر جملهٔ خوبان جهان پادشه افتاد
\\
تا پادشاه جملهٔ خوبان شده‌ای تو
&&
بس آتش سوزان که ز تو در سپه افتاد
\\
چون بوسه ستانم ز لبت چون مترصد
&&
با تیر و کمان چشم تو در پیشگه افتاد
\\
از عمد سر چاه زنخدان بنپوشید
&&
تا یوسف گم گشته درآمد به چه افتاد
\\
شهباز دلم زان چه سیمین نرهد زانک
&&
در خانهٔ مات است که این بار شه افتاد
\\
جانا دل عطار که دور از تو فتادست
&&
هرگز که بداند که چگونه تبه افتاد
\\
\end{longtable}
\end{center}
