\begin{center}
\section*{غزل شماره ۷۱: در سرم از عشقت این سودا خوش است}
\label{sec:071}
\addcontentsline{toc}{section}{\nameref{sec:071}}
\begin{longtable}{l p{0.5cm} r}
در سرم از عشقت این سودا خوش است
&&
در دلم از شوقت این غوغا خوش است
\\
من درون پرده جان می‌پرورم
&&
گر برون جان می کند اعدا خوش است
\\
چون جمالت برنتابد هیچ چشم
&&
جملهٔ آفاق نابینا خوش است
\\
همچو چرخ از شوق تو در هر دو کون
&&
هر که در خون می‌نگردد ناخوش است
\\
بندگی را پیش یک بند قبات
&&
صد کمر بر بسته بر جوزا خوش است
\\
جان فشان از خندهٔ جان‌پرورت
&&
زاهد خلوت نشین رسوا خوش است
\\
گر زبانم گنگ شد در وصف تو
&&
اشک خون آلود من گویا خوش است
\\
چون تو خونین می‌کنی دل در برم
&&
گرچه دل می‌سوزدم اما خوش است
\\
این جهان فانی است گر آن هم بود
&&
تو بسی، مه این مه آن یکتا خوش است
\\
گر نباشد هر دو عالم گو مباش
&&
تو تمامی با توام تنها خوش است
\\
ماه‌رویا سیرم اینجا از وجود
&&
بی وجودم گر بری آنجا خوش است
\\
پرده از رخ برفکن تا گم شوم
&&
کان تماشا بی وجود ما خوش است
\\
الحق آنجا کآفتاب روی توست
&&
صد هزاران بی سر و بی پا خوش است
\\
صد جهان بر جان و بر دل تا ابد
&&
والهٔ آن طلعت زیبا خوش است
\\
پرتو خورشید چون صحرا شود
&&
ذرهٔ سرگشته ناپروا خوش است
\\
چون تو پیدا آمدی چون آفتاب
&&
گر شدم چون سایه ناپیدا خوش است
\\
از درون چاه جسمم دل گرفت
&&
قصد صحرا می‌کنم صحرا خوش است
\\
دی اگر چون قطره‌ای بودم ضعیف
&&
این زمان دریا شدم دریا خوش است
\\
وای عجب تا غرق این دریا شدم
&&
بانگ می‌دارم که استسقا خوش است
\\
غرق دریا تشنه می‌میرم مدام
&&
این چه سودایی است این سودا خوش است
\\
ز اشتیاقت روز و شب عطار را
&&
دیده پر خون و دلی شیدا خوش است
\\
\end{longtable}
\end{center}
