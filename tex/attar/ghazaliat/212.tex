\begin{center}
\section*{غزل شماره ۲۱۲: خطت خورشید را در دامن آورد}
\label{sec:212}
\addcontentsline{toc}{section}{\nameref{sec:212}}
\begin{longtable}{l p{0.5cm} r}
خطت خورشید را در دامن آورد
&&
ز مشک ناب خرمن خرمن آورد
\\
چنان خطت برآوردست دستی
&&
که با خورشید و مه در گردن آورد
\\
کله‌دار فلک از عشق خطت
&&
چو گل کرده قبا پیراهن آورد
\\
خط مشکینت جوشی در دل انداخت
&&
لب شیرینت جوشی در من آورد
\\
فلک را عشق تو در گردش انداخت
&&
جهان را شوق تو در شیون آورد
\\
ندانم تا فلک در هیچ دوری
&&
به خوبی تو یک سیمین‌تن آورد
\\
فلک چون هر شبی زلف تو می‌دید
&&
که چندین حلقهٔ مردافکن آورد
\\
ز چشم بد بترسید از کواکب
&&
سر زلف تو را چوبک‌زن آورد
\\
از آن سر رشته گم کردم که رویت
&&
دهانی همچو چشم سوزن آورد
\\
از آن سرگشته دل ماندم که لعلت
&&
گهر سی‌دانه در یک ارزن آورد
\\
ز بهر ذره‌ای وصل تو هر روز
&&
اگر خورشید وجهی روشن آورد
\\
چون آن ذره نیافت از خجلت آن
&&
فرو شد زرد و سر در دامن آورد
\\
دل عطار در وصلت ضمیری
&&
به اسرار سخن آبستن آورد
\\
\end{longtable}
\end{center}
