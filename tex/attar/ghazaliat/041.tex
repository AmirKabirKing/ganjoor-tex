\begin{center}
\section*{غزل شماره ۴۱: قبلهٔ ذرات عالم روی توست}
\label{sec:041}
\addcontentsline{toc}{section}{\nameref{sec:041}}
\begin{longtable}{l p{0.5cm} r}
قبلهٔ ذرات عالم روی توست
&&
کعبهٔ اولاد آدم کوی توست
\\
میل خلق هر دو عالم تا ابد
&&
گر شناسند و اگر نی سوی توست
\\
چون به جز تو دوست نتوان داشتن
&&
دوستی دیگران بر بوی توست
\\
هر پریشانی که در هر دو جهان
&&
هست و خواهد بود از یک موی توست
\\
هر کجا در هر دو عالم فتنه‌ای است
&&
ترکتاز طرهٔ هندوی توست
\\
پهلوانان درت بس بی‌دلند
&&
دل ندارد هر که در پهلوی توست
\\
نیست پنهان آنکه از من دل ربود
&&
هست همچون آفتاب آن روی توست
\\
عقل چون طفل ره عشق تو بود
&&
شیرخوار از لعل پر لؤلؤی توست
\\
تیربارانی که چشمت می‌کند
&&
بر دلم پیوسته از ابروی توست
\\
گفتم ابرویت اگر طاقم فکند
&&
این گناه نرگس جادوی توست
\\
گفتم ای عاقل برو چون تیر راست
&&
کین کمان هرگز نه بر بازوی توست
\\
این همه عطار دور از روی تو
&&
درد از آن دارد که بی داروی توست
\\
\end{longtable}
\end{center}
