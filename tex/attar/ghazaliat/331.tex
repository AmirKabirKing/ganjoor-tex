\begin{center}
\section*{غزل شماره ۳۳۱: مرد یک موی تو فلک نبود}
\label{sec:331}
\addcontentsline{toc}{section}{\nameref{sec:331}}
\begin{longtable}{l p{0.5cm} r}
مرد یک موی تو فلک نبود
&&
محرم کوی تو ملک نبود
\\
ماه دو هفته گرچه هست تمام
&&
از جمال تو هفت یک نبود
\\
چون جمال تو آشکار شود
&&
همه باشی تو هیچ شک نبود
\\
ملک حسن آفتاب روی تورا
&&
با کسی نیز مشترک نبود
\\
نتوان دید ذره‌ای رخ تو
&&
تا دو عالم دو مردمک نبود
\\
آنچه در ذره ذره هست از تو
&&
در زمین نیست در فلک نبود
\\
لیک چون ذره در تو محو شود
&&
محو را ذره‌ای برک نبود!
\\
زر خورشید ذره ذره شود
&&
اگرش خال تو محک نبود
\\
هیچکس را در آفرینش حق
&&
در شکر این همه نمک نبود
\\
سر زلفت به چین رسید از هند
&&
هیچکس را چنین یزک نبود
\\
گر خسک در ره من اندازی
&&
چون تو اندازی آن خسک نبود
\\
هرچه عطار در صفات تو گفت
&&
بر محک جاودانش حک نبود
\\
\end{longtable}
\end{center}
