\begin{center}
\section*{غزل شماره ۲۰۱: تا دوست بر دلم در عالم فراز کرد}
\label{sec:201}
\addcontentsline{toc}{section}{\nameref{sec:201}}
\begin{longtable}{l p{0.5cm} r}
تا دوست بر دلم در عالم فراز کرد
&&
دل را به عشق خویش ز جان بی نیاز کرد
\\
دل از شراب عشق چو بر خویشتن فتاد
&&
از جان بشست دست و به جانان دراز کرد
\\
فریاد برکشید چو مست از شراب عشق
&&
بیخود شد و ز ننگ خودی احتراز کرد
\\
چون دل بشست از بد و نیک همه جهان
&&
تکبیر کرد بر دل و بر وی نماز کرد
\\
بر روی دوست دیده چو بر دوخت از دو کون
&&
این دیده چون فراز شد آن دیده باز کرد
\\
پیش از اجل بمرد و بدان زندگی رسید
&&
ادریس وقت گشت که جان چشم باز کرد
\\
چندان که رفت راه به آخر نمی‌رسید
&&
در هر قدم هزار حقیقت مجاز کرد
\\
عطار شرح چون دهد اندر هزار سال
&&
آن نیکویی که با دل او دلنواز کرد
\\
\end{longtable}
\end{center}
