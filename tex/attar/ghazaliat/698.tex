\begin{center}
\section*{غزل شماره ۶۹۸: ای دو عالم پرتوی از روی تو}
\label{sec:698}
\addcontentsline{toc}{section}{\nameref{sec:698}}
\begin{longtable}{l p{0.5cm} r}
ای دو عالم پرتوی از روی تو
&&
جنت الفردوس خاک کوی تو
\\
صد جهان پر عاشق سرگشته را
&&
هیچ وجهی نیست الا روی تو
\\
صد هزارن قصه دارم دردناک
&&
دور از روی تو با هر موی تو
\\
کور باید گشت از دید دو کون
&&
تا توان کردن نگاهی سوی تو
\\
یافت هندوخان لقب بر خوان چرخ
&&
ترک گردون تا که شد هندوی تو
\\
پشت صد صد پهلوان می‌بشکند
&&
تیر یک یک غمزهٔ جادوی تو
\\
دی مرا خواندی به تیر غمزه پیش
&&
تا کمان بر زه کنم ز ابروی تو
\\
خود سپر بفکندم و بگریختم
&&
کان کمان هم هست بر بازوی تو
\\
نه ز تو بگریختم از بیم سنگ
&&
زانکه دیدم سنگ در پهلوی تو
\\
شد زبان در وصف تو عطار را
&&
درفشان چون حلقهٔ لؤلؤی تو
\\
\end{longtable}
\end{center}
