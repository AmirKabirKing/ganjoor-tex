\begin{center}
\section*{غزل شماره ۱۲۴: تاب روی تو آفتاب نداشت}
\label{sec:124}
\addcontentsline{toc}{section}{\nameref{sec:124}}
\begin{longtable}{l p{0.5cm} r}
تاب روی تو آفتاب نداشت
&&
بوی زلف تو مشک ناب نداشت
\\
خازن خلد هشت خلد بگشت
&&
در خور جام تو شراب نداشت
\\
ذره‌ای پیش لعل سیرابت
&&
چشمهٔ آفتاب آب نداشت
\\
لعلت از آفتاب کرد سؤال
&&
کانچه او داشت آفتاب نداشت
\\
گفت تا سرگشاد چشمهٔ تو
&&
آب حیوان چون گلاب نداشت
\\
همچو من آب خضر و کوثر هم
&&
زیر سی لؤلؤ خوشاب نداشت
\\
چشمه بی‌آب کی به کار آید
&&
زین سخن آفتاب تاب نداشت
\\
همه دعوی او زوال آمد
&&
زرد از آن شد که یک جواب نداشت
\\
دور از روی همچو خورشیدت
&&
چشم من نیم ذره خواب نداشت
\\
کیست کز چشم مست خونریزت
&&
باده ناخورده دل خراب نداشت
\\
کیست کز دست فرق مشکینت
&&
دست بر فرق چون رباب نداشت
\\
کیست کز عشق لالهٔ رخ تو
&&
رخ چو لاله به خون خضاب نداشت
\\
گرچه صیدم مرا مکش به عذاب
&&
کس چو من صید را عذاب نداشت
\\
من چنان لاغرم که پهلوی من
&&
جز دل از لاغری کباب نداشت
\\
کس به خون‌ریزی چنان لاغر
&&
تا که فربه نشد شتاب نداشت
\\
تا که صید تو شد دل عطار
&&
سینه خالی ز اضطراب نداشت
\\
\end{longtable}
\end{center}
