\begin{center}
\section*{غزل شماره ۶۸: وشاقی اعجمی با دشنه در دست}
\label{sec:068}
\addcontentsline{toc}{section}{\nameref{sec:068}}
\begin{longtable}{l p{0.5cm} r}
وشاقی اعجمی با دشنه در دست
&&
به خون آلوده دست و زلف چون شست
\\
کمر بسته کله کژ برنهاده
&&
گره بر ابرو و پر خشم و سرمست
\\
درآمد در میان خرقه‌پوشان
&&
به کس در ننگرست از پای ننشست
\\
بزد یک دشته بر دل پیر ما را
&&
دلش بگشاد و زناریش بربست
\\
چو کرد این کار ناپیدا شد از چشم
&&
چون آتش پاره‌ای آن پیر در جست
\\
درآشامید دریاهای اسرار
&&
ز جام نیستی در صورت هست
\\
خودی او به کلی زو فرو ریخت
&&
ز ننگ خویشتن بینی برون رست
\\
جهان گم بد درو اما هنوز او
&&
بدان مطلوب خود عور و تهی دست
\\
چو مرغ همتش زان دانه بد دور
&&
قفس از بس که پر زد خرد بشکست
\\
ببرید و نشان و نام از او رفت
&&
ندانم تا کجا شد در که پیوست
\\
ازین دریا که کس با سر نیامد
&&
اگر خونین شود جان جای آن هست
\\
دلی پر خون درین هیبت بماندست
&&
فلک پشتی دو تا در سوک بنشست
\\
دریغا جان پر اسرار عطار
&&
که شد در پای این سرگشتگی پست
\\
\end{longtable}
\end{center}
