\begin{center}
\section*{غزل شماره ۵۵۶: چو خود را پاک دامن می ندانم}
\label{sec:556}
\addcontentsline{toc}{section}{\nameref{sec:556}}
\begin{longtable}{l p{0.5cm} r}
چو خود را پاک دامن می ندانم
&&
مقامی به ز گلخن می ندانم
\\
چرا اندر صف مردان نشینم
&&
چو خود را مرد جوشن می ندانم
\\
بیا تا ترک خود گیرم که خود را
&&
بتر از خویش دشمن می ندانم
\\
دلی کز آرزوها گشت پر بت
&&
من آن دل را مزین می ندانم
\\
چو عیسی از یکی سوزن فروماند
&&
من این بت کم ز سوزن می ندانم
\\
مرا جانان فروشد در غمت جان
&&
اگرچه جان معین می ندانم
\\
چنان در عشق تو سرگشته گشتم
&&
که جانم گم شد و تن می ندانم
\\
مرا هم کشتی و هم سوختی زار
&&
چه می‌خواهی تو از من می ندانم
\\
گهی گویی که تن زن صبر کن صبر
&&
علاج صبر کردن می ندانم
\\
گهی گویی مرا بستان ورستی
&&
ز صد خرمن یک ارزن می‌ندانم
\\
چون من یک ذره‌ام نه هست و نه نیست
&&
همه خورشید روشن می ندانم
\\
فرو رفتم در این وادی کم و کاست
&&
تو می‌دانی اگر من می ندانم
\\
درین حیرت دل حیران خود را
&&
طریقی به ز مردن می ندانم
\\
که گیرد دامن عطار ازین پس
&&
چو او را هیچ دامن می ندانم
\\
\end{longtable}
\end{center}
