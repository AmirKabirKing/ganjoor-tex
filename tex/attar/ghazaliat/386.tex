\begin{center}
\section*{غزل شماره ۳۸۶: اگر خورشید خواهی سایه بگذار}
\label{sec:386}
\addcontentsline{toc}{section}{\nameref{sec:386}}
\begin{longtable}{l p{0.5cm} r}
اگر خورشید خواهی سایه بگذار
&&
چو مادر هست شیر دایه بگذار
\\
چو با خورشید هم‌تک می‌توان شد
&&
ز پس در تک زدن چون سایه بگذار
\\
چو همسایه است با جان تو جانان
&&
بده جان و حق همسایه بگذار
\\
تو را سرمایهٔ هستی بلایی است
&&
زیانت سود کن سرمایه بگذار
\\
چو مردان جوشن و شمشیر برگیر
&&
نه‌ای آخر چو زن پیرایه بگذار
\\
فلک طشت است و اختر خایه در طشت
&&
خیال علم طشت و خایه بگذار
\\
فروتر پایهٔ تو عرش اعلاست
&&
تو برتر رو فروتر پایه بگذار
\\
فرید از مایهٔ هستی جدا شد
&&
تو هم مردی شو و این مایه بگذار
\\
\end{longtable}
\end{center}
