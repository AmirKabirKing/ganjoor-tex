\begin{center}
\section*{غزل شماره ۴۸: این گره کز تو بر دل افتادست}
\label{sec:048}
\addcontentsline{toc}{section}{\nameref{sec:048}}
\begin{longtable}{l p{0.5cm} r}
این گره کز تو بر دل افتادست
&&
کی گشاید که مشکل افتادست
\\
ناگشاده هنوز یک گرهم
&&
صد گره نیز حاصل افتادست
\\
چون نهد گام آنکه هر روزیش
&&
سیصد و شصت منزل افتادست
\\
چون رود راه آنکه هر میلش
&&
ینزل‌الله مقابل افتادست
\\
چونکه از خوف این چنین شب و روز
&&
عرش را رخت در گل افتادست
\\
من که باشم که دم زنم آنجا
&&
ور زنم زهر قاتل افتادست
\\
هست دیوانه‌ای علی الاطلاق
&&
هر که زین قصه غافل افتادست
\\
عقل چبود که صد جهان آتش
&&
نقد در جان و در دل افتادست
\\
فلک آبستن است این سر را
&&
زان بدین سیر مایل افتادست
\\
همچو آبستنان نقط بر روی
&&
می‌رود گرچه حامل افتادست
\\
نیست آگاه کسی ازین سر ازانک
&&
بیشتر خلق غافل افتادست
\\
قعر دریا چگونه داند باز
&&
آن کسی کو به ساحل افتادست
\\
گر رجوعی کند سوی قعرش
&&
گوهری سخت قابل افتادست
\\
ور کند حبس ساحلش محبوس
&&
در مضیق مشاغل افتادست
\\
هست در معرض بسی گرداب
&&
هر که را این مسایل افتادست
\\
خاک آنم که او درین دریا
&&
ترک جان گفته کامل افتادست
\\
هر که صد بحر یافت بس تنها
&&
قطره‌ای خرد مدخل افتادست
\\
جان عطار را درین دریا
&&
نفس تاریک حایل افتادست
\\
\end{longtable}
\end{center}
