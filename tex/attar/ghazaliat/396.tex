\begin{center}
\section*{غزل شماره ۳۹۶: پیر ما می‌رفت هنگام سحر}
\label{sec:396}
\addcontentsline{toc}{section}{\nameref{sec:396}}
\begin{longtable}{l p{0.5cm} r}
پیر ما می‌رفت هنگام سحر
&&
اوفتادش بر خراباتی گذر
\\
نالهٔ رندی به گوش او رسید
&&
کای همه سرگشتگان را راهبر
\\
نوحه از اندوه تو تا کی کنم
&&
تا کیم داری چنین بی خواب و خور
\\
در ره سودای تو درباختم
&&
کفر و دین و گرم و سرد و خشک و تر
\\
من همی دانم که چون من مفسدم
&&
ننگ می‌آید تو را زین بی هنر
\\
گرچه من رندم ولیکن نیستم
&&
دزد و شب رو رهزن و درویزه گر
\\
نیستم مرد ریا و زرق و فن
&&
فارغم از ننگ و نام و خیر و شر
\\
چون ندارم هیچ گوهر در درون
&&
می‌نمایم خویشتن را بد گهر
\\
این سخن ها همچو تیر راست‌رو
&&
بر دل آن پیر آمد کارگر
\\
دردیی بستد از آن رند خراب
&&
درکشید و آمد از خرقه بدر
\\
دردی عشقش به یک‌دم مست کرد
&&
در خروش آمد که‌ای دل الحذر
\\
ساغر دل اندر آن دم دم بدم
&&
پر همی کرد از خم خون جگر
\\
اندر آن اندیشه چون سرگشتگان
&&
هر زمان از پای می‌آمد به سر
\\
نعره می‌زد کاخر این دل را چه بود
&&
کین چنین یکبارگی شد بی خبر
\\
گرچه پیر راه بودم شصت سال
&&
می‌ندانستم درین راه این قدر
\\
هر که را از عشق دل از جای شد
&&
تا ابد او پند نپذیرد دگر
\\
هر که را در سینه نقد درد اوست
&&
گو به یک جوهر دو عالم را مخر
\\
بگسلان پیوند صورت را تمام
&&
پس به آزادی درین معنی نگر
\\
زانچه مر عطار را داده است دوست
&&
در دو عالم گشت او زان نامور
\\
\end{longtable}
\end{center}
