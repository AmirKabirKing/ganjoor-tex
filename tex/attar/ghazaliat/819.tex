\begin{center}
\section*{غزل شماره ۸۱۹: خواجه تا چند حساب زر و دینار کنی}
\label{sec:819}
\addcontentsline{toc}{section}{\nameref{sec:819}}
\begin{longtable}{l p{0.5cm} r}
خواجه تا چند حساب زر و دینار کنی
&&
سود و سرمایهٔ دین بر سر بازار کنی
\\
شب عمرت بشد و صبح اجل نزدیک است
&&
خویشتن را گه آن نیست که بیدار کنی
\\
چیست این عجب و تفاخر به جهان ساکن باش
&&
چند با صد من و من سیم و زر اظهار کنی
\\
پنج روزی همه کامی ز جهان حاصل گیر
&&
عاقبت هم سر پر کبر نگونسار کنی
\\
آن نه کام است که ناکام بجا بگذری
&&
وان نه برگ است که بر جان خودش بار کنی
\\
جمع تو بار گنه باشد و دیوان سیاه
&&
نه هم آخر تو خوشی نام سیه بار کنی
\\
چون همی دانی کت خانه لحد خواهد بود
&&
خانه را نقش چرا بر در و دیوار کنی
\\
سهو کارا به تک خاک همی باید خفت
&&
طاق و ایوان به چه تا گنبد دوار کنی
\\
مرگ در پیش و حساب از پس و دوزخ در راه
&&
به چه شادی خرفا خندهٔ بسیار کنی
\\
تو که بر روبه مسکین بدری پوست چو سگ
&&
عنکبوتانه کجا پردهٔ احرار کنی
\\
این همه دانی و کارت همه بی وجه بود
&&
خود ستم کم کن اگر منع ستمکار کنی
\\
به فصاحت ببری گوی ز میدان سخن
&&
لیک خود را به ستم بیهده رهوار کنی
\\
خویش و همسایهٔ تو گرسنه وز پر طمعی
&&
نفروشی به کسی غله در انبار کنی
\\
جامه در تنگ و دلت تنگ و در اندیشهٔ آن
&&
تا دگر ره ز کجا جامه و دستار کنی
\\
بر ضعیفان نکنی رحم به یک قرص جوین
&&
وانگه از ناز به مرغ و بره پروار کنی
\\
مستراحی است جهان و اهل جهان کناسند
&&
به تعزز سزد ار در همه نظار کنی
\\
نافه داری بر هر خشک دمانی مگشا
&&
اول آن به که طلبکاری عطار کنی
\\
\end{longtable}
\end{center}
