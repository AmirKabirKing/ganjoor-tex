\begin{center}
\section*{غزل شماره ۱۶۲: گر پرده ز خورشید جمال تو برافتد}
\label{sec:162}
\addcontentsline{toc}{section}{\nameref{sec:162}}
\begin{longtable}{l p{0.5cm} r}
گر پرده ز خورشید جمال تو برافتد
&&
گل جامه قبا کرده ز پرده به در افتد
\\
چون چشم چمن چهرهٔ گلرنگ تو بیند
&&
خون از دهن غنچه ز تشویر برافتد
\\
بشکافت تنم غمزهٔ تو گرچه چو مویی است
&&
یک تیر ندیدم که چنین کارگر افتد
\\
گر بر جگرم آب نمانده است عجب نیست
&&
کاتش ز رخت هر نفس اندر جگر افتد
\\
گر چه دل من مرغ بلند است چو سیمرغ
&&
لیکن چو دمت خورد به دام تو درافتد
\\
گر گلشکری این دل بیمار کند راست
&&
آتش ز لب و روی تو در گلشکر افتد
\\
بر چشم و لبم زآتش عشق تو بترسم
&&
کین آتش از آن است که در خشک و تر افتد
\\
من خاک توام پا نهم بر سر افلاک
&&
چون باد، گرت بر من خاکی گذر افتد
\\
بی یاد تو عطار اگر جان به لب آرد
&&
جانش همه خون گردد و دل در خطر افتد
\\
\end{longtable}
\end{center}
