\begin{center}
\section*{غزل شماره ۶۰۲: هر آن نقشی که بر صحرا نهادیم}
\label{sec:602}
\addcontentsline{toc}{section}{\nameref{sec:602}}
\begin{longtable}{l p{0.5cm} r}
هر آن نقشی که بر صحرا نهادیم
&&
تو زیبا بین که ما زیبا نهادیم
\\
سر مویی ز زلف خود نمودیم
&&
جهان را در بسی غوغا نهادیم
\\
چو آدم را فرستادیم بیرون
&&
جمال خویش بر صحرا نهادیم
\\
جمال ما ببین کین راز پنهان
&&
اگر چشمت بود پیدا نهادیم
\\
وگر چشمت نباشد همچنان دان
&&
که گوهر پیش نابینا نهادیم
\\
کسی ننهاد و نتواند نهادن
&&
طلسماتی که هر دم ما نهادیم
\\
مباش احوال مسمی جز یکی نیست
&&
اگرچه این همه اسما نهادیم
\\
یقین می‌دان که چندینی عجایب
&&
برای یک دل دانا نهادیم
\\
ز چندینی عجایب حصهٔ تو
&&
اگر دانا نه‌ای سودا نهادیم
\\
مشو مغرور چندین نقش زیراک
&&
بنای جمله بر دریا نهادیم
\\
اگر موجی از آن دریا برآید
&&
شود ناچیز هرچه اینجا نهادیم
\\
اگر اینجا ز دریا برکناری
&&
جهانی پر غمت آنجا نهادیم
\\
وگر همرنگ دریا گردی امروز
&&
تو را سلطانی فردا نهادیم
\\
دل عطار را در عشق این راه
&&
چه گویی بی سر و بی پا نهادیم
\\
\end{longtable}
\end{center}
