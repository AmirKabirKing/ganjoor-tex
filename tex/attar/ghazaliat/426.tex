\begin{center}
\section*{غزل شماره ۴۲۶: آنکه سر دارد کلاهت نرسدش}
\label{sec:426}
\addcontentsline{toc}{section}{\nameref{sec:426}}
\begin{longtable}{l p{0.5cm} r}
آنکه سر دارد کلاهت نرسدش
&&
وانکه پر آب است جاهت نرسدش
\\
هر که پست بارگاه فقر نیست
&&
در بلندی دستگاهت نرسدش
\\
هر که در خود ماند چون گردون بسی
&&
گر نگردد گرد راهت نرسدش
\\
تا نباشد همچو یوسف خواجه‌ای
&&
بندگی در قعر چاهت نرسدش
\\
تا کسی دارد به یک ذره پناه
&&
عرش اگر باشد پناهت نرسدش
\\
عرش اگر کرسی نهد در زیر پای
&&
دست بر زلف سیاهت نرسدش
\\
گرچه سر در عرش ساید آفتاب
&&
پرتو روی چو ماهت نرسدش
\\
نیم ترک چرخ در سر گشت از آنک
&&
بو که بر ترک کلاهت نرسدش
\\
تا کسی نشکست کلی قلب نفس
&&
لاف از خیل و سپاهت نرسدش
\\
تا نسوزد جملهٔ شب شمع زار
&&
یک نسیم صبحگاهت نرسدش
\\
تا کسی بر سر نگردد چون فلک
&&
طوف گرد بارگاهت نرسدش
\\
تا کسی جان ندهد از درد خمار
&&
می ز لعل عذر خواهت نرسدش
\\
گر نشد عطار یکتا همچو موی
&&
مشک از زلف دو تاهت نرسدش
\\
\end{longtable}
\end{center}
