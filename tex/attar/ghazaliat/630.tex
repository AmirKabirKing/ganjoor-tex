\begin{center}
\section*{غزل شماره ۶۳۰: چون نیاید سر عشقت در بیان}
\label{sec:630}
\addcontentsline{toc}{section}{\nameref{sec:630}}
\begin{longtable}{l p{0.5cm} r}
چون نیاید سر عشقت در بیان
&&
همچو طفلان مهر دارم بر زبان
\\
چون عبارت محرم عشق تو نیست
&&
چون دهد نامحرم از پیشان نشان
\\
آنک ازو سگ می‌کند پهلو تهی
&&
دوستکانی چون خورد با پهلوان
\\
چون زبان در عشق تو بر هیچ نیست
&&
لب فرو بستم قلم کردم زبان
\\
همچو مرغ نیم بسمل در رهت
&&
در میان خاک و خون گشتم نهان
\\
دور از تو جان ز من گیرد کنار
&&
گر مرا بیرون نیاری زین میان
\\
دوش عشق تو درآمد نیم شب
&&
از رهی دزدیده یعنی راه جان
\\
گفت صد دریا ز خون دل بیار
&&
تا در آشامم که مستم این زمان
\\
مرغ دل آوارهٔ دیرینه بود
&&
باز یافت از عشق حالی آشیان
\\
در پرید و عشق را در بر گرفت
&&
عقل و جان را کارد آمد به استخوان
\\
عقل فانی گشت و جان معدوم شد
&&
عشق و دل ماندند با هم جاودان
\\
عشق با دل گشت و دل با عشق شد
&&
زین عجب‌تر قصه نبود در جهان
\\
دیدن و دانستن اینجا باطل است
&&
بودن آن کار نه علم و بیان
\\
چون بباشی فانی مطلق ز خویش
&&
هست مطلق گردی اندر لامکان
\\
جان و جانان هر دو نتوان یافتن
&&
گر همی جانانت باید جان‌فشان
\\
تا کی ای عطار گویی راز عشق
&&
راز می‌گویی طلب کن رازدان
\\
\end{longtable}
\end{center}
