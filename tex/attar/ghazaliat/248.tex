\begin{center}
\section*{غزل شماره ۲۴۸: یک شرر از عین عشق دوش پدیدار شد}
\label{sec:248}
\addcontentsline{toc}{section}{\nameref{sec:248}}
\begin{longtable}{l p{0.5cm} r}
یک شرر از عین عشق دوش پدیدار شد
&&
طای طریقت بتافت عقل نگونسار شد
\\
مرغ دلم همچو باد گرد دو عالم بگشت
&&
هرچه نه از عشق بود از همه بیزار شد
\\
بر دل آن کس که تافت یک سر مو زین حدیث
&&
صومعه بتخانه گشت خرقه چو زنار شد
\\
گر تف خورشید عشق یافته‌ای ذره‌شو
&&
زود که خورشید عمر بر سر دیوار شد
\\
ماه رخا هر که دید زلف تو کافر بماند
&&
لیک هر آنکس که دید روی تو دین‌دار شد
\\
دام سر زلف تو باد صبا حلقه کرد
&&
جان خلایق چو مرغ جمله گرفتار شد
\\
یک شکن از زلف تو وقت سحر کشف گشت
&&
جان همه منکران واقف اسرار شد
\\
باز چو زلف تو کرد بلعجبی آشکار
&&
زاهد پشمینه پوش ساکن خمار شد
\\
هر که ز دین گشته بود چون رخ خوب تو دید
&&
پای بدین در نهاد باز به اقرار شد
\\
وانکه مقر گشته بود حجت اسلام را
&&
چون سر زلف تو دید با سر انکار شد
\\
روی تو و موی تو کایت دین است و کفر
&&
رهبر عطار گشت ره زن عطار شد
\\
\end{longtable}
\end{center}
