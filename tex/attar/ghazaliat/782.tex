\begin{center}
\section*{غزل شماره ۷۸۲: چه عجب کسی تو جانا که ندانمت چه چیزی}
\label{sec:782}
\addcontentsline{toc}{section}{\nameref{sec:782}}
\begin{longtable}{l p{0.5cm} r}
چه عجب کسی تو جانا که ندانمت چه چیزی
&&
تو مگر که جان جانی که چو جان جان عزیزی
\\
ز کجات جویم ای جان که کست نیافت هرگز
&&
ز که خواهمت که با کس ننشینی و نخیزی
\\
تن و جان برفته از هش ز تو تا تو خود چه گنجی
&&
دل و دین بمانده واله ز تو تا تو خود چه چیزی
\\
بنگر که چند عاشق ز تو خفته‌اند در خون
&&
ز کمال غیرت خود تو هنوز می ستیزی
\\
چه کشی مرا که من خود ز غم تو کشته گردم
&&
چو منی بدان نیرزد که تو خون من بریزی
\\
چو ز زلف خود شکنجی به میان ما فکندی
&&
به میان در آی آخر ز میان چه می‌گریزی
\\
چو نیافت جان عطار اثری ز ذوق عشقت
&&
بفروخت ز اشتیاقت ز دل آتش غریزی
\\
\end{longtable}
\end{center}
