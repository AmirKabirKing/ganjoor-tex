\begin{center}
\section*{غزل شماره ۱۹۹: چون شراب عشق در دل کار کرد}
\label{sec:199}
\addcontentsline{toc}{section}{\nameref{sec:199}}
\begin{longtable}{l p{0.5cm} r}
چون شراب عشق در دل کار کرد
&&
دل ز مستی بیخودی بسیار کرد
\\
شورشی اندر نهاد دل فتاد
&&
دل در آن شورش هوای یار کرد
\\
جامهٔ دریوزه بر آتش نهاد
&&
خرقهٔ پیروزه را زنار کرد
\\
هم ز فقر خویشتن بیزار شد
&&
هم ز زهد خویش استغفار کرد
\\
نیکویی‌هائی که در اسلام یافت
&&
بر سر جمع مغان ایثار کرد
\\
از پی یک قطره درد درد دوست
&&
روی اندر گوشهٔ خمار کرد
\\
چون ببست از هر دو عالم دیده را
&&
در میان بیخودی دیدار کرد
\\
هستی خود زیر پای آورد پست
&&
وز بلندی دست در اسرار کرد
\\
آنچه یافت از یاری عطار یافت
&&
وآنچه کرد از همت عطار کرد
\\
\end{longtable}
\end{center}
