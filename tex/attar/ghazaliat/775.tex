\begin{center}
\section*{غزل شماره ۷۷۵: تو را تا سر بود برجا کجا داری کله داری}
\label{sec:775}
\addcontentsline{toc}{section}{\nameref{sec:775}}
\begin{longtable}{l p{0.5cm} r}
تو را تا سر بود برجا کجا داری کله داری
&&
که شمع از بی سری یابد کلاه از نور جباری
\\
سر یک موی سر مفراز و سر در باز و سر بر نه
&&
اگر پیش سر اندازان سزای تن، سری داری
\\
چو بار آمد سر یحیی سرش بر تیرگی ماند
&&
درین سر باختن این سر بدان گر مرد اسراری
\\
مبر مویی وجود آنجا که دایم آن وجودت بس
&&
که مویی نیست تدبیرت مگر از خویش بیزاری
\\
اگر یک پرتو این نور بر هر دو جهان افتد
&&
شود هر دو جهان از شرم چون یک ذره متواری
\\
چو عالم ذره‌ای است اینجا ز عالم چند باشی تو
&&
که در پیش چنین کاری کمر بندی به عیاری
\\
چو شد ذات و صفت بندت مرو با این و آن آنجا
&&
چو گل زانجا برند آنجا چه خواهی برد جز زاری
\\
صفات نیک و بد آنجا بسوزد آتش غیرت
&&
مبر جز هیچ آنجا هیچ تا برهی به دشواری
\\
چه می‌گویم نه‌ای تو مرد این اسرار دین‌پرور
&&
که تو از دنیی جافی بماندی در نگونساری
\\
به دنیا عمر در جوجو بسر بردی عجب این است
&&
که در عقباب خواهد بود زان جوجو گرفتاری
\\
به دنیا و به عقبی در چو خر در جو به جو ماندی
&&
ز روح عیسوی بویی به تو نرسید پنداری
\\
چو در جانت ز دنیا بار بسیار است و از دین نه
&&
تو را زین بار جان دین رفت و دنیا هم به سر باری
\\
اگر از زندگی خود نکردی ذره‌ای حاصل
&&
چه داری غم چو کردی جمع این دنیای مرداری
\\
دل عطار خونی شد ازین دریای بوقلمون
&&
چه دنیا دیو مردم‌خوار و چندین خلق پرواری
\\
\end{longtable}
\end{center}
