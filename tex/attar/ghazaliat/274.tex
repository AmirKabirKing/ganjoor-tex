\begin{center}
\section*{غزل شماره ۲۷۴: کارم از عشق تو به جان آمد}
\label{sec:274}
\addcontentsline{toc}{section}{\nameref{sec:274}}
\begin{longtable}{l p{0.5cm} r}
کارم از عشق تو به جان آمد
&&
دلم از درد در فغان آمد
\\
تا می عشق تو چشید دلم
&&
از بد و نیک بر کران آمد
\\
از سر نام و ننگ و روی و ریا
&&
با سر درد جاودان آمد
\\
سالها در رهت قدمها زد
&&
عمرها بر پیت دوان آمد
\\
شب نخفت و به روز نارامید
&&
تا ز هستی خود به جان آمد
\\
وز تو کس را دمی درین وادی
&&
بی خبر بود و بی نشان آمد
\\
چون ز مقصود خود ندیدم بوی
&&
سود عمرم همه زیان آمد
\\
دل حیوان چو مرد کار نبود
&&
چون زنان پیش دیگران آمد
\\
دین هفتاد ساله داد به باد
&&
مرد میخانه و مغان آمد
\\
کم زن و همنشین رندان شد
&&
سگ مردان کاردان آمد
\\
با خراباتیان دردی کش
&&
خرقه بنهاد و در میان آمد
\\
چون به ایمان نیامدی در دست
&&
کافری را به امتحان آمد
\\
ترک دین گفت تا مگر بی دین
&&
بوک در خورد تو توان آمد
\\
دل عطار چون زبان دربست
&&
از بد و نیک در کران آمد
\\
\end{longtable}
\end{center}
