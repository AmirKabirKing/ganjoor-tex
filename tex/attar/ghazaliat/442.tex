\begin{center}
\section*{غزل شماره ۴۴۲: هر که هست اندر پی بهبود خویش}
\label{sec:442}
\addcontentsline{toc}{section}{\nameref{sec:442}}
\begin{longtable}{l p{0.5cm} r}
هر که هست اندر پی بهبود خویش
&&
دور افتادست از مقصود خویش
\\
تو ایازی پوستین را یاد دار
&&
تا نیفتی دور از محمود خویش
\\
عاشقی باید که بر هم سوزد او
&&
عالمی از آه خون آلود خویش
\\
نیست از تو یک نفس خشنود دوست
&&
تا تو هستی یک نفس خشنود خویش
\\
زاهد افسرده چوب سنجد است
&&
خوش بسوز ای عاشق اکنون عود خویش
\\
حلقهٔ معشوق گیر و وقف کن
&&
بر در او جان غم فرسود خویش
\\
چون درین سودا زیان از سود به
&&
پس درین سودا زیان کن سود خویش
\\
تا کی از بود تو و نابود تو
&&
درگذر از بود و از نابود خویش
\\
آتشی در هستی تاریک زن
&&
پس برون آی از میان دود خویش
\\
گر فنا گردی چو عطار از وجود
&&
فال گیر از طالع مسعود خویش
\\
\end{longtable}
\end{center}
