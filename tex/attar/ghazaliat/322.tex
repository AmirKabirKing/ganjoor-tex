\begin{center}
\section*{غزل شماره ۳۲۲: عشق بی درد ناتمام بود}
\label{sec:322}
\addcontentsline{toc}{section}{\nameref{sec:322}}
\begin{longtable}{l p{0.5cm} r}
عشق بی درد ناتمام بود
&&
کز نمک دیگ را طعام بود
\\
نمک این حدیث درد دل است
&&
عشق بی درد دل حرام بود
\\
کشته عشق گرد و سوخته شو
&&
زانکه بی این دو کار خام بود
\\
کشتهٔ عشق را به خون شویند
&&
آب اگر نیست خون تمام بود
\\
کفن عاشقان ز خون سازند
&&
کفنی به ز خون کدام بود
\\
از ازل تا ابد ز مستی عشق
&&
بی قراری علی‌الدوام بود
\\
در ره عاشقان دلی باید
&&
که منزه ز دال و لام بود
\\
نه خریدار نیک و بد باشد
&&
نه گرفتار ننگ و نام بود
\\
سرفرازی و خواجگی نخرد
&&
جملهٔ خلق را غلام بود
\\
نبود تیغش و اگر باشد
&&
با همه خلق در نیام بود
\\
همچو خود بی قرار و مست کند
&&
هر که را پیش او مقام بود
\\
گاه‌گاهی چنین شود عطار
&&
بو که این دولتش مدام بود
\\
\end{longtable}
\end{center}
