\begin{center}
\section*{غزل شماره ۶۶۷: گر با تو بگویم غم افزون شدهٔ من}
\label{sec:667}
\addcontentsline{toc}{section}{\nameref{sec:667}}
\begin{longtable}{l p{0.5cm} r}
گر با تو بگویم غم افزون شدهٔ من
&&
خونین شودت دل ز غم خون شدهٔ من
\\
زان روی که چون زلف تو تیره است و پریشان
&&
تو دانی و بس حال دگرگون شدهٔ من
\\
خاکی شده‌ام تا چو قدم رنجه کنی تو
&&
با خاک ببینی تن هامون شدهٔ من
\\
بیم است که ذرات جهان جمله بسوزد
&&
زین آتس از سینه به گردون شدهٔ من
\\
دی گفته‌ام ای جان سر زلف تو چه چیز است
&&
گو دام تو ای مرغ همایون شدهٔ من
\\
پرسیده‌ام ای لیلی من آن که ای تو
&&
گو آن تو ای عاشق مجنون شدهٔ من
\\
گفتم که دهانت چو الف هیچ ندارد
&&
گفتی بنگر طرهٔ چون نون شدهٔ من
\\
آن روز مبادا که بدین چشم ببینم
&&
هندو بچه‌ای را به شبیخون شدهٔ من
\\
جانا به خدا بخش دلم را که گزیده است
&&
مقبول تو را این دل مفتون شدهٔ من
\\
خون دل عطار چه ریزی که نیابی
&&
هم طبع سخن پرور موزون شدهٔ من
\\
\end{longtable}
\end{center}
