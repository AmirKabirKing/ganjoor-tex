\begin{center}
\section*{غزل شماره ۲۱۰: بی لعل لبت وصف شکر می‌نتوان کرد}
\label{sec:210}
\addcontentsline{toc}{section}{\nameref{sec:210}}
\begin{longtable}{l p{0.5cm} r}
بی لعل لبت وصف شکر می‌نتوان کرد
&&
بی عکس رخت فهم قمر می‌نتوان کرد
\\
چون صدقه ستانی است شکر لعل لبت را
&&
وصف لب لعلت به شکر می‌نتوان کرد
\\
مویی ز میان تو نشان می‌نتوان داد
&&
صفری ز دهان تو خبر می‌نتوان کرد
\\
برگ گلت آزرده شود از نظر تیز
&&
زان در رخ تو تیز نظر می‌نتوان کرد
\\
چون زلف تو زیر و زبری همه خلق است
&&
بی زلف تو دل زیر و زبر می‌نتوان کرد
\\
در واقعهٔ عشق رخت از همه نوعی
&&
کردیم بسی حیله دگر می‌نتوان کرد
\\
این کار به افسانه به سر می‌نتوان برد
&&
وافسانهٔ عشق تو زبر می‌نتوان کرد
\\
از تو کمری می‌نتوان بست به صد سال
&&
چون با تو به هم دست و کمر می‌نتوان کرد
\\
بی توشهٔ خون جگرم گر نخوری تو
&&
در وادی عشق تو سفر می‌نتوان کرد
\\
گفتی چو بسوزم جگرت آن تو باشم
&&
این سوخته را سوخته‌تر می‌نتوان کرد
\\
گفتی تو که مرغ منی آهنگ به من کن
&&
آهنگ بدین بال و بدین پر نتوان کرد
\\
کی در تو رسم گرد تو دریای پر آتش
&&
چون قصد تو از بیم خطر می‌نتوان کرد
\\
بی اشک چو خونم ز غم نقش خیالت
&&
نقاشی این روی چو زر می‌نتوان کرد
\\
ترک غم تو کرد مرا اشک چنین سرخ
&&
در گردن هندوی بصر می‌نتوان کرد
\\
چون هر چه که آن پیش من آید ز تو آید
&&
از آتش سوزنده حذر می‌نتوان کرد
\\
در پای غم از دست دل عاشق عطار
&&
افتاده چنانم که گذر می‌نتوان کرد
\\
\end{longtable}
\end{center}
