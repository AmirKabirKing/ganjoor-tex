\begin{center}
\section*{غزل شماره ۴۹: مرا در عشق او کاری فتادست}
\label{sec:049}
\addcontentsline{toc}{section}{\nameref{sec:049}}
\begin{longtable}{l p{0.5cm} r}
مرا در عشق او کاری فتادست
&&
که هر مویی به تیماری فتادست
\\
اگر گویم که می‌داند که در عشق
&&
چگونه مشکلم کاری فتادست
\\
مرا گوید اگر دانی وگرنه
&&
چنین در عشق بسیاری فتادست
\\
اگر گویم همه غمها به یک بار
&&
نصیب جان غمخواری فتادست
\\
مرا گوید مرا زین هیچ غم نیست
&&
همه غمها تو را آری فتادست
\\
چو خونم می‌بریزی زود بشتاب
&&
که الحق تیز بازاری فتادست
\\
مرا چون خون بریزی زود بفروش
&&
که بس نیکم خریداری فتادست
\\
مرا جانا ز عشقت بود صد بار
&&
به سرباری کنون باری فتادست
\\
دل مستم چو مرغ نیم بسمل
&&
به دام چون تو دلداری فتادست
\\
از آن دل دست باید شست دایم
&&
که در دست چو تو یاری فتادست
\\
کجا یابد گل وصل تو عطار
&&
که هر دم در رهش خاری فتادست
\\
\end{longtable}
\end{center}
