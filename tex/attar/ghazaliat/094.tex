\begin{center}
\section*{غزل شماره ۹۴: هر که درین دیرخانه مرد یگانه است}
\label{sec:094}
\addcontentsline{toc}{section}{\nameref{sec:094}}
\begin{longtable}{l p{0.5cm} r}
هر که درین دیرخانه مرد یگانه است
&&
تا به دم صور مست درد مغانه است
\\
ور به دم صور باهش آید ازین می
&&
نیست مبارز مخنث بن خانه است
\\
بر محک دیرخانه ناسره آید
&&
هر که گمان می‌برد که شیر ژیان است
\\
در بن این دیر درس عشق که گوید
&&
آنکه ز کونین بی نشان و نشانه است
\\
هر که دلی شاخ شاخ یافت چو شانه
&&
سالک آن زلف شاخ شاخ چو شانه‌است
\\
بر سر جمعی که بحر تشنهٔ آنهاست
&&
هرچه رود جز حدیث عشق فسانه است
\\
عاشق ره را هزار گونه جنیبت
&&
در پس و در پیش این طریق روانه است
\\
عشق که اندر خزانهٔ دو جهان نیست
&&
در بن صندوق سینه کنج خزانه است
\\
چون رخ معشوق را نه شبه و نه مثل است
&&
سلطنت عشق را نه سر نه کرانه است
\\
چشمه و کاریز و جوی و بحر یک آب است
&&
عاشق و معشوق و عشق هر سه بهانه است
\\
ذره اگر بی‌عدد به راه برآید
&&
ذره که باشد چو آفتاب عیان است
\\
هر دو جهان دام و دانه است ولیکن
&&
دیده و دل را وجود دام چو دانه است
\\
تا که زبانم به نطق عشق درآمد
&&
در دل عطار صد هزار زبانه است
\\
\end{longtable}
\end{center}
