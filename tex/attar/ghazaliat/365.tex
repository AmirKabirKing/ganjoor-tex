\begin{center}
\section*{غزل شماره ۳۶۵: گر رخ او ذره‌ای جمال نماید}
\label{sec:365}
\addcontentsline{toc}{section}{\nameref{sec:365}}
\begin{longtable}{l p{0.5cm} r}
گر رخ او ذره‌ای جمال نماید
&&
طلعت خورشید را زوال نماید
\\
ور ز رخش لحظه‌ای نقاب برافتد
&&
هر دو جهان بازی خیال نماید
\\
ذرهٔ سرگشته در برابر خورشید
&&
نیست عجب گر ضعیف حال نماید
\\
مرد مسلمان اگر ز زلف سیاهش
&&
کفر نیارد مرا محال نماید
\\
هر که به عشقش فروخت عقل به نقصان
&&
جمله نقصان او کمال نماید
\\
دوش غمش خون من بریخت و مرا گفت
&&
خون توام چشمه زلال نماید
\\
عشق حرامت بود اگر تو ندانی
&&
کین همه خون‌ها مرا حلال نماید
\\
در دهن مار نفس در بن چاه است
&&
هر که درین راه جاه و مال نماید
\\
گر تو درین راه خاک راه نگردی
&&
خاک تو را زود گوشمال نماید
\\
چند چو طاوس در مقابل خورشید
&&
مرغ وجود تو پر و بال نماید
\\
درنگر ای خودنمای تا سر مویی
&&
هر دو جهان پیش آن جمال نماید
\\
هر که درین دیرخانه دردکش افتاد
&&
کور شود از دو کون و لال نماید
\\
دیر که دولت سرای عالم عشق است
&&
دردکشی در هزار سال نماید
\\
مثل و مثالم طلب مکن تو درین دیر
&&
کاینه عطار را مثال نماید
\\
\end{longtable}
\end{center}
