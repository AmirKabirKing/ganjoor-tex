\begin{center}
\section*{غزل شماره ۳۰۰: عاشقانی کز نسیم دوست جان می‌پرورند}
\label{sec:300}
\addcontentsline{toc}{section}{\nameref{sec:300}}
\begin{longtable}{l p{0.5cm} r}
عاشقانی کز نسیم دوست جان می‌پرورند
&&
جمله وقت سوختن چون عود اندر مجمرند
\\
فارغند از عالم و از کار عالم روز و شب
&&
والهٔ راهی شگرف و غرق بحری منکرند
\\
هر که در عالم دویی می‌بیند آن از احولی است
&&
زانکه ایشان در دو عالم جز یکی را ننگرند
\\
گر صفتشان برگشاید پردهٔ صورت ز روی
&&
از ثری تا عرش اندر زیر گامی بسپرند
\\
آنچه می‌جویند بیرون از دوعالم سالکان
&&
خویش را یابند چون این پرده از هم بردرند
\\
هر دو عالم تخت خود بینند از روی صفت
&&
لاجرم در یک نفس از هر دو عالم بگذرند
\\
از ره صورت ز عالم ذره‌ای باشند و بس
&&
لیکن از راه صفت عالم به چیزی نشمرند
\\
فوق ایشان است در صورت دو عالم در نظر
&&
لیکن ایشان در صفت از هر دو عالم برترند
\\
عالم صغری به صورت عالم کبری به اصل
&&
اصغرند از صورت و از راه معنی اکبرند
\\
جمله غواصند در دریای وحدت لاجرم
&&
گرچه بسیارند لیکن در صفت یک گوهرند
\\
روز و شب عطار را از بهر شرح راه عشق
&&
هم به همت دل دهند و هم به دل جان پرورند
\\
\end{longtable}
\end{center}
