\begin{center}
\section*{غزل شماره ۹۶: خراباتی است پر رندان سرمست}
\label{sec:096}
\addcontentsline{toc}{section}{\nameref{sec:096}}
\begin{longtable}{l p{0.5cm} r}
خراباتی است پر رندان سرمست
&&
ز سر مستی همه نه نیست و نه هست
\\
فرو رفته همه در آب تاریک
&&
برآورده همه در کافری دست
\\
همه فارغ ز امروز و ز فردا
&&
همه آزاد از هشیار و از مست
\\
مگر افتاد پیر ما بر آن قوم
&&
مرقع چاک زد زنار در بست
\\
یقینش گشت کار و بی گمان شد
&&
درستش گشت فقر و توبه بشکست
\\
سیاهیی که در هر دو جهان بود
&&
فرود آمد به جان او و بنشست
\\
نقاب جان او شد آن سیاهی
&&
سیاهی آمد و در کفر پیوست
\\
چو آب خضر در تاریکی افتاد
&&
کنون هم او ز خلق و خلق ازو رست
\\
دل عطار خون گشت و حق اوست
&&
که تیری آنچنان ناگه ازو جست
\\
\end{longtable}
\end{center}
