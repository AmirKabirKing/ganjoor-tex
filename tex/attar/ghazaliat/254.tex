\begin{center}
\section*{غزل شماره ۲۵۴: چون عشق تو داعی عدم شد}
\label{sec:254}
\addcontentsline{toc}{section}{\nameref{sec:254}}
\begin{longtable}{l p{0.5cm} r}
چون عشق تو داعی عدم شد
&&
نتوان به وجود متهم شد
\\
جایی که وجود عین شرک است
&&
آنجا نتوان مگر عدم شد
\\
جانا می عشق تو دلی خورد
&&
کو محو وجود جام‌جم شد
\\
در پرتو نیستی عشقت
&&
بیش از همه بود و کم ز کم شد
\\
بر لوح فتاد ذره‌ای عشق
&&
لوح از سر بی‌خوردی قلم شد
\\
عشق تو دلم در آتش افکند
&&
تا گرد همه جهان علم شد
\\
دل در سر زلف تو قدم زد
&&
ایمانش نثار آن قدم شد
\\
دل در ره تو نداشت جز درد
&&
با درد دلم دریغ ضم شد
\\
رازی که دلم نهفته می‌داشت
&&
بر چهرهٔ من به خون رقم شد
\\
تا تو بنواختی چو چنگم
&&
رگ بر تن من چو زیر و بم شد
\\
عطار به نقد نیم جان داشت
&&
وان نیز به محنت تو هم شد
\\
\end{longtable}
\end{center}
