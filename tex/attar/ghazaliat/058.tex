\begin{center}
\section*{غزل شماره ۵۸: لعلت از شهد و شکر نیکوتر است}
\label{sec:058}
\addcontentsline{toc}{section}{\nameref{sec:058}}
\begin{longtable}{l p{0.5cm} r}
لعلت از شهد و شکر نیکوتر است
&&
رویت از شمس و قمر نیکوتر است
\\
خادم زلف تو عنبر لایق است
&&
هندوی رویت بصر نیکوتر است
\\
حلقه‌های زلف سرگردانت را
&&
سر ز پا و پا ز سر نیکوتر است
\\
از مفرح‌ها دل بیمار را
&&
از لب تو گلشکر نیکوتر است
\\
بوسه‌ای را می‌دهم جانی به تو
&&
کار با تو سر به سر نیکوتر است
\\
رستهٔ دندانت در بازار حسن
&&
استخوانی از گهر نیکوتر است
\\
هیچ بازاری چنان رسته ندید
&&
زانکه هریک زان دگر نیکوتر است
\\
عارضت کازرده گردد از نظر
&&
هر زمانی در نظر نیکوتر است
\\
چون کسی را بر میانت دست نیست
&&
دست با تو در کمر نیکوتر است
\\
چون لب لعلت نمک دارد بسی
&&
گر خورم چیزی جگر نیکوتر است
\\
کار رویم تا به تو رو کرده‌ام
&&
دور از رویت ز زر نیکوتر است
\\
گر دل عطار شد زیر و زبر
&&
دل ز تو زیر و زبر نیکوتر است
\\
\end{longtable}
\end{center}
