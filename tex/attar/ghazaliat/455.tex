\begin{center}
\section*{غزل شماره ۴۵۵: عشق جانی داد و بستد والسلام}
\label{sec:455}
\addcontentsline{toc}{section}{\nameref{sec:455}}
\begin{longtable}{l p{0.5cm} r}
عشق جانی داد و بستد والسلام
&&
چند گویی آخر از خود والسلام
\\
تو چنان انگار کاندر راه عشق
&&
یک نفس بود این شد آمد والسلام
\\
شیشه‌ای اندر دمید استاد کار
&&
بعد از آنش بر زمین زد والسلام
\\
گر تو اینجا ره بری با اصل کار
&&
رو که نبود چون تو بخرد والسلام
\\
ور بماند جان تو دربند خویش
&&
جان تو نانی نیرزد والسلام
\\
خلق را چون نیست بویی زین حدیث
&&
از یکی درگیر تا صد والسلام
\\
هر که را این ذوق نبود مرده‌ای است
&&
گر همه نیک است و گر بد والسلام
\\
عشق باید کز تو بستاند تورا
&&
چون تورا از خویش بستد والسلام
\\
عشق نبود آن که بنویسد قلم
&&
وانچه برخوانی ز کاغذ والسلام
\\
عشق دریایی است چون غرقت کند
&&
آن زمان عشق از تو زیبد والسلام
\\
ناخوشت می‌آید اما چون کنم
&&
عشق نبود در خوش آمد والسلام
\\
جان عطار از سپاه سر عشق
&&
در دو عالم شد سپهبد والسلام
\\
\end{longtable}
\end{center}
