\begin{center}
\section*{غزل شماره ۴۸۴: دوش چشم خود ز خون دریای گوهر یافتم}
\label{sec:484}
\addcontentsline{toc}{section}{\nameref{sec:484}}
\begin{longtable}{l p{0.5cm} r}
دوش چشم خود ز خون دریای گوهر یافتم
&&
منبع هر گوهری دریای دیگر یافتم
\\
زین چنین دریا که گرد من درآمد از سرشک
&&
گر کشتی بقا گرداب منکر یافتم
\\
موج این دریا چرا فوق‌الثریا نگذرد
&&
خاصه از تحت الثری قعرش فروتر یافتم
\\
در چنین بحری نیارم کرد عزم آشنا
&&
زانکه من این بحر را نه پا و نه سر یافتم
\\
یعلم الله گر به عمر خویش از بی قوتی
&&
هیچ عاشق را درین دریا شناگر یافتم
\\
شرم دارم کز گریبان سر برآرم خشک مو
&&
چون ز بحر چشم خود را دامن‌تر یافتم
\\
با چنین تردامنی بس ایمنم از خشک‌سال
&&
کز تر و وز خشک صد دریا میسر یافتم
\\
هفت دریا را زکوة از بحر چشم من گشاد
&&
لاجرم هر هفت را هفتاد کشور یافتم
\\
صد بیابان را که خشکی از لب خشکم گرفت
&&
سر به سر زین بحر پر خونم مصور یافتم
\\
در تعجب مانده‌ام از قطره‌های چشم خویش
&&
زانکه در هر قطره صد بحر مضمر یافتم
\\
ای عجب هر قطره اشکم که بگشادم ز هم
&&
قرب صد دریای خون در وی مجاور یافتم
\\
مد و جزر و قطره و دریا به هم هر دو یک است
&&
زانکه هر یک را مدار از بحر اخضر یافتم
\\
از کنار بحر اخضر دیده‌ام وز خون خویش
&&
از کنار خویش اکنون بحر احمر یافتم
\\
مردم آبی چشمم را درین دریای اشک
&&
گاه در خون غوطه گاه از آه منبر یافتم
\\
کی نماید آب رویم در چنین دریا که من
&&
روی خود چون مرد دریای مزعفر یافتم
\\
منت ایزد را که این دریا اگر آبم ببرد
&&
در عوض چشمم ازو دریای گوهر یافتم
\\
اندرین دریای خون هر قطرهٔ خونین که هست
&&
هر یکی را سوی دردی نیز رهبر یافتم
\\
خواستم تا ره برم بر روی آن دریای خون
&&
راه گم کردم که ره سرد صر صر یافتم
\\
دل که دارد تا بگردد گرد این دریا که من
&&
هر نفس در وی هزار و صد دلاور یافتم
\\
گر درین دریا کسی کشتی امید افکند
&&
باد سردش بادبان و صبر لنگر یافتم
\\
سینهٔ گردون که موجش آتشی زد زآفتاب
&&
روز و شب از رشک این بحرش پر اخگر یافتم
\\
گرچه دریای فلک را گوهر بسیار هست
&&
دایمش در جنب این دریا محقر یافتم
\\
زانکه این دریا ز دل می‌خیزد آن دریا ز خون
&&
درد را همچون عرض، دل را چو جوهر یافتم
\\
تا دلم بر روی دریا خون معنی گسترد
&&
خاطر عطار را چون قرص خاور یافتم
\\
\end{longtable}
\end{center}
