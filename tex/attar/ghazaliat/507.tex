\begin{center}
\section*{غزل شماره ۵۰۷: سواد خط تو چون نافع نظر دیدم}
\label{sec:507}
\addcontentsline{toc}{section}{\nameref{sec:507}}
\begin{longtable}{l p{0.5cm} r}
سواد خط تو چون نافع نظر دیدم
&&
روایتی که ازو رفت معتبر دیدم
\\
مرا چو زلف تو بر حرف می فرو گیرد
&&
حروف زلف تو برخواندم و خطر دیدم
\\
چه گویم از الف وصل تو که هیچ نداشت
&&
من اینکه هیچ نداشت از همه بتر دیدم
\\
تو را میان الف است و الف ندارد هیچ
&&
که من ورای الف هیچ در کمر دیدم
\\
کمند زلف تورا کافتاب دارد زیر
&&
هزار حلقه گرفتار یکدگر دیدم
\\
به حلق آمده جان در درون هر حلقه
&&
هزار عاشق گم کرده پا و سر دیدم
\\
سزد که هندی تو نام نرگس است از آنک
&&
دو هندوی رخ تو نرگس بصر دیدم
\\
چگونه شور نیارم ز آرزوی لبت
&&
کز آرزوی لبت شور در شکر دیدم
\\
ورای دولت وصل تو هیچ دولت نیست
&&
ولی چه سود که آن نیز برگذر دیدم
\\
چگونه وصل تو دارم طمع که من خود را
&&
ز تر و خشک لب‌خشک و چشم‌تر دیدم
\\
به عالم که ز وصلت سخن رود آنجا
&&
هزار تشنه به خون غرقه بیشتر دیدم
\\
ز مشرقی که ازو آفتاب حسن تو تافت
&&
هزار عرش اگر بود مختصر دیدم
\\
چو در صفات توام آبروی می‌بایست
&&
فرید را سخنی همچو آب زر دیدم
\\
\end{longtable}
\end{center}
