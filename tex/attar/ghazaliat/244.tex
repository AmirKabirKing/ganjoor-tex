\begin{center}
\section*{غزل شماره ۲۴۴: شکن زلف چو زنار بتم پیدا شد}
\label{sec:244}
\addcontentsline{toc}{section}{\nameref{sec:244}}
\begin{longtable}{l p{0.5cm} r}
شکن زلف چو زنار بتم پیدا شد
&&
پیر ما خرقهٔ خود چاک زد و ترسا شد
\\
عقل از طرهٔ او نعره‌زنان مجنون گشت
&&
روح از حلقهٔ او رقص‌کنان رسوا شد
\\
تا که آن شمع جهان پرده برافکند از روی
&&
بس دل و جان که چو پروانهٔ نا پروا شد
\\
هر که امروز معایینه رخ یار ندید
&&
طفل راه است اگر منتظر فردا شد
\\
همه سرسبزی سودای رخش می‌خواهم
&&
که همه عمر من اندر سر این سودا شد
\\
ساقیا جام می عشق پیاپی درده
&&
که دلم از می عشق تو سر غوغا شد
\\
نه چه حاجت به شراب تو که خود جان ز الست
&&
مست آمد به وجود از عدم و شیدا شد
\\
عاشقا هستی خود در ره معشوق بباز
&&
زانکه با هستی خود می‌نتوان آنجا شد
\\
روی صحرا چو همه پرتو خورشید گرفت
&&
کی تواند نفسی سایه بدان صحرا شد
\\
قطره‌ای بیش نه‌ای چند ز خویش اندیشی
&&
قطره‌ای چبود اگر گم شد و گر پیدا شد
\\
بود و نابود تو یک قطرهٔ آب است همی
&&
که ز دریا به کنار آمد و با دریا شد
\\
هرچه غیر است ز توحید به کل میل کشم
&&
زانکه چشم و دل عطار به کل بینا شد
\\
\end{longtable}
\end{center}
