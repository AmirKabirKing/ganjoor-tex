\begin{center}
\section*{غزل شماره ۳۸۱: دوش آمد و ز مسجدم اندر کران کشید}
\label{sec:381}
\addcontentsline{toc}{section}{\nameref{sec:381}}
\begin{longtable}{l p{0.5cm} r}
دوش آمد و ز مسجدم اندر کران کشید
&&
مویم گرفت و در صف دردی کشان کشید
\\
مستم بکرد و گرد جهانم به تک بتاخت
&&
تا نفس خوار خواری هر خاکدان کشید
\\
هر جزو من مشاهده تیغی دگر بخورد
&&
هر عضو من معاینه کوهی گران کشید
\\
گفتار خویش بگذر اگر می‌توان گذشت
&&
یعنی بلای من کش اگر می‌توان کشید
\\
گفتم هزار جان گرامی فدای تو
&&
از حکم تو چگونه توانم عنان کشید
\\
چون جان من به قوت او مرد کار شد
&&
از هرچه کرد عاقبتش بر کران کشید
\\
در بی نشانیم بنشاند و مرا بسوخت
&&
وانگه به گرد من رقمی بی نشان کشید
\\
عمری در آن میانه چو بودم به نیستی
&&
خوش خوش از آن میانه مرا در میان کشید
\\
چون چشم باز کرد و دل خویش را بدید
&&
سر بر خطش نهاد و خطی بر جهان کشید
\\
بس آه پرده‌سوز که از قعر دل بزد
&&
بس نعرهٔ عجیب که از مغز جان کشید
\\
پایان کار دل چو نگه کرد نیک نیک
&&
دلدار کرده بود، نه دل آنچه آن کشید
\\
عطار آشکار از آن دید نور عشق
&&
کان دلفروز سرمهٔ عشقش نهان کشید
\\
\end{longtable}
\end{center}
