\begin{center}
\section*{غزل شماره ۲۱۶: چو جان و دل ز می عشق دوش جوش بر آورد}
\label{sec:216}
\addcontentsline{toc}{section}{\nameref{sec:216}}
\begin{longtable}{l p{0.5cm} r}
چو جان و دل ز می عشق دوش جوش بر آورد
&&
دلم ز دست در افتاد و جان خروش بر آورد
\\
شراب عشق نخوردست هر که تا به قیامت
&&
ز ذوق مستی عشقت دمی به هوش بر آورد
\\
بیار دردی اندوه و صاف عشق دلم را
&&
که عقل پنبهٔ پندار خود ز گوش بر آورد
\\
بیار درد که معشوق من گرفت مرا مست
&&
میان درد و به بازار درد نوش بر آورد
\\
فکند خرقه و زنار داد و مست و خرابم
&&
به گرد شهر چو رندان می فروش بر آورد
\\
مرا به خلق نمود و برفت دل ز پی او
&&
چنان نمود که از راه دیده جوش بر آورد
\\
به یک شراب که در حلق پیر قوم فرو ریخت
&&
هزار نعره از آن پیر فوطه‌پوش بر آورد
\\
ز آرزوی رخ او دلم چنانست که بیزار
&&
هزار آه ز شوق رخ نکوش بر آورد
\\
سخن چگونه نیوشم برو که خاطر عطار
&&
مرا به عشق ز عقل سخن نیوش بر آورد
\\
\end{longtable}
\end{center}
