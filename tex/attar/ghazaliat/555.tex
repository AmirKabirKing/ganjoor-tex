\begin{center}
\section*{غزل شماره ۵۵۵: من این دانم که مویی می ندانم}
\label{sec:555}
\addcontentsline{toc}{section}{\nameref{sec:555}}
\begin{longtable}{l p{0.5cm} r}
من این دانم که مویی می ندانم
&&
بجز مرگ آرزویی می ندانم
\\
مرا مبشول مویی زانکه در عشق
&&
چنان غرقم که مویی می ندانم
\\
چنین رنگی که بر من سایه افکند
&&
ز دو کونش رکویی می ندانم
\\
چنانم در خم چوگان فگنده
&&
که پا و سر چو گویی می ندانم
\\
بسی بر بوی سر عشق رفتم
&&
نبردم بوی و بویی می ندانم
\\
بسی هر کار را روی است از ما
&&
به از تسلیم رویی می ندانم
\\
به از تسلیم و صبر و درد و خلوت
&&
درین ره چارسویی می ندانم
\\
شدم در کوی اهل دل چو خاکی
&&
که به زین کوی کویی می ندانم
\\
دلم را راه جوی عشق کردم
&&
که به زو راه جویی می ندانم
\\
درون دل بسی خود را بجستم
&&
که به زین جست و جویی می ندانم
\\
به خون دل بشستم دست از جان
&&
که به زین شست و شویی می ندانم
\\
بسی این راز نادانسته گفتم
&&
که به زین گفت و گویی می ندانم
\\
چو کردم جوی چشمان همچو عطار
&&
که به زین آب جویی می ندانم
\\
\end{longtable}
\end{center}
