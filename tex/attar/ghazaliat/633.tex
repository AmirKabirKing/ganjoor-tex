\begin{center}
\section*{غزل شماره ۶۳۳: ای نهان از دیده و در دل عیان}
\label{sec:633}
\addcontentsline{toc}{section}{\nameref{sec:633}}
\begin{longtable}{l p{0.5cm} r}
ای نهان از دیده و در دل عیان
&&
از جهان بیرون ولی در قعر جان
\\
هر کسی جان و جهان می‌خواندت
&&
خود تویی از هر دو بیرون جاودان
\\
هم جهان در جانت می‌جوید مدام
&&
هم ز جان می‌جویدت دایم جهان
\\
تو جهانی، لیک چون آیی پدید
&&
نه که جانی، لیک چون گردی نهان
\\
چون پدید آیی چو پنهانی مدام
&&
چون نهان گردی چو جاویدی عیان
\\
هم نهانی هم عیانی هر دویی
&&
هم نه اینی هم نه آن هم این هم آن
\\
جان ز پنهانی تو در داده تن
&&
تن ز پیدایی تو جان بر میان
\\
جان چو بی چون است چون آید به راه
&&
تن چو در جوش است چون یابد نشان
\\
چون ز تو جان نفی و تن اثبات یافت
&&
زین دو وصفند این دو جوهر در گمان
\\
هر دو گر بی‌وصف گردند آنگهی
&&
قرب بی وصفیت یابند آن زمان
\\
ز اشتیاق در وصلت چون قلم
&&
می‌روم بسته میان بر سر دوان
\\
من نیم تنها که ذرات دو کون
&&
جان‌فشانند این طلب را جان‌فشان
\\
آن چه جویم چون نیاید در طلب
&&
زان چه گویم چون نیاید در بیان
\\
بر زبانم چون بگردد نام وصل
&&
پر زبانه گرددم حالی زبان
\\
شرح این اسرار از عطار خواه
&&
او بگفت اسرار کو اسراردان
\\
\end{longtable}
\end{center}
