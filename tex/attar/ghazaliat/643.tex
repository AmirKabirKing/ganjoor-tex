\begin{center}
\section*{غزل شماره ۶۴۳: بندگی چیست به فرمان رفتن}
\label{sec:643}
\addcontentsline{toc}{section}{\nameref{sec:643}}
\begin{longtable}{l p{0.5cm} r}
بندگی چیست به فرمان رفتن
&&
پیش امر از بن دندان رفتن
\\
همه دشواری تو از طمع است
&&
ترک خود گفتن و آسان رفتن
\\
سر فدا کردن و سامان جستن
&&
وانگهی بی سر و سامان رفتن
\\
قابل امر شدن همچون گوی
&&
پس به یک ضربه به پایان رفتن
\\
از گران‌باری خود ترسیدن
&&
پس سبکبار به پیشان رفتن
\\
در پی شمع شریعت شب و روز
&&
همچو پروانه به پیمان رفتن
\\
آبرو باش تو در جوی طریق
&&
تا توانی تو بیابان رفتن
\\
برگ ره ساز که بی برگ رهی
&&
در چنین بادیه نتوان رفتن
\\
گر تو دنیا همه زندان دیدی
&&
فرخت باد ز زندان رفتن
\\
ور ندانی تو به جز دنیا هیچ
&&
مرده باید به فراوان رفتن
\\
تا کی از خواب درآموز آخر
&&
یک شب از گنبد گردان رفتن
\\
قرن‌ها شد که نمی‌آسایند
&&
از تو شب خفتن وزیشان رفتن
\\
عاشقان راست مسلم نه تو را
&&
در ره دوست به مژگان رفتن
\\
سر فدا کردن و چون عیاران
&&
جان به کف بر در جانان رفتن
\\
ترک عطار به گفتن کلی
&&
پس درین بادیه ترسان رفتن
\\
\end{longtable}
\end{center}
