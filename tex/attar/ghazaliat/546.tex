\begin{center}
\section*{غزل شماره ۵۴۶: کجایی ساقیا می ده مدامم}
\label{sec:546}
\addcontentsline{toc}{section}{\nameref{sec:546}}
\begin{longtable}{l p{0.5cm} r}
کجایی ساقیا می ده مدامم
&&
که من از جان غلامت را غلامم
\\
میم در ده تهی دستم چه داری
&&
که از خون جگر پر گشت جامم
\\
چه می‌خواهی ز جانم ای سمن بر
&&
که من بی روی تو خسته روانم
\\
چو بر جانم زدی شمشیر عشقت
&&
تمامم کن که رندی ناتمامم
\\
گهم زاهد همی خوانند و گه رند
&&
من مسکین ندانم تا کدامم
\\
ز ننگ من نگوید نام من کس
&&
چو من مردم چه مرد ننگ و نامم
\\
ز من چو شمع تا یک ذره باقی است
&&
نخواهد بود جز آتش مقامم
\\
مرا جز سوختن کاری دگر نیست
&&
بیا تا خوش بسوزم زانکه خامم
\\
دل عطار مرغی دانه چین است
&&
دریغ افتد چنین مرغی به دامم
\\
\end{longtable}
\end{center}
