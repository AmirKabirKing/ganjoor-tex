\begin{center}
\section*{غزل شماره ۱۳: برقع از ماه برانداز امشب}
\label{sec:013}
\addcontentsline{toc}{section}{\nameref{sec:013}}
\begin{longtable}{l p{0.5cm} r}
برقع از ماه برانداز امشب
&&
ابرش حسن برون تاز امشب
\\
دیده بر راه نهادم همه روز
&&
تا درآیی تو به اعزاز امشب
\\
من و تو هر دو تمامیم بهم
&&
هیچکس را مده آواز امشب
\\
کارم انجام نگیرد که چو دوش
&&
سرکشی می‌کنی آغاز امشب
\\
گرچه کار تو همه پرده‌دری است
&&
پرده زین کار مکن باز امشب
\\
تو چو شمعی و جهان از تو چو روز
&&
من چو پروانهٔ جانباز امشب
\\
همچو پروانه به پای افتادم
&&
سر ازین بیش میفراز امشب
\\
عمر من بیش شبی نیست چو شمع
&&
عمر شد، چند کنی ناز امشب
\\
بوده‌ام بی تو به‌صد سوز امروز
&&
چکنی کشتن من ساز امشب
\\
مرغ دل در قفس سینه ز شوق
&&
می‌کند قصد به پرواز امشب
\\
دانه از مرغ دلم باز مگیر
&&
که شد از بانگ تو دمساز امشب
\\
دل عطار نگر شیشه صفت
&&
سنگ بر شیشه مینداز امشب
\\
\end{longtable}
\end{center}
