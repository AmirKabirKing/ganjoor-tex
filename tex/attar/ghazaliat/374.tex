\begin{center}
\section*{غزل شماره ۳۷۴: واقعهٔ عشق را نیست نشانی پدید}
\label{sec:374}
\addcontentsline{toc}{section}{\nameref{sec:374}}
\begin{longtable}{l p{0.5cm} r}
واقعهٔ عشق را نیست نشانی پدید
&&
واقعه‌ای مشکل است بسته دری بی کلید
\\
تا تو تویی عاشقی از تو نیاید درست
&&
خویش بباید فروخت عشق بباید خرید
\\
پی نبری ذره‌ای زانچه طلب می‌کنی
&&
تا نشوی ذره‌وار زانچه تویی ناپدید
\\
واقعه‌ای بایدت تا بتوانی شنید
&&
حوصله‌ای بایدت تا بتوانی چشید
\\
تا بنبینی جمال عشق نگیرد کمال
&&
تا شنوی حسب حال راست بباید شنید
\\
کار کن ار عاشقی بار کش ار مفلسی
&&
زانکه بدین سرسری یار نگردد پدید
\\
سوخته شو تا مگر در تو فتد آتشی
&&
کاتش او چون بجست سوخته را بر گزید
\\
درد نگر رنج بین کانچه همی جسته‌ام
&&
راست که بنمود روی عمر به پایان رسید
\\
راست که سلطان عشق خیمه برون زد ز جان
&&
یار در اندر شکست عقل دم اندر کشید
\\
هر تر و خشکم که بود پاک به یکدم بسوخت
&&
پرده ز رخ برگرفت پردهٔ ما بر درید
\\
ای دل غافل مخسب خیز که معشوق ما
&&
در بر آن عاشقان پیش ز ما آرمید
\\
تا دل عطار گشت بلبل بستان درد
&&
هر دمش از عشق یار تازه گلی بشکفید
\\
\end{longtable}
\end{center}
