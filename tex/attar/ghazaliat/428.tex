\begin{center}
\section*{غزل شماره ۴۲۸: عاشقی نه دل نه دین می‌بایدش}
\label{sec:428}
\addcontentsline{toc}{section}{\nameref{sec:428}}
\begin{longtable}{l p{0.5cm} r}
عاشقی نه دل نه دین می‌بایدش
&&
من چنینم چون چنین می‌بایدش
\\
هر کجا رویی چو ماه آسمان است
&&
پیش رویش بر زمین می‌بایدش
\\
زن صفت هرگز نبیند آستانش
&&
مرد جان در آستین می‌بایدش
\\
می‌کشد هر روز عاشق صد هزار
&&
این چه باشد بیش ازین می‌بایدش
\\
شادمانی از غرور است از غرور
&&
دایما اندوهگین می‌بایدش
\\
برهم افتاده هزاران عرش هست
&&
حجره از قلب حزین می‌بایدش
\\
در ره عشقش چو آتش گرم خیز
&&
زانکه آتش همنشین می‌بایدش
\\
سر گنج او به خامی کس نیافت
&&
سوز عشق و درد دین می‌بایدش
\\
آه سرد از نفس خام آید پدید
&&
آه گرم آتشین می‌بایدش
\\
آن امانت کان دو عالم برنتافت
&&
هست صد عالم امین می‌بایدش
\\
گنج عشقش گر ندیدی کور شو
&&
زانکه کوری راه‌بین می‌بایدش
\\
سر گنج او همه عالم پر است
&&
اهل آن گنج یقین می‌بایدش
\\
می‌تواند داد هر دم خرمنی
&&
لیک مرد خوشه چین می‌بایدش
\\
شرق تا غرب جهان خوان می‌نهد
&&
و از تو یک نان جوین می‌بایدش
\\
اوست شاه تاج بخش اما ایاز
&&
در میان پوستین می‌بایدش
\\
گنج‌ها بخشید و از تو وام خواست
&&
تا شوی گستاخ این می‌بایدش
\\
امتحان را زلف هر دم کژ کند
&&
زانکه عاشق راستین می‌بایدش
\\
نه فلک فیروزه‌ای از کان اوست
&&
وز دل تو یک نگین می‌بایدش
\\
دست کس بر دامن او کی رسد
&&
لیک خلقی در کمین می‌بایدش
\\
عاشقان را دست و پای از کار شد
&&
ای عجب مرد آهنین می‌بایدش
\\
آفتابی ای عجب با ما بهم
&&
جای چرخ چارمین می‌بایدش
\\
ذره‌ای را بار می‌ندهد ولیک
&&
ذره ذره زیر زین می‌بایدش
\\
پای بگسل از دو عالم ای فرید
&&
کین قدر حبل المتین می‌بایدش
\\
\end{longtable}
\end{center}
