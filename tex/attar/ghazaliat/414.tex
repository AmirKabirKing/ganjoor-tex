\begin{center}
\section*{غزل شماره ۴۱۴: چند جویی در جهان یاری ز کس}
\label{sec:414}
\addcontentsline{toc}{section}{\nameref{sec:414}}
\begin{longtable}{l p{0.5cm} r}
چند جویی در جهان یاری ز کس
&&
یک کست در هر دو عالم یار بس
\\
تو چو طاوسی بدین ره در خرام
&&
کاندرین ره کم نیایی از مگس
\\
مرد باش و هر دو عالم ده طلاق
&&
پای در نه زانکه داری دست رس
\\
گر برآری یک نفس بی عشق او
&&
از تو با حضرت بنالد آن نفس
\\
هر نفس سرمایهٔ صد دولت است
&&
تا کی اندر یک نفس چندین هوس
\\
سرنگونساری تو از حرص توست
&&
باز کش آخر عنان را باز پس
\\
تا ز دانگی دوست تر داری دودانگ
&&
نیستی تو این سخن را هیچ کس
\\
گر گهر خواهی به دریا شو فرو
&&
بر سر دریا چه گردی همچو خس
\\
بر در او گر نداری حرمتی
&&
چون توانی رفت راه پر عسس
\\
چون تو ای عطار حرمت یافتی
&&
بر سر افلاک تازانی فرس
\\
\end{longtable}
\end{center}
