\begin{center}
\section*{غزل شماره ۳۳۹: هر گدایی مرد سلطان کی شود}
\label{sec:339}
\addcontentsline{toc}{section}{\nameref{sec:339}}
\begin{longtable}{l p{0.5cm} r}
هر گدایی مرد سلطان کی شود
&&
پشه‌ای آخر سلیمان کی شود
\\
نی عجب آن است کین مرد گدا
&&
چون که سلطان نیست سلطان کی شود
\\
بس عجب کاری است بس نادر رهی
&&
این چو عین آن بود آن کی شود
\\
گر بدین برهان کنی از من طلب
&&
این سخن روشن به برهان کی شود
\\
تا نگردی از وجود خود فنا
&&
بر تو این دشوار آسان کی شود
\\
گفتمش فانی شو و باقی تویی
&&
هر دو یکسان نیست یکسان کی شود
\\
گرچه هم دریای عمان قطره‌ای است
&&
قطره‌ای دریای عمان کی شود
\\
گر کسی را دیده دریابین نشد
&&
قطره‌بین باشد مسلمان کی شود
\\
تا نگردد قطره و دریا یکی
&&
سنگ کفرت لعل ایمان کی شود
\\
جمله یک خورشید می‌بینم ولیک
&&
می‌ندانم بر تو رخشان کی شود
\\
هر که خورشید جمال او ندید
&&
جان‌فشان بر روی جانان کی شود
\\
صد هزاران مرد می‌بینم ز عشق
&&
منتظر بنشسته تا جان کی شود
\\
چند اندایی به گل خورشید را
&&
گل بدین درگه نگهبان کی شود
\\
از کفی گل کان وجود آدم است
&&
آن چنان خورشید پنهان کی شود
\\
گر به کلی برنگیری گل ز راه
&&
پای در گل ره به پایان کی شود
\\
نه چه می‌گویم تو مرد این نه‌ای
&&
هر صبی رستم به دستان کی شود
\\
کی توانی شد تو مرد این حدیث
&&
هر مخنث مرد میدان کی شود
\\
تا نباشد همچو موسی عاشقی
&&
هر عصا در دست ثعبان کی شود
\\
عمرت ای عطار تاوان کرده‌ای
&&
بر تو آن خورشید تابان کی شود
\\
\end{longtable}
\end{center}
