\begin{center}
\section*{غزل شماره ۱۴: چه شاهدی است که با ماست در میان امشب}
\label{sec:014}
\addcontentsline{toc}{section}{\nameref{sec:014}}
\begin{longtable}{l p{0.5cm} r}
چه شاهدی است که با ماست در میان امشب
&&
که روشن است ز رویش همه جهان امشب
\\
نه شمع راست شعاعی، نه ماه را تابی
&&
نه زهره راست فروغی در آسمان امشب
\\
میان مجلس ما صورتی همی تابد
&&
که آفتاب شد از شرم او نهان امشب
\\
بسی سعادت از این شب پدید خواهد شد
&&
که هست مشتری و زهره را قران امشب
\\
شبی خوش است و ز اغیار نیست کس بر ما
&&
غنیمت است ملاقات دوستان امشب
\\
دمی خوش است مکن صبح دم دمی مردی
&&
که همدم است مرا یار مهربان امشب
\\
میان ما و تو امشب کسی نمی گنجد
&&
که خلوتی است مرا با تو در نهان امشب
\\
بساز مطرب از آن پرده‌های شور انگیز
&&
نوای تهنیت بزم عاشقان امشب
\\
همه حکایت مطبوع درد عطار است
&&
ترانهٔ خوش شیرین مطربان امشب
\\
\end{longtable}
\end{center}
