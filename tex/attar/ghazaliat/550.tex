\begin{center}
\section*{غزل شماره ۵۵۰: درد دل را دوا نمی‌دانم}
\label{sec:550}
\addcontentsline{toc}{section}{\nameref{sec:550}}
\begin{longtable}{l p{0.5cm} r}
درد دل را دوا نمی‌دانم
&&
گم شدم سر ز پا نمی‌دانم
\\
از می نیستی چنان مستم
&&
که صواب از خطا نمی‌دانم
\\
چند از من کنی سؤال که من
&&
درد را از دوا نمی‌دانم
\\
حل این مشکلم که افتادست
&&
در خلا و ملا نمی‌دانم
\\
به چه داد و ستد کنم با خلق
&&
که قبول از عطا نمی‌دانم
\\
هرچه از ماه تا به ماهی هست
&&
هیچ از خود جدا نمی‌دانم
\\
وانچه در اصل و فرع جمله تویی
&&
یا منم جمله یا نمی‌دانم
\\
گر یک است این همه یکی بگذار
&&
که عدد را قفا نمی‌دانم
\\
ور یکی نی و صد هزار است این
&&
صد و یک من چرا نمی‌دانم
\\
حیرتم کشت و من درین حیرت
&&
ره به کار خدا نمی‌دانم
\\
چشم دل را که نفس پردهٔ اوست
&&
در جهان توتیا نمی‌دانم
\\
آنچه عطار در پی آن رفت
&&
این زمان هیچ جا نمی‌دانم
\\
\end{longtable}
\end{center}
