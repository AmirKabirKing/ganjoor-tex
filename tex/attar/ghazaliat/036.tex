\begin{center}
\section*{غزل شماره ۳۶: تا آفتاب روی تو مشکین نقاب بست}
\label{sec:036}
\addcontentsline{toc}{section}{\nameref{sec:036}}
\begin{longtable}{l p{0.5cm} r}
تا آفتاب روی تو مشکین نقاب بست
&&
جان را شب اندر آمد و دل در عذاب بست
\\
ترسید زلف تو که کند چشم بد اثر
&&
خورشید را ز پردهٔ مشکین نقاب بست
\\
ناگاه آفتاب رخت تیغ برکشید
&&
پس تیغ تیز در تتق مشک ناب بست
\\
گر چهرهٔ تو در نگشادی فتوح را
&&
می‌خواست طرهٔ تو ره فتح باب بست
\\
عالم که بود تیره‌تر از زلف تو بسی
&&
روی تو کرد روشن و بر آفتاب بست
\\
تا هست روی تو که سر آفتاب داشت
&&
تا هست آب خضر که دل در سراب بست
\\
یک شعله آتش از رخ تو بر جهان فتاد
&&
سیلاب عشق در دل مشتی خراب بست
\\
بس در شگفت آمده‌ام تا مرا به حکم
&&
چشمت چگونه جست به یک غمزه خواب بست
\\
در خط شدم ز لعل لبت تا دهان تو
&&
از قفل لعل چو در در خوشاب بست
\\
جادو شنیده‌ام که ببندد به حکم آب
&&
وان بود نرگس تو که بر رویم آب بست
\\
نقاش صنع را همه لطف تو بود قصد
&&
بر گل نوشت نقش تو و بر گلاب بست
\\
چون خیمهٔ جمال تو از پیش برفگند
&&
از زلف عنبرین تو بر وی طناب بست
\\
جانی که گشت خیمه‌نشین جمال تو
&&
یکبارگی در هوس جاه و آب بست
\\
مسکین فرید کز همه عالم دلی که داشت
&&
بگسست پاک و در تو به صد اضطراب بست
\\
\end{longtable}
\end{center}
