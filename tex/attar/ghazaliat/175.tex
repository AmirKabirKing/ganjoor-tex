\begin{center}
\section*{غزل شماره ۱۷۵: اگر دردت دوای جان نگردد}
\label{sec:175}
\addcontentsline{toc}{section}{\nameref{sec:175}}
\begin{longtable}{l p{0.5cm} r}
اگر دردت دوای جان نگردد
&&
غم دشوار تو آسان نگردد
\\
که دردم را تواند ساخت درمان
&&
اگر هم درد تو درمان نگردد
\\
دمی درمان یک دردم نسازی
&&
که بر من درد صد چندان نگردد
\\
که یابد از سر زلف تو مویی
&&
که دایم بی سر و سامان نگردد
\\
که یابد از سر کوی تو گردی
&&
که همچون چرخ سرگردان نگردد
\\
که یابد از می عشق تو بویی
&&
که جانش مست جاویدان نگردد
\\
ندانم تا چه خورشیدی است عشقت
&&
که جز در آسمان جان نگردد
\\
دلا هرگز بقای کل نیابی
&&
که تا جان فانی جانان نگردد
\\
یقین می‌دان که جان در پیش جانان
&&
نیابد قرب تا قربان نگردد
\\
اگر قربان نگردد نیست ممکن
&&
که بر تو عمر تو تاوان نگردد
\\
چو خفاشی بمیری چشم بسته
&&
اگر خورشید تو رخشان نگردد
\\
اگر آدم کفی گل بود گو باش
&&
به گل خورشید تو پنهان نگردد
\\
در آن خورشید حیران گشت عطار
&&
چنان جایی کسی حیران نگردد
\\
\end{longtable}
\end{center}
