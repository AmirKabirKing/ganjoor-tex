\begin{center}
\section*{غزل شماره ۶۹: نیم شبی سیم برم نیم مست}
\label{sec:069}
\addcontentsline{toc}{section}{\nameref{sec:069}}
\begin{longtable}{l p{0.5cm} r}
نیم شبی سیم برم نیم مست
&&
نعره‌زنان آمد و در در نشست
\\
هوش بشد از دل من کو رسید
&&
جوش بخاست از جگرم کو نشست
\\
جام می آورد مرا پیش و گفت
&&
نوش کن این جام و مشو هیچ مست
\\
چون دل من بوی می عشق یافت
&&
عقل زبون گشت و خرد زیر دست
\\
نعره برآورد و به میخانه شد
&&
خرقه به خم در زد و زنار بست
\\
کم زن و اوباش شد و مهره دزد
&&
ره زن اصحاب شد و می‌پرست
\\
نیک و بد خلق به یکسو نهاد
&&
نیست شد و هست شد و نیست هست
\\
چون خودی خویش به کلی بسوخت
&&
از خودی خویش به کلی برست
\\
در بر عطار بلندی ندید
&&
خاک شد و در بر او گشت پست
\\
\end{longtable}
\end{center}
