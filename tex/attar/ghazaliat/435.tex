\begin{center}
\section*{غزل شماره ۴۳۵: می‌شد سر زلف در زمین کش}
\label{sec:435}
\addcontentsline{toc}{section}{\nameref{sec:435}}
\begin{longtable}{l p{0.5cm} r}
می‌شد سر زلف در زمین کش
&&
چون شرح دهم تو را که آن خوش
\\
از تیزی و تازگی که او بود
&&
گویی همه آب بود و آتش
\\
پر کرده ز چشم نرگسینش
&&
از تیر جفا هزار ترکش
\\
زیر قدشم هزار مشتاق
&&
از مردم دیده کرده مفرش
\\
جان همه کاملان ز زلفش
&&
همچون سر زلف او مشوش
\\
روی همه عاشقان ز عشقش
&&
از خون جگر شده منقش
\\
گل چهره و گل فشان و گل بوی
&&
مه طلعت و مه جبین و مهوش
\\
صد تشنه ز خون دیده سیراب
&&
از دشنهٔ چشم آن پریوش
\\
گه دل گه جان خروش می‌کرد
&&
کای غالیه زلف زلف برکش
\\
عطار ز زلف دلکش او
&&
تا حشر فتاده در کشاکش
\\
\end{longtable}
\end{center}
