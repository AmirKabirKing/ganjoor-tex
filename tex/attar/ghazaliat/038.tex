\begin{center}
\section*{غزل شماره ۳۸: چون به اصل اصل در پیوسته بی‌تو جان توست}
\label{sec:038}
\addcontentsline{toc}{section}{\nameref{sec:038}}
\begin{longtable}{l p{0.5cm} r}
چون به اصل اصل در پیوسته بی‌تو جان توست
&&
پس تویی بی‌تو که از تو آن تویی پنهان توست
\\
این تویی جزوی به نفس و آن تویی کلی به دل
&&
لیک تو نه این نه آنی بلکه هر دو آن توست
\\
تو درین و تو در آن تو کی رسی هرگز به تو
&&
زانکه اصل تو برون از نفس توست و جان توست
\\
بود تو اینجا حجاب افتاد و نابودت حجاب
&&
بود و نابودت چه خواهی کرد چون نقصان توست
\\
چون ز نابود و ز بود خویش بگذشتی تمام
&&
می‌ندانم تا به جز تو کیست کو سلطان توست
\\
هر چه هست و بود و خواهد بود هر سه ذره است
&&
ذره را منگر چو خورشید است کو پیشان توست
\\
تو مبین و تو مدان، گر دید و دانش بایدت
&&
کانچه تو بینی و تو دانی همه زندان توست
\\
بی سر و پا گر برون آیی ازین میدان چو گو
&&
تا ابد گر هست گویی در خم چوگان توست
\\
عین عینت چون به غیب الغیب در پوشیده‌اند
&&
پس یقین می‌دان که عینت غیب جاویدان توست
\\
صدر غیب‌الغیب را سلطان جاویدان تویی
&&
جز تو گر چیزی است در هر دو جهان دوران توست
\\
هم ز جسم و جان تو خاست این جهان و آن جهان
&&
هم بهشت و دوزخ از کفر تو و ایمان توست
\\
هم خداوندت سرشت و هم ملایک سجده کرد
&&
پس تویی معشوق خاص و چرخ سرگردان توست
\\
ای عجب تو کور خویش و ذره ذره در دو کون
&&
با هزاران دیده دایم تا ابد حیران توست
\\
بر دل عطار روشن گشت همچون آفتاب
&&
کاسمان نیلگون فیروزه‌ای از کان توست
\\
\end{longtable}
\end{center}
