\begin{center}
\section*{غزل شماره ۲۳۹: جان در مقام عشق به جانان نمی‌رسد}
\label{sec:239}
\addcontentsline{toc}{section}{\nameref{sec:239}}
\begin{longtable}{l p{0.5cm} r}
جان در مقام عشق به جانان نمی‌رسد
&&
دل در بلای درد به درمان نمی‌رسد
\\
درمان دل وصال و جمال است و این دو چیز
&&
دشوار می‌نماید و آسان نمی‌رسد
\\
ذوقی که هست جمله در آن حضرت است نقد
&&
وز صد یکی به عالم عرفان نمی‌رسد
\\
وز هرچه نقد عالم عرفان است از هزار
&&
جزوی به کل گنبد گردان نمی‌رسد
\\
وز صد هزار چیز که بر چرخ می‌رود
&&
صد یک به سوی جوهر انسان نمی‌رسد
\\
وز هرچه یافت جوهر انسان ز شوق و ذوق
&&
بویی به جنس جملهٔ حیوان نمی‌رسد
\\
مقصود آنکه از می ساقی حضرتش
&&
یک قطره درد درد به دو جهان نمی‌رسد
\\
چندین حجاب در ره تو خود عجب مدار
&&
گر جان تو به حضرت جانان نمی‌رسد
\\
جانان چو گنج زیر طلسم جهان نهاد
&&
گنجی که هیچ کس به سر آن نمی‌رسد
\\
زان می که می‌دهند از آن حسن قسم تو
&&
جز درد واپس آمد ایشان نمی‌رسد
\\
تو قانعی به لذت جسمی چو گاو و خر
&&
چون دست تو به معرفت جان نمی‌رسد
\\
تا کی چو کرم پیله تنی گرد خویشتن
&&
بر خود متن که خود به تو چندان نمی‌رسد
\\
خود را قدم قدم به مقام بر پران
&&
چندان پران که رخصت امکان نمی‌رسد
\\
زیرا که مرد راه نگیرد به هیچ روی
&&
یکدم قرار تا که به پیشان نمی‌رسد
\\
چندین هزار حاجب و دربان که در رهند
&&
شاید اگر کسی بر سلطان نمی‌رسد
\\
در راه او رسید قدم‌های سالکان
&&
وین راه بی‌کرانه به پایان نمی‌رسد
\\
پایان ندید کس ز بیابان عشق از آنک
&&
هرگز دلی به پای بیابان نمی‌رسد
\\
چندان به بوی وصل که در خود سفر کند
&&
عطار را به جز غم هجران نمی‌رسد
\\
\end{longtable}
\end{center}
