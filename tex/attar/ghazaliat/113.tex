\begin{center}
\section*{غزل شماره ۱۱۳: در ره عشاق نام و ننگ نیست}
\label{sec:113}
\addcontentsline{toc}{section}{\nameref{sec:113}}
\begin{longtable}{l p{0.5cm} r}
در ره عشاق نام و ننگ نیست
&&
عاشقان را آشتی و جنگ نیست
\\
عاشقی تردامنی گر تا ابد
&&
دامن معشوقت اندر چنگ نیست
\\
ننگ بادت هر دو عالم جاودان
&&
گر دو عالم بر تو بی‌او تنگ نیست
\\
پیک راه عاشقان دوست را
&&
در زمین و آسمان فرسنگ نیست
\\
مرغ دل از آشیانی دیگر است
&&
عقل و جان را سوی او آهنگ نیست
\\
ساقیا خون جگر در جام ریز
&&
تا شود پر خون دلی کز سنگ نیست
\\
آتش عشق و محبت برفروز
&&
تا بسوزد هر که او یک رنگ نیست
\\
کار ما بگذشت از فرهنگ و سنگ
&&
بیدلان عشق را فرهنگ نیست
\\
راست ناید نام و ننگ و عاشقی
&&
درد در ده جای نام و ننگ نیست
\\
نیست منصور حقیقی چون حسین
&&
هر که او از دار عشق آونگ نیست
\\
شد چنان عطار فارغ از جهان
&&
کآسمان با همتش هم سنگ نیست
\\
\end{longtable}
\end{center}
