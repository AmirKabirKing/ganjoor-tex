\begin{center}
\section*{غزل شماره ۷۲۰: ای یک کرشمهٔ تو صد خون حلال کرده}
\label{sec:720}
\addcontentsline{toc}{section}{\nameref{sec:720}}
\begin{longtable}{l p{0.5cm} r}
ای یک کرشمهٔ تو صد خون حلال کرده
&&
روی چو آفتابت ختم جمال کرده
\\
نیکوییی که هرگز نی روز دید نی شب
&&
هر سال ماه رویت با ماه و سال کرده
\\
خورشید طلعت تو ناگه فکنده عکسی
&&
اجسام خیره گشته ارواح حال کرده
\\
ماهی که قاف تا قاف از عکس اوست روشن
&&
چون روی تو بدیده پشتی چو دال کرده
\\
اول چو بدرهٔ سیم از نور بدر بوده
&&
وآخر ز شرم رویت خود را هلال کرده
\\
یک غمزهٔ ضعیفت صد سرکش قوی را
&&
هم دست خوش گرفته هم پایمال کرده
\\
روی تو مهر و مه را در زیر پر گرفته
&&
با هر یکی به خوبی صد پر و بال کرده
\\
زلف تو چون به شبرنگ آفاق در نوشته
&&
خورشید بر کمینه عزم زوال کرده
\\
دل را شده پریشان حالی و روزگاری
&&
تا از کمند زلفت مویی خیال کرده
\\
چون مرغ دل ز زلفت خسته برون ز در شد
&&
چندین مراغه در خون زان خط و خال کرده
\\
با آنکه بوی وصلت نه دل شنید و نه جان
&&
ما و دلی و جانی وقت وصال کرده
\\
گویاترین کسی را کو تیزبین‌تر آمد
&&
خط تو چشم بسته خال تو لال کرده
\\
شعر فرید کرده شرح لب تو شیرین
&&
تا او به وصف چشمت سحر حلال کرده
\\
\end{longtable}
\end{center}
