\begin{center}
\section*{غزل شماره ۶۲۳: ما ره ز قبله سوی خرابات می‌کنیم}
\label{sec:623}
\addcontentsline{toc}{section}{\nameref{sec:623}}
\begin{longtable}{l p{0.5cm} r}
ما ره ز قبله سوی خرابات می‌کنیم
&&
پس در قمارخانه مناجات می‌کنیم
\\
گاهی ز درد درد هیاهوی می‌زنیم
&&
گاهی ز صاف میکده هیهات می‌کنیم
\\
چون یک نفس به صومعه هشیار نیستیم
&&
مست و خراب کار خرابات می‌کنیم
\\
پیرا بیا ببین که جوانان رند را
&&
از بهر دردیی چه مراعات می‌کنیم
\\
طاماتیان ز دردی ما توبه می‌کنند
&&
ما بی‌نفاق توبه ز طامات می‌کنیم
\\
نه لاف پاک‌بازی و مردمی همی زنیم
&&
نه دعوی مقام و مقامات می‌کنیم
\\
ما را کجاست کشف و کرامات کین همه
&&
بر آرزوی کشف و کرامات می‌کنیم
\\
دردی کشیم و تا به نباشیم مرد دین
&&
بر اهل دین به کفر مباحات می‌کنیم
\\
گو بد کنید در حق ما خلق زانکه ما
&&
با کس نه داوری نه مکافات می‌کنیم
\\
ای ساقی اهل درد درین حلقه حاضرند
&&
می‌ده که کار می به مهمات می‌کنیم
\\
سلطان یک سوارهٔ نطع دو رنگ را
&&
بی یک پیاده بر رخ تو مات می‌کنیم
\\
ما شب‌روان بادیهٔ کعبهٔ دلیم
&&
با شاهدان روح ملاقات می‌کنیم
\\
در کسب علم و عقل چو عطار این زمان
&&
هم یک دو روز کار خرابات می‌کنیم
\\
\end{longtable}
\end{center}
