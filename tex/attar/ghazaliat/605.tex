\begin{center}
\section*{غزل شماره ۶۰۵: ما ز خرابات عشق مست الست آمدیم}
\label{sec:605}
\addcontentsline{toc}{section}{\nameref{sec:605}}
\begin{longtable}{l p{0.5cm} r}
ما ز خرابات عشق مست الست آمدیم
&&
نام بلی چون بریم چون همه مست آمدیم
\\
پیش ز ما جان ما خورد شراب الست
&&
ما همه زان یک شراب مست الست آمدیم
\\
خاک بد آدم که دوست جرعه بدان خاک ریخت
&&
ما همه زان جرعهٔ دوست به دست آمدیم
\\
ساقی جام الست چون و سقیهم بگفت
&&
ما ز پی نیستی عاشق هست آمدیم
\\
دوست چهل بامداد در گل ما داشت دست
&&
تا چو گل از دست دوست دست به دست آمدیم
\\
شست درافکند یار بر سر دریای عشق
&&
تا ز پی چل صباح جمله به شست آمدیم
\\
خیز و دلا مست شو از می قدسی از آنک
&&
ما نه بدین تیره جای بهر نشست آمدیم
\\
دوست چو جبار بود هیچ شکستی نداشت
&&
گفت شکست آورید ما به شکست آمدیم
\\
گوهر عطار یافت قدر و بلندی ز عشق
&&
گرچه ز تأثیر جسم جوهر پست آمدیم
\\
\end{longtable}
\end{center}
