\begin{center}
\section*{غزل شماره ۲۵۹: در راه عشق هر دل کو خصم خویشتن شد}
\label{sec:259}
\addcontentsline{toc}{section}{\nameref{sec:259}}
\begin{longtable}{l p{0.5cm} r}
در راه عشق هر دل کو خصم خویشتن شد
&&
فارغ ز نیک و بد گشت ایمن ز ما و من شد
\\
نی نی که نیست کس را جز نام عشق حاصل
&&
کان دم که عشق آمد از ننگ تن به تن شد
\\
در تافت روز اول یک ذره عشق از غیب
&&
افلاک سرنگون گشت ارواح نعره‌زن شد
\\
آن ذره عشق ناگه چون سینه‌ها ببویید
&&
کس را ندید محرم با جای خویشتن شد
\\
زان ذره عشق خلقی در گفتگو فتادند
&&
وان خود چنان که آمد هم بکر با وطن شد
\\
در عشق زنده باید کز مرده هیچ ناید
&&
عاشق نمرد هرگز کو زنده در کفن شد
\\
کو زنده‌ای که هرگز از بهر نفس کشتن
&&
مردود خلق آمد رسوای انجمن شد
\\
هر زنده را کزین می بویی نصیب آمد
&&
هر موی بر تن او گویای بی سخن شد
\\
چون جان و تن درین ره دو بند صعب آمد
&&
عطار همچو مردان در خون جان و تن شد
\\
\end{longtable}
\end{center}
