\begin{center}
\section*{غزل شماره ۷۲۴: ای ز صفات لبت عقل به جان آمده}
\label{sec:724}
\addcontentsline{toc}{section}{\nameref{sec:724}}
\begin{longtable}{l p{0.5cm} r}
ای ز صفات لبت عقل به جان آمده
&&
از سر زلفت شکست در دو جهان آمده
\\
چشمهٔ آب حیات بی‌لب سیراب تو
&&
تشنه دایم شده خشک دهان آمده
\\
نرگس خون‌ریز تو تیر جفا ریخته
&&
دلشدگان تورا کار به جان آمده
\\
پستهٔ تو در سخن تا شکرافشان شده
&&
عقل ز تشویر او بسته دهان آمده
\\
در طلب روی تو گرد جهان فراخ
&&
ابرش فکرت مدام تنگ‌عنان آمده
\\
عاشقت از جان و دل با دل و جان برطبق
&&
پیش نثار رخت نعره‌زنان آمده
\\
تا دل پر خون من جسته ز وصلت نشان
&&
نام دلم گم شده و او به نشان آمده
\\
چون شده روشن که نیست راه به تو تا ابد
&&
جملهٔ عشاق را ره به کران آمده
\\
تا که فتاده ز تو در دل عطار شور
&&
مرغ دلش در قفس در خفقان آمده
\\
\end{longtable}
\end{center}
