\begin{center}
\section*{غزل شماره ۴۳۷: ترسا بچهٔ شکر لبم دوش}
\label{sec:437}
\addcontentsline{toc}{section}{\nameref{sec:437}}
\begin{longtable}{l p{0.5cm} r}
ترسا بچهٔ شکر لبم دوش
&&
صد حلقهٔ زلف در بناگوش
\\
صد پیر قوی به حلقه می‌داشت
&&
زان حلقهٔ زلف حلقه در گوش
\\
آمد بر من شراب در دست
&&
گفتا که به یاد من کن این نوش
\\
در پرده اگر حریف مایی
&&
چون می‌نوشی خموش و مخروش
\\
زیرا که دلی نگشت گویا
&&
تا مرد زبان نکرد خاموش
\\
دل چون بشنود این سخن زود
&&
ناخورده شراب گشت مدهوش
\\
چون بستدم آن شراب و خوردم
&&
در سینهٔ من فتاد صد جوش
\\
دادم همه نام و ننگ بر باد
&&
کردم همه نیک و بد فراموش
\\
از دست بشد مرا دل و جان
&&
وز پای درآمدم تن و توش
\\
یک قطره از آن شراب مشکل
&&
آورد دو عالمم در آغوش
\\
یک ذره سواد فقر در تافت
&&
شد هر دو جهان از آن سیه‌پوش
\\
جانم ز سر دو کون برخاست
&&
در شیوهٔ فقر شد وفا کوش
\\
هر که بخرد به جان و دل فقر
&&
بر جان و دلش دو کون بفروش
\\
ور دین تو نیست دین عطار
&&
کفر آیدت این حدیث منیوش
\\
\end{longtable}
\end{center}
