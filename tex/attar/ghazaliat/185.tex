\begin{center}
\section*{غزل شماره ۱۸۵: لب تو مردمی دیده دارد}
\label{sec:185}
\addcontentsline{toc}{section}{\nameref{sec:185}}
\begin{longtable}{l p{0.5cm} r}
لب تو مردمی دیده دارد
&&
ولی زلف تو سر گردیده دارد
\\
که داند تا سر زلف تو در چین
&&
چه زنگی بچه ناگردیده دارد
\\
چو حسنت می‌نگنجد در جهانی
&&
به جانم چون رهی دزدیده دارد
\\
چو مژه بر سر چشمت نشاند
&&
سر یک مژه هر کو دیده دارد
\\
وصال تو مگر در چین زلف است
&&
که چندین پردهٔ دریده دارد
\\
کنون هر کو به جان وصل تو می‌جست
&&
اگر دارد طمع بریده دارد
\\
از آن شوریده‌ام از پستهٔ تو
&&
که شور او بسی شوریده دارد
\\
خیال روی تو استاد در قلب
&&
ز بهر کین زره پوشیده دارد
\\
اگر آهنگ خون ریزی ندارد
&&
چرا چندین به خون غلطیده دارد
\\
فرید از تو دلی دارد چو بحری
&&
که بحری خون چنین جوشیده دارد
\\
\end{longtable}
\end{center}
