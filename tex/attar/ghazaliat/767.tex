\begin{center}
\section*{غزل شماره ۷۶۷: ای کاش درد عشقت درمان‌پذیر بودی}
\label{sec:767}
\addcontentsline{toc}{section}{\nameref{sec:767}}
\begin{longtable}{l p{0.5cm} r}
ای کاش درد عشقت درمان‌پذیر بودی
&&
یا از تو جان و دل را یک‌دم گزیر بودی
\\
در آرزوی رویت چندین غمم نبودی
&&
گر در همه جهانت مثل و نظیر بودی
\\
می‌خواستم که جان را بر روی تو فشانم
&&
ور بر فشاندمی جان چیزی حقیر بودی
\\
عشقت مرا نکشتی گر یک‌دمی وصالت
&&
یا پایمرد گشتی یا دستگیر بودی
\\
کی پای دل به سختی در قیر باز ماندی
&&
گر نی به گرد ماهت زلف چو قیر بودی
\\
زان می که خورد حلاج گر هر کسی بخوردی
&&
بر دار صد هزاران برنا و پیر بودی
\\
گفتی که با تو روزی وصلی به هم برآرم
&&
این وعده بس خوشستی گر دلپذیر بودی
\\
گر شاد کردیی تو عطار را به وصلت
&&
نه جان نژند گشتی نه دل اسیر بودی
\\
\end{longtable}
\end{center}
