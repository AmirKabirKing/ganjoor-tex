\begin{center}
\section*{غزل شماره ۸۷: رهی کان ره نهان اندر نهان است}
\label{sec:087}
\addcontentsline{toc}{section}{\nameref{sec:087}}
\begin{longtable}{l p{0.5cm} r}
رهی کان ره نهان اندر نهان است
&&
چو پیدا شد عیان اندر عیان است
\\
چه می‌گویم چه پیدا و چه پنهان
&&
که این بالای پیدا و نهان است
\\
چه می‌گویم چه بالا و چه پستی
&&
که این بیرون ازین است و از آن است
\\
چه می‌گویم چه بیرون چه درون است
&&
که بیرون و درون گفت زبان است
\\
چگویم آنچه هرگز کس نگفته است
&&
چه دانم آنکه هرگز کس ندانست
\\
گمانی چون برم چون کس نبرده است
&&
نشانی چون دهم چون بی‌نشان است
\\
مکن روباه بازی شیر مردا
&&
خموشی پیشه کن کین ره عیان است
\\
برو از پوست بیرون آی کین کار
&&
نه کار توست کار مغز جان است
\\
برو عطار و ترک این سخن گیر
&&
که این را مستمع در لامکان است
\\
\end{longtable}
\end{center}
