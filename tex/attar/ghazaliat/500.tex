\begin{center}
\section*{غزل شماره ۵۰۰: تا جمال تو بدیدم مست و مدهوش آمدم}
\label{sec:500}
\addcontentsline{toc}{section}{\nameref{sec:500}}
\begin{longtable}{l p{0.5cm} r}
تا جمال تو بدیدم مست و مدهوش آمدم
&&
عاشق لعل شکربارش گهر پوش آمدم
\\
نامهٔ عشقت بخواندم عاشق دردت شدم
&&
حلقهٔ زلفت بدیدم حلقه در گوش آمدم
\\
سرخ رو از چشم بودم پیش ازین از خون دل
&&
زردرو از سبزهٔ آن چشمهٔ نوش آمدم
\\
شغبهٔ آن شکرستان شکربار ار شدم
&&
فتنهٔ آن سنبلستان بناگوش آمدم
\\
خواب خرگوشم بسی دادی ندانستم ولیک
&&
هم به آخر در جوال خواب خرگوش آمدم
\\
کی بگردانم ز تو از هر جفایی روی از آنک
&&
تو جفا کیش آمدی و من وفا کوش آمدم
\\
عشق تو کاندر میان جان من شد معتکف
&&
کی فراموشش کنم گر من فراموش آمدم
\\
وصف می‌کرد از تو عطار اندر آفاق جهان
&&
نک سخن ناگفته حالی گنگ و مدهوش آمدم
\\
\end{longtable}
\end{center}
