\begin{center}
\section*{غزل شماره ۷۶۶: گر مرد این حدیثی زنار کفر بندی}
\label{sec:766}
\addcontentsline{toc}{section}{\nameref{sec:766}}
\begin{longtable}{l p{0.5cm} r}
گر مرد این حدیثی زنار کفر بندی
&&
دین از تو دور دور است بر خویشتن چه خندی
\\
از کفر ناگذشته دعوی دین مکن تو
&&
گر محو کفر گردی بنیاد دین فکندی
\\
اندر نهاد گبرت پنجه هزار دیوست
&&
زنار کفر تو خود گبری اگر نبندی
\\
هر ذره‌ای ز عالم سدی است در ره تو
&&
از ذره ذره بگذر گر مرد هوشمندی
\\
چون گویمت که خود را می‌سوز چون سپندی
&&
زیرا که چشم بد را تو در پی سپندی
\\
مردانه پای در نه گر شیر مرد راهی
&&
ورنه به گوشه‌ای رو گر مرد مستمندی
\\
ای پست نفس مانده تا کنی تو دعوی
&&
کافزون ز عالم آمد جان من از بلندی
\\
هیچ است هر دو عالم در جنب این حقیقت
&&
آخر ز هر دو عالم خود را ببین که چندی
\\
عطار مرد عشقی فانی شو از دو عالم
&&
کز لنگر نهادت در بند تخته بندی
\\
\end{longtable}
\end{center}
