\begin{center}
\section*{غزل شماره ۸۰۴: دردی است درین دلم نهانی}
\label{sec:804}
\addcontentsline{toc}{section}{\nameref{sec:804}}
\begin{longtable}{l p{0.5cm} r}
دردی است درین دلم نهانی
&&
کان درد مرا دوا تو دانی
\\
تو مرهم درد بیدلانی
&&
دانم که مرا چنین نمانی
\\
من بندهٔ بی کس ضعیفم
&&
تو یار کسان بی کسانی
\\
گر مورچه‌ای در تو کوبد
&&
آنی تو که ضایعش نمانی
\\
از من گنه آید و من اینم
&&
وز تو کرم آید و تو آنی
\\
یارب به در که باز گردم
&&
گر تو ز در خودم برانی
\\
از خواندن و راندنم چه باک است
&&
خواه این کن و خواه آن تو دانی
\\
گویم «ارنی» و زار گریم
&&
ترسم ز جواب «لن ترانی»
\\
پیری بشنید و جان به حق داد
&&
عطار سخن مگو که جانی
\\
\end{longtable}
\end{center}
