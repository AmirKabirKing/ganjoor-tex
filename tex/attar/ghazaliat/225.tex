\begin{center}
\section*{غزل شماره ۲۲۵: دست در دامن جان خواهم زد}
\label{sec:225}
\addcontentsline{toc}{section}{\nameref{sec:225}}
\begin{longtable}{l p{0.5cm} r}
دست در دامن جان خواهم زد
&&
پای بر فرق جهان خواهم زد
\\
اسب بر جسم و جهت خواهم تاخت
&&
بانگ بر کون و مکان خواهم زد
\\
وانگه آن دم که میان من و اوست
&&
از همه خلق نهان خواهم زد
\\
چون مرا نام و نشان نیست پدید
&&
دم ز بی نام و نشان خواهم زد
\\
هان مبر ظن که من سوخته دل
&&
آن دم از کام و زبان خواهم زد
\\
تن پلید است بخواهم انداخت
&&
وثان دم پاک به جان خواهم زد
\\
در شکم چون زند آن طفل نفس
&&
من بی‌خویش چنان خواهم زد
\\
از دلم مشعله‌ای خواهم ساخت
&&
نفس شعله‌فشان خواهم زد
\\
از سر صدق و صفا صبح صفت
&&
آن نفس نی به دهان خواهم زد
\\
چون عیان گشت مرا آنچه مپرس
&&
لاف از عین عیان خواهم زد
\\
لاف این نیست یقین است یقین
&&
پس چرا دم به گمان خواهم زد
\\
من نیم مطبخی زیر و زبر
&&
دم بی کفک و دخان خواهم زد
\\
چون سر و پای روان نیست مرا
&&
قدم از پای روان خواهم زد
\\
خصم نفس است گرم عشوه دهد
&&
بر سر خصم سنان خواهم زد
\\
تا که از وسوسهٔ نفس پلید
&&
نفس از سود و زیان خواهم زد
\\
به خرابات فرو خواهم شد
&&
دست بر رطل گران خواهم زد
\\
آن دم انگشت گزان می‌زده‌ام
&&
این دم انگشت زنان خواهم زد
\\
تیر را پیک بلا خواهم ساخت
&&
تیغ را زخم میان خواهم زد
\\
فتنه بیدار چنان خواهم کرد
&&
کز سر فتنه نشان خواهم زد
\\
هر شبان موسی عمران نبود
&&
من دم گرگ شبان خواهم زد
\\
تا کی از شعر فرید آتش عشق
&&
در همه نطق و بیان خواهم زد
\\
\end{longtable}
\end{center}
