\begin{center}
\section*{غزل شماره ۵۶۳: درین نشیمن خاکی بدین صفت که منم}
\label{sec:563}
\addcontentsline{toc}{section}{\nameref{sec:563}}
\begin{longtable}{l p{0.5cm} r}
درین نشیمن خاکی بدین صفت که منم
&&
میان نفس و هوا دست و پای چند زنم
\\
هزار بار برآمد مرا که یکباری
&&
ز دست چرخ فلک جامه پاره پاره کنم
\\
گره چگونه گشایم ز سر خود که ز چرخ
&&
هزار گونه گره در فتاده در سخنم
\\
ز هر کسی چه شکایت کنم چو می‌دانم
&&
که جرم من ز من است و بلای خویش منم
\\
به هیچ روی مرا نیست رستگاری روی
&&
که هست دشمن من در میان پیرهنم
\\
حساب بر نتوانم گرفت بر خود از آنک
&&
به هر حساب که هستم اسیر خویشتنم
\\
هزار بار به یک روز عقل را ز صراط
&&
به قعر دوزخ نفس و هوا فرو فکنم
\\
اگر موافق طبعم ندیم ابلیسم
&&
وگر متابع نفسم حریف اهرمنم
\\
به گرد بلبل روحم قرار چون گیرد
&&
میان خار چو گلزار جان بود وطنم
\\
سزد که پیرهن کاغذین کند عطار
&&
که شد ز نفس بدآموز پیرهن کفنم
\\
\end{longtable}
\end{center}
