\begin{center}
\section*{غزل شماره ۷۸۳: گر مرد این حدیثی زین باده مست باشی}
\label{sec:783}
\addcontentsline{toc}{section}{\nameref{sec:783}}
\begin{longtable}{l p{0.5cm} r}
گر مرد این حدیثی زین باده مست باشی
&&
صد توبه در زمانی بر هم شکست باشی
\\
نه مست بودن از می کار تنگدلان است
&&
گر هوشیار عشقی از دوست مست باشی
\\
تا کی ز ناتمامی در حلقهٔ تمامان
&&
گه خودنمای گردی گه خود پرست باشی
\\
آخر دمی چنان شو کز دست ساقی جان
&&
جامی بخورد باشی وز خود برست باشی
\\
ای بر کنار مانده برخیز از دو عالم
&&
تا در میان مردان ز اهل نشست باشی
\\
در صحبت بلندان خود را بلند گردان
&&
تا کی ز نفس خودبین چون خاک پست باشی
\\
گر کاملی درین ره چون عاشقان کامل
&&
از خویش نیست گردی وز دوست هست باشی
\\
تا بسته‌ای به مویی زان موی در حجابی
&&
چه کوهی و چه کاهی چون پای‌بست باشی
\\
عطار اگر بر اصلی اصلا ز خود فنا شو
&&
کانگه که نیست گردی با او به دست باشی
\\
\end{longtable}
\end{center}
