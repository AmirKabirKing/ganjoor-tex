\begin{center}
\section*{غزل شماره ۲۰۶: زلف تو مرا بند دل و غارت جان کرد}
\label{sec:206}
\addcontentsline{toc}{section}{\nameref{sec:206}}
\begin{longtable}{l p{0.5cm} r}
زلف تو مرا بند دل و غارت جان کرد
&&
عشق تو مرا رانده به گرد دو جهان کرد
\\
گویی که بلا با سر زلف تو قرین بود
&&
گویی که قضا با غم عشق تو قران کرد
\\
اندر طلب زلف تو عمری دل من رفت
&&
چون یافت ره زلف تو یک حلقه نشان کرد
\\
وقت سحری باد درآمد ز پس و پیش
&&
وان حلقه ز چشم من سرگشته نهان کرد
\\
چون حلقهٔ زلف تو نهان گشت دلم برد
&&
چون برد دلم آمد و آهنگ به جان کرد
\\
جان نیز به سودای سر زلف تو برخاست
&&
پیش آمد و عمری چو دلم در سر آن کرد
\\
ناگه سر مویی ز سر زلف تو در تاخت
&&
جان را ز پس پردهٔ خود موی کشان کرد
\\
فی‌الجمله بسی تک که زدم تا که یقین گشت
&&
کز زلف تو یک موی نشان می نتوان کرد
\\
گرچه نتوان کرد بیان سر زلفت
&&
آن مایه که عطار توانست بیان کرد
\\
\end{longtable}
\end{center}
