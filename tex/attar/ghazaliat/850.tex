\begin{center}
\section*{غزل شماره ۸۵۰: رخ تو چگونه ببینم که تو در نظر نیایی}
\label{sec:850}
\addcontentsline{toc}{section}{\nameref{sec:850}}
\begin{longtable}{l p{0.5cm} r}
رخ تو چگونه بینم که تو در نظر نیایی
&&
نرسی به کس چو دانم که تو خود به سر نیایی
\\
وطن تو از که جویم که تو در وطن نگنجی
&&
خبر تو از که پرسم که تو در خبر نیایی
\\
چه کسی تو باری ای جان که ز غایت کمالت
&&
چو به وصف تو درآیم تو به وصف در نیایی
\\
گهری عجب تر از تو نشنیدم و ندیدیم
&&
که به بحر در نگنجی و ز قعر بر نیایی
\\
چو به پرده در نشینی چه بود که عاشقان را
&&
چو شکر همی نبخشی نمک جگر نیایی
\\
همه دل فرو گرفتی به تو کی رسم که گر من
&&
در دل بسی بکوبم تو ز دل به در نیایی
\\
تو بیا که جان عطار اگرت خوش آمد از وی
&&
به تو بخش آن ولیکن تو بدین قدر نیایی
\\
\end{longtable}
\end{center}
