\begin{center}
\section*{غزل شماره ۵۲۴: نظری به کار من کن که ز دست رفت کارم}
\label{sec:524}
\addcontentsline{toc}{section}{\nameref{sec:524}}
\begin{longtable}{l p{0.5cm} r}
نظری به کار من کن که ز دست رفت کارم
&&
به کسم مکن حواله که به جز تو کس ندارم
\\
منم و هزار حسرت که در آرزوی رویت
&&
همه عمر من برفت و بنرفت هیچ کارم
\\
اگر به دستگیری بپذیری اینت منت
&&
واگر نه رستخیزی ز همه جهان برآرم
\\
چه کمی درآید آخر به شرابخانهٔ تو
&&
اگر از شراب وصلت ببری ز سر خمارم
\\
چو نیم سزای شادی ز خودم مدار بی غم
&&
که درین چنین مقامی غم توست غمگسارم
\\
ز غم تو همچو شمعم که چو شمع در غم تو
&&
چو نفس زنم بسوزم چو بخندم اشکبارم
\\
چو زکار شد زبانم بروم به پیش خلقی
&&
غم تو به خون دیده همه بر رخم نگارم
\\
ز توام من آنچه هستم که تو گرنه‌ای نیم من
&&
که تویی که آفتابی و منم که ذره‌وارم
\\
اگر از تو جان عطار اثر کمال یابد
&&
منم آنکه از دو عالم به کمال اختیارم
\\
\end{longtable}
\end{center}
