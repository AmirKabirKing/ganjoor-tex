\begin{center}
\section*{غزل شماره ۷۷۲: جانا دهنی چو پسته داری}
\label{sec:772}
\addcontentsline{toc}{section}{\nameref{sec:772}}
\begin{longtable}{l p{0.5cm} r}
جانا دهنی چو پسته داری
&&
در پسته گهر دو رسته داری
\\
صد شور به پسته در فتاده است
&&
زان قند که مغز پسته داری
\\
قندیم فرست و مرهمی ساز
&&
زین بیش مرا چه خسته داری
\\
در هر سر موی زلف شستت
&&
صد فتنهٔ نانشسته داری
\\
گفتی به درست عهد کردم
&&
صد عهد چنین شکسته داری
\\
در تاز و جهان بگیر کز حسن
&&
صد ابلق تنگ بسته داری
\\
یک گل ندهی ز رخ به عطار
&&
وانگاه هزار دسته داری
\\
\end{longtable}
\end{center}
