\begin{center}
\section*{غزل شماره ۲۲۱: چون زلف بیقرارش بر رخ قرار گیرد}
\label{sec:221}
\addcontentsline{toc}{section}{\nameref{sec:221}}
\begin{longtable}{l p{0.5cm} r}
چون زلف بیقرارش بر رخ قرار گیرد
&&
از رشک روی مه را در صد نگار گیرد
\\
از بس که حلقه بینی در زلف مشکبارش
&&
صد دست باید آنجا تا در شمار گیرد
\\
گر زاهدی ببیند میگونی لب او
&&
تا روز رستخیزش زان می خمار گیرد
\\
گر ماه لاله گونش تابد به نرگس و گل
&&
گلزار پای تا سر از رشک خار گیرد
\\
گر از کمان ابرو بادام نرگسینش
&&
یک تیر برگشاید صیدی هزار گیرد
\\
خورشید کو ز تنگی بر چرخ می‌کشد تیغ
&&
از بیم تیر چشمش گردون حصار گیرد
\\
او آفتاب حسن است از پرده گر بتابد
&&
دهر خرف ز رویش طبع بهار گیرد
\\
عاشق که از میانش مویی خبر ندارد
&&
در آرزوی مویش از جان کنار گیرد
\\
عطار را به وعده دل می‌دهد ولیکن
&&
اندر میان آتش دل چون قرار گیرد
\\
\end{longtable}
\end{center}
