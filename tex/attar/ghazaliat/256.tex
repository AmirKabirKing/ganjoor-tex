\begin{center}
\section*{غزل شماره ۲۵۶: هر که در بادیهٔ عشق تو سرگردان شد}
\label{sec:256}
\addcontentsline{toc}{section}{\nameref{sec:256}}
\begin{longtable}{l p{0.5cm} r}
هر که در بادیهٔ عشق تو سرگردان شد
&&
همچو من در طلبت بی سر و بی سامان شد
\\
بی سر و پای از آنم که دلم گوی صفت
&&
در خم زلف چو چوگان تو سرگردان شد
\\
هر که از ساقی عشق تو چو من باده گرفت
&&
بی‌خود و بی‌خرد و بی‌خبر و حیران شد
\\
سالک راه تو بی نام و نشان اولیتر
&&
در ره عشق تو با نام و نشان نتوان شد
\\
در منازل منشین خیز که آن کس بیند
&&
چهرهٔ مقصد و مقصود که تا پایان شد
\\
تا ابد کس ندهد نام و نشان از وی باز
&&
دل که در سایهٔ زلف تو چنین پنهان شد
\\
حسنت امروز همی بینم و صد چندان است
&&
لاجرم در دل من عشق تو صد چندان شد
\\
شادم ای دوست که در عشق تو دشواری‌ها
&&
بر من امروز به اقبال غمت آسان شد
\\
بر سر نفس نهم پای که در حالت رقص
&&
مرد راه از سر این عربده دست‌افشان شد
\\
رو که در مملکت عشق سلیمانی تو
&&
دیو نفست اگر از وسوسه در فرمان شد
\\
همچو عطار درین درد بساز ار مردی
&&
کان نبد مرد که او در طلب درمان شد
\\
\end{longtable}
\end{center}
