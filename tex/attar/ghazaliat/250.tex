\begin{center}
\section*{غزل شماره ۲۵۰: چو خورشید جمالت جلوه‌گر شد}
\label{sec:250}
\addcontentsline{toc}{section}{\nameref{sec:250}}
\begin{longtable}{l p{0.5cm} r}
چو خورشید جمالت جلوه‌گر شد
&&
چو ذره هر دو عالم مختصر شد
\\
ز هر ذره چو صد خورشید می‌تافت
&&
همه عالم به زیر سایه در شد
\\
چو خورشید از رخ تو ذره‌ای یافت
&&
بزد یک نعره وز حلقه به در شد
\\
جهان آشفته و شوریده‌دل گشت
&&
فلک سرگشته و دریوزه‌گر شد
\\
هزاران قرن پوشیده کبودی
&&
ز سر آمد به پا وز پا به سر شد
\\
ازین چندین بگردید او که ناگاه
&&
خبر یافت از تو وز خود بی خبر شد
\\
بسا رستم که اینجا زن‌صفت گشت
&&
بسا مطرب که اینجا نوحه‌گر شد
\\
قدر کاینجا رسید از خویش گم گشت
&&
قضا کانجا رسید اندک قدر شد
\\
بشست از جان و از دل دست جاوید
&&
کسی کو مرد راه این سفر شد
\\
درین ره هر که نعلینی بینداخت
&&
هزاران راهرو را تاج سر شد
\\
ولی چون سر بباخت اول درین راه
&&
ازین نعلین آخر تاجور شد
\\
درین منزل کسی کو پیشتر رفت
&&
به هر گامش تحیر بیشتر شد
\\
عجب کارا که موری می‌نداند
&&
که با عرش معظم در کمر شد
\\
شبی موجی ازین دریا برآمد
&&
از آن وقتی فلک زیر و زبر شد
\\
چو کرسی عرش حیران ماند برجای
&&
چو دنیا و آخرت یک ره گذر شد
\\
چه دریایی است این کز هیبت آن
&&
جهان هر ساعتی رنگ دگر شد
\\
ازین دریا چو عکسی سایه انداخت
&&
جدا هر ذره‌ای بحر گهر شد
\\
ازین دریا دو عالم شور بگرفت
&&
که تا ترتیب عالم معتبر شد
\\
درآمد موج دیگر آخرالامر
&&
دو عالم محو گشت و بی اثر شد
\\
ز حل و عقد شرح این مقالات
&&
دل عطار در خون جگر شد
\\
\end{longtable}
\end{center}
