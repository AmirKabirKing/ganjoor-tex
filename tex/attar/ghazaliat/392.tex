\begin{center}
\section*{غزل شماره ۳۹۲: اشک ریز آمدم چو ابر بهار}
\label{sec:392}
\addcontentsline{toc}{section}{\nameref{sec:392}}
\begin{longtable}{l p{0.5cm} r}
اشک ریز آمدم چو ابر بهار
&&
ساقیا هین بیا و باده بیار
\\
توبهٔ من درست نیست خموش
&&
وز من دلشکسته دست بدار
\\
جام درده پیاپی ای ساقی
&&
تا کنم جان خویش بر تو نثار
\\
تا که جامی تهی کنم در عشق
&&
پر برآرم ز خون دیده کنار
\\
در ره عشق چون فلک هر روز
&&
کار گیرم ز سر زهی سر و کار
\\
منم و دردیی و درد دلی
&&
دردی و درد هر دو با هم یار
\\
سر فرو برده‌ای درین گلخن
&&
فارغ از توبه و ز استغفار
\\
درس عشاق گفته در بن دیر
&&
پای منبر نهاده بر سر دار
\\
فانی و باقیم و هیچ و همه
&&
روح محضیم و صورت دیوار
\\
ساقیا گر برآرم از دل دم
&&
ز دم من برآید از تو دمار
\\
بادهٔ ما ز جام دیگر ده
&&
که نه مستیم ما و نه هشیار
\\
موضع عاشقان بی سر و بن
&&
هست بالای کعبه و خمار
\\
گر برآرند یک نفس بی دوست
&&
دلق و تسبیحشان شود زنار
\\
ما همه کشتگان این راهیم
&&
سیر گشته ز جان قلندروار
\\
مست عشقیم و روی آورده
&&
در رهی دور و عقبه‌ای دشوار
\\
زاد ما مانده مرکب افتاده
&&
وادییی تیره و رهی پر خار
\\
بی نهایت رهی که هر ساعت
&&
کشتهٔ اوست صد هزار هزار
\\
چون بدین ره بسی فرو رفتیم
&&
باز ماندیم آخر از رفتار
\\
گه به پهلوی عجز می‌گشتیم
&&
گه به سر می‌شدیم چون پرگار
\\
آخر از گوشه‌ای منادی خاست
&&
کای فروماندگان بی‌مقدار
\\
آنچه جستید در گلیم شماست
&&
لیس فی الدار غیرکم دیار
\\
این چنین وادیی به پای تو نیست
&&
سر خود گیر و رفتی ای عطار
\\
\end{longtable}
\end{center}
