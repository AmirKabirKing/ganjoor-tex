\begin{center}
\section*{غزل شماره ۸۰۷: به هر کویی مرا تا کی دوانی}
\label{sec:807}
\addcontentsline{toc}{section}{\nameref{sec:807}}
\begin{longtable}{l p{0.5cm} r}
به هر کویی مرا تا کی دوانی
&&
ز هر زهری مرا تا کی چشانی
\\
چو زهرم می‌چشاند چرخ گردون
&&
به تریاک سعادت کی رسانی
\\
گهی تابوتم اندازی به دریا
&&
گهی بر تخت فرعونم نشانی
\\
برآری برفراز طور سینا
&&
شراب الفت وصلم چشانی
\\
چو بنده مست شد دیدار خود را
&&
خطاب آید که موسی لن‌ترانی
\\
ایا موسی سخن گستاخ تا چند
&&
نه آنی که شعیبم را شبانی
\\
من آنم که شعیبت را شبانم
&&
تو آنی که شبانی را بخوانی
\\
منم موسی تویی جبار عالم
&&
گرم خوانی ورم رانی تو دانی
\\
شبانی را کجا آن قدر باشد
&&
که تو بی‌واسطه وی را بخوانی
\\
سخن گویی بدو در طور سینا
&&
درو در و گهر سازی نهانی
\\
ایا موسی تو رخت خویش بربند
&&
که تا خود را به منزلگه رسانی
\\
نه ایوبم که چندین صبر دارم
&&
نیم یوسف که در چاهم نشانی
\\
برون آمد گل زرد از گل سرخ
&&
مکن در باغ ویران باغبانی
\\
نشان وصل ما موی سفید است
&&
رسول آشکارا نه نهانی
\\
زهی عطار کز بحر معانی
&&
به الماس سخن در می‌چکانی
\\
\end{longtable}
\end{center}
