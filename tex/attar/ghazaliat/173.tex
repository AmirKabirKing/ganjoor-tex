\begin{center}
\section*{غزل شماره ۱۷۳: گر نکوییت بیشتر گردد}
\label{sec:173}
\addcontentsline{toc}{section}{\nameref{sec:173}}
\begin{longtable}{l p{0.5cm} r}
گر نکوییت بیشتر گردد
&&
آسمان در زمین به سر گردد
\\
آفتابی که هر دو عالم را
&&
کار ازو همچو آب زر گردد
\\
زآرزوی رخ تو هر روزی
&&
روی بر خاک دربدر گردد
\\
نرسد آفتاب در گردت
&&
گرچه صد قرن گرد در گردد
\\
گر بیابد کمال تو جزوی
&&
عقل کل مست و بیخبر گردد
\\
صبح از شرم سر به جیب کشد
&&
دامن آفتاب تر گردد
\\
هر که بر یاد چشمهٔ نوشت
&&
زهر قاتل خورد شکر گردد
\\
درد عشق تو را که افزون باد
&&
گر کنم چاره بیشتر گردد
\\
چون ز عشقت سخن رود جایی
&&
سخن عقل مختصر گردد
\\
چه دهی دم مرا دلم برسوز
&&
کاتش از باد تیزتر گردد
\\
بر رخم گرچه خون دل گرم است
&&
از دم سرد من جگر گردد
\\
دل عطار هر زمان بی تو
&&
در میان غمی دگر گردد
\\
\end{longtable}
\end{center}
