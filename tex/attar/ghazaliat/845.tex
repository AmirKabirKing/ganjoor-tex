\begin{center}
\section*{غزل شماره ۸۴۵: دوش از سر بیهوشی و ز غایت خودرایی}
\label{sec:845}
\addcontentsline{toc}{section}{\nameref{sec:845}}
\begin{longtable}{l p{0.5cm} r}
دوش از سر بیهوشی و ز غایت خودرایی
&&
رفتم گذری کردم بر یار ز شیدایی
\\
قلاش و قلندرسان رفتم به در جانان
&&
حلقه بزدم گفتا نه مرد در مایی
\\
گفتم که مرا بنما دیدار که تا بینم
&&
گفتا برو و بنشین ای عاشق هرجایی
\\
این چیست که می‌گویی وین چیست که می‌جویی
&&
مانا که دگر مستی یا واله و سودایی
\\
با قالب جسمانی با ما نرود کاری
&&
جسمانی و روحانی بگذار به یغمایی
\\
رو خرقهٔ جسمت را در آب فنا می‌زن
&&
تا بو که وجودت را از غیر بپالایی
\\
تا با تو تو خواهی بود بنشین چو دگر یاران
&&
از خود چو شدی بیخود برخیز چه میپایی
\\
سیلی جفا می‌خور گر طالب این راهی
&&
از نوح بلا مگریز گر عاشق دریایی
\\
ناقوس هوا بشکن گر زانکه نه گبری تو
&&
زنار ریا بگسل گر زانکه نه ترسایی
\\
دردی‌کش درد ما در راه کسی باید
&&
کو هست چو سربازان جان داده به رسوایی
\\
تو زاهد و مستوری در هستی خود مانده
&&
تا نیست نگردی تو کی محرم ما آیی
\\
خود را چو تو نشناسی حقا که چو نسناسی
&&
بیخود شو و پس خود را بنگر که چه زیبایی
\\
هم خوانچه‌کش صنعی هم مائده و خوانی
&&
هم مخزن اسراری هم مطرح یغمایی
\\
آیینهٔ دیداری جسم تو حجاب توست
&&
اندر تو پدید آید چون آینه بزدایی
\\
\end{longtable}
\end{center}
