\begin{center}
\section*{غزل شماره ۱۸۰: صبح بر شب شتاب می‌آرد}
\label{sec:180}
\addcontentsline{toc}{section}{\nameref{sec:180}}
\begin{longtable}{l p{0.5cm} r}
صبح بر شب شتاب می‌آرد
&&
شب سر اندر نقاب می‌آرد
\\
گریهٔ شمع وقت خندهٔ صبح
&&
مست را در عذاب می‌آرد
\\
ساقیا آب لعل ده که دلم
&&
ساعتی سر به آب می‌آرد
\\
خیز و خون سیاوش آر که صبح
&&
تیغ افراسیاب می‌آرد
\\
خیز ای مطرب و بخوان غزلی
&&
هین که زهره رباب می‌آرد
\\
صبحدم چون سماع گوش کنی
&&
دیده را سخت خواب می‌آرد
\\
مطرب ما رباب می‌سازد
&&
ساقی ما شراب می‌آرد
\\
همه اسباب عیش هست ولیک
&&
مرگ تیغ از قراب می‌آرد
\\
عالمی عیش با اجل هیچ است
&&
این سخن را که تاب می‌آرد
\\
ای دریغا که گر درنگ کنم
&&
عمر بر من شتاب می‌آرد
\\
در غم مرگ بی‌نمک عطار
&&
از دل خود کباب می‌آرد
\\
\end{longtable}
\end{center}
