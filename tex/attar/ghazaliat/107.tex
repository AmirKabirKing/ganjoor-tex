\begin{center}
\section*{غزل شماره ۱۰۷: هر که درین درد گرفتار نیست}
\label{sec:107}
\addcontentsline{toc}{section}{\nameref{sec:107}}
\begin{longtable}{l p{0.5cm} r}
هر که درین درد گرفتار نیست
&&
یک نفسش در دو جهان کار نیست
\\
هر که دلش دیدهٔ بینا نیافت
&&
دیدهٔ او محرم دیدار نیست
\\
هر که ازین واقعه بویی نبرد
&&
جز به صفت صورت دیوار نیست
\\
خوار شود در ره او همچو خاک
&&
هرکه در این بادیه خونخوار نیست
\\
ای دل اگر دم زنی از سر عشق
&&
جای تو جز آتش و جز دار نیست
\\
پردهٔ این راز که در قمر جان است
&&
جز قدح دردی خمار نیست
\\
آنکه سزاوار در گلخن است
&&
در حرم شاه سزاوار نیست
\\
گلخنی مفلس ناشسته روی
&&
مرد سراپردهٔ اسرار نیست
\\
کعبهٔ جانان اگرت آرزوست
&&
در گذر از خود ره بسیار نیست
\\
گرچه حجاب تو برون از حد است
&&
هیچ حجابیت چو پندار نیست
\\
پردهٔ پندار بسوز و بدانک
&&
در دو جهانت به ازین کار نیست
\\
چند کنی از سر هستی خروش
&&
نیست شو اندر طلب یار، نیست
\\
از طمع خام درین واقعه
&&
سوخته‌تر از دل عطار نیست
\\
\end{longtable}
\end{center}
