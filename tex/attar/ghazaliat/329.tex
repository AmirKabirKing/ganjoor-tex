\begin{center}
\section*{غزل شماره ۳۲۹: هر که را ذره‌ای وجود بود}
\label{sec:329}
\addcontentsline{toc}{section}{\nameref{sec:329}}
\begin{longtable}{l p{0.5cm} r}
هر که را ذره‌ای وجود بود
&&
پیش هر ذره در سجود بود
\\
نه همه بت ز سیم و زر باشد
&&
که بت رهروان وجود بود
\\
هر که یک ذره می‌کند اثبات
&&
نفس او گبر یا جهود بود
\\
در حقیقت چو جمله یک بودست
&&
پس همه بودها نبود بود
\\
نقطهٔ آتش است در باطن
&&
دود دیدن ازو چه سود بود
\\
هر که آن نقطه دید هر دو جهانش
&&
محو گشته ز چشم زود بود
\\
زانکه دو کون پیش دیدهٔ دل
&&
چون سرابی همه نمود بود
\\
هر که یک ذره غیر می‌بیند
&&
همچو کوری میان دود بود
\\
همچو عطار در فنا می‌سوز
&&
تا دمی گر زنی چو عود بود
\\
\end{longtable}
\end{center}
