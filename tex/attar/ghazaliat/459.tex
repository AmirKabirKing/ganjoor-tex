\begin{center}
\section*{غزل شماره ۴۵۹: صبح بر افراخت علم ای غلام}
\label{sec:459}
\addcontentsline{toc}{section}{\nameref{sec:459}}
\begin{longtable}{l p{0.5cm} r}
صبح بر افراخت علم ای غلام
&&
رنجه کن از لطف قدم ای غلام
\\
خیز که بشکفت گل و یاسمین
&&
تا بنشینیم به هم ای غلام
\\
باده خوریم و ز جهان بگذریم
&&
زانکه جهان شد چو ارم ای غلام
\\
بس که بریزد گل نازک ز باد
&&
ما شده در خاک دژم ای غلام
\\
زین گذران عمر چه نازیم ما
&&
زندگیی ماند و دو دم ای غلام
\\
پس چو چنین است یقین عمر خویش
&&
چند گذاریم به غم ای غلام
\\
این همه خود بگذرد و جان و دل
&&
وا رهد از جور و ستم ای غلام
\\
وقت درآمد که به پشتی تو
&&
باز بر آریم شکم ای غلام
\\
آب نجوییم ز خضر ای پسر
&&
جام نخواهیم ز جم ای غلام
\\
در نگر و خلق جهان را ببین
&&
روی نهاده به عدم ای غلام
\\
چون همه در معرض محو آمدند
&&
محو شوی زود تو هم ای غلام
\\
خود تو یقین دان که نیرزد ز مرگ
&&
جمله جهان نیم درم ای غلام
\\
عاقبت الامر چو مرگ است راه
&&
عمر تو چه بیش و چه کم ای غلام
\\
پس غم عطار درین وقت گل
&&
دفع کن از می به کرم ای غلام
\\
\end{longtable}
\end{center}
