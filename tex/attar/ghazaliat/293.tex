\begin{center}
\section*{غزل شماره ۲۹۳: روی تو کافتاب را ماند}
\label{sec:293}
\addcontentsline{toc}{section}{\nameref{sec:293}}
\begin{longtable}{l p{0.5cm} r}
روی تو کافتاب را ماند
&&
آسمان را به سر بگرداند
\\
مرکب عشق تو چو برگذرد
&&
خاک در چشم عقل افشاند
\\
هر که عکس لب تو می‌بیند
&&
دهنش پهن باز می‌ماند
\\
زلف شبرنگ و روی گلگونت
&&
می‌کند هر جفا که بتواند
\\
گاه شب‌رنگ زلفت آن تازد
&&
گاه گلگون عشقت این راند
\\
عشقت آتش فکند در جانم
&&
این چنین آتشی که بنشاند
\\
خط خونین که می‌نویسم من
&&
بر رخ چون زرم که برخواند
\\
پای تا سر چو ابر اشک شود
&&
از غمم هر که حال من داند
\\
اوفتادم ز پای دستم گیر
&&
آخر افتاده را که رنجاند
\\
دلم از زلف پیچ بر پیچت
&&
یک سر موی سر نپیچاند
\\
گر دلم بستدی و دم دادی
&&
آه من از تو داد بستاند
\\
هر که درماندهٔ تو شد نرهد
&&
همچو عطار با تو درماند
\\
\end{longtable}
\end{center}
