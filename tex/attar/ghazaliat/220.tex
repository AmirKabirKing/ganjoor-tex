\begin{center}
\section*{غزل شماره ۲۲۰: درد من هیچ دوا نپذیرد}
\label{sec:220}
\addcontentsline{toc}{section}{\nameref{sec:220}}
\begin{longtable}{l p{0.5cm} r}
درد من هیچ دوا نپذیرد
&&
زانکه حسن تو فنا نپذیرد
\\
گر من از عشق رخت توبه کنم
&&
هرگز آن توبه خدا نپذیرد
\\
از لطافت که رخت را دیدم
&&
نقش تو دیدهٔ ما نپذیرد
\\
نتوانم که تو را بینم از آنک
&&
چشم خفاش ضیا نپذیرد
\\
گرچه زلف تو دل ما می‌خواست
&&
سر گرفته است عطا نپذیرد
\\
ما بدادیم دل اما چه کنیم
&&
اگر آن زلف دوتا نپذیرد
\\
هرچه پیش تو کشم لعل لبت
&&
از من بی سر و پا نپذیرد
\\
می‌کشم پیش‌کش لعل تو جان
&&
این قدر تحفه چرا نپذیرد
\\
در ره عشق تو جان می‌بازم
&&
زانکه جان بی تو بها نپذیرد
\\
چه دغا می‌دهی آخر در جان
&&
جان عزیز است دغا نپذیرد
\\
گر بگویم که چه دیدم از تو
&&
هیچکس گفت گدا نپذیرد
\\
ور نگویم، ز غمت کشته شوم
&&
کشته دانی که دوا نپذیرد
\\
تو مرا کشتی و خلقیت گواه
&&
کس ز قول تو گوا نپذیرد
\\
خستگی دل عطار از تو
&&
مرهمی به ز وفا نپذیرد
\\
\end{longtable}
\end{center}
