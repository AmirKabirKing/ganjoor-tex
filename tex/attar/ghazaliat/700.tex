\begin{center}
\section*{غزل شماره ۷۰۰: جانا بسوخت جان من از آرزوی تو}
\label{sec:700}
\addcontentsline{toc}{section}{\nameref{sec:700}}
\begin{longtable}{l p{0.5cm} r}
جانا بسوخت جان من از آرزوی تو
&&
دردم ز حد گذشت ز سودای روی تو
\\
چندین حجاب و بنده به ره بر گرفته‌ای
&&
تا هیچ خلق پی نبرد راه کوی تو
\\
چون مشک در حجاب شدی در میان جان
&&
تا ناقصان عشق نیابند بوی تو
\\
گشتی چو گنج زیر طلسم جهان نهان
&&
تا جز تو هیچ‌کس نبرده ره به سوی تو
\\
در غایت علوی تو ارواح پست شد
&&
کو دیده‌ای که در نظر آرد علوی تو
\\
در وادی غم تو دل مستمند ما
&&
خالی نبود یک نفس از جستجوی تو
\\
بسیار جست و جوی توکردم که عاقبت
&&
عمرم رسید و می نرسد گفت و گوی تو
\\
از بس که انتظار تو کردم به روز و شب
&&
عطار را بسوخت دل از آرزوی تو
\\
\end{longtable}
\end{center}
