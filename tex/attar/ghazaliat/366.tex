\begin{center}
\section*{غزل شماره ۳۶۶: رخت را ماه نایب می‌نماید}
\label{sec:366}
\addcontentsline{toc}{section}{\nameref{sec:366}}
\begin{longtable}{l p{0.5cm} r}
رخت را ماه نایب می‌نماید
&&
خطت را مشک کاتب می‌نماید
\\
رخت سلطان حسن یک سوار است
&&
که دو ابروش حاجب می‌نماید
\\
رخت را صبح صادق کس ندیده است
&&
اگرچه صد عجایب می‌نماید
\\
چو در عشق صادق نیست یک تن
&&
همیشه صبح کاذب می‌نماید
\\
ندانم تا چو رویت آفتابی
&&
مشارق یا مغارب می‌نماید
\\
چو زلفت نیز زناری به صد سال
&&
نه رهبان و نه راهب می‌نماید
\\
چه شیوه دارد آخر غمزهٔ تو
&&
که خون‌ریزیش واجب می‌نماید
\\
ز دیوان جهان هر روز صد خونش
&&
چنین دانم که راتب می‌نماید
\\
عجب برجی است درج دلستانت
&&
که دو رسته کواکب می‌نماید
\\
ز عشقت چون کنم توبه که از عشق
&&
نخستین مست تایب می‌نماید
\\
بسی با عشق تو عقلم چخیده است
&&
ولی عشق تو غالب می‌نماید
\\
دلم بردی و گفتی دل نگه‌دار
&&
که دل در عشق راغب می‌نماید
\\
چگونه دل نگه دارم ز عشقت
&&
که گر دل هست غایب می‌نماید
\\
غم عشقت به جان بخرید عطار
&&
که چون شادی مناسب می‌نماید
\\
\end{longtable}
\end{center}
