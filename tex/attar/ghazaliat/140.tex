\begin{center}
\section*{غزل شماره ۱۴۰: ای آفتاب طفلی در سایهٔ جمالت}
\label{sec:140}
\addcontentsline{toc}{section}{\nameref{sec:140}}
\begin{longtable}{l p{0.5cm} r}
ای آفتاب طفلی در سایهٔ جمالت
&&
شیر و شکر مزیده از چشمهٔ زلالت
\\
هم هر دو کون برقی از آفتاب رویت
&&
هم نه سپهر مرغی در دام زلف و خالت
\\
بر باد داده دل را آوازهٔ فراقت
&&
در خواب کرده جان را افسانهٔ وصالت
\\
عقلی که در حقیقت بیدار مطلق آمد
&&
تا حشر مست خفته در خلوت خیالت
\\
خورشید کاسمان را سر رزمهٔ می‌گشاید
&&
یک تار می‌نسنجد در رزمه جمالت
\\
ترک فلک که هست او در هندوی تو دایم
&&
سر پا برهنه گردان در وادی کمالت
\\
سیمرغ مطلقی تو بر کوه قاف قربت
&&
پرورده هر دو گیتی در زیر پر و بالت
\\
صف قتال مردان صف‌های مژه توست
&&
صد قلب برشکسته در هر صف قتالت
\\
عطار شد چو مویی بی روی همچو روزت
&&
تا بو که راه یابد در زلف شب مثالت
\\
\end{longtable}
\end{center}
