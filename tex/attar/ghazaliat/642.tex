\begin{center}
\section*{غزل شماره ۶۴۲: کافری است از عشق دل برداشتن}
\label{sec:642}
\addcontentsline{toc}{section}{\nameref{sec:642}}
\begin{longtable}{l p{0.5cm} r}
کافری است از عشق دل برداشتن
&&
اقتدا در دین به کافر داشتن
\\
در ملا تحقیق کردن آشکار
&&
در خلا دین مزور داشتن
\\
از برون گفتن که شیطان گمره است
&&
وز درونش پیر و رهبر داشتن
\\
چون درآید تیرباران بلا
&&
در هزیمت دامن تر داشتن
\\
کار مردان چیست بیکار آمدن
&&
پس به هر دم کار دیگر داشتن
\\
خاک ره بر خود نمایان ریختن
&&
خویشتن را خاک این در داشتن
\\
غرقهٔ این بحر گشتن ناامید
&&
وانگهی امید گوهر داشتن
\\
دست بر سر پای در گل آمدن
&&
خشت بالین، خاک بستر داشتن
\\
دام تن در راه معنی سوختن
&&
مرغ جان بی‌بال و بی‌پر داشتن
\\
هر سری کان از تو سر برمی‌زند
&&
از برای تیغ و خنجر داشتن
\\
چون فلک خورشید را بر سر کشید
&&
کی تواند پای بر سر داشتن
\\
پای بر سر نه که اینجا کافری است
&&
سر برای تاج و افسر داشتن
\\
همچو عطار این سگ درنده را
&&
زهر دادن یا مسخر داشتن
\\
\end{longtable}
\end{center}
