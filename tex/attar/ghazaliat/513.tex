\begin{center}
\section*{غزل شماره ۵۱۳: ازین کاری که من دارم نه جان دارم نه تن دارم}
\label{sec:513}
\addcontentsline{toc}{section}{\nameref{sec:513}}
\begin{longtable}{l p{0.5cm} r}
ازین کاری که من دارم نه جان دارم نه تن دارم
&&
چون من من نیستم، آخر چرا گویم که من دارم
\\
تن و جان محو شد از من، ز بهر آنکه تا هستم
&&
حقیقت بهر دل دارم شریعت بهر تن دارم
\\
همه عالم پر است از من ولی من در میان پنهان
&&
مگر گنج همه عالم نهان با خویشتن دارم
\\
اگر خواهی که این گنجت شود معلوم دم درکش
&&
که سر این چنین گنجی نه بهر انجمن دارم
\\
اگر ذرات این عالم زبان من شود دایم
&&
نیارم گفت ازو یک حرف و چندانی سخن دارم
\\
مرا گویی که حرفی گوی از اسرار گنج جان
&&
چه گویم چون درین معرض نه نطق و نه دهن دارم
\\
میان خیل نا اهلان سخن چون با میان آرم
&&
که من اینجا به یک یک گام صد صد راهزن دارم
\\
چو از کونین آزادم، نگویم سر خود با کس
&&
مرا این بس که من در سینه سر سرفکن دارم
\\
اگر از سر این گنجت خبر باید به خاکم رو
&&
بپرس از من در آن ساعت که سر زیر کفن دارم
\\
از آن سلطان کونینم که دارالملک وحدت را
&&
درون گلخنی مانده نه خرقه نی وطن دارم
\\
چو زلفش را دو صد گونه شکن دیدم ز پیش و پس
&&
میان بسته به زناری سر یک یک شکن دارم
\\
نسیمی گر نمی‌یابم ز زلف یوسف قدسم
&&
ندارم هیچ نومیدی که بوی پیرهن دارم
\\
چه می‌گویم که زلف او مرا برهاند از چنبر
&&
به گرد جملهٔ عالم در آورده رسن دارم
\\
فرید از یک شکن زنار اگر بربست من با او
&&
به سوی صد شکن دیگر ز صد سو تاختن دارم
\\
\end{longtable}
\end{center}
