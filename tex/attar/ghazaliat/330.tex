\begin{center}
\section*{غزل شماره ۳۳۰: هر که را در عشق تو کاری بود}
\label{sec:330}
\addcontentsline{toc}{section}{\nameref{sec:330}}
\begin{longtable}{l p{0.5cm} r}
هر که را در عشق تو کاری بود
&&
هر سر مویی برو خاری بود
\\
یک زمان مگذار بی درد خودم
&&
تا مرا در هجر تو یاری بود
\\
مست گشتم از تو گفتی صبر کن
&&
صبر کردن کار هشیاری بود
\\
دل ز من بردی و گفتی غم مخور
&&
گر دلی نبود نه بس کاری بود
\\
گر تو را در عشق دین و دل نماند
&&
این چنین در عشق بسیاری بود
\\
دل شد از دست و ز جان ترسم ازانک
&&
طره تو چست طراری بود
\\
بی نمکدان لبت در هر دو کون
&&
می‌ندانم تا جگر خواری بود
\\
گر بخندی عاشق بیمار را
&&
وقت بیماری شکرباری بود
\\
رسته دندانت در بازار حسن
&&
تا قیامت روز بازاری بود
\\
گر بهای بوسه خواهی جز به جان
&&
می‌ندانم تا خریداری بود
\\
نافه وصلت که بویی کس نیافت
&&
کی سزای ناسزاواری بود
\\
ای عجب بی زلف عنبر بیزتو
&&
هر کسی خواهد که عطاری بود
\\
\end{longtable}
\end{center}
