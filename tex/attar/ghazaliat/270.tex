\begin{center}
\section*{غزل شماره ۲۷۰: نور روی تو را نظر نکشد}
\label{sec:270}
\addcontentsline{toc}{section}{\nameref{sec:270}}
\begin{longtable}{l p{0.5cm} r}
نور روی تو را نظر نکشد
&&
سوز عشق تو را جگر نکشد
\\
باد خاک سیاه بر سر آنک
&&
خاک کوی تو در بصر نکشد
\\
آتش عشق بیدلان تو را
&&
هفت آتش گه سقر نکشد
\\
از درازی و دوری راهت
&&
هیچ کس راه تو به سر نکشد
\\
که رهت جز به قدر و قوت ما
&&
قدر یک گام بیشتر نکشد
\\
درد هر کس به قدر طاقت اوست
&&
کانچه عیسی کشید خر نکشد
\\
کوه اندوه و بار محنت تو
&&
چون کشد دل که بحر و بر نکشد
\\
خود عجب نبود آنکه از ره عجز
&&
پشه‌ای پیل را به بر نکشد
\\
با کمان فلک به هیچ سبیل
&&
بازوی هیچ پشه در نکشد
\\
هیچکس عشق چون تو معشوقی
&&
به ترازوی عقل بر نکشد
\\
چون کشد کوه بی نهایت را
&&
آن ترازو که بیش زر نکشد
\\
وزن عشق تو عقل کی داند
&&
عشق تو عقل مختصر نکشد
\\
عشقت از دیرها نگردد باز
&&
تا که ابدال را بدر نکشد
\\
دل عطار در غم تو چنان است
&&
که غم دیگران دگر نکشد
\\
\end{longtable}
\end{center}
