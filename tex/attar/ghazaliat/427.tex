\begin{center}
\section*{غزل شماره ۴۲۷: عشق آن باشد که غایت نبودش}
\label{sec:427}
\addcontentsline{toc}{section}{\nameref{sec:427}}
\begin{longtable}{l p{0.5cm} r}
عشق آن باشد که غایت نبودش
&&
هم نهایت هم بدایت نبودش
\\
تا به کی گویم که آنجا کی رسم
&&
کی بود کی چون نهایت نبودش
\\
گر هزاران سال بر سر می‌روی
&&
همچنان می‌رو که غایت نبودش
\\
گر فرو استد کسی مرتد شود
&&
بعد از آن هرگز هدایت نبودش
\\
گر فرود آید به یک دل ذره‌ای
&&
تا به صد عالم سرایت نبودش
\\
صد هزاران خون بریزد همچو باد
&&
زانکه چون آتش حمایت نبودش
\\
نیستی خواهد که از هر نیک و بد
&&
از کسی شکر و شکایت نبودش
\\
تو مباش اصلا که اندر حق تو
&&
تا تو می‌باشی عنایت نبودش
\\
هر که بی پیری ازینجا دم زند
&&
کار بیرون از حکایت نبودش
\\
بر پی پیری برو تا پی بری
&&
کانکه تنها شد کفایت نبودش
\\
وانکه پیری می‌کشد بی دیده‌ای
&&
زین بتر هرگز جنایت نبودش
\\
چون نبیند پیر ره را گام گام
&&
کور باشد این ولایت نبودش
\\
سلطنت کی یابد ای عطار پیر
&&
تا رعیت را رعایت نبودش
\\
\end{longtable}
\end{center}
