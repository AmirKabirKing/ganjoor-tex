\begin{center}
\section*{غزل شماره ۲۰۵: ترسا بچه‌ای ناگه قصد دل و جانم کرد}
\label{sec:205}
\addcontentsline{toc}{section}{\nameref{sec:205}}
\begin{longtable}{l p{0.5cm} r}
ترسا بچه‌ای ناگه قصد دل و جانم کرد
&&
سودای سر زلفش رسوای جهانم کرد
\\
زو هر که نشان دارد دل بر سر جان دارد
&&
ترسا بچه آن دارد دیوانه از آنم کرد
\\
دوش آن بت شنگانه می‌داد به پیمانه
&&
وز کعبه به بتخانه زنجیر کشانم کرد
\\
کردم ز پریشانی در بتکده دربانی
&&
چون رفت مسلمانی بس نوحه که جانم کرد
\\
دل کفر به دین‌داری زو کرد خریداری
&&
دردا که به سر باری اسلام زیانم کرد
\\
آزاد جهان بودم بی داد و ستان بودم
&&
انگشت زنان بودم انگشت گزانم کرد
\\
دل دادم و بد کردم یک درد به صد کردم
&&
وین جرم چو خود کردم با خود چه توانم کرد
\\
دی گفت نکو خواهی توبه است تورا راهی
&&
از روی چنان ماهی من توبه ندانم کرد
\\
آخر چو فرو ماندم ترسا بچه را خواندم
&&
بسیار سخن راندم تا راه بیانم کرد
\\
بنهاد ز درویشی صد تعبیه اندیشی
&&
در پردهٔ بی خویشی از خویش نهانم کرد
\\
چون دست ز خود شستم از بند برون جستم
&&
هر چیز که می‌جستم در حال عیانم کرد
\\
من بی من و بی‌مایی افتاده بدم جایی
&&
تا در بن دریایی بی نام و نشانم کرد
\\
عطار دمی گر زد بس دست که بر سر زد
&&
هم مهر به لب بر زد هم بند زبانم کرد
\\
\end{longtable}
\end{center}
