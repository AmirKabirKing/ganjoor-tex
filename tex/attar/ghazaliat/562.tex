\begin{center}
\section*{غزل شماره ۵۶۲: ازین دریا که غرق اوست جانم}
\label{sec:562}
\addcontentsline{toc}{section}{\nameref{sec:562}}
\begin{longtable}{l p{0.5cm} r}
ازین دریا که غرق اوست جانم
&&
برون جستم ولیکن در میانم
\\
بسی رفتم درین دریا و گفتم
&&
گشاده شد به دریا دیدگانم
\\
چون نیکو باز جستم سر دریا
&&
سر مویی ز دریا می ندانم
\\
کسی کو روی این دریا بدید است
&&
دهد خوش خوش نشانی هر زمانم
\\
ولیکن آنکه در دریاست غرقه
&&
ندانم تا دهد هرگز نشانم
\\
چو چشمم نیست دریابین، چه مقصود
&&
اگر من غرق این دریا بمانم
\\
چو نابینای مادرزاد، کشتی
&&
درین دریا همه بر خشک رانم
\\
چو در دریا جنب می‌بایدم مرد
&&
چنین لب خشک و تر دامن از آنم
\\
کسی در آب حیوان تشنه میرد
&&
چه گویند آخر آن کس را من آنم
\\
دریغا کانچه می‌جستم ندیدم
&&
وزین غم پر دریغا ماند جانم
\\
ندارم یکشبه حاصل ولیکن
&&
به انواع سخن گوهر فشانم
\\
مرا از عالمی علم شکر به
&&
که باشد یک شکر اندر دهانم
\\
دلم کلی ز علم انکار بگرفت
&&
کنون من در پی کار عیانم
\\
اگر کاری عیان من نگردد
&&
چو مرداری شوم در خاکدانم
\\
اگر عطار را فانی بیابم
&&
به بحر دولتش باقی رسانم
\\
\end{longtable}
\end{center}
