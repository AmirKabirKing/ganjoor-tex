\begin{center}
\section*{غزل شماره ۸۳۳: ای لب گلگونت جام خسروی}
\label{sec:833}
\addcontentsline{toc}{section}{\nameref{sec:833}}
\begin{longtable}{l p{0.5cm} r}
ای لب گلگونت جام خسروی
&&
پیشهٔ شبرنگ زلفت شبروی
\\
پهلوی خورشید مشک‌آلود کرد
&&
خط تو یعنی که هستم پهلوی
\\
مردم چشمت بدان خردی که هست
&&
می‌ببندد دست چرخ از جادوی
\\
کی توان گفت از دهان تو سخن
&&
زانکه صورت نیست آن جز معنوی
\\
گاه همچون آفتابی از جمال
&&
گاه همچون ماه از بس نیکوی
\\
من ندانم کافتابی یا مهی
&&
کژ چه گویم راست به از هر دوی
\\
عاشقان را جامه می‌گردد قبا
&&
تو کله بنهاده کژ خوش می‌روی
\\
گفته بودی آنکه دل برد از تو کیست
&&
من ندارم زهره تا گویم توی
\\
ور بگویم من که تو بردی دلم
&&
دل به من ندهی و هرگز نشنوی
\\
دل ندارم زان ضعیفم همچو موی
&&
تو دلم ده تا شود کارم قوی
\\
من که تخم نیکوی کشتم مدام
&&
بر نخوردم از تو الا بدخوی
\\
تو که با من تخم کین کاری همه
&&
درو نبود کانچه کاری بدروی
\\
در سخن عطار اگر معجز نمود
&&
تو به اعجاز سخن می‌نگروی
\\
\end{longtable}
\end{center}
