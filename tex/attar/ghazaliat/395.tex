\begin{center}
\section*{غزل شماره ۳۹۵: ای تو را با هر دلی کاری دگر}
\label{sec:395}
\addcontentsline{toc}{section}{\nameref{sec:395}}
\begin{longtable}{l p{0.5cm} r}
ای تو را با هر دلی کاری دگر
&&
در پس هر پرده غمخواری دگر
\\
چون بسی کار است با هر کس تورا
&&
هر کسی را هست پنداری دگر
\\
لاجرم هرکس چنان داند که نیست
&&
با کست بیرون ازو کاری دگر
\\
چون جمالت صد هزاران روی داشت
&&
بود در هر ذره دیداری دگر
\\
لاجرم هر ذره را بنموده‌ای
&&
از جمال خویش رخساری دگر
\\
تا نماند هیچ ذره بی نصیب
&&
داده‌ای هر ذره را یاری دگر
\\
لاجرم دادی تو یک یک ذره را
&&
در درون پرده بازاری دگر
\\
چون یک است اصل این عدد از بهر آنست
&&
تا بود هر دم گرفتاری دگر
\\
ای دل سرگشته تا کی باشدت
&&
هر زمانی درد و تیماری دگر
\\
کی رسد از دین سر مویی به تو
&&
زیر هر موییت زناری دگر
\\
خیز و ایمان آر و زنارت ببر
&&
توبه کن مردانه یکباری دگر
\\
دل منه بر هیچ چون عطار هیچ
&&
تا کیت هر لحظه دلداری دگر
\\
\end{longtable}
\end{center}
