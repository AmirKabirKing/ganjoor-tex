\begin{center}
\section*{غزل شماره ۳۳۷: چه سازی سرای و چه گویی سرود}
\label{sec:337}
\addcontentsline{toc}{section}{\nameref{sec:337}}
\begin{longtable}{l p{0.5cm} r}
چه سازی سرای و چه گویی سرود
&&
فروشو بدین خاک تیره فرود
\\
یقین‌دان که همچون تو بسیار کس
&&
فکندست در چرخ چرخ کبود
\\
چه برخیزد از خود و آهن تو را
&&
چو سر آهنین نیست در زیر خود
\\
اگر جامهٔ عمر تو زآهن است
&&
اجل بگسلد از همش تار و پود
\\
اگر سر کشی زین پل هفت طاق
&&
سر و سنگ ماننده آب رود
\\
ز سرگشتگی زیر چوگان چرخ
&&
چو گویی ندانی فراز از فرود
\\
چو دور سپهرت نخواهد گذاشت
&&
ز دور سپهری چه نالی چو رود
\\
رفیقان هم‌راز را کن وداع
&&
عزیزان همدرد را کن درود
\\
درخت بتر بودن از بن بکن
&&
ز شاخ بهی کن کلوخ آمرود
\\
مکن همچو عطار عمر عزیز
&&
همه ضایع اندر سرای و سرود
\\
\end{longtable}
\end{center}
