\begin{center}
\section*{غزل شماره ۱۴۱: ای بی نشان محض نشان از که جویمت}
\label{sec:141}
\addcontentsline{toc}{section}{\nameref{sec:141}}
\begin{longtable}{l p{0.5cm} r}
ای بی نشان محض نشان از که جویمت
&&
گم گشت در تو هر دو جهان از که جویمت
\\
تو گم نه‌ای و گمشدهٔ تو منم ولیک
&&
تا یافت یافت می‌نتوان از که جویمت
\\
دل در فنای وحدت و جان در بقای صرف
&&
من گمشده درین دو میان از که جویمت
\\
پیدا بسی بجستمت اما نیافتم
&&
اکنون مرا بگو که نهان از که جویمت
\\
چون در رهت یقین و گمانی همی رود
&&
ای برتر از یقین و گمان از که جویمت
\\
در بحر بی نهایت عشقت چو قطره‌ای
&&
گم شد نشان مه به نشان از که جویمت
\\
تا بود که بویی از تو بیابد دلم چو جان
&&
بیرون شد از زمان و مکان از که جویمت
\\
در جست و جوی تو دلم از پرده اوفتاد
&&
ای در درون پردهٔ جان از که جویمت
\\
عطار اگرچه یافت به عین یقین تورا
&&
ای بس عیان به عین عیان از که جویمت
\\
\end{longtable}
\end{center}
