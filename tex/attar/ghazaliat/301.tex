\begin{center}
\section*{غزل شماره ۳۰۱: از می عشق نیستی هر که خروش می‌زند}
\label{sec:301}
\addcontentsline{toc}{section}{\nameref{sec:301}}
\begin{longtable}{l p{0.5cm} r}
از می عشق نیستی هر که خروش می‌زند
&&
عشق تو عقل و جانش را خانه فروش می‌زند
\\
عاشق عشق تو شدم از دل و جان که عشق تو
&&
پرده نهفته می‌درد زخم خموش می‌زند
\\
دل چو ز درد درد تو مست خراب می‌شود
&&
عمر وداع می‌کند عقل خروش می‌زند
\\
گرچه دل خراب من از می عشق مست شد
&&
لیک صبوح وصل را نعره به هوش می‌زند
\\
دل چو حریف درد شد ساقی اوست جان ما
&&
دل می عشق می‌خورد جام دم نوش می‌زند
\\
تا دل من به مفلسی از همه کون درگذشت
&&
از همه کینه می‌کشد بر همه دوش می‌زند
\\
تا ز شراب شوق تو دل بچشید جرعه‌ای
&&
جملهٔ پند زاهدان از پس گوش می‌زند
\\
ای دل خسته نیستی مرد مقام عاشقی
&&
سیر شدی ز خود مگر خون تو جوش می‌زند
\\
جان فرید از بلی مست می الست شد
&&
شاید اگر به بوی او لاف سروش می‌زند
\\
\end{longtable}
\end{center}
