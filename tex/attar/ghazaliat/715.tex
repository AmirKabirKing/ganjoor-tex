\begin{center}
\section*{غزل شماره ۷۱۵: سر پا برهنگانیم اندر جهان فتاده}
\label{sec:715}
\addcontentsline{toc}{section}{\nameref{sec:715}}
\begin{longtable}{l p{0.5cm} r}
سر پا برهنگانیم اندر جهان فتاده
&&
جان را طلاق گفته دل را به باد داده
\\
مردان راه‌بین را در گبرکی کشیده
&&
رندان ره‌نشین را میخانه در گشاده
\\
با گوشه‌ای نشسته دست از جهان بشسته
&&
در پیش دردنوشان بر پای ایستاده
\\
اندر میان مستان چندان گناه کرده
&&
کز چشم خلق عالم یکبارگی فتاده
\\
هرجا که مفلسان را جمعیتی است روزی
&&
ماییم جان و دل را اندر میان نهاده
\\
ما خود که‌ایم ما را خون ریختن حلال است
&&
رهزن شدند ما را مشتی حرام‌زاده
\\
زنهار الله الله تا کی ز کفر و ایمان
&&
گه روی سوی قبله گه دست سوی باده
\\
نه مؤمنم نه کافر گه اینم و گه آنم
&&
رفتم به خاک تاریک از هر دو خر پیاده
\\
عطار اگر دگر ره در راه دین درآیی
&&
دل بایدت که گردد از هرچه هست ساده
\\
\end{longtable}
\end{center}
