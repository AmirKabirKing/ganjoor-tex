\begin{center}
\section*{غزل شماره ۴۰۳: چو پیشهٔ تو شیوه و ناز است چه تدبیر}
\label{sec:403}
\addcontentsline{toc}{section}{\nameref{sec:403}}
\begin{longtable}{l p{0.5cm} r}
چو پیشهٔ تو شیوه و ناز است چه تدبیر
&&
چون مایهٔ من درد و نیاز است چه تدبیر
\\
آن در که به روی همه باز است نگارا
&&
چون بر من بیچار فراز است چه تدبیر
\\
گفتی که اگر راست روی راه بدانی
&&
این راه چو پر شیب و فراز است چه تدبیر
\\
گفتی که اگر صبر کنی کام بیابی
&&
لعاب فلک شعبده‌باز است چه تدبیر
\\
گویی نه درست است نماز از سر غفلت
&&
چون عشق توام پیش‌نماز است چه تدبیر
\\
گفتم که کنم قصهٔ سودای تو کوتاه
&&
چون قصهٔ عشق تو دراز است چه تدبیر
\\
گفتم که کنم توبه ز عشق تو ولیکن
&&
عشق تو حقیقت نه مجاز است چه تدبیر
\\
گفتم ندهم دل به تو چون روی تو بینم
&&
چون غمزهٔ تو عربده‌ساز است چه تدبیر
\\
بیچار دلم صعوهٔ خرد است چه چاره
&&
در صید دلم عشق تو باز است چه تدبیر
\\
بر مجمر سودای تو همچون شکر و عود
&&
عطار چو در سوز و گذار است چه تدبیر
\\
\end{longtable}
\end{center}
