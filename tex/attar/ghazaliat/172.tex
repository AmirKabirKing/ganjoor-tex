\begin{center}
\section*{غزل شماره ۱۷۲: بودی که ز خود نبود گردد}
\label{sec:172}
\addcontentsline{toc}{section}{\nameref{sec:172}}
\begin{longtable}{l p{0.5cm} r}
بودی که ز خود نبود گردد
&&
شایستهٔ وصل زود گردد
\\
چوبی که فنا نگردد از خود
&&
ممکن نبود که عود گردد
\\
این کار شگرف در طریقت
&&
بر بود تو و نبود گردد
\\
هرگه که وجود تو عدم گشت
&&
حالی عدمت وجود گردد
\\
ای عاشق خویش وقت نامد
&&
کابلیس تو در سجود گردد
\\
دل در ره نفس باختی پاک
&&
تا نفس تو جفت سود گردد
\\
دل نفس شد و شگفتت آید
&&
گر یک علوی جهود گردد
\\
هر دم که به نفس می برآری
&&
در دیدهٔ دل چو دود گردد
\\
بی شک دل تو از آن چنان دود
&&
کوری شود و کبود گردد
\\
عطار بگفت آنچه دانست
&&
باقی همه بر شنود گردد
\\
\end{longtable}
\end{center}
