\begin{center}
\section*{غزل شماره ۲۷۵: ره عشاق بی ما و من آمد}
\label{sec:275}
\addcontentsline{toc}{section}{\nameref{sec:275}}
\begin{longtable}{l p{0.5cm} r}
ره عشاق بی ما و من آمد
&&
ورای عالم جان و تن آمد
\\
درین ره چون روی کژ چون روی راست
&&
که اینجا غیر ره بین رهزن آمد
\\
رهی پیش من آمد بی نهایت
&&
که بیش از وسع هر مرد و زن آمد
\\
هزارن قرن گامی می‌توان رفت
&&
چه راه راست این که در پیش من آمد
\\
شود اینجا کم از طفل دو روزه
&&
اگر صد رستم در جوشن آمد
\\
درین ره عرش هر روزی به صد بار
&&
ز هیبت با سر یک سوزن آمد
\\
درین ره هست مرغان کاسمانشان
&&
درون حوصله یک ارزن آمد
\\
رهی است آیینه وارآن کس که در رفت
&&
هم او در دیدهٔ خود روشن آمد
\\
کسی کو اندرین ره دانه‌ای یافت
&&
سپهری خوشه‌چین خرمن آمد
\\
نهان باید که داری سر این راه
&&
که خصمت با تو در پیراهن آمد
\\
کسی را گر شود گویی بیانش
&&
ازین سر باخبر تر دامن آمد
\\
کسی مرد است کین سر چون بدانست
&&
نه مستی کرد ونه آبستن آمد
\\
علاج تو درین ره تا تویی تو
&&
چو شمعت سوختن یا مردن آمد
\\
بمیر از خویش تا زنده بمانی
&&
که بی شک گرد ران با گردن آمد
\\
دل عطار سر دوستی یافت
&&
ولی وقتی که خود را دشمن آمد
\\
\end{longtable}
\end{center}
