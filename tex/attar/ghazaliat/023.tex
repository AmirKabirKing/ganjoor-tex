\begin{center}
\section*{غزل شماره ۲۳: آه‌های آتشینم پرده‌های شب بسوخت}
\label{sec:023}
\addcontentsline{toc}{section}{\nameref{sec:023}}
\begin{longtable}{l p{0.5cm} r}
آه‌های آتشینم پرده‌های شب بسوخت
&&
بر دل آمد وز تف دل هم زبان هم لب بسوخت
\\
دوش در وقت سحر آهی برآوردم ز دل
&&
در زمین آتش فتاد و بر فلک کوکب بسوخت
\\
جان پر خونم که مشتی خاک دامن گیر اوست
&&
گاه اندر تاب ماند و گاه اندر تب بسوخت
\\
پردهٔ پندار کان چون سد اسکندر قوی است
&&
آه خون آلود من هر شب به یک یارب بسوخت
\\
روز دیگر پردهٔ دیگر برون آمد ز غیب
&&
پردهٔ دیگر به یارب‌های دیگرشب بسوخت
\\
هر که او خام است گو در مذهب ما نه قدم
&&
زانکه دعوی خام شد هر کو درین مذهب بسوخت
\\
باز عشقش چون دل عطار در مخلب گرفت
&&
از دل گرمش عجب نبود اگر مخلب بسوخت
\\
\end{longtable}
\end{center}
