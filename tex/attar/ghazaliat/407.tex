\begin{center}
\section*{غزل شماره ۴۰۷: هر که زو داد یک نشانی باز}
\label{sec:407}
\addcontentsline{toc}{section}{\nameref{sec:407}}
\begin{longtable}{l p{0.5cm} r}
هر که زو داد یک نشانی باز
&&
ماند محجوب جاودانی باز
\\
چون کس از بی نشان نشان دهدت
&&
یا تو هم چون دهی نشانی باز
\\
مرده دل گر ازو نشان طلبد
&&
گو ز سر گیر زندگانی باز
\\
چون جمالی است بی نشان جاوید
&&
نتوان یافت جز نهانی باز
\\
ارنی گر بسی خطاب کنی
&&
بانگ آید به لن‌ترانی باز
\\
من گرفتم که این همه پرده
&&
شود از مرکز معانی باز
\\
چون تو بیگانه وار زیسته‌ای
&&
چون ببینی کجاش دانی باز
\\
پس رونده که کرد دعوی آنک
&&
رسته‌ام از جهان فانی باز
\\
خود چو در ره فتوح دید بسی
&&
ماند از اندک از معانی باز
\\
گرچه کردند از یقین دعوی
&&
همه گشتند بر گمانی باز
\\
هر که را این جهان ز راه ببرد
&&
نبود راه آن جهانی باز
\\
تو اگر عاشقی به هر دو جهان
&&
ننگری جز به سرگرانی باز
\\
جان مده در طریق عشق چنان
&&
که ستانی اگر توانی باز
\\
خود ز جان دوستی تو هرگز جان
&&
ندهی ور دهی ستانی باز
\\
گر چو پروانه عاشقی که به صدق
&&
پیش آید به جان فشانی باز
\\
چه بود ای دل فرو رفته
&&
خبری گر به من رسانی باز
\\
تا کجایی چه می‌کنی چونی
&&
این گره کن به مهربانی باز
\\
گر ز عطار بشنوی تو سخن
&&
راه یابد به خوش بیانی باز
\\
\end{longtable}
\end{center}
