\begin{center}
\section*{غزل شماره ۷۸۵: هر دمم مست به بازار کشی}
\label{sec:785}
\addcontentsline{toc}{section}{\nameref{sec:785}}
\begin{longtable}{l p{0.5cm} r}
هر دمم مست به بازار کشی
&&
راستی چست و به هنجار کشی
\\
می عشقم بچشانی و مرا
&&
مست گردانی و در کار کشی
\\
گاهم از کفر به دین باز آری
&&
گاهم از کعبه به خمار کشی
\\
گاهم از راه یقین دور کنی
&&
گاهم اندر ره اسرار کشی
\\
گه ز مسجد به خرابات بری
&&
گاهم از میکده در غار کشی
\\
چون ز اسلام منت ننگ آید
&&
از مصلام به زنار کشی
\\
چون مرا ننگ ره دین بینی
&&
هر دمم در ره کفار کشی
\\
بس که پیران حقیقت‌بین را
&&
اندرین واقعه بر دار کشی
\\
ای دل سوخته گر مرد رهی
&&
خون خوری تن زنی و بار کشی
\\
بر امید گل وصلش شب و روز
&&
همچو گلبن ستم خار کشی
\\
آتش اندر دل ایام زنی
&&
خاک در دیدهٔ اغیار کشی
\\
بویی از مجمرهٔ عشق بری
&&
باده بر چهرهٔ دلدار کشی
\\
غم معشوق که شادی دل است
&&
در ره عشق چو عطار کشی
\\
\end{longtable}
\end{center}
