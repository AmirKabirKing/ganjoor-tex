\begin{center}
\section*{غزل شماره ۴۰۹: ای روی تو شمع پردهٔ راز}
\label{sec:409}
\addcontentsline{toc}{section}{\nameref{sec:409}}
\begin{longtable}{l p{0.5cm} r}
ای روی تو شمع پردهٔ راز
&&
در پردهٔ دل غم تو دمساز
\\
بی مهر رخت برون نیاید
&&
از باطن هیچ پرده آواز
\\
از شوق تو می‌کند همه روز
&&
خورشید درون پرده پرواز
\\
هر جا که شگرف پرده بازی است
&&
در پردهٔ زلف توست جان‌باز
\\
در مجمع سرکشان عالم
&&
چون زلف تو نیست یک سرافراز
\\
خون دل من بریخت چشمت
&&
پس گفت نهفته دار این راز
\\
چون خونی بود غمزهٔ تو
&&
شد سرخی غمزهٔ تو غماز
\\
گفتی که چو زر عزیز مایی
&&
زان همچو زرت نهیم در گاز
\\
هرچه از تو رسد به جان پذیرم
&&
این واسطه از میان بینداز
\\
ما را به جنایتی که ما راست
&&
خود زن به زنندگان مده باز
\\
یک لحظه تو غمگسار ما باش
&&
تا نوحهٔ تو کنیم آغاز
\\
تا کی باشم من شکسته
&&
در بادیهٔ تو در تک و تاز
\\
گر وقت آمد به یک عنایت
&&
این خانهٔ من ز شک بپرداز
\\
بیش است به تو نیازمندیم
&&
چندان که تو بیش می‌کنی ناز
\\
عطار ز دیرگاه بی تو
&&
بیچارهٔ توست، چاره‌ای ساز
\\
\end{longtable}
\end{center}
