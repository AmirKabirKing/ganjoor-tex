\begin{center}
\section*{غزل شماره ۶۹۵: دلا چون کس نخواهد ماند دایم هم نمانی تو}
\label{sec:695}
\addcontentsline{toc}{section}{\nameref{sec:695}}
\begin{longtable}{l p{0.5cm} r}
دلا چون کس نخواهد ماند دایم هم نمانی تو
&&
قدم در نه اگر هستی طلب‌کار معانی تو
\\
گرفتم صد هزاران علم در مویی بدانستی
&&
چو مرگت سایه اندازد سر مویی چه دانی تو
\\
چو کامت بر نمی‌آید به ناکامی فرو ده تن
&&
که در زندان ناکامی نیابی کامرانی تو
\\
به چیزی زندگی باید که نبود زین جهان لابد
&&
که تا چون زین جهان رفتی بدان زنده بمانی تو
\\
وگر زنده به دنیا باشی ای غافل در آن عالم
&&
بمانی مرده و هرگز نیابی زندگانی تو
\\
اگر تو پر و بال دنیی و عقبی بیندازی
&&
خطابت آید از پیشان که هرچ آن جستی آنی تو
\\
بلی هر دم پیامت آید از حضرت که ای محرم
&&
چو حی‌لایموتی تو چرا بر خود نخوانی تو
\\
چو گشتی زین خطاب آگاه جانت را یقین گردد
&&
که سلطان جهان‌افروز دارالملک جانی تو
\\
زهی عطار کز بحر معانی چون مدد داری
&&
توانی کرد هر ساعت بسی گوهرفشانی تو
\\
\end{longtable}
\end{center}
