\begin{center}
\section*{غزل شماره ۴۴۹: عقل کجا پی برد شیوهٔ سودای عشق}
\label{sec:449}
\addcontentsline{toc}{section}{\nameref{sec:449}}
\begin{longtable}{l p{0.5cm} r}
عقل کجا پی برد شیوهٔ سودای عشق
&&
باز نیابی به عقل سر معمای عشق
\\
عقل تو چون قطره‌ای است مانده ز دریا جدا
&&
چند کند قطره‌ای فهم ز دریای عشق
\\
خاطر خیاط عقل گرچه بسی بخیه زد
&&
هیچ قبایی ندوخت لایق بالای عشق
\\
گر ز خود و هر دو کون پاک تبرا کنی
&&
راست بود آن زمان از تو تولای عشق
\\
ور سر مویی ز تو با تو بماند به هم
&&
خام بود از تو خام پختن سودای عشق
\\
عشق چو کار دل است دیدهٔ دل باز کن
&&
جان عزیزان نگر مست تماشای عشق
\\
دوش درآمد به جان دمدمهٔ عشق او
&&
گفت اگر فانیی هست تو را جای عشق
\\
جان چو قدم در نهاد تا که همی چشم زد
&&
از بن و بیخش بکند قوت و غوغای عشق
\\
چون اثر او نماند محو شد اجزای او
&&
جای دل و جان گرفت جملهٔ اجزای عشق
\\
هست درین بادیه جملهٔ جانها چو ابر
&&
قطرهٔ باران او درد و دریغای عشق
\\
تا دل عطار یافت پرتو این آفتاب
&&
گشت ز عطار سیر، رفت به صحرای عشق
\\
\end{longtable}
\end{center}
