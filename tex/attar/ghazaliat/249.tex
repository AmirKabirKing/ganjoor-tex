\begin{center}
\section*{غزل شماره ۲۴۹: در راه تو هر که راهبر شد}
\label{sec:249}
\addcontentsline{toc}{section}{\nameref{sec:249}}
\begin{longtable}{l p{0.5cm} r}
در راه تو هر که راهبر شد
&&
هر لحظه به طبع خاک تر شد
\\
هر خاک که ذرهٔ قدم گشت
&&
در عالم عشق تاج سر شد
\\
تا تو نشوی چو ذره ناچیز
&&
نتوانی ازین قفس به در شد
\\
هر کو به وجود ذره آمد
&&
فارغ ز وجود خیر و شر شد
\\
در هستی خود چو ذره گم گشت
&&
ذاتی که ز عشق معتبر شد
\\
ذره ز که پرسد و چه پرسد
&&
زیرا که ز خویش بی‌خبر شد
\\
خورشید ز خویش ذره‌ای دید
&&
وآنگه به دهان شیر در شد
\\
گر ذرهٔ راه نیست خورشید
&&
پیوسته چرا چنین به سر شد
\\
چون ذره کسی که پیشتر رفت
&&
سرگشتهٔ راه بیشتر شد
\\
در عشق چو ذره شو که عشقش
&&
بر آهن و سنگ کارگر شد
\\
بنمود نخست پردهٔ زلف
&&
در پرده نشست و پرده در شد
\\
درداد ندا که همچو ذره
&&
فانی صفتی که در سفر شد
\\
موی سر زلف ماش جاوید
&&
همراهی کرد و راهبر شد
\\
عطار چو ذره تا فنا گشت
&&
در دیدهٔ خویش مختصر شد
\\
\end{longtable}
\end{center}
