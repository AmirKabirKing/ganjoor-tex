\begin{center}
\section*{غزل شماره ۶۷۴: ای صبا گر بگذری بر زلف مشک افشان او}
\label{sec:674}
\addcontentsline{toc}{section}{\nameref{sec:674}}
\begin{longtable}{l p{0.5cm} r}
ای صبا گر بگذری بر زلف مشک افشان او
&&
همچو من شو گرد یک یک حلقه سرگردان او
\\
منت صد جان بیار و بر سر ما نه به حکم
&&
وز سر زلفش نشانی آر ما را زان او
\\
گاه از چوگان زلفش حلقهٔ مشکین ربای
&&
گاه خود را گوی گردان در خم چوگان او
\\
خوش خوش اندر پیچ زلفش پیچ تا مشکین کنی
&&
شرق تا غرب جهان از زلف مشک افشان او
\\
نی خطا گفتم ادب نیست آنچه گفتم جهد کن
&&
تا پریشانی نیاید زلف عنبرسان او
\\
گر مرا دل زنده خواهی کرد جامی جانفزای
&&
نوش کن بر یاد من از چشمهٔ حیوان او
\\
گر تو جان داری چه کن بر کن به دندان پشت دست
&&
چون ببینی جانفزایی لب و دندان او
\\
گو فلانی از میان جانت می‌گوید سلام
&&
گو به جان تو فرو شد روز اول جان او
\\
جان او در جان تو گم گشت و دل از دست رفت
&&
درد او از حد بشد گر می‌کنی درمان او
\\
چون رسی آنجا اجازت خواه اول بعد از آن
&&
عرضه کن این قصهٔ پر درد در دیوان او
\\
چشم آنجا بر مگیر از پشت پای و گوش‌دار
&&
ورنه حالی بر زمین دوزد تو را مژگان او
\\
هرچه گوید یادگیر و یک به یک بر دل نویس
&&
تا چنان کو گفت برسانی به من فرمان او
\\
چند گریی ای فرید از عشق رویش همچو شمع
&&
صبح را مژده رسان از پستهٔ خندان او
\\
\end{longtable}
\end{center}
