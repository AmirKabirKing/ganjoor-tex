\begin{center}
\section*{غزل شماره ۷۶۹: گر از همه عاشقان وفا دیدی}
\label{sec:769}
\addcontentsline{toc}{section}{\nameref{sec:769}}
\begin{longtable}{l p{0.5cm} r}
گر از همه عاشقان وفا دیدی
&&
چون من به وفای خود که را دیدی
\\
دانی تو که جز وفا ندیدی خود
&&
در جملهٔ عمر تا مرا دیدی
\\
من از تو به جان خود جفا دیدم
&&
تو از من خسته دل وفا دیدی
\\
این است جفا که زود بگذشتی
&&
از بی رویی چو روی ما دیدی
\\
برگشتی تو ز بی دلی هر دم
&&
این مصلحت آخر از کجا دیدی
\\
می‌بگذری و روی تو از پیشم
&&
ما را تو به راه آسیا دیدی
\\
بیگانه مباش چون دو چشمم را
&&
از خون جگر در آشنا دیدی
\\
تا روی چو آفتاب بنمودی
&&
بس دل که چو ذره در هوا دیدی
\\
عطار ز دست رفت و تو با او
&&
دیدی که چه کردی و چها دیدی
\\
\end{longtable}
\end{center}
