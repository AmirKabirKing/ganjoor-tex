\begin{center}
\section*{غزل شماره ۱۲۷: ای دلم مست چشمهٔ نوشت}
\label{sec:127}
\addcontentsline{toc}{section}{\nameref{sec:127}}
\begin{longtable}{l p{0.5cm} r}
ای دلم مست چشمهٔ نوشت
&&
در خطم از خط سیه پوشت
\\
باد سرسبزی خطت که به لطف
&&
سر برون زد ز چشمهٔ نوشت
\\
حلقه در گوش کرد خلق را
&&
حلقهٔ زلف بر بناگوشت
\\
همچو من صد هزار سرگشته
&&
حلقه در گوش حلقهٔ گوشت
\\
گشت معلوم من که جان نبرد
&&
دلم از طرهٔ سیه پوشت
\\
تو به جان و دلی جفا کوشم
&&
من به جان و دلم وفا کوشت
\\
عشوه مفروش زانکه من پس ازین
&&
نخرم نیز خواب خرگوشت
\\
یاد کن از کسی که در همه عمر
&&
نکند لحظه‌ای فراموشت
\\
مست از آنم چنین که در بر خویش
&&
مست در خواب دیده‌ام دوشت
\\
بو که تعبیر خوابم آن باشد
&&
که شوم امشبی هم آغوشت
\\
دل عطار باده ناخورده
&&
تا قیامت بمانده مدهوشت
\\
\end{longtable}
\end{center}
