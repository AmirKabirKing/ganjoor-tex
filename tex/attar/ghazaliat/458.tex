\begin{center}
\section*{غزل شماره ۴۵۸: خورد بر شب صبحدم شام ای غلام}
\label{sec:458}
\addcontentsline{toc}{section}{\nameref{sec:458}}
\begin{longtable}{l p{0.5cm} r}
خورد بر شب صبحدم شام ای غلام
&&
زنده گردان جانم از جام ای غلام
\\
جام در ده و این دل پر درد را
&&
وارهان از ننگ و از نام ای غلام
\\
جملهٔ شب همچو شمعی سوختم
&&
صبح دم زد ما چنین خام ای غلام
\\
دست ایامم به روی اندر فکند
&&
هین که رفت از دست ایام ای غلام
\\
گام بیرون نه که دست روزگار
&&
ندهدت پیشی به یک گام ای غلام
\\
چند باشی بر امید دانه‌ای
&&
همچو مرغی مانده در دام ای غلام
\\
چند باشی در میان خرقه گیر
&&
تازه گردان زود اسلام ای غلام
\\
گر همی خواهی که از خود وارهی
&&
با قلندر دردی آشام ای غلام
\\
عاشق ره شو که کار مرد عشق
&&
برتر است از مدح و دشنام ای غلام
\\
بی سر و بن شو چو گویی زانکه عشق
&&
هست بی آغاز و انجام ای غلام
\\
هر که او در عشق بی‌آرام نیست
&&
کی تواند یافت آرام ای غلام
\\
گاه مرد مسجدی گه رند دیر
&&
هر دو نبود کام و ناکام ای غلام
\\
یا مرو در مسجد و زنار بند
&&
یا مده در دیر ابرام ای غلام
\\
چون تو اندر راه باشی ناتمام
&&
کی رسد کارت به اتمام ای غلام
\\
رو تو خاص خاص شو یا عام عام
&&
تا به کی نه خاص و نه عام ای غلام
\\
گفت عطار آنچه می‌دانست باز
&&
یادت آید این به هنگام ای غلام
\\
\end{longtable}
\end{center}
