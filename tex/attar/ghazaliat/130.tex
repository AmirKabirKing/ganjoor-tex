\begin{center}
\section*{غزل شماره ۱۳۰: تا دل ز کمال تو نشان یافت}
\label{sec:130}
\addcontentsline{toc}{section}{\nameref{sec:130}}
\begin{longtable}{l p{0.5cm} r}
تا دل ز کمال تو نشان یافت
&&
جان عشق تو در میان جان یافت
\\
پروانهٔ شمع عشق شد جان
&&
چون سوخته شد ز تو نشان یافت
\\
جان بود نگین عشق و مهرت
&&
چون نقش نگین در آن میان یافت
\\
جان بارگه تورا طلب کرد
&&
در مغز جهان لامکان یافت
\\
جان را به درت نگاهی افتاد
&&
صد حلقه برو چو آسمان یافت
\\
هر جان که به کوی تو فرو شد
&&
از بوی تو جان جاودان یافت
\\
فریاد و خروش عاشقانت
&&
در کون و مکان نمی‌توان یافت
\\
از درد تو جان ما بنالید
&&
درمان تو درد بی‌کران یافت
\\
چون درد تو یافت زیر هر درد
&&
درمان همه جهان نهان یافت
\\
هرچیز که جان ما همی جست
&&
چون در تو نگاه کرد آن یافت
\\
هر مقصودی که عقل را بود
&&
در شعلهٔ روی تو عیان یافت
\\
عطار چو این سخن بیان کرد
&&
بیرون ز جهان بسی جهان یافت
\\
\end{longtable}
\end{center}
