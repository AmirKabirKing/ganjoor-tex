\begin{center}
\section*{غزل شماره ۱۹۸: هر دل که وصال تو طلب کرد}
\label{sec:198}
\addcontentsline{toc}{section}{\nameref{sec:198}}
\begin{longtable}{l p{0.5cm} r}
هر دل که وصال تو طلب کرد
&&
شب خوش بادش که روز شب کرد
\\
در تاریکی میان خون مرد
&&
هر که آب حیات تو طلب کرد
\\
وآنکس که بنا در این گهر یافت
&&
بی خود شد و مدتی طرب کرد
\\
آن چیز که یافت بس عجب یافت
&&
وآن حال که کرد بس عجب کرد
\\
چون حوصله پر برآمد او را
&&
بانگی نه به وقت ازین سبب کرد
\\
عشق تو میان خون و آتش
&&
بردار کشیدش و ادب کرد
\\
عشق تو هزار طیلسان را
&&
در گردن عاشقان کنب کرد
\\
بس مرد شگرف را که این بحر
&&
لب برهم دوخت و خشک لب کرد
\\
بس جان عظیم را که این درد
&&
گه تاب بسوخت گاه تب کرد
\\
چون خار رطب بد و رطب خار
&&
عقل از چه عزیمت رطب کرد
\\
صد حقه و مهره هست و هیچ است
&&
این کار کدام بلعجب کرد
\\
چون نتوانی محمدی یافت
&&
باری مکن آنچه بولهب کرد
\\
عطار سزد که پشت گرم است
&&
چون روی به قبلهٔ عرب کرد
\\
\end{longtable}
\end{center}
