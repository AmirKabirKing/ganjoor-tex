\begin{center}
\section*{غزل شماره ۳۶۰: صبح از پرده به در می‌آید}
\label{sec:360}
\addcontentsline{toc}{section}{\nameref{sec:360}}
\begin{longtable}{l p{0.5cm} r}
صبح از پرده به در می‌آید
&&
اثر آه سحر می‌آید
\\
یا کسی مشک ختن می‌بیزد
&&
یا نسیم گل تر می‌آید
\\
خیز ای ساقی و می‌ده به صبوح
&&
که حریف چو شکر می‌آید
\\
پسری کز خط سبزش چو قلم
&&
دل عشاق به سر می‌آید
\\
ای پسر می ده و می نوش که عمر
&&
به سر تو که به سر می‌آید
\\
عمرت این یکدم حالی است تو را
&&
کیست ضامن که دگر می‌آید
\\
تویی و یکدم و آگاه نه‌ای
&&
کز دگر دم چه خبر می‌آید
\\
لیک دانی تو که بی صد غم نیست
&&
هر دمی کان ز تو بر می‌آید
\\
سنگ بر بام فلک زن به صبوح
&&
که فلک بر تو به در می‌آید
\\
داد بستان ز جهانی که درو
&&
بهتر خلق بتر می‌آید
\\
در جهانی که همه بی‌نمکی است
&&
قسم عطار جگر می‌آید
\\
\end{longtable}
\end{center}
