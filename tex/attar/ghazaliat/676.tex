\begin{center}
\section*{غزل شماره ۶۷۶: ای سراسیمه مه از رخسار تو}
\label{sec:676}
\addcontentsline{toc}{section}{\nameref{sec:676}}
\begin{longtable}{l p{0.5cm} r}
ای سراسیمه مه از رخسار تو
&&
سرو سر در پیش از رفتار تو
\\
ذره‌ای است انجم زخورشید رخت
&&
نقطه‌ای است افلاک از پرگار تو
\\
گل که باشد پیش رخسارت از آنک
&&
عقل کل جزوی است از رخسار تو
\\
پر شکر شد شرق تا غرب جهان
&&
از شکرریز شکر گفتار تو
\\
چشم گردد ذره ذره در دو کون
&&
بر امید ذره‌ای دیدار تو
\\
گنج پنهانی تو ای جان و جهان
&&
جان شعاع تو جهان آثار تو
\\
چون تو هستی هر زمان در خورد تو
&&
پس که خواهد بود جز تو یار تو
\\
چون کسی را نیست یارا در دو کون
&&
هست هر دم تیزتر بازار تو
\\
صد هزاران جان فروشد هر نفس
&&
کس نیامد واقف اسرار تو
\\
بیش می‌دانم هزار و صد هزار
&&
از فلک سرگشته‌تر در کار تو
\\
دم به دم می‌آفریند آنچه هست
&&
و آفریدن نیست جز اظهار تو
\\
خود نمی‌استد دمی یک ذره چیز
&&
تا نثار تو شود ایثار تو
\\
هر زمانی صد هزاران عالم است
&&
کان نثار توست انمودار تو
\\
تا ابد هرگز نبیند ذره‌ای
&&
خواری و غم هر که شد غمخوار تو
\\
زان حسین از دار تو منصور شد
&&
کز هزاران تخت بهتر دار تو
\\
گر همه آفاق عالم پر گل است
&&
زان همه گل خوشترم یک خار تو
\\
صد سپه هرلحظه گر ظاهر شود
&&
برهم اندازم به استظهار تو
\\
می‌بچربد بر جهانی خوش دلی
&&
در دل من ذره‌ای تیمار تو
\\
روی گردانید عطار از دو کون
&&
در لحد آورد و در دیوار تو
\\
عالمی در هستی خود مانده‌اند
&&
زین جهت شد نیست خود عطار تو
\\
\end{longtable}
\end{center}
