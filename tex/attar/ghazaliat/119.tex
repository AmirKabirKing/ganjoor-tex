\begin{center}
\section*{غزل شماره ۱۱۹: در ده خبر است این که ز مه ده خبری نیست}
\label{sec:119}
\addcontentsline{toc}{section}{\nameref{sec:119}}
\begin{longtable}{l p{0.5cm} r}
در ده خبر است این که ز مه ده خبری نیست
&&
وین واقعه را همچو فلک پای و سری نیست
\\
عقلم که جهان زیر و زبر کرد به فکرت
&&
بی خویش از آن شد که ز خویشش خبری نیست
\\
جان سوخته زان شد که از آنها که برفتند
&&
بسیار اثر جست و ز یک تن اثری نیست
\\
دل بر سر ره ماند که می‌دید که هستش
&&
مشکل سفری پیش که چون هر سفری نیست
\\
این کار برون نیست ز دو نوع به تحقیق
&&
یا هیچ نیم یا که به جز من دگری نیست
\\
در ماتم این درد که دورند از آن خلق
&&
آشفته و سرگشته چو من نوحه‌گری نیست
\\
زان مغز شود خشک و ترم هر شب و هر روز
&&
کز چرخ مرا جز لب و رخ خشک و تری نیست
\\
جانم که ز بستان فلک نیشکری خواست
&&
گفتا نه‌ای واقف که مرا نیشکری نیست
\\
از خوان فلک دل مطلب گر جگرت خورد
&&
زیرا که اگر دل دهدت بی جگری نیست
\\
عطار چو کس را خطری نیست درین راه
&&
تو نیز فرو شو که تورا هم خطری نیست
\\
\end{longtable}
\end{center}
