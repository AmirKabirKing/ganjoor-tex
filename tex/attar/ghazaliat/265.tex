\begin{center}
\section*{غزل شماره ۲۶۵: حدیث فقر را محرم نباشد}
\label{sec:265}
\addcontentsline{toc}{section}{\nameref{sec:265}}
\begin{longtable}{l p{0.5cm} r}
حدیث فقر را محرم نباشد
&&
وگر باشد مگر زآدم نباشد
\\
طبایع را نباشد آنچنان خوی
&&
که هرگز رخش چون رستم نباشد
\\
سخن می‌رفت دوش از لوح محفوظ
&&
نگه کردم چو جام جم نباشد
\\
هرآنکس کو ازین یک جرعه نوشید
&&
مر او را کعبه و زمزم نباشد
\\
سلیمان‌وار می‌شو منطق‌الطیر
&&
روا گر تخت ور خاتم نباشد
\\
پس اکنون کیست محرم در ره فقر
&&
دلی کو را نشاط و غم نباشد
\\
مجرد باش دایم چونکه عطار
&&
سوار فقر را پرچم نباشد
\\
\end{longtable}
\end{center}
