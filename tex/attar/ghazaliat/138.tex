\begin{center}
\section*{غزل شماره ۱۳۸: گر نبودی در جهان امکان گفت}
\label{sec:138}
\addcontentsline{toc}{section}{\nameref{sec:138}}
\begin{longtable}{l p{0.5cm} r}
گر نبودی در جهان امکان گفت
&&
کی توانستی گل معنی شکفت
\\
جان ما را تا به حق شد چشم باز
&&
بس که گفت و بس گل معنی که رفت
\\
بی قراری پیشه کرد و روز و شب
&&
یک نفس ننشست و یک ساعت نخفت
\\
بس گهر کز قعر دریای ضمیر
&&
بر سر آورد و به خون دل بسفت
\\
پاک‌رو داند که در اسرار عشق
&&
بهتر از ما راهبر نتوان گرفت
\\
آنچه ما دیدیم در عالم که دید
&&
وآنچه ما گفتیم در عالم که گفت
\\
آنچه بعد از ما بگویند آن ماست
&&
زانکه راز گفت نیست از ما نهفت
\\
تربیت ما را ز جان مصطفاست
&&
لاجرم خود را نمی‌یابیم جفت
\\
تا تویی عطار زیر بار عشق
&&
گردنان را زیر بار توست سفت
\\
صورت جان است شعرت لاجرم
&&
عقل را نظم تو می‌آید شگفت
\\
\end{longtable}
\end{center}
