\begin{center}
\section*{غزل شماره ۱۶۰: هر آن دردی که دلدارم فرستد}
\label{sec:160}
\addcontentsline{toc}{section}{\nameref{sec:160}}
\begin{longtable}{l p{0.5cm} r}
هر آن دردی که دلدارم فرستد
&&
شفای جان بیمارم فرستد
\\
چو درمان است درد او دلم را
&&
سزد گر درد بسیارم فرستد
\\
اگر بی او دمی از دل برآرم
&&
که داند تا چه تیمارم فرستد
\\
وگر در عشق او از جان برآیم
&&
هزاران جان به ایثارم فرستد
\\
وگر در جویم از دریای وصلش
&&
به دریا در نگونسارم فرستد
\\
وگر از راز او رمزی بگویم
&&
ز غیرت بر سر دارم فرستد
\\
چو در دیرم دمی حاضر نبیند
&&
ز مسجد سوی خمارم فرستد
\\
چو دام زرق بیند در برم دلق
&&
بسوزد دلق و زنارم فرستد
\\
چو گبر نفس بیند در نهادم
&&
به آتشگاه کفارم فرستد
\\
به دیرم درکشد تا مست گردم
&&
به صد عبرت به بازارم فرستد
\\
چو بی کارم کند از کار عالم
&&
پس آنگه از پی کارم فرستد
\\
چو در خدمت چنان گردم که باید
&&
به خلوت پیش عطارم فرستد
\\
\end{longtable}
\end{center}
