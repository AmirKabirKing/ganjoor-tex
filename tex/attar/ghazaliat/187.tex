\begin{center}
\section*{غزل شماره ۱۸۷: زین درد کسی خبر ندارد}
\label{sec:187}
\addcontentsline{toc}{section}{\nameref{sec:187}}
\begin{longtable}{l p{0.5cm} r}
زین درد کسی خبر ندارد
&&
کین درد کسی دگر ندارد
\\
تا در سفر اوفکند دردم
&&
می‌سوزم و کس خبر ندارد
\\
کور است کسی که ذره‌ای را
&&
بیند که هزار در ندارد
\\
چه جای هزار و صد هزار است
&&
یک ذره چو پا و سر ندارد
\\
چندان که شوی به ذره‌ای در
&&
مندیش که ره دگر ندارد
\\
چون نامتناهی است ذره
&&
خواجه سر این سفر ندارد
\\
آن کس گوید که ذره‌خرد است
&&
کو دیدهٔ دیده‌ور ندارد
\\
چون دیده پدید گشت خورشید
&&
از ذره بزرگتر ندارد
\\
از یک اصل است جمله پیدا
&&
اما دل تو نظر ندارد
\\
در ذره تو اصل بین که ذره
&&
از ذره شدن خبر ندارد
\\
اصل است که فرع می‌نماید
&&
زان اصل کسی گذر ندارد
\\
عطار اگر زبون فرغ است
&&
جان چشم زاصل بر ندارد
\\
\end{longtable}
\end{center}
