\begin{center}
\section*{غزل شماره ۱۹۲: عشق تو به سینه تاختن برد}
\label{sec:192}
\addcontentsline{toc}{section}{\nameref{sec:192}}
\begin{longtable}{l p{0.5cm} r}
عشق تو به سینه تاختن برد
&&
وآرام و قرار من ز من برد
\\
تن چند زنم که چشم مستت
&&
جانی که نداشتم ز تن برد
\\
صد گونه قرار از دل من
&&
زلفت به طلسم پرشکن برد
\\
عشق تو نمود دستبردی
&&
مردی و زنی ز مرد و زن برد
\\
با چشم تو عقل خویشتن را
&&
بی خویشتنی ز خویشتن برد
\\
عیسی لب روح‌بخش تو دید
&&
در حال خرش شد و رسن برد
\\
خضر آب حیات کی توانست
&&
بی‌یاد لب تو در دهن برد
\\
جمشید کجا جهان‌نمایی
&&
بی عکس رخت به جام ظن برد
\\
سیمرغ ز بیم دام زلفت
&&
بگریخت و به قاف تاختن برد
\\
گفتند بتان که چهرهٔ ما
&&
قدر گل و رونق سمن برد
\\
درتافت ستارهٔ رخ تو
&&
وآب همه از چه ذقن برد
\\
عطار چو شرح آن ذقن داد
&&
گوی از همه کس بدین سخن برد
\\
\end{longtable}
\end{center}
