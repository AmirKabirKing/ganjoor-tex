\begin{center}
\section*{غزل شماره ۵۶۶: زهره ندارم که سلامت کنم}
\label{sec:566}
\addcontentsline{toc}{section}{\nameref{sec:566}}
\begin{longtable}{l p{0.5cm} r}
زهره ندارم که سلامت کنم
&&
چون طمع وصل مدامت کنم
\\
گرچه جوابم ندهی این بسم
&&
چون شنوی تو که سلامت کنم
\\
چون نتوانم که به گردت رسم
&&
گرد به گرد در و بامت کنم
\\
مرغ تو حلاج سزد من کیم
&&
تا هوس حلقهٔ دامت کنم
\\
خاک شدم تا نفس خویش را
&&
هم نفس جرعهٔ جامت کنم
\\
گر به حسامم بکشی نقد جان
&&
پیشکش زخم حسامت کنم
\\
نیست مرا دل وگرم صد بود
&&
سوختهٔ وعدهٔ خامت کنم
\\
یک شکرت خواسته‌ام گفته‌ای
&&
می‌طلبم باز که وامت کنم
\\
گر چه حلال است تو را خون من
&&
گر ندهی بوسه حرامت کنم
\\
چون همه خوبی جهان وقف توست
&&
گنگ شدم وصف کدامت کنم
\\
خطبهٔ جانم چو به نام تو رفت
&&
سکهٔ تن نیز به نامت کنم
\\
نی که تنی نیست دو من استخوانست
&&
پیش سگ کوی غلامت کنم
\\
مشک جهان گر همه عطار داشت
&&
وقف خط غالیه فامت کنم
\\
\end{longtable}
\end{center}
