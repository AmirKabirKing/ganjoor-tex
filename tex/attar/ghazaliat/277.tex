\begin{center}
\section*{غزل شماره ۲۷۷: دی پیر من از کوی خرابات برآمد}
\label{sec:277}
\addcontentsline{toc}{section}{\nameref{sec:277}}
\begin{longtable}{l p{0.5cm} r}
دی پیر من از کوی خرابات برآمد
&&
وز دلشدگان نعرهٔ هیهات برآمد
\\
شوریده به محراب فنا سر به برافکند
&&
سرمست به معراج مناجات برآمد
\\
چون دردی جانان به ره سینه فرو ریخت
&&
از مشرق جان صبح تحیات برآمد
\\
چون دوست نقاب از رخ پر نور برانداخت
&&
با دوست فرو شد به مقامات برآمد
\\
آن دیده کزان دیده توان دید جمالش
&&
آن دیده پدید آمد و حاجات برآمد
\\
مقصود به حاصل شد و مطلوب به تعین
&&
محبوب قرین گشت و مهمات برآمد
\\
بد باز جهان بود بدان کوی فروشد
&&
واقبال بدان بود که شهمات برآمد
\\
دین داشت و کرامات و به یک جرعه می عشق
&&
بیخود شد و از دین و کرامات برآمد
\\
عطار بدین کوی سراسیمه همی گشت
&&
تا نفی شد و از ره اثبات برآمد
\\
\end{longtable}
\end{center}
