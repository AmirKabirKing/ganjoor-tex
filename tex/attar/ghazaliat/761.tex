\begin{center}
\section*{غزل شماره ۷۶۱: تا تو ز هستی خود زیر و زبر نگردی}
\label{sec:761}
\addcontentsline{toc}{section}{\nameref{sec:761}}
\begin{longtable}{l p{0.5cm} r}
تا تو ز هستی خود زیر و زبر نگردی
&&
در نیستی مطلق مرغی بپر نگردی
\\
زین ابر تر چو باران بیرون شو و سفر کن
&&
زیرا که بی سفر تو هرگز گهر نگردی
\\
این پردهٔ نهادت بر در ز هم که هرگز
&&
در پرده ره نیابی تا پرده‌در نگردی
\\
گر با تو خلق عالم آید برون به خصمی
&&
گر مرد این حدیثی زنهار برنگردی
\\
ور بر تو نیز بارد ذرات هر دو عالم
&&
هان تا به دفع کردن گرد سپر نگردی
\\
گرچه میان دریا جاوید غرقه گشتی
&&
هش دار تا ز دریا یک موی تر نگردی
\\
گر عاقل جهانی کس عاقلت نخواند
&&
تا تو ز عشق هر دم دیوانه‌تر نگردی
\\
گر تو کبود پوشی همچون فلک درین راه
&&
همچون فلک چرا تو دایم به سر نگردی
\\
عطار خاک ره شو زیرا که اندرین راه
&&
بادت به دست ماند خاک ره ار نگردی
\\
\end{longtable}
\end{center}
