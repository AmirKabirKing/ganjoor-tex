\begin{center}
\section*{غزل شماره ۳۲۷: هر که را اندیشهٔ درمان بود}
\label{sec:327}
\addcontentsline{toc}{section}{\nameref{sec:327}}
\begin{longtable}{l p{0.5cm} r}
هر که را اندیشهٔ درمان بود
&&
درد عشق تو برو تاوان بود
\\
بر کسی درد تو گردد آشکار
&&
کو ز چشم خویشتن پنهان بود
\\
گرچه دارد آفتابی در درون
&&
لیک همچون ذره سرگردان بود
\\
ای دل محجوب بگذر از حجاب
&&
زانکه محجوبی عذاب جان بود
\\
گر هزاران سال باشی در عذاب
&&
می‌توان گفتن که بس آسان بود
\\
لیک گر افتد حجابی در رهت
&&
این عذاب سخت صد چندان بود
\\
چند اندیشی بمیر از خویش پاک
&&
تا نمیری کی تو را درمان بود
\\
چون بمیرد شمع برهد از بلا
&&
نه دگر سوزنده نه گریان بود
\\
هر دم از سر گیر چو شمع و بسوز
&&
زانکه سوز شمع تا پایان بود
\\
چون بسوزی پاک پیش چشم تو
&&
هر دو کون و ذره‌ای یکسان بود
\\
عرش را گر چشم جان آید پدید
&&
تا ابد در خردلی حیران بود
\\
عرش و خردل و آنچه در هر دو جهان است
&&
ذره ذره جامهٔ جانان بود
\\
تو درون جامهٔ جانان مدام
&&
تا ایازت دایما سلطان بود
\\
صد هزاران چیز داند شد به طبع
&&
آن عصا کان لایق ثعبان بود
\\
آن عصا کان سحرهٔ فرعون خورد
&&
نی عصای موسی عمران بود
\\
وان نفس کان مردگان را زنده کرد
&&
نی دم عیسی حکمت‌دان بود
\\
آن عصا آنجا یدالله بود و بس
&&
وان نفس بی شک دم رحمان بود
\\
وان هزاران خلق کز داود مرد
&&
آن نه زین الحان که زان الحان بود
\\
در بر مردی که این سر پی برد
&&
مردی رستم همه دستان بود
\\
گر ندانستی تو این سر تن بزن
&&
تا در آن ساعت که وقت آن بود
\\
تن زن ای عطار و تن زن دم مزن
&&
زانکه اینجا دم زدن نقصان بود
\\
\end{longtable}
\end{center}
