\begin{center}
\section*{غزل شماره ۱۱۶: سرو چون قد خرامان تو نیست}
\label{sec:116}
\addcontentsline{toc}{section}{\nameref{sec:116}}
\begin{longtable}{l p{0.5cm} r}
سرو چون قد خرامان تو نیست
&&
لعل چون پستهٔ خندان تو نیست
\\
نیست یک کس که به لب آمده جان
&&
زآرزوی لب و دندان تو نیست
\\
هیچ جمعیت اگر یافت کسی
&&
از جز آن زلف پریشان تو نیست
\\
مرده آن دل که به صد جان نه به یک
&&
زندهٔ چشمهٔ حیوان تو نیست
\\
غرقه باد آنکه به صد سوختگی
&&
تشنهٔ چاه زنخدان تو نیست
\\
به ز جان عاشق دیدار تو را
&&
سپر ناوک مژگان تو نیست
\\
چشم یک عاقل و هشیار ندید
&&
که چو من واله و حیران تو نیست
\\
می وصلم ده آخر که مرا
&&
بیش ازین طاقت هجران تو نیست
\\
ای دل سوخته در درد بسوز
&&
زانکه جز درد تو درمان تو نیست
\\
چند باشی تو از آن خود از آنک
&&
تا تو آن خودی او آن تو نیست
\\
گر بدو نیست رهت جان درباز
&&
زحمت جان تو جز جان تو نیست
\\
که کشد درد دلت ای عطار
&&
شرح آن لایق دیوان تو نیست
\\
\end{longtable}
\end{center}
