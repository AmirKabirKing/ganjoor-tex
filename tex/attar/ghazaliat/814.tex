\begin{center}
\section*{غزل شماره ۸۱۴: ترسا بچه‌ای به دلستانی}
\label{sec:814}
\addcontentsline{toc}{section}{\nameref{sec:814}}
\begin{longtable}{l p{0.5cm} r}
ترسا بچه‌ای به دلستانی
&&
در دست شراب ارغوانی
\\
دوش آمد و تیز و تازه بنشست
&&
چون آتش و آب زندگانی
\\
دانی که خوشی او چه سان بود
&&
چون عشق به موسم جوانی
\\
در بسته میان خود به زنار
&&
بگشاده دهن به دلستانی
\\
در هر خم زلف دلفریبش
&&
صد عالم کافری نهانی
\\
آمد بنشست و پیر ما را
&&
بنهاد محک به امتحانی
\\
القصه چو پیر روی او دید
&&
از دست بشد ز ناتوانی
\\
دردی ستد و درود دین کرد
&&
یارب ز بلای ناگهانی
\\
دردا که چنان بزرگواری
&&
برخاست ز راه خرده دانی
\\
ترسا بچه را به پیش خود خواند
&&
پس گفت نشان ره چه دانی
\\
گفتا که نشان راه جایی است
&&
کانجا نه تویی و نه نشانی
\\
چون پیر سخن شنید جان داد
&&
عطار سخن بگو که جانی
\\
\end{longtable}
\end{center}
