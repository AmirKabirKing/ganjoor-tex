\begin{center}
\section*{غزل شماره ۳۸۹: بردار صراحیی ز خمار}
\label{sec:389}
\addcontentsline{toc}{section}{\nameref{sec:389}}
\begin{longtable}{l p{0.5cm} r}
بردار صراحیی ز خمار
&&
بربند به روی خرقه زنار
\\
با دردکشان دردپیشه
&&
بنشین و دمی مباش هشیار
\\
یا پیش هوا به سجده درشو
&&
یا بند هوا ز پای بردار
\\
تا چند نهان کنی به تلبیس
&&
این دین مزورت ز اغیار
\\
تا کی ز مذبذبین بوی تو
&&
یک لحظه نخفته و نه بیدار
\\
گر زن صفتی به کوی سر نه
&&
ور مرد رهی درآی در کار
\\
سر در نه و هرچه بایدت کن
&&
گه کعبه مجوی و گاه خمار
\\
چون سیر شدی ز هرزه کاری
&&
آنگاه به دین درآی یکبار
\\
گه آیی و گاه بازگردی
&&
این نیست نشان مرد دین‌دار
\\
چیزی که صلاح تو در آن است
&&
بنیوش که با تو گفت عطار
\\
\end{longtable}
\end{center}
