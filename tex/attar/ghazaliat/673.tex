\begin{center}
\section*{غزل شماره ۶۷۳: ای چو گویی گشته در میدان او}
\label{sec:673}
\addcontentsline{toc}{section}{\nameref{sec:673}}
\begin{longtable}{l p{0.5cm} r}
ای چو گویی گشته در میدان او
&&
تا ابد چون گوی سرگردان او
\\
همچو گویی خویشتن تسلیم کن
&&
پس به سر می‌گرد در میدان او
\\
جان اگر زو داری و جانانت اوست
&&
تن فرو ده درخم چوگان او
\\
سوز عشقش بس بود در جان تو را
&&
دل منه بر وصل و بر هجران او
\\
با وصال و هجر او کاریت نیست
&&
اینت بس یعنی که عشقت زان او
\\
این کمالت بس که در وادی عشق
&&
خویش را بینی همی حیران او
\\
تو که‌ای در راه عشقش قطره‌ای
&&
غرقه در دریای بی پایان او
\\
وانگه از هر سوی می‌پرسی خبر
&&
تا کجا دارد کسی دیوان او
\\
تن زن ای عطار و جان پروانه وار
&&
برفشان چون در رسد فرمان او
\\
\end{longtable}
\end{center}
