\begin{center}
\section*{غزل شماره ۲۱: تا به عمدا ز رخ نقاب انداخت}
\label{sec:021}
\addcontentsline{toc}{section}{\nameref{sec:021}}
\begin{longtable}{l p{0.5cm} r}
تا به عمدا ز رخ نقاب انداخت
&&
خاک در چشم آفتاب انداخت
\\
سر زلفش چو شیر پنجه گشاد
&&
آهوان را به مشک ناب انداخت
\\
تیر چشمش که عالمی خون داشت
&&
اشتری را به یک کباب انداخت
\\
لب شیرینش چون تبسم کرد
&&
شور در لؤلؤ خوشاب انداخت
\\
تاب در زلف داد و هر مویش
&&
در دلم صد هزار تاب انداخت
\\
خیمهٔ عنبرینت ای مهوش
&&
در همه حلقها طناب انداخت
\\
شوق روی چو آفتاب تو بود
&&
کاسمان را در انقلاب انداخت
\\
شکری از لبت به سرکه رسید
&&
سرکه را باز در شراب انداخت
\\
عرقی کرد عارض چو گلت
&&
نظرم بر گل و گلاب انداخت
\\
روی ناشسته خوشتری بنشین
&&
کاتشی روی تو در آب انداخت
\\
از لب تو فرید آبی خواست
&&
در دلش آتش عذاب انداخت
\\
\end{longtable}
\end{center}
