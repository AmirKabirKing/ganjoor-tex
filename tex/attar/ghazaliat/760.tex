\begin{center}
\section*{غزل شماره ۷۶۰: درج یاقوت درفشان کردی}
\label{sec:760}
\addcontentsline{toc}{section}{\nameref{sec:760}}
\begin{longtable}{l p{0.5cm} r}
درج یاقوت درفشان کردی
&&
دیو بودی و قصد جان کردی
\\
شکری خواستم از لعل لبت
&&
هر دو لب را شکرستان کردی
\\
گفتم این لحظه یافتم شکری
&&
روی از آستین نهان کردی
\\
وا گرفتی ز بیدلی شکری
&&
با چنین لب چرا چنان کردی
\\
از سبک روحی تو این نسزد
&&
گر تو بر خشم سر گران کردی
\\
عشوه دادی مرا در اول کار
&&
دلم از وصل شادمان کردی
\\
آخر کار چون ز دست شدم
&&
چشمم از هجر خون‌فشان کردی
\\
ریختی تیر غمزه بر رویم
&&
تا مرا پشت چون کمان کردی
\\
چون دلم پیش خود هدف دیدی
&&
دل من بد بتر از آن کردی
\\
آن چه کردی ز جور با عطار
&&
شیوهٔ دور آسمان کردی
\\
\end{longtable}
\end{center}
