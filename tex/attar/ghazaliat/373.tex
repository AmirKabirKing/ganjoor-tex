\begin{center}
\section*{غزل شماره ۳۷۳: تا که گشت این خیال‌خانه پدید}
\label{sec:373}
\addcontentsline{toc}{section}{\nameref{sec:373}}
\begin{longtable}{l p{0.5cm} r}
تا که گشت این خیال‌خانه پدید
&&
هر زمان گشت صد بهانه پدید
\\
ناپدید است عیسی مریم
&&
قصهٔ سوزن است و شانه پدید
\\
صد جهان ناپدید شد که نشد
&&
ذره‌ای کس درین دهانه پدید
\\
گرچه تو صد هزار می‌بینی
&&
هیچکس نیست در میانه پدید
\\
چون دو گیتی به جز خیالی نیست
&&
کیست غمگین و شادمانه پدید
\\
زین همه نقش‌های گوناگون
&&
نیست جز نقش یک یگانه پدید
\\
روشنی از یک آفتاب بود
&&
گر شود در هزار خانه پدید
\\
مرغ در دام اوفتاده بسی است
&&
وی عجب نیست مرغ و دانه پدید
\\
می‌نماید بسی خیال ولیک
&&
نه زمان است و نه زمانه پدید
\\
زین همه کار و بار و گفت و شنود
&&
اثری نیست جاودانه پدید
\\
صد جهان خلق همچو تیر برفت
&&
نه نشان است و نه نشانه پدید
\\
قطره بس ناپدید بینم از آنک
&&
هست دریای بی کرانه پدید
\\
نه که خود قطره کی خبر دارد
&&
که پدید است بحر یا نه پدید
\\
دو جهان پر و بال سیمرغ است
&&
نیست سیمرغ و آشیانه پدید
\\
ره به سیمرغ چون توان بردن
&&
بیش هر گام صد ستانه پدید
\\
قدر خلعت کنون بدانستم
&&
که بشد خازن و خزانه پدید
\\
گر درین شرح شد زبان از کار
&&
از دل آمد بسی زبانه پدید
\\
سر فروپوش چند گویی از آنک
&&
نیست پایان این فسانه پدید
\\
گر شود گوش ذره‌های دو کون
&&
نشود سر این ترانه پدید
\\
شیرمردان مرد را اینجا
&&
عالمی عذر شد زنانه پدید
\\
ندهد شرح این کسی چو فرید
&&
کاسمان هست از آسمانه پدید
\\
\end{longtable}
\end{center}
