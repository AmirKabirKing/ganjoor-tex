\begin{center}
\section*{غزل شماره ۲۲۴: چو قفل لعل بر درج گهر زد}
\label{sec:224}
\addcontentsline{toc}{section}{\nameref{sec:224}}
\begin{longtable}{l p{0.5cm} r}
چو قفل لعل بر درج گهر زد
&&
جهانی خلق را بر یکدگر زد
\\
لب لعلش جهان را برهم انداخت
&&
خط سبزش قضا را بر قدر زد
\\
نبات خط او چون از شکر رست
&&
ز خجلت چون عسل حل شد طبر زد
\\
به رخش حسن چون بر عاشقان تاخت
&&
نیندیشید و لاف لاتذر زد
\\
رخ او تاب در خورشید و مه داد
&&
لب او بانگ بر تنگ شکر زد
\\
چو نقاش ازل از بهر خطش
&&
به سیمین لوح او بیرنگ برزد
\\
چو خط بنوشت گویی نقطهٔ لعل
&&
درونش سی ستاره بر قمر زد
\\
بسی می‌زد به مژگان بر دلم تیر
&&
بدو گفتم که کم زن بیشتر زد
\\
دلم از طره چون زیر و زبر کرد
&&
گره بر طرهٔ زیر و زبر زد
\\
دلم خون کرد تا از پاش بفکند
&&
عقیقی گشت آنگه بر کمر زد
\\
دلم با او چو دستی در کمر کرد
&&
کمربند فلک را دست در زد
\\
فرید او را گزید از هر دو عالم
&&
به یک‌دم آتشی در خشک و تر زد
\\
\end{longtable}
\end{center}
