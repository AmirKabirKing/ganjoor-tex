\begin{center}
\section*{غزل شماره ۱۷۱: هر دل که ز خویشتن فنا گردد}
\label{sec:171}
\addcontentsline{toc}{section}{\nameref{sec:171}}
\begin{longtable}{l p{0.5cm} r}
هر دل که ز خویشتن فنا گردد
&&
شایستهٔ قرب پادشا گردد
\\
هر گل که به رنگ دل نشد اینجا
&&
اندر گل خویش مبتلا گردد
\\
امروز چو دل نشد جدا از گل
&&
فردا نه ز یکدگر جدا گردد
\\
خاک تن تو شود همه ذره
&&
هر ذره کبوتر هوا گردد
\\
ور در گل خویشتن بماند دل
&&
از تنگی گور کی رها گردد
\\
دل آینه‌ای است پشت او تیره
&&
گر بزدایی بروی وا گردد
\\
گل دل گردد چو پشت گردد رو
&&
ظلمت چو رود همه ضیا گردد
\\
هرگاه که پشت و روی یکسان شد
&&
آن آینه غرق کبریا گردد
\\
ممکن نبود که هیچ مخلوقی
&&
گردید خدای یا خدا گردد
\\
اما سخن درست آن باشد
&&
کز ذات و صفات خود فنا گردد
\\
هرگه که فنا شود ازین هر دو
&&
در عین یگانگی بقا گردد
\\
حضرت به زبان حال می‌گوید
&&
کس ما نشود ولی ز ما گردد
\\
چیزی که شود چو بود کی باشد
&&
کی نادایم چو دایما گردد
\\
گر می‌خواهی که جان بیگانه
&&
با این همه کار آشنا گردد
\\
در سایهٔ پیر شو که نابینا
&&
آن اولیتر که با عصا گردد
\\
کاهی شو و کوه عجب بر هم زن
&&
تا پیر تو را چو کهربا گردد
\\
ور این نکنی که گفت عطارت
&&
هر رنج که می‌بری هبا گردد
\\
\end{longtable}
\end{center}
