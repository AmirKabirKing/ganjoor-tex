\begin{center}
\section*{غزل شماره ۱۴۹: عکس روی تو بر نگین افتاد}
\label{sec:149}
\addcontentsline{toc}{section}{\nameref{sec:149}}
\begin{longtable}{l p{0.5cm} r}
عکس روی تو بر نگین افتاد
&&
حلقه بشکست و بر زمین افتاد
\\
شد جهان همچو حلقه‌ای بر من
&&
تا که چشمم بر آن نگین افتاد
\\
دور از رویت آتشم در دل
&&
زان لب همچو انگبین افتاد
\\
آب رویم مبر که بی رویت
&&
قسم من آه آتشین افتاد
\\
تا که خورشید چهرهٔ تو بتافت
&&
شور در چرخ چارمین افتاد
\\
خوشهٔ عنبرین زلفت تورا
&&
ماه و خورشید خوشه چین افتاد
\\
زلف مگشای و کفر برمفشان
&&
که خروشی در اهل دین افتاد
\\
مشک از چین طلب که نیم شبی
&&
چینی از زلف تو به چین افتاد
\\
در ز چشمم طلب که هر اشکی
&&
به حقیقت دری ثمین افتاد
\\
دست شست از وجود هر که دمی
&&
در غم چون تو نازنین افتاد
\\
دل ندارم ملامتم چه کنی
&&
بی دل افتاده‌ام چنین افتاد
\\
می ندانم تو را بدین سختی
&&
با من مهربان چه کین افتاد
\\
دل عطار چون نه مرغ تو بود
&&
ضعف در مخلبش ازین افتاد
\\
\end{longtable}
\end{center}
