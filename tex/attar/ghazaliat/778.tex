\begin{center}
\section*{غزل شماره ۷۷۸: درآمد دوش دلدارم به یاری}
\label{sec:778}
\addcontentsline{toc}{section}{\nameref{sec:778}}
\begin{longtable}{l p{0.5cm} r}
درآمد دوش دلدارم به یاری
&&
مرا گفتا بگو تا در چه کاری
\\
حرامت باد اگر بی ما زمانی
&&
برآوردی دمی یا می برآری
\\
چو با ما می‌توانی بود هر شب
&&
روا نبود که بی ما شب گذاری
\\
چو با ما غمگساری می‌توان کرد
&&
چرا با دیگری غم می گساری
\\
خوشی با دشمن ما در نشستی
&&
نباشد این دلیل دوستداری
\\
بدان می‌داریم کز عزت خویش
&&
تو را در خاک اندازم به خواری
\\
به تنهاییت بگذارم که تا تو
&&
بمانی تا ابد در بیقراری
\\
چو بشنیدم ز جانان این سخن‌ها
&&
بدو گفتم که دست از جمله داری
\\
ولیکن چون تو یار غمگنانی
&&
مرا از ننگ من برهان به یاری
\\
که گر عطار در هستی بماند
&&
برو گریند عالمیان به زاری
\\
\end{longtable}
\end{center}
