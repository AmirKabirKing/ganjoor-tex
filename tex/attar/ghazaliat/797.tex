\begin{center}
\section*{غزل شماره ۷۹۷: هزاران جان سزد در هر زمانی}
\label{sec:797}
\addcontentsline{toc}{section}{\nameref{sec:797}}
\begin{longtable}{l p{0.5cm} r}
هزاران جان سزد در هر زمانی
&&
نثار روی چون تو دلستانی
\\
توان کردن هزاران جان به یک دم
&&
فدای روی تو چه جای جانی
\\
نثار تو کنم منت پذیرم
&&
اگر جانم بود هر دم جهانی
\\
بجز عشقت ندارم کیش و دینی
&&
بجز کویت ندارم خان و مانی
\\
نیارم داد شرح ذوق عشقت
&&
اگر هر موی من گردد زبانی
\\
اگر هر دو جهان بر من بشورند
&&
ز شور عشق کم نکنم زمانی
\\
مرا جانا از آن خویشتن خوان
&&
توانی دید خود را تا توانی
\\
تو سلطانی اگر محرم نیم من
&&
قبولم کن به جای پاسپانی
\\
چه می‌گویم چه مرد این حدیثم
&&
خطار رفت این سخن یارب امانی
\\
اگر صد بار خواهم کوفت این در
&&
نخواهد گفت کس کامد فلانی
\\
نشان کی ماند از عطار در عشق
&&
چو می‌جوید نشان از بی نشانی
\\
\end{longtable}
\end{center}
