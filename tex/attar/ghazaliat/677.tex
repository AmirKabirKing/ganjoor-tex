\begin{center}
\section*{غزل شماره ۶۷۷: ماییم دل بریده ز پیوند و ناز تو}
\label{sec:677}
\addcontentsline{toc}{section}{\nameref{sec:677}}
\begin{longtable}{l p{0.5cm} r}
ماییم دل بریده ز پیوند و ناز تو
&&
کوتاه کرده قصهٔ زلف دراز تو
\\
تا ترکتاز هندوی زلف تو دیده‌ام
&&
زنگی دلم ز شادی بی ترکتاز تو
\\
هرگز نساخت در ره عشاق پرده‌ای
&&
کان راست بود ترک کج پرده ساز تو
\\
سر در نشیب مانده‌ام از غم چو مست عشق
&&
از شوق زلف عنبری سرفراز تو
\\
گر بود پیش قامت تو سرو در نماز
&&
آزاد شد ز قامت تو در نماز تو
\\
خطت که آفتاب رخت را روان بود
&&
زان خط محقق است که شد نسخ ناز تو
\\
نی نی که هست خط تو سرسبز طوطیی
&&
پرورده است از شکر دلنواز تو
\\
شهباز حسن تو چو ز خط یافت پر و بال
&&
طوطی گرفت غاشیهٔ دلنواز تو
\\
هر روز احتراز تو بیش است سوی من
&&
از حد گذشت شوق من و احتراز تو
\\
از بس که هست در ره سودای تو طلسم
&&
واقف نگشت هیچ‌کس از گنج راز تو
\\
چون از کسی حقیقت رویت طلب کنم
&&
چون کس نبود محرم کوی مجاز تو
\\
سر باز زن چو شمع به گازی فرید را
&&
گر سر دمی چو شمع بتابد ز گاز تو
\\
\end{longtable}
\end{center}
