\begin{center}
\section*{غزل شماره ۴: گفتم اندر محنت و خواری مرا}
\label{sec:004}
\addcontentsline{toc}{section}{\nameref{sec:004}}
\begin{longtable}{l p{0.5cm} r}
گفتم اندر محنت و خواری مرا
&&
چون ببینی نیز نگذاری مرا
\\
بعد از آن معلوم من شد کان حدیث
&&
دست ندهد جز به دشواری مرا
\\
از می عشقت چنان مستم که نیست
&&
تا قیامت روی هشیاری مرا
\\
گر به غارت می‌بری دل باک‌نیست
&&
دل تو را باد و جگرخواری مرا
\\
از تو نتوانم که فریاد آورم
&&
زآنکه در فریاد می‌ناری مرا
\\
گر بنالم زیر بار عشق تو
&&
بار بفزایی به سر باری مرا
\\
گر زمن بیزار گردد هرچه هست
&&
نیست از تو روی بیزاری مرا
\\
از من بیچاره بیزاری مکن
&&
چون همی بینی بدین زاری مرا
\\
گفته بودی کاخرت یاری دهم
&&
چون بمردم کی دهی یاری مرا
\\
پرده بردار و دل من شاد کن
&&
در غم خود تا به کی داری مرا
\\
چبود از بهر سگان کوی خویش
&&
خاک کوی خویش انگاری مرا
\\
مدتی خون خوردم و راهم نبود
&&
نیست استعداد بیزاری مرا
\\
نی غلط گفتم که دل خاکی شدی
&&
گر نبودی از تو دلداری مرا
\\
مانع خود هم منم در راه خویش
&&
تا کی از عطار و عطاری مرا
\\
\end{longtable}
\end{center}
