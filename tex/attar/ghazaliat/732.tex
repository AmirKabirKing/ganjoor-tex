\begin{center}
\section*{غزل شماره ۷۳۲: ای جان ما شرابی از جام تو کشیده}
\label{sec:732}
\addcontentsline{toc}{section}{\nameref{sec:732}}
\begin{longtable}{l p{0.5cm} r}
ای جان ما شرابی از جام تو کشیده
&&
سرمست اوفتاده دل از جهان بریده
\\
وی جان ما به یک دم صد زندگی گرفته
&&
تا از رخت نسیمی بر جان ما وزیده
\\
ای جان پاکبازان در قعر هر دو عالم
&&
مثل تو هیچ گوهر نه دیده نه شنیده
\\
جان‌های عاشقانت چون مرغ بال بسته
&&
در زیر دام دنیا بر بویت آرمیده
\\
آنجا که آتش تو بالا گرفته در دل
&&
هم شمع جان نهاده هم صبح دل دمیده
\\
وآنجا که عرضه داده عشقت امانت خود
&&
هم کوه پست گشته هم چرخ در رمیده
\\
گردون سالخورده بویی شنیده از تو
&&
در جست و جویت از جان چندان به سر دویده
\\
عشقت به لاابالی بر چار سوی عالم
&&
پیران راه‌بین را بر دارها کشیده
\\
در راه انتظارت جان‌ها ز اشتیاقت
&&
چون مرغ نیم بسمل در خاک و خون تپیده
\\
تو فارغ از دو عالم مشغول خویش دایم
&&
وز سختی ره تو کس در تو نارسیده
\\
الحق شگرف مرغی کز تو دو کون پر شد
&&
نه بال باز کرده نه ز آشیان پریده
\\
ای در حجاب عزت پنهان شده ز غیرت
&&
نادیده گرد کویت مردان کار دیده
\\
تو همچو آفتابی در پرده‌ها نشسته
&&
یک آه عاشقانت صد پرده بر دریده
\\
ای جان ما چو آدم شادی هشت جنت
&&
داده به یک دو گندم واندوه تو خریده
\\
در چشم ما نیایی گویی که نور چشمی
&&
یا نور چشم جانی هم جای خود گزیده
\\
بر جان فتاده نورت وز جان فتاده بر دل
&&
وز دل رسیده بویی زان نور سوی دیده
\\
چون صنع توست جمله فارغ ز صنع خویشی
&&
زان دوستی نداری با هیچ آفریده
\\
جمله تویی ولیکن کس دیده‌ای ندارد
&&
زیرا که پرده بینم بر دیده‌ها کشیده
\\
کو دیده‌ای که او را توحید کرده سرمه
&&
تا فرش راز بیند بر کون گستریده
\\
هر بی خبر نشاید این راز را که این را
&&
جانی شگرف باید ذوق لقا چشیده
\\
بحری است حضرت تو جان‌ها جواهر آن
&&
وان بحر سر جان را موجی برآوریده
\\
ای صد هزار کامل در وصف قدرت تو
&&
جان‌های دور فکرت در عجز پروریده
\\
در کشف سر عشقت گردن کشان دین را
&&
سلطان غیرت تو بر خاک خوابنیده
\\
عطار دوربین را اندر مقام وحدت
&&
پروانه‌وار جانش در شمع تو پریده
\\
\end{longtable}
\end{center}
