\begin{center}
\section*{غزل شماره ۳۶۳: تشنه را از سراب چگشاید}
\label{sec:363}
\addcontentsline{toc}{section}{\nameref{sec:363}}
\begin{longtable}{l p{0.5cm} r}
تشنه را از سراب چگشاید
&&
سایه را ز آفتاب چگشاید
\\
آب حیوان چو هست در ظلمات
&&
از نسیم گلاب چگشاید
\\
نیست این کار جنبش و آرام
&&
از درنگ و شتاب چگشاید
\\
قطره‌ای را که او نبود و نه هست
&&
غرق دریای آب چگشاید
\\
بسی ستون است خیمهٔ عالم
&&
از هزاران طناب چگشاید
\\
صد درت گر گشاد پنداری است
&&
این چنین فتح باب چگشاید
\\
چون نبردی بر آب هرگز پی
&&
پی بری بر سر اب چگشاید
\\
گرچه بغنوده‌ای بهر نفسی
&&
عالمی ماهتاب چگشاید
\\
رو که این رهروان چو تشنه شدند
&&
از دو ساغر شرآب چگشاید
\\
خون بسته است اگر کباب خوری
&&
خون خوری از کباب چگشاید
\\
چون کمیت فلک طبق آورد
&&
از خری در خلاب چگشاید
\\
تا بتان در زمین همی ریزند
&&
چرخ را ز انقلاب چگشاید
\\
کار چون ذره‌ای به علت نیست
&&
از خطا و صواب چگشاید
\\
سر یک یک چو او همی داند
&&
از حساب و کتاب چگشاید
\\
از همه چون به از همه است آگاه
&&
از سؤال و جواب چگشاید
\\
چون من از هر دو کون گم گشتم
&&
از ثواب و عقاب چگشاید
\\
گنج می‌جسته‌ام به معموری
&&
هست جای خراب چگشاید
\\
هر چه بیدار دیده‌ام هیچ است
&&
گر ببینم به خواب چگشاید
\\
آفتابی است ذره ذره ولی
&&
هست زیر نقاب چگشاید
\\
ای فرید آسمان نه‌ای آخر
&&
زین همه اضطراب چگشاید
\\
\end{longtable}
\end{center}
