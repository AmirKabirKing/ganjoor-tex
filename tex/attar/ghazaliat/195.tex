\begin{center}
\section*{غزل شماره ۱۹۵: هرچه نشان کنی تویی، راه نشان نمی‌برد}
\label{sec:195}
\addcontentsline{toc}{section}{\nameref{sec:195}}
\begin{longtable}{l p{0.5cm} r}
هرچه نشان کنی تویی، راه نشان نمی‌برد
&&
وآنچه نشان‌پذیر نی، این سخن آن نمی‌برد
\\
گفت زبان ز سر بنه خاک بباش و سر بنه
&&
زانک ز لطف این سخن، گفت زبان نمی‌برد
\\
در دل مرد جوهری است از دوجهان برون شده
&&
پی چو بکرده‌اند گم کس پی آن نمی‌برد
\\
ماه رخا رخ تو را پی نبرد به هیچ روی
&&
هر که به ذوق نیستی راه به جان نمی‌برد
\\
زنده بمردم از غمت خام بسوختم ز تو
&&
تا به کی این فغان برم نیز فغان نمی‌برد
\\
یک سر موی ازین سخن باز نیاید آن کسی
&&
کو بدر تو عقل را موی کشان نمی‌برد
\\
آنچه فرید یافتست از ره عشق ساعتی
&&
هیچ کسی به عمر خود با سر آن نمی‌برد
\\
\end{longtable}
\end{center}
