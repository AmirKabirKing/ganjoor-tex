\begin{center}
\section*{غزل شماره ۵۷۰: رفت وجودم به عدم چون کنم}
\label{sec:570}
\addcontentsline{toc}{section}{\nameref{sec:570}}
\begin{longtable}{l p{0.5cm} r}
رفت وجودم به عدم چون کنم
&&
هیچ شدم هیچ نیم چون کنم
\\
تو همه من هیچ به هم هر دو را
&&
چون به هم اندازم وضم چون کنم
\\
با منی و من ز توام بی خبر
&&
با تو بهم بی تو بهم چون کنم
\\
ای غم عشق تو مرا سوخته
&&
سوخته‌ام بی تو ز غم چون کنم
\\
واقعهٔ عشق توام زنده کرد
&&
یکدم ازین واقعه کم چون کنم
\\
گرچه بسی گرم تر از آتشم
&&
در طلب خویش علم چون کنم
\\
در هوست سر چو درانداختم
&&
پیش‌کشت سر چو قلم چون کنم
\\
چون نتوان کرد ز تو صورتی
&&
صورت محض است صنم چون کنم
\\
ای همه بر هیچ ز تو چون بود
&&
نقش پی نقش رقم چون کنم
\\
کی به دمم نرم شوی زانکه تو
&&
موم نه‌ای نرم بدم چون کنم
\\
ره به درنگ است و درم سوی تو
&&
من نه درنگ و نه درم چون کنم
\\
چون نه مقرم من و نه منکرم
&&
بر سخنی لا و نعم چون کنم
\\
در حرم عشق چو نامحرمم
&&
نیست مرا ره به حرم چون کنم
\\
بر صفت شمع گرفتست سوز
&&
فرق سرم تا به قدم چون کنم
\\
تا بودم یک سر موی از وجود
&&
عزم بیابان عدم چون کنم
\\
بازوی جود است کمال فرید
&&
فربهیش هست ورم چون کنم
\\
\end{longtable}
\end{center}
