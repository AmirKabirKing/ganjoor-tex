\begin{center}
\section*{غزل شماره ۶۳: اگر تو عاشقی معشوق دور است}
\label{sec:063}
\addcontentsline{toc}{section}{\nameref{sec:063}}
\begin{longtable}{l p{0.5cm} r}
اگر تو عاشقی معشوق دور است
&&
وگر تو زاهدی مطلوب حور است
\\
ره عاشق خراب اندر خراب است
&&
ره زاهد غرور اندر غرور است
\\
دل زاهد همیشه در خیال است
&&
دل عاشق همیشه در حضور است
\\
نصیب زاهدان اظهار راه است
&&
نصیب عاشقان دایم حضور است
\\
جهانی کان جهان عاشقان است
&&
جهانی ماورای نار و نور است
\\
درون عاشقان صحرای عشق است
&&
که آن صحرا نه نزدیک و نه دور است
\\
در آن صحرا نهاده تخت معشوق
&&
به گرد تخت دایم جشن و سور است
\\
همه دلها چو گلهای شکفته است
&&
همه جان‌ها چو صف‌های طیور است
\\
سراینده همه مرغان به صد لحن
&&
که در هر لحن صد سور و سرور است
\\
ازان کم می‌رسد هرجان بدین جشن
&&
که ره بس دور و جانان بس غیور است
\\
طریق تو اگر این جشن خواهی
&&
ز جشن عقل و جان و دل عبور است
\\
اگر آنجا رسی بینی وگرنه
&&
دلت دایم ازین پاسخ نفور است
\\
خردمندا مکن عطار را عیب
&&
اگر زین شوق جانش ناصبور است
\\
\end{longtable}
\end{center}
