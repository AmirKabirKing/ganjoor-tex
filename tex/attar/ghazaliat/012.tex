\begin{center}
\section*{غزل شماره ۱۲: روز و شب چون غافلی از روز و شب}
\label{sec:012}
\addcontentsline{toc}{section}{\nameref{sec:012}}
\begin{longtable}{l p{0.5cm} r}
روز و شب چون غافلی از روز و شب
&&
کی کنی از سر روز و شب طرب
\\
روی او چون پرتو افکند اینت روز
&&
زلف او چون سایه انداخت اینت شب
\\
گه کند این پرتو آن سایه نهان
&&
گه کند این سایه آن پرتو طلب
\\
صد هزاران محو در اثبات هست
&&
صد هزار اثبات در محو ای عجب
\\
چون تو در اثبات اول مانده‌ای
&&
مانده‌ای از ننگ خود سردرکنب
\\
تا نمیری و نگردی زنده باز
&&
صد هزاران بار هستی بی ادب
\\
هر که او جایی فرود آمد همی
&&
هست او را مرددون‌همت لقب
\\
چون ز پرده اوفتادی می‌شتاب
&&
تا ابد هرگز مزن دم بی‌طلب
\\
طالب آن باشد که جانش هر نفس
&&
تشنه‌تر باشد ولیکن بی سبب
\\
نه سبب نه علتش باشد پدید
&&
نه بود از خود نه از غیرش نسب
\\
چون نباشد او صفت چون باشدش
&&
خود همه اوست اینت کاری بوالعجب
\\
گر تو را باید که این سر پی بری
&&
خویش را از سلب او سازی سلب
\\
بر کنار گنج ماندی خاک بیز
&&
در میان بحر ماندی خشک لب
\\
چون رطب آمد غرض از استخوان
&&
استخوان تا چند خائی بی رطب
\\
هین شراب صرف درکش مردوار
&&
پس دو عالم پر کن از شور و شعب
\\
مست جاویدان شو و فانی بباش
&&
تا شوی جاوید آزاد از تعب
\\
چون تو آزاد آیی از ننگ وجود
&&
راستت آن وقت گیرد حکم چپ
\\
از دم آن کس که این می نوش کرد
&&
دوزخ سوزنده را بگرفت تب
\\
همچو عطار این شراب صاف عشق
&&
نوش کن از دست ساقی عرب
\\
\end{longtable}
\end{center}
