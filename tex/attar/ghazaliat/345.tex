\begin{center}
\section*{غزل شماره ۳۴۵: رهبان دیر را سبب عاشقی چه بود}
\label{sec:345}
\addcontentsline{toc}{section}{\nameref{sec:345}}
\begin{longtable}{l p{0.5cm} r}
رهبان دیر را سبب عاشقی چه بود
&&
کو روی را ز دیر به خلقان نمی‌نمود
\\
از نیستی دو دیده به کس می‌نکرد باز
&&
ور راستی روان خلایق همی ربود
\\
چون در فتاد در محن عشق زان سپس
&&
در مهر دل عبادت عیسی همی شنود
\\
در ملت مسیح روا نیست عاشقی
&&
او عاشق از چه بود و چرا در بلا فزود
\\
مانا که یار ما به خرابات برگذشت
&&
وز حال دل به نغمه سرودی همی سرود
\\
می‌گفت هر که دوست کند در بلا فتد
&&
عاشق زیان کند دو جهان از برای سود
\\
رهبان طواف دیر همی کرد ناگهان
&&
کاواز آن نگار خراباتیان شنود
\\
برشد به بام دیر چو رخسار او بدید
&&
از آرزوش روی به خاک‌اندرون بسود
\\
دیوانه شد ز عشق و برآشفت در زمان
&&
زنجیر نعت صورت عیسی برید زود
\\
آتش به دیر در زد و بتخانه در شکست
&&
وز سقف دیر او به سما بر رسید دود
\\
باده ز دست دوست دمادم همی کشید
&&
زنگ بلا ز ساغر و مطرب همی زدود
\\
سرمست و بیقرار همی گفت و می‌گریست
&&
ناکردنی بکردم و نابودنی ببود
\\
\end{longtable}
\end{center}
