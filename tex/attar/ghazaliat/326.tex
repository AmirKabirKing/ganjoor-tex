\begin{center}
\section*{غزل شماره ۳۲۶: هر که را با لب تو پیمان بود}
\label{sec:326}
\addcontentsline{toc}{section}{\nameref{sec:326}}
\begin{longtable}{l p{0.5cm} r}
هر که را با لب تو پیمان بود
&&
اجل او از آب حیوان بود
\\
هر که روی چو آفتاب تو دید
&&
همچو من تا که بود حیران بود
\\
در نکویی پسندهٔ جایی
&&
که نکوتر از آن بنتوان بود
\\
چون بدیدم لب جگر رنگت
&&
نمکی داشت و شکرافشان بود
\\
یک شکر آرزوم کرد الحق
&&
لیک بیمم ز تیر مژگان بود
\\
بی رخت بر رخم نوشت به خون
&&
دیده هر راز دل که پنهان بود
\\
خواستم تا نفس زنم بی تو
&&
نزدم زانکه آن نفس جان بود
\\
جان من گر بود وگر نبود
&&
کی مرا در جهان غم آن بود
\\
لیک جان زان سبب ندادم من
&&
که نه در خورد چون تو جانان بود
\\
جان بدادم چو روی تو دیدم
&&
زانکه جان دادن من آسان بود
\\
جان عطار تا که بود از تو
&&
هستی و نیستیش یکسان بود
\\
\end{longtable}
\end{center}
