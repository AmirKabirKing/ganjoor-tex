\begin{center}
\section*{غزل شماره ۵۵۳: کجا بودم کجا رفتم کجاام من نمی‌دانم}
\label{sec:553}
\addcontentsline{toc}{section}{\nameref{sec:553}}
\begin{longtable}{l p{0.5cm} r}
کجا بودم کجا رفتم کجاام من نمی‌دانم
&&
به تاریکی در افتادم ره روشن نمی‌دانم
\\
ندارم من درین حیرت به شرح حال خود حاجت
&&
که او داند که من چونم اگرچه من نمی‌دانم
\\
چو من گم گشته‌ام از خود چه جویم باز جان و تن
&&
که گنج جان نمی‌بینم طلسم تن نمی‌دانم
\\
چگونه دم توانم زد درین دریای بی پایان
&&
که درد عاشقان آنجا به جز شیون نمی‌دانم
\\
برون پرده گر مویی کنی اثبات شرک افتد
&&
که من در پرده جز نامی ز مرد و زن نمی‌دانم
\\
در آن خرمن که جان من در آنجا خوشه می‌چیند
&&
همه عالم و مافیها به نیم ارزن نمی‌دانم
\\
از آنم سوخته خرمن که من عمری درین صحرا
&&
اگرچه خوشه می‌چینم ره خرمن نمی‌دانم
\\
چو از هر دو جهان خود را نخواهم مسکنی هرگز
&&
سزای درد این مسکین یکی مسکن نمی‌دانم
\\
چو آن گلشن که می‌جویم نخواهد یافت هرگز کس
&&
ره عطار را زین غم به جز گلخن نمی‌دانم
\\
\end{longtable}
\end{center}
