\begin{center}
\section*{غزل شماره ۳۷۰: کسی کو هرچه دید از چشم جان دید}
\label{sec:370}
\addcontentsline{toc}{section}{\nameref{sec:370}}
\begin{longtable}{l p{0.5cm} r}
کسی کو هرچه دید از چشم جان دید
&&
هزاران عرش در مویی عیان دید
\\
عدد از عقل خاست اما دل پاک
&&
عدد گردید از گفت زبان دید
\\
چو این آن است و آن این است جاوید
&&
چرا پس عقل احول این و آن دید
\\
چو دریا عقل دایم قطره بیند
&&
به چشم او نشاید جاودان دید
\\
کسی کو بر احد حکم عدد کرد
&&
جمال بی نشانی را نشان دید
\\
به جان بین هرچه می‌بینی که توحید
&&
کسی کو محو شد از چشم جان دید
\\
چو دو عالم ز یک جوهر برآمد
&&
در اندک جوهری بسیار کان دید
\\
ازل را و ابد را نقطه‌ای یافت
&&
همه کون و مکان را لامکان دید
\\
یقین می‌دان که چشم جان چنان است
&&
که در هر ذره‌ای هفت آسمان دید
\\
ولی هر ذره‌ای از آسمان نیز
&&
به عینه هم زمین و هم زمان دید
\\
چه جای آسمان است و زمین است
&&
که در هر ذره‌ای هر دو جهان دید
\\
چه می‌گویم که عالم صد هزاران
&&
ورای هر دو عالم می‌توان دید
\\
همی در هرچه خواهی هرچه خواهی
&&
به چشم جان توانی بی‌گمان دید
\\
تو در قدرت نگر تا آشکارا
&&
ببینی آنچه غیر تو نهان دید
\\
چو هر دو کون در جنب حقیقت
&&
بسی کمتر ز تاری ریسمان دید
\\
اگر یک ذره رنگ کل پذیرد
&&
عجب نبود چنین باید چنان دید
\\
اگر یک ذره را در قرص خورشید
&&
کسی گم کرد چه سود و زیان دید
\\
کسی کز ذره ذره بند دارد
&&
نیارد ذره‌ای زان آستان دید
\\
اگر یک ذره سایه پیش خورشید
&&
پدید آمد ندانم تا امان دید
\\
دو عالم چیست از یک ذره سایه‌ست
&&
که آنجا ذره‌ای را خط روان دید
\\
طلسم نور و ظلمت بی‌قیاس است
&&
ولیکن گنج باید در میان دید
\\
کسی کان گنج می‌بیند طلسمش
&&
فنا شد تا دو عالم دلستان دید
\\
گزیرت نیست از چشمی که جاوید
&&
ندید او غیر هرگز غیب‌دان دید
\\
ز خود گم گرد ای عطار اینجا
&&
که تا خود را توانی کامران دید
\\
\end{longtable}
\end{center}
