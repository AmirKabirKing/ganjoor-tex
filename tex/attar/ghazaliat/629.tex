\begin{center}
\section*{غزل شماره ۶۲۹: چون قصهٔ زلف تو دراز است چگویم}
\label{sec:629}
\addcontentsline{toc}{section}{\nameref{sec:629}}
\begin{longtable}{l p{0.5cm} r}
چون قصهٔ زلف تو دراز است چگویم
&&
چون شیوهٔ چشمت همه ناز است چگویم
\\
این است حقیقت که ز وصل تو نشان نیست
&&
هر قصه که این نیست مجاز است چگویم
\\
خورشید که او چشم و چراغ است جهان را
&&
از شوق رخت در تک و تاز است چگویم
\\
چون شمع سحرگاه دل سوخته هر شب
&&
بی روی تو در سوز و گداز است چگویم
\\
تا دست به زلف تو رسد در همه عمرم
&&
چون زلف توام کار دراز است چگویم
\\
گر کرد مرا زلف تو با خاک برابر
&&
لعل لب تو بنده نواز است چگویم
\\
المنه‌لله که دلم گرچه ربودی
&&
از زلف تو در پردهٔ راز است چگویم
\\
گفتی که بگو تا چه کشیدی تو ز نازم
&&
کار من دلخسته نیاز است چگویم
\\
گفتم که در بسته مرا چند نمایی
&&
گفتی که درم بر همه باز است چگویم
\\
گر بر همه باز است در وصل تو جانا
&&
چون بر من سرگشته فراز است چگویم
\\
عطار درین کوی اگر نیک و اگر بد
&&
پروانهٔ آن شمع طراز است چگویم
\\
\end{longtable}
\end{center}
