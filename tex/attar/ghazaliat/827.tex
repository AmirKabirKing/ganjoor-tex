\begin{center}
\section*{غزل شماره ۸۲۷: نگر تا ای دل بیچاره چونی}
\label{sec:827}
\addcontentsline{toc}{section}{\nameref{sec:827}}
\begin{longtable}{l p{0.5cm} r}
نگر تا ای دل بیچاره چونی
&&
چگونه می‌روی سر در نگونی
\\
چگونه می‌کشی صد بحر آتش
&&
چو اندر نفس خود یک قطره خونی
\\
زمانی در تماشای خیالی
&&
زمانی در تمنای جنونی
\\
اگر خواهی که باشی از بزرگان
&&
مباش از خرده‌گیران کنونی
\\
چرا باشی نه کافر نه مسلمان
&&
که تو نه رهروی نه رهنمونی
\\
ز یک یک ذره سوی دوست راه است
&&
ولی ره نیست بهتر از زبونی
\\
زبون عشق شو تا بر کشندت
&&
که هرگاهی که کم گشتی فزونی
\\
خود از رفعت ورای هر دو کونی
&&
چرا هم‌صحبت این نفس دونی
\\
دلا تو چیستی هستی تو یا نه
&&
وگر نه نیستی نه هست چونی
\\
منی یا نه منی عینی تو یا غیر
&&
و یا از هرچه اندیشم برونی
\\
چه می‌گویم تو خود از خود نهانی
&&
که دو انگشت حق را در درونی
\\
تو ای عطار اگر چه دل نداری
&&
ولیکن اهل دل را ذوفنونی
\\
\end{longtable}
\end{center}
