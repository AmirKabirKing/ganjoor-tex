\begin{center}
\section*{غزل شماره ۱۱۸: کیست که از عشق تو پردهٔ او پاره نیست}
\label{sec:118}
\addcontentsline{toc}{section}{\nameref{sec:118}}
\begin{longtable}{l p{0.5cm} r}
کیست که از عشق تو پردهٔ او پاره نیست
&&
وز قفس قالبش مرغ دل آواره نیست
\\
وزن کجا آورد خاصه به میزان عشق
&&
گر زر عشاق را سکهٔ رخساره نیست
\\
هر نفسم همچو شمع زاربکش پیش خویش
&&
گر دل پر خون من کشتهٔ صد پاره نیست
\\
گر تو ز من فارغی من ز تو فارغ نیم
&&
چارهٔ کارم بکن کز تو مرا چاره نیست
\\
هر که درین راه یافت بوی می عشق تو
&&
مست شود تا ابد گر دلش از خاره نیست
\\
هست همه گفتگو با می عشقش چه کار
&&
هرکه درین میکده مفلس و این کاره نیست
\\
درد ره و درد دیر هست محک مرد را
&&
دلق بیفکن که زرق لایق میخواره نیست
\\
در بن این دیر اگر هست میت آرزو
&&
درد خور اینجا که دیر موضع نظاره نیست
\\
گشت هویدا چو روز بر دل عطار از آنک
&&
عهد ندارد درست هر که درین پاره نیست
\\
\end{longtable}
\end{center}
