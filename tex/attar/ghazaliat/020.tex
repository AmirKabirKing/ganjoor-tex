\begin{center}
\section*{غزل شماره ۲۰: ای شکر خوشه‌چین گفتارت}
\label{sec:020}
\addcontentsline{toc}{section}{\nameref{sec:020}}
\begin{longtable}{l p{0.5cm} r}
ای شکر خوشه‌چین گفتارت
&&
سرو آزاد کرد رفتارت
\\
بس که طوطی جان بزد پر و بال
&&
ز اشتیاق لب شکر بارت
\\
خار در پای گل شکست هزار
&&
ز آرزوی رخ چو گلنارت
\\
هر شبی با هزار دیده سپهر
&&
مانده در انتظار دیدارت
\\
لعل از جان بشسته دست به خون
&&
شده مبهوت جزع خون‌خوارت
\\
نرگس تر که ساقی چمن است
&&
حلقه در گوش چشم مکارت
\\
هرکه را از هزار گونه جفا
&&
دل ببردی به‌جان گرفتارت
\\
بحر از آن جوش می‌زند لب خشک
&&
که بدیدست در شهوارت
\\
آسمان می‌کند زمین بوست
&&
زانکه سرگشته گشت در کارت
\\
گشت دندان عاشقان همه کند
&&
زانکه بس تیز گشت بازارت
\\
بر دل و جان من جهان مفروش
&&
که به جان و دلم خریدارت
\\
بر بناگوش توست حلقهٔ زلف
&&
حلقه در گوش کرده عطارت
\\
\end{longtable}
\end{center}
