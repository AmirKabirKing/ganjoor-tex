\begin{center}
\section*{غزل شماره ۲۶۴: چه دانستم که این دریای بی پایان چنین باشد}
\label{sec:264}
\addcontentsline{toc}{section}{\nameref{sec:264}}
\begin{longtable}{l p{0.5cm} r}
چه دانستم که این دریای بی پایان چنین باشد
&&
بخارش آسمان گردد کف دریا زمین باشد
\\
لب دریا همه کفر است و دریا جمله دین‌داری
&&
ولیکن گوهر دریا ورای کفر و دین باشد
\\
اگر آن گوهر و دریا به هم هر دو به دست آری
&&
تورا آن باشد و این هم ولی نه آن نه این باشد
\\
یقین می‌دان که هم هر دو بود هم هیچیک نبود
&&
یقین نبود گمان باشد گمان نبود یقین باشد
\\
درین دریا که من هستم نه من هستم نه دریا هم
&&
نداند هیچکس این سر مگر آن کو چنین باشد
\\
اگر خواهی کزین دریا وزین گوهر نشان یابی
&&
نشانی نبودت هرگز چو نفست همنشین باشد
\\
اگر صد سال روز و شب ریاضت می‌کشی دایم
&&
مباش ایمن یقین می‌دان که نفست در کمین باشد
\\
چو تو نفسی ز سر تا پای کی دانی کمال دل
&&
کمال دل کسی داند که مردی راه‌بین باشد
\\
تو صاحب نفسی ای غافل میان خاک خون می خور
&&
که صاحبدل اگر زهری خورد آن انگبین باشد
\\
نداند کرد صاحب‌نفس کار هیچ صاحبدل
&&
وگر گوید توانم کرد ابلیس لعین باشد
\\
اگر خواهی که بشناسی که کاری راستین هستت
&&
قدم در شرع محکم کن که کارت راستین باشد
\\
اگر از نقطهٔ تقوی بگردد یک دمت دیده
&&
سزای دیدهٔ گردیده میل آتشین باشد
\\
تو ای عطار محکم کن قدم در جادهٔ معنی
&&
که اندر خاتم معنی لقای حق نگین باشد
\\
\end{longtable}
\end{center}
