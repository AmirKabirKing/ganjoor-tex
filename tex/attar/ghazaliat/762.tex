\begin{center}
\section*{غزل شماره ۷۶۲: خطی از غالیه بر غالیه‌دان آوردی}
\label{sec:762}
\addcontentsline{toc}{section}{\nameref{sec:762}}
\begin{longtable}{l p{0.5cm} r}
خطی از غالیه بر غالیه‌دان آوردی
&&
دل این سوخته را کار به جان آوردی
\\
نه که منشور نکویی تو بی طغرا بود
&&
رفتی از غالیه طغرا و نشان آوردی
\\
تا به ماهت نرسد چشم بد هیچ کسی
&&
ماه را در زره مشک‌فشان آوردی
\\
نیست از جانب من تا به تو یک موی میان
&&
تو چرا بیهده از موی میان آوردی
\\
هرکه او از سر کوی تو به مویی سر تافت
&&
با سر موی خودش موی کشان آوردی
\\
گفتم از لعل لبت یک شکر آرم بر زخم
&&
گفت آری شدی و زخم زبان آوردی
\\
خواست از لعل تو عطار به عمری شکری
&&
جگرش خوردی و کارش به زیان آوردی
\\
\end{longtable}
\end{center}
