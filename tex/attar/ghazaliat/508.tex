\begin{center}
\section*{غزل شماره ۵۰۸: عشق بالای کفر و دین دیدم}
\label{sec:508}
\addcontentsline{toc}{section}{\nameref{sec:508}}
\begin{longtable}{l p{0.5cm} r}
عشق بالای کفر و دین دیدم
&&
بی نشان از شک و یقین دیدم
\\
کفر و دین و شک و یقین گر هست
&&
همه با عقل همنشین دیدم
\\
چون گذشتم ز عقل صد عالم
&&
چون بگویم که کفر و دین دیدم
\\
هرچه هستند سد راه خودند
&&
سد اسکندری من این دیدم
\\
فانی محض گرد تا برهی
&&
راه نزدیکتر همین دیدم
\\
چون من اندر صفات افتادم
&&
چشم صورت صفات بین دیدم
\\
هر صفت را که محو می‌کردم
&&
صفتی نیز در کمین دیدم
\\
جان خود را چو از صفات گذشت
&&
غرق دریای آتشین دیدم
\\
خرمن من چو سوخت زان دریا
&&
ماه و خورشید خوشه‌چین دیدم
\\
گفتی آن بحر بی نهایت را
&&
جنت عدن و حور عین دیدم
\\
چون گذر کردم از چنان بحری
&&
رخش خورشید زیر زین دیدم
\\
حلقه‌ای یافتم دو عالم را
&&
دل در آن حلقه چون نگین دیدم
\\
آخر الامر زیر پردهٔ غیب
&&
روی آن ماه نازنین دیدم
\\
آسمان را که حلقهٔ در اوست
&&
پیش او روی بر زمین دیدم
\\
بر رخ او که عکس اوست دو کون
&&
برقع از زلف عنبرین دیدم
\\
نقش های دو کون را زان زلف
&&
گره و تاب و بند و چین دیدم
\\
هستی خویش پیش آن خورشید
&&
سایهٔ یار راستین دیدم
\\
دامنش چون به دست بگرفتم
&&
دست او اندر آستین دیدم
\\
هر که او سر این حدیث شناخت
&&
نقطهٔ دولتش قرین دیدم
\\
جان عطار را نخستین گام
&&
برتر از چرخ هفتمین دیدم
\\
\end{longtable}
\end{center}
