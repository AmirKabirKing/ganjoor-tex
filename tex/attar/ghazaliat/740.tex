\begin{center}
\section*{غزل شماره ۷۴۰: ماه را در مشک پنهان کرده‌ای}
\label{sec:740}
\addcontentsline{toc}{section}{\nameref{sec:740}}
\begin{longtable}{l p{0.5cm} r}
ماه را در مشک پنهان کرده‌ای
&&
مشک را بر مه پریشان کرده‌ای
\\
چشم عقل دوربین را روز و شب
&&
بر جمال خویش حیران کرده‌ای
\\
از شکنج زلف رستم افکنت
&&
هر زمان صد گونه دستان کرده‌ای
\\
دام مشکین است زلف عنبرینت
&&
دام مشکین عنبر افشان کرده‌ای
\\
من دل و جان خوانمت از جان و دل
&&
تو چنین قصد دل و جان کرده‌ای
\\
یوسف عهدی کزان چاه چو سیم
&&
پوست بر من همچو زندان کرده‌ای
\\
گفتمت بردی به بازی دل ز من
&&
این خصومت باز بازان کرده‌ای
\\
چشم تو می‌گوید از تو خامشی
&&
کین چنین بازی فراوان کرده‌ای
\\
در صفات حسن خود عطار را
&&
تا که جان دارد ثناخوان کرده‌ای
\\
\end{longtable}
\end{center}
