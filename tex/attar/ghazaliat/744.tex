\begin{center}
\section*{غزل شماره ۷۴۴: بحری است عشق و عقل ازو برکناره‌ای}
\label{sec:744}
\addcontentsline{toc}{section}{\nameref{sec:744}}
\begin{longtable}{l p{0.5cm} r}
بحری است عشق و عقل ازو برکناره‌ای
&&
کار کنارگی نبود جز نظاره‌ای
\\
در بحر عشق عقل اگر راهبر بدی
&&
هرگز کجا فتادی ازو برکناره‌ای
\\
وانجا که بحر عشق درآید به جان و دل
&&
عقل است اعجمی و خرد شیرخواره‌ای
\\
در پردهٔ وجود ز هستی عدم شوند
&&
آنها که ره برند درین پرده پاره‌ای
\\
بسیار چاره می‌طلبی تا که سر عشق
&&
یک دم شود به پیش تو چون آشکاره‌ای
\\
گر صد هزار سال درین ره قدم زنی
&&
تا تو تویی تو را نتوان کرد چاره‌ای
\\
تو درد عشق خود چه شناسی که چون بود
&&
تا بر دلت ز عشق نیاید کتاره‌ای
\\
در هر هزار سال به برج دلی رسد
&&
از آسمان عشق بدین سان ستاره‌ای
\\
عطار اگر پیاده شوی از دو کون تو
&&
در هر دو کون چون تو نباشد سواره‌ای
\\
\end{longtable}
\end{center}
