\begin{center}
\section*{غزل شماره ۳۵۵: یک ذره نور رویت گر ز آسمان برآید}
\label{sec:355}
\addcontentsline{toc}{section}{\nameref{sec:355}}
\begin{longtable}{l p{0.5cm} r}
یک ذره نور رویت گر ز آسمان برآید
&&
افلاک درهم افتد خورشید بر سرآید
\\
آخر چه طاقت آرد اندر دو کون هرگز
&&
تا با فروغ رویت اندر برابر آید
\\
یارب چه آفتابی کانجا که پرتو توست
&&
هم و هم تیره گردد هم فهم ابتر آید
\\
چه جای وهم و فهم است کاندر حوالی تو
&&
نه روح لایق افتد نه عقل در خور آید
\\
هر کو ز ناتمامی از تو وصال جوید
&&
در عشق تو بسوزد از جان و دل برآید
\\
ور از عنایت تو جان را رسد نسیمی
&&
اقبال جاودانی جان را ز در درآید
\\
هرگه که شرح رویت عطار پیش گیرد
&&
کام و لبش ز معنی پر در و گوهر آید
\\
\end{longtable}
\end{center}
