\begin{center}
\section*{غزل شماره ۸۰۲: ای یک کرشمه تو غارتگر جهانی}
\label{sec:802}
\addcontentsline{toc}{section}{\nameref{sec:802}}
\begin{longtable}{l p{0.5cm} r}
ای یک کرشمه تو غارتگر جهانی
&&
دشنام تو خریده ارزان خران به جانی
\\
آشفتهٔ رخ تو هرجا که ماهرویی
&&
دلداهٔ لب تو هر جا که دلستانی
\\
گر از دهان تنگت بوسی به من فرستی
&&
جان‌های تنگ بسته برهم نهم جهانی
\\
تو خود دهان نداری چون بوسه خواهم از تو
&&
هرگز برون نگنجد بوس از چنین دهانی
\\
چون تو میان نداری من با کنار رفتم
&&
چون دست درکش آرد کس با چنان میانی
\\
تو یوسفی و هر دم زلف تو از نسیمی
&&
کرده روان به کنعان از مشک کاروانی
\\
دیری است تا دل من از درد توست سوزان
&&
آخر دلت نسوزد بر درد من زمانی
\\
گفتی بخواه چیزی کان سودمندت آید
&&
کز سود کردن تو نبود مرا زیانی
\\
وقت بهار خواهم در نور شمع رویت
&&
من کرده در رخ تو هر لحظه گلفشانی
\\
عطار اگرت بیند یک شب چنان که گفتم
&&
صد جان تازه یابد آنگاه هر زمانی
\\
\end{longtable}
\end{center}
