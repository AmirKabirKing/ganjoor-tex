\begin{center}
\section*{غزل شماره ۵۳۲: خبرت هست که خون شد جگرم}
\label{sec:532}
\addcontentsline{toc}{section}{\nameref{sec:532}}
\begin{longtable}{l p{0.5cm} r}
خبرت هست که خون شد جگرم
&&
وز می عشق تو چون بی خبرم
\\
زآرزوی سر زلف تو مدام
&&
چون سر زلف تو زیر و زبرم
\\
نتوان گفت به صد سال آن غم
&&
کز سر زلف تو آمد به سرم
\\
می‌تپم روز و شب و می‌سوزم
&&
تا که بر روی تو افتد نظرم
\\
خود ز خونابهٔ چشمم نفسی
&&
نتوانم که به تو در نگرم
\\
گر به روز اشک چو در می‌بارم
&&
می‌بر آید دل پر خون ز برم
\\
چون نبینم نظری روی تو من
&&
به تماشای خیال تو درم
\\
گر نخوردی غم این سوخته دل
&&
غم عشق تو بخوردی جگرم
\\
چند گویی که تو خود زر داری
&&
پشت گرمی تو غمت را چه خورم
\\
دور از روی تو گر درنگری
&&
پشت گرمی است ز روی چو زرم
\\
روی عطار چو زر زان بشکست
&&
که زری نیست به وجه دگرم
\\
\end{longtable}
\end{center}
