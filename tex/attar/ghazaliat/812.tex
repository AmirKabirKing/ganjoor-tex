\begin{center}
\section*{غزل شماره ۸۱۲: بس نادره جهانی ای جان و زندگانی}
\label{sec:812}
\addcontentsline{toc}{section}{\nameref{sec:812}}
\begin{longtable}{l p{0.5cm} r}
بس نادره جهانی ای جان و زندگانی
&&
جان و دلم نماند گر تو چنین بمانی
\\
شاهی خوب رویان ختم است بر تو اکنون
&&
بستان خراج خوبی در ملک کامرانی
\\
از چشم نیم مستت پر فتنه شد جهانی
&&
آخر بدین شگرفی چه فتنهٔ جهانی
\\
نه گفته‌ای کزین پس فتنه نخواهم انگیخت
&&
پس طره نیز مفشان گر فتنه می‌نشانی
\\
تا دید آب حیوان لعل چو آتش تو
&&
شد از جهان به یکسو از شرم در نهانی
\\
چون هر نفس لب تو جانی دگر ببخشد
&&
کس ننگرد به عمری در آب زندگانی
\\
هرچند جان شیرین بردی به تلخی از من
&&
تلخیم کرد لیکن شیرین ترم ز جانی
\\
چون جان شوربختم شیرینی از تو دارد
&&
شاید اگر به تلخی جانم به لب رسانی
\\
عطار از غم تو زحمت کشید عمری
&&
گر بر من ستمکش رحمت کنی توانی
\\
\end{longtable}
\end{center}
