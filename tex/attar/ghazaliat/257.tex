\begin{center}
\section*{غزل شماره ۲۵۷: هر که در راه حقیقت از حقیقت بی‌نشان شد}
\label{sec:257}
\addcontentsline{toc}{section}{\nameref{sec:257}}
\begin{longtable}{l p{0.5cm} r}
هر که در راه حقیقت از حقیقت بی‌نشان شد
&&
مقتدای عالم آمد پیشوای انس و جان شد
\\
هر که مویی آگه است از خویشتن یا از حقیقت
&&
او ز خود بیرون نیامد چون به نزد او توان شد
\\
آن خبر دارد ازو کو در حقیقت بی‌خبر گشت
&&
وان اثر دارد که او در بی‌نشانی بی نشان شد
\\
تا تو در اثبات و محوی مبتلایی فرخ آن کس
&&
کو ازین هر دو کناری جست و ناگه از میان شد
\\
گم شدن از محو، پیدا گشتن از اثبات تا کی
&&
مرد آن را دان که چون مردان ورای این و آن شد
\\
هر که از اثبات آزاد آمد و از محو فارغ
&&
هرچه بودش آرزو تا چشم برهم زد عیان شد
\\
هست بال مرغ جان اثبات و پرش محو مطلق
&&
بال و پر فرع است بفکن تا توانی اصل جان شد
\\
تن در اثبات است و جان در محو ازین هر دو برون شو
&&
کانک ازین هر دو برون شد او عزیز جاودان شد
\\
آنکه بیرون شد ازین هر دو نهان و آشکارا
&&
کی توان گفتن که این کس آشکارا یا نهان شد
\\
تا خلاصی یافت عطار از میان این دو دریا
&&
غرقهٔ دریای دیگر گشت و دایم کامران شد
\\
\end{longtable}
\end{center}
