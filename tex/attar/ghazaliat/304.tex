\begin{center}
\section*{غزل شماره ۳۰۴: آفتاب رخ آشکاره کند}
\label{sec:304}
\addcontentsline{toc}{section}{\nameref{sec:304}}
\begin{longtable}{l p{0.5cm} r}
آفتاب رخ آشکاره کند
&&
جگرم ز اشتیاق پاره کند
\\
از پس پرده روی بنماید
&&
مهر و مه را دو پیشکاره کند
\\
شوق رویش چو روی پر از اشک
&&
روی خورشید پر ستاره کند
\\
لعل دانی که چیست رخش لبش
&&
خون خارا ز سنگ خاره کند
\\
هر که او روی چو گلش خواهد
&&
مدتی خار پشتواره کند
\\
در میان با کسی همی آید
&&
کان کس اول ز جان کناره کند
\\
عاشقانی که وصل او طلبند
&&
همه را دوع در کواره کند
\\
بالغان در رهش چو طفل رهند
&&
جمله را گور گاهواره کند
\\
تا کسی روی او نداند باز
&&
چهرهٔ مردم آشکاره کند
\\
نور رویش ز هر دریچهٔ چشم
&&
چون سیه پوش شد نظاره کند
\\
عشق او در غلط بسی فکند
&&
چون نداند کسی چه چاره کند
\\
نتوانیم توبه کرد ز عشق
&&
توبه را صد هزار باره کند
\\
شیر عشقش چو پنجه بگشاید
&&
عقل را طفل شیرخواره کند
\\
زور یک ذره عشق چندان است
&&
که ز هر سو جهان گذاره کند
\\
ضربت عشق با فرید آن کرد
&&
که ندانم که صد کتاره کند
\\
\end{longtable}
\end{center}
