\begin{center}
\section*{غزل شماره ۷۸۰: دوش سرمست به وقت سحری}
\label{sec:780}
\addcontentsline{toc}{section}{\nameref{sec:780}}
\begin{longtable}{l p{0.5cm} r}
دوش سرمست به وقت سحری
&&
می‌شدم تا به بر سیم‌بری
\\
تیز کرده سر دندان که مگر
&&
بربایم ز لب او شکری
\\
چون ربودم شکری از لب او
&&
بنشستم به امید دگری
\\
جگرم سوخت که از لعل لبش
&&
شکری می نرسد بی جگری
\\
گاهگاهی شکری می‌دهدم
&&
بر سر پای روان در گذری
\\
زین چنین بوسه چه در کیسه کنم
&&
وای از غصهٔ بیدادگری
\\
زان همه تنگ شکر کو راهست
&&
از قضا قسم من آمد قدری
\\
تا خبر یافته‌ام از شکرش
&&
نیست از هستی خویشم خبری
\\
کارم از دست شد و کار مرا
&&
نیست چون دایره پایی و سری
\\
وقت نامد که شوم جملهٔ عمر
&&
همچو نی با شکری در کمری
\\
ماه‌رویا دل عطار بسوخت
&&
مکن و در دل او کن نظری
\\
\end{longtable}
\end{center}
