\begin{center}
\section*{غزل شماره ۳۷۹: درد کو تا دردوار خواهم رسید}
\label{sec:379}
\addcontentsline{toc}{section}{\nameref{sec:379}}
\begin{longtable}{l p{0.5cm} r}
درد کو تا دردوا خواهم رسید
&&
خوت کو تا در رجا خواهم رسید
\\
چون تهی دستم ز علم و از عمل
&&
پس چگونه در جزا خواهم رسید
\\
بی سر و پای است این راه عظیم
&&
من به سر یا من به پا خواهم رسید
\\
در چنین راهی قوی کاری بود
&&
گر به یک بانگ درا خواهم رسید
\\
می‌روم پیوسته در قعر دلم
&&
می‌ندانم تا کجا خواهم رسید
\\
جان توان دادن درین دریای خون
&&
تا مگر در آشنا خواهم رسید
\\
پی کسی بر آب دریا کی برد
&&
من به گرداب بلا خواهم رسید
\\
هر دم این دریا جهانی خلق خورد
&&
گرچه من بر ناشتا خواهم رسید
\\
علم در علم است این دریای ژرف
&&
من چنین جاهل کجا خواهم رسید
\\
گر هزاران ساله علم آنجا برم
&&
آن زمان از روستا خواهم رسید
\\
هیچ نتوان بردن آنجا جز فنا
&&
کز بقا بس مبتلا خواهم رسید
\\
هر که فانی شد درین دریا برست
&&
وای بر من گر به پا خواهم رسید
\\
بیخودی است اینجا صواب هر دو کون
&&
گر رسم با خود خطا خواهم رسید
\\
شبنمی‌ام ذره‌ای دارم فنا
&&
کی به دریای بقا خواهم رسید
\\
برنتابم این فنا سختی کشم
&&
خوش بود گر در فنا خواهم رسید
\\
کی شود عطار الا لا شود
&&
زانچه بر الا بلا خواهم رسید
\\
\end{longtable}
\end{center}
