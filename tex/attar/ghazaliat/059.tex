\begin{center}
\section*{غزل شماره ۵۹: آن دهان نیست که تنگ شکر است}
\label{sec:059}
\addcontentsline{toc}{section}{\nameref{sec:059}}
\begin{longtable}{l p{0.5cm} r}
آن دهان نیست که تنگ شکر است
&&
وان میان نیست که مویی دگر است
\\
زان تنم شد چو میانت باریک
&&
کز دهان تو دلم تنگ‌تر است
\\
به دهان و به میانت ماند
&&
چشم سوزن که به دو رشته در است
\\
هر که مویی ز میان و ز دهانت
&&
خبری باز دهد بی‌خبر است
\\
از میان تو سخن چون مویی است
&&
وز دهان تو سخن چون شکر است
\\
نه کمر را ز میانت وطنی است
&&
نه سخن را ز دهانت گذر است
\\
میم دیدی که به جای دهن است
&&
موی دیدی که میان کمر است
\\
چه میان چون الفی معدوم است
&&
چه دهان چون صدفی پر گوهر است
\\
چون میان تو سخن گفت فرید
&&
چون دهان تو از آن نامور است
\\
\end{longtable}
\end{center}
