\begin{center}
\section*{غزل شماره ۷۱۳: ای دل اندر عشق، دل در یار ده}
\label{sec:713}
\addcontentsline{toc}{section}{\nameref{sec:713}}
\begin{longtable}{l p{0.5cm} r}
ای دل اندر عشق، دل در یار ده
&&
کار او کن جان و دل در کار ده
\\
چند باشی در حجاب خود نهان
&&
دلبرت صد بار آمد بار ده
\\
یا برو گر مؤمنی اسلام آر
&&
یا بیا گر کافری اقرار ده
\\
خون خوری بر روی آن دلدار خور
&&
خون دهی بر روی آن دلدار ده
\\
آرزوهای تو بت‌های تواند
&&
جمله بت‌هات بر دیوار ده
\\
پس در آتش چون خلیل بت‌شکن
&&
جانت را شایستگی یار ده
\\
ساقیا خمخانه را بگشای در
&&
عاشقان را بادهٔ ابرار ده
\\
زاهدان را از وجود خویشتن
&&
وارهان و دردی خمار ده
\\
چند پوشی دلق دام زرق را
&&
دلق پوشان را کنون زنار ده
\\
چون شود شایستهٔ ره جان تو
&&
اهل دل را تحفه چون عطار ده
\\
\end{longtable}
\end{center}
