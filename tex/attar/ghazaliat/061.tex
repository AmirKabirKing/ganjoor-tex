\begin{center}
\section*{غزل شماره ۶۱: ذره‌ای اندوه تو از هر دو عالم خوشتر است}
\label{sec:061}
\addcontentsline{toc}{section}{\nameref{sec:061}}
\begin{longtable}{l p{0.5cm} r}
ذره‌ای اندوه تو از هر دو عالم خوشتر است
&&
هر که گوید نیست دانی کیست آن کس کافر است
\\
کافری شادی است و آن شادی نه از اندوه تو
&&
نی که کار او ز اندوه و ز شادی برتر است
\\
آن کزو غافل بود دیوانه‌ای نامحرم است
&&
وانکه زو فهمی کند دیوانه‌ای صورتگر است
\\
کس سر مویی ندارد از مسما آگهی
&&
اسم می‌گویند و چندان کاسم گویی دیگر است
\\
هرچه در فهم تو آید آن بود مفهوم تو
&&
کی بود مفهوم تو او کو از آن عالی‌تر است
\\
ای عجب بحری است پنهان لیک چندان آشکار
&&
کز نم او ذره ذره تا ابد موج‌آور است
\\
صورتی کان در درون آینه از عکس توست
&&
در درون آینه هر جا که گویی مضمر است
\\
گر تو آن صورت در آئینه ببینی عمرها
&&
زو نیابی ذره‌ای کان در محلی انور است
\\
ای عجب با جملهٔ آهن به هم آن صورت است
&&
گرچه بیرون است ازآن آهن بدان آهن در است
\\
صورتی چون هست با چیزی و بی چیزی به هم
&&
در صفت رهبر چنین گر جان پاکت رهبر است
\\
ور مثالی دیگرت باید به حکم او نگر
&&
صورتش خاک است و برتر سنگ و برتر زان زر است
\\
تا که در دریای دل عطار کلی غرق شد
&&
گوییا تیغ زبانش ابر باران گوهر است
\\
\end{longtable}
\end{center}
