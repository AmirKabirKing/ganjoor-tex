\begin{center}
\section*{غزل شماره ۵۷۳: بی رخت در جهان نظر چکنم}
\label{sec:573}
\addcontentsline{toc}{section}{\nameref{sec:573}}
\begin{longtable}{l p{0.5cm} r}
بی رخت در جهان نظر چکنم
&&
بی لبت عالمی شکر چکنم
\\
رویت ای ترک اگر نخواهم دید
&&
زحمت هندوی بصر چکنم
\\
چون دریغ آیدم رخت به نظر
&&
رخت آلودهٔ نظر چکنم
\\
دو جهان گرچه سخت با خطر است
&&
من خطیری نیم خطر چکنم
\\
چون سر موی تو به از دو جهان
&&
از سر کوی تو گذر چکنم
\\
گر عزیز است عمر مختصر است
&&
من بدین عمر مختصر چکنم
\\
همه عالم جمال و آواز است
&&
چشم کور است و گوش کر چکنم
\\
چون خبر دادن از تو ممکن نیست
&&
من حیران بی خبر چکنم
\\
گرچه جان موج می‌زند از تو
&&
چون زبان نیست کارگر چکنم
\\
چون ز کاهی بسی ضعیف ترم
&&
دست با کوه در کمر چکنم
\\
گر کنم صد هزار قرن سجود
&&
هیچ باشد من این قدر چکنم
\\
گفته بودی که خشک و تر در باز
&&
با لب خشک و چشم تر چکنم
\\
آتش دل به است بی تو مرا
&&
بی تو با آب بر جگر چکنم
\\
گفتیم بال و پر زن از طلبم
&&
چون ز هم ریخت بال و پر چکنم
\\
چون مسافر تویی و من هیچم
&&
من هیچ آخر این سفر چکنم
\\
چون تو جویندهٔ خودی بر من
&&
من سرگشته پا و سر چکنم
\\
چون درونی تو و برون کس نیست
&&
من چو حلقه برون در چکنم
\\
در درون کش مرا و محرم کن
&&
تا تو باشی همه دگر چکنم
\\
محو شد درغم تو فرد فرید
&&
فرد باید مرا حشر چکنم
\\
\end{longtable}
\end{center}
