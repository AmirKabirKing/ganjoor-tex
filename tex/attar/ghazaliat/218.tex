\begin{center}
\section*{غزل شماره ۲۱۸: خطی سبز از زنخدان می بر آورد}
\label{sec:218}
\addcontentsline{toc}{section}{\nameref{sec:218}}
\begin{longtable}{l p{0.5cm} r}
خطی سبز از زنخدان می بر آورد
&&
مرا از دل نه کز جان می بر آورد
\\
خطش خوش خوان از آن آمد که بی کلک
&&
مداد از لعل خندان می بر آورد
\\
مداد اینجا چه باشد لوح سیمش
&&
ز نقره خط خوش‌خوان می بر آورد
\\
کدامین خط خطا رفت آنچه گفتم
&&
مگر خار از گلستان می بر آورد
\\
چنین باغی چه جای خار باشد
&&
که از گلبرگ ریحان می بر آورد
\\
چه می‌گویم که ریحان خادم اوست
&&
که سنبل از نمکدان می برآورد
\\
چه جای سنبل تاریک‌روی است
&&
که سبزه زاب حیوان می برآورد
\\
نبات اینجا چه ذوق آرد ولیکن
&&
زمرد را ز مرجان می بر آورد
\\
ز سبزه هیچ شیرینی نیاید
&&
نبات از شکرستان می بر آورد
\\
چه سنجد در چنین موضع زمرد
&&
که مشک از ماه تابان می بر آورد
\\
که داند تا به سرسبزی خط او
&&
چه شیرینی ز دیوان می بر آورد
\\
به خون در می‌کشد دامن جهانی
&&
چو او سر از گریبان می بر آورد
\\
خدایا داد من بستان ز خطش
&&
که دل از جورش افغان می بر آورد
\\
جهانی خلق را مانند عطار
&&
ز اسلام و ز ایمان می بر آورد
\\
\end{longtable}
\end{center}
