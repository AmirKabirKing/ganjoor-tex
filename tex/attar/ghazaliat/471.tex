\begin{center}
\section*{غزل شماره ۴۷۱: بیشتر عمر چنان بوده‌ام}
\label{sec:471}
\addcontentsline{toc}{section}{\nameref{sec:471}}
\begin{longtable}{l p{0.5cm} r}
بیشتر عمر چنان بوده‌ام
&&
کز نظر خویش نهان بوده‌ام
\\
گه به مناجات به سر گشته‌ام
&&
گه به خرابات دوان بوده‌ام
\\
گاه ز جان سود بسی کرده‌ام
&&
گاه ز تن عین زیان بوده‌ام
\\
راستی آن است که از هیچ وجه
&&
من نه درین و نه در آن بوده‌ام
\\
من چکنم کان که چنان خواستند
&&
گر بد و گر نیک چنان بوده‌ام
\\
گرچه به خورشید مرا علم هست
&&
طالب یک ذره عیان بوده‌ام
\\
نی که خطا رفت چه علم و چه عین
&&
دلشدهٔ سوخته‌جان بوده‌ام
\\
گرچه سبکدل شده‌ام هم ز خود
&&
بر دل خود سخت گران بوده‌ام
\\
بحر جهان بس عجب آمد مرا
&&
غرق تحیر ز جهان بوده‌ام
\\
گرچه ز هر نوع سخن گفته‌ام
&&
کوردلی گنگ زبان بوده‌ام
\\
زآنچه که اصل است چو آگه نیم
&&
پس همه پندار و گمان بوده‌ام
\\
هیچ نمی‌دانم و در عمر خویش
&&
منتظر یک همه دان بوده‌ام
\\
چون همه دانی نتوان زد به تیر
&&
لاجرم از غم چو کمان بوده‌ام
\\
غرقهٔ خون شد ز تحیر فرید
&&
زانکه بسی اشک‌فشان بوده‌ام
\\
\end{longtable}
\end{center}
