\begin{center}
\section*{غزل شماره ۶۰۷: دردا که درین بادیه بسیار دویدیم}
\label{sec:607}
\addcontentsline{toc}{section}{\nameref{sec:607}}
\begin{longtable}{l p{0.5cm} r}
دردا که درین بادیه بسیار دویدیم
&&
در خود برسیدیم و بجایی نرسیدیم
\\
بسیار درین بادیه شوریده برفتیم
&&
بسیار درین واقعه مردانه چخیدیم
\\
گه نعره‌زنان معتکف صومعه بودیم
&&
گه رقص‌کنان گوشهٔ خمار گزیدیم
\\
کردیم همه کار ولی هیچ نکردیم
&&
دیدیم همه چیز ولی هیچ ندیدیم
\\
بر درج دل ماست یکی قفل گران سنگ
&&
در بند ازینیم که در بند کلیدیم
\\
از خون رحم چون به گو خاک فتادیم
&&
از طفل مزاجی همه انگشت مزیدیم
\\
چون شیر ز انگشت براهیم برآمد
&&
انگشت مزیدان چه که انگشت گزیدیم
\\
وامروز که بالغ شدگانیم به صورت
&&
یک پر بنماند ارچه به صد پر بپریدیم
\\
از دست فتادیم نه دیده نه چشیده
&&
زان باده که از جرعهٔ او بوی شنیدیم
\\
چون هستی عطار درین راه حجاب است
&&
از هستی عطار به یکبار بریدیم
\\
\end{longtable}
\end{center}
