\begin{center}
\section*{غزل شماره ۱۷۸: خطش مشک از زنخدان می برآرد}
\label{sec:178}
\addcontentsline{toc}{section}{\nameref{sec:178}}
\begin{longtable}{l p{0.5cm} r}
خطش مشک از زنخدان می برآرد
&&
مرا از دل نه از جان می برآرد
\\
خطش خوانا از آن آمد که بی کلک
&&
مداد از لعل خندان می برآرد
\\
مداد آنجا که باشد لوح سیمینش
&&
ز نقره خط چون جان می برآرد
\\
کدامین خط خطا رفت آنچه گفتم
&&
مگر خار از گلستان می برآرد
\\
چنین جایی چه خای خار باشد
&&
که از گل برگ ریحان می برآرد
\\
چه می‌گویم که ریحان خادم اوست
&&
که سنبل از نمکدان می برآرد
\\
چه جای سنبل تاریک روی است
&&
که سبزه زاب حیوان می برآرد
\\
ز سبزه هیچ شیرینی نیاید
&&
نبات از شکرستان می برآرد
\\
نبات آنجا چه وزن آرد ولیکن
&&
زمرد را ز مرجان می برآرد
\\
چه سنجد در چنین موقع زمرد
&&
که مشک از ماه تابان می برآرد
\\
که داند تا به سرسبزی خط او
&&
چه شیرینی ز دیوان می برآرد
\\
به یک دم کافر زلفش به مویی
&&
دمار از صد مسلمان می برآرد
\\
ز سنگ خاره خون، یعنی که یاقوت
&&
به زخم تیر مژگان می برآرد
\\
میان شهر می‌گردد چو خورشید
&&
خروش از چرخ گردان می برآرد
\\
دلم از عشق رویش زیر بر او
&&
نفس دزدیده پنهان می برآرد
\\
چو می‌ترسد ز چشم بد نفس را
&&
نهان از خویشتن زان می برآرد
\\
فرید از دست او صد قصه هر روز
&&
به پیش چشم سلطان می برآرد
\\
\end{longtable}
\end{center}
