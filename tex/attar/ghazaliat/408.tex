\begin{center}
\section*{غزل شماره ۴۰۸: هر که سر رشتهٔ تو یابد باز}
\label{sec:408}
\addcontentsline{toc}{section}{\nameref{sec:408}}
\begin{longtable}{l p{0.5cm} r}
هر که سر رشتهٔ تو یابد باز
&&
درش از سوزنی کنند فراز
\\
عاشق تو کسی بود که چو شمع
&&
نفسی می‌زند به سوز و گداز
\\
باز خندد چو گل به شکرانه
&&
گر سر او جدا کنند به گاز
\\
آنکه بر جان خویش می‌لرزد
&&
کی تواند چو شمع شد جان‌باز
\\
تا که خوف و رجات می‌ماند
&&
هست نام تو در جریدهٔ ناز
\\
چون نه خوفت بماند و نه رجا
&&
برهی هم ز زنار و هم ز نیاز
\\
هست این راه بی‌نهایت دور
&&
توی بر توی بر مثال پیاز
\\
هر حقیقت که توی اول داشت
&&
در دوم توی هست عین مجاز
\\
ره چنین است و پیش هر قدمی
&&
صد هزاران هزار شیب و فراز
\\
با لبی تشنه و دلی پر خون
&&
خلق کونین مانده در تک و تاز
\\
از فنایی که چارهٔ تو فناست
&&
توشهٔ این ره دراز بساز
\\
تا که باقی است از تو یک سر موی
&&
سر مویی به عشق سر مفراز
\\
گرچه هستی تو مرد پرده‌شناس
&&
نیست از پردهٔ تو این آواز
\\
پرده بر خود مدر که در دو جهان
&&
کس درین پرده نیست پرده‌نواز
\\
گر بسی مایه داری آخر کار
&&
حیرت و عجز را کنی انباز
\\
نیست هر مرغ مرغ این انجیر
&&
نیست هر باز باز این پرواز
\\
مگسی بیش نیستی به وجود
&&
بو که در دامت اوفتد شهباز
\\
یک زمانت فراغت او نیست
&&
باری اول ز خویش واپرداز
\\
در دریای عشق آن کس یافت
&&
که به خون گشت سالهای دراز
\\
تو طمع می‌کنی که بعد از مرگ
&&
برخوری از وصال شمع طراز
\\
هر که در زندگی نیافت ورا
&&
چون بمیرد چگونه یابد باز
\\
زنده چون ره نبرد در همه عمر
&&
مرده چون ره برد به پردهٔ راز
\\
گر به نادر کس این گهر یابد
&&
خویش را گم کند هم از آغاز
\\
پای در نه درین ره ای عطار
&&
سر گردن‌کشان همی انداز
\\
\end{longtable}
\end{center}
