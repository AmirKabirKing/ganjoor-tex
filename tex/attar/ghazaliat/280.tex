\begin{center}
\section*{غزل شماره ۲۸۰: سرمست به بوستان برآمد}
\label{sec:280}
\addcontentsline{toc}{section}{\nameref{sec:280}}
\begin{longtable}{l p{0.5cm} r}
سرمست به بوستان برآمد
&&
از سرو و ز گل فغان برآمد
\\
با حسن نظارهٔ رخش کرد
&&
هر گل که ز بوستان برآمد
\\
نرگس چو بدید چشم مستش
&&
مخمور ز گلستان برآمد
\\
چون لاله فروغ روی او یافت
&&
دلسوخته شد ز جان برآمد
\\
سوسن چو ز بندگی او گفت
&&
آزاده و ده زبان برآمد
\\
بگذشت به کاروان چو یوسف
&&
فریاد ز کاروان برآمد
\\
از شیرینی خندهٔ اوست
&&
هر شور که از جهان برآمد
\\
وز سر تیزی غمزهٔ اوست
&&
هر تیر که از کمان برآمد
\\
کردم شکری طلب ز تنگش
&&
از شرم رخش چنان برآمد
\\
کز روی چو گلستانش گویی
&&
صد دستهٔ ارغوان برآمد
\\
خورشید رخ ستاره ریزش
&&
از کنگرهٔ عیان برآمد
\\
از یک یک ذرهٔ دو عالم
&&
ماهی مه از آسمان برآمد
\\
در خود نگریستم بدان نور
&&
نقشیم به امتحان برآمد
\\
یک موی حجاب در میان بود
&&
چون موی تنم از آن برآمد
\\
در حقه مکن مرا که کارم
&&
زان حقهٔ درفشان برآمد
\\
از هر دو جهان کناره کردم
&&
اندوه تو از میان برآمد
\\
هر مرغ که کرد وصفت آغاز
&&
آواره ز آشیان برآمد
\\
زیرا که به وصفت از دو عالم
&&
آوازهٔ بی نشان برآمد
\\
در وصف تو شد فرید خیره
&&
وز دانش و از بیان برآمد
\\
\end{longtable}
\end{center}
