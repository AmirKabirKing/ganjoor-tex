\begin{center}
\section*{غزل شماره ۲۶: آن نه روی است ماه دو هفته است}
\label{sec:026}
\addcontentsline{toc}{section}{\nameref{sec:026}}
\begin{longtable}{l p{0.5cm} r}
آن نه روی است ماه دو هفته است
&&
وان نه قد است سرو برفته است
\\
پیش ماه دو هفتهٔ رخ تو
&&
ماه و خورشید طفل یک هفته است
\\
ذره‌ای عشق آفتاب رخش
&&
همه دلها به جان پذیرفته است
\\
نرگس اوست ای عجب بیمار
&&
دل عشاق درد بگرفته است
\\
هر کجا صف کشیده مژه او
&&
فتنه بیدار و عافیت خفته است
\\
از دهانش که هست معدومی
&&
نیست عالم تهی پر آشفته است
\\
به دهانش خوش آمد است محال
&&
هر که حرفی از آن دهان گفته است
\\
در دهانش که هست سی و دو در
&&
در پس یک عقیق ناسفته است
\\
می‌نبینی دهانش اگر بینی
&&
کاشکار است آنکه بنهفته است
\\
تا درافشان شد از دهانش فرید
&&
بر سر طاق عالمش جفته است
\\
\end{longtable}
\end{center}
