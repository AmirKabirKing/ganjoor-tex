\begin{center}
\section*{غزل شماره ۳۳۸: گر نسیم یوسفم پیدا شود}
\label{sec:338}
\addcontentsline{toc}{section}{\nameref{sec:338}}
\begin{longtable}{l p{0.5cm} r}
گر نسیم یوسفم پیدا شود
&&
هر که نابینا بود بینا شود
\\
بس که پیراهن بدرم تا مگر
&&
بویی از پیراهنش پیدا شود
\\
گر برافتد برقع از پیش رخش
&&
زاهد منکر سر غوغا شود
\\
ور برافشاند سر زلف دو تا
&&
دل ز زلفش کافری یکتا شود
\\
هر دلی کز زلف او زنار ساخت
&&
بی‌شک آن دلمؤمنی حقا شود
\\
گر بیابد عقل بوی زلف او
&&
عقل از لایعقلی رسوا شود
\\
از دو عالم فارغ آید تا ابد
&&
هر که او مشغول این سودا شود
\\
گر کسی پرسد که پیش روی او
&&
دل چرا شوریده و شیدا شود
\\
تو جوابش ده که پیش آفتاب
&&
ذره سرگردان و ناپروا شود
\\
ای دل از دریا چرا تنها شدی
&&
از چنین دریا کسی تنها شود
\\
هر که دور افتد ز جای خویشتن
&&
می‌دود تا زودتر آنجا شود
\\
ماهی از دریا چو بر خاک اوفتد
&&
می‌تپد تا چون سوی دریا شود
\\
گر تو بنشینی به بیکاری مدام
&&
کارت ای غافل کجا زیبا شود
\\
گر دل عطار با دریا رسد
&&
گوهری بی‌مثل و بی‌همتا شود
\\
\end{longtable}
\end{center}
