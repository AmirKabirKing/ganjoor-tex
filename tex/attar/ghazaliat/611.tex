\begin{center}
\section*{غزل شماره ۶۱۱: چون زلف تاب دهد آن ترک لشکریم}
\label{sec:611}
\addcontentsline{toc}{section}{\nameref{sec:611}}
\begin{longtable}{l p{0.5cm} r}
چون زلف تاب دهد آن ترک لشکریم
&&
هندوی خویش کند هر دم به دلبریم
\\
چون زلف کافر او آهنگ دین کندم
&&
در حال بند کند در دام کافریم
\\
مویی اگر همه خلق در من نگه نکنند
&&
مویی تمام بود زان زلف عنبریم
\\
ای ساقی از می عشق دلقم بشو و بیا
&&
چون دلق زرق من است چند از سیه گریم
\\
تا کی ز رد و قبول دردی بیار که من
&&
مست ملامتیم رند قلندریم
\\
تا کی ز روی و ریا بت ساختن ز هوا
&&
زین پس به بتکده‌ها مرد مقامریم
\\
گر دی به صومعه در، مرد خلیل بدم
&&
امروز پیش مغان چون گبر آزریم
\\
گرچه به صورت تن، از مؤمنان رهم
&&
لیکن ز روی یقین گبرم چو بنگریم
\\
عطار تا که نهاد در راه فقر قدم
&&
کرد آن حقیقت فقر از جان و دل بریم
\\
\end{longtable}
\end{center}
