\begin{center}
\section*{غزل شماره ۶۵۹: بیم است که صد آه برآرم ز جگر من}
\label{sec:659}
\addcontentsline{toc}{section}{\nameref{sec:659}}
\begin{longtable}{l p{0.5cm} r}
بیم است که صد آه برآرم ز جگر من
&&
تا بی تو چرا می‌برم این عمر به سر من
\\
آگاه از آنم که به جز تو دگری نیست
&&
و آگاه نیم از بد و از نیک دگر من
\\
عمری ره تو جستم و چون راه ندیدم
&&
کم آمدم آنجا ز سگ راهگذر من
\\
دل سوخته زانم که کنون از سرخامی
&&
کردم همه کردار نکو زیر و زبر من
\\
در کوی خرابات و خرافات فتادم
&&
وآنگاه بشستم به میی دامن‌تر من
\\
پر کردم از اندوه به یک کوزهٔ دردی
&&
هر لحظه کناری ز خم خون‌جگر من
\\
وامروز درین حادثه دانی به چه مانم
&&
در نزع فرومانده چون شمع سحر من
\\
مردان چو نگین مانده در حلقهٔ معنی
&&
وز حلقه به درمانده چو حلقه به در من
\\
ای دوست به عطار نظر کن که ندارم
&&
جز بی خبری از ره تو هیچ خبر من
\\
\end{longtable}
\end{center}
