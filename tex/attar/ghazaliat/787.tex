\begin{center}
\section*{غزل شماره ۷۸۷: هر دمم در امتحان چندی کشی}
\label{sec:787}
\addcontentsline{toc}{section}{\nameref{sec:787}}
\begin{longtable}{l p{0.5cm} r}
هر دمم در امتحان چندی کشی
&&
دامنم در خون جان چندی کشی
\\
مهربان خویشتن گفتم تو را
&&
کینهٔ آن هر زمان چندی کشی
\\
همچو خاکم بر زمین افتاده خوار
&&
سر ز من بر آسمان چندی کشی
\\
چون جهان سر بر خطت دارد مدام
&&
چون قلم خط بر جهان چندی کشی
\\
در غمت چون با کناری رفته‌ام
&&
تو به زورم در میان چندی کشی
\\
بر تو دارم چشم از روی جهان
&&
بر من از مژگان سنان چندی کشی
\\
همچو شمعی سر نهادم در میان
&&
بر سرم تیغ از میان چندی کشی
\\
پیشکش می‌سازمت گلگون اشک
&&
رخش کبرت را عنان چندی کشی
\\
چون سپر بفکندم و بگریختم
&&
تو به کین من کمان چندی کشی
\\
کینت از صد مهر خوشتر آیدم
&&
کین ز چون من مهربان چندی کشی
\\
بر سرم آمد لاشهٔ صبرم ز عجز
&&
تنگ اسب امتحان چندی کشی
\\
بس سبک دل گشتی از عشق ای فرید
&&
جان بده بار گران چندی کشی
\\
\end{longtable}
\end{center}
