\begin{center}
\section*{غزل شماره ۱۳۵: هر دل که ز عشق بی نشان رفت}
\label{sec:135}
\addcontentsline{toc}{section}{\nameref{sec:135}}
\begin{longtable}{l p{0.5cm} r}
هر دل که ز عشق بی نشان رفت
&&
در پردهٔ نیستی نهان رفت
\\
از هستی خویش پاک بگریز
&&
کین راه به نیستی توان رفت
\\
تا تو نکنی ز خود کرانه
&&
کی بتوانی ازین میان رفت
\\
صد گنج میان جان کسی یافت
&&
کین بادیه از میان جان رفت
\\
راهی که به عمرها توان رفت
&&
مرد ره او به یک زمان رفت
\\
هان ای دل خفته عمر بگذشت
&&
تا کی خسبی که کاروان رفت
\\
ای جان و جهان چه می‌نشینی
&&
برخیز که جان شد و جهان رفت
\\
از جملهٔ نیستان این راه
&&
آن برد سبق که بی نشان رفت
\\
چون نیستی از زمین توان برد
&&
کی هست توان بر آسمان رفت
\\
محتاج به دانهٔ زمین بود
&&
مرغی که ز شاخ لامکان رفت
\\
عطار چو ذوق نیستی یافت
&&
از هستی خویش بر کران رفت
\\
\end{longtable}
\end{center}
