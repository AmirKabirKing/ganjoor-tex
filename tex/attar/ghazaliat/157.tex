\begin{center}
\section*{غزل شماره ۱۵۷: دلم قوت کار می‌برنتابد}
\label{sec:157}
\addcontentsline{toc}{section}{\nameref{sec:157}}
\begin{longtable}{l p{0.5cm} r}
دلم قوت کار می‌برنتابد
&&
تنم این همه بار می‌برنتابد
\\
دل من ز انبارها غم چنان شد
&&
که این بار آن بار می‌برنتابد
\\
چگونه کشد نفس کافر غم تو
&&
چو دانم که دین‌دار می‌برنتابد
\\
پس پردهٔ پندار می‌سوزم اکنون
&&
که این پرده پندار می‌برنتابد
\\
دل چون گلم را منه خار چندین
&&
گلی این همه خار می‌برنتابد
\\
چنان شد دل من که بار فراقت
&&
نه اندک نه بسیار می‌برنتابد
\\
چنان زار می‌بینمش دور از تو
&&
که یک نالهٔ زار می‌برنتابد
\\
سزد گر نهی مرهمی از وصالش
&&
که زین بیش تیمار می‌برنتابد
\\
جهانی است عشقت چنان پر عجایب
&&
که تسبیح و زنار می‌برنتابد
\\
نه در کفر می‌آید و نه در ایمان
&&
که اقرار و انکار می‌برنتابد
\\
دلم مست اسرار عشقت چنان شد
&&
که بویی ز اسرار می‌برنتابد
\\
مرا دیده‌ای بخش دیدار خود را
&&
که این دیده دیدار می‌برنتابد
\\
چگونه جمال تو را چشم دارم
&&
که این چشم اغیار می‌برنتابد
\\
گرفتاری عشق سودای رویت
&&
دلی جز گرفتار می‌برنتابد
\\
خلاصی ده از من مرا این چه عار است
&&
که عطار این عار می‌برنتابد
\\
\end{longtable}
\end{center}
