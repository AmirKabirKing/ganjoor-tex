\begin{center}
\section*{غزل شماره ۴۲۰: گر مرد رهی ز رهروان باش}
\label{sec:420}
\addcontentsline{toc}{section}{\nameref{sec:420}}
\begin{longtable}{l p{0.5cm} r}
گر مرد رهی ز رهروان باش
&&
در پردهٔ سر خون نهان باش
\\
بنگر که چگونه ره سپردند
&&
گر مرد رهی تو آن چنان باش
\\
خواهی که وصال دوست یابی
&&
با دیده درآی و بی زبان باش
\\
از بند نصیب خویش برخیز
&&
دربند نصیب دیگران باش
\\
در کوی قلندری چو سیمرغ
&&
می‌باش به نام و بی نشان باش
\\
بگذر تو ازین جهان فانی
&&
زنده به حیات جاودان باش
\\
در یک قدم این جهان و آن نیز
&&
بگذار جهان و در جهان باش
\\
منگر تو به دیدهٔ تصرف
&&
بیرون ز دو کون این و آن باش
\\
عطار ز مدعی بپرهیز
&&
رو گوشه‌نشین و در میان باش
\\
\end{longtable}
\end{center}
