\begin{center}
\section*{غزل شماره ۲۳۳: گرچه ز تو هر روزم صد فتنه دگر خیزد}
\label{sec:233}
\addcontentsline{toc}{section}{\nameref{sec:233}}
\begin{longtable}{l p{0.5cm} r}
گرچه ز تو هر روزم صد فتنه دگر خیزد
&&
در عشق تو هر ساعت دل شیفته‌تر خیزد
\\
لعلت که شکر دارد حقا که یقینم من
&&
گر در همه خوزستان زین شیوه شکر خیزد
\\
هرگه که چو چوگانی زلف تو به پای افتد
&&
دل در خم زلف تو چون گوی به سر خیزد
\\
گفتی به بر سیمین زر از تو برانگیزم
&&
آخر ز چو من مفلس دانی که چه زر خیزد
\\
قلبی است مرا در بر رویی است مرا چون زر
&&
این قلب که برگیرد زان وجه چه برخیزد
\\
تا در تو نظر کردم رسوای جهان گشتم
&&
آری همه رسوایی اول ز نظر خیزد
\\
گفتی چو منی بگزین تا من برهم از تو
&&
آری چو تو بگزینم، گر چون تو دگر خیزد
\\
بیچاره دلم بی کس کز شوق رخت هر شب
&&
بر خاک درت افتد در خون جگر خیزد
\\
چو خاک توام آخر خونم به چه می‌ریزی
&&
از خون چو من خاکی چه خیزد اگر خیزد
\\
عطار اگر روزی رخ تازه بود بی تو
&&
آن تازگی رویش از دیدهٔ‌تر خیزد
\\
\end{longtable}
\end{center}
