\begin{center}
\section*{غزل شماره ۷۵۸: ای آفتاب از ورق رویت آیتی}
\label{sec:758}
\addcontentsline{toc}{section}{\nameref{sec:758}}
\begin{longtable}{l p{0.5cm} r}
ای آفتاب از ورق رویت آیتی
&&
در جنب جام لعل تو کوثر حکایتی
\\
هرگز ندید هیچ کس از مصحف جمال
&&
سرسبزتر ز خط سیاه تو آیتی
\\
بر نیت خطت که دلم جای وقف دید
&&
کرد از حروف زلف تو عالی روایتی
\\
از مشک خط خود جگرم سوختی ولیک
&&
دل ندهدم که در قلم آرم شکایتی
\\
آب حیات در ظلمات ظلالت است
&&
تا کی ز عکس لعل تو یابد هدایتی
\\
خورشید را که سلطنت سخت روشن است
&&
بنگر گرفت سایهٔ زلفت حمایتی
\\
هر دم ز زلف تو شکنی دیگرم رسد
&&
زان پی نمی‌برم شکنش را نهایتی
\\
چون زلف تو به تاب درم تا کیم رسد
&&
از زلف عنبر تو نسیم عنایتی
\\
زلف توراست از در دربند تا ختن
&&
زان دل فرو گرفت زهی خوش ولایتی
\\
عطار تا که بود، نبودش به هیچ روی
&&
جز دوستی روی تو هرگز جنایتی
\\
\end{longtable}
\end{center}
