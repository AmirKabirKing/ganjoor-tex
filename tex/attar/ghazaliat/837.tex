\begin{center}
\section*{غزل شماره ۸۳۷: جان به لب آورده‌ام تا از لبم جانی دهی}
\label{sec:837}
\addcontentsline{toc}{section}{\nameref{sec:837}}
\begin{longtable}{l p{0.5cm} r}
جان به لب آورده‌ام تا از لبم جانی دهی
&&
دل ز من بربوده‌ای باشد که تاوانی دهی
\\
از لبت جانی همی خواهم برای خویش نه
&&
زانکه هم بر تو فشانم گر مرا جانی دهی
\\
تو همی خواهی که هر تابی اندر زلف توست
&&
همچو زلف خویش در کار پریشانی دهی
\\
من چو گویی پا و سر گم کرده‌ام تا تو مرا
&&
زلف بفشانی و از هر حلقه چوگانی دهی
\\
من کیم مهمان تو، تو تنگ‌ها داری شکر
&&
می‌سزد گر یک شکر آخر به مهمانی دهی
\\
من سگ کوی توام شیری شوم گر گاه گاه
&&
چون سگان کوی خویشم ریزهٔ خوانی دهی
\\
چون نمی‌یابند از وصل تو شاهان ذره‌ای
&&
نیست ممکن گر چنان ملکی به دربانی دهی
\\
من که باشم تا به خون من بیالایی تو دست
&&
این به دست من برآید گر تو فرمانی دهی
\\
کی رسم در گرد وصل تو که تا می‌بنگرم
&&
هر دمم تشنه جگر سر در بیابانی دهی
\\
داد از بیداد تو عطار مسکین دل ز دست
&&
دست آن داری که تو داد سخن دانی دهی
\\
\end{longtable}
\end{center}
