\begin{center}
\section*{غزل شماره ۷۰۹: ای چشم بد را برقعی بر روی ماه آویخته}
\label{sec:709}
\addcontentsline{toc}{section}{\nameref{sec:709}}
\begin{longtable}{l p{0.5cm} r}
ای چشم بد را برقعی بر روی ماه آویخته
&&
صد یوسف گم گشته را زلفت به چاه آویخته
\\
ماه است روی خرمت دام است زلف پر خمت
&&
دلها چو مرغ اندر غمت از دامگاه آویخته
\\
فرش بقا انداخته کوس فنا بنواخته
&&
میزان عزت ساخته پیش سپاه آویخته
\\
مردان ره را بارها بر لب زده مسمارها
&&
پس جمله را بر دارها از چار راه آویخته
\\
شمع طرب افروخته تا راز شمع آموخته
&&
دل بی جنایت سوخته جان بی گناه آویخته
\\
ای داده در دلها ندا، تا کرده دلها جان فدا
&&
سرهای پیران هدی بر شاهراه آویخته
\\
آن خواجهٔ روز جزا، بر چارسوی کبریا
&&
از بهر دست آویز ما زلف سیاه آویخته
\\
ابلیس را حالی عجب در بحر حرمان خشک لب
&&
از بهر یک ترک ادب از سجدگاه آویخته
\\
عطار این تفصیل‌دان وین قصه بی تأویل‌دان
&&
عالم یکی قندیل دان، ز ایوان شاه آویخته
\\
\end{longtable}
\end{center}
