\begin{center}
\section*{غزل شماره ۷۲: چشم خوشش مست نیست لیک چو مستان خوش است}
\label{sec:072}
\addcontentsline{toc}{section}{\nameref{sec:072}}
\begin{longtable}{l p{0.5cm} r}
چشم خوشش مست نیست لیک چو مستان خوش است
&&
خوشی چشمش از آنست کین همه دستان خوش است
\\
نرگس دستان گرش دست دل از حیله برد
&&
هرچه کند چشم او ور ببرد جان خوش است
\\
زلف پریشانش را حلقه به گوشم از آنک
&&
بر رخ چون ماه او زلف پریشان خوش است
\\
خندهٔ شیرین او گریهٔ من تلخ کرد
&&
گریهٔ خونین من زان لب خندان خوش است
\\
پستهٔ شیرین او شور دل عاشقانش
&&
شور دل عاشقانش زین شکرستان خوش است
\\
چون سخنش را گذر بر لب شیرین اوست
&&
آن سخن تلخ او همچو شکر زان خوش است
\\
عقل لبش را مرید از بن دندان شده است
&&
نیست درین هیچ شک کان لب و دندان خوش است
\\
سبزهٔ خطش دمید بر لب آب حیات
&&
با خط سرسبز او چشمهٔ حیوان خوش است
\\
بحر صفت شد به نطق خاطر عطار ازو
&&
در صفت حسن او بحر درافشان خوش است
\\
\end{longtable}
\end{center}
