\begin{center}
\section*{غزل شماره ۱۳۲: پیشگاه عشق را پیشان که یافت}
\label{sec:132}
\addcontentsline{toc}{section}{\nameref{sec:132}}
\begin{longtable}{l p{0.5cm} r}
پیشگاه عشق را پیشان که یافت
&&
پایگاه فقر را پایان که یافت
\\
در میان این دو ششدر کل خلق
&&
جمله مردند و اثر زیشان که یافت
\\
رخنه می‌جویی خلاص خویشتن
&&
رخنه‌ای جز مرگ ازین زندان که یافت
\\
ذره‌ای وصلش چو کس طاقت نداشت
&&
قسم موجودات جز هجران که یافت
\\
ذره‌ای این درد عالم سوز را
&&
در زمین و آسمان درمان که یافت
\\
آفتاب آسمان غیب را
&&
در فروغش کفر با ایمان که یافت
\\
چون بتافت آن آفتاب آواز داد
&&
کان هزاران ذره سرگردان که یافت
\\
ابر بر دریا بسی بگریست زار
&&
لیک دریا گشت و آن باران که یافت
\\
گشت مستهلک درین دریا دو کون
&&
گر کفی گل بود و ور طوفان که یافت
\\
چون دو عالم هست فرزند عدم
&&
پس وجودی بی سر و سامان که یافت
\\
چون دو عالم نیست جز یک آفتاب
&&
ذره‌ای در سایه‌ای پنهان که یافت
\\
چون همه مردند و می‌میرند نیز
&&
آب حیوان زین همه حیوان که یافت
\\
بر فلک رو این دم از عیسی بپرس
&&
تا خری رهوار بی پالان که یافت
\\
صد هزاران چشم صدیقان راه
&&
گشت خون‌باران همه، باران که یافت
\\
صد هزاران جان صدیقان راه
&&
غرقهٔ این راه شد جانان که یافت
\\
ای فرید از فرش تا عرش مجید
&&
ذره‌ای هستی درین دیوان که یافت
\\
\end{longtable}
\end{center}
