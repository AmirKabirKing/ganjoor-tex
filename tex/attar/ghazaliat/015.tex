\begin{center}
\section*{غزل شماره ۱۵: سحرگاهی شدم سوی خرابات}
\label{sec:015}
\addcontentsline{toc}{section}{\nameref{sec:015}}
\begin{longtable}{l p{0.5cm} r}
سحرگاهی شدم سوی خرابات
&&
که رندان را کنم دعوت به طامات
\\
عصا اندر کف و سجاده بر دوش
&&
که هستم زاهدی صاحب کرامات
\\
خراباتی مرا گفتا که ای شیخ
&&
بگو تا خود چه کار است از مهمات
\\
بدو گفتم که کارم توبهٔ توست
&&
اگر توبه کنی یابی مراعات
\\
مرا گفتا برو ای زاهد خشک
&&
که تر گردی ز دردی خرابات
\\
اگر یک قطره دردی بر تو ریزم
&&
ز مسجد بازمانی وز مناجات
\\
برو مفروش زهد و خودنمائی
&&
که نه زهدت خرند اینجا نه طامات
\\
کسی را اوفتد بر روی، این رنگ
&&
که در کعبه کند بت را مراعات
\\
بگفت این و یکی دردی به من داد
&&
خرف شد عقلم و رست از خرافات
\\
چو من فانی شدم از جان کهنه
&&
مرا افتاد با جانان ملاقات
\\
چو از فرعون هستی باز رستم
&&
چو موسی می‌شدم هر دم به میقات
\\
چو خود را یافتم بالای کونین
&&
چو دیدم خویشتن را آن مقامات
\\
برآمد آفتابی از وجودم
&&
درون من برون شد از سماوات
\\
بدو گفتم که ای دانندهٔ راز
&&
بگو تا کی رسم در قرب آن ذات
\\
مرا گفتا که ای مغرور غافل
&&
رسد هرگز کسی هیهات هیهات
\\
بسی بازی ببینی از پس و پیش
&&
ولی آخر فرومانی به شهمات
\\
همه ذرات عالم مست عشقند
&&
فرومانده میان نفی و اثبات
\\
در آن موضع که تابد نور خورشید
&&
نه موجود و نه معدوم است ذرات
\\
چه می‌گویی تو ای عطار آخر
&&
که داند این رموز و این اشارات
\\
\end{longtable}
\end{center}
