\begin{center}
\section*{غزل شماره ۲۴۰: در صفت عشق تو شرح و بیان نمی‌رسد}
\label{sec:240}
\addcontentsline{toc}{section}{\nameref{sec:240}}
\begin{longtable}{l p{0.5cm} r}
در صفت عشق تو شرح و بیان نمی‌رسد
&&
عشق تو خود عالی است عقل در آن نمی‌رسد
\\
آنچه که از عشق تو معتکف جان ماست
&&
گرچه بگویم بسی سوی زبان نمی‌رسد
\\
جان چو ز میدان عشق گوی وصال تو برد
&&
تاختنی دو کون در پی جان نمی‌رسد
\\
گرچه نشانه بسی است لیک دراز است راه
&&
سوی تو بی نور تو کس به نشان نمی‌رسد
\\
عاشق دل خسته را تا نرسد هرچه هست
&&
در اثر درد تو هر دو جهان نمی‌رسد
\\
بادیهٔ عشق تو بادیه‌ای است بی‌کران
&&
پس به چنین بادیه کس به کران نمی‌رسد
\\
سوی تو عطار را موی‌کشان برد عشق
&&
بی خبری سوی تو موی کشان نمی‌رسد
\\
\end{longtable}
\end{center}
