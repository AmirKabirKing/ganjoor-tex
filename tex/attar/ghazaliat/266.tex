\begin{center}
\section*{غزل شماره ۲۶۶: عشقت ایمان و جان به ما بخشد}
\label{sec:266}
\addcontentsline{toc}{section}{\nameref{sec:266}}
\begin{longtable}{l p{0.5cm} r}
عشقت ایمان و جان به ما بخشد
&&
لیک بی‌علتی عطا بخشد
\\
نیست علت که ملک صد سلطان
&&
در زمانی به یک گدا بخشد
\\
گر همه طاعتی به جای آری
&&
هر یکی را صدت جزا بخشد
\\
لیک گنجی که قسم عشاق است
&&
عشق بی چون و بی چرا بخشد
\\
نیست کس را خبر که پرتو عشق
&&
به کجا آید و کجا بخشد
\\
ذره‌ای گر ز پرده در تابد
&&
شرق تا غرب کیمیا بخشد
\\
گر بقا بیندت فنا کندت
&&
ور فنا بایدت بقا بخشد
\\
هر نفس صد هزار خاک شوند
&&
تا چنین دولتی کرا بخشد
\\
چون ببازی تو جمله تو بر تو
&&
گر تو بی تو شوی تو را بخشد
\\
گر تو را چشم راه بین است بران
&&
راه چشم تو را ضیا بخشد
\\
وگرت چشم تیرگی دارد
&&
راهت از گرد توتیا بخشد
\\
همچو نی شو تهی ز دعوی و لاف
&&
تا دمت روح را صفا بخشد
\\
گر بسوزی ز شعله نور دهد
&&
ور بسازی بسی نوا بخشد
\\
گر درین ره فرید کشته شود
&&
اولین گام خونبها بخشد
\\
\end{longtable}
\end{center}
