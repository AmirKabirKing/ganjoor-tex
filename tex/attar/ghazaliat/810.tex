\begin{center}
\section*{غزل شماره ۸۱۰: ز سگان کویت ای جان که دهد مرا نشانی}
\label{sec:810}
\addcontentsline{toc}{section}{\nameref{sec:810}}
\begin{longtable}{l p{0.5cm} r}
ز سگان کویت ای جان که دهد مرا نشانی
&&
که ندیدم از تو بوی و گذشت زندگانی
\\
دل من نشان کویت ز جهان بجست عمری
&&
که خبر نبود دل را که تو در میان جانی
\\
ز غمت چو مرغ بسمل شب و روز می‌طپیدم
&&
چو به لب رسید جانم پس ازین دگر تو دانی
\\
به عتاب گفته بودی که برآتشت نشانم
&&
چو مرا بسوخت عشقت چه بر آتشم نشانی
\\
همه بندها گشادی به طریق دلفریبی
&&
همه دست‌ها ببستی به کمال دلستانی
\\
تو چه گنجی آخر ای جان که به کون در نگنجی
&&
تو چه گوهری که در دل شده‌ای بدین نهانی
\\
دو جهان پر از گهر شد ز فروغ تو ولیکن
&&
به تو کی توان رسیدن که تو گنج بی کرانی
\\
همه عاشقان عالم همه مفلسان عاشق
&&
ز تو مانده‌اند حیران که به هیچ می نمانی
\\
چو به سر کشی در آیی همه سروران دین را
&&
ز سر نیازمندی چو قلم به سر دوانی
\\
دل تشنگان عاشق ز غم تو سوخت در بر
&&
چه شود اگر شرابی بر تشنگان رسانی
\\
اگر از پی تو عطار اثر وصال یابد
&&
دو جهان به سر برآرد ز جواهر معانی
\\
\end{longtable}
\end{center}
