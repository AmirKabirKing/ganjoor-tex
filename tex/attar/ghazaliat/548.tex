\begin{center}
\section*{غزل شماره ۵۴۸: ای جان و جهان رویت پیدا نکنی دانم}
\label{sec:548}
\addcontentsline{toc}{section}{\nameref{sec:548}}
\begin{longtable}{l p{0.5cm} r}
ای جان و جهان رویت پیدا نکنی دانم
&&
تا جان و جهانی را شیدا نکنی دانم
\\
پشت من یکتا دل از زلف دوتا کردی
&&
و آن زلف دوتا هرگز یکتا نکنی دانم
\\
گر جور کنی ور نی تا کار تو می‌ماند
&&
زین شیوه بسی افتد عمدا نکنی دانم
\\
در غارت جان و دل در زلف و لبت بازی
&&
زیرا که چنین کاری تنها نکنی دانم
\\
چون عاشق غم‌کش را در خاک کنی پنهان
&&
بر خویش نظر آری پیدا نکنی دانم
\\
گفتی کنم از بوسی روزی دهنت شیرین
&&
این خود به زبان گویی اما نکنی دانم
\\
اندر عوض بوسی گر جان و تنم بردی
&&
تا عاشق سودایی رسوا نکنی دانم
\\
گفتی که شبی با تو دستی کنم اندر کش
&&
یارب چه دروغ است این با ما نکنی دانم
\\
گفتی که جفا کردم در حق تو ای عطار
&&
آخر همه کس داند کانها نکنی دانم
\\
\end{longtable}
\end{center}
