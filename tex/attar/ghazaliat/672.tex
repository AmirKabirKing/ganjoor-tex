\begin{center}
\section*{غزل شماره ۶۷۲: هر که جان درباخت بر دیدار او}
\label{sec:672}
\addcontentsline{toc}{section}{\nameref{sec:672}}
\begin{longtable}{l p{0.5cm} r}
هر که جان درباخت بر دیدار او
&&
صد هزاران جان شود ایثار او
\\
تا توانی در فنای خویش کوش
&&
تا شوی از خویش برخودار او
\\
چشم مشتاقان روی دوست را
&&
نسیه نبود پرتو رخسار او
\\
نقد باشد اهل دل را روز و شب
&&
در مقام معرفت دیدار او
\\
دوست یک دم نیست خاموش از سخن
&&
گوش کو تا بشنود گفتار او
\\
پنبه را از گوش بر باید کشید
&&
بو که یکدم بشنوی اسرار او
\\
نور و نار او بهشت و دوزخ است
&&
پای برتر نه ز نور و نار او
\\
دوزخ مردان بهشت دیگران است
&&
درگذر زین هر دو در زنهار او
\\
کز امید وصل و از بیم فراق
&&
جان مردان خون شد اندر کار او
\\
عاشقان خسته دل بین صد هزار
&&
سرنگون آویخته از دار او
\\
همچو مرغ نیم بسمل مانده‌اند
&&
بیخود و سرگشته از تیمار او
\\
صد هزاران رفته‌اند و کس ندید
&&
تا که دید از رفتگان آثار او
\\
زاد عطار اندرین ره هیچ نیست
&&
جز امید رحمت بسیار او
\\
\end{longtable}
\end{center}
