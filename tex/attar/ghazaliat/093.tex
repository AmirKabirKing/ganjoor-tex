\begin{center}
\section*{غزل شماره ۹۳: بت ترسای من مست شبانه است}
\label{sec:093}
\addcontentsline{toc}{section}{\nameref{sec:093}}
\begin{longtable}{l p{0.5cm} r}
بت ترسای من مست شبانه است
&&
چه شور است این کزان بت در زمانه است
\\
سر زلفش نگر کاندر دو عالم
&&
ز هر موییش جویی خون روانه است
\\
دل من صاف دین در راه او باخت
&&
که این دل مست دردی مغانه است
\\
چو عقلم مات شد بر نطع عشقش
&&
چه بازم چون نه بازی و نه خانه است
\\
دل بیمار را در عشق آن بت
&&
شفا از نعره‌های عاشقانه است
\\
درآمد دوش و گفت ای غرهٔ خود
&&
دلت غمگین و نفست شادمانه است
\\
به بوی دانه مرغت مانده در دام
&&
چه مرغی آنکه عرشش آشیانه است
\\
بدو گفتند چون در دام ماندی
&&
بخور دانه که غم خوردن فسانه است
\\
به زاری مرغ گفتا ای عزیزان
&&
به دام اندر که را پروای دانه است
\\
کز آنگاهی که خورد آن دانه آدم
&&
به دام افتاده سر بر آستانه است
\\
عزیزا کار تو بس مشکل افتاد
&&
چه گویم چون زبانم پر زبانه است
\\
ببین کایینهٔ کونین عالم
&&
جمال بی نشانی را نشانه است
\\
نگاهی می‌کند در آینه یار
&&
که او خود عاشق خود جاودانه است
\\
به خود می‌بازد از خود عشق با خود
&&
خیال آب و گل در ره بهانه است
\\
اگر احول نباشی زود ببینی
&&
که کلی هر دو عالم یک یگانه است
\\
تو هرجایی از آن می بازمانی
&&
که راهی دور و بحری بی‌کرانه است
\\
بر آن ایوان کز اینجا رفت این حرف
&&
دو عالم همچو نقش آسمان است
\\
دل عطار از روز ازل باز
&&
ز صاف عشق مخمور شبانه است
\\
\end{longtable}
\end{center}
