\begin{center}
\section*{غزل شماره ۶۲: مرکب لنگ است و راه دور است}
\label{sec:062}
\addcontentsline{toc}{section}{\nameref{sec:062}}
\begin{longtable}{l p{0.5cm} r}
مرکب لنگ است و راه دور است
&&
دل را چکنم که ناصبور است
\\
این راه پریدنم خیال است
&&
وین شیوه گرفتنم غرور است
\\
صد قرن چو باد اگر بپویم
&&
هم باد بود که یار دور است
\\
با این همه گر دمی برآرم
&&
بی او همه فسق یا فجور است
\\
دانی تو که سر کافری چیست
&&
آن دم که همی نه در حضور است
\\
بی او نفسی مزن که ناگاه
&&
تیغت زند او که بس غیور است
\\
بگذر ز رجا و خوف کین‌جا
&&
چه جای خیال نار و نور است
\\
جایی است که صد جهان اگر نیست
&&
ور هست نه ماتم و نه سور است
\\
مردی که بدین صفت رسیده است
&&
دایم هم ازین صفت نفور است
\\
همچون دریا بود که پیوست
&&
لب خشک بماند از قصور است
\\
این حرف ز بی نهایتی رفت
&&
چون زین بگذشت زرق و زور است
\\
یک ذره‌گی فرید اینجا
&&
بالای هزار خلد و حور است
\\
\end{longtable}
\end{center}
