\begin{center}
\section*{غزل شماره ۵۴۱: بی لبت از آب حیوان می‌بسم}
\label{sec:541}
\addcontentsline{toc}{section}{\nameref{sec:541}}
\begin{longtable}{l p{0.5cm} r}
بی لبت از آب حیوان می‌بسم
&&
بی رخت از ماه تابان می‌بسم
\\
کار روی حسن تو گردان بس است
&&
ز آفتاب چرخ گردان می‌بسم
\\
سر گرانم من ز چین زلف تو
&&
از همه چین مشک ارزان می‌بسم
\\
گر ندارم آبرویی پیش تو
&&
آب روی از چشم گریان می‌بسم
\\
تا لب لعل تو در چشم من است
&&
تا ابد از بحر و از کان می‌بسم
\\
از همه ملک دو عالم یک نفس
&&
با تو گر دستم دهد آن می‌بسم
\\
گفته‌ای زارت بخواهم سوختن
&&
آتش شوق تو در جان می‌بسم
\\
زآتش دیگر چه می‌سوزی مرا
&&
چون یک آتش هست سوزان می‌بسم
\\
ساقیا در ده شرابی آشکار
&&
کز دلی پر کفر پنهان می‌بسم
\\
زین همه زنار از تشویر خلق
&&
کرده پنهان زیر خلقان می‌بسم
\\
درد ده تا درد بفزاید مرا
&&
زانکه با دردت ز درمان می‌بسم
\\
غرق دریا گر مرا کرده است نفس
&&
تشنه می‌میرم بیابان می‌بسم
\\
مست لایعقل کن این ساعت مرا
&&
کز دم عقل سخن دان می‌بسم
\\
عقل خود را مصلحت جوید مدام
&&
زین چنین عقل تن آسان می‌بسم
\\
کارساز است او ز پیش و پس ولی
&&
هم ز پایان هم ز پیشان می‌بسم
\\
عقل را بگذار اگر اهل دلی
&&
زانکه چون دل هست از جان می‌بسم
\\
نقد ابن الوقت قلب است ای فرید
&&
دل طلب کز عقل حیران می‌بسم
\\
\end{longtable}
\end{center}
