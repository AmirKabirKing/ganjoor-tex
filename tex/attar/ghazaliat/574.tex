\begin{center}
\section*{غزل شماره ۵۷۴: چاره نیست از توام چه چاره کنم}
\label{sec:574}
\addcontentsline{toc}{section}{\nameref{sec:574}}
\begin{longtable}{l p{0.5cm} r}
چاره نیست از توام چه چاره کنم
&&
تا به تو از همه کناره کنم
\\
چکنم تا همه یکی بینم
&&
به یکی در همه نظاره کنم
\\
آنچه زو هیچ ذره پنهان نیست
&&
همچو خورشید آشکاره کنم
\\
ذره‌آی چون هزار عالم هست
&&
پرده بر ذره ذره پاره کنم
\\
تا که هر ذره را چو خورشیدی
&&
بر براق فلک سواره کنم
\\
صد هزاران هزار عالم را
&&
پیش روی تو پیشکاره کنم
\\
پس به یک یک نفس هزار جهان
&&
تحفهٔ چون تو ماه پاره کنم
\\
چون کنم قصد این سلوک شگرف
&&
کوکب کفش از ستاره کنم
\\
شیر دوشم هزار دریا بیش
&&
لیک پستان ز سنگ خاره کنم
\\
ذره‌های دو کون را زان شیر
&&
همچو اطفال شیرخواره کنم
\\
چون کمال بلوغ ممکن نیست
&&
چکنم گور گاهواره کنم
\\
ای عجب چون بسازم این همه کار
&&
هیچ باشد همه چه چاره کنم
\\
عاقبت چون فلک فرو ریزم
&&
این روش گر هزار باره کنم
\\
همه چون چرخ گرد خود گردم
&&
گرچه خورشید پشتواره کنم
\\
نرهم از دو کون یک سر موی
&&
مگر از خویشتن گذاره کنم
\\
چون ز معشوق محو گشت فرید
&&
تا کیش مرغ عشق باره کنم
\\
\end{longtable}
\end{center}
