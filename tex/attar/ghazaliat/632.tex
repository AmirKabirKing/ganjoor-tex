\begin{center}
\section*{غزل شماره ۶۳۲: ای گرفته حسن تو هر دو جهان}
\label{sec:632}
\addcontentsline{toc}{section}{\nameref{sec:632}}
\begin{longtable}{l p{0.5cm} r}
ای گرفته حسن تو هر دو جهان
&&
در جمالت خیره چشم عقل و جان
\\
جان تن جان است و جان جان تویی
&&
در جهان جانی و در جانی جهان
\\
های و هوی عاشقانت هر سحر
&&
می نگنجد در زمین و آسمان
\\
بوالعجب مرغی است جان عاشقت
&&
کز دو کونش می نیابد آشیان
\\
جملهٔ عالم همی بینم به تو
&&
وز تو در عالم نمی‌بینم نشان
\\
ای ز پیدایی و پنهانی تو
&&
جان من هم در یقین هم در گمان
\\
تن همی داند که هستی بر کنار
&&
جان همی داند که هستی در میان
\\
بس سخن گویی از آنی بس خموش
&&
بس هویدایی از آنی بس نهان
\\
کی تواند دید نور آفتاب
&&
چشم اعمی چون ندارد جای آن
\\
ما همه عیبیم چون یابد وصال
&&
عیب‌دان در بارگاه غیب‌دان
\\
تا نگردد جان ما از عیب پاک
&&
کی شوی با عاشقانش هم عنان
\\
آستین نا کرده پر خون هر شبی
&&
کی شود شایستهٔ آن آستان
\\
همچو عطار از دو کون آزاد گرد
&&
بندهٔ یکتای او شو جاودان
\\
\end{longtable}
\end{center}
