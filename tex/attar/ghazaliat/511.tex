\begin{center}
\section*{غزل شماره ۵۱۱: آن در که بسته باید تا چند باز دارم}
\label{sec:511}
\addcontentsline{toc}{section}{\nameref{sec:511}}
\begin{longtable}{l p{0.5cm} r}
آن در که بسته باید تا چند باز دارم
&&
کامروز وقتش آمد کان در فراز دارم
\\
با هر که از حقیقت رمزی دمی بگویم
&&
گوید مگوی یعنی برگ مجاز دارم
\\
تا لاجرم به مردی با پاره پاره جانی
&&
در جان خویش گفتم چندان که راز دارم
\\
چون این جهان و آن یک با صد جهان دیگر
&&
در چشم من فروشد چون چشم باز دارم
\\
چیزی برفت از من و اینجا نماند چیزی
&&
تا این شود چون آن یک کاری دراز دارم
\\
جانی که داشتم من، شد محو عشق جانان
&&
جان من است جانان، جان دلنواز دارم
\\
نی نی اگر چو شمعی این دم زدم ز گرمی
&&
اکنون چو شمع از آن دم سر زیر گاز دارم
\\
چون عز و ناز ختم است بر تو همیشه دایم
&&
تا چند خویشتن را در عز و ناز دارم
\\
کارم فتاد و از من تو فارغی به غایت
&&
نه صبر می‌توانم نه کارساز دارم
\\
از بس که بی نیازی است آنجا که حضرت توست
&&
من زاد این بیابان عجز و نیاز دارم
\\
شوریدهٔ جهانم چون قربت تو جویم
&&
محمود نیستم من، خو با ایاز دارم
\\
بازی اگر نشیند بر دوش من نگیرم
&&
ورنه کسی نبوده است البته باز دارم
\\
من شمع جمع عشقم نه جان به تن بمانده
&&
جان در میان آتش تن در گداز دارم
\\
لاف ای فرید کم زن زیرا که در ره او
&&
چون سرنگون نه‌ای تو صد سرفراز دارم
\\
\end{longtable}
\end{center}
