\begin{center}
\section*{غزل شماره ۱۴۶: رطل گران ده صبوح زانکه رسیده است صبح}
\label{sec:146}
\addcontentsline{toc}{section}{\nameref{sec:146}}
\begin{longtable}{l p{0.5cm} r}
رطل گران ده صبوح زانکه رسیده است صبح
&&
تا سر شب بشکند تیغ کشیده است صبح
\\
روی نهفته است تیر روی نهاده است مهر
&&
پشت بداده است ماه هین که رسیده است صبح
\\
بر سر زنگی شب همچو کلاه است ماه
&&
بر در قفل سحر همچو کلید است صبح
\\
ای بت بربط‌نواز پردهٔ مستان بساز
&&
کز رخ هندوی شب پرده دریده است صبح
\\
صبح برآمد زکوه وقت صبوح است خیز
&&
کز جهت غافلان صور دمیده است صبح
\\
سوخته گردد شرار کز نفس سوخته
&&
گنبد فیروزه را فرق بریده است صبح
\\
بوی خوش باد صبح مشک دمد گوییا
&&
کز دم آهوی چین مشک مزید است صبح
\\
نی که از آن است صبح مشک فشان کز هوا
&&
نافهٔ عطار را بوی شنیده است صبح
\\
\end{longtable}
\end{center}
