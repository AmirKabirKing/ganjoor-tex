\begin{center}
\section*{غزل شماره ۶۲۵: گر مردی خویشتن ببینیم}
\label{sec:625}
\addcontentsline{toc}{section}{\nameref{sec:625}}
\begin{longtable}{l p{0.5cm} r}
گر مردی خویشتن ببینیم
&&
اندر پس دوکدان نشینیم
\\
دیگر نزنیم لاف مردی
&&
وز شرم ره زنان گزینیم
\\
کاری عجب اوفتاده ما را
&&
پیمانهٔ زهر و انگبینیم
\\
تا زهر چو انگبین نگردد
&&
یک ذره جمال او نبینیم
\\
سر رشتهٔ دل ز دست دادیم
&&
کین چیست که ما کنون درینیم
\\
ای ساقی درد درد در ده
&&
کامروز ورای کفر و دینیم
\\
ما در ره یار سر ببازیم
&&
وانگه پس کار خود نشینیم
\\
آبی در ده صبوحیان را
&&
کز عشق به سینه آتشینیم
\\
صبح رخ او پدید آمد
&&
ما جمله صبوحیان ازینیم
\\
ما مستانیم و همچو عطار
&&
از مستی خویش شرمگینیم
\\
\end{longtable}
\end{center}
