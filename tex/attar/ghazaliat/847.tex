\begin{center}
\section*{غزل شماره ۸۴۷: ترسا بچه‌ای دیشب در غایت ترسایی}
\label{sec:847}
\addcontentsline{toc}{section}{\nameref{sec:847}}
\begin{longtable}{l p{0.5cm} r}
ترسا بچه‌ای دیشب در غایت ترسایی
&&
دیدم به در دیری چون بت که بیارایی
\\
زنار کمر کرده وز دیر برون جسته
&&
طرف کله اشکسته از شوخی و رعنایی
\\
چون چشم و لبش دیدم صد گونه بگردیدم
&&
ترسا بچه چون دیدم بی توش و توانایی
\\
آمد بر من سرمست زنار و می اندر دست
&&
اندر بر من بنشست گفتا اگر از مایی
\\
امشب بر ما باشی تاج سر ما باشی
&&
ما از تو بیاساییم وز ما تو بیاسایی
\\
از جان کنمت خدمت بی منت و بی علت
&&
دارم ز تو صد منت کامشب بر ما آیی
\\
رفتم به در دیرش خوردم ز می عشقش
&&
در حال دلم دریافت راهی ز هویدایی
\\
عطار ز عشق او سرگشته و حیران شد
&&
در دیر مقیمی شد دین داد به ترسایی
\\
\end{longtable}
\end{center}
