\begin{center}
\section*{غزل شماره ۷۲۵: ای ز شراب غفلت مست و خراب مانده}
\label{sec:725}
\addcontentsline{toc}{section}{\nameref{sec:725}}
\begin{longtable}{l p{0.5cm} r}
ای ز شراب غفلت مست و خراب مانده
&&
با سایه خو گرفته وز آفتاب مانده
\\
تا چند باشی آخر از حرص نفس کافر
&&
ایمان به باد داده در خورد و خواب مانده
\\
اندیشه کن تو روزی کین خفتگان ره را
&&
گه در حجاب بینی گه در عذاب مانده
\\
آنجا که نقدها را ناقد عیار خواهد
&&
مردان مرد بینی در اضطراب مانده
\\
وانجا که باز خواهند از جان و دل نشانی
&&
هم دل سیاه بینی هم جان خراب مانده
\\
وانجا که عاشقان را از صدق باز پرسند
&&
بس عاشق مجازی کاندر جواب مانده
\\
ای اوفتاده از ره بگشای چشم و بنگر
&&
پیران راه‌بین را سر در طناب مانده
\\
عیسی پاک‌رو را از سوزنی شکسته
&&
حیران میان این ره چون در خلاب مانده
\\
ترسم که هیچ عاشق پیشان ره نبیند
&&
وان ماه‌رخ بماند اندر نقاب مانده
\\
در بحر عشق دری است از چشم خلق پنهان
&&
ما جمله غرقه گشته وان در درآب مانده
\\
بر آتش محبت از شرح این عجایب
&&
عطار را دل و جان در تف و تاب مانده
\\
\end{longtable}
\end{center}
