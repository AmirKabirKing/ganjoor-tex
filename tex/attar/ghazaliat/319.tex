\begin{center}
\section*{غزل شماره ۳۱۹: هر که سرگردان این سودا بود}
\label{sec:319}
\addcontentsline{toc}{section}{\nameref{sec:319}}
\begin{longtable}{l p{0.5cm} r}
هر که سرگردان این سودا بود
&&
از دو عالم تا ابد یکتا بود
\\
هر که نادیده در اینجا دم زند
&&
چو حدیث مرد نابینا بود
\\
کی تواند بود مرد راهبر
&&
هر که او همچون زنان رعنا بود
\\
راهبر تا درگه حق گام گام
&&
هم بره بینا و هم دانا بود
\\
هر که او را دیده بینا شد به کل
&&
در وجود خویش نابینا بود
\\
دیده آن دارد که اسرار دو کون
&&
ذره ذره بر دلش صحرا بود
\\
جملهٔ عالم به دریا اندرند
&&
فرخ آنکس کاندرو دریا بود
\\
تا تو در بحری ندارد کار نور
&&
بحر در تو نور کار اینجا بود
\\
قطرهٔ بحرت اگر در دل فتاد
&&
قطره نبود لؤلؤ لالا بود
\\
هر که در دریاست تر دامن بود
&&
وانکه دریا اوست او از ما بود
\\
تا تو دربند خودی خود را بتی
&&
بت‌پرستی از تو کی زیبا بود
\\
تا گرفتاری تو در عقل لجوج
&&
از تو این سودا همه سودا بود
\\
مرد ره آن است کز لایعقلی
&&
در صف مستان سر غوغا بود
\\
گوی آنکس می‌برد در راه عشق
&&
کو چو گویی بی سر و بی پا بود
\\
آن کس آزادی گرفت از مردمان
&&
کو میان مردمان رسوا بود
\\
هر که چون عطار فارغ شد ز خلق
&&
دی و امروزش همه فردا بود
\\
\end{longtable}
\end{center}
