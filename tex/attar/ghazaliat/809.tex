\begin{center}
\section*{غزل شماره ۸۰۹: کجایی ای دل و جانم مگر که در دل و جانی}
\label{sec:809}
\addcontentsline{toc}{section}{\nameref{sec:809}}
\begin{longtable}{l p{0.5cm} r}
کجایی ای دل و جانم مگر که در دل و جانی
&&
که کس نمی‌دهد از تو به هیچ جای نشانی
\\
به هیچ جای نشانی نداد هیچ کس از تو
&&
نشانی از تو کسی چون دهد که برتر از آنی
\\
عجب بمانده‌ام از ذات و از صفات تو دایم
&&
کز آفتاب هویداتری اگرچه نهانی
\\
چه گوهری تو که در عرصهٔ دو کون نگنجی
&&
همه جهان ز تو پر گشت و تو برون ز جهانی
\\
منم که هستی من بند ره شدست درین ره
&&
تویی که از تویی خود مرا ز من برهانی
\\
من از خودی خود افتاده‌ام به چاه طبیعت
&&
مرا ز چاه به ماه ار بر آوری تو توانی
\\
در آرزوی تو عمری به سر دویدم و اکنون
&&
چو در سر آمدم آخر مرا به سر چه دوانی
\\
چه باشد ار ز سر لطف جان تشنه لبان را
&&
از آن شراب دل آشوب قطره‌ای بچشانی
\\
امید ما همه آن است در ره تو که یک‌دم
&&
ز بوی خویش نسیمی به جان ما برسانی
\\
ز اشتیاق تو عطار از دو کون فنا شد
&&
از آن او بود این و از آن خویش تو دانی
\\
\end{longtable}
\end{center}
