\begin{center}
\section*{غزل شماره ۱۱۱: از قوت مستیم ز هستیم خبر نیست}
\label{sec:111}
\addcontentsline{toc}{section}{\nameref{sec:111}}
\begin{longtable}{l p{0.5cm} r}
از قوت مستیم ز هستیم خبر نیست
&&
مستم ز می عشق و چو من مست دگر نیست
\\
در جشن می عشق که خون جگرم ریخت
&&
نقل من دلسوخته جز خون جگر نیست
\\
مستان می‌عشق درین بادیه رفتند
&&
من ماندم و از ماندن من نیز اثر نیست
\\
در بادیهٔ عشق نه نقصان نه کمال است
&&
چون من دو جهان خلق اگر هست و اگر نیست
\\
گویند برو تا به درش برگذری بوک
&&
هیهات که گر باد شوم روی گذر نیست
\\
زین پیش دلی بود مرا عاشق و امروز
&&
جز بی‌خبریم از دل خود هیچ خبر نیست
\\
جانا اگرم در سر کار تو رود جان
&&
از دادن صد جان دگرم بیم خطر نیست
\\
در دامن تو دست کسی می‌زند ای دوست
&&
کو در ره سودای تو با دامن تر نیست
\\
دانی که چه خواهم من دلسوخته از تو
&&
خواهم که نخواهم، دگرم هیچ نظر نیست
\\
عطار چنان غرق غمت شد که دلش را
&&
یک دم دل دل نیست زمانی سر سر نیست
\\
\end{longtable}
\end{center}
