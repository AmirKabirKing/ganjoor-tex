\begin{center}
\section*{غزل شماره ۴۷۷: تو بلندی عظیم و من پستم}
\label{sec:477}
\addcontentsline{toc}{section}{\nameref{sec:477}}
\begin{longtable}{l p{0.5cm} r}
تو بلندی عظیم و من پستم
&&
چکنم تا به تو رسد دستم
\\
تا که سر زیر پای تو ننهم
&&
نرسم بر چنان که خود هستم
\\
تا چنین هستیی حجابم بود
&&
آن ز من بود رخت بربستم
\\
چون ز هستی خویش نیست شدم
&&
لاجرم یا نه نیست یا هستم
\\
گرچه وصل تو نیست یک نفسم
&&
اشتیاق تو هست پیوستم
\\
خود تو دانی کز اشتیاق تو بود
&&
در دو عالم به هرچه پیوستم
\\
دوش عشقت درآمد از در دل
&&
من ز غیرت ز پای ننشستم
\\
گفت بنشین و جام و جم در ده
&&
تا ز جام جمت کنی مستم
\\
گفتمش جام جام به دستم بود
&&
طفل بودم ز جهل بشکستم
\\
گفت اگر جام جم شکست تورا
&&
دیگری به از آنت بفرستم
\\
سخت درمانده بودم و عاجز
&&
چون شنیدم من این سخن رستم
\\
آفتابی برآمد از جانم
&&
من ز هر دو جهان برون جستم
\\
از بلندی که جان من بر شد
&&
عرش و کرسی به جمله شد پستم
\\
چون شوم من ورای هر دو جهان
&&
ماه و ماهی فتاد در شستم
\\
عمر عطار شد هزاران قرن
&&
چند گویی ز پنجه و شستم
\\
\end{longtable}
\end{center}
