\begin{center}
\section*{غزل شماره ۴۵۳: زهی در کوی عشقت مسکن دل}
\label{sec:453}
\addcontentsline{toc}{section}{\nameref{sec:453}}
\begin{longtable}{l p{0.5cm} r}
زهی در کوی عشقت مسکن دل
&&
چه می‌خواهی ازین خون خوردن دل
\\
چکیده خون دل بر دامن جان
&&
گرفته جان پرخون دامن دل
\\
از آن روزی که دل دیوانهٔ توست
&&
به صد جان من شدم در شیون دل
\\
منادی می‌کنند در شهر امروز
&&
که خون عاشقان در گردن دل
\\
چو رسوا کرد ما را درد عشقت
&&
همی کوشم به رسوا کردن دل
\\
چو عشقت آتشی در جان من زد
&&
برآمد دود عشق از روزن دل
\\
زهی خال و زهی روی چو ماهت
&&
که دل هم دام جان هم ارزن دل
\\
مکن جانا دل ما را نگه‌دار
&&
که آسان است بر تو بردن دل
\\
چو گل اندر هوای روی خوبت
&&
به خون درمی‌کشم پیراهن دل
\\
بیا جانا دل عطار کن شاد
&&
که نزدیک است وقت رفتن دل
\\
\end{longtable}
\end{center}
