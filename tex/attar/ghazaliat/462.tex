\begin{center}
\section*{غزل شماره ۴۶۲: شیفتهٔ حلقهٔ گوش توام}
\label{sec:462}
\addcontentsline{toc}{section}{\nameref{sec:462}}
\begin{longtable}{l p{0.5cm} r}
شیفتهٔ حلقهٔ گوش توام
&&
سوختهٔ چشمهٔ نوش توام
\\
ماهرخ با خط و خال منی
&&
دلشدهٔ بی تن و توش توام
\\
ترک منی گوش به من دار از آنک
&&
هندوک حلقه به گوش توام
\\
خانه بیاراسته‌ام چون نگار
&&
منتظر خانه فروش توام
\\
چون دلم از خشم تو آید به جوش
&&
عاشق خشم تو و جوش توام
\\
خط چه کشی بر من غمکش از آنک
&&
مست خط غالیه‌پوش توام
\\
هوش به من باز کی آید که من
&&
تا به ابد رفته ز هوش توام
\\
گرچه به گویایی من نیست کس
&&
یک شکرم ده که خموش توام
\\
چون بگریزی تو ز عطار از آنک
&&
با تو به هم دوش به دوش توام
\\
\end{longtable}
\end{center}
