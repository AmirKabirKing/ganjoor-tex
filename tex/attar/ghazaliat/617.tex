\begin{center}
\section*{غزل شماره ۶۱۷: ساقیا خیز که تا رخت به خمار کشیم}
\label{sec:617}
\addcontentsline{toc}{section}{\nameref{sec:617}}
\begin{longtable}{l p{0.5cm} r}
ساقیا خیز که تا رخت به خمار کشیم
&&
تائبان را به شرابی دو سه در کار کشیم
\\
زاهد خانه‌نشین را به یکی کوزه درد
&&
اوفتان خیزان از خانه به بازار کشیم
\\
هوست هست که صافی دل و صوفی گردی
&&
خیز تا پیش مغان دردی خمار کشیم
\\
هر که را در ره اسلام قدم ثابت نیست
&&
به یکی جرعه میش در صف کفار کشیم
\\
هر که دعوی اناالحق کند و حق گوید
&&
انا گویان خودی را به سر دار کشیم
\\
چند داریم نهان زیر مرقع زنار
&&
وقت نامد که خط اندر خط زنار کشیم
\\
هیچکس را ندهد دنیی و دین دست بهم
&&
هرکه گوید که دهد، خنجر انکار کشیم
\\
گر تو دین می‌طلبی از سر دنیی برخیز
&&
که ز دین بار نیابیم مگر بار کشیم
\\
گر ازین شاخ گل وصل طمع می‌داریم
&&
اندرین راه غم عشق چو عطار کشیم
\\
\end{longtable}
\end{center}
