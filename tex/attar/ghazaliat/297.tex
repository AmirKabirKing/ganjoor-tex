\begin{center}
\section*{غزل شماره ۲۹۷: آن را که غمت به خویش خواند}
\label{sec:297}
\addcontentsline{toc}{section}{\nameref{sec:297}}
\begin{longtable}{l p{0.5cm} r}
آن را که غمت به خویش خواند
&&
شادی جهان غم تو داند
\\
چون سلطنتت به دل درآید
&&
از خویشتنش فراستاند
\\
ور هیچ نقاب برگشایی
&&
یک ذره وجود کس نماند
\\
چون نیست شوند در ره هست
&&
جان را به کمال دل رساند
\\
زان پس نظرت به دست گیری
&&
عشق تو قیامتی براند
\\
جان را دو جهان تمام باید
&&
تا بر سگ کوی تو فشاند
\\
چون بگشایی ز پای دل بند
&&
جان بند نهاد بگسلاند
\\
هر پرده که پیش او درآید
&&
از قوت عشق بردراند
\\
ساقی محبتش به هر گام
&&
ذوق می عشق می‌چشاند
\\
وقت است که جان مست عطار
&&
ابلق ز جهان برون جهاند
\\
\end{longtable}
\end{center}
