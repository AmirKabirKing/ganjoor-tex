\begin{center}
\section*{غزل شماره ۵۰۹: دریغا کانچه جستم آن ندیدم}
\label{sec:509}
\addcontentsline{toc}{section}{\nameref{sec:509}}
\begin{longtable}{l p{0.5cm} r}
دریغا کانچه جستم آن ندیدم
&&
نجات تن خلاص جان ندیدم
\\
دلم می‌سوزد از درد و چه سازم
&&
که درد خویش را درمان ندیدم
\\
به کار افتادگی خویش هرگز
&&
ندیدم هیچ سرگردان ندیدم
\\
بگردیدم چو گردون گرد عالم
&&
چو خود واله چو خود حیران ندیدم
\\
شدم چون گوی سرگردان که خود را
&&
حریفی درد در میدان ندیدم
\\
درین حیرت ندارم صبر و غم اینت
&&
که گشتن خویش را قربان ندیدم
\\
درین وادی بسی از پیش رفتم
&&
ولی یک ذره از پیشان ندیدم
\\
کنون از پس شدم عمری ولیکن
&&
سر یک مویی از انسان ندیدم
\\
چو راهی بی نهایت می‌نماید
&&
سر و بن یافتن امکان ندیدم
\\
چو شمعی خویش را در آتش و دود
&&
اگر دیدم به جز گریان ندیدم
\\
گزیرم نیست از خوناب دیده
&&
که من هرگز چنین طوفان ندیدم
\\
ز عالم شربتی بی خون نخوردم
&&
ز گیتی بی جگر یک نان ندیدم
\\
ندیدم در جهان یک ذره شادی
&&
که تا اندوه صد چندان ندیدم
\\
چه گر خورشید عمرم بود تاوان
&&
چو بر من تافت جز تاوان ندیدم
\\
حکایت چون کنم از ملک یوسف
&&
که من جز چاه و جز زندان ندیدم
\\
خطا گفتم بسی دیدم نکویی
&&
ولی خود را سزای آن ندیدم
\\
کمال دیگران بر خود چه بندم
&&
که من در خویش جز نقصان ندیدم
\\
صدف را آن بود بهتر که گوید
&&
که من در عمر خود باران ندیدم
\\
فقیری بایدم همدرد و همدم
&&
که می‌گوید که من سلطان ندیدم
\\
تو ای عطار چون اینجا رسیدی
&&
سخن گفتن تورا سامان ندیدم
\\
\end{longtable}
\end{center}
