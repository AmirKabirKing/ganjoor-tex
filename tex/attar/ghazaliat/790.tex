\begin{center}
\section*{غزل شماره ۷۹۰: جانا ز فراق تو این محنت جان تا کی}
\label{sec:790}
\addcontentsline{toc}{section}{\nameref{sec:790}}
\begin{longtable}{l p{0.5cm} r}
جانا ز فراق تو این محنت جان تا کی
&&
دل در غم عشق تو رسوای جهان تا کی
\\
چون جان و دلم خون شد در درد فراق تو
&&
بر بوی وصال تو دل بر سر جان تا کی
\\
نامد گه آن آخر کز پرده برون آیی
&&
آن روی بدان خوبی در پرده نهان تا کی
\\
در آرزوی رویت ای آرزوی جانم
&&
دل نوحه کنان تا چند، جان نعره‌زنان تا کی
\\
بشکن به سر زلفت این بند گران از دل
&&
بر پای دل مسکین این بند گران تا کی
\\
دل بردن مشتاقان از غیرت خود تا چند
&&
خون خوردن و خاموشی زین دلشدگان تا کی
\\
ای پیر مناجاتی در میکده رو بنشین
&&
درباز دو عالم را این سود و زیان تا کی
\\
اندر حرم معنی از کس نخرند دعوی
&&
پس خرقه بر آتش نه زین مدعیان تا کی
\\
گر طالب دلداری از کون و مکان بگذر
&&
هست او ز مکان برتر از کون و مکان تا کی
\\
گر عاشق دلداری ور سوختهٔ یاری
&&
بی نام و نشان می‌رو زین نام و نشان تا کی
\\
گفتی به امید تو بارت بکشم از جان
&&
پس بارکش ار مردی این بانگ و فغان تا کی
\\
عطار همی بیند کز بار غم عشقش
&&
عمر ابدی یابد عمر گذران تا کی
\\
\end{longtable}
\end{center}
