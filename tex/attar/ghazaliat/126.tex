\begin{center}
\section*{غزل شماره ۱۲۶: در عشق تو عقل سرنگون گشت}
\label{sec:126}
\addcontentsline{toc}{section}{\nameref{sec:126}}
\begin{longtable}{l p{0.5cm} r}
در عشق تو عقل سرنگون گشت
&&
جان نیز خلاصهٔ جنون گشت
\\
خود حال دلم چگونه گویم
&&
کان کار به جان رسیده چون گشت
\\
بر خاک درت به زاری زار
&&
از بس که به خون بگشت خون گشت
\\
خون دل ماست یا دل ماست
&&
خونی که ز دیده‌ها برون گشت
\\
درمان چه طلب کنم که عشقت
&&
ما را سوی درد رهنمون گشت
\\
آن مرغ که بود زیرکش نام
&&
در دام بلای تو زبون گشت
\\
لختی پر و بال زد به آخر
&&
از پای فتاد و سرنگون گشت
\\
تا دور شدم من از در تو
&&
از ناله دلم چو ارغنون گشت
\\
تا قوت عشق تو بدیدم
&&
سرگشتگیم بسی فزون گشت
\\
تا درد تو را خرید عطار
&&
قد الفش بسان نون گشت
\\
عطار که بود کشتهٔ تو
&&
دریاب که کشته‌تر کنون گشت
\\
\end{longtable}
\end{center}
