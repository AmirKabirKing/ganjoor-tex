\begin{center}
\section*{غزل شماره ۳۲۳: آنچه نقد سینهٔ مردان بود}
\label{sec:323}
\addcontentsline{toc}{section}{\nameref{sec:323}}
\begin{longtable}{l p{0.5cm} r}
آنچه نقد سینهٔ مردان بود
&&
زآرزوی آن فلک گردان بود
\\
گر از آن یک ذره گردد آشکار
&&
هر دو عالم تا ابد پنهان بود
\\
در گذر از کون تا تاب آوری
&&
خود که را در کون تاب آن بود
\\
آن فلک کان در درون عاشق است
&&
آفتاب آن رخ جانان بود
\\
گر فرو استد ز دوران این فلک
&&
آن فلک را تا ابد دوران بود
\\
نور این خورشید اگر زایل شود
&&
نور آن خورشید جاویدان بود
\\
زود بیند آن فلک و آن آفتاب
&&
هر که را یک ذره نور جان بود
\\
وانکه نور جان ندارد ذره‌ای
&&
تا بود در کار خود حیران بود
\\
چند گویی کین چنین و آن چنان
&&
تا چنینی عمر تو تاوان بود
\\
کی بود پروای خلقش ذره‌ای
&&
هر که او در کار سرگردان بود
\\
پای در نه راه را پایان مجوی
&&
زانکه راه عشق بی‌پایان بود
\\
عشق را دردی بباید بی قرار
&&
آن چنان دردی که بی درمان بود
\\
گر زند عطار بی این سر نفس
&&
آن نفس بر جان او تاوان بود
\\
\end{longtable}
\end{center}
