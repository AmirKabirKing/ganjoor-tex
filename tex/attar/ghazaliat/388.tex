\begin{center}
\section*{غزل شماره ۳۸۸: درآمد دوش ترکم مست و هشیار}
\label{sec:388}
\addcontentsline{toc}{section}{\nameref{sec:388}}
\begin{longtable}{l p{0.5cm} r}
درآمد دوش ترکم مست و هشیار
&&
ز سر تا پای او اقرار و انکار
\\
ز هشیاری نه دیوانه نه عاقل
&&
ز سرمستی نه در خواب و نه بیدار
\\
به یک دم از هزاران سوی می‌گشت
&&
فلک از گشت او می‌گشت دوار
\\
به هر سوئی که می‌گشت او همی ریخت
&&
ز هر جزویش صورت‌های بسیار
\\
چو باران از سر هر موی زلفش
&&
ز بهر عاشقان می‌ریخت پندار
\\
زمانی کفر می‌افشاند بر دین
&&
زمانی تخت می‌انداخت بردار
\\
زمانی شهد می‌پوشید در زهر
&&
زمانی گل نهان می‌کرد در خار
\\
زمانی صاف می‌آمیخت با درد
&&
زمانی نور می‌انگیخت از نار
\\
چو بوقلمون به هر دم رنگ دیگر
&&
ولیکن آن همه رنگش به یکبار
\\
همه اضدادش اندر یک مکان جمع
&&
همه الوانش اندر یک زمان یار
\\
زمانش دایما عین مکانش
&&
ولی نه این و نه آنش پدیدار
\\
دو ضدش در زمانی و مکانی
&&
به هم بودند و از هم دور هموار
\\
تو مینوش این که از طامات حرفی است
&&
وگر این می‌نیوشی عقل بگذار
\\
که گر با عقل گرد این بگردی
&&
به بتخانه میان بندی به زنار
\\
چو دیدم روی او گفتم چه چیزی
&&
که من هرگز ندیدم چون تو دلدار
\\
جوابم داد کز دریای قدرت
&&
منم مرغی، دو عالم زیر منقار
\\
علی‌الجمله در او گم گشت جانم
&&
دگر کفر است چون گویم زهی کار
\\
اگر گویم به صد عمر آنچه دیدم
&&
سر مویی نیاید زان به گفتار
\\
چه بودی گر زبان من نبودی
&&
که گنگان راست نیکو شرح اسرار
\\
زبان موسی از آتش از آن سوخت
&&
که تا پاس زبان دارد به هنجار
\\
چو چیزی در عبارت می‌نیاید
&&
فضولی باشد آن گفتن به اشعار
\\
که گر صد بار در روزی بمیری
&&
ندانی سر این معنی چو عطار
\\
\end{longtable}
\end{center}
