\begin{center}
\section*{غزل شماره ۳۱۷: دل ز جان برگیر تا راهت دهند}
\label{sec:317}
\addcontentsline{toc}{section}{\nameref{sec:317}}
\begin{longtable}{l p{0.5cm} r}
دل ز جان برگیر تا راهت دهند
&&
ملک دو عالم به یک آهت دهند
\\
چون تو برگیری دل از جان مردوار
&&
آنچه می‌جویی هم آنگاهت دهند
\\
گر بسوزی تا سحر هر شب چو شمع
&&
تحفه از نقد سحرگاهت دهند
\\
گر گدای آستان او شوی
&&
هر زمانی ملک صد شاهت دهند
\\
گر بود آگاه جانت را جز او
&&
گوش مال جان به ناگاهت دهند
\\
لذت دنیی اگر زهرت شود
&&
شربت خاصان درگاهت دهند
\\
تا نگردی بی نشان از هر دو کون
&&
کی نشان آن حرم گاهت دهند
\\
چون به تاریکی در است آب حیات
&&
گنج وحدت در بن چاهت دهند
\\
چون سپیدی تفرقه است اندر رهش
&&
در سیاهی راه کوتاهت دهند
\\
بی‌سواد فقر تاریک است راه
&&
گر هزاران روی چون ماهت دهند
\\
چون درون دل شد از فقرت سیاه
&&
ره برون زین سبز خرگاهت دهند
\\
در سواد اعظم فقر است آنک
&&
نقطهٔ کلی به اکراهت دهند
\\
ای فرید اینجا چو کوهی صبر کن
&&
تا ازین خرمن یکی کاهت دهند
\\
\end{longtable}
\end{center}
