\begin{center}
\section*{غزل شماره ۳۹۷: آتش عشق تو دلم، کرد کباب ای پسر}
\label{sec:397}
\addcontentsline{toc}{section}{\nameref{sec:397}}
\begin{longtable}{l p{0.5cm} r}
آتش عشق تو دلم، کرد کباب ای پسر
&&
زیر و زبر شدم ز تو، چیست صواب ای پسر
\\
چون من خسته دل ز تو، زیر و زبر بمانده‌ام
&&
زیر و زبر چه می‌کنی، زلف بتاب ای پسر
\\
تا که بدید چشم من، چهرهٔ جانفزای تو
&&
ساخته‌ام ز خون دل، چره خضاب ای پسر
\\
جان من از جهان غم، سوخته شد به جان تو
&&
جام بیا و درفکن، بادهٔ ناب ای پسر
\\
آب حیات جان من، جام شراب می‌دهد
&&
زانکه به جان همی رسد، جام شراب ای پسر
\\
چند غم جهان خوری، چیست جهان، خرابه‌ای
&&
ما همه در خرابه‌ای، مست و خراب ای پسر
\\
هین که نشست آسمان، در پی گوشمال تو
&&
خیز و بمال اندکی، گوش رباب ای پسر
\\
نقل چه می‌کنیم ما، قند لب تو نقل بس
&&
زان دو لب شکرفشان، هین بشتاب ای پسر
\\
شمع چه می‌کنیم ما، نور رخ تو شمع بس
&&
برفکن از رخ چو مه، خیز نقاب ای پسر
\\
نرگس نیم خواب را، باز کن و شراب خور
&&
غفلت ماست خواب ما، چند ز خواب ای پسر
\\
زان دو لب تو یک شکر، بنده سال می‌کند
&&
مفتی این سخن تویی، چیست جواب ای پسر
\\
گرچه تو آفتاب را، رخ بنهاده‌ای به رخ
&&
با من دلشده مرا، خر به خلاب ای پسر
\\
وصف تو گر فرید را، ورد زبان همی شود
&&
آب شود ز رشک او، در خوشاب ای پسر
\\
\end{longtable}
\end{center}
