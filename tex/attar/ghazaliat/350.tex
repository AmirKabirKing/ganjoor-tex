\begin{center}
\section*{غزل شماره ۳۵۰: هر که را ذوق دین پدید آید}
\label{sec:350}
\addcontentsline{toc}{section}{\nameref{sec:350}}
\begin{longtable}{l p{0.5cm} r}
هر که را ذوق دین پدید آید
&&
شهد دنیاش کی لذیذ آید
\\
چه کنی در زمانه‌ای که درو
&&
پیر چون طفل نا رسید آید
\\
آنچنان عقل را چه خواهی کرد
&&
که نگونسار یک نبید آید
\\
عقل بفروش و جمله حیرت خر
&&
که تو را سود زین خرید آید
\\
این نه آن عالمی است ای غافل
&&
که درو هیچکس پدید آید
\\
نشود باز این چنین قفلی
&&
گر دو عالم پر از کلید آید
\\
گر در آیند ذره ذره به بانگ
&&
آن همه بانگ ناشنید آید
\\
چه شود بیش و کم ازین دریا
&&
خواجه گر پاک و گر پلید آید
\\
هر که دنیا خرید ای عطار
&&
خر بود کز پی خوید آید
\\
\end{longtable}
\end{center}
