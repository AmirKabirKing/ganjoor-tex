\begin{center}
\section*{غزل شماره ۷۰: دوش ناگه آمد و در جان نشست}
\label{sec:070}
\addcontentsline{toc}{section}{\nameref{sec:070}}
\begin{longtable}{l p{0.5cm} r}
دوش ناگه آمد و در جان نشست
&&
خانه ویران کرد و در پیشان نشست
\\
عالمی بر منظر معمور بود
&&
او چرا در خانهٔ ویران نشست
\\
گنج در جای خراب اولیتر است
&&
گنج بود او در خرابی زان نشست
\\
هیچ یوسف دیده‌ای کز تخت و تاج
&&
چون دلش بگرفت در زندان نشست
\\
گرچه پیدا برد دل از دست من
&&
آمد و بر جان من پنهان نشست
\\
چون مرا تنها بدید آن ماه روی
&&
گفت تنها بیش ازین نتوان نشست
\\
جان بده وانگه نشست ما طلب
&&
که توان با جان بر جانان نشست
\\
از سر جان چون تو برخیزی تمام
&&
من کنم آن ساعتت در جان نشست
\\
چون ز جانان این سخن بشنید جان
&&
خویش را درباخت و سرگردان نشست
\\
خویشتن را خویشتن آن وقت دید
&&
کو چو گویی در خم چوگان نشست
\\
دایما در نیستی سرگشته بود
&&
زان چنین عطار زان حیران نشست
\\
\end{longtable}
\end{center}
