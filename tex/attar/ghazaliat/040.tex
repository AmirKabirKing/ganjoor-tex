\begin{center}
\section*{غزل شماره ۴۰: عقل مست لعل جان افزای توست}
\label{sec:040}
\addcontentsline{toc}{section}{\nameref{sec:040}}
\begin{longtable}{l p{0.5cm} r}
عقل مست لعل جان افزای توست
&&
دل غلام نرگس رعنای توست
\\
نیکویی را در همه روی زمین
&&
گر قبایی هست بر بالای توست
\\
چون کسی را نیست حسن روی تو
&&
سیر مهر و مه به حسن رای توست
\\
نور ذره ذره بخش هر دو کون
&&
آفتاب طلعت زیبای توست
\\
در جهان هرجا که هست آرایشی
&&
پرتو از روی جهان‌آرای توست
\\
تا رخت شد ملک‌بخش هر دو کون
&&
مالک الملک جهان مولای توست
\\
خون اگر در آهوی چین مشک شد
&&
هم ز چین زلف عنبرسای توست
\\
گرچه آب خضر جام جم بشد
&&
تشنهٔ جام جهان افزای توست
\\
خلق عالم در رهت سر باختند
&&
ور کسی را هست سر همپای توست
\\
آسمان سر بر زمین هر جای تو
&&
در طواف عشق یک یک جای توست
\\
آفتاب بی سر و بن ذره‌وار
&&
این چنین سرگشته در سودای توست
\\
این جهان و آن جهان و هرچه هست
&&
شبنمی لب تشنه از دریای توست
\\
چون به جز تو در دو عالم نیست کس
&&
در دو عالم کیست کوهمتای توست
\\
هر که را هر ذره‌ای چشمی شود
&&
هم گر انصاف است نابینای توست
\\
گر فرید امروز چون شوریده‌ای است
&&
عاقل خلق است چون شیدای توست
\\
\end{longtable}
\end{center}
