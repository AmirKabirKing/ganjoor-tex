\begin{center}
\section*{غزل شماره ۷۸۹: در ده می عشق یک دم ای ساقی}
\label{sec:789}
\addcontentsline{toc}{section}{\nameref{sec:789}}
\begin{longtable}{l p{0.5cm} r}
در ده می عشق یک دم ای ساقی
&&
تا عقل کند گزاف در باقی
\\
زین عقل گزاف گوی پر دعوی
&&
بگذر که گذشت عمر ای ساقی
\\
دردی در ده که توبه بشکستم
&&
تا کی ز نفاق و زرق و خناقی
\\
ما ننگ وجود پارسایانیم
&&
از روی و ریا نهفته زراقی
\\
ای ساقی جان بیار جام می
&&
کامروز تو دست گیر عشاقی
\\
تا باز رهیم یک زمان از خود
&&
فانی گردیم و جاودان باقی
\\
رفتیم به بوی تو همه آفاق
&&
تو خود نه ز فوق و نه ز آفاقی
\\
کس می نرسد به آستان تو
&&
زیرا که تو در خودی خود طاقی
\\
بس جان که بسوختند مشتاقان
&&
بر آتش عشق تو ز مشتاقی
\\
بنمای به خلق رخ که خود گفتی
&&
با ما که تخلقوا به اخلاقی
\\
عطار برو که در ره معنی
&&
امروز محققی به اطلاقی
\\
\end{longtable}
\end{center}
