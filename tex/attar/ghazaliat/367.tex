\begin{center}
\section*{غزل شماره ۳۶۷: نه یار هرکسی را رخسار می‌نماید}
\label{sec:367}
\addcontentsline{toc}{section}{\nameref{sec:367}}
\begin{longtable}{l p{0.5cm} r}
نه یار هرکسی را رخسار می‌نماید
&&
نه هر حقیر دل را دیدار می‌نماید
\\
در آرزوی رویش در خاک خفت و خون خور
&&
کان ماه‌روی رخ را دشوار می‌نماید
\\
بر چار سوی دعوی از بی‌نیازی خود
&&
سرهای سرکشان بین کز دار می‌نماید
\\
سلطان غیرت او خون همه عزیزان
&&
بر خاک اگر بریزد بس خوار می‌نماید
\\
گر مرد ره نه‌ای تو بر بوی گل چه پویی
&&
رو باز گرد کین ره پر خار می‌نماید
\\
زنهار تا بپویی بی رهبری درین ره
&&
زیرا که این بیابان خون‌خوار می‌نماید
\\
گر مردیی نداری پرهیز کن که چون تو
&&
سرگشتگان گمره بسیار می‌نماید
\\
در راه کفر و ایمان مرد آن بود که خود را
&&
دایم چنانکه باشد در کار می‌نماید
\\
در کار اگر تمامی در نه قدم درین ره
&&
کاحوال ناتمامان بس زار می‌نماید
\\
کو آتشی که بر وی این خرقه را بسوزم
&&
کین خرقه در بر من زنار می‌نماید
\\
اندر میان غفلت در خواب شد دل من
&&
کو هیچ دل که یک دم بیدار می‌نماید
\\
جمله ز خود نمایی اندر نفاق مستند
&&
کو عاشقی که در دین هشیار می‌نماید
\\
در بند دین و دنیی لیکن نه دین و دنیی
&&
سرگشته روزگاری عطار می‌نماید
\\
\end{longtable}
\end{center}
