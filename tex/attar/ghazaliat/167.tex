\begin{center}
\section*{غزل شماره ۱۶۷: جانا حدیث حسنت در داستان نگنجد}
\label{sec:167}
\addcontentsline{toc}{section}{\nameref{sec:167}}
\begin{longtable}{l p{0.5cm} r}
جانا حدیث حسنت در داستان نگنجد
&&
رمزی ز راز عشقت در صد زبان نگنجد
\\
جولانگه جلالت در کوی دل نباشد
&&
جلوه گه جمالت در چشم و جان نگنجد
\\
سودای زلف و خالت در هر خیال ناید
&&
اندیشهٔ وصالت جز در گمان نگنجد
\\
در دل چو عشقت آمد سودای جان نماند
&&
در جان چو مهرت افتد عشق روان نگنجد
\\
پیغام خستگانت در کوی تو که آرد
&&
کانجا ز عاشقانت باد وزان نگنجد
\\
دل کز تو بوی یابد در گلستان نپوید
&&
جان کز تو رنگ گیرد خود در جهان نگنجد
\\
آن دم که عاشقان را نزد تو بار باشد
&&
مسکین کسی که آنجا در آستان نگنجد
\\
بخشای بر غریبی کز عشق می‌نمیرد
&&
وانگه در آشیانت خود یک زمان نگنجد
\\
جان داد دل که روزی در کوت جای یابد
&&
نشناخت او که آخر جای چنان نگنجد
\\
آن دم که با خیالت دل را ز عشق گوید
&&
عطار اگر شود جان اندر میان نگنجد
\\
\end{longtable}
\end{center}
