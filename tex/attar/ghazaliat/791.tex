\begin{center}
\section*{غزل شماره ۷۹۱: گر یک شکر از لعلت در کار کنی حالی}
\label{sec:791}
\addcontentsline{toc}{section}{\nameref{sec:791}}
\begin{longtable}{l p{0.5cm} r}
گر یک شکر از لعلت در کار کنی حالی
&&
صد کافر منکر را دین‌دار کنی حالی
\\
ور زلف پریشان را درهم فکنی حلقه
&&
تسبیح همه مردان زنار کنی حالی
\\
روزی که ز گلزاری بی روی تو گل چینم
&&
گلزار ز چشم من گلزار کنی حالی
\\
چون دیدهٔ من هر دم گلبرگ رخت بیند
&&
از ناوک مژگانش پر خار کنی حالی
\\
صد گونه جفا داری چون روی مرا بینی
&&
بر من به جوانمردی ایثار کنی حالی
\\
صد بلعجبی دانی کابلیس نداند آن
&&
ما را چو زبون بینی در کار کنی حالی
\\
بردی دلم از من جان چون با تو کنم دعوی
&&
خود را عجمی سازی انکار کنی حالی
\\
چون صبح صبا زان‌رو در خاک کفت مالد
&&
کز بوی سر زلفش عطار کنی حالی
\\
\end{longtable}
\end{center}
