\begin{center}
\section*{غزل شماره ۲۶۷: هر زمانم عشق ماهی در کشاکش می‌کشد}
\label{sec:267}
\addcontentsline{toc}{section}{\nameref{sec:267}}
\begin{longtable}{l p{0.5cm} r}
هر زمانم عشق ماهی در کشاکش می‌کشد
&&
آتش سودای او جانم در آتش می‌کشد
\\
تا دل مسکین من در آتش حسنش فتاد
&&
گاه می‌سوزد چو عود و گه دمی خوش می‌کشد
\\
شحنهٔ سودای او شوریدگان عشق را
&&
هر نفس چون خونیان اندر کشاکش می‌کشد
\\
عشق را با هفت چرخ و شش جهت آرام نیست
&&
لاجرم نه بار هفت و نی غم شش می‌کشد
\\
جمع باید بود بر راهی چو موران روز و شب
&&
هر که را دل سوی آن زلف مشوش می‌کشد
\\
خاطر عطار از نور معانی در سخن
&&
آفتاب تیر بر چرخ منقش می‌کشد
\\
\end{longtable}
\end{center}
