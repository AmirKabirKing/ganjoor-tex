\begin{center}
\section*{غزل شماره ۸۴۱: ای راه تو را دراز نایی}
\label{sec:841}
\addcontentsline{toc}{section}{\nameref{sec:841}}
\begin{longtable}{l p{0.5cm} r}
ای راه تو را دراز نایی
&&
نه راه تو را سری نه پایی
\\
این راه دراز سالکان را
&&
کوته نکند مگر فنایی
\\
عاشق ز فنا چگونه ترسد
&&
چون عین فنا بود بقایی
\\
چون از تو نماند هیچ بر جای
&&
آنجاست اگر رسی بجایی
\\
ای دل بنشسته‌ای همه روز
&&
بر بوی وصال جانفزایی
\\
در لجهٔ بحر عشق جانت
&&
شد غرقه به بوی آشنایی
\\
دری که به هر دو کون ارزد
&&
دانی نرسد به ناسزایی
\\
هرگز دیدی که هیچ سلطان
&&
بر تخت نشست با گدایی
\\
هرگز دیدی که رند گلخن
&&
می خورد ز دست پادشایی
\\
ای دل خون خور که آن چنان ماه
&&
فارغ بود از غم چو مایی
\\
ای بس که من اندرین بیابان
&&
ره پیمودم ز تنگنایی
\\
دردا که ز اشتران راهش
&&
بانگی نشنیدم از درایی
\\
باری چه بدی که غول را هم
&&
دل خوش کندی به مرحبایی
\\
چون در خور صومعه نیم من
&&
اکنون منم و کلیسیایی
\\
در بسته چهار گوشه زنار
&&
از حلقهٔ زلف دلربایی
\\
بس پرگره است زلفش و هست
&&
زان هر گرهی گره‌گشایی
\\
گر خون دلم بریزد آن زلف
&&
خون‌ریزی اوست خون بهایی
\\
گر تو سر عین عشق داری
&&
دیری است که گفتمی صلایی
\\
ورنه ز درم برو که در پاش
&&
دادند نشان پارسایی
\\
عطار تو خویشتن نگه دار
&&
از آفت خویشتن نمایی
\\
\end{longtable}
\end{center}
