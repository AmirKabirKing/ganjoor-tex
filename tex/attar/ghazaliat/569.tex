\begin{center}
\section*{غزل شماره ۵۶۹: دل ندارم، صبر بی دل چون کنم}
\label{sec:569}
\addcontentsline{toc}{section}{\nameref{sec:569}}
\begin{longtable}{l p{0.5cm} r}
دل ندارم، صبر بی دل چون کنم
&&
صبر و دل در عشق حاصل چون کنم
\\
در بیابانی که پایان کس ندید
&&
کاروان بگذشت، منزل چون کنم
\\
همرهان رفتند و من بی روی و راه
&&
دست بر سر پای در گل چون کنم
\\
همچو مرغ نیم بسمل بال و پر
&&
می‌زنم تا خویش بسمل چون کنم
\\
بر امید قطره‌ای آب حیات
&&
نوش کردن زهر قاتل چون کنم
\\
چون دلم خون گشت و جان بر لب رسید
&&
چاره جان داروی دل چون کنم
\\
هر کسی گوید که این دردت ز چیست
&&
پیش دارم کار مشکل چون کنم
\\
مبتلا شد دل به جهل نفس شوم
&&
با بلای نفس جاهل چون کنم
\\
نفس، گرگ بد رگ است و سگ پرست
&&
همچو روح‌القدس عاقل چون کنم
\\
ناقصی کو در دم خر می‌زید
&&
از دم عیسیش کامل چون کنم
\\
مدبری کز جرعه دردی خوش است
&&
از می معنیش مقبل چون کنم
\\
چون ز غفلت درد من از حد گذشت
&&
داروی عطار غافل چون کنم
\\
\end{longtable}
\end{center}
