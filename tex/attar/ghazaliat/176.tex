\begin{center}
\section*{غزل شماره ۱۷۶: قد تو به آزادی بر سرو چمن خندد}
\label{sec:176}
\addcontentsline{toc}{section}{\nameref{sec:176}}
\begin{longtable}{l p{0.5cm} r}
قد تو به آزادی بر سرو چمن خندد
&&
خط تو به سرسبزی بر مشک ختن خندد
\\
تا یاد لبت نبود گلهای بهاری را
&&
حقا که اگر هرگز یک گل ز چمن خندد
\\
از عکس تو چون دریا از موج برآرد دم
&&
یاقوت و گهر بارد بر در عدن خندد
\\
گر کشته شود عاشق از دشنهٔ خونریزت
&&
در روی تو همچون گل از زیر کفن خندد
\\
چه حیله نهم برهم چون لعل شکربارت
&&
چندان که کنم حیله بر حیلهٔ من خندد
\\
تو هم‌نفس صبحی زیرا که خدا داند
&&
تا حقهٔ پر درت هرگز به دهن خندد
\\
من هم‌نفس شمعم زیرا که لب و چشمم
&&
بر فرقت جان گرید بر گریهٔ تن خندد
\\
عطار چو در چیند از حقهٔ پر درت
&&
در جنب چنان دری بر در سخن خندد
\\
\end{longtable}
\end{center}
