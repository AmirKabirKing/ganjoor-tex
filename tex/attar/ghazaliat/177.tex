\begin{center}
\section*{غزل شماره ۱۷۷: عاشق تو جان مختصر که پسندد}
\label{sec:177}
\addcontentsline{toc}{section}{\nameref{sec:177}}
\begin{longtable}{l p{0.5cm} r}
عاشق تو جان مختصر که پسندد
&&
فتنه تو عقل بی خبر که پسندد
\\
روی تو کز ترک آفتاب دریغ است
&&
در نظر هندوی بصر که پسندد
\\
روی تو را تاب قوت نظری نیست
&&
در رخ تو تیزتر نظر که پسندد
\\
چون بنگنجد شکر برون ز دهانت
&&
از لب تو خواستن شکر که پسندد
\\
چون نتوان بی کمر میان تو دیدن
&&
موی میان تو را کمر که پسندد
\\
چون به کمان برنهی خدنگ جگردوز
&&
پیش تو جز جان خود سپر که پسندد
\\
چون به جفا تیغت از نیام برآری
&&
در همه عالم حدیث سر که پسندد
\\
چون غم عشقت به جان خرند و به ارزد
&&
در غم تو حیله و حذر که پسندد
\\
تا غم عشق تو هست در همه عالم
&&
هیچ دلی را غمی دگر که پسندد
\\
وصل تو جستم به نیم جان محقر
&&
وصل تو آخر بدین قدر که پسندد
\\
هر سحر از عشق تو بسا که بسوزم
&&
سوز چو من شمع هر سحر که پسندد
\\
چون تو جگر گوشهٔ دل منی آخر
&&
قوت من از گوشهٔ جگر که پسندد
\\
شد دل عطار پاره پاره ز شوقت
&&
کار دل او ازین بتر که پسندد
\\
\end{longtable}
\end{center}
