\begin{center}
\section*{غزل شماره ۵۸۶: ما به عهد حسن تو ترک دل و جان گفته‌ایم}
\label{sec:586}
\addcontentsline{toc}{section}{\nameref{sec:586}}
\begin{longtable}{l p{0.5cm} r}
ما به عهد حسن تو ترک دل و جان گفته‌ایم
&&
با رخ و زلف تو شرح کفر و ایمان گفته‌ایم
\\
یاد زلفت کرده‌ایم و نام زلفت برده‌ایم
&&
هم پریشان گشته‌ایم و هم پریشان گفته‌ایم
\\
تا تو جان از بس لطیفی در نیابد کس تو را
&&
ما تو را از استعارت در سخن جان گفته‌ایم
\\
همچو من در عشقت ای جان ترک جان‌ها گفته‌اند
&&
تا به جانبازان عالم وصف جانان گفته‌ایم
\\
درد عشقت را چو درمانی نمی‌دیدیم ما
&&
درد را تسکین دل را عین درمان گفته‌ایم
\\
وصل و هجران با تو و از تو خیال عشق توست
&&
قرب و بعد خویشتن را وصل و هجران گفته‌ایم
\\
چون سر و سامان حجاب راهت آمد در رهت
&&
از سر سر رفته‌ایم و ترک سامان گفته‌ایم
\\
با خیالت چون یکی محرم نمی‌دیدیم ما
&&
داستان عشق خود را تا به پایان گفته‌ایم
\\
خویشتن را در میان قبض و بسط و صحو سکر
&&
گه گدا را خوانده‌ایم و گاه سلطان گفته‌ایم
\\
مرد وصلت نیست کس بشنو درین معنی که ما
&&
بس دلیل آورده‌ایم و چند برهان گفته‌ایم
\\
گرچه عطاریم ما کاسرار راه عشق تو
&&
گاه پیدا کرده‌ایم و گاه پنهان گفته‌ایم
\\
\end{longtable}
\end{center}
