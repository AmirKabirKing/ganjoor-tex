\begin{center}
\section*{غزل شماره ۳۷۵: تا خطت آمد به شبرنگی پدید}
\label{sec:375}
\addcontentsline{toc}{section}{\nameref{sec:375}}
\begin{longtable}{l p{0.5cm} r}
تا خطت آمد به شبرنگی پدید
&&
فتنه شد از چند فرسنگی پدید
\\
چون ز تنگت نیست رایج یک شکر
&&
جان کجا آید ز دلتنگی پدید
\\
پیش خورشید رخت چون ذره‌ای
&&
عقل ناید از سبک سنگی پدید
\\
در زمستان روی چون گل جلوه کن
&&
تا کند بلبل خوش آهنگی پدید
\\
خون من خوردست چشم شنگ تو
&&
چشم تو تا کی کند شنگی پدید
\\
بی تو عمری صبر کردم وین زمان
&&
اسب صبرم می‌کند لنگی پدید
\\
می‌کشم خواری رنگارنگ تو
&&
آخر آید بو که یک رنگی پدید
\\
طفلکی‌ام هندوی وصلت مکن
&&
هجر را بر صورت زنگی پدید
\\
گر شود عطار خاکت آفتاب
&&
بر درش آید به سرهنگی پدید
\\
\end{longtable}
\end{center}
