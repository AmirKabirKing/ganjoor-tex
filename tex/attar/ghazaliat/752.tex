\begin{center}
\section*{غزل شماره ۷۵۲: از من بی خبر چه می‌طلبی}
\label{sec:752}
\addcontentsline{toc}{section}{\nameref{sec:752}}
\begin{longtable}{l p{0.5cm} r}
از من بی خبر چه می‌طلبی
&&
سوختم خشک و تر چه می‌طلبی
\\
گر چه شهباز معرفت بودم
&&
ریختم بال و پر چه می‌طلبی
\\
در دو عالم ز هرچه بود و نبود
&&
بگسستم دگر چه می‌طلبی
\\
مانده‌ام همچو گوی در ره تو
&&
گم شده پا و سر چه می‌طلبی
\\
من آشفته را ز عشق رخت
&&
هر دم آشفته‌تر چه می‌طلبی
\\
پیش طرف کلاه گوشهٔ تو
&&
کرده‌ام جان کمر چه می‌طلبی
\\
گفته‌ای درد تو همی طلبم
&&
درد ازین بیشتر چه می‌طلبی
\\
با دلی پر ز درد تو شب و روز
&&
شده‌ام نوحه‌گر چه می‌طلبی
\\
بی خبر مانده‌ام ز مستی عشق
&&
هستت آخر خبر چه می‌طلبی
\\
پرده برگیر و بیش ازین آخر
&&
پردهٔ من مدر چه می‌طلبی
\\
چند باشم نه دل نه جان بی تو
&&
راندهٔ در بدر چه می‌طلبی
\\
بی تو عطار را روا نبود
&&
خون گرفته جگر چه می‌طلبی
\\
\end{longtable}
\end{center}
