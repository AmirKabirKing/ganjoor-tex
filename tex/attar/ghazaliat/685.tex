\begin{center}
\section*{غزل شماره ۶۸۵: ای دل و جان کاملان، گم شده در کمال تو}
\label{sec:685}
\addcontentsline{toc}{section}{\nameref{sec:685}}
\begin{longtable}{l p{0.5cm} r}
ای دل و جان کاملان، گم شده در کمال تو
&&
عقل همه مقربان، بی خبر از وصال تو
\\
جمله تویی به خود نگر جمله ببین که دایما
&&
هجده هزار عالم است آینهٔ جمال تو
\\
تا دل طالبانت را از تو دلالتی بود
&&
هرچه که هست در جهان هست همه مثال تو
\\
جملهٔ اهل دیده را از تو زبان ز کار شد
&&
نیست مجال نکته‌ای در صفت کمال تو
\\
چرخ رونده قرن‌ها بی سر و پای در رهت
&&
پشت خمیده می‌رود در غم گوشمال تو
\\
تا ابدش نشان و نام از دو جهان بریده شد
&&
هر که دمی جلاب خورد از قدح جلال تو
\\
مانده‌اند دور دور اهل دو کون از رهت
&&
زانکه وجود گم کند خلق در اتصال تو
\\
خشک شدیم بر زمین پرده ز روی برفکن
&&
تا لب خشک عاشقان تر شود از زلال تو
\\
گرچه فرید در جهان هست فصیح‌تر کسی
&&
رد مکنش که در سخن هست زبانش لال تو
\\
\end{longtable}
\end{center}
