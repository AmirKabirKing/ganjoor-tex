\begin{center}
\section*{غزل شماره ۹۲: شیر در کار عشق مسکین است}
\label{sec:092}
\addcontentsline{toc}{section}{\nameref{sec:092}}
\begin{longtable}{l p{0.5cm} r}
شیر در کار عشق مسکین است
&&
عشق را بین که با چه تمکین است
\\
نکشد کس کمان عشق به زور
&&
عشق شاه همه سلاطین است
\\
دلم از دلبران بتی بگزید
&&
کو به رخ همچو ماه و پروین است
\\
از لطیفی که هست آن دلبر
&&
فخر خوبان چین و ماچین است
\\
وصف خوبی او چه دانم گفت
&&
هرچه گویم هزار چندین است
\\
خوب رویی شگرف گفتاری
&&
که به صورت فرشته آیین است
\\
آن نگاری که روی او قمر است
&&
طره‌اش مشک عنبرآگین است
\\
من چو فرهاد در غمش زارم
&&
کو به حسن و جمال شیرین است
\\
صفتش در زمانه ممتاز است
&&
دیدنش روح را جهان بین است
\\
آن ستم کز صنم کشید فرید
&&
بی‌گمان آفت دل و دین است
\\
\end{longtable}
\end{center}
