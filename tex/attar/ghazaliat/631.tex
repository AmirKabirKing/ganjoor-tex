\begin{center}
\section*{غزل شماره ۶۳۱: ای روی تو شمع بت‌پرستان}
\label{sec:631}
\addcontentsline{toc}{section}{\nameref{sec:631}}
\begin{longtable}{l p{0.5cm} r}
ای روی تو شمع بت‌پرستان
&&
یاقوت تو قوت تنگدستان
\\
زلف تو و صد هزار حلقه
&&
چشم تو و صد هزار دستان
\\
خورشید نهاده چشم بر در
&&
تا تو به درآیی از شبستان
\\
گردون به هزار چشم هر شب
&&
واله شده در تو همچو مستان
\\
آنچ از رخ تو رود در اسلام
&&
هرگز نرود به کافرستان
\\
پیران ره حروف زلفت
&&
ابجد خوانان این دبستان
\\
در عشق تو نیستان که هستند
&&
هستند نه نیستان نه هستان
\\
ممکن نبود به لطف تو خلق
&&
از دینداران و بت‌پرستان
\\
گوی تو که آب خضر بوده است
&&
هر شیر که خورده‌ای ز پستان
\\
ای بر شده بس بلند آخر
&&
به زین نگرید سوی بستان
\\
گلگون جمال در جهان تاز
&&
وز عمر رونده داد بستان
\\
کین گلبن نوبهار عمرت
&&
درهم ریزد به یک زمستان
\\
مشغول مشو به گل که ماراست
&&
پنهان ز تو خفته در گلستان
\\
زخمی زندت به چشم زخمی
&&
گورستانت کند ز بستان
\\
تو گلبن گلستان حسنی
&&
عطار تورا هزاردستان
\\
\end{longtable}
\end{center}
