\begin{center}
\section*{غزل شماره ۳۷۶: در ره عشق تو پایان کس ندید}
\label{sec:376}
\addcontentsline{toc}{section}{\nameref{sec:376}}
\begin{longtable}{l p{0.5cm} r}
در ره عشق تو پایان کس ندید
&&
راه بس دور است و پیشان کس ندید
\\
گرد کویت چون تواند دید کس
&&
زانکه تو در جانی و جان کس ندید
\\
از نهانی کس ندیدت آشکار
&&
وز هویداییت پنهان کس ندید
\\
بلعجب دردی است دردت کاندرو
&&
تا قیامت روی درمان کس ندید
\\
در خرابات خراب عشق تو
&&
یک حریف آب دندان کس ندید
\\
گوهر وصلت از آن در پرده ماند
&&
کز جهان شایستهٔ آن‌کس ندید
\\
در بیابانت ز چندین سوخته
&&
یک نشان از صد هزاران کس ندید
\\
بس دل شوریده کاندر راه عشق
&&
جان بداد و روی جانان کس ندید
\\
جمله در راهت فرو رفته به خاک
&&
بوالعجب تر زین بیابان کس ندید
\\
خون خور ای عطار و تن در صبر ده
&&
کانچه می‌جویی تو آسان کس ندید
\\
\end{longtable}
\end{center}
