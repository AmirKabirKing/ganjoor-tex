\begin{center}
\section*{غزل شماره ۴۱۶: در عشق روی او ز حدوث و قدم مپرس}
\label{sec:416}
\addcontentsline{toc}{section}{\nameref{sec:416}}
\begin{longtable}{l p{0.5cm} r}
در عشق روی او ز حدوث و قدم مپرس
&&
گر مرد عاشقی ز وجود و عدم مپرس
\\
مردانه بگذر از ازل و از ابد تمام
&&
کم گوی از ازل ز ابد نیز هم مپرس
\\
زین چار رکن چون بگذشتی حرم ببین
&&
وانگاه دیده برکن و نیز از حرم مپرس
\\
آنجا که نیست هستی توحید، هیچ نیست
&&
زانجای درگذر به دمی و ز دم مپرس
\\
لوح و قلم به قطع دماغ و زبان توست
&&
لوح و قلم بدان و ز لوح و قلم مپرس
\\
کرسی است سینهٔ تو و عرش است دل درو
&&
وین هر دو نیست جز رقمی وز رقم مپرس
\\
چون تو بدین مقام رسیدی دگر مباش
&&
گم گرد در فنا و دگر بیش و کم مپرس
\\
یک ذره سایه باش تو اینجا در آفتاب
&&
اینجا چو تو نه‌ای تو ز شادی و غم مپرس
\\
هر چیز کان تو فهم کنی آن همه تویی
&&
پس تا که تو تویی ز حدوث و قدم مپرس
\\
عطار اگر رسیدی اینجایگاه تو
&&
در لذت حقیقت خود از الم مپرس
\\
\end{longtable}
\end{center}
