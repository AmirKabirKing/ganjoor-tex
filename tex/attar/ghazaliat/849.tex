\begin{center}
\section*{غزل شماره ۸۴۹: ترسا بچه‌ایم افکند از زهد به ترسایی}
\label{sec:849}
\addcontentsline{toc}{section}{\nameref{sec:849}}
\begin{longtable}{l p{0.5cm} r}
ترسا بچه‌ایم افکند از زهد به ترسایی
&&
اکنون من و زناری در دیر به تنهایی
\\
دی زاهد دین بودم سجاده نشین بودم
&&
ز ارباب یقین بودم سر دفتر دانایی
\\
امروز دگر هستم دردی کشم و مستم
&&
در بتکده بنشستم دین داده به ترسایی
\\
نه محرم ایمانم نه کفر همی دانم
&&
نه اینم و نه آنم تن داده به رسوایی
\\
دوش از غم فکر و دین یعنی که نه آن نه این
&&
بنشسته بدم غمگین شوریده و سودایی
\\
ناگه ز درون جان در داد ندا جانان
&&
کای عاشق سرگردان تا چند ز رعنایی
\\
روزی دو سه گر از ما گشتی تو چنین تنها
&&
باز آی سوی دریا تو گوهر دریایی
\\
پس گفت در این معنی نه کفر نه دین اولی
&&
برتو شو ازین دعوی گر سوختهٔ مایی
\\
هرچند که پر دردی کی محرم ما گردی
&&
فانی شو اگر مردی تا محرم ما آیی
\\
عطار چه دانی تو وین قصه چه خوانی تو
&&
گر هیچ نمانی تو اینجا شوی آنجایی
\\
\end{longtable}
\end{center}
