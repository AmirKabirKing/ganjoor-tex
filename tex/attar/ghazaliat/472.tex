\begin{center}
\section*{غزل شماره ۴۷۲: روی تو در حسن چنان دیده‌ام}
\label{sec:472}
\addcontentsline{toc}{section}{\nameref{sec:472}}
\begin{longtable}{l p{0.5cm} r}
روی تو در حسن چنان دیده‌ام
&&
کاینهٔ هر دو جهان دیده‌ام
\\
جمله از آن آینه پیدا نمود
&&
واینه از جمله نهان دیده‌ام
\\
هست در آیینه نشان صد هزار
&&
واینه فارغ ز نشان دیده‌ام
\\
صورت در آینه از آینه
&&
نیست خبردار چنان دیده‌ام
\\
جمله درین آینه جلوه‌گرند
&&
واینه را حافظ آن دیده‌ام
\\
صورت آن آینه چون جسم بود
&&
پرتو آن آینه جان دیده‌ام
\\
جوهر آن آینه چون کس ندید
&&
من چه زنم دم که عیان دیده‌ام
\\
لیک کسی را ز چنان جوهری
&&
هیچ نه شرح و نه بیان دیده‌ام
\\
جملهٔ ذرات ازو بر کنار
&&
با همه او را به میان دیده‌ام
\\
یافته‌ام از همه بس فارغش
&&
پس همه را کرده ضمان دیده‌ام
\\
با تو و بی تو چه دهم شرح این
&&
چون به ندانم که چه سان دیده‌ام
\\
یک همه دان در دو جهان کس ندید
&&
چون دو جهان یک همه دان دیده‌ام
\\
جملهٔ مردان جهان دیده را
&&
در غم این نعره‌زنان دیده‌ام
\\
دایم ازین واقعه عطار را
&&
نوحه‌گری اشک فشان دیده‌ام
\\
\end{longtable}
\end{center}
