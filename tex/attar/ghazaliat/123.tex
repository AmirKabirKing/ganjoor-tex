\begin{center}
\section*{غزل شماره ۱۲۳: هر دیده که بر تو یک نظر داشت}
\label{sec:123}
\addcontentsline{toc}{section}{\nameref{sec:123}}
\begin{longtable}{l p{0.5cm} r}
هر دیده که بر تو یک نظر داشت
&&
از عمر تمام بهره برداشت
\\
سرمایهٔ عمر دیدن توست
&&
وان دید تو را که یک نظر داشت
\\
کور است کسی که هر زمانی
&&
در دید تو دیدهٔ دگر داشت
\\
جاوید ز خویش بی‌خبر شد
&&
هر دل که ز عشق تو خبر داشت
\\
مرغی بپرید در هوایت
&&
کز شوق تو صد هزار پر داشت
\\
در شوق رخ تو بیشتر سوخت
&&
هر کو به تو قرب بیشتر داشت
\\
دل بی رخ تو دمی سر کس
&&
سوگند به جان تو اگر داشت
\\
در عشق رخ تو یک سر موی
&&
ننهاد قدم کسی که سر داشت
\\
بس مرده که زنده کرد در حال
&&
بادی که به کوی تو گذر داشت
\\
با چشم تو کارگر نیامد
&&
هر حیله که چرخ پاک برداشت
\\
خوارم کردی چنان که عشقت
&&
بر خاک درم چو خاک در داشت
\\
خوار از چه سبب کنی کسی را
&&
کز جان خودت عزیزتر داشت
\\
با بوالعجبی غمزهٔ تو
&&
نه دل قیمت نه جان خطر داشت
\\
در پیش لبت ز شرم بگداخت
&&
هر شیرینی که آن شکر داشت
\\
در جنب لب تو آب حیوان
&&
هر شیوه که داشت مختصر داشت
\\
در نقرهٔ عارضت فروشد
&&
هر نازکییی که آب زر داشت
\\
بر گرد میان تو کمر گشت
&&
آن حرف که در میان کمر داشت
\\
شکل دهن تو طرفه برخاست
&&
زان نقطهٔ طرفه بر زبر داشت
\\
چون روی تو زیر پردهٔ زلف
&&
چه صد که هزار پرده در داشت
\\
در هر بن موی بی رخ تو
&&
عطار هزار نوحه‌گر داشت
\\
\end{longtable}
\end{center}
