\begin{center}
\section*{غزل شماره ۲۳۲: اگر ز پیش جمالت نقاب برخیزد}
\label{sec:232}
\addcontentsline{toc}{section}{\nameref{sec:232}}
\begin{longtable}{l p{0.5cm} r}
اگر ز پیش جمالت نقاب برخیزد
&&
ز ذره ذره هزار آفتاب برخیزد
\\
جهان ز فتنهٔ بیدار رستخیز شود
&&
چو چشم نیم‌خمارش ز خواب برخیزد
\\
به مجلسی که زند خنده لعل میگونش
&&
خرد اگر بنشیند خراب برخیزد
\\
اگر به خنده در آید لبش ز هر سویی
&&
هزار نعره‌زن بی شراب برخیزد
\\
زمرد خط تو چون ز لعل برجوشد
&&
هزار جوش ز لعل خوشاب برخیزد
\\
ز بس که بوی گل عارضش عرق گیرد
&&
ز خار رشک، خروش از گلاب برخیزد
\\
ز بس که اهل جهان را چو صور دم دهد او
&&
قیامتی از جهان خراب برخیزد
\\
جنابتی که ز دعوی عشق او بنشست
&&
چو غسل سازی از خون ناب برخیزد
\\
که آن چنان حدثی تا که تو نگریی خون
&&
گمان مبر که به دریای آب برخیزد
\\
خبر کراست که از بهر تف هر جگری
&&
ز زلف مشک فشانش چه تاب برخیزد
\\
نشان کراست که از بهر غارت دو جهان
&&
ز آفتاب رخش کی نقاب برخیزد
\\
اگر ادا کند از لفظ خویش شعر فرید
&&
ز پیش چشمهٔ حیوان حجاب برخیزد
\\
\end{longtable}
\end{center}
