\begin{center}
\section*{غزل شماره ۱۰۵: سخن عشق جز اشارت نیست}
\label{sec:105}
\addcontentsline{toc}{section}{\nameref{sec:105}}
\begin{longtable}{l p{0.5cm} r}
سخن عشق جز اشارت نیست
&&
عشق در بند استعارت نیست
\\
دل شناسد که چیست جوهر عشق
&&
عقل را ذره‌ای بصارت نیست
\\
در عبارت همی نگنجد عشق
&&
عشق از عالم عبارت نیست
\\
هر که را دل ز عشق گشت خراب
&&
بعد از آن هرگزش عمارت نیست
\\
عشق بستان و خویشتن بفروش
&&
که نکوتر ازین تجارت نیست
\\
گر شود فوت لحظه‌ای بی عشق
&&
هرگز آن لحظه را کفارت نیست
\\
دل خود را ز گور نفس برآر
&&
که دلت را جز این زیارت نیست
\\
تن خود را به خون دیده بشوی
&&
که تنت را جز این طهارت نیست
\\
پر شد از دوست هر دو کون ولیک
&&
سوی او زهرهٔ اشارت نیست
\\
دل شوریدگان چو غارت کرد
&&
بانگ بر زد که جای غارت نیست
\\
تن در این کار در ده ای عطار
&&
زانکه این کار ما حقارت نیست
\\
\end{longtable}
\end{center}
