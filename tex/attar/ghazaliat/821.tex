\begin{center}
\section*{غزل شماره ۸۲۱: ای دل اندر عشق غوغا چون کنی}
\label{sec:821}
\addcontentsline{toc}{section}{\nameref{sec:821}}
\begin{longtable}{l p{0.5cm} r}
ای دل اندر عشق غوغا چون کنی
&&
خویش را بیهوده رسوا چون کنی
\\
آنچه کل خلق نتوانست کرد
&&
تو محال‌اندیش تنها چون کنی
\\
دم مزن خون می‌خور و صفرا مکن
&&
پشه‌ای با باد صفرا چون کنی
\\
تو همی خواهی که دانی سر عشق
&&
کس بدین سر نیست دانا چون کنی
\\
چون تو اندر عشق او پنهان شدی
&&
سر عشقش آشکارا چون کنی
\\
چون تبرا نیستت از خویشتن
&&
پس به عشق او تولا چون کنی
\\
عشق را سرمایه‌ای باید شگرف
&&
پس تو بی سرمایه سودا چون کنی
\\
چون تو را هر دم حجابی دیگر است
&&
چشم جان خویش بینا چون کنی
\\
چون به یک قطره دلت قانع ببود
&&
جان خود را کل دریا چون کنی
\\
غرق دریا گرد و ناپیدا بباش
&&
خویش را زین بیش پیدا چون کنی
\\
چون تو سایه باشی و او آفتاب
&&
پیش او خود را هویدا چون کنی
\\
هر که او پیداست درصد تفرقه است
&&
چون نباشی جمع آنجا چون کنی
\\
چون نکردی خویش را امروز جمع
&&
می‌ندانم تا که فردا چون کنی
\\
مذهب عطار گیر و نیست شو
&&
هستی خود را محابا چون کنی
\\
\end{longtable}
\end{center}
