\begin{center}
\section*{غزل شماره ۵۷۱: دل ز دستم رفت و جان هم، بی دل و جان چون کنم}
\label{sec:571}
\addcontentsline{toc}{section}{\nameref{sec:571}}
\begin{longtable}{l p{0.5cm} r}
دل ز دستم رفت و جان هم، بی دل و جان چون کنم
&&
سر عشقم آشکارا گشت پنهان چون کنم
\\
هرکسم گوید که درمانی کن آخر درد را
&&
چون به دردم دایما مشغول درمان چون کنم
\\
چون خروشم بشنود هر بی خبر گوید خموش
&&
می‌تپد دل در برم می‌سوزدم جان چون کنم
\\
عالمی در دست من، من همچو مویی در برش
&&
قطره‌ای خون است دل، در زیر طوفان چون کنم
\\
در تموزم مانده جان خسته و تن تب زده
&&
وآنگهم گویند براین ره به پایان چون کنم
\\
چون ندارم یک نفس اهلیت صف النعال
&&
پیشگه چون جویم و آهنگ پیشان چون کنم
\\
در بن هر موی صد بت بیش می‌بینم عیان
&&
در میان این همه بت عزم ایمان چون کنم
\\
نه ز ایمانم نشانی نه ز کفرم رونقی
&&
در میان این و آن درمانده حیران چون کنم
\\
چون نیامد از وجودم هیچ جمعیت پدید
&&
بیش ازین عطار را از خود پریشان چون کنم
\\
\end{longtable}
\end{center}
