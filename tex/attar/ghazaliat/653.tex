\begin{center}
\section*{غزل شماره ۶۵۳: گر سر این کار داری کار کن}
\label{sec:653}
\addcontentsline{toc}{section}{\nameref{sec:653}}
\begin{longtable}{l p{0.5cm} r}
گر سر این کار داری کار کن
&&
ور نه‌ای این کار را انکار کن
\\
خلق عالم جمله مست غفلتند
&&
مست منگر خویش را هشیار کن
\\
چون بدانستی و دیدی خویش را
&&
تا بمیری روی در دیوار کن
\\
گر طمع داری وصال آفتاب
&&
ذره‌ای این شیوه را اقرار کن
\\
گر ز تو یک ذره باقی مانده است
&&
خرقه و تسبیح با زنار کن
\\
با منی شرک است استغفار تو
&&
پس ز استغفار استغفار کن
\\
یار بیزار است از تو تا تویی
&&
اول از خود خویش را بیزار کن
\\
گر جمال یار می‌خواهی عیان
&&
چشم در خورد جمال یار کن
\\
نیست پنهان آفتاب لایزال
&&
تو چو ذره خویش را ایثار کن
\\
تا ابد هم از عدم هم از وجود
&&
دیده بر دوز آنگهی دیدار کن
\\
چند گردی گرد عالم بی خبر
&&
دل سرای خلوت دلدار کن
\\
در درج عشق بر طاق دل است
&&
مرد دل شو جمع‌گرد و کار کن
\\
نقطهٔ توحید با جان در میان است
&&
گرد جان برگرد و چون پرگار کن
\\
چون فرو رفتی به قعر بحر جان
&&
عزم خلوتخانهٔ اسرار کن
\\
درس اسرار است نقش جان تو
&&
در نه تعلیق و نه تکرار کن
\\
پس چه کن در لوح جان خود نگر
&&
پس زبان در نطق گوهربار کن
\\
گر کسی را اهل بینی باز گوی
&&
ورنه درج نطق را مسمار کن
\\
ور به ترک هر دو عالم گفته‌ای
&&
ذره‌ای مندیش و چون عطار کن
\\
\end{longtable}
\end{center}
