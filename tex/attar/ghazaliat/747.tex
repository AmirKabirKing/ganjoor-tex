\begin{center}
\section*{غزل شماره ۷۴۷: آن را که نیست در دل ازین سر سکینه‌ای}
\label{sec:747}
\addcontentsline{toc}{section}{\nameref{sec:747}}
\begin{longtable}{l p{0.5cm} r}
آن را که نیست در دل ازین سر سکینه‌ای
&&
نبود کم از کم و بود از کم کمینه‌ای
\\
خواهی که از قرینه بدانی که عشق چیست
&&
ناخورده می ز عشق ندانی قرینه‌ای
\\
در دار ملک عشق خلیفه کسی بود
&&
کو را بود ز در حقیقت خزینه‌ای
\\
مرغی است جان عاشق و چندانش حوصله
&&
کز هر دو کون لایق او نیست چینه‌ای
\\
شه‌بیت سر عشق که مطلوب جمله اوست
&&
بیتی است بس عجب مطلب از سفینه‌ای
\\
عمری ز عرش و فرش طلب کردی این حدیث
&&
چل روز نیز واطلب از قعر سینه‌ای
\\
در عشق اگر سکینه پدید آیدت نکوست
&&
لیکن به زهد هیچ نیرزد سکینه‌ای
\\
طوفان عشق چون ز پس و پیش در رسد
&&
جز در درون سینه نیابی سفینه‌ای
\\
ای ساقی امشب از سر این جمع برمخیز
&&
هر لحظه پر کن از می دوشین قنینه‌ای
\\
چندان شراب ده تو که با منکر و مقر
&&
در سینه‌ای نه مهر بماند نه کینه‌ای
\\
بشکن پیاله بر در زهاد تا مگر
&&
در پای زاهدی شکند آبگینه‌ای
\\
عطار در بقای حق و در فنای خود
&&
چون بوسعیدمهنه نیابی مهینه‌ای
\\
\end{longtable}
\end{center}
