\begin{center}
\section*{غزل شماره ۵۲۸: چون من ز همه عالم ترسا بچه‌ای دارم}
\label{sec:528}
\addcontentsline{toc}{section}{\nameref{sec:528}}
\begin{longtable}{l p{0.5cm} r}
چون من ز همه عالم ترسا بچه‌ای دارم
&&
دانم که ز ترسایی هرگز نبود عارم
\\
تا زلف چو زنارش دیدم به کنار مه
&&
پیوسته میان خود بربسته به زنارم
\\
تا از شکن زلفش شد کشف مرا صد سر
&&
برخاست ز پیش دل اقرارم و انکارم
\\
هر لحظه به رغم من در زلف دهد تابی
&&
با تاب چنان زلفی من تاب نمی‌آرم
\\
چون از سر هر مویش صد فتنه فرو بارد
&&
از هر مژه طوفانی چون ابر فروبارم
\\
آن رفت که می‌آمد از دست مرا کاری
&&
اکنون چو سر زلفش، از دست بشد کارم
\\
هر شب ز فراق او چون شمع همی سوزم
&&
واو بر صفت شمعی هر روز کشد زارم
\\
گفتم به جز از عشوه چیزی نفروشی تو
&&
بفروخت جهان بر من زیرا که خریدارم
\\
نه در صف درویشی شایستهٔ آن ماهم
&&
نه در ره ترسایی اهلیت او دارم
\\
نه مرد مناجاتم نه رند خراباتم
&&
نه محرم محرابم نه در خور خمارم
\\
نه مؤمن توحیدم نه مشرک تقلیدم
&&
نه منکر تحقیقم نه واقف اسرارم
\\
از بس که چو کرم قز بر خویش تنم پرده
&&
پیوسته چو کردم قز در پردهٔ پندارم
\\
از زحمت عطارم بندی است قوی در ره
&&
کو کس که کند فارغ از زحمت عطارم
\\
\end{longtable}
\end{center}
