\begin{center}
\section*{غزل شماره ۱۶۱: هر شب دل پر خونم بر خاک درت افتد}
\label{sec:161}
\addcontentsline{toc}{section}{\nameref{sec:161}}
\begin{longtable}{l p{0.5cm} r}
هر شب دل پر خونم بر خاک درت افتد
&&
تا بو که چو روز آید بر وی گذرت افتد
\\
کار دو جهان من جاوید نکو گردد
&&
گر بر من سرگردان یک دم نظرت افتد
\\
از دست چو من عاشق دانی که چه برخیزد
&&
کاید به سر کویت در خاک درت افتد
\\
گر عاشق روی خود سرگشته همی خواهی
&&
حقا که اگر از من سرگشته‌ترت افتد
\\
این است گناه من کت دوست همی دارم
&&
خطی به گناه من درکش اگرت افتد
\\
دانم که بدت افتد زیرا که دلم بردی
&&
ور در تو رسد آهم از بد بترت افتد
\\
گر تو همه سیمرغی از آه دلم می‌ترس
&&
کاتش ز دلم ناگه در بال و پرت افتد
\\
خون جگرم خوردی وز خویش نپرسیدی
&&
آخر چکنی جانا گر بر جگرت افتد
\\
پا بر سر درویشان از کبر منه یارا
&&
در طشت فنا روزی بی تیغ سرت افتد
\\
بیچاره من مسکین در دست تو چون مومم
&&
بیچاره تو گر روزی مردی به سرت افتد
\\
هش دار که این ساعت طوطی خط سبزت
&&
می‌آید و می‌جوشد تا بر شکرت افتد
\\
گفتی شکری بخشم عطار سبک دل را
&&
این بر تو گران آید رایی دگرت افتد
\\
\end{longtable}
\end{center}
