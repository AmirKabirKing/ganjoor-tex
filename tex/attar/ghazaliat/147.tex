\begin{center}
\section*{غزل شماره ۱۴۷: صبح دم زد ساقیا هین الصبوح}
\label{sec:147}
\addcontentsline{toc}{section}{\nameref{sec:147}}
\begin{longtable}{l p{0.5cm} r}
صبح دم زد ساقیا هین الصبوح
&&
خفتگان را در قدح کن قوت روح
\\
در قدح ریز آب خضر از جام جم
&&
باز نتوان گشت ازین در بی فتوح
\\
توبه بشکن تا درست آیی ز کار
&&
چند گویی توبه‌ای دارم نصوح
\\
مطربا قولی بگو از راهوی
&&
راه راه راهوی است اندر صبوح
\\
دل ز مستی قول کس می‌نشنود
&&
زانکه بشنوده است قول بوالفتوح
\\
چون سرانجام تو طوفان بلاست
&&
عمر تو چه یک نفس چه عمر نوح
\\
گر ز عطار این سخن می‌نشنوی
&&
بشنو از مرغ سحر صور صلوح
\\
\end{longtable}
\end{center}
