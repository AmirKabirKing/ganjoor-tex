\begin{center}
\section*{غزل شماره ۲۸۳: مستغرقی که از خود هرگز به سر نیامد}
\label{sec:283}
\addcontentsline{toc}{section}{\nameref{sec:283}}
\begin{longtable}{l p{0.5cm} r}
مستغرقی که از خود هرگز به سر نیامد
&&
صد ره بسوخت هر دم دودی به در نیامد
\\
گفتم که روی او را روزی سپند سوزم
&&
زیرا که از چو من کس کاری دگر نیامد
\\
چون نیک بنگرستم آن روی بود جمله
&&
از روی او سپندی کس را به سر نیامد
\\
جانان چو رخ نمودی هرجا که بود جانی
&&
فانی شدند جمله وز کس خبر نیامد
\\
آخر سپند باید بهر چنان جمالی
&&
دردا که هیچ کس را این کار برنیامد
\\
پیش تو محو گشتند اول قدم همه کس
&&
هرگز دوم قدم را یک راهبر نیامد
\\
چون گام اول از خود جمله شدند فانی
&&
کس را به گام دیگر رنج گذر نیامد
\\
ما سایه و تو خورشید آری شگفت نبود
&&
خورشید سایه‌ای را گر در نظر نیامد
\\
که سر نهاد روزی بر پای درد عشقت
&&
تا در رهت چو گویی بی پا و سر نیامد
\\
که گوشهٔ جگر خواند او از میان جانت
&&
تا از میان جانش بوی جگر نیامد
\\
چندان که برگشادم بر دل در معانی
&&
عطار را از آن در جز دردسر نیامد
\\
\end{longtable}
\end{center}
