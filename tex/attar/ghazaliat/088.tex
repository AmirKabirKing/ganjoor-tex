\begin{center}
\section*{غزل شماره ۸۸: چون دلبر من سبز خط و پسته دهان است}
\label{sec:088}
\addcontentsline{toc}{section}{\nameref{sec:088}}
\begin{longtable}{l p{0.5cm} r}
چون دلبر من سبز خط و پسته دهان است
&&
دل بر خط حکمش چو قلم بسته میان است
\\
سرسبزی خطش همه سرسبزی خلق است
&&
شور لب لعلش همه شیرینی جان است
\\
نقاش که بنگاشت رخ او به تعجب
&&
از غایت حسن رخش انگشت گزان است
\\
جانا نبرم جان ز تو زیرا که تو ترکی
&&
وابروی تو در تیز زدن سخت کمان است
\\
از غالیه دانت شکری نیست امیدم
&&
کان خال سیه مشرف آن غالیه دان است
\\
از بس دل پرتاب که زلف تو ربوده است
&&
زلف تو چنین تافته پیوسته از آن است
\\
قربان کندم چشم تو از تیر که پیوست
&&
خون ریختن و تیر از آن کیش روان است
\\
خورشید که رویش به جهان پشت سپاه است
&&
بر پشتی روی تو دل افروز جهان است
\\
تا روی دلفروز تو عطار بدیده است
&&
حقا که چنان کش دل و جان خواست چنان است
\\
\end{longtable}
\end{center}
