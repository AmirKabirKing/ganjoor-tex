\begin{center}
\section*{غزل شماره ۸۱: هر شور وشری که در جهان است}
\label{sec:081}
\addcontentsline{toc}{section}{\nameref{sec:081}}
\begin{longtable}{l p{0.5cm} r}
هر شور وشری که در جهان است
&&
زان غمزهٔ مست دلستان است
\\
گفتم لب اوست جان، خرد گفت
&&
جان چیست مگو چه جای جان است
\\
وصفش چه کنی که هرچه گویی
&&
گویند مگو که بیش از آن است
\\
غمهاش به جان اگر فروشند
&&
می‌خر که هنوز رایگان است
\\
در عشق فنا و محو و مستی
&&
سرمایهٔ عمر جاودان است
\\
در عشق چو یار بی نشان شو
&&
کان یار لطیف بی‌نشان است
\\
تو آینهٔ جمال اویی
&&
و آیینهٔ تو همه جهان است
\\
ای ساقی بزم ما سبک‌خیز
&&
می‌ده که سرم ز می گران است
\\
در جام جهان نمای ما ریز
&&
آن باده که کیمیای جان است
\\
ای مطرب ساده ساز بنواز
&&
کامشب شب بزم عاشقان است
\\
از رفته و نامده چه گویم
&&
چون حاصل عمرم این زمان است
\\
ما را سر بودن جهان نیست
&&
ما را سر یار مهربان است
\\
این کار نه کار توست خاموش
&&
کین راه به پای رهروان است
\\
کاری است بزرگ راه عطار
&&
وین کار نه بر سر زبان است
\\
\end{longtable}
\end{center}
