\begin{center}
\section*{غزل شماره ۵۸۱: در درد عشق یک دل بیدار می نبینم}
\label{sec:581}
\addcontentsline{toc}{section}{\nameref{sec:581}}
\begin{longtable}{l p{0.5cm} r}
در درد عشق یک دل بیدار می نبینم
&&
مستند جمله در خود هشیار می نبینم
\\
جمله ز خودپرستی مشغول کار خویشند
&&
در راه او دلی را بر کار می نبینم
\\
عمری بسر دویدم گفتم مگر رسیدم
&&
با دست هرچه دیدم چون یار می نبینم
\\
گفتم مگر که باشم از خاصگان کویش
&&
خود از سگان کویش آثار می نبینم
\\
دعوی است جمله دعوی کو عاشقی و کو عشق
&&
کز کشتگان عشقش دیار می نبینم
\\
گر عاشقی برآور از جان دم اناالحق
&&
زیرا که جای عاشق جز دار می نبینم
\\
چون مرد دین نبودم کیش مغان گزیدم
&&
دین رفت و بر میان جز زنار می نبینم
\\
اکنون ز نا تمامی نه مغ نه مؤمنم من
&&
اندک ز دست دادم بسیار می نبینم
\\
دردا که داد چون گل عطار دل به بادش
&&
وز گلبن وصالش یک خار می نبینم
\\
\end{longtable}
\end{center}
