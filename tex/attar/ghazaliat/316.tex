\begin{center}
\section*{غزل شماره ۳۱۶: آنها که در حقیقت اسرار می‌روند}
\label{sec:316}
\addcontentsline{toc}{section}{\nameref{sec:316}}
\begin{longtable}{l p{0.5cm} r}
آنها که در حقیقت اسرار می‌روند
&&
سرگشته همچو نقطهٔ پرگار می‌روند
\\
هم در کنار عرش سرافراز می‌شوند
&&
هم در میان بحر نگونسار می‌روند
\\
هم در سلوک گام به تدریج می‌نهند
&&
هم در طریق عشق به هنجار می‌روند
\\
راهی که آفتاب به صد قرن آن برفت
&&
ایشان به حکم وقت به یکبار می‌روند
\\
گر می‌رسند سخت سزاوار می‌رسند
&&
ور می‌روند سخت سزاوار می‌روند
\\
در جوش و در خروش از آنند روز و شب
&&
کز تنگنای پردهٔ پندار می‌روند
\\
از زیر پرده فارغ و آزاد می‌شوند
&&
گرچه به پرده باز گرفتار می‌روند
\\
هرچند مطلقند ز کونین و عالمین
&&
در مطلقی گرفتهٔ اسرار می‌روند
\\
بار گران عادت و رسم اوفکنده‌اند
&&
وآزاد همچو سرو سبکبار می‌روند
\\
چون نیست محرمی که بگویند سر خویش
&&
سر در درون کشیده چو طومار می‌روند
\\
چون سیر بی نهایت و چون عمر اندک است
&&
در اندکی هر آینه بسیار می‌روند
\\
تا روی که بود که به بینند روی دوست
&&
روی پر اشک و روی به دیوار می‌روند
\\
بی وصف گشته‌اند ز هستی و نیستی
&&
تا لاجرم نه مست و نه هشیار می‌روند
\\
از ذات و از صفات چنان بی صفت شدند
&&
کز خود نه گم شده نه پدیدار می‌روند
\\
از مشک این حدیث مگر بوی برده‌اند
&&
بر بوی آن به کلبه عطار می‌روند
\\
\end{longtable}
\end{center}
