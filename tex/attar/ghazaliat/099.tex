\begin{center}
\section*{غزل شماره ۹۹: نور ایمان از بیاض روی اوست}
\label{sec:099}
\addcontentsline{toc}{section}{\nameref{sec:099}}
\begin{longtable}{l p{0.5cm} r}
نور ایمان از بیاض روی اوست
&&
ظلمت کفر از سر یک موی اوست
\\
ذره ذره در دو عالم هر چه هست
&&
پرده‌ای در آفتاب روی اوست
\\
هر که را در هر دو عالم قبله‌ای است
&&
گرچه نیست آگاه آنکس سوی اوست
\\
هر دو عالم هیچ می‌دانی که چیست
&&
هر دو عکس طاق دو ابروی اوست
\\
چون کمان ابروی او درکشیم
&&
کان کمان پیوسته بر بازوی اوست
\\
آن همه غوغای روز رستخیز
&&
از مصاف غمزهٔ جادوی اوست
\\
رستخیز آری کلمح باالبصر
&&
از خدنگ چشم چون آهوی اوست
\\
هم زمین از راه او گردی است بس
&&
هر فلک سرگشته‌ای در کوی اوست
\\
زان سیه گردد قیامت آفتاب
&&
تا شود روشن که او هندوی اوست
\\
آسمان را از درش بویی رسید
&&
تا قیامت سرنگون بر بوی اوست
\\
خلق هر دو کون را درد گناه
&&
بر امید ذره‌ای داروی اوست
\\
تا که بویی یافت عطار از درش
&&
دل نمی‌داند که در پهلوی اوست
\\
\end{longtable}
\end{center}
