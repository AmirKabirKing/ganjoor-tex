\begin{center}
\section*{غزل شماره ۴۲: آنکه چندین نقش ازو برخاسته است}
\label{sec:042}
\addcontentsline{toc}{section}{\nameref{sec:042}}
\begin{longtable}{l p{0.5cm} r}
آنکه چندین نقش ازو برخاسته است
&&
یارب او در پرده چون آراسته است
\\
چون ز پرده دم به دم می تافته است
&&
هر دو عالم دم به دم می‌کاسته است
\\
چون شود یک ره ز پرده آشکار
&&
تو یقین دان کان قیامت خاسته است
\\
محو گردد در قیامت زان جمال
&&
هر که نقشی در جهان پیراسته است
\\
ذره‌ای معشوق کی آید پدید
&&
چون دو عالم پر زر و پر خواسته است
\\
در قیامت سوی خود کس ننگرد
&&
چون جمال آن چنان آراسته است
\\
ذره‌ای گشت است ظاهر زان جمال
&&
شور از هر دو جهان برخاسته است
\\
ای فرید اینجا چه خواهی کار و بار
&&
راه تو نادانی و ناخواسته است
\\
\end{longtable}
\end{center}
