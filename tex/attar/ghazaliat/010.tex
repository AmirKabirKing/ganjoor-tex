\begin{center}
\section*{غزل شماره ۱۰: در دلم بنشسته‌ای بیرون میا}
\label{sec:010}
\addcontentsline{toc}{section}{\nameref{sec:010}}
\begin{longtable}{l p{0.5cm} r}
در دلم بنشسته‌ای بیرون میا
&&
نی برون آی از دلم در خون میا
\\
چون ز دل بیرون نمی‌آیی دمی
&&
هر زمان در دیده دیگرگون میا
\\
چون کست یک ذره هرگز پی نبرد
&&
تو به یک یک ذره بوقلمون میا
\\
غصه‌ای باشد که چون تو گوهری
&&
آید از دریا برون بیرون میا
\\
سرنگون غواص خود پیش آیدت
&&
تو ز فقر بحر در هامون میا
\\
گر پدید آیی دو عالم گم شود
&&
بیش از این ای لولو مکنون میا
\\
نی برون آی و دو عالم محو کن
&&
گو برون از تو کسی اکنون، میا
\\
چون تو پیدا می‌شوی گم می‌شوم
&&
لطف کن وز وسع من افزون میا
\\
چون به یک مویت ندارم دست رس
&&
دست بر نه برتر از گردون میا
\\
چون ز هشیاری به جان آمد دلم
&&
بی‌شرابی پیش این مجنون میا
\\
بدرهٔ موزون شعرت ای فرید
&&
بستهٔ این بدرهٔ موزون میا
\\
\end{longtable}
\end{center}
