\begin{center}
\section*{غزل شماره ۶۴۹: عشق چیست از خویش بیرون آمدن}
\label{sec:649}
\addcontentsline{toc}{section}{\nameref{sec:649}}
\begin{longtable}{l p{0.5cm} r}
عشق چیست از خویش بیرون آمدن
&&
غرقه در دریای پر خون آمدن
\\
گر بدین دریا فرو خواهی شدن
&&
نیست هرگز روی بیرون آمدن
\\
ور سر کم کاستی دارای در آی
&&
زانکه اینجا نیست افزون آمدن
\\
لازمت باشد اگر عاشق شوی
&&
ترک کردن عقل و مجنون آمدن
\\
از ازل آزاد گشتن وز ابد
&&
محرم سر هم اکنون آمدن
\\
چون توان بودن به صورت بارکش
&&
پس به معنی فوق گردون آمدن
\\
سر بریده راه رفتن چون قلم
&&
پا و سر افکنده چون نون آمدن
\\
سرنگون رفتن درین دریای ژرف
&&
پس نهان چون در مکنون آمدن
\\
چون دهم شرحت همی گم بودگی است
&&
محرم این بحر بیچون آمدن
\\
تا ابد یکرنگ بودن با فنا
&&
نی همی هردم دگرگون آمدن
\\
چیست ای عطار کفر راه عشق
&&
سست دین از همت دون آمدن
\\
\end{longtable}
\end{center}
