\begin{center}
\section*{غزل شماره ۸۵: جهانی جان چو پروانه از آن است}
\label{sec:085}
\addcontentsline{toc}{section}{\nameref{sec:085}}
\begin{longtable}{l p{0.5cm} r}
جهانی جان چو پروانه از آن است
&&
که آن ترسا بچه شمع جهان است
\\
به ترسایی درافتادم که پیوست
&&
مرا زنار زلفش بر میان است
\\
درآمد دوش آن ترسا بچه مست
&&
مرا گفتا که دین من عیان است
\\
درین دین گر بقا خواهی فنا شو
&&
که گر سودی کنی آنجا زیان است
\\
بدو گفتم نشانی ده ازین راه
&&
مرا گفتا که این ره بی نشان است
\\
ز پیدایی هویدا در هویداست
&&
ز پنهانی نهان اندر نهان است
\\
فنا اندر فنای است و عجب این
&&
که اندر وی بقای جاودان است
\\
چو پیدا و نهان دانستی این راه
&&
یقین می‌دان که نه این و نه آن است
\\
به دین ما درآ گر مرد کفری
&&
که عاشق غیر این دین کفر دان است
\\
یقین می‌دان که کفر عاشقی را
&&
بنا بر کافری جاودان است
\\
اگر داری سر این پای در نه
&&
به ترک جان بگو چه جای جان است
\\
وگرنه با سلامت رو که با تو
&&
سخن گفتن ز دلق و طیلسان است
\\
برو عطار و تن زن زانکه این شرح
&&
نه کار توست کار رهبران است
\\
\end{longtable}
\end{center}
