\begin{center}
\section*{غزل شماره ۴۳۰: بنمود رخ از پرده، دل گشت گرفتارش}
\label{sec:430}
\addcontentsline{toc}{section}{\nameref{sec:430}}
\begin{longtable}{l p{0.5cm} r}
بنمود رخ از پرده، دل گشت گرفتارش
&&
دانی که کجا شد دل در زلف نگونسارش
\\
از بس که سر زلفش در خون دل من شد
&&
در نافهٔ زلف او دل گشت جگرخوارش
\\
چون مشک و جگر دید او در ناک دهی آمد
&&
ناک از چه دهد آخر خاکی شده عطارش
\\
ای کاش چو دل برد او بارش دهدی باری
&&
چون بار دهد دل را چون دل ندهد بارش
\\
جانا چو دلم دارد درد از سر زلف تو
&&
بگذار در آن دردش وز دست بمگذارش
\\
بردی دلم و پایش بستی به سر زلفت
&&
دل باز نمی‌خواهم اما تو نکو دارش
\\
تا بو که به دست آرم یک ذره وصال تو
&&
جان می‌بفروشم من کس نیست خریدارش
\\
چون نیست وصالت را در کون خریداری
&&
عطار کجا افتد یک ذره سزاوارش
\\
\end{longtable}
\end{center}
