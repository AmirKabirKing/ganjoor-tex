\begin{center}
\section*{غزل شماره ۷۰۶: شب را ز تیغ صبحدم خون است عمدا ریخته}
\label{sec:706}
\addcontentsline{toc}{section}{\nameref{sec:706}}
\begin{longtable}{l p{0.5cm} r}
شب را ز تیغ صبحدم خون است عمدا ریخته
&&
اینک ببین خون شفق در طشت مینا ریخته
\\
لالای شب در هر قدم لؤلؤ بر آورده بهم
&&
وز یک نسیم صبحدم لؤلؤی لالا ریخته
\\
خورشید زرکش تافته زربفت عیسی بافته
&&
زنار زرین یافته زر بر مسیحا ریخته
\\
مطرب ز دیوان فرح پروانه را آورده صح
&&
ساقی شراب اندر قدح از حوض حورا ریخته
\\
موسی کف عیسی زبان فرعونیی کرده روان
&&
زنار زلفش هر زمان صد خون ترسا ریخته
\\
ساقی به گردش سر گران زرین نطاقی بر میان
&&
وز شرم او از کهکشان جوجو چو جوزا ریخته
\\
ما کرده از مستی همی بر جام ساقی جان فدی
&&
وز دیده تا تحت‌الثری عقد ثریا ریخته
\\
از تائبی سر تافته صد توبه برهم بافته
&&
چون بوی زلفش یافته می بر مصلا ریخته
\\
چون قطره دریا کش زبون اشک وی از دریا فزون
&&
دریای دل یک قطره خون یک قطره دریا ریخته
\\
آنجا که قومی همنفس می می‌دهند از پیش و پس
&&
طاوس جان‌ها چون مگس بال و پر آنجا ریخته
\\
جان غرقهٔ سودای دل تن نیز ناپروای دل
&&
عطار از دریای دل صد گنج پیدا ریخته
\\
\end{longtable}
\end{center}
