\begin{center}
\section*{غزل شماره ۱۰۶: عشق را اندر دو عالم هیچ پذرفتار نیست}
\label{sec:106}
\addcontentsline{toc}{section}{\nameref{sec:106}}
\begin{longtable}{l p{0.5cm} r}
عشق را اندر دو عالم هیچ پذرفتار نیست
&&
چون گذشتی از دو عالم هیچکس را بار نیست
\\
هر دو عالم چیست رو نعلین بیرون کن ز پای
&&
تا رسی آنجا که آنجا نام و نور و نار نیست
\\
چون رسی آنجا نه تو مانی و نه غیر تو هم
&&
پس چه ماند هیچ، کانجا هیچ غیر از یار نیست
\\
چون نمانی تو، تو مانی جمله و این فهم را
&&
در خیال آفرینش هیچ استظهار نیست
\\
چون رسیدی تو به تو هم هیچ باشی هم همه
&&
چه همه چه هیچ چون اینجا سخن بر کار نیست
\\
آنچه می‌جویی تویی و آنچه می‌خواهی تویی
&&
پس ز تو تا آنچه گم کردی ره بسیار نیست
\\
کل کل چون جان تو آمد اگر در هر دو کون
&&
هیچکس را هست صاعی جز تو را دربار نیست
\\
چون به جان فانی شدی آسان به جانان ره بری
&&
زانکه از جان تا به جانان تو ره دشوار نیست
\\
جان چو در جانان فرو شد جمله جانان ماند و بس
&&
خود به جز جانان کسی را هیچ استقرار نیست
\\
جمله اینجا روی در دیوار جان خواهند داد
&&
گر علاجی هست دیگر جز سر و دیوار نیست
\\
گر گمان خلق ازین بیش است سودایی است بس
&&
ور خیال غیر در راه است جز پندار نیست
\\
هر که آمد هیچ آمد هر که شد هم هیچ شد
&&
هم ازین و هم از آن در هر دو کون آثار نیست
\\
هیچ چون جوید همه یا هیچ چون آید همه
&&
چون همه باشد همه پس هیچ را مقدار نیست
\\
راه وصلش چون روم چون نیست منزلگه پدید
&&
حلقه بر در چون زنم چون در درون دیار نیست
\\
هست گنجی از دو عالم مانده پنهان تا ابد
&&
جای او جز کنج خلوتخانهٔ اسرار نیست
\\
در زمین و آسمان این گنج کی یابی تو باز
&&
زانکه آن جز در درون مرد معنی‌دار نیست
\\
در درون مرد پنهان وی عجب مردان مرد
&&
جمله کور از وی که آنجا دیده و دیدار نیست
\\
تا تو بر جایی طلسم گنج بر جای است نیز
&&
چون تو گم گشتی کسی از گنج برخوردار نیست
\\
گر تو باشی گنج نی و گر نباشی گنج هست
&&
بشنو این مشنو که این اقرار با انکار نیست
\\
تا دل عطار بیخود شد درین مستی فتاد
&&
بیخودی آمد ز خود او نیست شد عطار نیست
\\
\end{longtable}
\end{center}
