\begin{center}
\section*{غزل شماره ۴۹۱: بر درد تو دل از آن نهادم}
\label{sec:491}
\addcontentsline{toc}{section}{\nameref{sec:491}}
\begin{longtable}{l p{0.5cm} r}
بر درد تو دل از آن نهادم
&&
کان درد برای جان نهادم
\\
از مال جهانم نیم جان بود
&&
با درد تو در میان نهادم
\\
از در سرشک و گوهر اشک
&&
بس گنج که رایگان نهادم
\\
هر روز هزار بار خود را
&&
در بوتهٔ امتحان نهادم
\\
از بوته چو پا برون گرفتم
&&
مهر غم تو بر آن نهادم
\\
آن سر که ببند کس نیاید
&&
از دست تو در جهان نهادم
\\
شوریده به شهر در فتادم
&&
بنیاد جنون چنان نهادم
\\
کز یک دم خویش هفت دوزخ
&&
در جنب نه آسمان نهادم
\\
بس شب که در اشتیاق رویت
&&
سر بر سر آستان نهادم
\\
بس روز که دل کباب کردم
&&
در پیش سگانت خوان نهادم
\\
سودای تو سر چو بر نمی‌تافت
&&
با مغز در استخوان نهادم
\\
چه سود که بی تو بر من آمد
&&
هر تیر که در کمان نهادم
\\
صد ساله ذخیرهٔ ملامت
&&
زان غمزهٔ دلستان نهادم
\\
صد لقمهٔ زهر در دهانم
&&
زان لعل شکرفشان نهادم
\\
هر فکر که از لب تو کردم
&&
بندی است که بر دهان نهادم
\\
عطار به جان رسیده را مهر
&&
از مهر تو بر زبان نهادم
\\
\end{longtable}
\end{center}
