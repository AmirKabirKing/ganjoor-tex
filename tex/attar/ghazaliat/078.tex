\begin{center}
\section*{غزل شماره ۷۸: غم بسی دارم چه جای صد غم است}
\label{sec:078}
\addcontentsline{toc}{section}{\nameref{sec:078}}
\begin{longtable}{l p{0.5cm} r}
غم بسی دارم چه جای صد غم است
&&
زانکه هر موییم در صد ماتم است
\\
غم نباشد کانچه پیشان است و پس
&&
کم ز کم نبود نصیبم زان کم است
\\
عالمی است اشراق نور آفتاب
&&
کور را زانچه اگر صد عالم است
\\
عالمی در دست بر جانم ولی
&&
چون ازوست این درد جانم خرم است
\\
درد زخم او کشیدن خوش بود
&&
گر پس از صد زخم او یک مرهم است
\\
گر بسی عمرم بود تا جان بود
&&
آن من گر هست عمری یک دم است
\\
گر کسی را آن دم اینجا دست داد
&&
او خلیفه‌زاده‌ای از آدم است
\\
ور کسی زان دم ندارد آگهی
&&
مرده دل زاد است اگر از مریم است
\\
بی خیال و صورت وهم و قیاس
&&
چیست آن دم، شیر و روغن درهم است
\\
نی که دایم روغن است و شیر نه
&&
زانکه گر شیر است بس نامحرم است
\\
گر فرید این جایگه با خویش نیست
&&
آن دمش در پردهٔ جان همدم است
\\
\end{longtable}
\end{center}
