\begin{center}
\section*{غزل شماره ۶۹۱: ای دل مبتلای من شیفتهٔ هوای تو}
\label{sec:691}
\addcontentsline{toc}{section}{\nameref{sec:691}}
\begin{longtable}{l p{0.5cm} r}
ای دل مبتلای من شیفتهٔ هوای تو
&&
دیده دلم بسی بلا آن همه از برای تو
\\
رای مرا به یک زمان جمله برای خود مران
&&
چون ز برای خود کنم چند کشم بلای تو
\\
نی ز برای تو به جان بار بلای تو کشم
&&
عشق تو و بلای جان، جان من و وفای تو
\\
باد جهان بی وفا دشمن من ز جان و دل
&&
گر نکنم ز دوستی از دل و جان هوای تو
\\
پرده ز روی برفکن زانکه بماند تا ابد
&&
جملهٔ جان عاشقان مست می لقای تو
\\
جان و دلی است بنده را بر تو فشانم اینکه هست
&&
نی که محقری است خود کی بود این سزای تو
\\
چشم من از گریستن تیره شدی اگر مرا
&&
گاه و به‌گاه نیستی سرمه ز خاک پای تو
\\
گر ببری به دلبری از سر زلف جان من
&&
زنده شوم به یک نفس از لب جانفزای تو
\\
هست ز مال این جهان نقد فرید نیم جان
&&
می نپذیری این ازو پس چه کند برای تو
\\
\end{longtable}
\end{center}
