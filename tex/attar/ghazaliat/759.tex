\begin{center}
\section*{غزل شماره ۷۵۹: گر مرد راه عشقی ره پیش بر به مردی}
\label{sec:759}
\addcontentsline{toc}{section}{\nameref{sec:759}}
\begin{longtable}{l p{0.5cm} r}
گر مرد راه عشقی ره پیش بر به مردی
&&
ورنه به خانه بنشین چه مرد این نبردی
\\
درمان عشق جانان هم درد اوست دایم
&&
درمان مجوی دل را گر زنده دل به دردی
\\
گفتی به ره سپردن گردی برآرم از ره
&&
نه هیچ ره سپردی نه هیچ گرد کردی
\\
گرچه ز قوت دل چون کوه پایداری
&&
در پیش عشق سرکش چون پیش باد گردی
\\
مردان مرد اینجا در پرده چون زنانند
&&
تو پیش صف چه آیی چون نه زنی نه مردی
\\
مردان هزار دریا خوردند و تشنه مردند
&&
تو مست از چه گشتی چون جرعه‌ای نخوردی
\\
گر سالها به پهلو می‌گردی اندرین ره
&&
مرتد شوی اگر تو یک دم ملول گردی
\\
باید که هر دو عالم یک جزء جانت آید
&&
گر تو به جان کلی در راه عشق فردی
\\
بگذر ز راه دعوی در جمع اهل معنی
&&
مرهم طلب ازیشان گر یار سوز و دردی
\\
عطار اگر به‌کلی از خود خلاص یابد
&&
یک جزو جانش آید نه چرخ لاجوردی
\\
\end{longtable}
\end{center}
