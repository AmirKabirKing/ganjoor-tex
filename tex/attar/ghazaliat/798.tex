\begin{center}
\section*{غزل شماره ۷۹۸: زلف را تاب داد چندانی}
\label{sec:798}
\addcontentsline{toc}{section}{\nameref{sec:798}}
\begin{longtable}{l p{0.5cm} r}
زلف را تاب داد چندانی
&&
که نه عقلی گذاشت نه جانی
\\
نیست در چار حد جمع جهان
&&
بی سر زلف او پریشانی
\\
کس چو زلف و لبش نداد نشان
&&
ظلماتی و آب حیوانی
\\
دهن اوست در همه عالم
&&
عالمی قند در نمکدانی
\\
دی برای شکر ربودن ازو
&&
می‌شدم تیز کرده دندانی
\\
لیک گفتم به قطع جان نبرم
&&
او چنین تیز کرده مژگانی
\\
بامدادی که تیغ زد خورشید
&&
مگر از حسن کرد جولانی
\\
گوی سیمین او چو ماه بتافت
&&
گشت خورشید تنگ میدانی
\\
لاجرم شد ز رشک او جاوید
&&
زرد رویی کبود خلقانی
\\
جرم خورشید بود کز سر جهل
&&
پیش رویش نمود برهانی
\\
هست نازان رخش چنانکه به حکم
&&
هرچه او کرد نیست تاوانی
\\
ماه رویا اسیر تو شده‌اند
&&
هر کجا کافر و مسلمانی
\\
صد جهان عاشقند جان بر دست
&&
جمله در انتظار فرمانی
\\
پرده برگیر تا برافشانند
&&
هرکجا هست جان و ایمانی
\\
چند سازی ز زلف خم در خم
&&
دار اسلام کافرستانی
\\
تا به دامن ز عشق تو شق کرد
&&
هر که سر بر زد از گریبانی
\\
ندمد در بهارگاه دو کون
&&
سبزتر از خط تو ریحانی
\\
نتواند شکفت در فردوس
&&
تازه‌تر از رخت گلستانی
\\
من چنانم ز لعل سیرابت
&&
که بود تشنه در بیابانی
\\
گر دهی شربتیم آب زلال
&&
شوم از عشق آتش‌افشانی
\\
ورنه در موکب ممالک تو
&&
کرده گیر از فرید قربانی
\\
\end{longtable}
\end{center}
