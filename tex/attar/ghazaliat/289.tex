\begin{center}
\section*{غزل شماره ۲۸۹: پیش رفتن را چو پیشان بسته‌اند}
\label{sec:289}
\addcontentsline{toc}{section}{\nameref{sec:289}}
\begin{longtable}{l p{0.5cm} r}
پیش رفتن را چو پیشان بسته‌اند
&&
بازگشتن را چو پایان بسته‌اند
\\
پس نه از پس راه داری نه ز پیش
&&
کز دو سو ره بر تو حیران بسته‌اند
\\
پس تو را حیران میان این دو راه
&&
عالمی زنجیر در جان بسته‌اند
\\
بی قراری زانکه در جان و دلت
&&
این همه زنجیر جنبان بسته‌اند
\\
چون عدد گویی تو دایم نه احد
&&
هم عدد در تو فراوان بسته‌اند
\\
حرص زنجیر است این سر فهم کن
&&
تا بری پی هرچه زینسان بسته‌اند
\\
حرص باید تا تو زر جمع‌آوری
&&
تا کند وام از تو این زان بسته‌اند
\\
چون عوض خواهی تو زر را گویدت
&&
چار طاقت خلد رضوان بسته‌اند
\\
چون رسی در خلد گوید نفس خلد
&&
از برای نفس انسان بسته‌اند
\\
مرد جانی جمع شود بگذر ز نفس
&&
زانکه دل در تو پریشان بسته‌اند
\\
در علفزاری چه خواهی کرد تو
&&
چون تو را در قید سلطان بسته‌اند
\\
قرب سلطان جوی و مهمانی مخواه
&&
کان خیال از بهر مهمان بسته‌اند
\\
جان به ما ده تا همه جانان شوی
&&
کین همه از بهر جانان بسته‌اند
\\
هم چنین یک یک صفت می کن قیاس
&&
کان همه زنجیر از اینسان بسته‌اند
\\
تو به یک‌یک راه می‌بر سوی دوست
&&
لیک دشوار است و آسان بسته‌اند
\\
چون به پیشان راه بردی، برگشاد
&&
بر تو هر در کان ز پیشان بسته‌اند
\\
چون رسی آنجا شود روشن تو را
&&
پرده‌ای کز کفر و ایمان بسته‌اند
\\
جز به توحیدت نگردد آشکار
&&
آنچه در جان تو پنهان بسته‌اند
\\
جان عطار ای عجب چون سایه‌ای است
&&
لیک در خورشید رخشان بسته‌اند
\\
\end{longtable}
\end{center}
