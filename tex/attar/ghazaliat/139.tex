\begin{center}
\section*{غزل شماره ۱۳۹: ای زلف تو دام و دانه خالت}
\label{sec:139}
\addcontentsline{toc}{section}{\nameref{sec:139}}
\begin{longtable}{l p{0.5cm} r}
ای زلف تو دام و دانه خالت
&&
هر صید که می‌کنی حلالت
\\
خورشید دراوفتاده پیوست
&&
در حلقهٔ دام شب مثالت
\\
همچون نقطی سیه پدیدار
&&
بر چهرهٔ آفتاب خالت
\\
دل فتنهٔ طرهٔ سیاهت
&&
جان تشنهٔ چشمهٔ زلالت
\\
از عالم حسن دایه لطف
&&
آورده به صد هزار سالت
\\
رخ زرد و کبود جامه خورشید
&&
سرگشتهٔ ذرهٔ وصالت
\\
تو خفته و اختران همه شب
&&
مبهوت بمانده در جمالت
\\
تو ماه تمامی و عجب آنک
&&
انگشت نمای شد هلالت
\\
مرغی عجبی که می‌نگنجد
&&
در صحن سپهر پر و بالت
\\
چون در تو توان رسید چون کس
&&
هرگز نرسید در خیالت
\\
پی گم کردی چنانکه هرگز
&&
کس پی نبرد به هیچ حالت
\\
خواهد که بسی بگوید از تو
&&
عطار ولی بود ملالت
\\
\end{longtable}
\end{center}
