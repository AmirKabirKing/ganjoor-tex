\begin{center}
\section*{غزل شماره ۹۸: بی تو از صد شادیم یک غم به است}
\label{sec:098}
\addcontentsline{toc}{section}{\nameref{sec:098}}
\begin{longtable}{l p{0.5cm} r}
بی تو از صد شادیم یک غم به است
&&
با تو یک زخمم ز صد مرهم به است
\\
گر ز مشرق تا به مغرب دعوت است
&&
چون نمی‌بینم تو را ماتم به است
\\
از میان جان ز سوز عشق تو
&&
گر کنم آهی ز دو عالم به است
\\
می‌نگویم از بتر بودن سخن
&&
می چه پرسی حال من هر دم به است
\\
گرمی می‌باید و عشقت مدام
&&
زانکه نفت عشق تو از نم به است
\\
هست آب چشم کروبی بسی
&&
آتش جان بنی آدم به است
\\
چون بشست افتاد دست آویز را
&&
زلف تو پر حلقه و پر خم به است
\\
چون تویی محرم مرا در هر دو کون
&&
خلق عالم جمله نامحرم به است
\\
شادی وصلت چو بر بالای توست
&&
پس نصیب خلق مشتی غم به است
\\
توسن عشق تو رام توست و بس
&&
زانکه رخش تند را رستم به است
\\
رنگ بسیار است در عالم ولیک
&&
بر رکوی عیسی مریم به است
\\
پشه‌ای را دیده‌ای هرگز که گفت
&&
همنشینم گنبد اعظم به است
\\
نی که تو سلطانی و ما گلخنی
&&
عز تو با ذل ما بر هم به است
\\
چون فرید از ناله همچون چنگ شد
&&
هر رگ او همچو زیر و بم به است
\\
\end{longtable}
\end{center}
