\begin{center}
\section*{غزل شماره ۴۳۳: هر مرد که نیست امتحانش}
\label{sec:433}
\addcontentsline{toc}{section}{\nameref{sec:433}}
\begin{longtable}{l p{0.5cm} r}
هر مرد که نیست امتحانش
&&
خوابی و خوری است در جهانش
\\
می‌خفتد و می‌خورد شب و روز
&&
تا مغز بود در استخوانش
\\
فربه کند از غرور پهلو
&&
تا نام نهند پهلوانش
\\
مرد آن باشد که همچو شمعی
&&
آتش بارد ز ریسمانش
\\
از بسکه در امتحان کشندش
&&
پیدا گردد همه نهانش
\\
چون پاک شود ز هرچه دارد
&&
آنگاه نهند در میانش
\\
صد مغز یقین دهندش آنگاه
&&
در پوست کشند از گمانش
\\
تا هیچ فریفته نگردد
&&
ایمن نبود ز مکر جانش
\\
چون پاک شد از دو کون کلی
&&
آیند دو کون میهمانش
\\
نقدیش بود که مثل نبود
&&
در هفت زمین و آسمانش
\\
دانی تو که آن چه نقش یابد
&&
تا خرج کنند جاودانش
\\
تو جوهر مرد کی شناسی
&&
نا کرده هزار امتحانش
\\
در هر صفتش بجوی صد بار
&&
در علم مبین و در عیانش
\\
گر قلب بود بدر برون کن
&&
ور نی بنشین بر آستانش
\\
مردی که تو را به خویش خواند
&&
در حال ز پیش خود برانش
\\
وان مرد که از تو می‌گریزد
&&
گنجی است درون خاکدانش
\\
وان کو نگریزد از تو با تو
&&
چون باد ز پس شوی دوانش
\\
این هم رنگ است و می‌توان کرد
&&
رسوای زمانه هر زمانش
\\
شرحت دادم که بی نشان کیست
&&
بپذیر چو جان بدین نشانش
\\
خاک ره او به چشم درکش
&&
کز سود تو ببود زیانش
\\
زیبا محکی نهاد عطار
&&
زین شرح که رفت بر زبانش
\\
\end{longtable}
\end{center}
