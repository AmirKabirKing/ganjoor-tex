\begin{center}
\section*{غزل شماره ۴۵۶: صبح رخ از پرده نمود ای غلام}
\label{sec:456}
\addcontentsline{toc}{section}{\nameref{sec:456}}
\begin{longtable}{l p{0.5cm} r}
صبح رخ از پرده نمود ای غلام
&&
چند کنی گفت و شنود ای غلام
\\
دیر شد آخر قدحی می بیار
&&
چند زنم بانگ که زود ای غلام
\\
درد خرابات مپیمای کم
&&
هین که بسی درد فزود ای غلام
\\
در دلم آتش فکن از می که می
&&
آینهٔ دل بزدود ای غلام
\\
آتش تر ده به صبوحی که عمر
&&
می‌گذرد زود چو دود ای غلام
\\
عمر تو چون اول افسانه‌ای
&&
هرچه همی بود نبود ای غلام
\\
روی زمین گر همه ملک تو شد
&&
در پی تو مرگ چه سود ای غلام
\\
پشت بده زانکه بلایی دگر
&&
هر نفست روی نمود ای غلام
\\
گوشه‌نشین باش که چوگان چرخ
&&
گوی ز پیش تو ربود ای غلام
\\
دانهٔ امید چه کاری که دهر
&&
دانهٔ ناکشته درود ای غلام
\\
صد قدح خونش بباید کشید
&&
هر که دمی خوش بغنود ای غلام
\\
بر دل عطار فلک هر نفس
&&
صد در اندوه گشود ای غلام
\\
\end{longtable}
\end{center}
