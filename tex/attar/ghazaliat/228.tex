\begin{center}
\section*{غزل شماره ۲۲۸: ترسا بچهٔ مستم گر پرده براندازد}
\label{sec:228}
\addcontentsline{toc}{section}{\nameref{sec:228}}
\begin{longtable}{l p{0.5cm} r}
ترسا بچهٔ مستم گر پرده براندازد
&&
بس سر که ز هر سویی بر یکدگر اندازد
\\
از دیر برون آمد سرمست و پریشان مو
&&
یارب که چه آتش‌ها در هر جگر اندازد
\\
چون زلف پریشان را زنار برافشاند
&&
صد رهبر ایمان را در رهگذر اندازد
\\
هم غمزهٔ غمازش بی تیر جگر دوزد
&&
هم طرهٔ طرارش بی تیغ سر اندازد
\\
در وقت ترش‌رویی چون تلخ سخن گوید
&&
بس شور به شیرینی کاندر شکر اندازد
\\
کو عیسی روحانی تا معجز خود بیند
&&
کو یوسف کنعانی تا چشم بر اندازد
\\
گر عارض خوب او از پرده برون آید
&&
صد چون پسر ادهم تاج و کمر اندازد
\\
گر تائب صد ساله بیند شکن زلفش
&&
حالی به سراندازی دستار در اندازد
\\
ور صوفی صافی دل رویش به خیال آرد
&&
زنار کمر سازد خرقه بدر اندازد
\\
گر تر بکند دریا از چشمهٔ خضرش لب
&&
دایم به نثار او موج گهر اندازد
\\
ور طشت فلک روزی در زر کندش پنهان
&&
همچون گهرش حالی زر باز بر اندازد
\\
خورشید که هر روزی بس تیغ زنان آید
&&
از رشک رخش هر شب آخر سپر اندازد
\\
چون دوستی آن بت در سینه فرود آید
&&
دل دشمن جان گردد جان در خطر اندازد
\\
در دیده و دل هرگز چه خشک و ترم ماند
&&
چون هر نفسم آتش در خشک و تر اندازد
\\
عطار اگر روزی نو دولت عشق آید
&&
یکبار دگر آخر بر وی نظر اندازد
\\
\end{longtable}
\end{center}
