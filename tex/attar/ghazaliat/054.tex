\begin{center}
\section*{غزل شماره ۵۴: پی آن گیر کاین ره پیش بردست}
\label{sec:054}
\addcontentsline{toc}{section}{\nameref{sec:054}}
\begin{longtable}{l p{0.5cm} r}
پی آن گیر کاین ره پیش بردست
&&
که راه عشق پی بردن نه خردست
\\
عدو جان خویش و خصم تن گشت
&&
در اول گام هرک این ره سپردست
\\
کسی داند فراز و شیب این راه
&&
که سرگردانی این راه بردست
\\
گهی از چشم خود خون می‌فشاندست
&&
گهی از روی خود خون می‌ستردست
\\
گرش هر روز صد جان می‌رسیدست
&&
صد و یک جان به جانان می‌سپردست
\\
دلش را صد حیات زنده بودست
&&
اگر آن نفس یک ساعت بمردست
\\
ز سندانی که بر سر می‌زنندش
&&
قدم در عشق محکم‌تر فشردست
\\
کسی چون ذره گردد این هوا را
&&
که دم اندر هوای خود شمردست
\\
بسا آتش که چون اینجا رسیدست
&&
شدست آبی و همچون یخ فسردست
\\
بسا دریا کش پاکیزه گوهر
&&
که اینجا قطره‌ای آبش ببردست
\\
مشو پیش صف ای نه مرد و نه زن
&&
که خفتان تو اطلس نیست بردست
\\
مده خود را ز پری این تهی باد
&&
که در جام تو نه صاف و نه دردست
\\
درین وادی دل وحشی عطار
&&
ز حیرت جلف‌تر زان مرد کردست
\\
\end{longtable}
\end{center}
