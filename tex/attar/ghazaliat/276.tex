\begin{center}
\section*{غزل شماره ۲۷۶: لعل تو به جان فزایی آمد}
\label{sec:276}
\addcontentsline{toc}{section}{\nameref{sec:276}}
\begin{longtable}{l p{0.5cm} r}
لعل تو به جان فزایی آمد
&&
چشم تو به دلربایی آمد
\\
چون صد گرهم فتاد در کار
&&
زلفت به گره‌گشایی آمد
\\
با زنگی خال تو که بر ماه
&&
در جلوهٔ خودنمایی آمد
\\
در دیدهٔ آفتاب روشن
&&
چون نقطهٔ روشنایی آمد
\\
با چشم تو می‌بباختم جان
&&
چون چشم تو دردغایی آمد
\\
بگریخت دلم ز چشم تو زود
&&
وآواره ز بی وفایی آمد
\\
در حلقهٔ زلفت آن دم افتاد
&&
کز چشم تواش رهایی آمد
\\
هرگاه که بگذری به بازار
&&
گویند به جان فزایی آمد
\\
یکتایی ماه شق شد از رشک
&&
تا سرو تو در دوتایی آمد
\\
بنشین و دگر مرو اگرچه
&&
در کار تو صد روایی آمد
\\
دانی نبود صواب اسلام
&&
آنجا که بت ختایی آمد
\\
بردی دلم و بحل بکردم
&&
واشکم همه در گوایی آمد
\\
در کار من جدا فتاده
&&
چندین خلل از جدایی آمد
\\
بیگانه مباش زانکه عطار
&&
پیش تو به آشنایی آمد
\\
\end{longtable}
\end{center}
