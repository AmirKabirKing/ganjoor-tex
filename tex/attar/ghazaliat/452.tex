\begin{center}
\section*{غزل شماره ۴۵۲: صورت نبندد ای صنم، بی زلف تو آرام دل}
\label{sec:452}
\addcontentsline{toc}{section}{\nameref{sec:452}}
\begin{longtable}{l p{0.5cm} r}
صورت نبندد ای صنم، بی زلف تو آرام دل
&&
دل فتنه شد بر زلف تو، ای فتنهٔ ایام دل
\\
ای جان به مولای تو، دل غرقهٔ دریای تو
&&
دیری است تا سودای تو، بگرفت هفت اندام دل
\\
تا جان به عشقت بنده شد، زین بندگی تابنده شد
&&
تا دل ز نامت زنده شد، پر شد دو عالم نام دل
\\
جانا دلم از چشم بد، نه هوش دارد نه خرد
&&
تا از شراب عشق خود، پر باده کردی جام دل
\\
پیغامت آمد از دلم، کای ماه حل کن مشکلم
&&
کی خواهد آمد حاصلم، ای فارغ از پیغام دل
\\
از رخ مه گردون تویی، وز لب می گلگون تویی
&&
کام دل من چون تویی، هرگز نیابم کام دل
\\
ای همگنان را همدمی، شادی من از تو غمی
&&
عطار را در هر دمی، جانا تویی آرام دل
\\
\end{longtable}
\end{center}
