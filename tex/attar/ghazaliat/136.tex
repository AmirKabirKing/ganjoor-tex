\begin{center}
\section*{غزل شماره ۱۳۶: دوش جان دزدیده از دل راه جانان برگرفت}
\label{sec:136}
\addcontentsline{toc}{section}{\nameref{sec:136}}
\begin{longtable}{l p{0.5cm} r}
دوش جان دزدیده از دل راه جانان برگرفت
&&
دل خبر یافت و به تک خاست و دل از جان برگرفت
\\
جان چو شد نزدیک جانان دید دل را نزد او
&&
غصه‌ها کردش ز پشت دست دندان برگرفت
\\
ناگهی بادی برآمد مشکبار از پیش و پس
&&
برقع صورت ز پیش روی جانان برگرفت
\\
جان ز خود فانی شد و دل در عدم معدوم گشت
&&
عقل حیلت‌گر به کلی دست ازیشان برگرفت
\\
بی نشان شد جان کدامین جان که گنجی داشت او
&&
گاه پیدایش نهاد و گاه پنهان برگرفت
\\
فرخ آن اقبال باری کاندرین دریای ژرف
&&
ترک جان گفت و سر این نفس حیوان برگرفت
\\
شکر یزدان را که گنج دین درین کنج خراب
&&
بی غم و رنجی دل عطار آسان برگرفت
\\
\end{longtable}
\end{center}
