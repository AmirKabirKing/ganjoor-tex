\begin{center}
\section*{غزل شماره ۷۷۷: ای بوس تو اصل هر شماری}
\label{sec:777}
\addcontentsline{toc}{section}{\nameref{sec:777}}
\begin{longtable}{l p{0.5cm} r}
ای بوس تو اصل هر شماری
&&
چشم سیهت سفید کاری
\\
زلف تو ز حلقه درشکستی
&&
ماه تو ز مشک در غباری
\\
از زلف تو مشک وام کرده
&&
باد سحری به هر بهاری
\\
روی تو که شمع نه سپهر است
&&
از هشت بهشت یادگاری
\\
هرگز نکشید هیچ نقاش
&&
چون صورت روی تو نگاری
\\
سرسبزتر از خط تو ایام
&&
گل را ننهاد هیچ خاری
\\
شد آب روان ز چشمهٔ چشم
&&
چون خط تو دید سبزه‌زاری
\\
می‌خواستم از لب تو بوسی
&&
گفتی که همی دهم قراری
\\
گفتم که قرار چیست گفتی
&&
هر بوسی را کنی نثاری
\\
جانی بستان بهای بوسی
&&
یا دست ز جان بدار باری
\\
چون هست زکات بر تو واجب
&&
یک بوسه ببخش از هزاری
\\
گر بوسه بسی نگاه داری
&&
هرگز ناید به هیچ کاری
\\
گفتی به شمار بوسه بستان
&&
کی کار مرا بود شماری
\\
چون خوزستان لب تو دارد
&&
کی بوس تو را بود کناری
\\
خود بی جگری نیافت عطار
&&
از لعل تو بوسه هیچ باری
\\
\end{longtable}
\end{center}
