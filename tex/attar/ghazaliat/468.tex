\begin{center}
\section*{غزل شماره ۴۶۸: کار بر خود سخت مشکل کرده‌ام}
\label{sec:468}
\addcontentsline{toc}{section}{\nameref{sec:468}}
\begin{longtable}{l p{0.5cm} r}
کار بر خود سخت مشکل کرده‌ام
&&
زانکه استعداد باطل کرده‌ام
\\
چون به مقصد ره برم چون در سفر
&&
در هوای خویش منزل کرده‌ام
\\
راه خون آلوده می‌بینم همه
&&
کین سفر چون مرغ بسمل کرده‌ام
\\
گر گل‌آلود آورم پایم رواست
&&
کز سرشکم خاک ره گل کرده‌ام
\\
راه بر من هر زمان مشکلتر است
&&
زانکه عزم راه مشکل کرده‌ام
\\
عیش شیرینم برای لذتی
&&
تلخ‌تر از زهر قاتل کرده‌ام
\\
روی جان با نفس کم بینم از آنک
&&
روح ناقص نفس کامل کرده‌ام
\\
حاصل عمرم همه بی حاصلی است
&&
آه از این حاصل که حاصل کرده‌ام
\\
قصهٔ جانم چو کس می‌نشنود
&&
غصهٔ بسیار در دل کرده‌ام
\\
هست دریای معانی بس عظیم
&&
کشتی پندار حایل کرده‌ام
\\
سخت می‌ترسم ازین دریای ژرف
&&
لاجرم ره سوی ساحل کرده‌ام
\\
بیم من از غرقه گشتن چون بسی است
&&
خویش را مشغول شاغل کرده‌ام
\\
چون نمی‌یارم شدن مطلق به خویش
&&
خویشتن را در سلاسل کرده‌ام
\\
بر امید غرقه گشتن چون فرید
&&
روی سوی بحر هایل کرده‌ام
\\
\end{longtable}
\end{center}
