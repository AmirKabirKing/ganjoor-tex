\begin{center}
\section*{غزل شماره ۷۳۱: ای روی تو زهر سو رویی دگر نموده}
\label{sec:731}
\addcontentsline{toc}{section}{\nameref{sec:731}}
\begin{longtable}{l p{0.5cm} r}
ای روی تو زهر سو رویی دگر نموده
&&
لطف تو از کفی گل گنجی گهر نموده
\\
دریای در عشقت در اصل لطف پاک است
&&
اما نخست هیبت چندین خطر نموده
\\
در قرن‌ها فلک‌ها در راه تو شب و روز
&&
از سر به پای رفته وز پای سر نموده
\\
طاوس چرخ پیشت پروانه‌وار رفته
&&
وز نور شمع رویت بی بال و پر نموده
\\
از درگه تو نوری بر جان و دل فتاده
&&
وز دل به چشم رفته نور بصر نموده
\\
تو آمده به قدرت و قدرت به فعل پیدا
&&
فعلت به گشت گشته چندین صور نموده
\\
ناگه به دست قدرت بنموده یک اشارت
&&
این یک اشارت تو چندین اثر نموده
\\
چون در دو کون کس را چشم یگانگی نیست
&&
زان صد هزار حیرت اندر نظر نموده
\\
عطار کز جهانش جانی است عاشق تو
&&
از بحر سینه هر دم دری دگر نموده
\\
\end{longtable}
\end{center}
