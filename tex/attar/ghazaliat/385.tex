\begin{center}
\section*{غزل شماره ۳۸۵: میی درده که در ده نیست هشیار}
\label{sec:385}
\addcontentsline{toc}{section}{\nameref{sec:385}}
\begin{longtable}{l p{0.5cm} r}
میی درده که در ده نیست هشیار
&&
چه خفتی عمر شد برخیز و هشدار
\\
ز نام و ننگ بگریز و چو مردان
&&
ز دردی کوزه‌ای بستان ز خمار
\\
چو مست عشق گشتی کوزه در دست
&&
قلندروار بیرون شو به بازار
\\
لباس خواجگی از بر بیفکن
&&
به میخانه فرو انداز دستار
\\
برآور نعره‌ای مستانه از جان
&&
تهی کن سر ز باد عجب و پندار
\\
ز روی خویشتن بت بر زمین زن
&&
ز زیر خرقه بیرون آر زنار
\\
چو خلقانت بدانند و برانند
&&
تو فارغ گردی از خلقان به یکبار
\\
چنان فارغ شوی از خلق عالم
&&
که یکسانت بود اقرار و انکار
\\
نماند در همه عالم به یک جو
&&
نه کس را نه تو را نزد تو مقدار
\\
چو ببریدی ز خویش و خلق کلی
&&
همی بر جانت افتد پرتو یار
\\
تو هر دم در خروش آیی که احسنت
&&
زهی یار و زهی کار و زهی بار
\\
چو در وادی عشقت راه دادند
&&
در آن وادی به سر می‌رو قلم‌وار
\\
زمانی نعره‌زن از وصل جانان
&&
زمانی رقص کن از فهم اسرار
\\
اگر تو راه جویی نیک بندیش
&&
که راه عشق ظاهر کرد عطار
\\
\end{longtable}
\end{center}
