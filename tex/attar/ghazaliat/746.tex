\begin{center}
\section*{غزل شماره ۷۴۶: شعله زد شمع جمال او ز دولتخانه‌ای}
\label{sec:746}
\addcontentsline{toc}{section}{\nameref{sec:746}}
\begin{longtable}{l p{0.5cm} r}
شعله زد شمع جمال او ز دولتخانه‌ای
&&
گشت در هر دو جهان هر ذره‌ای پروانه‌ای
\\
ای عجب هر شعله‌ای از آفتاب روی او
&&
گشتت زنجیری و در هر حلقه‌ای دیوانه‌ای
\\
هر که با هر حلقه در دنیا نیفتاد آشنا
&&
همچو حلقه تا ابد بر در بود بیگانه‌ای
\\
نیک در هر حلقه او را باز می‌باید شناخت
&&
ورنه گردد بر تو آن هر حلقه‌ای بتخانه‌ای
\\
در درون چاه و زندانش بدان و انس گیر
&&
زانکه نه گلشن بود پیوسته نه کاشانه‌ای
\\
یا اگر هر دم به نوعی نیز بینی آن جمال
&&
تو یقین می‌دان که آن گنجی است در ویرانه‌ای
\\
ور به یک صورت فرو ریزی چو گلبرگی ز بار
&&
کی رسد دریا به تو، تو مست از پیمانه‌ای
\\
قفل عشقش کی گشایی گر کلیدی نبودت
&&
هر دم از انسی نو و دردی نوش دندانه‌ای
\\
من چه‌گویم چون درین دریا دو عالم محو شد
&&
شبنمی را کی رسد از پیشگه پروانه‌ای
\\
هر که خواهد داد از وصلش سر مویی خبر
&&
در حقیقت آن سخن دانی که چیست افسانه‌ای
\\
از مسما کس نخواهد یافت هرگز شمه‌ای
&&
گر به تو اسمی رسد واجب بود شکرانه‌ای
\\
گر جزین چیزی که می‌گویم طلب داری دمی
&&
تا ابد در دام مانی از برای دانه‌ای
\\
شبنمی را فهم کی در بحر بی پایان رسد
&&
گر نمی‌فهمش بود باشد قوی مردانه‌ای
\\
چون رسد آن نم بدو جاوید در پی باشدش
&&
تا کند هم چون خودش از فر خود فرزانه‌ای
\\
یک سر سوزن ندیدی روی دولت ای فرید
&&
ده زبان تا چند خواهی بود همچون شانه‌ای
\\
\end{longtable}
\end{center}
