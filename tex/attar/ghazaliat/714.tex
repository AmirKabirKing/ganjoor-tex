\begin{center}
\section*{غزل شماره ۷۱۴: ساقیا گر پخته‌ای می خام ده}
\label{sec:714}
\addcontentsline{toc}{section}{\nameref{sec:714}}
\begin{longtable}{l p{0.5cm} r}
ساقیا گر پخته‌ای می خام ده
&&
جان بی آرام را آرام ده
\\
خیزو بزمی در صبوحی راست کن
&&
یک صراحی باده ما را وام ده
\\
صبح پیدا گشت و شب اندر شکست
&&
خفتگان مست را دشنام ده
\\
چون بخواهی ریخت همچون گل ز بار
&&
بار کم کش بادهٔ گلفام ده
\\
همچو گل شو بادهٔ گلفام نوش
&&
همچو بلبل سوی گل پیغام ده
\\
داد خود بستان که ایام گل است
&&
یا نه خوش خوش داد این ایام ده
\\
گر سراسر نیست دردی در فکن
&&
نیم مستان را پیاپی جام ده
\\
چون اجل دامی گلوگیر آمده است
&&
چون درآید وقت تن در دام ده
\\
خاطر عطار سودا می‌پزد
&&
سوخت از غم هین شرابش خام ده
\\
\end{longtable}
\end{center}
