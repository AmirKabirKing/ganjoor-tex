\begin{center}
\section*{غزل شماره ۳۴۴: هرچه در هر دو جهان جانان نمود}
\label{sec:344}
\addcontentsline{toc}{section}{\nameref{sec:344}}
\begin{longtable}{l p{0.5cm} r}
هرچه در هر دو جهان جانان نمود
&&
تو یقین می‌دان که آن از جان نمود
\\
هست جانت را دری اما دو روی
&&
دوست از دو روی او دو جهان نمود
\\
کرد از یک روی دنیا آشکار
&&
وز دگر روی آخرت پنهان نمود
\\
آخرت آن روی و دنیا این دگر
&&
ای عجب یک چیز این و آن نمود
\\
هر دو عالم نیست بیرون زین دو روی
&&
هرچه آن دشوار یا آسان نمود
\\
در میان این دو دربند عظیم
&&
چون نگه کردم یکی ایوان نمود
\\
یک درش دنیا و دیگر آخرت
&&
بلکه دو کونش چو دو دوران نمود
\\
باز پرسیدم ز دل کان قصر چیست
&&
گفت خلوتخانهٔ جانان نمود
\\
گفتم آخر قصر سلطان جان ماست
&&
جان نمود این قصر یا سلطان نمود
\\
گفت دایم بر تو سلطان است جان
&&
بارگاه خویش در جان زان نمود
\\
پرتو او بی‌نهایت اوفتاد
&&
لاجرم بی‌حد و بی پایان نمود
\\
تا ابد گر پیش گیری راه جان
&&
ذره‌ای نتوانی از پیشان نمود
\\
پرتوی کان دور بود آن کفر بود
&&
وانکه آن نزدیک بود ایمان نمود
\\
چند گویم این جهان و آن جهان
&&
از دو روی جان همی نتوان نمود
\\
گرد جان در گرد چون مردان بسی
&&
تا توانی عشق را برهان نمود
\\
در جهان جان بسی سرگشته‌اند
&&
کمترین یک چرخ سرگردان نمود
\\
می‌رو و یک دم میاسا از روش
&&
کین سفر در روح جاویدان نمود
\\
گر تورا افتاد یک ساعت درنگ
&&
صد درنگ از عالم هجران نمود
\\
همچو گویی گشت سرگردان مدام
&&
هر که خود را مرد این میدان نمود
\\
خود در این میدان فروشد هر که رفت
&&
وانکه یکدم ماند هم حیران نمود
\\
تا ابد در درد این، عطار را
&&
ذره ذره کلبهٔ احزان نمود
\\
\end{longtable}
\end{center}
