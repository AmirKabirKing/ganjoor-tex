\begin{center}
\section*{غزل شماره ۷۸۶: چون خط شبرنگ بر گلگون کشی}
\label{sec:786}
\addcontentsline{toc}{section}{\nameref{sec:786}}
\begin{longtable}{l p{0.5cm} r}
چون خط شبرنگ بر گلگون کشی
&&
حلقه در گوش مه گردون کشی
\\
گر ببینی روی خود در خط شده
&&
سرکشی و هر زمان افزون کشی
\\
گفته بودی در خط خویشت کشم
&&
تا لباس سرکشی بیرون کشی
\\
خط تو بر ماه و من در قعر چاه
&&
در خط خویشم ندانم چون کشی
\\
گر بریزی بر زمین خونم رواست
&&
بلکه آن خواهم که تیغ اکنون کشی
\\
لیک زلفت از درازی بر زمین است
&&
خون شود جانم اگر در خون کشی
\\
می‌کشی در خاک زلفت تا مرا
&&
هر نفس در بند دیگرگون کشی
\\
چون منم دیوانه تو زنجیر زلف
&&
می‌کشی تا بر من مجنون کشی
\\
دام مشکین می‌نهی عطار را
&&
تا به دام مشکش از افسون کشی
\\
\end{longtable}
\end{center}
