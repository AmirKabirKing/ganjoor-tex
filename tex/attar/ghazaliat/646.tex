\begin{center}
\section*{غزل شماره ۶۴۶: با تو سری در میان خواهد بدن}
\label{sec:646}
\addcontentsline{toc}{section}{\nameref{sec:646}}
\begin{longtable}{l p{0.5cm} r}
با تو سری در میان خواهد بدن
&&
کان ورای جسم و جان خواهد بدن
\\
هر که زان سر یافت یک ذره نشان
&&
از دو عالم بی نشان خواهد بدن
\\
محرم آن شو که گر آن نبودت
&&
تا ابد عمرت زیان خواهد بدن
\\
هر نفس کان در حضور او زنی
&&
عمر تو آن است و آن خواهد بدن
\\
ور نخواهد بود همراهت حضور
&&
پس عذاب جاودان خواهد بدن
\\
وای بر حال کسی کو بر مجاز
&&
زان حقیقت بر کران خواهد بدن
\\
مرد دایم همچنان کاینجا زید
&&
چون بمیرد همچنان خواهد بدن
\\
تا نپنداری که هر کو خار بود
&&
روز محشر گلستان خواهد بدن
\\
هرچه اینجا ذره ذره می‌کنی
&&
جمله در پیشت عیان خواهد بدن
\\
این همه آمد شد و وعد و وعید
&&
از برای امتحان خواهد بدن
\\
تو بکوش و جهد کن تا پی بری
&&
زانکه کار ناگهان خواهد بدن
\\
هر که بی او آستین در خون گرفت
&&
محرم آن آستان خواهد بدن
\\
محرم او شو که کار هر دو کون
&&
محو و گم در یک زمان خواهد بدن
\\
ترک کن کار زمین و آسمان
&&
زانکه این کف وان دخان خواهد بدن
\\
چون به حضرت زود نتوان رفت از آنک
&&
پرده در پرده نهان خواهد بدن
\\
جملهٔ ذرات عالم لاجرم
&&
سوی آن حضرت دوان خواهد بدن
\\
بر کناره می‌شو از هر سایه‌ای
&&
زانکه کاری در میان خواهد بدن
\\
در بر آن کار عالی کار خلق
&&
اشتری بر نردبان خواهد بدن
\\
کار ما در پیش او چون ذره‌ای
&&
در بر هفت آسمان خواهد بدن
\\
چون جهان آنجا کف و دودی بود
&&
پس چه جای صد جهان خواهد بدن
\\
چون برافتد پرده از روی دو کون
&&
آن حقیقت ترجمان خواهد بدن
\\
گوییا هر ذره‌ای را تا ابد
&&
جاودانی صد زبان خواهد بدن
\\
همچو باران ز آسمان سلطنت
&&
خط استغنا روان خواهد بدن
\\
در چنین جایی کجا عطار را
&&
یک سخن یا یک بیان خواهد بدن
\\
\end{longtable}
\end{center}
