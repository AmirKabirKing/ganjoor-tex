\begin{center}
\section*{غزل شماره ۲۶۰: تا نور او دیدم دو کون از چشم من افتاده شد}
\label{sec:260}
\addcontentsline{toc}{section}{\nameref{sec:260}}
\begin{longtable}{l p{0.5cm} r}
تا نور او دیدم دو کون از چشم من افتاده شد
&&
پندار هستی تا ابد از جان و تن افتاده شد
\\
روزی برون آمد ز شب طالب فنا گشت از طلب
&&
شور جهان‌سوزی عجب در انجمن افتاده شد
\\
رویت ز برقع ناگهان یک شعله زد آتش فشان
&&
هر لحظه آتش صد جهان در مرد و زن افتاده شد
\\
چون لب گشادی در سخن جان من آمد سوی تن
&&
تا مرده بیخود نعره‌زن مست از کفن افتاده شد
\\
برقی برون جست از قدم برکند گیتی را ز هم
&&
پس نور وحدت زد علم تا ما و من افتاده شد
\\
ما چون فتادیم از وطن زان خسته‌ایم و ممتحن
&&
دل کی نهد بر خویشتن آن کز وطن افتاده شد
\\
حلاج همچون رستمی خوش با وطن آمد همی
&&
کاندر گلوی وی دمی بند از رسن افتاده شد
\\
ساقی به جای مصحفش جامی نهاده بر کفش
&&
وآتش ز جان پر تفش در پیرهن افتاده شد
\\
می خورد تا شد نعره‌زن پس نعره زد بی ما و من
&&
آزاد گشت از خویشتن بی خویشتن افتاده شد
\\
چون قوت دیگر داشت او زان صبر دیگر داشت او
&&
یک لقمه‌ای برداشت او باز از دهن افتاده شد
\\
در هیبت حالی چنان گشتند مردان چون زنان
&&
چه خیزد از تر دامنان چو تهمتن افتاده شد
\\
در جنب این کار گران گشتند فانی صفدران
&&
هم بت شد و هم بتگران هم بت شکن افتاده شد
\\
عطار ازین معنی همی دارد بدل در عالمی
&&
چون می نیابد محرمی دل بر سخن افتاده شد
\\
\end{longtable}
\end{center}
