\begin{center}
\section*{غزل شماره ۲۹۰: ز لعلت زکاتی شکر می‌ستاند}
\label{sec:290}
\addcontentsline{toc}{section}{\nameref{sec:290}}
\begin{longtable}{l p{0.5cm} r}
ز لعلت زکاتی شکر می‌ستاند
&&
ز رویت براتی قمر می‌ستاند
\\
به یک لحظه چشمت ز عشاق صد جان
&&
به یک غمزهٔ حیله‌گر می‌ستاند
\\
سزد گر ز رشک نظر خون شود دل
&&
که داد از جمالت نظر می‌ستاند
\\
خطت طوطی است آب حیوانش در بر
&&
کزان آب حیوان شکر می‌ستاند
\\
زهی ترکتازی که لوح چو سیمت
&&
خطی سبزم آورد و زرمی‌ستاند
\\
مرا نیست زر چون دهم زر ولیکن
&&
دهم در عوض جان اگر می‌ستاند
\\
مرا گفت جان را خطر نیست زر ده
&&
که چشمم زر بی‌خطر می‌ستاند
\\
اگرچه لبت خشک و چشمت تر آمد
&&
مخور غم که زر خشک و تر می‌ستاند
\\
عیار از رخ زرد عطار دارد
&&
زری کان بت سیم‌بر می‌ستاند
\\
\end{longtable}
\end{center}
