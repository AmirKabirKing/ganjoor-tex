\begin{center}
\section*{غزل شماره ۷۹۲: ماییم ز عالم معالی}
\label{sec:792}
\addcontentsline{toc}{section}{\nameref{sec:792}}
\begin{longtable}{l p{0.5cm} r}
ماییم ز عالم معالی
&&
رندی دو سه اندرین حوالی
\\
در عشق دلی و نیم جانی
&&
بر داده به باد لاابالی
\\
بگذشته ز هستی و گرفته
&&
چون صوفی ابن‌وقت حالی
\\
در صفهٔ عاشقان حضرت
&&
از برهنگی فکنده قالی
\\
پس یافته برترین مقامی
&&
احسنت زهی مقام عالی
\\
ما را چه مرقع و چه اطلس
&&
چه نیک کنی چه بد سگالی
\\
ای زاهد کهنه درد نقد است
&&
برخیز که گوشه‌ای است خالی
\\
تا نالهٔ عاشقان نیوشی
&&
بر خلق ز زهد چند نالی
\\
آن می که تو می‌خوری حرام است
&&
ما می نخوریم جز حلالی
\\
ما بر سر آتشیم پیوست
&&
مستغرق بحر لایزالی
\\
ما بی خوابیم و چون بود خواب
&&
در حضرت قرب ذوالجلالی
\\
چون خواب کند کسی که او را
&&
از ریگ روان بود نهالی
\\
عطار برو که دست بردی
&&
از جملهٔ عالم معالی
\\
\end{longtable}
\end{center}
