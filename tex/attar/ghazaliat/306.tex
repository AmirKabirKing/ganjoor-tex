\begin{center}
\section*{غزل شماره ۳۰۶: دل ز میان جان و دل قصد هوات می‌کند}
\label{sec:306}
\addcontentsline{toc}{section}{\nameref{sec:306}}
\begin{longtable}{l p{0.5cm} r}
دل ز میان جان و دل قصد هوات می‌کند
&&
جان به امید وصل تو عزم وفات می‌کند
\\
گرچه ندید جان و دل از تو وفا به هیچ روی
&&
بر سر صد هزار غم یاد جفات می‌کند
\\
می‌نکند به صد قران ترک کلاه‌دار چرخ
&&
آنچه میان عاشقان بند قبات می‌کند
\\
خسرو یک سواره را بر رخ نطع نیلگون
&&
لعل تو طرح می‌نهد روی تو مات می‌کند
\\
جان و دلم به دلبری زیر و زبر همی کنی
&&
وین تو نمی‌کنی بتا زلف دوتات می‌کند
\\
خود تو چه آفتی که چرخ از پی گوشمال من
&&
هر نفسی به داوری بر سر مات می‌کند
\\
گرچه فرید، از جفا می‌نکند سزای تو
&&
خط تو خود به دست خود با تو سزات می‌کند
\\
\end{longtable}
\end{center}
