\begin{center}
\section*{غزل شماره ۷۲۲: ای هر دهان ز یاد لبت پر عسل شده}
\label{sec:722}
\addcontentsline{toc}{section}{\nameref{sec:722}}
\begin{longtable}{l p{0.5cm} r}
ای هر دهان ز یاد لبت پر عسل شده
&&
در هر زبان خوشی لب تو مثل شده
\\
آوازهٔ وصال تو کوس ابد زده
&&
مشاطهٔ جمال تو لطف ازل شده
\\
از نیم ذره پرتو خورشید روی تو
&&
ارواح حال کرده و اجسام حل شده
\\
جان‌ها ز راه حلق برافکنده خویشتن
&&
در حلقه‌های زلف تو صاحب محل شده
\\
ترک رخت که هندوک اوست آفتاب
&&
آورده خط به خون من و در عمل شده
\\
بر توچون من به دل نگریدم روا مدار
&&
آبی که می‌خورم ز تو با خون بدل شده
\\
ای از کمال روی تو نقصان گرفته کفر
&&
وز کافری زلف تو در دین خلل شده
\\
چون دیده‌ام نزول تو در خون جان خویش
&&
در خون جان خویشتنم زین قبل شده
\\
در وصف تو فرید که از چاکران توست
&&
سلطان عالم است بدین یک غزل شده
\\
\end{longtable}
\end{center}
