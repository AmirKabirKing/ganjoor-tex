\begin{center}
\section*{غزل شماره ۴۲۹: چون دربسته است درج ناپدیدش}
\label{sec:429}
\addcontentsline{toc}{section}{\nameref{sec:429}}
\begin{longtable}{l p{0.5cm} r}
چون دربسته است درج ناپدیدش
&&
به یک بوسه توان کرد کلیدش
\\
شکر دارد لبش هرگز نمیری
&&
اگر یک ذره بتوانی چشیدش
\\
ندید از خود سر یک موی بر جای
&&
کسی کز دور و از نزدیک دیدش
\\
مگر طراری بسیار می‌کرد
&&
کمند طره‌اش زان سر بریدش
\\
اگر نبود کمند طرهٔ او
&&
که یارد سوی خود هرگز کشیدش
\\
اگرچه او جهان بفروخت بر من
&&
به صد جان جان پرخونم خریدش
\\
ز جان بیزار شو در عشق جانان
&&
اگر خواهی به جای جان گزیدش
\\
دلم جایی رسید از عشق رویش
&&
که کار از غم به جان خواهد رسیدش
\\
اگر بر گویم ای عطار آن غم
&&
کزو دل خورد نتوانی شنیدش
\\
\end{longtable}
\end{center}
