\begin{center}
\section*{غزل شماره ۱۴۸: کشتی عمر ما کنار افتاد}
\label{sec:148}
\addcontentsline{toc}{section}{\nameref{sec:148}}
\begin{longtable}{l p{0.5cm} r}
کشتی عمر ما کنار افتاد
&&
رخت در آب رفت و کار افتاد
\\
موی همرنگ کفک دریا شد
&&
وز دهان در شاهوار افتاد
\\
روز عمری که بیخ بر باد است
&&
با سر شاخ روزگار افتاد
\\
سر به ره در نهاد سیل اجل
&&
شورشی سخت در حصار افتاد
\\
مستییی بود عهد برنایی
&&
این زمان کار با خمار افتاد
\\
چون به مقصد رسم که بر سر راه
&&
خر نگونسار گشت و بار افتاد
\\
گل چگویم ز گلستان جهان
&&
که به یک گل هزار خار افتاد
\\
هر که در گلستان دنیا خفت
&&
پای او در دهان مار افتاد
\\
هر که یک دم شمرد در شادی
&&
در غم و رنج بی شمار افتاد
\\
بی قراری چرا کنی چندین
&&
چه کنی چون چنین قرار افتاد
\\
چه توان کرد اگر ز سکهٔ عمر
&&
نقد عمر تو کم عیار افتاد
\\
تو مزن دم خموش باش خموش
&&
که نه این کار اختیار افتاد
\\
گر نبودی امید، وای دلم
&&
لیک عطار امیدوار افتاد
\\
\end{longtable}
\end{center}
