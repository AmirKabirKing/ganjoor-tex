\begin{center}
\section*{غزل شماره ۱۰۰: شمع رویت را دلم پروانه‌ای است}
\label{sec:100}
\addcontentsline{toc}{section}{\nameref{sec:100}}
\begin{longtable}{l p{0.5cm} r}
شمع رویت را دلم پروانه‌ای است
&&
لیک عقل از عشق چون بیگانه‌ای است
\\
پر زنان در پیش شمع روی تو
&&
جان ناپروای من پروانه‌ای است
\\
بر سر موی است جان کز دیرگاه
&&
یک سر موی توام در شانه‌ای است
\\
زلف تو زنار خواهم کرد از آنک
&&
هر شکن از زلف تو بتخانه‌ای است
\\
واندران بتخانه درد عشق را
&&
جان خون آلود من پیمانه‌ای است
\\
وصل تو گنجی است پنهان از همه
&&
هر که گوید یافتم دیوانه‌ای است
\\
در خرابات خرابی می‌روم
&&
زانکه گر گنجی است در ویرانه‌ای است
\\
مرغ آدم دانهٔ وصل تو جست
&&
لاجرم در بند دام از دانه‌ای است
\\
خفته‌ای کز وصل تو گوید سخن
&&
خواب خوش بادش که خوش افسانه‌ای است
\\
وصلت آن کس یافت کز خود شد فنا
&&
هر که فانی شد ز خود مردانه‌ای است
\\
گر مرا در عشق خود فانی کنی
&&
باقیت بر جان من شکرانه‌ای است
\\
بیدقی عطار در عشق تو راند
&&
گر به فرزینی رسد فرزانه‌ای است
\\
\end{longtable}
\end{center}
