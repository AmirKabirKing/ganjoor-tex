\begin{center}
\section*{غزل شماره ۵۵۲: بجز غم خوردن عشقت غمی دیگر نمی‌دانم}
\label{sec:552}
\addcontentsline{toc}{section}{\nameref{sec:552}}
\begin{longtable}{l p{0.5cm} r}
بجز غم خوردن عشقت غمی دیگر نمی‌دانم
&&
که شادی در همه عالم ازین خوشتر نمی‌دانم
\\
گر از عشقت برون آیم به ما و من فرو نایم
&&
ولیکن ما و من گفتن به عشقت در نمی‌دانم
\\
ز بس کاندر ره عشق تو از پای آمدم تا سر
&&
چنان بی پا و سرگشتم که پای از سر نمی‌دانم
\\
به هر راهی که دانستم فرو رفتم به بوی تو
&&
کنون عاجز فرو ماندم رهی دیگر نمی‌دانم
\\
به هشیاری می از ساغر جدا کردن توانستم
&&
کنون از غایت مستی می از ساغر نمی‌دانم
\\
به مسجد بتگر از بت باز می‌دانستم و اکنون
&&
درین خمخانهٔ رندان بت از بتگر نمی‌دانم
\\
چو شد محرم ز یک دریا همه نامی که دانستم
&&
درین دریای بی نامی دو نام‌آور نمی‌دانم
\\
یکی را چون نمی‌دانم سه چون دانم که از مستی
&&
یکی راه و یکی رهرو یکی رهبر نمی‌دانم
\\
کسی کاندر نمکسار اوفتد گم گردد اندر وی
&&
من این دریای پر شور از نمک کمتر نمی‌دانم
\\
دل عطار انگشتی سیه رو بود و این ساعت
&&
ز برق عشق آن دلبر به جز اخگر نمی‌دانم
\\
\end{longtable}
\end{center}
