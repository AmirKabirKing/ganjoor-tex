\begin{center}
\section*{غزل شماره ۴۸۷: دوش درون صومعه، دیر مغانه یافتم}
\label{sec:487}
\addcontentsline{toc}{section}{\nameref{sec:487}}
\begin{longtable}{l p{0.5cm} r}
دوش درون صومعه، دیر مغانه یافتم
&&
راهنمای دیر را، پیر یگانه یافتم
\\
چون بر پیر در شدم، پیر ز خویش رفته بود
&&
کز می عشق پیر را، مست شبانه یافتم
\\
از طلبی که داشتم، چون بنشستم اندکی
&&
از کف پیر میکده، درد مغانه یافتم
\\
راست که درد خورده شد، موج بخاست از دلم
&&
تا ز دو چشم خون فشان، سیل روانه یافتم
\\
گرچه امام دین بدم، تا که به دیر در شدم
&&
در بن دیر خویش را، رند زمانه یافتم
\\
نعره‌زنان برون شدم، دلق و سجاده سوختم
&&
طاعت و زاهدی خود، زیر میانه یافتم
\\
چون دل من به نیستی، حلقه نشین دیر شد
&&
دشمن جان خویش را، در بن خانه یافتم
\\
بی سر و سروری شدم، قبلهٔ کافری شدم
&&
رند و قلندری شدم، زهد فسانهٔافتم
\\
چون بنمود ناگهم، آینهٔ وجود روی
&&
ذره به ذره را درو، عشق نشانه یافتم
\\
عاشق و یار دایما، در دو جهان هموست بس
&&
زانکه خیال آب و گل، جمله بهانه یافتم
\\
نه الم فراق را، هیچ دوا رقم زدم
&&
نه ره دور عشق را، هیچ کرانه یافتم
\\
در ره عشق چون روم، چون ره بی نهایت است
&&
خاصه که پیش هر قدم، چاه و ستانه یافتم
\\
گر تو به عشق فی‌المثل، عیسی وقتی ای فرید
&&
لاف مزن چو رهزنت، سوزن و شانه یافتم
\\
\end{longtable}
\end{center}
