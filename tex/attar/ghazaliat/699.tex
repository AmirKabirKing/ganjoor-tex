\begin{center}
\section*{غزل شماره ۶۹۹: ای دو عالم یک فروغ از روی تو}
\label{sec:699}
\addcontentsline{toc}{section}{\nameref{sec:699}}
\begin{longtable}{l p{0.5cm} r}
ای دو عالم یک فروغ از روی تو
&&
هشت جنت خاک‌بوس کوی تو
\\
هر دو عالم را درین چاه حدوث
&&
تا ابد حبل‌المتین یک موی تو
\\
هر دو عالم گرچه عالی اوفتاد
&&
یک سر موی است پیش روی تو
\\
در رهت تا حشر دو سرگشته‌اند
&&
روز رومی و شب هندوی تو
\\
بس که بر پهلو بگردید آفتاب
&&
تا شود یک ذره هم‌زانوی تو
\\
پس برفت و دید و روی آن ندید
&&
کو نهد از بیم گامی سوی تو
\\
آفتاب آخر چه سنجد چون دو کون
&&
ذره است آنجا که آید روی تو
\\
چون شکستی چرخ گردان را کمان
&&
کی تواند گشت هم‌بازوی تو
\\
جان خود از اندیشهٔ تو محو گشت
&&
چون شود اندیشه هم‌پهلوی تو
\\
حقهٔ گردون چرا پر لولوی است
&&
از فروغ حقهٔ لولوی تو
\\
صد هزاران جادوان را صف شکست
&&
یک مژه از نرگس جادوی تو
\\
همچو ابروی تواش چشمی رسید
&&
چشم هر که افتاد بر ابروی تو
\\
شعر بس نیکو از آن گوید فرید
&&
کو بسوخت از روی بس نیکوی تو
\\
\end{longtable}
\end{center}
