\begin{center}
\section*{غزل شماره ۴۴۳: ای از همه بیش و از همه پیش}
\label{sec:443}
\addcontentsline{toc}{section}{\nameref{sec:443}}
\begin{longtable}{l p{0.5cm} r}
ای از همه بیش و از همه پیش
&&
از خود همه دیده وز همه خویش
\\
در ششدر خاک و خون فتاده
&&
در وصف تو عقل حکمت اندیش
\\
در عالم عشق عاشقان را
&&
قربان شدن است در رهت کیش
\\
هر دم که زنند عاشقانت
&&
بی یاد تو در دهن شود نیش
\\
درویش که لاف معرفت زد
&&
از عجز نبود آن سخن پیش
\\
در هر دو جهان ز خجلت تو
&&
زآن است سیاه‌روی درویش
\\
چون فقر سرای عاشقان است
&&
عاشق شو و از وجود مندیش
\\
در عشق وجودت ار عدم شد
&&
دولت نبود تو را ازین بیش
\\
عطار ز عشق او فنا شو
&&
تا باز رهی ازین دل ریش
\\
\end{longtable}
\end{center}
