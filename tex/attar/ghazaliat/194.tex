\begin{center}
\section*{غزل شماره ۱۹۴: درد من از عشق تو درمان نبرد}
\label{sec:194}
\addcontentsline{toc}{section}{\nameref{sec:194}}
\begin{longtable}{l p{0.5cm} r}
درد من از عشق تو درمان نبرد
&&
زانکه دلم خون شد و فرمان نبرد
\\
دل که به جان آمدهٔ درد توست
&&
درد بسی برد که درمان نبرد
\\
جان نبرم از تو من خسته‌دل
&&
کانکه به تو داد دل او جان نبرد
\\
هر که پریشان نشد از زلف تو
&&
بویی از آن زلف پریشان نبرد
\\
تا به ابد گمره جاوید ماند
&&
هر که به تو راه ز پیشان نبرد
\\
پاک‌بری تا دو جهان در نباخت
&&
آنچه که می‌جست ز تو آن نبرد
\\
پاک توان باخت درین ره که کس
&&
دست درین راه به دستان نبرد
\\
گرچه به سر گشت فلک قرن‌ها
&&
یک نفس این راه به پایان نبرد
\\
چرخ چو از خویش نیامد به سر
&&
واقعهٔ عشق تو پی زان نبرد
\\
کی ببرم وصل تو دست تهی
&&
هیچ ملخ ملک سلیمان نبرد
\\
آه که اندر ظلمات جهان
&&
مرده‌دلی چشمهٔ حیوان نبرد
\\
تا که نشد مات فرید از دو کون
&&
نرد غم عشق تو آسان نبرد
\\
\end{longtable}
\end{center}
