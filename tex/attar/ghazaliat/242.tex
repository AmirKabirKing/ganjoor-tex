\begin{center}
\section*{غزل شماره ۲۴۲: مرد ره عشق تو از دامن تر ترسد}
\label{sec:242}
\addcontentsline{toc}{section}{\nameref{sec:242}}
\begin{longtable}{l p{0.5cm} r}
مرد ره عشق تو از دامن تر ترسد
&&
آن کس که بود نامرد از دادن سر ترسد
\\
گر با تو دوصد دریا آتش بودم در ره
&&
نه دل ز خود اندیشد نه جان ز خطر ترسد
\\
جانی که بر افروزد از شمع جمال تو
&&
می‌دان که ز پروانه کفر است اگر ترسد
\\
جایی که جگر سوزد مردان و جگرخواران
&&
در خون جگر میرد هر کو ز جگر ترسد
\\
گفتی دلت از هجرم می‌ترسد و می‌سوزد
&&
بی وصل تو هر ساعت دل‌سوخته‌تر ترسد
\\
از آه دل عطار آخر به نمی‌ترسی
&&
کانکس که خبر دارد از آه سحر ترسد
\\
\end{longtable}
\end{center}
