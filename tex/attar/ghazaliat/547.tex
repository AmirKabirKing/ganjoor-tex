\begin{center}
\section*{غزل شماره ۵۴۷: خویش را چند ز اندیشه به سر گردانم}
\label{sec:547}
\addcontentsline{toc}{section}{\nameref{sec:547}}
\begin{longtable}{l p{0.5cm} r}
خویش را چند ز اندیشه به سر گردانم
&&
وز تحیر دل خود زیر و زبر گردانم
\\
دل من سوختهٔ حیرت گوناگون است
&&
تا کی از فکرت خود سوخته‌تر گردانم
\\
چون درین راه به یک موی خطر نیست مرا
&&
پس چرا خاطر خود گرد خطر گردانم
\\
می نیاید ز جهان هم نفسی در نظرم
&&
گرچه بسیار ز هر سوی نظر گردانم
\\
چون ز دلتنگی و غم در جگرم آب نماند
&&
چند بر چهره ز غم خون جگر گردانم
\\
نیست در مذهب من هیچ به از تنهایی
&&
گر بسی بنگرم و مسئله برگردانم
\\
نان خشکم بود و گر به تکلف بزیم
&&
از دو چشم آب برو ریزم و تر گردانم
\\
آری ای دوست به جز دانهٔ خود نتوان خورد
&&
خویش را فی‌المثل ار مرغ بپر گردانم
\\
تا کی از غصه و غم غصه و غم ای عطار
&&
سر فرو پوش که سرگشته و سرگردانم
\\
\end{longtable}
\end{center}
