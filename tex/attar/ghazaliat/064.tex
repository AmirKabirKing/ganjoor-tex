\begin{center}
\section*{غزل شماره ۶۴: چه رخساره که از بدر منیر است}
\label{sec:064}
\addcontentsline{toc}{section}{\nameref{sec:064}}
\begin{longtable}{l p{0.5cm} r}
چه رخساره که از بدر منیر است
&&
لبش شکر فروش جوی شیر است
\\
سر هر موی زلفش از درازی
&&
جهان سرنگون را دستگیر است
\\
قمر ماند از خط او پای در قیر
&&
که در گرد خطش هم جوی قیر است
\\
خطا گفتم مگر مشک ختاست او
&&
که در پیرامن بدر منیر است
\\
خط نو خیزش از سبزی جوان است
&&
که کمتر خط پیشش عقل پیر است
\\
نیاید در ضمیر کس که آن خط
&&
چگونه نوبهاری در ضمیر است
\\
جهان جان سزای وصل او هست
&&
که او در جنب وصل او حقیر است
\\
کجا زو بر تواند خورد عاشق
&&
کزو ناز است و از عاشق نفیر است
\\
مرا از جان گریز است ار بگویم
&&
که یک ساعت از آن دلبر گزیر است
\\
مکن ای عشق شمع خوبان ناز چندین
&&
که شمع حسن خوبان زود میر است
\\
فرید یک دلت را یک شکر ده
&&
که در صاحب نصابی او حقیر است
\\
\end{longtable}
\end{center}
