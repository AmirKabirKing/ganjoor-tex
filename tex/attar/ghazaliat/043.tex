\begin{center}
\section*{غزل شماره ۴۳: بیا که قبلهٔ ما گوشهٔ خرابات است}
\label{sec:043}
\addcontentsline{toc}{section}{\nameref{sec:043}}
\begin{longtable}{l p{0.5cm} r}
بیا که قبلهٔ ما گوشهٔ خرابات است
&&
بیار باده که عاشق نه مرد طامات است
\\
پیاله‌ای‌دو به من ده که صبح پرده درید
&&
پیاده‌ای‌دو فرو کن که وقت شه‌مات است
\\
در آن مقام که دلهای عاشقان خون شد
&&
چه جای دردفروشان دیر آفات است
\\
کسی که دیرنشین مغانست پیوسته
&&
چه مرد دین و چه شایستهٔ عبادات است
\\
مگو ز خرقه و تسبیح ازانکه این دل مست
&&
میان ببسته به زنار در مناجات است
\\
ز کفر و دین و ز نیک و بد و ز علم و عمل
&&
برون گذر که برون زین بسی مقامات است
\\
اگر دمی به مقامات عاشقی برسی
&&
شود یقینت که جز عاشقی خرافات است
\\
چه داند آنکه نداند که چیست لذت عشق
&&
از آنکه لذت عاشق ورای لذات است
\\
مقام عاشق و معشوق از دو کون برون است
&&
که حلقهٔ در معشوق ما سماوات است
\\
بنوش درد و فنا شو اگر بقا خواهی
&&
که زادراه فنا دردی خرابات است
\\
به کوی نفی فرو شو چنان که برنایی
&&
که گرد دایرهٔ نفی عین اثبات است
\\
نگه مکن به دو عالم از آنکه در ره دوست
&&
هر آنچه هست به جز دوست عزی و لات است
\\
مخند از پی مستی که بر زمین افتد
&&
که آن سجود وی از جملهٔ مناجات است
\\
اگرچه پاک‌بری مات هر گدایی شو
&&
که شاه نطع یقین آن بود که شهمات است
\\
بباز هر دو جهان و ممان که سود کنی
&&
از آنکه در ره ناماندنت مباهات است
\\
ز هر دو کون فنا شود درین ره ای عطار
&&
که باقی ره عشاق فانی ذات است
\\
\end{longtable}
\end{center}
