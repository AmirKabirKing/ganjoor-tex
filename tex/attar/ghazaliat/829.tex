\begin{center}
\section*{غزل شماره ۸۲۹: به وادییی که درو گوی راه سر بینی}
\label{sec:829}
\addcontentsline{toc}{section}{\nameref{sec:829}}
\begin{longtable}{l p{0.5cm} r}
به وادییی که درو گوی راه سر بینی
&&
به هر دمی که زنی ماتمی دگر بینی
\\
ز هرچه می‌دهدت روزگار عمر بهست
&&
ولی چه سود که آن نیز بر گذر بینی
\\
ز دولتی به چه نازی که تا که چشم زنی
&&
اثر نبینی ازو در جهان اگر بینی
\\
مساز قبهٔ زرین که تیز شمشیر است
&&
سزای قبهٔ زرین که بر سپر بینی
\\
اگر سلوک کنی صد هزار قرن هنوز
&&
چو مرد رهگذری جمله رهگذر بینی
\\
چو هر چه هست همه اصل خویش می‌جویند
&&
ز شوق جملهٔ ذرات در سفر بینی
\\
چو کل اصل جهان از یک اصل خاسته‌اند
&&
سزد که کل جهان را به یک نظر بینی
\\
مکن ز نفس تکبر تو چشم باز گشای
&&
که تا همه شکم خاک سیم و زر بینی
\\
به باد بر زبر خاک گنجه چند کنی
&&
که تا که رنجه شوی خاک بر زبر بینی
\\
چگونه پای نهی در خزانه‌ای که درو
&&
به هر سویی که روی صد هزار سر بینی
\\
نه لحظه‌ای ز همه خفتگان خبر شنوی
&&
نه ذره‌ای ز همه رفتگان اثر بینی
\\
ز بس که خون جگر می‌فروخورد به زمین
&&
زمین ز خون جگر بسته چون جگر بینی
\\
اگر جهان همه از پس کنی نمی‌دانم
&&
که در جهان ز دریغا چه بیشتر بینی
\\
درین مصیبت و سرگشتگی محال بود
&&
که در زمانه چو عطار نوحه‌گر بینی
\\
\end{longtable}
\end{center}
