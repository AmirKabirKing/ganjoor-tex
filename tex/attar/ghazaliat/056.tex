\begin{center}
\section*{غزل شماره ۵۶: ره عشاق راهی بی‌کنار است}
\label{sec:056}
\addcontentsline{toc}{section}{\nameref{sec:056}}
\begin{longtable}{l p{0.5cm} r}
ره عشاق راهی بی‌کنار است
&&
ازین ره دور اگر جانت به کار است
\\
وگر سیری ز جان در باز جان را
&&
که یک جان را عوض آنجا هزار است
\\
تو هر وقتی که جانی برفشانی
&&
هزاران جان نو بر تو نثار است
\\
وگر در یک قدم صد جان دهندت
&&
نثارش کن که جان‌ها بی‌شمار است
\\
چه خواهی کرد خود را نیم‌جانی
&&
چو دایم زندگی تو بیاراست
\\
کسی کز جان بود زنده درین راه
&&
ز جرم خود همیشه شرمسار است
\\
درآمد دوش در دل عشق جانان
&&
خطابم کرد کامشب روز بار است
\\
کنون بی‌خود بیا تا بار یابی
&&
که شاخ وصل بی باران به بار است
\\
چو شد فانی دلت در راه معشوق
&&
قرار عشق جانان بی‌قرار است
\\
تو را اول قدم در وادی عشق
&&
به زارش کشتن است آنگاه دار است
\\
وزان پس سوختن تا هم بوینی
&&
که نور عاشقان در مغز نار است
\\
چو خاکستر شوی و ذره گردی
&&
به رقص آیی که خورشید آشکار است
\\
تو را از کشتن و وز سوختن هم
&&
چه غم چون آفتابت غمگسار است
\\
کسی سازد رسن از نور خورشید
&&
که اندر هستی خود ذره‌وار است
\\
کسی کو در وجود خویش ماندست
&&
مده پندش که بندش استوار است
\\
درین مجلس کسی باید که چون شمع
&&
بریده سر نهاده بر کنار است
\\
شبانروزی درین اندیشه عطار
&&
چو گل پر خون و چون نرگس نزار است
\\
\end{longtable}
\end{center}
