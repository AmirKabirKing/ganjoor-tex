\begin{center}
\section*{غزل شماره ۷۱۲: ای روی همچو ماهت یک پرده بر گرفته}
\label{sec:712}
\addcontentsline{toc}{section}{\nameref{sec:712}}
\begin{longtable}{l p{0.5cm} r}
ای روی همچو ماهت یک پرده بر گرفته
&&
جان های بی قراران فریاد در گرفته
\\
در پیش نور رویت پیران شست ساله
&&
با صد هزار خجلت ایمان ز سر گرفته
\\
عشقت به دلربایی بگشاده دست بر ما
&&
ناگاه جان و دل را بس بی خبر گرفته
\\
دل هر دم از فراقت داغی دگر کشیده
&&
جان هر دم از کمالت راهی دگر گرفته
\\
از بس که رهزنانند اندر رهت ز غیرت
&&
هر ذره ذرهٔ تو صد راه بر گرفته
\\
چون آفتاب رویت بر جان فکند پرتو
&&
عشقت به جان رسیده دل را به‌در گرفته
\\
عشق تو چون همایی پر بر کشیده از هم
&&
جان‌های عاشقان را در زیر پر گرفته
\\
مستان عشق هر شب همچون صبوح خیزان
&&
بر آرزوی رویت راه سحر گرفته
\\
آنجا که حسن رویت بوی نمک نموده
&&
صحرای هر دو عالم خون جگر گرفته
\\
عطار در غم تو شادی هر دو عالم
&&
هم از نظر فکنده هم مختصر گرفته
\\
\end{longtable}
\end{center}
