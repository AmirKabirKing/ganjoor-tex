\begin{center}
\section*{غزل شماره ۱۵۵: عشق تو پرده، صد هزار نهاد}
\label{sec:155}
\addcontentsline{toc}{section}{\nameref{sec:155}}
\begin{longtable}{l p{0.5cm} r}
عشق تو پرده، صد هزار نهاد
&&
پرده در پرده بی شمار نهاد
\\
پس هر پرده عالمی پر درد
&&
گه نهان و گه آشکار نهاد
\\
صد جهان خون و صد جهان آتش
&&
پس هر پرده استوار نهاد
\\
پرده بازی چنان عجایب کرد
&&
که یکی در یکی هزار نهاد
\\
پردهٔ دل به یک زمان بگرفت
&&
پرده بر روی اختیار نهاد
\\
کرد با دل ز جور آنچه مپرس
&&
جرم بر جان بی قرار نهاد
\\
جان مضطر چو خاک راهش گشت
&&
روی بر خاک اضطرار نهاد
\\
شیرمرد همه جهان بودم
&&
عشق بر دست من نگار نهاد
\\
که بداند که دور از رویت
&&
گل روی توام چه خار نهاد
\\
دوش آمد خیال تو سحری
&&
تا مرا در هزار کار نهاد
\\
همچو لاله فکند در خونم
&&
بر دلم داغ انتظار نهاد
\\
سر من همچو شمع باز برید
&&
پس بیاورد و در کنار نهاد
\\
چون همی بازگشت از بر من
&&
درد هجرم به یادگار نهاد
\\
هر زمان عقبه‌ای ز درد فراق
&&
پیش عطار دل فگار نهاد
\\
\end{longtable}
\end{center}
