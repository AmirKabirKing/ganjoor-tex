\begin{center}
\section*{غزل شماره ۹۱: عشق جمال جانان دریای آتشین است}
\label{sec:091}
\addcontentsline{toc}{section}{\nameref{sec:091}}
\begin{longtable}{l p{0.5cm} r}
عشق جمال جانان دریای آتشین است
&&
گر عاشقی بسوزی زیرا که راه این است
\\
جایی که شمع رخشان ناگاه بر فروزند
&&
پروانه چون نسوزد کش سوختن یقین است
\\
گر سر عشق خواهی از کفر و دین گذر کن
&&
کانجا که عشق آمد چه جای کفر و دین است
\\
عاشق که در ره آید اندر مقام اول
&&
چون سایه‌ای به خواری افتاده در زمین است
\\
چون مدتی برآید سایه نماند اصلا
&&
کز دور جایگاهی خورشید در کمین است
\\
چندین هزار رهرو دعوی عشق کردند
&&
برخاتم طریقت منصور چون نگین است
\\
هرکس که در معنی زین بحر بازیابد
&&
در ملک هر دو عالم جاوید نازنین است
\\
کاری قوی است عالی کاندر ره طریقت
&&
بر هر هزار سالی یک مرد راه‌بین است
\\
تو مرد ره چه دانی زیرا که مرد ره را
&&
اول قدم درین ره بر چرخ هفتمین است
\\
عطار اندرین ره جایی فتاد کانجا
&&
برتر ز جسم و جان است بیرون ز مهر و کین است
\\
\end{longtable}
\end{center}
