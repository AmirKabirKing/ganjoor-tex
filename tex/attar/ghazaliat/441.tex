\begin{center}
\section*{غزل شماره ۴۴۱: دلا در سر عشق از سر میندیش}
\label{sec:441}
\addcontentsline{toc}{section}{\nameref{sec:441}}
\begin{longtable}{l p{0.5cm} r}
دلا در سر عشق از سر میندیش
&&
بده جان و ز جان دیگر میندیش
\\
چو سر در کار و جان در یار بازی
&&
خوشی خویش ازین خوشتر میندیش
\\
رسن از زلف جانان ساز جان را
&&
وزین فیروزه‌گون چنبر میندیش
\\
چو پروانه گرت پر سوزد آن شمع
&&
به پهلو می‌رو و از پر میندیش
\\
چو عشاق را نه کفر است و نه ایمان
&&
ز کار مؤمن و کافر میندیش
\\
مقامرخانهٔ رندان طلب کن
&&
سر اندر باز و از افسر میندیش
\\
چو سر در باختی بشناختی سر
&&
چو سر بشناختی از سر میندیش
\\
همه بتها چو ابراهیم بشکن
&&
هم از آذر هم از آزر میندیش
\\
چو آن حلاج برکش پنبه از گوش
&&
هم از دار و هم از منبر میندیش
\\
اگر عشقت بسوزد بر سر دار
&&
دهد بر باد خاکستر میندیش
\\
چو انگشت سیه‌رو گشت اخگر
&&
تو آن انگشت جز اخگر میندیش
\\
چو می با ساغر صافی یکی گشت
&&
دویی گم شد می و ساغر میندیش
\\
چو مس در زر گدازد مرد صراف
&&
مس آنجا زر بود جز زر میندیش
\\
مشو اینجا حلولی لیکن این رمز
&&
جز استغراق در دلبر میندیش
\\
اگر خواهی که گوهر بیابی
&&
درین دریا به جز گوهر میندیش
\\
بسی کشتی جان بر خشک راندی
&&
تو کشتی ران ز خشک و تر میندیش
\\
چنان فربه نه‌ای تو هم درین کار
&&
اگر صیدی فتد لاغر میندیش
\\
چو تو دایم به پهنا می‌شوی باز
&&
ازین وادی پهناور میندیش
\\
درین دریای پر گرداب حسرت
&&
کس از عطار حیران‌تر میندیش
\\
\end{longtable}
\end{center}
