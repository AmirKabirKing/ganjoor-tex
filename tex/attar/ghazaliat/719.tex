\begin{center}
\section*{غزل شماره ۷۱۹: ترسا بچه‌ای دیدم زنار کمر کرده}
\label{sec:719}
\addcontentsline{toc}{section}{\nameref{sec:719}}
\begin{longtable}{l p{0.5cm} r}
ترسا بچه‌ای دیدم زنار کمر کرده
&&
در معجزهٔ عیسی صد درس ز بر کرده
\\
با زلف چلیپاوش بنشسته به مسجد خوش
&&
وز قبلهٔ روی خود محراب دگر کرده
\\
از تختهٔ سیمینش یعنی که بناگوشش
&&
خورشید خجل گشته رخساره چو زر کرده
\\
از جادویی چشمش برخاسته صد غوغا
&&
تا بر سر بازاری یکبار گذر کرده
\\
چون مه به کله‌داری پیروزه قبا بسته
&&
زنار سر زلفش عشاق کمر کرده
\\
روزی که ز بد کردن بگرفت دلش کلی
&&
بگذاشته دست از بد صد بار بتر کرده
\\
صد چشمهٔ حیوان است اندر لب سیرابش
&&
وین عاشق بی دل را بس تشنه جگر کرده
\\
دوش آمد پیر ما در صومعه بد تنها
&&
گفت ای ز سر عجبی در خویش نظر کرده
\\
از خویش پرستیدن در صومعه بنشسته
&&
خلق همه عالم را از خویش خبر کرده
\\
بگریخته نفس تو از یار ز نامردی
&&
چون بار گران دیده از خلق حذر کرده
\\
برخیزی اگر مردی در شیوهٔ ما آیی
&&
تا شیوهٔ ما بینی در سنگ اثر کرده
\\
یک دردی درد ما در عالم رسوایی
&&
صد زاهد خودبین را با دامن تر کرده
\\
در حلقه چو دیدی خود دردی خور و مستی کن
&&
وانگاه ببین خود را از حلقه به در کرده
\\
چون کوری قرایان عطار عیان دیده
&&
بینایی پیر خود صد نوع سمر کرده
\\
\end{longtable}
\end{center}
