\begin{center}
\section*{غزل شماره ۵۷۹: دردا که ز یک همدم آثار نمی‌بینم}
\label{sec:579}
\addcontentsline{toc}{section}{\nameref{sec:579}}
\begin{longtable}{l p{0.5cm} r}
دردا که ز یک همدم آثار نمی‌بینم
&&
دل باز نمی‌یابم دلدار نمی‌بینم
\\
در عالم پر حسرت بسیار بگردیدم
&&
از خیل وفاداران دیار نمی‌بینم
\\
در چار سوی عالم شش گوشهٔ توتویش
&&
یک دوست نمی‌بینم یک یار نمی‌بینم
\\
بسیار وفا جستم اندک قدم از هرکس
&&
در روی زمین اندک بسیار نمی‌بینم
\\
چندان که در آن وادی کردم طلب یک گل
&&
در عرصهٔ این وادی جز خار نمی‌بینم
\\
تا چند درین وادی بر جان و دلم لرزم
&&
کانجا به دو جود جان را مقدار نمی‌بینم
\\
تا چند ز نادانی دیوان جهان دارم
&&
چون مورد درین دیوان جز مار نمی‌بینم
\\
هر روز ازین دیوان صد غم برما آید
&&
دردا که درین صد غم غمخوار نمی‌بینم
\\
گر زانکه اثر بودی در روی زمین کس را
&&
زانگونه اثر کم شد کاثار نمی‌بینم
\\
عطار دلت بر کن از کار جهان کلی
&&
کز کار جهان یک دل بر کار نمی‌بینم
\\
\end{longtable}
\end{center}
