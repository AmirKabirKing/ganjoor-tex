\begin{center}
\section*{غزل شماره ۸۲: تا چشم برندوزی از هرچه در جهان است}
\label{sec:082}
\addcontentsline{toc}{section}{\nameref{sec:082}}
\begin{longtable}{l p{0.5cm} r}
تا چشم برندوزی از هرچه در جهان است
&&
در چشم دل نیاید چیزی که مغز جان است
\\
در عشق درد خود را هرگز کران نبینی
&&
زیرا که عشق جانان دریای بی‌کران است
\\
تا چند جویی آخر از جان نشان جانان
&&
در باز جان و دل را کین راه بی نشان است
\\
تا کی ز هستی تو کز هستی تو باقی
&&
گر نیست بیش مویی صد کوه در میان است
\\
هر جان که در ره آمد لاف یقین بسی زد
&&
لیکن نصیب جان زان پندار یا گمان است
\\
اندیشه کن تو با خود تا در دو کون هرگز
&&
یک قطره آب تیره دریا کجا بدان است
\\
رند شراب خواره، چون مست مست گردد
&&
گوید که هر دو عالم در حکم من روان است
\\
لیکن چو باهش آید در خود کند نگاهی
&&
حالی خجل بماند داند که نه چنان است
\\
عطار مست عشقی از عشق چند لافی
&&
گر طالبی فنا شو مطلوب بس عیان است
\\
\end{longtable}
\end{center}
