\begin{center}
\section*{غزل شماره ۸۰۸: خاک کوی توام تو می‌دانی}
\label{sec:808}
\addcontentsline{toc}{section}{\nameref{sec:808}}
\begin{longtable}{l p{0.5cm} r}
خاک کوی توام تو می‌دانی
&&
خاک در روی من چه افشانی
\\
سر نگردانم از ره تو دمی
&&
گر به خون صد رهم بگردانی
\\
با چو من کس که ناتوان توام
&&
بتوان کرد هرچه بتوانی
\\
گر به خونم درافکنی ز درت
&&
بر نگیرم ز خاک پیشانی
\\
سر مهر غم تو در دل من
&&
راز عشقت بس است پنهانی
\\
گر به رویم نظر کنی نفسی
&&
همه از روی من فرو خوانی
\\
من ز درمان به جان شدم بیزار
&&
جان من درد توست می‌دانی
\\
گر مرا درد تو نخواهد بود
&&
سر بگردانم از مسلمانی
\\
هیچ درمان مرا مکن هرگز
&&
که نیم جز به درد ارزانی
\\
گفته بودی که دل ز تو ببرم
&&
که ز دلداری و پریشانی
\\
نتوانی که دل ز من ببری
&&
دل چگونه بری چو درمانی
\\
من ز عطار جان بخواهم برد
&&
برهد از هزار حیرانی
\\
\end{longtable}
\end{center}
