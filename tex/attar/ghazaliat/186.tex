\begin{center}
\section*{غزل شماره ۱۸۶: بر در حق هر که کار و بار ندارد}
\label{sec:186}
\addcontentsline{toc}{section}{\nameref{sec:186}}
\begin{longtable}{l p{0.5cm} r}
بر در حق هر که کار و بار ندارد
&&
نزد حق او هیچ اعتبار ندارد
\\
جان به تماشای گلشن در حق بر
&&
خوش بود آن گلشنی که خار ندارد
\\
مست خراب شراب شوق خدا شو
&&
زانکه شراب خدا خمار ندارد
\\
خدمت حق کن به هر مقام که باشی
&&
خدمت مخلوق افتخار ندارد
\\
تا بتند عنکبوت بر در هر غار
&&
پردهٔ عصمت که پود و تار ندارد
\\
ساختن پرده آنچنان ز که آموخت
&&
از در آنکس که پرده‌دار ندارد
\\
تا دل عطار در دو کون فروشد
&&
از پی آن بار بار بار ندارد
\\
\end{longtable}
\end{center}
