\begin{center}
\section*{غزل شماره ۴۰۵: عشق تو مرا ستد ز من باز}
\label{sec:405}
\addcontentsline{toc}{section}{\nameref{sec:405}}
\begin{longtable}{l p{0.5cm} r}
عشق تو مرا ستد ز من باز
&&
وافگند مرا ز جان و تن باز
\\
تا خاص خودم گرفت کلی
&&
می‌نگذارد مرا به من باز
\\
بگرفت مرا چنان که مویی
&&
نتوان آمد به خویشتن باز
\\
آن جامه که از تو جان ما یافت
&&
می نتوان کرد از شکن باز
\\
روزی ز شکن کنند بازش
&&
کز چهرهٔ ما شود کفن باز
\\
کی در تو رسد کسی که جاوید
&&
در راه تو ماند مرد و زن باز
\\
چون در تو نمی‌توان رسیدن
&&
نومید نمی‌توان شدن باز
\\
درد تو رسیدهٔ تمام است
&&
من بی تو دریده پیرهن باز
\\
چون لاف وصال تو می‌زنم من
&&
چون پرده کنم ازین سخن باز
\\
چون می‌دانم که روز آخر
&&
حسرت ماند ز من به تن باز
\\
از قرب تو کان وطنگهم بود
&&
دل مانده ز نفس راهزن باز
\\
عطار از آن وطن فتاده است
&&
او را برسان بدان وطن باز
\\
\end{longtable}
\end{center}
