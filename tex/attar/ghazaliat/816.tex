\begin{center}
\section*{غزل شماره ۸۱۶: ای گشته نهان از همه از بس که عیانی}
\label{sec:816}
\addcontentsline{toc}{section}{\nameref{sec:816}}
\begin{longtable}{l p{0.5cm} r}
ای گشته نهان از همه از بس که عیانی
&&
دیده ز تو بینا و تو از دیده نهانی
\\
گر من طلبم دولت وصلت نتوانم
&&
گر تو بنمایی رخ خویشم بتوانی
\\
شد در سر کار تو همه مایهٔ عمرم
&&
آخر تو چه چیزی که نه سودی نه زیانی
\\
یک چند در اندیشهٔ روی تو نشستم
&&
معلوم نشد خود که چه چیزی به چه مانی
\\
ای جان و جهان نیست به هر جان و جهان هیچ
&&
آخر تو کدامی که نه جانی نه جهانی
\\
دل گفت که جانی و خرد گفت جهانی
&&
چون نیک بدیدم تو نه اینی و نه آنی
\\
تبدیل نداری که توان خواند جهانت
&&
تغییر نداری که توان گفت که جانی
\\
عطار عیان است که محتاج بیان است
&&
گر اهل عیانی به چه در بند عیانی
\\
\end{longtable}
\end{center}
