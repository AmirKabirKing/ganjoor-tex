\begin{center}
\section*{غزل شماره ۳۰۲: چون لبش درج گهر باز کند}
\label{sec:302}
\addcontentsline{toc}{section}{\nameref{sec:302}}
\begin{longtable}{l p{0.5cm} r}
چون لبش درج گهر باز کند
&&
عقل را حاملهٔ راز کند
\\
یارب از عشق شکر خندهٔ او
&&
طوطی روح چه پرواز کند
\\
هیچ کس زهره ندارد که دمی
&&
صفت آن لب دمساز کند
\\
تیرباران همهٔ شادی دل
&&
غم آن غمزهٔ غماز کند
\\
راست کان ترک پریچهره چو صبح
&&
زلف شبرنگ ز رخ باز کند
\\
نتوان گفت که هندوی بصر
&&
از چه زنگی دل آغاز کند
\\
ناز او چون خوشم آید نکند
&&
ور کند ناز به صد ناز کند
\\
ماه رویت چو ز رخ درتابد
&&
ذره را با فلک انباز کند
\\
همه ذرات جهان را رخ تو
&&
همچو خورشید سرافراز کند
\\
وه که دیوانگی عشق تو را
&&
عقل پر حیله چه اعزاز کند
\\
ماه در دق و ورم مانده و باز
&&
بر امید تو تک و تاز کند
\\
گفته بودی که برو ور نروی
&&
زلف من کشتن تو ساز کند
\\
سر نپیچم اگر از هر سر موی
&&
سر زلف تو سرانداز کند
\\
به سخن گرچه منم عیسی دم
&&
جزع تو دعوی ایجاز کند
\\
عنبر زلف تو عطارم کرد
&&
واطلس روی تو بزاز کند
\\
\end{longtable}
\end{center}
