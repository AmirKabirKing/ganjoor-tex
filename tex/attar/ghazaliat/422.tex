\begin{center}
\section*{غزل شماره ۴۲۲: منم اندر قلندری شده فاش}
\label{sec:422}
\addcontentsline{toc}{section}{\nameref{sec:422}}
\begin{longtable}{l p{0.5cm} r}
منم اندر قلندری شده فاش
&&
در میان جماعتی اوباش
\\
همه افسوس خواره و همه رند
&&
همه دردی کش و همه قلاش
\\
ترک نیک و بد جهان گفته
&&
که جهان خواه باش و خواه مباش
\\
دام دیوانگی بگسترده
&&
تا به دام اوفتاده عقل معاش
\\
ساقیا چند خسبی آخر خیز
&&
که سپهرت نمی‌دهد خشخاش
\\
بنشان از دلم غبار به می
&&
که تویی صحن سینه را فراش
\\
گر تو در معرفت شکافی موی
&&
ور زبان تو هست گوهر پاش
\\
یک سر موی بیش و کم نشود
&&
زانچه بنگاشت در ازل نقاش
\\
تو چه دانی که در نهاد کثیف
&&
آفتاب است روح یا خفاش
\\
عاشقی خواه اوفتاده ز شوق
&&
بر سر فرش شمع همچو فراش
\\
چه کنی زاهدی که از سردی
&&
بجهد بیست رش ز بیم رشاش
\\
زاهد خام خویش‌بین هرگز
&&
نشود پخته گر نهی در داش
\\
هست زاهد چو آن دروگر بد
&&
که کند سوی خود همیشه تراش
\\
مرد ایثار باش و هیچ مترس
&&
که نترسد ز مردگان نباش
\\
من نیم خرده گیر و خرده شناس
&&
که ندارم ز خرده هیچ قماش
\\
دور باشید از کسی که مدام
&&
کفر دارد نهفته، ایمان فاش
\\
چون نیم زاهد و نیم فاسق
&&
از چه قومم بدانمی ای کاش
\\
چه خبر داری این دم ای عطار
&&
تا قدم درنهی درین ره باش
\\
\end{longtable}
\end{center}
