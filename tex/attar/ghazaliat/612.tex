\begin{center}
\section*{غزل شماره ۶۱۲: ما در غمت به شادی جان باز ننگریم}
\label{sec:612}
\addcontentsline{toc}{section}{\nameref{sec:612}}
\begin{longtable}{l p{0.5cm} r}
ما در غمت به شادی جان باز ننگریم
&&
در عشق تو به هر دو جهان باز ننگریم
\\
خوش خوش چو شمع ز آتش عشق تو فی‌المثل
&&
گر جان ما بسوخت به جان باز ننگریم
\\
هر طاعتی که خلق جهان کرد و می‌کنند
&&
گر نقد ماست جمله بدان باز ننگریم
\\
سود دو کون در طلبت گر زیان کنیم
&&
ما در طلب به سود و زیان باز ننگریم
\\
گر عین ما شود همه ذرات کاینات
&&
یک ذره ما به عین عیان باز ننگریم
\\
اسرار تو ز کون و مکان چون منزه است
&&
ما تا ابد به کون و مکان باز ننگریم
\\
چون شد یقین ما که تویی اصل هرچه هست
&&
در پردهٔ یقین به گمان باز ننگریم
\\
در کوی تو دو اسبه بتازیم مردوار
&&
هرگز به مرکب و به عنان باز ننگریم
\\
عطار چو کناره گرفت از میان ما
&&
ما از کنار او به میان باز ننگریم
\\
\end{longtable}
\end{center}
