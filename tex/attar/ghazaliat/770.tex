\begin{center}
\section*{غزل شماره ۷۷۰: ای آنکه هیچ جایی آرام جان ندیدی}
\label{sec:770}
\addcontentsline{toc}{section}{\nameref{sec:770}}
\begin{longtable}{l p{0.5cm} r}
ای آنکه هیچ جایی آرام جان ندیدی
&&
رنج جهان کشیدی گنج جهان ندیدی
\\
هرچند جهد کردی کاری به سر نبردی
&&
چندان که پیش رفتی ره را کران ندیدی
\\
زان گوهری که گردون از عشق اوست گردان
&&
قانع شدی به نامی اما نشان ندیدی
\\
مرد شنو چه باشی مردانه رو سخن دان
&&
چه حاصل از شنیدن چون در عیان ندیدی
\\
می‌دان که روز معنی بیرون پرده مانی
&&
گر در درون پرده خود را نهان ندیدی
\\
آن نافه‌ای که جستی هم با تو در گلیم است
&&
تو از سیه گلیمی بویی از آن ندیدی
\\
گر جان بر او فشانی صد جان عوض ستانی
&&
بر جان مگرد چندین انگار جان ندیدی
\\
عمری بپروریدی این نفس سگ صفت را
&&
چه سود چون ز مکرش یک‌دم امان ندیدی
\\
نا آزموده گفتی هستم چنان که باید
&&
لیکن چو آزمودی هرگز چنان ندیدی
\\
افسوس می‌خورم من کافسوس خواره‌ای را
&&
جز هم‌نفس نگفتی جز مهربان ندیدی
\\
تو مرغ بام عرشی در قعر چاه مانده
&&
هم در زمین بمردی هم آسمان ندیدی
\\
آخر چو شیر مردان بر پر ز چاه و رفتی
&&
انگار نفس سگ را در خاکدان ندیدی
\\
دل را به باد دادی وانگه به کام این سگ
&&
یک‌پاره نان نخوردی یک استخوان ندیدی
\\
عطار در غم خود عمرت به آخر آمد
&&
چه سود کز غم خود غیر از زیان ندیدی
\\
\end{longtable}
\end{center}
