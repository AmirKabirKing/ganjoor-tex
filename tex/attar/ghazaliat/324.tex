\begin{center}
\section*{غزل شماره ۳۲۴: عشق را پیر و جوان یکسان بود}
\label{sec:324}
\addcontentsline{toc}{section}{\nameref{sec:324}}
\begin{longtable}{l p{0.5cm} r}
عشق را پیر و جوان یکسان بود
&&
نزد او سود و زیان یکسان بود
\\
هم ز یکرنگی جهان عشق را
&&
نو بهار و مهرگان یکسان بود
\\
زیر او بالا و بالا هست زیر
&&
کش زمین و آسمان یکسان بود
\\
بارگاه عشق همچون دایره است
&&
صد او با آستان یکسان بود
\\
یار اگر سوزد وگر سازد رواست
&&
عاشقان را این و آن یکسان بود
\\
در طریق عاشقان خون ریختن
&&
با حیات جاودان یکسان بود
\\
سایه از کل دان که پیش آفتاب
&&
آشکارا و نهان یکسان بود
\\
کی بود دلدار چون دل ای فرید
&&
باز کی با آشیان یکسان بود
\\
\end{longtable}
\end{center}
