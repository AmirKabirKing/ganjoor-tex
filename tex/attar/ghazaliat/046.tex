\begin{center}
\section*{غزل شماره ۴۶: چون کنم معشوق عیار آمدست}
\label{sec:046}
\addcontentsline{toc}{section}{\nameref{sec:046}}
\begin{longtable}{l p{0.5cm} r}
چون کنم معشوق عیار آمدست
&&
دشنه در کف سوی بازار آمدست
\\
دشنهٔ او تشنهٔ خون دل است
&&
لاجرم خونریز و خونخوار آمدست
\\
همچنان کان پسته می‌ریزد شکر
&&
همچنان آن دشنه خونبار آمدست
\\
هست ترک و من به جان هندوی او
&&
لاجرم با تیغ در کار آمدست
\\
صبحدم هر روز با کرباس و تیغ
&&
پیش تیغ او به زنهار آمدست
\\
آینه بر روی خود می‌داشتست
&&
تا به خود بر عاشق زار آمدست
\\
از وصال او کسی کی برخورد
&&
کو به عشق خود گرفتار آمدست
\\
او ز جمله فارغ است و هر کسی
&&
اندرین دعوی پدیدار آمدست
\\
لیک چون تو بنگری در راه عشق
&&
قسم هر کس محض پندار آمدست
\\
عاشق او و عشق او معشوقه اوست
&&
کیستی تو چون همه یار آمدست
\\
جز فنائی نیست چون می‌بنگرم
&&
آنچه از وی قسم عطار آمدست
\\
\end{longtable}
\end{center}
