\begin{center}
\section*{غزل شماره ۱۷۰: تا زلف تو همچو مار می‌پیچد}
\label{sec:170}
\addcontentsline{toc}{section}{\nameref{sec:170}}
\begin{longtable}{l p{0.5cm} r}
تا زلف تو همچو مار می‌پیچد
&&
جان بی دل و بی قرار می‌پیچد
\\
دل بود بسی در انتظار تو
&&
در هر پیچی هزار می‌پیچد
\\
زان می‌پیچم که تاج را چندین
&&
زلف تو کمندوار می‌پیچد
\\
بس جان که ز پیچ حلقهٔ زلفت
&&
در حلقهٔ بی شمار می‌پیچد
\\
بس دل که ز زلف تابدار تو
&&
چو زلف تو تابدار می‌پیچد
\\
بس تن که ز بار عشق یک مویت
&&
بی روی تو زیر دار می‌پیچد
\\
تو می‌گذری ز ناز بس فارغ
&&
و او بر سر دار زار می‌پیچد
\\
هر دل که شکار زلف تو گردد
&&
جان می‌دهد و چو مار می‌پیچد
\\
ترکانه و چست هندوی زلفت
&&
بس نادره در شکار می‌پیچد
\\
هر دل که ز دام زلف تو بجهد
&&
زان چهرهٔ چون نگار می‌پیچد
\\
چون می‌پیچد فرید بپذیرش
&&
زیرا که به اضطرار می‌پیچد
\\
\end{longtable}
\end{center}
