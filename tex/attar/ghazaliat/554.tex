\begin{center}
\section*{غزل شماره ۵۵۴: زلف و رخت از شام و سحر باز ندانم}
\label{sec:554}
\addcontentsline{toc}{section}{\nameref{sec:554}}
\begin{longtable}{l p{0.5cm} r}
زلف و رخت از شام و سحر باز ندانم
&&
خال و لبت از مشک و شکر باز ندانم
\\
از فرقت رویت ز دل پر شرر خویش
&&
آهی که برآرم ز شرر باز ندانم
\\
روی تو که هرگز ز خیالم نشود دور
&&
از بس که بگریم به نظر باز ندانم
\\
گویی که مرا باز ندانی چو ببینی
&&
شاید چو نمی‌بینمت ار باز ندانم
\\
اشکم که همی از دم سردم چو جگر بست
&&
بر چهرهٔ زردم ز جگر باز ندانم
\\
با پشت دوتا از غم روی تو چنانم
&&
کز دست غمت پای ز سر باز ندانم
\\
زانگاه که عطار تو را تنگ شکر خواند
&&
در وصف تو شعرم ز شکر باز ندانم
\\
\end{longtable}
\end{center}
