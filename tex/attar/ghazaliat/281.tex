\begin{center}
\section*{غزل شماره ۲۸۱: چو ترک سیم برم صبحدم ز خواب درآمد}
\label{sec:281}
\addcontentsline{toc}{section}{\nameref{sec:281}}
\begin{longtable}{l p{0.5cm} r}
چو ترک سیم برم صبحدم ز خواب درآمد
&&
مرا ز خواب برانگیخت و با شراب درآمد
\\
به صد شتاب برون رفت عقل جامه به دندان
&&
چو دید دیده که آن بت به صد شتاب درآمد
\\
چو زلف او دل پر تاب من ببرد به غارت
&&
ز زلف او به دل من هزار تاب درآمد
\\
خراب گشتم و بیخود اگر چه باده نخوردم
&&
چو ترک من ز سر بیخودی خراب درآمد
\\
نهاد شمع و شرابی که شیشه شعله زد از وی
&&
چو باد خورد چو آتش به کار آب درآمد
\\
شراب و شاهد و شمع من و ز گوشهٔ مجلس
&&
همی نسیم گل و نور ماهتاب درآمد
\\
شکست توبهٔ سنگینم آبگینه چنان خوش
&&
کزان خوشی به دل من صد اضطراب درآمد
\\
چو توبهٔ من بی دل شکستی ای بت دلبر
&&
نمک بده ز لبت کز دلم کباب درآمد
\\
بیار باده و زلفت گره مزن به ستیزه
&&
که فتنه از گره زلف تو ز خواب درآمد
\\
شراب نوش که از سرخی رخ چو گل تو
&&
هزار زردی خجلت به آفتاب درآمد
\\
که می‌نماید عطار را رهی که گریزد
&&
که همچو سیل ز هر سو نبید ناب درآمد
\\
\end{longtable}
\end{center}
