\begin{center}
\section*{غزل شماره ۴۱۹: غیرت آمد بر دلم زد دور باش}
\label{sec:419}
\addcontentsline{toc}{section}{\nameref{sec:419}}
\begin{longtable}{l p{0.5cm} r}
غیرت آمد بر دلم زد دور باش
&&
یعنی ای نااهل ازین در دور باش
\\
تو گدایی دور شو از پادشاه
&&
ورنه بر جان تو آید دور باش
\\
گر وصال شاه می‌داری طمع
&&
از وجود خویشتن مهجور باش
\\
ترک جانت گوی آخر این که گفت
&&
کز ضلالت نفس را مزدور باش
\\
تو درافکن خویش و قسم تو ز دوست
&&
خواه ماتم باش و خواهی سور باش
\\
چون بسوزی همچو پروانه ز شمع
&&
دایما نظارگی نور باش
\\
گر می وصلش به دریا درکشی
&&
مست لایعقل مشو مخمور باش
\\
نه چو بی مغزان به یک می مست شو
&&
نه به یک دردی همه معذور باش
\\
ور به دریاها درآشامی شراب
&&
تا ابد از تشنگی رنجور باش
\\
همچو آن حلاج بدمستی مکن
&&
یا حسینی باش یا منصور باش
\\
چون نفخت فیه من روحی توراست
&&
روح پاکی فوق نفخ صور باش
\\
کنج وحدت گیر چون عطار پیش
&&
پس به کنجی درشو و مستور باش
\\
\end{longtable}
\end{center}
