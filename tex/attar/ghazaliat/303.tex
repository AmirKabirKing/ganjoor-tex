\begin{center}
\section*{غزل شماره ۳۰۳: هر که درین دایره دوران کند}
\label{sec:303}
\addcontentsline{toc}{section}{\nameref{sec:303}}
\begin{longtable}{l p{0.5cm} r}
هر که درین دایره دوران کند
&&
نقطهٔ دل آینهٔ جان کند
\\
چون رخ جان ز آینه دل بدید
&&
جان خود آئینهٔ جانان کند
\\
گر کند اندر رخ جانان نظر
&&
شرط وی آن است که پنهان کند
\\
ور نظرش از نظر آگه بود
&&
دور فتد از ره و تاوان کند
\\
گر همه یک مور ادب گوش داشت
&&
رونق خود همچو سلیمان کند
\\
مرد ره آن است که در راه عشق
&&
هرچه کند جمله به فرمان کند
\\
کی بود آن مرد گدا مرد آنک
&&
عزم به خلوتگه سلطان کند
\\
کار تو آن است که پروانه‌وار
&&
جان تو بر شمع سرافشان کند
\\
راست چو پروانه به سودای شمع
&&
تیز برون تازد و جولان کند
\\
طاقت شمعش نبود خویش را
&&
روی به شمع آرد و قربان کند
\\
عشق رخش بس که درین دایره
&&
همچو من و همچو تو حیران کند
\\
زلف پریشانش به یک تار موی
&&
جملهٔ اسلام پریشان کند
\\
لیک ز عکس رخ او ذره‌ای
&&
بتکده‌ها جمله پر ایمان کند
\\
در غم عشقش دل عطار را
&&
درد ز حد رفت چه درمان کند
\\
\end{longtable}
\end{center}
