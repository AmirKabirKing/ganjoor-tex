\begin{center}
\section*{غزل شماره ۲۱۱: چون باد صبا سوی چمن تاختن آورد}
\label{sec:211}
\addcontentsline{toc}{section}{\nameref{sec:211}}
\begin{longtable}{l p{0.5cm} r}
چون باد صبا سوی چمن تاختن آورد
&&
گویی به غنیمت همه مشک ختن آورد
\\
زان تاختنش یوسف دل گر نشد افگار
&&
پس از چه سبب غرقه به خون پیرهن آورد
\\
اشکال بدایع همه در پردهٔ رشکند
&&
زین شکل که از پرده برون یاسمن آورد
\\
هرگز ز گل و مشک نیفتاد به صحرا
&&
زین بوی که از نافه به صحرا سمن آورد
\\
صد بیضهٔ عنبر نخرد کس به جوی نیز
&&
زین رسم که در باغ کنون نسترن آورد
\\
هر لحظه صبا از پی صد راز نهانی
&&
از مشک برافکند و به گوش چمن آورد
\\
آن راز به طفلی همه عیسی صفتان را
&&
در مهد چو عیسی به شکر در سخن آورد
\\
چون کرد گل سرخ عرق از رخ یارم
&&
آبی چو گلابش ز صفا در دهن آورد
\\
لاله چو شهیدان همه آغشته به خون شد
&&
سر از غم کم عمری خود در کفن آورد
\\
اول نفس از مشک چو عطار همی زد
&&
آخر جگری سوخته دل‌تر ز من آورد
\\
\end{longtable}
\end{center}
