\begin{center}
\section*{غزل شماره ۲۴۷: قصهٔ عشق تو چون بسیار شد}
\label{sec:247}
\addcontentsline{toc}{section}{\nameref{sec:247}}
\begin{longtable}{l p{0.5cm} r}
قصهٔ عشق تو چون بسیار شد
&&
قصه‌گویان را زبان از کار شد
\\
قصهٔ هرکس چو نوعی نیز بود
&&
ره فراوان گشت و دین بسیار شد
\\
هر یکی چون مذهبی دیگر گرفت
&&
زین سبب ره سوی تو دشوار شد
\\
ره به خورشید است یک یک ذره را
&&
لاجرم هر ذره دعوی‌دار شد
\\
خیر و شر چون عکس روی و موی توست
&&
گشت نور افشان و ظلمت‌بار شد
\\
ظلمت مویت بیافت انکار کرد
&&
پرتو رویت بتافت اقرار شد
\\
هر که باطل بود در ظلمت فتاد
&&
وانکه بر حق بود پر انوار شد
\\
مغز نور از ذوق نورالنور گشت
&&
مغز ظلمت از تحسر نار شد
\\
مدتی در سیر آمد نور و نار
&&
تا زوال آمد ره و رفتار شد
\\
پس روش برخاست پیدا شد کشش
&&
رهروان را لاجرم پندار شد
\\
چون کشش از حد و غایت درگذشت
&&
هم وسایط رفت و هم اغیار شد
\\
نار چون از موی خاست آنجا گریخت
&&
نور نیز از پرده با رخسار شد
\\
موی از عین عدد آمد پدید
&&
روی از توحید بنمودار شد
\\
ناگهی توحید از پیشان بتافت
&&
تا عدد هم‌رنگ روی یار شد
\\
بر غضب چون داشت رحمت سبقتی
&&
گر عدد بود از احد هموار شد
\\
کل شیء هالک الا وجهه
&&
سلطنت بنمود و برخوردار شد
\\
چیست حاصل عالمی پر سایه بود
&&
هر یکی را هستییی مسمار شد
\\
صد حجب اندر حجب پیوسته گشت
&&
تا رونده در پس دیوار شد
\\
مرتفع چو شد به توحید آن حجب
&&
خفته از خواب هوس بیدار شد
\\
گرچه در خون گشت دل عمری دراز
&&
این زمان کودک همه دلدار شد
\\
هرکه او زین زندگی بویی نیافت
&&
مرده زاد از مادر و مردار شد
\\
وان کزین طوبی مشک‌افشان دمی
&&
برد بویی تا ابد عطار شد
\\
\end{longtable}
\end{center}
