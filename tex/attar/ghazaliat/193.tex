\begin{center}
\section*{غزل شماره ۱۹۳: نام وصلش به زبان نتوان برد}
\label{sec:193}
\addcontentsline{toc}{section}{\nameref{sec:193}}
\begin{longtable}{l p{0.5cm} r}
نام وصلش به زبان نتوان برد
&&
ور کسی برد ندانم جان برد
\\
وصل او گوهر بحری است شگرف
&&
ره بدو می‌نتوان آسان برد
\\
دوش سرمست درآمد ز درم
&&
تا قرار از من سرگردان برد
\\
زلف کژ کرد و برافشاند دلم
&&
برد شکلی که چنان نتوان برد
\\
دل من تا که خبر بود مرا
&&
راه دزدیده بدو پنهان برد
\\
زلف چوگان صفتش در صف کفر
&&
گوی از کوکبهٔ ایمان برد
\\
از فلک نرگس او نرد دغا
&&
قرب صد دست به یک دستان برد
\\
ذره‌ای پرتو خورشید رخش
&&
آفتاب از فلک گردان برد
\\
لمعه‌ای لعل خوشاب لب او
&&
رونق لاله و لالستان برد
\\
گفتم ای جان و جهان جان عزیز
&&
کس ازین بادیهٔ هجران برد
\\
گفت جان در ره ما باز و بدانک
&&
آن بود جان که ز تو جانان برد
\\
دل عطار چو این نکته شنید
&&
جان بدو داد و به جان فرمان برد
\\
\end{longtable}
\end{center}
