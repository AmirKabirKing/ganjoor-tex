\begin{center}
\section*{غزل شماره ۱۴۳: ای مشک خطا خط سیاهت}
\label{sec:143}
\addcontentsline{toc}{section}{\nameref{sec:143}}
\begin{longtable}{l p{0.5cm} r}
ای مشک خطا خط سیاهت
&&
خورشید درم خرید ماهت
\\
هرگز به خطا خطی نیفتاد
&&
سر سبزتر از خط سیاهت
\\
در عالم حسن پادشاهی
&&
جان همه عاشقان سپاهت
\\
چون بنده شدند پادشاهانت
&&
می‌نتوان خواند پادشاهت
\\
گردان گردان سپهر سرکش
&&
جویان جویان ز دیر گاهت
\\
بر خاک از آن فتاد خورشید
&&
تا ذره بود ز خاک راهت
\\
چون چین قبا به هم درافتند
&&
عشاق چو کژ نهی کلاهت
\\
در عشق تو زهد چون توان کرد
&&
چون کس نرسد به یک گناهت
\\
بس آه که عاشقانت کردند
&&
دل نرم نشد ز هیچ آهت
\\
هرگز نرسد ور آن همه آه
&&
درهم بندی به بارگاهت
\\
آن دم که ز پرده رخ نمایی
&&
صد فتنه نشسته در پناهت
\\
وانگه که ز لب شکر گشایی
&&
صد خوزستان زکات خواهت
\\
گر تو شکری دهی به عطار
&&
این صدقه فتد به جایگاهت
\\
\end{longtable}
\end{center}
