\begin{center}
\section*{غزل شماره ۶۰۱: ما بار دگر گوشهٔ خمار گرفتیم}
\label{sec:601}
\addcontentsline{toc}{section}{\nameref{sec:601}}
\begin{longtable}{l p{0.5cm} r}
ما بار دگر گوشهٔ خمار گرفتیم
&&
دادیم دل از دست و پی یار گرفتیم
\\
دعوی دو کون از دل خود دور فکندیم
&&
پس در ره جانان پی اسرار گرفتیم
\\
از هر دو جهان مهر یکی را بگزیدیم
&&
و از آرزوی او کم اغیار گرفتیم
\\
گفتند خودی تو درین راه حجاب است
&&
ترک خودی خویش به یکبار گرفتیم
\\
ای بس که چو پروانهٔ پر سوخته از شمع
&&
در کوی رجا دامن پندار گرفتیم
\\
از کعبهٔ جان چون که ندیدیم نشانی
&&
از کعبهٔ ظاهر ره خمار گرفتیم
\\
از خرقه و تسبیح چو جز نام ندیدیم
&&
چه خرقه چه تسبیح که زنار گرفتیم
\\
زین دین به تزویر چو دل خیره فروماند
&&
اندر ره دین شیوهٔ کفار گرفتیم
\\
چون هرچه جز او هست درین راه حجاب است
&&
پس ما به یقین مذهب عطار گرفتیم
\\
\end{longtable}
\end{center}
