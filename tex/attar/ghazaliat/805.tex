\begin{center}
\section*{غزل شماره ۸۰۵: ترسا بچهٔ لولی همچون بت روحانی}
\label{sec:805}
\addcontentsline{toc}{section}{\nameref{sec:805}}
\begin{longtable}{l p{0.5cm} r}
ترسا بچهٔ لولی همچون بت روحانی
&&
سرمست برون آمد از دیر به نادانی
\\
زنار و بت اندر بر ناقوس ومی اندر کف
&&
در داد صلای می از ننگ مسلمانی
\\
چون نیک نگه کردم در چشم و لب و زلفش
&&
بر تخت دلم بنشست آن ماه به سلطانی
\\
بگرفتم زنارش در پای وی افتادم
&&
گفتم چکنم جانا گفتا که تو می‌دانی
\\
گر وصل منت باید ای پیر مرقع پوش
&&
هم خرقه بسوزانی هم قبله بگردانی
\\
با ما تو به دیر آیی محراب دگر گیری
&&
وز دفتر عشق ما سطری دو سه بر خوانی
\\
اندر بن دیر ما شرطت بود این هر سه
&&
کز خویش برون آیی وز جان و دل فانی
\\
می خور تو به دیر اندر تا مست شوی بیخود
&&
کز بی خبری یابی آن چیز که جویانی
\\
هر گه که شود روشن بر تو که تویی جمله
&&
فریاد اناالحق زن در عالم انسانی
\\
عطار ز راه خود برخیز که تا بینی
&&
خود را ز خودی برهان کز خویش تو پنهانی
\\
\end{longtable}
\end{center}
