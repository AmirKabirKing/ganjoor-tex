\begin{center}
\section*{غزل شماره ۵۷۲: آه گر من زعشق آه کنم}
\label{sec:572}
\addcontentsline{toc}{section}{\nameref{sec:572}}
\begin{longtable}{l p{0.5cm} r}
آه گر من زعشق آه کنم
&&
همه روی جهان سیاه کنم
\\
آه من در جهان نمی‌گنجد
&&
در جهان پس چگونه آه کنم
\\
هر دو عالم شود چو انگشتی
&&
گر من آهی ز جایگاه کنم
\\
گر دمی آتشین زنم ز دلم
&&
به دمی دفع صد سپاه کنم
\\
بحر خون در دلم چو موج زند
&&
من به خون در روم شناه کنم
\\
موج آن خون چو بگذرد از حد
&&
خون دل را به دیده راه کنم
\\
خون بریزم ز دیده چندانی
&&
که بسی خلق را تباه کنم
\\
عالمی خون خویشتن بینم
&&
از پس و پیش اگر نگاه کنم
\\
با چنین حالتی عجب که مراست
&&
گر کنم طاعتی گناه کنم
\\
هیچ خلقی گداتر از من نیست
&&
گرچه دعوی پادشاه کنم
\\
ره به گلخن نمی‌دهند مرا
&&
وین عجب عزم بارگاه کنم
\\
شربتی آب چاه نیست مرا
&&
وی عجب عزم فخر آب جاه کنم
\\
همچو لاله کلاه در خونم
&&
چه حدیث سر و کلاه کنم
\\
سر درودم فرید را چو گیاه
&&
پس کنون کره در گیاه کنم
\\
همچو عطار مست عشق شوم
&&
گر دمی در رخش نگاه کنم
\\
\end{longtable}
\end{center}
