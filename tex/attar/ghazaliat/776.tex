\begin{center}
\section*{غزل شماره ۷۷۶: پروانه شبی ز بی قراری}
\label{sec:776}
\addcontentsline{toc}{section}{\nameref{sec:776}}
\begin{longtable}{l p{0.5cm} r}
پروانه شبی ز بی قراری
&&
بیرون آمد به خواستاری
\\
از شمع سؤال کرد کاخر
&&
تا کی سوزی مرا به خواری
\\
در حال جواب داد شمعش
&&
کای بی سر و بن خبر نداری
\\
آتش مپرست تا نباشد
&&
در سوختنت گریفتاری
\\
تو در نفسی بسوختی زود
&&
رستی ز غم و ز غمگساری
\\
من مانده‌ام ز شام تا صبح
&&
در گریه و سوختن به زاری
\\
گه می‌خندم ولیک بر خویش
&&
گه می‌گریم ز سوکواری
\\
می‌گویندم بسوز خوش خوش
&&
تا بیخ ز انگبین برآری
\\
هر لحظه سرم نهند در پیش
&&
گویند چرا چنین نزاری
\\
شمعی دگر است لیک در غیب
&&
شمعی است نه روشن و نه تاری
\\
پروانهٔ او منم چنین گرم
&&
زان یافته‌ام مزاج زاری
\\
من می‌سوزم ازو تو از من
&&
این است نشان دوستداری
\\
چه طعن زنی مرا که من نیز
&&
در سوختنم به بیقراری
\\
آن شمع اگر بتابد از غیب
&&
پروانه بسی فتد شکاری
\\
تا می‌ماند نشان عطار
&&
می‌خواهد سوخت شمع واری
\\
\end{longtable}
\end{center}
