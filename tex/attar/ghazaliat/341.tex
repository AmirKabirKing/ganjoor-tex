\begin{center}
\section*{غزل شماره ۳۴۱: هر که صید چون تو دلداری شود}
\label{sec:341}
\addcontentsline{toc}{section}{\nameref{sec:341}}
\begin{longtable}{l p{0.5cm} r}
هر که صید چون تو دلداری شود
&&
عاجزی گردد گرفتاری شود
\\
هر که خار مژهٔ تو بنگرد
&&
هر گلی در چشم او خاری شود
\\
باز چون گلبرگ روی تو بدید
&&
بی شکش هر خار گلزاری شود
\\
شیر دل پیش نمکدان لبت
&&
چون به جان آید جگر خواری شود
\\
گر لبت در ابر خندد همچو برق
&&
ابر تا محشر شکرباری شود
\\
در طواف نقطهٔ خالت ز شوق
&&
چرخ سرگردان چو پرگاری شود
\\
مس اگرچه زر تواند شد ولیک
&&
وصف خط تو چو بسیاری شود
\\
پیش سرسبزی خطت زاشتیاق
&&
زر کند بدرود و زنگاری شود
\\
سرفرازی کو سر زلف تو دید
&&
تا بجنبد سرنگونساری شود
\\
میل زلف تو به ترسایی است از آنک
&&
گه چلیپا گاه زناری شود
\\
گو بیا و مذهب زلف تو گیر
&&
هر که می‌خواهد که دینداری شود
\\
گر فروشی بر من غمکش جهان
&&
هر سر مویم خریداری شود
\\
هر که او دل‌زنده عشق تو نیست
&&
گر همه مشک است مرداری شود
\\
نیست آسان هیچ کار عشق تو
&&
زان به تن بردن چو دشواری شود
\\
پی چو گم کردند کار عشق را
&&
عاشقی کو کز پی کاری شود
\\
عشق را هرگز نماند رونقی
&&
هر کسی گر صاحب‌اسراری شود
\\
صد هزاران قطره گردد ناپدید
&&
تا یکی زان در شهواری شود
\\
چون کسی را بوی نبود زین حدیث
&&
کی شود ممکن که عطاری شود
\\
\end{longtable}
\end{center}
