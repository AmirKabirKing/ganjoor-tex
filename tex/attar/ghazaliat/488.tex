\begin{center}
\section*{غزل شماره ۴۸۸: دوش دل را در بلایی یافتم}
\label{sec:488}
\addcontentsline{toc}{section}{\nameref{sec:488}}
\begin{longtable}{l p{0.5cm} r}
دوش دل را در بلایی یافتم
&&
خانه چون ماتم سرایی یافتم
\\
گفتم ای دل چیست حال آخر بگو
&&
گفت بوی آشنایی یافتم
\\
همچو گویی در خم چوگان عشق
&&
خویش را نه سر نه پایی یافتم
\\
خواستم تا دل نثار او کنم
&&
زانکه جانم را سزایی یافتم
\\
پیش از من جان بر او رفته بود
&&
گرچه من بی‌جان بقایی یافتم
\\
آن بقا از جان نبود از عشق بود
&&
زانکه عشق جان فزایی یافتم
\\
مردم چشم خودش خوانم از آنک
&&
دایمش در دیده جایی یافتم
\\
گرچه زلف او گره بسیار داشت
&&
هر گره مشکل‌گشایی یافتم
\\
با چنان مشکل‌گشایی حل نشد
&&
آنچه من از دلربایی یافتم
\\
چون به خون خویشتن بستم سجل
&&
هر سرشکی را گوایی یافتم
\\
چون سجل بندم به خون چون پیش ازین
&&
از لب او خون بهایی یافتم
\\
عقل از زلفش ز بس کاندیشه کرد
&&
حاصلش تاریکنایی یافتم
\\
با دهانش تا دوچاری خورد دل
&&
دایمش در تنگنایی یافتم
\\
در هوای او دل عطار را
&&
ذره کردم چون هبایی یافتم
\\
\end{longtable}
\end{center}
