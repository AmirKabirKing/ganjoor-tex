\begin{center}
\section*{غزل شماره ۲۲۹: گر از گره زلفت جانم کمری سازد}
\label{sec:229}
\addcontentsline{toc}{section}{\nameref{sec:229}}
\begin{longtable}{l p{0.5cm} r}
گر از گره زلفت جانم کمری سازد
&&
در جمع کله‌داران از خویش سری سازد
\\
گردون که همه کس را زو دست بود بر سر
&&
از دست سر زلفت هر شب حشری سازد
\\
طاوس فلک هر شب شد سوخته بال و پر
&&
هم شمع رخت سوزد گر بال و پری سازد
\\
بنمای لب و رویت تا این دل بیمارم
&&
یا به بتری گردد یا گلشکری سازد
\\
جان عزم سفر دارد زین بیش مخور خونش
&&
تا بو که ز خون دل زاد سفری سازد
\\
این عاشق بی زر را زر نیست تو می‌خواهی
&&
چون وجه زرش نبود از وجه زری سازد
\\
تا زر نبود اول تا جان ندهد آخر
&&
دیوانه بود هر کو با سیم‌بری سازد
\\
دیری است که می‌سازم تا بو که بسازی تو
&&
چون توبه نمی‌سازی دل با دگری سازد
\\
چون نیست ز یاقوتت هم قوت و هم قوتم
&&
عطار کنون بی تو قوت از جگری سازد
\\
\end{longtable}
\end{center}
