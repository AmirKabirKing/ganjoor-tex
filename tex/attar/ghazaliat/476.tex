\begin{center}
\section*{غزل شماره ۴۷۶: از عشق تو من به دیر بنشستم}
\label{sec:476}
\addcontentsline{toc}{section}{\nameref{sec:476}}
\begin{longtable}{l p{0.5cm} r}
از عشق تو من به دیر بنشستم
&&
زنار مغانهٔ بر میان بستم
\\
چون حلقهٔ زلف توست زناری
&&
زنار چرا همیشه نپرستم
\\
گر دین و دلم ز دست شد شاید
&&
چون حلقه زلف توست در دستم
\\
دست‌آویزی نکو به دست آمد
&&
در زلف تو دست تا بپیوستم
\\
چون ترسایی درست شد بر من
&&
خوردم می عشق و توبه بشکستم
\\
زان می که به جرعه‌ای که من خوردم
&&
گویی ز هزار سالگی مستم
\\
در سینه دریچه‌ای پدید آمد
&&
بسیار بر آن دریچه بنشستم
\\
صد بحر از آن دریچه پیدا شد
&&
من چشمهٔ دل به بحر پیوستم
\\
طاقت چو نداشتم شدم غرقه
&&
زان صید که اوفتاد در شستم
\\
جانم چو ز عشق آن جهانی شد
&&
از رسم و رسوم این جهان رستم
\\
باور نکنند اگر به نطق آرم
&&
امروز بدین صفت که من هستم
\\
نه موجودم نه نیز معدومم
&&
هیچم، همه‌ام، بلند و پستم
\\
عطار درین چنین خطرگاهی
&&
تو دانی و تو که من برون جستم
\\
\end{longtable}
\end{center}
