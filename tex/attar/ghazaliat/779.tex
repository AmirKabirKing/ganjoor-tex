\begin{center}
\section*{غزل شماره ۷۷۹: ترسا بچه‌ای شنگی زین نادره دلداری}
\label{sec:779}
\addcontentsline{toc}{section}{\nameref{sec:779}}
\begin{longtable}{l p{0.5cm} r}
ترسا بچه‌ای شنگی زین نادره دلداری
&&
زین خوش نمکی شوخی، زین طرفه جگرخواری
\\
از پستهٔ خندانش هرجا که شکر ریزی
&&
در چاه زنخدانش هر جا که نگونساری
\\
از هر سخن تلخش ره یافته بی دینی
&&
وز هر شکن زلفش گمره شده دین‌داری
\\
دیوانهٔ عشق او هرجا که خردمندی
&&
دردی کش درد او هرجا که طلب کاری
\\
آمد بر پیر ما می در سر و می در بر
&&
پس در بر پیر ما بنشست چو هشیاری
\\
گفتش که بگیر این می، این روی و ریا تا کی
&&
گر نوش کنی یک می، از خود برهی باری
\\
ای همچو یخ افسرده یک لحظه برم بنشین
&&
تا در تو زند آتش ترسا بچه یک باری
\\
بی خویش شو از هستی تا باز نمانی تو
&&
ای چون تو به هر منزل واماندهٔ بسیاری
\\
پیر از سر بی خویشی، می بستد و بیخود شد
&&
در حال پدید آمد در سینهٔ او ناری
\\
کاریش پدید آمد کان پیر نود ساله
&&
بر جست و میان حالی بر بست به زناری
\\
در خواب شد از مستی بیدار شد از هستی
&&
از صومعه بیرون شد بنشست چو خماری
\\
عطار ز کار او در مانده به صد حیرت
&&
هرکس که چنین بیند حیرت بودش آری
\\
\end{longtable}
\end{center}
