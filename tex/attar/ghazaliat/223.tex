\begin{center}
\section*{غزل شماره ۲۲۳: چون پرده ز روی ماه برگیرد}
\label{sec:223}
\addcontentsline{toc}{section}{\nameref{sec:223}}
\begin{longtable}{l p{0.5cm} r}
چون پرده ز روی ماه برگیرد
&&
از فرق فلک کلاه برگیرد
\\
بی روی چو ماه او دم سردم
&&
از روی سپهر ماه برگیرد
\\
صاحب‌نظری اگر دمم بیند
&&
هر دم که زنم به آه برگیرد
\\
در راه فتاده‌ام به بوی آنک
&&
چون سایه مرا ز راه برگیرد
\\
و او خود چو مرا تباه بیند حال
&&
سایه ز من تباه برگیرد
\\
خطش چو به خون من سجل بندد
&&
دو جادو را گواه برگیرد
\\
که حکم کند بدین گواه و خط
&&
جز آنکه دل از اله برگیرد
\\
هرگاه که زلف او نهد جرمم
&&
صد توبه به یک گناه برگیرد
\\
لیکن لب عذرخواه پیش آرد
&&
وز هم لب عذرخواه برگیرد
\\
جادو بچهٔ دو چشمش آن خواهد
&&
تا رسم گدا و شاه برگیرد
\\
صد بالغ را ببین که چون از راه
&&
جادو بچهٔ سیاه برگیرد
\\
عقل آید و عالمی حشر سازد
&&
وز صبر بسی سپاه برگیرد
\\
با قلب شکسته پیش صف آید
&&
تا پرده ز پیشگاه برگیرد
\\
چشمش به صف مژه به یک مویش
&&
با خیل و سپه ز راه برگیرد
\\
گفتم اگرم دهد پناه خود
&&
کنجی دلم از پناه برگیرد
\\
از نقد جهان فرید را قلبی است
&&
این قلب که گاه گاه برگیرد
\\
\end{longtable}
\end{center}
