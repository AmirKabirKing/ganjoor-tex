\begin{center}
\section*{غزل شماره ۱۳۷: آتش سودای تو عالم جان در گرفت}
\label{sec:137}
\addcontentsline{toc}{section}{\nameref{sec:137}}
\begin{longtable}{l p{0.5cm} r}
آتش سودای تو عالم جان در گرفت
&&
سوز دل عاشقانت هر دو جهان در گرفت
\\
جان که فروشد به عشق زندهٔ جاوید گشت
&&
دل که بدانست حال ماتم جان در گرفت
\\
از پس چندین هزار پرده که در پیش بود
&&
روی تو یک شعله زد کون و مکان در گرفت
\\
چون تو برانداختی برقع عزت ز پیش
&&
جان متحیر بماند عقل فغان در گرفت
\\
بر سر کوی تو عشق آتش دل برفروخت
&&
شمع دل عاشقانت جمله از آن در گرفت
\\
جرعهٔ اندوه تو تا دل من نوش کرد
&&
زآتش آه دلم کام و زبان در گرفت
\\
تا که ز رنگ رخت یافت دل من نشان
&&
روی من از خون دل رنگ و نشان در گرفت
\\
جان و دل عاشقان خرقه شد اندر میان
&&
زانکه سماع غمت در همگان در گرفت
\\
راست که عطار داد حسن و جمال تو شرح
&&
سینه برآورد جوش دل خفقان در گرفت
\\
\end{longtable}
\end{center}
