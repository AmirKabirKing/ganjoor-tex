\begin{center}
\section*{غزل شماره ۴۶۵: در خطت تا دل به جان در بسته‌ام}
\label{sec:465}
\addcontentsline{toc}{section}{\nameref{sec:465}}
\begin{longtable}{l p{0.5cm} r}
در خطت تا دل به جان در بسته‌ام
&&
چون قلم زان خط میان در بسته‌ام
\\
در تماشای خط سرسبز تو
&&
چشم بگشاده فغان در بسته‌ام
\\
نی که از خطت زبانم شد ز کار
&&
زان چنین دایم زبان در بسته‌ام
\\
تو چنین پسته دهان و من ز شوق
&&
گرچه می‌سوزم دهان در بسته‌ام
\\
آشکارا خون دل بگشاده‌ام
&&
تا به زلفت دل نهان در بسته‌ام
\\
پر گره دانست زلف تو که من
&&
دل به زلفت هر زمان در بسته‌ام
\\
چون جهان آرای دیدم روی تو
&&
چشم از روی جهان در بسته‌ام
\\
نیست در کار توام دلبستگی
&&
زانکه در کار تو جان در بسته‌ام
\\
گفته‌ای در بند با من تا به جان
&&
این چه باشد بیش از آن در بسته‌ام
\\
گفته‌ای در بند با من تا به جان
&&
این چه باشد بیش از آن در بسته‌ام
\\
گر بسوزد همچو خاکستر دو کون
&&
نگسلم از تو چنان در بسته‌ام
\\
تا بلای ناگهان دیدم ز هجر
&&
رخت رحلت ناگهان در بسته‌ام
\\
هم دل از عطار فارغ کرده‌ام
&&
هم در سود و زیان در بسته‌ام
\\
\end{longtable}
\end{center}
