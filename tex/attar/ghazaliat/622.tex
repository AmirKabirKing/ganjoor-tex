\begin{center}
\section*{غزل شماره ۶۲۲: وقت آن آمد که ما آن ماه را مهمان کنیم}
\label{sec:622}
\addcontentsline{toc}{section}{\nameref{sec:622}}
\begin{longtable}{l p{0.5cm} r}
وقت آن آمد که ما آن ماه را مهمان کنیم
&&
پیش او شکرانه جان خویش را قربان کنیم
\\
چون ز راه اندر رسد ما روی بر راهش نهیم
&&
وانگهی بر خاک راهش دیده خون‌افشان کنیم
\\
هرچه در صد سال گرد آورده باشیم این زمان
&&
گر همه جان است ایثار ره جانان کنیم
\\
گر نباشد ماحضر چیزی نیندیشیم از آن
&&
آتشی از دل برافروزیم و جان بریان کنیم
\\
شمع چون از سینه سوزد نقل از چشم آوریم
&&
باده چون از عشق باشد جام او از جان کنیم
\\
بر جمال دوست چندان می‌کشیم از جام جان
&&
کز تف او عقل را تا منتها حیران کنیم
\\
پای‌کوبان دست‌زن در های و هوی آییم مست
&&
هم پیاپی هم سراسر دورها گردان کنیم
\\
هر نفس بر بوی او عمری دگر پی افکنیم
&&
هر زمان بر روی او شادی دیگرسان کنیم
\\
گر در آن شب صبحدم ما را بود خلوت بسوز
&&
صبح را تا روز حشر از خون دل مهمان کنیم
\\
در نگنجد مویی آن دم گر بیاید ماه و چرخ
&&
ماه را بر در زنیم و چرخ را دربان کنیم
\\
در حضور او کسی ننشست تا فانی نشد
&&
گر سر مویی ز ما باقی بود تاوان کنیم
\\
چون حریفان جمله از مستی و هستی وا رهند
&&
جمله را بی خویشتن بر خویشتن گریان کنیم
\\
چون نه سر نه خرقه ماند از کمال نیستی
&&
خرقه را با سر بریم و کارها آسان کنیم
\\
گر دهد عطار را وصلی چنین یک لحظه دست
&&
هر که دردی دارد از درد خودش درمان کنیم
\\
\end{longtable}
\end{center}
