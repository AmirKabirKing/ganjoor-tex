\begin{center}
\section*{غزل شماره ۷۱۰: ای لبت حقهٔ گهر بسته}
\label{sec:710}
\addcontentsline{toc}{section}{\nameref{sec:710}}
\begin{longtable}{l p{0.5cm} r}
ای لبت حقهٔ گهر بسته
&&
دهنت شور در شکر بسته
\\
طوطیان خط تو پیش شکر
&&
بال بگشاده و کمر بسته
\\
خطت از پستهٔ تو بر رستهٔ است
&&
هست بر رستهٔ تو بر بسته
\\
زان خط سبز بر رخ زردم
&&
خون دل جمله چون جگر بسته
\\
عاشق از جان به صد هزاران دل
&&
در تو هر دم دلی دگر بسته
\\
هر که از تو کشیده مویی سر
&&
دستش از موی باز بر بسته
\\
به شکرخنده بر دهان بگشا
&&
که گره کس ندید بر بسته
\\
تا به کی همچو حلقه بر در تو
&&
سر زنم دایم و تو در بسته
\\
نظری کن دلم مسوز که هست
&&
کار جانم در آن نظر بسته
\\
کمترم سوز اگر نه فاش کنم
&&
مکر تو چند مکر سربسته
\\
چشم عطار سیل بگشاده
&&
دل ز هجر تو در خطر بسته
\\
\end{longtable}
\end{center}
