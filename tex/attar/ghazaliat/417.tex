\begin{center}
\section*{غزل شماره ۴۱۷: دوش آمد و گفت از آن ما باش}
\label{sec:417}
\addcontentsline{toc}{section}{\nameref{sec:417}}
\begin{longtable}{l p{0.5cm} r}
دوش آمد و گفت از آن ما باش
&&
در بوتهٔ امتحان ما باش
\\
گر خواهی بود زندهٔ جاوید
&&
زنده به وجود جان ما باش
\\
عمری است که تا از آن خویشی
&&
گر وقت آمد از آن ما باش
\\
مردانه به کوی ما فرود آی
&&
نعره زن و جان فشان ما باش
\\
گر محرم پیشگه نه‌ای تو
&&
هم صحبت آستان ما باش
\\
پریده زآشیان مایی
&&
جویندهٔ آشیان ما باش
\\
از ننگ وجود خود بپرهیز
&&
فانی شو و بی نشان ما باش
\\
ره نتوانی به خود بریدن
&&
در پهلوی پهلوان ما باش
\\
تا کی خفتی که کاروان رفت
&&
در رستهٔ کاروان ما باش
\\
چون می‌دانی که جمله ماییم
&&
با جمله مگو زبان ما باش
\\
چون اعجمیند خلق جمله
&&
تو با همه ترجمان ما باش
\\
تا چند ز داستان عطار
&&
مستغرق داستان ما باش
\\
\end{longtable}
\end{center}
