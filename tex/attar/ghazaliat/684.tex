\begin{center}
\section*{غزل شماره ۶۸۴: ای جلوه‌گر عالم، طاوس جمال تو}
\label{sec:684}
\addcontentsline{toc}{section}{\nameref{sec:684}}
\begin{longtable}{l p{0.5cm} r}
ای جلوه‌گر عالم، طاوس جمال تو
&&
سرسبزی و شب رنگی وصف خط و خال تو
\\
بدری که فرو شد زو خورشید به تاریکی
&&
در دق و ورم مانده از رشک هلال تو
\\
صد مرد چو رستم را چون بچهٔ یک روزه
&&
پرورده به زیر پر سیمرغ جمال تو
\\
زان درفکند خود را خورشید به هر روزن
&&
تا بو که به دست آرد یک ذره وصال تو
\\
مه گرچه به روز و شب دواسبه همی تازد
&&
نرسد به رخ خوب خورشید مثال تو
\\
گفتم ز خیال تو رنگی بودم یک شب
&&
خود هم تک برق آمد شبرنگ خیال تو
\\
گفتی که تو را از من صبر است اگر خواهی
&&
کشتن شودم واجب از گفت محال تو
\\
عطار به وصافی گرچه به کمال آمد
&&
شد گنگ زبان او در وصف کمال تو
\\
\end{longtable}
\end{center}
