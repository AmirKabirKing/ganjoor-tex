\begin{center}
\section*{غزل شماره ۲۵۳: بیچاره دلم در سر آن زلف به خم شد}
\label{sec:253}
\addcontentsline{toc}{section}{\nameref{sec:253}}
\begin{longtable}{l p{0.5cm} r}
بیچاره دلم در سر آن زلف به خم شد
&&
دل کیست که جان نیز درین واقعه هم شد
\\
انگشت نمای دو جهان گشت به عزت
&&
هر دل که سراسیمهٔ آن زلف به خم شد
\\
چون پرده برانداختی از روی چو خورشید
&&
هر جا که وجودی است از آن روی عدم شد
\\
راه تو شگرف است بسر می‌روم آن ره
&&
زآنروی که کفر است در آن ره به قدم شد
\\
عشاق جهان جمله تماشای تو دارند
&&
عالم ز تماشی تو چون خلد ارم شد
\\
تا مشعلهٔ روی تو در حسن بیفزود
&&
خوبان جهان را ز خجل مشعله کم شد
\\
تا روی چو خورشید تو از پرده علم زد
&&
خورشید ز پرده به‌در افتاد و علم شد
\\
تا لوح چو سیم تو خطی سبز برآورد
&&
جان پیش خط سبز تو بر سر چو قلم شد
\\
چون آه جگرسوز ز عطار برآمد
&&
با مشک خط تو جگر سوخته ضم شد
\\
\end{longtable}
\end{center}
