\begin{center}
\section*{غزل شماره ۳۰۸: عشق توام داغ چنان می‌کند}
\label{sec:308}
\addcontentsline{toc}{section}{\nameref{sec:308}}
\begin{longtable}{l p{0.5cm} r}
عشق توام داغ چنان می‌کند
&&
کآتش سوزنده فغان می‌کند
\\
بر دل من چون دل آتش بسوخت
&&
بر سر من اشک‌فشان می‌کند
\\
درنگر آخر که ز سوز دلم
&&
چون دل آتش خفقان می‌کند
\\
عشق تو بی‌رحم‌تر از آتش است
&&
کآتشم از عشق ضمان می‌کند
\\
آتش سوزنده به جز تن نسوخت
&&
عشق تو آهنگ به جان می‌کند
\\
هر که ز زلف تو کشد سر چو موی
&&
زلف تواش موی کشان می‌کند
\\
آنچه که جستند همه اهل دل
&&
مردم چشم تو عیان می‌کند
\\
وآنچه که صد سال کند رستمی
&&
زلف تو در نیم زمان می‌کند
\\
چون نزند چشم خوشت تیر چرخ
&&
کابروی تو چرخ کمان می‌کند
\\
گر همه خورشید سبک‌رو بود
&&
پیش رخت سایه گران میکند
\\
هر که کند وصف دهانت که نیست
&&
هست یقین کان به گمان می‌کند
\\
خط تو چون مهر نبوت به نسخ
&&
ختم همه حسن جهان می‌کند
\\
چون ز پی خضر همه سبز رست
&&
خط تو زان قصد نشان می‌کند
\\
چشمهٔ خضر است دهانت به حکم
&&
خط تو سرسبزی از آن می‌کند
\\
پسته وآن فستقی مغز او
&&
دعوی آن خط و دهان می‌کند
\\
بی خبری دی خط تو دید و گفت
&&
برگ گل از سبزه نهان می‌کند
\\
می‌نشناسد که دهانش ز خط
&&
غالیه در غالیه‌دان می‌کند
\\
چون دهنش ثقبهٔ سوزن فتاد
&&
رشتهٔ آن ثقبه میان می‌کند
\\
دی ز دهانش شکری خواستم
&&
گفت که نرمم به زبان می‌کند
\\
سود ندارد شکری بی جگر
&&
می‌ندهد زانکه زیان می‌کند
\\
کز نفس سردت و باران اشک
&&
لالهٔ من برگ خزان می‌کند
\\
شفقت او بین که رخم در سرشک
&&
چون رخ خود لاله‌ستان می‌کند
\\
شیوه او می‌نبد اندر فرید
&&
گرچه ز صد شیوه برآن می‌کند
\\
\end{longtable}
\end{center}
