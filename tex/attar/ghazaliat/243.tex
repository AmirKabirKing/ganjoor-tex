\begin{center}
\section*{غزل شماره ۲۴۳: ذوق وصلت به هیچ جان نرسد}
\label{sec:243}
\addcontentsline{toc}{section}{\nameref{sec:243}}
\begin{longtable}{l p{0.5cm} r}
ذوق وصلت به هیچ جان نرسد
&&
شرح رویت به هر زبان نرسد
\\
سر زلفت به دست چون آرم
&&
دست موری به آسمان نرسد
\\
با سر زلفت تو دو عالم را
&&
سر یک موی امتحان نرسد
\\
نرسد بوی زلف تو به دلم
&&
تا که کار دلم به جان نرسد
\\
ماه خواهد که چون رخ تو بود
&&
عمرها گردد و بدان نرسد
\\
پیش خطت که رایج است به خون
&&
هیچکس را خط امان نرسد
\\
تا قیامت چو طوطی خط تو
&&
هیچ طوطی شکرفشان نرسد
\\
عقل را زاب زندگانی تو
&&
تا نمیرد ز خود نشان نرسد
\\
گرچه کس نیست چو تو موی میان
&&
هر دو کونت فرا میان نرسد
\\
کاروان تواند خلق و ز تو
&&
بیش گردی به کاروان نرسد
\\
برسد صد هزار باره جهان
&&
که نظیر تو در جهان نرسد
\\
وصل تو چون به جان نمی‌یابند
&&
به چو من کس به رایگان نرسد
\\
آتش عشق تو چو شعله زند
&&
هیچ کس را از او امان نرسد
\\
تا ابد دل ز سود برگیرد
&&
هر که را در رهت زیان نرسد
\\
کرده‌ام دل کباب و اشک شراب
&&
که مرا چون تو میهمان نرسد
\\
آن زمان کت به جان بخواهم جست
&&
برسد جان و آن زمان نرسد
\\
تا که عطار را بیان تو هست
&&
هیچ گوینده را بیان نرسد
\\
\end{longtable}
\end{center}
