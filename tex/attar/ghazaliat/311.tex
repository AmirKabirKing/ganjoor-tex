\begin{center}
\section*{غزل شماره ۳۱۱: چو تاب در سر آن زلف دلستان فکند}
\label{sec:311}
\addcontentsline{toc}{section}{\nameref{sec:311}}
\begin{longtable}{l p{0.5cm} r}
چو تاب در سر آن زلف دلستان فکند
&&
هزار فتنه و آشوب در جهان فکند
\\
چو شور پستهٔ تو تلخیی کند به شکر
&&
هزار شور و شغب در شکرستان فکند
\\
چو خلق را به سر آستین به خود خواند
&&
به غمزه‌شان بکشد خون برآستان فکند
\\
چون جشن ساخت بتان را چو خاتمی شد ماه
&&
که بو که خاتم مه نیز در میان فکند
\\
به پیش خلق مرا دی بزد به زخم زبان
&&
که تا به طنز مرا خلق در زبان فکند
\\
بتا ز زلف تو زان خیره گشت روی زمین
&&
که سایه بر سر خورشید آسمان فکند
\\
اگر شبی برم آیی به جان تو که دلم
&&
بر آتش تو به جای سپند جان فکند
\\
دلم ببردی و عطار اگر ز پس آید
&&
چنان بود که پس تیر در، کمان فکند
\\
\end{longtable}
\end{center}
