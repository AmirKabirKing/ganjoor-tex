\begin{center}
\section*{غزل شماره ۲۵۲: بار دگر پیر ما مفلس و قلاش شد}
\label{sec:252}
\addcontentsline{toc}{section}{\nameref{sec:252}}
\begin{longtable}{l p{0.5cm} r}
بار دگر پیر ما مفلس و قلاش شد
&&
در بن دیر مغان ره زن اوباش شد
\\
میکدهٔ فقر یافت خرقهٔ دعوی بسوخت
&&
در ره ایمان به کفر در دو جهان فاش شد
\\
زآتش دل پاک سوخت مدعیان را به دم
&&
دردی اندوه خورد عاشق و قلاش شد
\\
پاک بری چست بود در ندب لامکان
&&
کم زن و استاد گشت حیله گر و طاش شد
\\
لاشهٔ دل را ز عشق بار گران برنهاد
&&
فانی و لاشییء گشت یار هویداش شد
\\
راست که بنمود روی آن مه خورشید چهر
&&
عقل چو طاوس گشت وهم چو خفاش شد
\\
وهم ز تدبیر او آزر بت‌ساز گشت
&&
عقل ز تشویر او مانی نقاش شد
\\
چون دل عطار را بحر گهربخش دید
&&
در سخن آمد به حرف ابر گهرپاش شد
\\
\end{longtable}
\end{center}
