\begin{center}
\section*{غزل شماره ۳۸۴: قدم درنه اگر مردی درین کار}
\label{sec:384}
\addcontentsline{toc}{section}{\nameref{sec:384}}
\begin{longtable}{l p{0.5cm} r}
قدم درنه اگر مردی درین کار
&&
حجاب تو تویی از پیش بردار
\\
اگر خواهی که مرد کار گردی
&&
مکن بی حکم مردی عزم این کار
\\
یقین دان کز دم این شیرمردان
&&
شود چون شیر بیشه شیر دیوار
\\
چو بازان جای خود کن ساعد شاه
&&
مشو خرسند چون کرکس به مردار
\\
دلیری شیرمردی باید این جا
&&
که صد دریا درآشامد به یکبار
\\
ز رعنایان نازک‌دل چه خیزد
&&
که این جا پردلی باید جگرخوار
\\
نه او را کفر دامن‌گیر و نه دین
&&
نه او را نور دامن‌سوز و نه نار
\\
دلا تا کی روی بر سر چو گردون
&&
قراری گیر و دم درکش زمین‌وار
\\
اگر خواهی که دریایی شوی تو
&&
چو کوهی خویش را برجای می دار
\\
کنون چون نقطه ساکن باش یکچند
&&
که سرگردان بسی گشتی چو پرگار
\\
اگر خواهی که در پیش افتی از خویش
&&
سه کارت می‌بباید کرد ناچار
\\
یکی آرام و دیگر صبر کردن
&&
سیم دایم زبان بستن ز گفتار
\\
اگر دستت دهد این هر سه حالت
&&
علم بر هر دو عالم زن چو عطار
\\
\end{longtable}
\end{center}
