\begin{center}
\section*{غزل شماره ۴۸۹: یک غمت را هزار جان گفتم}
\label{sec:489}
\addcontentsline{toc}{section}{\nameref{sec:489}}
\begin{longtable}{l p{0.5cm} r}
یک غمت را هزار جان گفتم
&&
شادی عمر جاودان گفتم
\\
عاشق ذره‌ای غمت دیدم
&&
هر دلی را که شادمان گفتم
\\
بر درت آفتاب را همه شب
&&
عاشقی سر بر آستان گفتم
\\
باز چون سایه‌ای همه روزش
&&
در بدر از پیت دوان گفتم
\\
ذره‌ای عکس را که از رخ توست
&&
آفتاب همه جهان گفتم
\\
تا که وصف دهان تو کردم
&&
قصه‌ای بس شکرفشان گفتم
\\
چون بدو وصف را طریق نبود
&&
ظلم کردم کزان دهان گفتم
\\
زان سبب شد مرا سخن باریک
&&
کز میان تو هر زمان گفتم
\\
ماه رویا هنوز یک موی است
&&
هرچه در وصل آن میان گفتم
\\
گفته بودم که در تو بازم سر
&&
بی توام ترک سر از آن گفتم
\\
گفتی از دل نگویی این هرگز
&&
راست گفتی که من ز جان گفتم
\\
باد بی‌تو سر زبانم شق
&&
گر من این از سر زبان گفتم
\\
خواستم ذره‌ای وصال از تو
&&
وین سخن هم به امتحان گفتم
\\
در تو نگرفت از هزار یکی
&&
گرچه صد گونه داستان گفتم
\\
چون نشان برده‌ای دل عطار
&&
هرچه گفتم بدان نشان گفتم
\\
\end{longtable}
\end{center}
