\begin{center}
\section*{غزل شماره ۲۹: دوش کان شمع نیکوان برخاست}
\label{sec:029}
\addcontentsline{toc}{section}{\nameref{sec:029}}
\begin{longtable}{l p{0.5cm} r}
دوش کان شمع نیکوان برخاست
&&
ناله از پیر و از جوان برخاست
\\
گل سرخ رخش چو عکس انداخت
&&
جوش آتش ز ارغوان برخاست
\\
آفتابی که خواجه‌تاش مه است
&&
به غلامیش مدح خوان برخاست
\\
از غم جام خسروی لبش
&&
شور از جان خسروان برخاست
\\
روی بگشاد تا ز هر مویم
&&
صد نگهبان و دیده‌بان برخاست
\\
یارب از تاب زلف هندوی او
&&
چه قیامت ز هندوان برخاست
\\
مشک از چین زلف می‌افشاند
&&
آه از ناف آهوان برخاست
\\
چشم جادوش آتشی در زد
&&
دود از مغز جادوان برخاست
\\
فتنه‌ای کان نشسته بود تمام
&&
باز از آن ماه مهربان برخاست
\\
پیش من آمد و زبان بگشاد
&&
گفت یوسف ز کاروان برخاست
\\
دل به من ده که گر به حق گویی
&&
در غم من ز جان توان برخاست
\\
دل چو رویش بدید دزدیده
&&
بگریخت از من و دوان برخاست
\\
آتش روی او بدید و بسوخت
&&
به تجلی چو آن شبان برخاست
\\
او چو سلطان به زیر پرده نشست
&&
دل تنها چو پاسبان برخاست
\\
چون همه عمر خویش یک مژه زد
&&
همه مغزش ز استخوان برخاست
\\
نتوان کرد شرح کز چه صفت
&&
دل عطار ناتوان برخاست
\\
\end{longtable}
\end{center}
