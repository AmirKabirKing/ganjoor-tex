\begin{center}
\section*{غزل شماره ۷۴۳: بوی زلفت در جهان افکنده‌ای}
\label{sec:743}
\addcontentsline{toc}{section}{\nameref{sec:743}}
\begin{longtable}{l p{0.5cm} r}
بوی زلفت در جهان افکنده‌ای
&&
خویشتن را بر کران افکنده‌ای
\\
از نسیم زلف مشک‌افشان خویش
&&
غلغلی اندر جهان افکنده‌ای
\\
وز کمال نور روی خویشتن
&&
آتشی در عقل و جان افکنده‌ای
\\
وز فروغ لعل روح‌افزای خویش
&&
شورشی در بحر و کان افکنده‌ای
\\
روز و شب از بهر عاشق خواندنت
&&
نعره در کون و مکان افکنده‌ای
\\
می نیایی در میان عاشقان
&&
عاشقان را در گمان افکنده‌ای
\\
بر امید وصل در صحرای دل
&&
بیدلان را در فغان افکنده‌ای
\\
مرغ دل را بر امید گنج وصل
&&
اندرین رنج آشیان افکنده‌ای
\\
روی چون مه زآستین گنج وصل
&&
خون ما بر آستان افکنده‌ای
\\
هر که را دردی است اندر عشق تو
&&
خویشتن در پیش آن افکنده‌ای
\\
دام سودای خود اندر حلق دل
&&
کس چه داند کز چه سان افکنده‌ای
\\
در بلای نیک و بد عطار را
&&
روز و شب در امتحان افکنده‌ای
\\
\end{longtable}
\end{center}
