\begin{center}
\section*{غزل شماره ۷۵۷: ای همه راحت روان، سرو روان کیستی}
\label{sec:757}
\addcontentsline{toc}{section}{\nameref{sec:757}}
\begin{longtable}{l p{0.5cm} r}
ای همه راحت روان، سرو روان کیستی
&&
ملک تو شد جهان جان، جان و جهان کیستی
\\
اینت جمال دلبری مثل تو کس ندیده‌ام
&&
هیچ ندانم ای پسر تا تو از آن کیستی
\\
از لب همچو شکرت پر گهر است عالمی
&&
ای گهر شریف جان گوهر کان کیستی
\\
بی تو چو جان و دل توی سیر شدم ز جان و دل
&&
ای دل و جان من بگو تا دل وجان کیستی
\\
ای زده راه بر دلم نرگس نیم مست تو
&&
رهزن دل شدی مرا روح روان کیستی
\\
عطار از هوای خود سود و زیان ز دست داد
&&
از پی وصل و هجر خود سود و زیان کیستی
\\
\end{longtable}
\end{center}
