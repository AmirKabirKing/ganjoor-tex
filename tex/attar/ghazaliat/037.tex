\begin{center}
\section*{غزل شماره ۳۷: تو را در ره خراباتی خراب است}
\label{sec:037}
\addcontentsline{toc}{section}{\nameref{sec:037}}
\begin{longtable}{l p{0.5cm} r}
تو را در ره خراباتی خراب است
&&
گر آنجا خانه‌ای گیری صواب است
\\
بگیر آن خانه تا ظاهر ببینی
&&
که خلق عالم و عالم سراب است
\\
در آن خانه تو را یکسان نماید
&&
جهانی گر پر آتش گر پر آب است
\\
خراباتی است بیرون از دو عالم
&&
دو عالم در بر آن همچو خواب است
\\
ببین کز بوی درد آن خرابات
&&
فلک را روز و شب چندین شتاب است
\\
به آسانی نیابی سر این کار
&&
که کاری سخن و سری تنک یاب است
\\
به عقل این راه مسپر کاندرین راه
&&
جهانی عقل چون خر در خلاب است
\\
مثال تو درین کنج خرابات
&&
مثال سایه‌ای در آفتاب است
\\
چگونه شرح آن گویم که جانم
&&
ز عشق این سخن مست و خراب است
\\
اگر پرسی ز سر این سؤالی
&&
چه گویم من که خاموشی جواب است
\\
برای جست و جوی این حقیقت
&&
هزاران حلق در دام طناب است
\\
ز درد این سخن پیران ره را
&&
محاسن‌ها به خون دل خضاب است
\\
جوانمردان دین را زین مصیبت
&&
جگرها تشنه و دلها کباب است
\\
ز شرح این سخن وز خجلت خویش
&&
دل عطار در صد اضطراب است
\\
\end{longtable}
\end{center}
