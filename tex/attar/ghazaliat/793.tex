\begin{center}
\section*{غزل شماره ۷۹۳: دی ز دیر آمد برون سنگین دلی}
\label{sec:793}
\addcontentsline{toc}{section}{\nameref{sec:793}}
\begin{longtable}{l p{0.5cm} r}
دی ز دیر آمد برون سنگین دلی
&&
با لبی پرخنده بس مستعجلی
\\
عالمی نظارگی حیران او
&&
دست بر دل مانده پای اندر گلی
\\
علم در وصف لبش لایعملی
&&
عقل در شرح رخش لایعقلی
\\
زلف همچون شست او می‌کرد صید
&&
هر کجا در شهر جانی و دلی
\\
عاشقان را از خیال زلف او
&&
تازه می‌شد هر زمانی مشکلی
\\
تا نگردی هندوی زلفش به جان
&&
نه مبارک باشی و نه مقبلی
\\
جمله پشت دست می‌خایند از او
&&
هست هرجا عالمی و عاقلی
\\
منزل عشقش دل پاک است و بس
&&
نیست عشقش در خور هر منزلی
\\
تا تو بی حاصل نگردی از دو کون
&&
هرگز از عشقش نیابی حاصلی
\\
شد دل عطار غرق بحر عشق
&&
کی تواند غرقه دیدن ساحلی
\\
\end{longtable}
\end{center}
