\begin{center}
\section*{غزل شماره ۲۴۶: پیر ما وقت سحر بیدار شد}
\label{sec:246}
\addcontentsline{toc}{section}{\nameref{sec:246}}
\begin{longtable}{l p{0.5cm} r}
پیر ما وقت سحر بیدار شد
&&
از در مسجد بر خمار شد
\\
از میان حلقهٔ مردان دین
&&
در میان حلقهٔ زنار شد
\\
کوزهٔ دردی به یک دم درکشید
&&
نعره‌ای دربست و دردی‌خوار شد
\\
چون شراب عشق در وی کار کرد
&&
از بد و نیک جهان بیزار شد
\\
اوفتان خیزان چو مستان صبوح
&&
جام می بر کف سوی بازار شد
\\
غلغلی در اهل اسلام اوفتاد
&&
کای عجب این پیر از کفار شد
\\
هر کسی می‌گفت کین خذلان چبود
&&
کان‌چنان پیری چنین غدار شد
\\
هرکه پندش داد بندش سخت کرد
&&
در دل او پند خلقان خار شد
\\
خلق را رحمت همی آمد بر او
&&
گرد او نظارگی بسیار شد
\\
آنچنان پیر عزیز از یک شراب
&&
پیش چشم اهل عالم خوار شد
\\
پیر رسوا گشته مست افتاده بود
&&
تا از آن مستی دمی هشیار شد
\\
گفت اگر بدمستیی کردم رواست
&&
جمله را می‌باید اندر کار شد
\\
شاید ار در شهر بد مستی کند
&&
هر که او پر دل شد و عیار شد
\\
خلق گفتند این گدیی کشتنی است
&&
دعوی این مدعی بسیار شد
\\
پیر گفتا کار را باشید هین
&&
کین گدای گبر دعوی‌دار شد
\\
صد هزاران جان نثار روی آنک
&&
جان صدیقان برو ایثار شد
\\
این بگفت و آتشین آهی بزد
&&
وانگهی بر نردبان دار شد
\\
از غریب و شهری و از مرد و زن
&&
سنگ از هر سو برو انبار شد
\\
پیر در معراج خود چون جان بداد
&&
در حقیقت محرم اسرار شد
\\
جاودان اندر حریم وصل دوست
&&
از درخت عشق برخوردار شد
\\
قصهٔ آن پیر حلاج این زمان
&&
انشراح سینهٔ ابرار شد
\\
در درون سینه و صحرای دل
&&
قصهٔ او رهبر عطار شد
\\
\end{longtable}
\end{center}
