\begin{center}
\section*{غزل شماره ۴۱۰: ای شیوهٔ تو کرشمه و ناز}
\label{sec:410}
\addcontentsline{toc}{section}{\nameref{sec:410}}
\begin{longtable}{l p{0.5cm} r}
ای شیوهٔ تو کرشمه و ناز
&&
تا چند کنی کرشمه آغاز
\\
بستی در دیده از جهانم
&&
بر روی تو دیده کی کنم باز
\\
ای جان تو در اشتیاق می‌سوز
&&
وی دیده در انتظار می‌ساز
\\
تا روز وصال در شب هجر
&&
بر آتش غم چو شمع بگداز
\\
در باز به عشق هرچه داری
&&
در صف مقامران جانباز
\\
پیمانهٔ هر دو کون درکش
&&
یعنی که دو کون را برانداز
\\
ای باز چو صید کون کردی
&&
بازآی به دست شه چو شهباز
\\
ای نوپر آشیان علوی
&&
بر پر سوی آشیانه شو باز
\\
گردون خرفی است بس زبون گیر
&&
گیتی زنکی است بس فسون ساز
\\
بر مرکب روح گرد راکب
&&
زین بادیه تازیان برون تاز
\\
چون غمزده قصهٔ غم خویش
&&
با غمزه مگو که هست غماز
\\
در مجلس کم زنان قدح نوش
&&
در خلوت عاشقان طرب ساز
\\
مقراض اجل گرت برد سر
&&
چون شمع سر آور از دم گاز
\\
خون خوار زمین گرت خورد خون
&&
مانند نبات شو سرافراز
\\
چون جوهر فرد باش یعنی
&&
از خلق زمانه باش ممتاز
\\
تا کی چون مقلدان غافل
&&
تا چند چو غافلان پر آز
\\
تا جان ندهی تو همچو عطار
&&
بیرون مده از درون دل راز
\\
\end{longtable}
\end{center}
