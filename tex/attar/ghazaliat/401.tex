\begin{center}
\section*{غزل شماره ۴۰۱: باد شمال می‌وزد، طرهٔ یاسمن نگر}
\label{sec:401}
\addcontentsline{toc}{section}{\nameref{sec:401}}
\begin{longtable}{l p{0.5cm} r}
باد شمال می‌وزد، طرهٔ یاسمن نگر
&&
وقت سحر ز عشق گل، بلبل نعره زن نگر
\\
سبزهٔ تازه روی را، نو خط جویبار بین
&&
لالهٔ سرخ روی را، سوخته‌دل چو من نگر
\\
خیری سرفکنده را، در غم عمر رفته بین
&&
سنبل شاخ شاخ را، مروحه چمن نگر
\\
یاسمن دوشیزه را، همچو عروس بکر بین
&&
باد مشاطه فعل را، جلوه‌گر سمن نگر
\\
نرگس نیم مست را، عاشق زرد روی بین
&&
سوسن شیرخواره را، آمده در سخن نگر
\\
لعبت شاخ ارغوان، طفل زبان گشاده بین
&&
ناوک چرخ گلستان، غنچهٔ بی دهن نگر
\\
تا که بنفشه باغ را، صوفی فوطه‌پوش کرد
&&
از پی ره زنی او، طرهٔ یاسمن نگر
\\
تا گل پادشاه وش، تخت نهاد در چمن
&&
لشکریان باغ را، خیمهٔ نسترن نگر
\\
خیز و دمی به وقت گل، باده بده که عمر شد
&&
چند غم جهان خوری، شادی انجمن نگر
\\
هین که گذشت وقت گل، سوی چمن نگاه کن
&&
راح نسیم صبح بین، ابر گلاب زن نگر
\\
نی بگذر ازین همه، وز سر صدق فکر کن
&&
وین شکن زمانه را، پر بت سیم‌تن نگر
\\
ای دل خفته عمر شد، تجربه گیر از جهان
&&
زندگیی به دست کن، مردن مرد و زن نگر
\\
از سر خاک دوستان، سبزه دمید خون گری
&&
ماتم دوستان مکن، رفتن خویشتن نگر
\\
جملهٔ خاک خفتگان، موج دریغ می‌زند
&&
درنگر و ز خاکشان، حسرت تن به تن نگر
\\
فکر کن و به چشم دل، حال گذشتگان ببین
&&
ریخته زیر خاکشان، طرهٔ پرشکن نگر
\\
آنکه حریر و خز نسود، از سر ناز این زمان
&&
چهرهٔ او ز خاک بین، قامتش از کفن نگر
\\
سوختی ای فرید تو، در غم هجر خود بسی
&&
دلشدهٔ فراق بین، سوختهٔ محن نگر
\\
\end{longtable}
\end{center}
