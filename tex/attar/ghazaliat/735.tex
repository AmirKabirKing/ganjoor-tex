\begin{center}
\section*{غزل شماره ۷۳۵: جهان جمله تویی تو در جهان نه}
\label{sec:735}
\addcontentsline{toc}{section}{\nameref{sec:735}}
\begin{longtable}{l p{0.5cm} r}
جهان جمله تویی تو در جهان نه
&&
همه عالم تویی تو در میان نه
\\
چه دریایی است این دریای پر موج
&&
همه در وی گم و از وی نشان نه
\\
چه راه است این نه سر پیدا و نه پای
&&
ولیکن راه محو و کاروان نه
\\
خیالی و سرابی می‌نماید
&&
چو بوقلمون هویدا و نهان نه
\\
همه تا بنگری ناچیز گردد
&&
همه چیزی چنین و آن چنان نه
\\
عجب کاری است کار سر معشوق
&&
جهان از وی پر و او در جهان نه
\\
همه دل پر ازو و دل درو محو
&&
نشسته در میان جان و جان نه
\\
اگر ظاهر شود مویی جز او نی
&&
وگر باطن بود مویی عیان نه
\\
عجب سری که یک یک ذره آن است
&&
چه می‌گویم همین است و همان نه
\\
دلی دارم درو صد عالم اسرار
&&
ولیکن شرح یک سر را زبان نه
\\
چنین جایی فرید آخر چه گوید
&&
زبان گنگ و سخن قطع و بیان نه
\\
\end{longtable}
\end{center}
