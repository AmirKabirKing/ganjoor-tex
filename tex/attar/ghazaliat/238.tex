\begin{center}
\section*{غزل شماره ۲۳۸: هم بلای تو به جان بی قراران می‌رسد}
\label{sec:238}
\addcontentsline{toc}{section}{\nameref{sec:238}}
\begin{longtable}{l p{0.5cm} r}
هم بلای تو به جان بی قراران می‌رسد
&&
هم غم عشقت نصیب غمگساران می‌رسد
\\
ذره‌ای غم از تو چون خواهد گدای کوی تو
&&
کین چنین میراث غم با شهسواران می‌رسد
\\
من ندارم زهره خاک پای تو کردن طمع
&&
زانکه این دولت به فرق تاجداران می‌رسد
\\
هر کسی از نقش روی تو خیالی می‌کند
&&
پس به بوی وصل تو چون خواستاران می‌رسد
\\
هیچ کس را در دمی صورت نبندد تا چرا
&&
نقش روی تو بدین صورت نگاران می‌رسد
\\
گل مگر لافی زد از خوبی کنون پیش رخت
&&
عذر خواه از ده زبان چون شرمساران می‌رسد
\\
پیش رویت بلبل ار در پیش می‌آید شفیع
&&
او عرق کرده ز پس چون میگساران می‌رسد
\\
دور از روی تو نتواند بروی کس رسید
&&
آنچه از رویت به روی دوستداران می‌رسد
\\
زلف شبرنگت چو بر گلگون سواری می‌کند
&&
عالمی فتنه به روی بی قراران می‌رسد
\\
رخ چو گلبرگ بهار از من چرا پوشی به زلف
&&
کاشک من دور از تو چون ابر بهاران می‌رسد
\\
بر خطت چون زار می‌گریم مکن منعم ازانک
&&
این همه سرسبزی سبزه ز باران می‌رسد
\\
کی رسد آشفتگی از روزگار بوالعجب
&&
آنچه از چشمت بدین آشفته‌کاران می‌رسد
\\
دل سپر بفکند از هر غمزهٔ چشم تو بس
&&
در کم از یک چشم زد صد تیرباران می‌رسد
\\
هیچ درمانم نکردی تا که یارم خوانده‌ای
&&
جملهٔ درد تو گویی قسم یاران می‌رسد
\\
چون طمع ببریدن از وصلت نشان کافری است
&&
لاجرم عطار چون امیدواران می‌رسد
\\
\end{longtable}
\end{center}
