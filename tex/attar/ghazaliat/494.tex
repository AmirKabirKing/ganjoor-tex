\begin{center}
\section*{غزل شماره ۴۹۴: تا روی تو قبلهٔ نظر کردم}
\label{sec:494}
\addcontentsline{toc}{section}{\nameref{sec:494}}
\begin{longtable}{l p{0.5cm} r}
تا روی تو قبلهٔ نظر کردم
&&
از کوی تو کعبهٔ دگر کردم
\\
تا روی به کعبهٔ تو آوردم
&&
صد گونه سجود معتبر کردم
\\
سرگشته شدم که گرد آن کعبه
&&
هر لحظه طواف بیشتر کردم
\\
روزی نه به اختیار می‌رفتم
&&
در دفتر عشق تو نظر کردم
\\
گویی که هزار سال می‌خواندم
&&
تا جمله به یک نفس زبر کردم
\\
چون جان و جهان خود تو را دیدم
&&
جان دادم و از جهان گذر کردم
\\
زآن روز که پردهٔ تو جان دیدم
&&
سوراخ به جان خویش در کردم
\\
بر روزن دل مقیم بنشستم
&&
جان پیش تو بر میان کمر کردم
\\
چون اصل همه جمال تو دیدم
&&
ترک بد و نیک و خیر و شر کردم
\\
آنگه که دلم چو آفتابی شد
&&
در خود همه چون فلک سفر کردم
\\
افسانهٔ دولت تو می‌گفتند
&&
من سوخته‌سر ز خاک بر کردم
\\
چون نعره‌زنان به میکده رفتم
&&
هم رقص‌کنان ز پای سر کردم
\\
چون بوی شراب عشق بشنودم
&&
خود را ز دو کون بی خبر کردم
\\
عطار شکسته را همی هر دم
&&
از عشق رخت درست تر کردم
\\
\end{longtable}
\end{center}
