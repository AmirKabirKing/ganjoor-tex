\begin{center}
\section*{غزل شماره ۳۹۴: ای در درون جانم و جان از تو بی خبر}
\label{sec:394}
\addcontentsline{toc}{section}{\nameref{sec:394}}
\begin{longtable}{l p{0.5cm} r}
ای در درون جانم و جان از تو بی خبر
&&
وز تو جهان پر است و جهان از تو بی خبر
\\
چون پی برد به تو دل و جانم که جاودان
&&
در جان و در دلی دل و جان از تو بی خبر
\\
ای عقل پیر و بخت جوان گرد راه تو
&&
پیر از تو بی نشان و جوان از تو بی خبر
\\
نقش تو در خیال و خیال از تو بی نصیب
&&
نام تو بر زبان و زبان از تو بی خبر
\\
از تو خبر به نام و نشان است خلق را
&&
وآنگه همه به نام و نشان از تو بی خبر
\\
جویندگان جوهر دریای کنه تو
&&
در وادی یقین و گمان از تو بی خبر
\\
چون بی خبر بود مگس از پر جبرئیل
&&
از تو خبر دهند و چنان از تو بی خبر
\\
شرح و بیان تو چه کنم زانکه تا ابد
&&
شرح از تو عاجز است و بیان از تو بی خبر
\\
عطار اگرچه نعرهٔ عشق تو می‌زند
&&
هستند جمله نعره‌زنان از تو بی خبر
\\
\end{longtable}
\end{center}
