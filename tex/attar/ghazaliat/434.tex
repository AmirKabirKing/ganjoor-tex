\begin{center}
\section*{غزل شماره ۴۳۴: ای ز عشقت این دل دیوانه خوش}
\label{sec:434}
\addcontentsline{toc}{section}{\nameref{sec:434}}
\begin{longtable}{l p{0.5cm} r}
ای ز عشقت این دل دیوانه خوش
&&
جان و دردت هر دو در یک خانه خوش
\\
گر وصال است از تو قسمم گر فراق
&&
هست هر دو بر من دیوانه خوش
\\
من چنان در عشق غرقم کز توام
&&
هم غرامت هست و هم شکرانه خوش
\\
دل بسی افسانهٔ وصل تو گفت
&&
تا که شد در خواب ازین افسانه خوش
\\
گر تو ای دل عاشقی پروانه‌وار
&&
از سر جان درگذر مردانه خوش
\\
نه که جان درباختن کار تو نیست
&&
جان فشاندن هست از پروانه خوش
\\
قرب سلطان جوی و پروانه مجوی
&&
روستایی باشد از پروانه خوش
\\
گر تو مرد آشنایی چون شوی
&&
از شرابی همچو آن بیگانه خوش
\\
هر که صد دریا ندارد حوصله
&&
تا ابد گردد به یک پیمانه خوش
\\
مرد این ره آن زمانی کز دو کون
&&
مفلسی باشی درین ویرانه خوش
\\
تو از آن مرغان مدان عطار را
&&
کز دو عالم آیدش یک دانه خوش
\\
\end{longtable}
\end{center}
