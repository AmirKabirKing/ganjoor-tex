\begin{center}
\section*{غزل شماره ۶۱: معشوقه از آن ظریفتر نیست}
\label{sec:061}
\addcontentsline{toc}{section}{\nameref{sec:061}}
\begin{longtable}{l p{0.5cm} r}
معشوقه از آن ظریفتر نیست
&&
زان عشوه‌فروش و عشوه خر نیست
\\
شهریست پر از شگرف لیکن
&&
زو هیچ بتی شگرف‌تر نیست
\\
مریم کده‌ها بسیست لیکن
&&
کس را چو مسیح یک پسر نیست
\\
فرزند بسیست چرخ را لیک
&&
انصاف بده چنو دگر نیست
\\
آن کیست که پیش تیر بالاش
&&
چون نیزه همه تنش کمر نیست
\\
چون او قمری قمار دل را
&&
در زیر ولایت قمر نیست
\\
شمشیرکشان چشم او را
&&
جز دیدهٔ عاشقان سپر نیست
\\
آن کیست کز آفتاب رویش
&&
چون کان همه خاطرش گهر نیست
\\
در تاب دو زلفش از بلاها
&&
یارب زنهار تا چه در نیست
\\
از بلعجبان نیایدش روی
&&
رویش گویان که روی گر نیست
\\
سم زهر بود به لفظ تازی
&&
زو هیچ خطیر با خطر نیست
\\
دندان و لب چو سین و میمش
&&
این نادره بین که جز شکر نیست
\\
در عشق و بلاش جان و دل را
&&
حقا که جز از حذر حذر نیست
\\
شادی و غمست عشق و ما را
&&
غم هست ولیک آن دگر نیست
\\
از رد و قبول دلبران را
&&
چه سود که هیچ بی‌جگر نیست
\\
او سیم‌بر است و سیم زی او
&&
گر زر نبود ترا خطر نیست
\\
ما را چه ز سیم او که ما را
&&
روی چو زرست و روی زر نیست
\\
حقا که ظریف روزگاران
&&
گر هست حریف ما دگر نیست
\\
ما را کلهی نهاد عشقش
&&
کان بر سر هیچ تاجور نیست
\\
اندر طلبش سوی سنایی
&&
غم تاج سرست و درد سر نیست
\\
\end{longtable}
\end{center}
