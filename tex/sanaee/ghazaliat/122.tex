\begin{center}
\section*{غزل شماره ۱۲۲: ما را ز مه عشق تو سالی دگر آمد}
\label{sec:122}
\addcontentsline{toc}{section}{\nameref{sec:122}}
\begin{longtable}{l p{0.5cm} r}
ما را ز مه عشق تو سالی دگر آمد
&&
دور از ره هجر تو وصالی دگر آمد
\\
در دیده خیالی که مرا بد ز رخ تو
&&
یکباره همه رفت و خیالی دگر آمد
\\
بر مرکب شایسته شهنشاه شکوهت
&&
بر تخت دل من به جمالی دگر آمد
\\
شد نقص کمالی که مرا بود به صورت
&&
در عالم تحقیق کمالی دگر آمد
\\
بر طبل طلب می‌زدم از حرص دوالی
&&
ناگاه بر آن طبل دوالی دگر آمد
\\
از سینه نهال امل از بیم بکندم
&&
با میوهٔ انصاف نهالی دگر آمد
\\
بر عشوه ز من رفت به تعریض نکالات
&&
آسوده به تصریح نکالی دگر آمد
\\
در وصف صفا حیدر اقبال به چشمم
&&
بر دلدل دولت به دلالی دگر آمد
\\
\end{longtable}
\end{center}
