\begin{center}
\section*{غزل شماره ۹۶: آنکس که ز عاشقی خبر دارد}
\label{sec:096}
\addcontentsline{toc}{section}{\nameref{sec:096}}
\begin{longtable}{l p{0.5cm} r}
آنکس که ز عاشقی خبر دارد
&&
دایم سر نیش بر جگر دارد
\\
جان را به قضای عشق بسپارد
&&
تن پیش بلا و غم سپر دارد
\\
گه دست بلا فراز دل گیرد
&&
گه سنگ تعب به زیر سر دارد
\\
پیوسته چو من فگنده تن گردد
&&
دل را ز هوای نفس بر دارد
\\
بگسسته شود ز شهر و ز مسکن
&&
هر دم زدنی رهی دگر دارد
\\
هر چند که زهر عشق می نوشد
&&
آن زهر به گونهٔ شکر دارد
\\
وان دیده به دست غیر بردوزد
&&
کو جز به جمال حق نظر دارد
\\
ای یار مقامر خراباتی
&&
طبع تو طریق مختصر دارد
\\
بنمای به من کسی که او چون من
&&
در کوی مقامری مقر دارد
\\
یا از ره کم زنان نشان جوید
&&
یا از دل بی دلان خبر دارد
\\
\end{longtable}
\end{center}
