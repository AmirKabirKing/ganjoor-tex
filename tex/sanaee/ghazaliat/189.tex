\begin{center}
\section*{غزل شماره ۱۸۹: ای ز خوبی مست هان هشیار باش}
\label{sec:189}
\addcontentsline{toc}{section}{\nameref{sec:189}}
\begin{longtable}{l p{0.5cm} r}
ای ز خوبی مست هان هشیار باش
&&
ور ز مستی خفته‌ای بیدار باش
\\
از شراب شوق رویت عالمی
&&
گشته مستانند هان هشیار باش
\\
گر مه میخواره خوانندت رواست
&&
می به شادی نوش و بی تیمار باش
\\
خویشتن‌داری کن اندر کارها
&&
خصم بر کارست هان بر کار باش
\\
گاه بزم افروز عاشق سوز باش
&&
گاه صاحب درد و دردی خوار باش
\\
زینهاری دارم اندر گردنت
&&
زینهار ای بت بران زنهار باش
\\
چون ز خصمان خویشتن داری کنی
&&
دستبردی بر جهان سالار باش
\\
هم چنین از خویشتن داری مدام
&&
تا توانی سر کش و عیار باش
\\
بر در دیوار خود ایمن مباش
&&
بر حذر هان از در و دیوار باش
\\
کار تو باید که باشد بر نظام
&&
کارهای عاشقان گو زار باش
\\
گر سنایی از تو برخوردار نیست
&&
تو ز بخت خویش برخوردار باش
\\
\end{longtable}
\end{center}
