\begin{center}
\section*{غزل شماره ۴۰۲: آن دلبر عیار من ار یار منستی}
\label{sec:402}
\addcontentsline{toc}{section}{\nameref{sec:402}}
\begin{longtable}{l p{0.5cm} r}
آن دلبر عیار من ار یار منستی
&&
کوس «لمن الملک» زدن کار منستی
\\
گر هیچ کلاهی نهدم از سر تشریف
&&
سیاره کنون ریشهٔ دستار منستی
\\
بر افسر شاهان جهانم بودی فخر
&&
کر پاردم مرکبش افسار منستی
\\
ور گل دهدی چشم مر از آن رخ چون باغ
&&
صحرای فلک جمله سمن زار منستی
\\
گرهیچ عزیز دهدم از پس خواری
&&
بالله همه گلهای جهان خار منستی
\\
جوزای کمرکش کشدی غاشیهٔ من
&&
گر حشمت او همره زنار منستی
\\
ور کژدم زلفش گزدی مر جگرم را
&&
هر چیز که آن مال جهان مار منستی
\\
هر روز دلی نو دهدم از دو لب خویش
&&
گر دیدهٔ شوخش نه جگر خوار منستی
\\
یاری که نسوزد نه بسازد ز لب او
&&
شایستی اگر در دل بیمار منستی
\\
گر هیچ قبولم کندی سایهٔ آن در
&&
خورشید کنون سایهٔ دیوار منستی
\\
گر لطف لبش نیستی از قهر دو زلفش
&&
هر چوب که افراخته‌تر دار منستی
\\
گویند که جز هیچ کسان را نخرد یار
&&
من هیچکسم کاش خریدار منستی
\\
ور داغ سنایی ننهادی صفت او
&&
کی خلق چنین سغبهٔ گفتار منستی
\\
\end{longtable}
\end{center}
