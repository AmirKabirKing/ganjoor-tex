\begin{center}
\section*{غزل شماره ۳۴۷: ای شکسته رونق بازار جان بازار تو}
\label{sec:347}
\addcontentsline{toc}{section}{\nameref{sec:347}}
\begin{longtable}{l p{0.5cm} r}
ای شکسته رونق بازار جان بازار تو
&&
عالمی دلسوخته از خامی گفتار تو
\\
توشه هر روزی مرا از گوشهٔ انده نهد
&&
گوشهٔ شبپوش تو بر طرهٔ طرار تو
\\
خوبی خوبان عالم گر بسنجی بی‌غلط
&&
صد یکی زان هیچ پیش کفهٔ معیار تو
\\
عشق تو مرغی‌ست کو را این خطابست از خرد
&&
ای دو عالم گشته عاجز در سر منقار تو
\\
حلقه بودن شرط باشد بر در هستی خود
&&
هر که در دیوار دارد روی از آزار تو
\\
نیست منزل صبر را یک لحظه پیش من چنانک
&&
نیست قیمت شرم را یک ذره در بازار تو
\\
زین گذشتست ای صنم در عشقبازی کار من
&&
زان گذشتست ای پسر در شوخ چشمی کار تو
\\
ترس من در عذر تو افزون بود از جنگ از آنک
&&
نفی استغفار باشد عین استغفار تو
\\
ایمنی از چشم بد زان کز صفا بینندگان
&&
جز که شکل خود نمی‌بینند در رخسار تو
\\
فارغی از بند پرده چون همی دانی که نیست
&&
هیچ پرده پیش دیدار تو چون دیدار تو
\\
\end{longtable}
\end{center}
