\begin{center}
\section*{غزل شماره ۴۱۰: روی چو ماه داری زلف سیاه داری}
\label{sec:410}
\addcontentsline{toc}{section}{\nameref{sec:410}}
\begin{longtable}{l p{0.5cm} r}
روی چو ماه داری زلف سیاه داری
&&
بر سرو ماه داری بر سر کلاه داری
\\
خال تو بوسه خواهد لیکن هم از لب تو
&&
هم بوسه جای داری هم بوسه خواه داری
\\
زلف تو بر دل من بندی نهاد محکم
&&
گفتم که بند دارم گفتا گناه داری
\\
یکره بپرس جانا زان زلف مشکبویت
&&
تا بر گل مورد چون خوابگاه داری
\\
دل جایگاه دارد اندر میان آتش
&&
تو در میان آن دل چون جایگاه داری
\\
مست ثنای عشقست در مجلست سنایی
&&
گر هیچ عقل داری او را نگاه داری
\\
\end{longtable}
\end{center}
