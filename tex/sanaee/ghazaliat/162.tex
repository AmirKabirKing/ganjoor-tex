\begin{center}
\section*{غزل شماره ۱۶۲: ای نهاده بر گل از مشک سیه پیچان دو مار}
\label{sec:162}
\addcontentsline{toc}{section}{\nameref{sec:162}}
\begin{longtable}{l p{0.5cm} r}
ای نهاده بر گل از مشک سیه پیچان دو مار
&&
هین که از عالم برآورد آن دو مار تو دمار
\\
روی تو در هر دلی افروخته شمع و چراغ
&&
زلف تو در هر تنی جان سوخته پروانه‌وار
\\
هر کجا بوییست خطت تاخته آنجا سپاه
&&
هر کجا رنگیست خالت ساخته آنجا قرار
\\
آتش عشقت ببرده عالمی را آبروی
&&
باد هجرانت نشانده کشوری را خاکسار
\\
تا ترا بر یاسمین رست از بنفشه برگ مورد
&&
عاشقان را زعفران رست از سمن بر لاله‌زار
\\
یوسف عصر ار نه‌ای پس چون که اندر عشق تو
&&
خونفشان یعقوب بینم هر زمانی صدهزار
\\
ماه را مانی غلط کردم که مر خورشید را
&&
نورمند از خاک پای تست نورانی عذار
\\
قیروان عشوه بگذارند غواصان دهر
&&
گر نهنگ عشق تو بخرامد از دریای قار
\\
گر براندازی نقاب از روی روح افزای خود
&&
رخت بردارد ز کیهان زحمت لیل و نهار
\\
هر که بر روی تو باشد عاشق ای جان جهان
&&
با جهان جان نباشد بود او را هیچ کار
\\
عالم کون و فساد از کفر و دین آراسته‌ست
&&
عالم عشق از دل بریان و چشم اشکبار
\\
در جهان عشق ازین رمز و حکایت هیچ نیست
&&
کاین مزخرف پیکران گویند بر سرهای دار
\\
وای اگر دستی برآرد در جهان انصاف تو
&&
در همه صحرای جان یک تن نماند پایدار
\\
بر تو کس در می‌نگنجد تالی الا الله چو لا
&&
حاجبی دارد کشیده تیغ در ایوان نار
\\
لاف گویان اناالله را ببین در عشق خویش
&&
بر بساط عشق بنهاده جبین اختیار
\\
من نه تنها عاشقم بر تو که بر هفت آسمان
&&
کشته هست از عشق تو چندان که ناید در شمار
\\
من شناسم مر ترا کز هفتمین چرخ آمدم
&&
بچهٔ عشق ترا پرورده بر دوش و کنار
\\
\end{longtable}
\end{center}
