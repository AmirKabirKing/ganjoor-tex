\begin{center}
\section*{غزل شماره ۱۴۱: لشکر شب رفت و صبح اندر رسید}
\label{sec:141}
\addcontentsline{toc}{section}{\nameref{sec:141}}
\begin{longtable}{l p{0.5cm} r}
لشکر شب رفت و صبح اندر رسید
&&
خیز و مهرویا فراز آور نبید
\\
چشم مست پر خمارت باز کن
&&
کز نشاطت صبرم از دل بر پرید
\\
مطرب سرمست را آواز ده
&&
چون ز میخانه عصیر اندر رسید
\\
پر مکن جام ای صنم امشب چو دوش
&&
کت همه جامه چکانه بر چکید
\\
نیست گویی آن حکایت راستی
&&
خون دل بر گرد چشم ما دوید
\\
کیست کز عشقت نه بر خاک اوفتاد
&&
کیست کز هجرت نه جامه بر درید
\\
چون خطت طغرای شاهنشاه یافت
&&
از فنا خط گردد عالم بر کشید
\\
از سنایی زارتر در عشق کیست
&&
یا چو تو دلبر به زیبایی که دید
\\
\end{longtable}
\end{center}
