\begin{center}
\section*{غزل شماره ۱۴: ساقیا دانی که مخموریم در ده جام را}
\label{sec:014}
\addcontentsline{toc}{section}{\nameref{sec:014}}
\begin{longtable}{l p{0.5cm} r}
ساقیا دانی که مخموریم در ده جام را
&&
ساعتی آرام ده این عمر بی آرام را
\\
میر مجلس چون تو باشی با جماعت در نگر
&&
خام در ده پخته را و پخته در ده خام را
\\
قالب فرزند آدم آز را منزل شدست
&&
انده پیشی و بیشی تیره کرد ایام را
\\
نه بهشت از ما تهی گردد نه دوزخ پر شود
&&
ساقیا در ده شراب ارغوانی فام را
\\
قیل و قال بایزید و شبلی و کرخی چه سود
&&
کار کار خویش دان اندر نورد این نام را
\\
تا زمانی ما برون از خاک آدم دم زنیم
&&
ننگ و نامی نیست بر ما هیچ خاص و عام را
\\
\end{longtable}
\end{center}
