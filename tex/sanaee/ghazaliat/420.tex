\begin{center}
\section*{غزل شماره ۴۲۰: به درگاه عشقت چه نامی چه ننگی}
\label{sec:420}
\addcontentsline{toc}{section}{\nameref{sec:420}}
\begin{longtable}{l p{0.5cm} r}
به درگاه عشقت چه نامی چه ننگی
&&
به نزد جلالت چه شاهی چه شنگی
\\
جهان پر حدیث وصال تو بینم
&&
زهی نارسیده به زلف تو چنگی
\\
همانا به صحرا نظر کرده‌ای تو
&&
که صحرا ز رویت گرفتست رنگی
\\
ز عکس رخ تو به هر مرغزاری
&&
ز دیبای چینی گشادست تنگی
\\
شگفت آهوی تو که صید تو سازد
&&
به هر چشم زخمی دلاور پلنگی
\\
ز جعدت کمندی و شهری پیاده
&&
جهانی سوار و ز چشمت خدنگی
\\
اگر خواهی ارواح مرغان علوی
&&
فرود آری از شاخ طوبا به سنگی
\\
به تو کی رسد هرگز از راه گفتی
&&
بر نار و نورت که دارد درنگی
\\
کیم من که از نوش وصل تو گویم
&&
نپوید پی شیر روباه لنگی
\\
من آن عاشقم کز تو خشنود باشم
&&
ز نوشی به زهری ز صلحی به جنگی
\\
\end{longtable}
\end{center}
