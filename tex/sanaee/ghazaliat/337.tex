\begin{center}
\section*{غزل شماره ۳۳۷: ای چون تو ندیده جم آخر چه جمالست این}
\label{sec:337}
\addcontentsline{toc}{section}{\nameref{sec:337}}
\begin{longtable}{l p{0.5cm} r}
ای چون تو ندیده جم آخر چه جمالست این
&&
وی چون تو به عالم کم آخر چه کمالست این
\\
تو با من و من پویان هر جای ترا جویان
&&
ای شمع نکورویان آخر چه وصالست این
\\
زان گلبن انسانی هر دم گلی افشانی
&&
ای میوهٔ روحانی آخر چه نهالست این
\\
در وصف تو عقل و جان چون من شده سرگردان
&&
ای وهم ز تو حیران آخر چه جمالست این
\\
گفتی که چو من دلبر داری وز من بهتر
&&
ای جادوی صورت گر آخر چه خیالست این
\\
ای از پی داغ ما آرایش باغ ما
&&
ای چشم و چراغ ما آخر چه مثالست این
\\
هر روز نپویی تو جز عشق نجویی تو
&&
ای ماه نکویی تو آخر چه خصالست این
\\
هر روز مرا نرمک بکشی تو به آزرمک
&&
ای شوخک بی‌شرمک آخر چه وبالست این
\\
پرسی: چو منی دلبر بینی تو به عالم در
&&
ای ماه نکو منظر آخر چه سوالست این
\\
ما را نه بدین سستی زین بیش همی جستی
&&
ای خسته از آن خستی آخر چه ملالست این
\\
گفتی همه جا با تو وصلست مرا با تو
&&
ای بی خود و با ما تو آخر چه دلالست این
\\
گفتی که سنایی خود داریم و ازو به صد
&&
ای ناقد نیک و بد آخر چه محالست این
\\
\end{longtable}
\end{center}
