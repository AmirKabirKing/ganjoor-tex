\begin{center}
\section*{غزل شماره ۱۹۸: الا ای دلربای خوش بیا کامد بهاری خوش}
\label{sec:198}
\addcontentsline{toc}{section}{\nameref{sec:198}}
\begin{longtable}{l p{0.5cm} r}
الا ای دلربای خوش بیا کامد بهاری خوش
&&
شراب تلخ ما را ده که هست این روزگاری خوش
\\
سزد گر ما به دیدارت بیاراییم مجلس را
&&
چو شد آراسته گیتی به بوی نوبهاری خوش
\\
همی بوییم هر ساعت همی نوشیم هر لحظه
&&
گل اندر بوستانی نو مل اندر مرغزاری خوش
\\
گهی از دست تو گیریم چون آتش می صافی
&&
گهی در وصف تو خوانیم شعر آبداری خوش
\\
کنون در انتظار گل سراید هر شبی بلبل
&&
غزلهای لطیف خوش به نغمه‌های زاری خوش
\\
شود صحرا همه گلشن شود گیتی همه روشن
&&
چو خرم مجلس عالی و باد مشکباری خوش
\\
\end{longtable}
\end{center}
