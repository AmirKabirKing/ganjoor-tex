\begin{center}
\section*{غزل شماره ۲۴۱: به دردم به دردم که اندیشه دارم}
\label{sec:241}
\addcontentsline{toc}{section}{\nameref{sec:241}}
\begin{longtable}{l p{0.5cm} r}
به دردم به دردم که اندیشه دارم
&&
کز آن یاسمین بر تهی شد کنارم
\\
به وقتی که دولت بپیوست با من
&&
بپیوست هجرش به غم روزگارم
\\
که داند که حالم چگونست بی تو
&&
که داند که شبها همی چون گذارم
\\
خیالش ربودست خواب از دو چشم
&&
گرفتنش باید همی استوارم
\\
ز من برد نرمک همی هوشیاری
&&
کنون با غم او نه بس هوشیارم
\\
اگر غمگنان را غم اندر دل آمد
&&
چرا غمگنم من چو من دل ندارم
\\
چون آن گوهر پاک از من جدا شد
&&
سزد گر من از چشم یاقوت بارم
\\
وگر من نپایم به آزاد مردی
&&
ببینند مردم که چون بی قرارم
\\
همی داد ندهد زمانه مهان را
&&
اگر داد دادی نرفتی نگارم
\\
چو من یادگارش دل راد دارم
&&
دهد هجر گویی به جان زینهارم
\\
\end{longtable}
\end{center}
