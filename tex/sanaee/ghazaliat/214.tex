\begin{center}
\section*{غزل شماره ۲۱۴: از حل و از حرام گذشتست کام عشق}
\label{sec:214}
\addcontentsline{toc}{section}{\nameref{sec:214}}
\begin{longtable}{l p{0.5cm} r}
از حل و از حرام گذشتست کام عشق
&&
هستی و نیستی ست حلال و حرام عشق
\\
تسبیح و دین و صومعه آمد نظام زهد
&&
زنار و کفر و میکده آمد نظام عشق
\\
خالیست راه عشق ز هستی بر آن صفت
&&
کز روی حرف پردهٔ عشقست نام عشق
\\
بر نظم عشق مهره فرو باز بهر آنک
&&
از عین و شین و قاف تبه شد قوام عشق
\\
چندین هزار جان مقیمان سفر گزید
&&
جانی هنوز تکیه نزد در مقام عشق
\\
این طرفه‌تر که هر دو جهان پاک شد ز دست
&&
با این هنوز گردن ما زیر وام عشق
\\
برخاست اختیار و تصرف ز فعل ما
&&
چون کم زدیم خویشتن از بهر کام عشق
\\
اندر کنشت و صومعه بی‌بیم و بی‌امید
&&
درباختیم صد الف از بهر لام عشق
\\
برداشت پرده‌های تشابه ز بهر ما
&&
تا روی داد سوی دل ما پیام عشق
\\
مستی همی کنم ز شراب بلا ولیک
&&
هر روز برترست چنین ازدحام عشق
\\
آزاده مانده‌ایم ز کام و هوای خویش
&&
تا گشته‌ایم از سر معنی غلام عشق
\\
دامست راه عشق و نهاده به شاهراه
&&
بادام و بند خلق سنایی به دام عشق
\\
زان دولتی که بی‌خبران را نصیبه‌ایست
&&
کم باد نام عاشق و گم باد نام عشق
\\
چون یوسف سعید بفرمودم این غزل
&&
بادا دوام دولت او چون دوام عشق
\\
\end{longtable}
\end{center}
