\begin{center}
\section*{غزل شماره ۲۵۰: چو دانستم که گردنده‌ست عالم}
\label{sec:250}
\addcontentsline{toc}{section}{\nameref{sec:250}}
\begin{longtable}{l p{0.5cm} r}
چو دانستم که گردنده‌ست عالم
&&
نیاید مرد را بنیاد محکم
\\
پس آن بهتر که ما در وی مقیمیم
&&
شبان و روز با هم مست و خرم
\\
مرا زان چه که چونان گفت ابلیس
&&
مرا زان چه که چونین کرد آدم
\\
تو گویی می مخور من می خورم می
&&
تو گویی کم مزن من می‌زنم کم
\\
فتادی تو به کعبه من به خاور
&&
الا تا چند ازین دوری و درهم
\\
من و خورشید و معشوق و می لعل
&&
تو و رکن و مقام و آب زمزم
\\
ترا کردم مسلم کوثر و خلد
&&
مسلم کن مرا باری جهنم
\\
به فردوس از چه طاعت شد سگ کهف
&&
به دوزخ از چه عصیان رفت بلعم
\\
تو گر هستی چو بلعم در عبادت
&&
من آخر از سگی کمتر نیم هم
\\
سرانجام من و تو روز محشر
&&
ندانم چون بود والله اعلم
\\
سخن‌گویی تو همواره ز اسلام
&&
همه اسلام تو صلوات و سلم
\\
زدن در کوی معنی دم نیاری
&&
همه پیراهن دعوی زنی دم
\\
\end{longtable}
\end{center}
