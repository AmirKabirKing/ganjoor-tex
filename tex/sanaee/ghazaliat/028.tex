\begin{center}
\section*{غزل شماره ۲۸: تا نقش خیال دوست با ماست}
\label{sec:028}
\addcontentsline{toc}{section}{\nameref{sec:028}}
\begin{longtable}{l p{0.5cm} r}
تا نقش خیال دوست با ماست
&&
ما را همه عمر خود تماشاست
\\
آنجا که جمال دلبر آمد
&&
والله که میان خانه صحراست
\\
وانجا که مراد دل برآمد
&&
یک خار به از هزار خرماست
\\
گر چه نفس هوا ز مشکست
&&
ورچه سلب زمین ز دیباست
\\
هر چند شکوفه بر درختان
&&
چون دو لب دوست پر ثریاست
\\
هر چند میان کوه لاله
&&
چون دیده میان روی حوراست
\\
چون دولت عاشقی در آمد
&&
اینها همه از میانه برخاست
\\
هرگز نشود به وصل مغرور
&&
هر دیده که در فراق بیناست
\\
اکنون که ز باغ زاغ کم شد
&&
بلبل ز گل آشیانه آراست
\\
بر هر سر شاخ عندلیبی‌ست
&&
زین شکر که زاغ کم شد و کاست
\\
فریاد همی کند که باری
&&
امروز زمانه نوبت ماست
\\
\end{longtable}
\end{center}
