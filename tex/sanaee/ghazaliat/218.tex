\begin{center}
\section*{غزل شماره ۲۱۸: در زلف تو دادند نگارا خبر دل}
\label{sec:218}
\addcontentsline{toc}{section}{\nameref{sec:218}}
\begin{longtable}{l p{0.5cm} r}
در زلف تو دادند نگارا خبر دل
&&
معذورم اگر آمده‌ام بر اثر دل
\\
یا دل بر من باز فرست ای بت مه رو
&&
یا راه مرا باز نما تو به بر دل
\\
نی نی که اگر نیست ترا هیچ سر ما
&&
ما بی تو نداریم دل خویش و سر دل
\\
چندین سر اندیشه و تیمار که دارد
&&
تا گه جگر یار خورد گه جگر دل
\\
بی عشق تو دل را خطری نیست بر ما
&&
هر چند که صعب ست نگارا خطر دل
\\
تا دل کم عشق تو در بست به شادی
&&
بستیم به جان بر غم عشقت کمر دل
\\
چاک زد جان پدر دست صبا دامن گل
&&
خیز تا هر دو خرامیم به پیرامن گل
\\
تیره شد ابر چو زلفین تو بر چهرهٔ رخ
&&
تا بیاراست چو روی تو رخ روشن گل
\\
همه شب فاخته تا روز همی گرید زار
&&
ز غم گل چو من از عشق تو ای خرمن گل
\\
زان که گل بندهٔ آن روی خوش خرم تست
&&
در هوای رخ تو دست من و دامن گل
\\
گل برون کرد سر از شاخ به دل بردن خلق
&&
تا بسی جلوه گری کرد هوا بر تن گل
\\
تا گل عارض تو دید فرو ریخت ز شرم
&&
با گل عارض تو راست نیاید فن گل
\\
\end{longtable}
\end{center}
