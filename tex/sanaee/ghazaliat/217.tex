\begin{center}
\section*{غزل شماره ۲۱۷: ای بلبل وصل تو طربناک}
\label{sec:217}
\addcontentsline{toc}{section}{\nameref{sec:217}}
\begin{longtable}{l p{0.5cm} r}
ای بلبل وصل تو طربناک
&&
وی غمزت زهر و خنده تریاک
\\
ای جان دو صدهزار عاشق
&&
آویخته از دوال فتراک
\\
افلاک توانگر از ستاره
&&
در جنب ستانهٔ تو مفلاک
\\
در بند تو سر زنان گردون
&&
با طوق تو گردنان سرناک
\\
از بهر شمارش ستاره
&&
پیشانی ماه تختهٔ خاک
\\
از زلف تو صد هزار منزل
&&
تا روی تو و همه خطرناک
\\
ای نقش نگین تو «لعمرک»
&&
وی خلعت خلقت تو «لولاک»
\\
بر بوی خط تو روح پاکان
&&
از عقل بشسته تخته‌ها پاک
\\
با نقش تو گفته نقش بندت
&&
«لولاک لما خلقت الا فلاک»
\\
از رشک تو آفتاب چون صبح
&&
هر روز قبای نو کند چاک
\\
با تابش تو به ماه نیسان
&&
گشته می صرف غوره بر تاک
\\
از گرد رکاب تو سنایی
&&
مانندهٔ مرکب تو چالاک
\\
با کیش نه از کس و گزافست
&&
آن تو و آنگه از کسش باک؟
\\
\end{longtable}
\end{center}
