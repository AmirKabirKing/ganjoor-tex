\begin{center}
\section*{غزل شماره ۴۲۱: الا ای لعبت ساقی ز می پر کن مرا جامی}
\label{sec:421}
\addcontentsline{toc}{section}{\nameref{sec:421}}
\begin{longtable}{l p{0.5cm} r}
الا ای لعبت ساقی ز می پر کن مرا جامی
&&
که پیدا نیست کارم را درین گیتی سرانجامی
\\
کنون چون توبه بشکستم به خلوت با تو بنشستم
&&
ز می باید که در دستم نهی هر ساعتی جامی
\\
نباید خورد چندین غم بباید زیستن خرم
&&
که از ما اندرین عالم نخواهد ماند جز نامی
\\
همی خور بادهٔ صافی ز غم آن به که کم لافی
&&
که هرگز عالم جافی نگیرد با کس آرامی
\\
منه بر خط گردون سر ز عمر خویش بر خور
&&
که عمرت را ازین خوشتر نخواهد بود ایامی
\\
چرا باشی چو غمناکی مدار از مفلسی باکی
&&
که ناگاهان شوی خاکی ندیده از جهان کامی
\\
مترس از کار نابوده مخور اندوه بیهوده
&&
دل از غم دار آسوده به کام خود بزن گامی
\\
ترا دهرست بدخواهی نشسته در کمین‌گاهی
&&
ز غداری به هر راهی بگسترده ترا دامی
\\
\end{longtable}
\end{center}
