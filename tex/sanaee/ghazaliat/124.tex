\begin{center}
\section*{غزل شماره ۱۲۴: گر شبی عشق تو بر تخت دلم شاهی کند}
\label{sec:124}
\addcontentsline{toc}{section}{\nameref{sec:124}}
\begin{longtable}{l p{0.5cm} r}
گر شبی عشق تو بر تخت دلم شاهی کند
&&
صدهزاران ماه آن شب خدمت ماهی کند
\\
باد لطفت گر به دارالملک انسان بروزد
&&
هر یکی را بر مثال یوسف چاهی کند
\\
من چه سگ باشم که در عشق تو خوش یک دم زنم
&&
آدم و ابلیس یک جا چون به همراهی کند
\\
هر که از تصدیق دل در خویشتن کافر شود
&&
بی خلافی صورت ایمانش دلخواهی کند
\\
بی خود ار در کفر و دین آید کسی محبوب نیست
&&
مختصر آنست کار از روی آگاهی کند
\\
خفتهٔ بیدار بنگر عاقل دیوانه بین
&&
کو ز روی معرفت بی وصل الاهی کند
\\
تا درین داری به جز بر عشق دارایی مکن
&&
عاشق آن کار خود از آه سحرگاهی کند
\\
ساحری دان مر سنایی را که او در کوی عقل
&&
عشقبازی با خیال ترک خرگاهی کند
\\
\end{longtable}
\end{center}
