\begin{center}
\section*{غزل شماره ۲۴۵: می ده پسرا که در خمارم}
\label{sec:245}
\addcontentsline{toc}{section}{\nameref{sec:245}}
\begin{longtable}{l p{0.5cm} r}
می ده پسرا که در خمارم
&&
آزردهٔ جور روزگارم
\\
تا من بزیم پیاله بادا
&&
بر دست زیار یادگارم
\\
می رنگ کند به جامم اندر
&&
بس خون که ز دیده می‌ببارم
\\
از حلقه و تاب و بند زلفت
&&
هم مومن و بستهٔ زنارم
\\
ای ماه در آتشم چه داری
&&
چون با تو ز نار نیست عارم
\\
تا مانده‌ام از تو برکناری
&&
جویست ز دیده بر کنارم
\\
خواهم که شکایت تو گویم
&&
از بیم دو زلف تو نیارم
\\
گر ماه رخان تو برآید
&&
از من ببرد دل و قرارم
\\
امروز که در کفم نبیدست
&&
اندوه جهان بتا چه دارم
\\
مولای پیالهٔ بزرگم
&&
فرمانبر دور بی‌شمارم
\\
در مغکده‌ها بود مقامم
&&
در مصطبه‌ها بود قرارم
\\
از شحنهٔ شهر نیست بیمم
&&
در خانهٔ هجر نیست کارم
\\
هر چند ز بخت بد به دردم
&&
هر چند به چشم خلق خوارم
\\
با رود و سرود و بادهٔ ناب
&&
ایام جهان همی گذارم
\\
\end{longtable}
\end{center}
