\begin{center}
\section*{غزل شماره ۲۴۶: چو آمد روی بر رویم که باشم من که من باشم}
\label{sec:246}
\addcontentsline{toc}{section}{\nameref{sec:246}}
\begin{longtable}{l p{0.5cm} r}
چو آمد روی بر رویم که باشم من که من باشم
&&
که آنگه خوش بود با من که من بی‌خویشتن باشم
\\
من آنگه خود کسی باشم که در میدان حکم او
&&
نه دل باشم نه جان باشم نه سر باشم نه تن باشم
\\
چه جای سرکشی باشد ز حکم او که در رویش
&&
چو شمع آنگاه خوش باشم که در گردن زدن باشم
\\
چو او با من سخن گوید چو یوسف وقت لا باشد
&&
چو من با او سخن گویم چو موسی گاه لن باشم
\\
سخن پیدا و پنهان‌ست و او آن دوستر دارد
&&
که چون با من سخن گوید من آنجا چون وثن باشم
\\
چو بیخود بر برش باشم ز وصف اندر کنف باشم
&&
چو با خود بر درش باشم ز هجر اندر کفن باشم
\\
مرا در عالم عشقش مپرس از شیب و از بالا
&&
مهم تا در فلک باشم گلم تا در چمن باشم
\\
مرا گر پایه‌ای بینی بدان کان پایه او باشد
&&
بر او گر سایه‌ای بینی بدان کان سایه من باشم
\\
سنایی خوانم آن ساعت که فانی گشتم از سنت
&&
سنایی آنگهی باشم که در بند سنن باشم
\\
\end{longtable}
\end{center}
