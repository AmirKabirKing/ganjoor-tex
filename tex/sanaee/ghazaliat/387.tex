\begin{center}
\section*{غزل شماره ۳۸۷: جانا نگویی آخر ما را که تو کجایی}
\label{sec:387}
\addcontentsline{toc}{section}{\nameref{sec:387}}
\begin{longtable}{l p{0.5cm} r}
جانا نگویی آخر ما را که تو کجایی
&&
کز تو ببرد آتش عشق تو آب مایی
\\
ما را ز عشق کردی چو آسیای گردان
&&
خود همچو دانه گشتی در ناو آسیایی
\\
گه در زمین دلها پنهان شوی چو پروین
&&
گاه از سپهر جانها چون ماه نو برآیی
\\
از بهر لطف مستان وز قهر خود پرستان
&&
چون برق میگریزی چون باد می‌ربایی
\\
بهر سماع دنیا بر شاخهای طوبا
&&
چون عندلیب بیدل همواره می‌سرایی
\\
خورشیدوار کردی چون ذره‌های عقلی
&&
دلهای عاشقان را در پردهٔ هوایی
\\
یاقوت بار کردی عشاق لاله رخ را
&&
از نوک کلک نرگس بر لوح کهربایی
\\
ای یافته جمالت در جلوهٔ نخستین
&&
منشور حسن و تمکین از خلعت خدایی
\\
روح‌القدس ندارددر خوبی و لطافت
&&
با خاک کف پایت یکذره آشنایی
\\
بردار پرده از رخ تا حضرت الاهی
&&
گردد ز مهر چهرت پر نور و روشنایی
\\
گویی مرا بجویی آخر کجا بجویم
&&
در گرد گوی ارضی یا حلقهٔ سمایی
\\
بگشای بند مرجان تا همچو طبع بی‌جان
&&
بندازد از جمالت جان تاج کبریایی
\\
ای تافته کمالت از چار سوی ارکان
&&
پنهان ز هر دو عالم در صدر پارسایی
\\
بر خیره چند جویم آنرا که او ندارد
&&
منزل به کوی رندی یا راه پارسایی
\\
ما ز انتظار مردیم از عشق تو ولیکن
&&
در حجرهٔ غریبان تو خود درون نیایی
\\
گیرم که بار ندهی ما را درون پرده
&&
کم زان مکن که بیرون رویی به ما نمایی
\\
بی روی تو نگارا چشم امید ما را
&&
باید ز نقش نامه نام تو توتیایی
\\
نادیده کس ولیکن از سنگ و چوب کویت
&&
بدهند اگر بپرسی بر حسن تو گوایی
\\
نی نی اگر ندیدی رویت چگونه گفتی
&&
در نظمهای عالی وصف ترا سنایی
\\
\end{longtable}
\end{center}
