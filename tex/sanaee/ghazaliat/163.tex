\begin{center}
\section*{غزل شماره ۱۶۳: هر کرا در دل بود بازار یار}
\label{sec:163}
\addcontentsline{toc}{section}{\nameref{sec:163}}
\begin{longtable}{l p{0.5cm} r}
هر کرا در دل بود بازار یار
&&
عمر و جان و دل کند در کار یار
\\
خاصه آن بی دل که چون من یک زمان
&&
بر زمین نشکیبد از دیدار یار
\\
کبک را بین تا چگونه شد خجل
&&
زان کرشمه کردن و رفتار یار
\\
بنگر اندر گل که رشوت چون دهد
&&
خون شود لعل از پی رخسار یار
\\
در جهان فردوس اعلا دارد آنک
&&
یک نفس بودست در پندار یار
\\
در همه عالم ندیدم لذتی
&&
خوشتر و شیرین‌تر از گفتار یار
\\
همچو سنگ آید مرا یاقوت سرخ
&&
بی لب یاقوت شکر بار یار
\\
باد نوشین دوش گفتی ناگهان
&&
چین زلف آشفت بر گلنار یار
\\
زان قبل امروز مشک آلود گشت
&&
خانه و بام و در و دیوار یار
\\
رشک لعل و لولو اندر کوه و بحر
&&
زان عقیق و لولو شهوار یار
\\
شد دلم مسکین من در غم نژند
&&
من ندانم پیش ازین هنجار یار
\\
دست بر سر ماند چون کژدم دلم
&&
زان دو زلفین سیه چون مار یار
\\
هوش و عقلم برده‌اند از دل تمام
&&
آن دو نرگس بر رخ چون نار یار
\\
مر سنایی را فتاد این نادره
&&
چون معزی گفت از اخبار یار
\\
آنچه من می‌بینم از آزار یار
&&
گر بگویم بشکنم بازار یار
\\
\end{longtable}
\end{center}
