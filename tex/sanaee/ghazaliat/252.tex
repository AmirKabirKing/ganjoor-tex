\begin{center}
\section*{غزل شماره ۲۵۲: در راه عشق ای عاشقان خواهی شفا خواهی الم}
\label{sec:252}
\addcontentsline{toc}{section}{\nameref{sec:252}}
\begin{longtable}{l p{0.5cm} r}
در راه عشق ای عاشقان خواهی شفا خواهی الم
&&
کاندر طریق عاشقی یک رنگ بینی بیش و کم
\\
روزی بیاید در میان تا عشق را بندی میان
&&
عیسی بباید ترجمان تا زنده گرداند به دم
\\
چون دیده کوته‌بین بود هر نقش حورالعین بود
&&
چون حاصل عشق این بود خواهی شفا خواهی الم
\\
یک جرعه زان می نوش کن سری ز حرفی گوش کن
&&
جان را ازان مدهوش کن کم کن حدیث بیش و کم
\\
دردت بود درمان شمر دشوارها آسان شمر
&&
در عاشقی یکسان شمر شیر فلک شیر علم
\\
از خویشتن آزاد زی از هر ملالی شاد زی
&&
هر جا که باشی راد زی چون یافتی از عشق شم
\\
رو کن شراب رنگ را وز سر بنه نیرنگ را
&&
بس کن تو نام و ننگ را بر فرق فرقد زن قدم
\\
بر سوز دل دمساز شو اول قدم جان باز شو
&&
زی سر معنی باز شو شکل حروف انگار کم
\\
بر زن زمانی کبر را بر طاق نه کبر و ریا
&&
خواهی وفا خواهی جفا چون دوست باشد محتشم
\\
عاشق که جام می کشد بر یاد روی وی کشد
&&
جز رخش رستم کی کشد رنج رکاب روستم
\\
چون از پی دلبر بود شاید که جان چاکر بود
&&
چون زهره خنیاگر بود از حور باید زیر و بم
\\
تا کی ازین سالوس و زه از بند چار ارکان بجه
&&
سر سوی کل خویش نه تا نور بینی بی ظلم
\\
از کل عالم شو بری بگذر ز چرخ چنبری
&&
تا هیچ چیزی نشمری تاج قباد و تخت جم
\\
گر بایدت حرفی ازین تا گرددت عین‌الیقین
&&
شو مدحت خورشید دین بر دفتر جان کن رقم
\\
\end{longtable}
\end{center}
