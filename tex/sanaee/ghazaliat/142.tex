\begin{center}
\section*{غزل شماره ۱۴۲: اقتدا بر عاشقان کن گر دلیلت هست درد}
\label{sec:142}
\addcontentsline{toc}{section}{\nameref{sec:142}}
\begin{longtable}{l p{0.5cm} r}
اقتدا بر عاشقان کن گر دلیلت هست درد
&&
ور نداری درد گرد مذهب رندان مگرد
\\
ناشده بی عقل و جان و دل درین ره کی شوی
&&
محرم درگاه عشقی با بت و زنار گرد
\\
هر که شد مشتاق او یکبارگی آواره شد
&&
هر که شد جویای او در جان و دل منزل نکرد
\\
مرد باید پاکباز و درد باید مرد سوز
&&
کان نگارین روی عاشق می نخواهد کرد مرد
\\
خاکپای خادمان درگه معشوق شو
&&
بوسه را بر خاک ده چون عاشقان از بهر درد
\\
هر کرا سودای وصل آن صنم در سر فتاد
&&
اندرین ره سر هم آخر در سر این کار کرد
\\
ای سنایی رنگ و بویی اندرین ره بیش نیست
&&
اندرین ره رو همی چون رنگ و بو خواهند کرد
\\
\end{longtable}
\end{center}
