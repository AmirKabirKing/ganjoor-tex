\begin{center}
\section*{غزل شماره ۱۴۷: هر که در عاشقی تمام بود}
\label{sec:147}
\addcontentsline{toc}{section}{\nameref{sec:147}}
\begin{longtable}{l p{0.5cm} r}
هر که در عاشقی تمام بود
&&
پخته خوانش اگر چه خام بود
\\
آنکه او شاد گردد از غم عشق
&&
خاص دانش اگر چه عام بود
\\
چه خبر دارد از حلاوت عشق
&&
هر که در بند ننگ و نام بود
\\
دوری از عشق اگر همی خواهی
&&
کز سلامت ترا سلام بود
\\
در ره عاشقی طمع داری
&&
که ترا کار بر نظام بود
\\
این تمنا و این هوس که تراست
&&
عشقبازی ترا حرام بود
\\
عشق جویی و عافیت طلبی
&&
عشق یا عافیت کدام بود
\\
بندهٔ عشق باش تا باشی
&&
تا سنایی ترا غلام بود
\\
\end{longtable}
\end{center}
