\begin{center}
\section*{غزل شماره ۱۳۵: با او دلم به مهر و مودت یگانه بود}
\label{sec:135}
\addcontentsline{toc}{section}{\nameref{sec:135}}
\begin{longtable}{l p{0.5cm} r}
با او دلم به مهر و مودت یگانه بود
&&
سیمرغ عشق را دل من آشیانه بود
\\
بر درگهم ز جمع فرشته سپاه بود
&&
عرش مجید جاه مرا آستانه بود
\\
در راه من نهاد نهان دام مکر خویش
&&
آدم میان حلقهٔ آن دام دانه بود
\\
می‌خواست تا نشانهٔ لعنت کند مرا
&&
کرد آنچه خواست آدم خاکی بهانه بود
\\
بودم معلم ملکوت اندر آسمان
&&
امید من به خلد برین جاودانه بود
\\
هفصد هزار سال به طاعت ببوده‌ام
&&
وز طاعتم هزار هزاران خزانه بود
\\
در لوح خوانده‌ام که یکی لعنتی شود
&&
بودم گمان به هر کس و بر خود گمان نبود
\\
آدم ز خاک بود من از نور پاک او
&&
گفتم یگانه من بوم و او یگانه بود
\\
گفتند مالکان که نکردی تو سجده‌ای
&&
چون کردمی که با منش این در میانه بود
\\
جانا بیا و تکیه به طاعات خود مکن
&&
کاین بیت بهر بینش اهل زمانه بود
\\
دانستم عاقبت که به ما از قضا رسید
&&
صد چشمه آن زمان زد و چشمم روانه بود
\\
ای عاقلان عشق مرا هم گناه نیست
&&
ره یافتن به جانبشان بی رضا نبود
\\
\end{longtable}
\end{center}
