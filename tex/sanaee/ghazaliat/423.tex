\begin{center}
\section*{غزل شماره ۴۲۳: تا به گرد روی آن شیرین پسر گردم همی}
\label{sec:423}
\addcontentsline{toc}{section}{\nameref{sec:423}}
\begin{longtable}{l p{0.5cm} r}
تا به گرد روی آن شیرین پسر گردم همی
&&
چون قلم گرد سر کویش به سر گردم همی
\\
بهر آن بو تا که خورشیدی به دست آرم چنو
&&
من به گرد کوی خیره خیره برگردم همی
\\
پس چو میدان فلک را نیست خورشیدی چو تو
&&
چون فلک هر روز گرد خاک در گردم همی
\\
آبروی عاشقان در خاکپایش تعبیه‌ست
&&
خاکپایش را ز بهر آب سر گردم همی
\\
از پی گرد سم شبدیز او وقت نثار
&&
گه ز دیده سیم و گه از روی زر گردم همی
\\
روی تا داریم به کویش در بهشتم در بهشت
&&
چون ز کویش بازگردم در سقر گردم همی
\\
که گهی از شرم‌تر گردم ز خشم آوردنش
&&
بلعجب مردی منم کز خشم تر گردم همی
\\
گر هنوز از دولبش جویم غذا نشگفت از آنک
&&
در هوای عشقش اکنون کفچه بر گردم همی
\\
تا چو شیر اورخ به خون دارد من از بهر غذاش
&&
همچو ناف آهو از خون بارور گردم همی
\\
روی زورد من ز عکس روی چون خورشید اوست
&&
زان چو سایه گرد آن دیوار و در گردم همی
\\
گر چه هستم با دل آهوی ماده وقت ضعف
&&
چون ز عشقش یادم آید شیر نر گردم همی
\\
هر چه پیشم پوستین درد همی نادر تر آنک
&&
من سلیم از پوستینش سغبه‌تر گردم همی
\\
با سنایی و سنایی گشتم اندر عشق او
&&
باز در وصف دهانش پر درر گردم همی
\\
\end{longtable}
\end{center}
