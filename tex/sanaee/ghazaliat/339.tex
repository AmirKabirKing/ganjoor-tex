\begin{center}
\section*{غزل شماره ۳۳۹: خواجه سلام علیک آن لب چون نوش بین}
\label{sec:339}
\addcontentsline{toc}{section}{\nameref{sec:339}}
\begin{longtable}{l p{0.5cm} r}
خواجه سلام علیک آن لب چون نوش بین
&&
توشهٔ جانها در آن گوشهٔ شبپوش بین
\\
پیش رکابت جمال کیست گرفته عنان
&&
چرخ جفا کیش بین لعل وفا کوش بین
\\
گردش ایام دوش تعبیه‌ای ساختست
&&
سوختهٔ عشق باش ساختهٔ دوش بین
\\
برگذر و کوی او غرقه چو من صد هزار
&&
عاشق جانباز بین مرد کفن‌پوش بین
\\
گوش مینبار و آن نغمه و دستان شنو
&&
دیده برانداز و آن خط و بناگوش بین
\\
در بر تنگ شکر مار جهانسوز بین
&&
بر سر سنگ سیاه صبر جگرجوش بین
\\
گر چه دل ریش ما بر سر سودای اوست
&&
بر دل او یاد ما جمله فراموش بین
\\
صف زده در پیش او خلق خروشان شده
&&
تن زده آن ماه را فارغ و خاموش بین
\\
بهرهٔ ما دیده‌ای ناله و فریاد ازو
&&
بهرهٔ هر ناکسی بوسه و آغوش بین
\\
ساقی فردوس را از پی بازار او
&&
بر در میخانه‌ها بلبله بر دوش بین
\\
زلفش یکسو فگن و آنگه در زیر زلف
&&
جان سنایی ز عشق خسته و مدهوش بین
\\
\end{longtable}
\end{center}
