\begin{center}
\section*{غزل شماره ۸۳: ساقیا می ده که جز می عشق را پدرام نیست}
\label{sec:083}
\addcontentsline{toc}{section}{\nameref{sec:083}}
\begin{longtable}{l p{0.5cm} r}
ساقیا می ده که جز می عشق را پدرام نیست
&&
وین دلم را طاقت اندیشهٔ ایام نیست
\\
پختهٔ عشقم شراب خام خواهی زان کجا
&&
سازگار پخته جانا جز شراب خام نیست
\\
با فلک آسایش و آرام چون باشد ترا
&&
چون فلک را در نهاد آسایش و آرام نیست
\\
عشق در ظاهر حرامست از پی نامحرمان
&&
زان که هر بیگانه‌ای شایستهٔ این نام نیست
\\
خوردن می نهی شد زان نیز در ایام ما
&&
کاندرین ایام هر دستی سزای جام نیست
\\
تا نیفتد بر امید عشق در دام هوا
&&
کاین ره خاصست اندر وی مجال عام نیست
\\
هست خاص و عام نی نزدیک هر فرزانه‌ای
&&
دانهٔ دام هوا جز جام جان انجام نیست
\\
جاهلان را در چراگه دام هست و دانه نی
&&
عاشقان را باز در ره دانه هست و دام نیست
\\
\end{longtable}
\end{center}
