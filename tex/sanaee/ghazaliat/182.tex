\begin{center}
\section*{غزل شماره ۱۸۲: با تابش زلف و رخت ای ماه دلفروز}
\label{sec:182}
\addcontentsline{toc}{section}{\nameref{sec:182}}
\begin{longtable}{l p{0.5cm} r}
با تابش زلف و رخت ای ماه دلفروز
&&
از شام تو قدر آید و از صبح تو نوروز
\\
از جنبش موی تو برآید دو گل از مشک
&&
وز تابش روی تو برآید دو شب از روز
\\
بر گرد یکی گرد دل ما و در آن دل
&&
گر جز غم خود یابی آتش زن و بفروز
\\
هر چند همه دفتر عشاق بخواندیم
&&
با این همه در عشق تو هستیم نو آموز
\\
در مملکت عاشقی از پسته و بادام
&&
بوس تو جهانگیر شد و غمزه جهانسوز
\\
تا دیدهٔ ما جز به تو آرام نگیرد
&&
از بوسه‌ش مهری کن و ز غمزه‌ش بردوز
\\
با هجر تو هر شب ز پی وصل تو گویم
&&
یارب تو شب عاشق و معشوق مکن روز
\\
\end{longtable}
\end{center}
