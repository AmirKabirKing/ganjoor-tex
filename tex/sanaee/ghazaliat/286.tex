\begin{center}
\section*{غزل شماره ۲۸۶: ما قد ترا بنده‌تر از سرو روانیم}
\label{sec:286}
\addcontentsline{toc}{section}{\nameref{sec:286}}
\begin{longtable}{l p{0.5cm} r}
ما قد ترا بنده‌تر از سرو روانیم
&&
ما خد ترا سغبه‌تر از عقل و روانیم
\\
بی روی تو لب خشک‌تر از پیکر تیریم
&&
با موی تو دل تیره‌تر از نقش کمانیم
\\
بیرون ز رخ و زلف تو ما قبله نداریم
&&
بیش از لقب و نام تو توحید نخوانیم
\\
در ره روش عقل تو ما کهتر عقلیم
&&
وز پرورش لفظ تو ما مهتر جانیم
\\
از تقویت جزع تو خردیم و بزرگیم
&&
وز تربیت عقل تو پیریم و جوانیم
\\
در کوی امید تو و اندر ره ایمان
&&
از نیستی و هستی بر بسته میانیم
\\
یک بار برانداز نقاب از رخ رنگین
&&
تا دل به تو بخشیم و خرد بر تو فشانیم
\\
وز نیز درین پرده جمال تو ببینیم
&&
شاید که بر امید تو این مایه توانیم
\\
گر ز آتش عشق تو چو شمع از ره تحقیق
&&
سوزیم همی خوش خوش تا هیچ نمانیم
\\
تا از رخ چون روز تو بی واسطهٔ کسب
&&
چون ماه ز خورشید فلک مایه ستانیم
\\
ما را غرض از خدمت تو جز لب تو نیست
&&
نه در پی جانیم نه در بند جهانیم
\\
شاید که شب و روز همه مدح تو گوییم
&&
در نامهٔ اقبال همه نام تو خوانیم
\\
زان باده که خواجه از کف اقبال تو خوردست
&&
درده تو سنایی را چون کشتهٔ آنیم
\\
فرخنده حکیمی که در اقلیم سنایی
&&
بگذشت ز اندازهٔ خوبی و ندانیم
\\
\end{longtable}
\end{center}
