\begin{center}
\section*{غزل شماره ۲۹۹: جام را نام ای سنایی گنج کن}
\label{sec:299}
\addcontentsline{toc}{section}{\nameref{sec:299}}
\begin{longtable}{l p{0.5cm} r}
جام را نام ای سنایی گنج کن
&&
راح در ده روح را بی رنج کن
\\
این دل و جان طبیعت سنج را
&&
یک زمان از می طریقت سنج کن
\\
تاج جان پاک را در راه دل
&&
مفرش جانان جان آهنج کن
\\
کدخدای روح را در ملک عشق
&&
بی تصرف چون شه شطرنج کن
\\
عقل دین‌دار سلامت جوی را
&&
سنگ شنگولی عشق الفنج کن
\\
یا همه رخ گرد چون گلنار باش
&&
یا همه دل باش و چون نارنج باش
\\
با عمارت چند سازی همچو رنج
&&
با خرابی ساز و همچون گنج باش
\\
خاک و باد و آب و آتش دشمنند
&&
برگذر زین چار و نوبت پنج کن
\\
\end{longtable}
\end{center}
