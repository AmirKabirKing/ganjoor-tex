\begin{center}
\section*{غزل شماره ۱۲۱: دل به تحفه هر که او در منزل جانان کشد}
\label{sec:121}
\addcontentsline{toc}{section}{\nameref{sec:121}}
\begin{longtable}{l p{0.5cm} r}
دل به تحفه هر که او در منزل جانان کشد
&&
از وجود نیستی باید که خط بر جان کشد
\\
در نوردد مفرش آزادگی از روی عقل
&&
رخت بدبختی ز دل از خانهٔ احزان کشد
\\
گر چه دشوارست کار عاشقی از بهر دوست
&&
از محبت بر دل و جان رخت عشق آسان کشد
\\
رهروی باید که اندر راه ایمان پی نهد
&&
تا ز دل پیمانهٔ غم بر سر پیمان کشد
\\
دین و پیمان و امانت در ره ایمان یکیست
&&
مرد کو تا فضل دین اندر ره ایمان کشد
\\
لشکر لا حول را بند قطیعت بگسلد
&&
وز تفاوت بر شعاع شرع شادروان کشد
\\
خلق پیغمبر کجا تا از بزرگان عرب
&&
جور و رنج ناسزایان از پی یزدان کشد
\\
صادقی باید که چون بوبکر در صدق و صواب
&&
زخم مار و بیم دشمن از بن دندان کشد
\\
یا نه چون عمر که در اسلام بعد از مصطفا
&&
از عرب لشکر ز جیحون سوی ترکستان کشد
\\
پارسایی کو که در محراب و مصحف بی گناه
&&
تا ز غوغا سوزش شمشیر چون عثمان کشد
\\
حیدر کرار کو کاندر مصاف از بهر دین
&&
در صف صفین ستم از لشکر مروان کشد
\\
\end{longtable}
\end{center}
