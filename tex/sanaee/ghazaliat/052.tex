\begin{center}
\section*{غزل شماره ۵۲: کار تو پیوسته آزارست گویی نیست هست}
\label{sec:052}
\addcontentsline{toc}{section}{\nameref{sec:052}}
\begin{longtable}{l p{0.5cm} r}
کار تو پیوسته آزارست گویی نیست هست
&&
زین سبب کار دلم زارست گویی نیست هست
\\
خصم تو بازار من بشکست و با خصم ای صنم
&&
مر ترا پیوسته بازارست گویی نیست هست
\\
تا به خروارست شکر لعل نوشین ترا
&&
در دلم عشقت به خروارست گویی نیست هست
\\
طرهٔ طرار تو دل دزدد از مردم همی
&&
شد یقین کان طره طرارست گویی نیست هست
\\
ماهرویا تا تو کردی رایت صحبت نگون
&&
رایت صبرم نگونسارست گویی نیست هست
\\
بوسه‌ای را زان لب چون لعل نوشینت به جان
&&
چاکر مسکین خریدارست گویی نیست هست
\\
نرگس خونخوار تو پیوسته خون ریزد همی
&&
نرگست بس شوخ و خونخوارست گویی نیست هست
\\
\end{longtable}
\end{center}
