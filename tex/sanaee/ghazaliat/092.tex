\begin{center}
\section*{غزل شماره ۹۲: تا کی کنم از طرهٔ تو فریاد}
\label{sec:092}
\addcontentsline{toc}{section}{\nameref{sec:092}}
\begin{longtable}{l p{0.5cm} r}
تا کی کنم از طرهٔ تو فریاد
&&
تا کی کشم از غمزهٔ تو بیداد
\\
یک شهر زن و مرد همی باز ندانند
&&
فریاد من از خنده و بیداد تو از داد
\\
آن روز که زلفین نگون تو بدیدند
&&
گشتند ترا بنده چو من بنده و آزاد
\\
هشیار نشد هر که ز گفتار تو شد مست
&&
غمناک نشد هر که ز دیدار تو شد شاد
\\
من با رخ چون لاله و با عارض چون مشک
&&
با قامت چون تیر ز وصل تو کنم یاد
\\
تو زر کنی از لاله و کافور کنی مشک
&&
چوگان کنی از تیز زهی جادوی استاد
\\
ویران کنی آن دل که درو سازی منزل
&&
هرگز نگذاری که بود منزلت آباد
\\
ای منزل تو گشته ز آشوب تو ویران
&&
آن شهر کزو خاستی آباد همی باد
\\
جیحون شده چشم من از آن زلف سمن سا
&&
بر باد شده زلف تو از قامت شمشاد
\\
مشهور جهان گشته سنایی ز غم تو
&&
از روی چو خورشید تو ای طرفهٔ بغداد
\\
تو مایهٔ خوبی شدی ای مایهٔ خوبان
&&
افگنده درین خسته دلم عشق تو بنیاد
\\
صد رحمت و صد شادی بر جان تو ای بت
&&
مادر که ترا زاد بر او نیز دعا باد
\\
\end{longtable}
\end{center}
