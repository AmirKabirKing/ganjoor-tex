\begin{center}
\section*{غزل شماره ۲۷۵: ما کلاه خواجگی اکنون ز سر بنهاده‌ایم}
\label{sec:275}
\addcontentsline{toc}{section}{\nameref{sec:275}}
\begin{longtable}{l p{0.5cm} r}
ما کلاه خواجگی اکنون ز سر بنهاده‌ایم
&&
تا که در بند کله‌دوزی اسیر افتاده‌ایم
\\
صد سر ار زد هر کلاهی کو همی دوزد ولیک
&&
ما بهای هر کله اکنون سری بنهاده‌ایم
\\
او کلاه عاشقان اکنون همی دوزد چو شمع
&&
ما از آن چون شمع در پیشش به جان استاده‌ایم
\\
بندهٔ او از سر چشمیم همچون سوزنش
&&
گر چه همچون سرو و سوسن نزد عقل آزاده‌ایم
\\
سینه چشم سوزن و تن تار ابریشم شدست
&&
تا غلام آن بهشتی روی حورا زاده‌ایم
\\
کار او چون بیشتر با سوزن و ابریشمست
&&
لاجرم ما از تن و دل هر دو را آماده‌ایم
\\
از لب خویش و لب او در فراق و در وصال
&&
چون چراغ و باغ و با هم با باد و هم با باده‌ایم
\\
برنتابد بار نازش دل همی از بهر آنک
&&
دل همی گوید گر او سادست ما هم ساده‌ایم
\\
لعل پاش و در فشانیم از دو دریا و دو کان
&&
تا اسیر آن دو لعل و آن دو تا بیجاده‌ایم
\\
ما ز خصمانش کی اندیشیم کاندر راه او
&&
خوان جان بنهاده و بانگ صلا در داده‌ایم
\\
تا سنایی وار دربستیم دل در مهر او
&&
ما دو چشم اندر سنایی جز به کین نگشاده‌ایم
\\
\end{longtable}
\end{center}
