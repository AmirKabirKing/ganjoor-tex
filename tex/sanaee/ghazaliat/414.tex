\begin{center}
\section*{غزل شماره ۴۱۴: عشق و شراب و یار و خرابات و کافری}
\label{sec:414}
\addcontentsline{toc}{section}{\nameref{sec:414}}
\begin{longtable}{l p{0.5cm} r}
عشق و شراب و یار و خرابات و کافری
&&
هر کس که یافت شد ز همه اندهان بری
\\
از راه کج به سوی خرابات راه یافت
&&
کفرش همه هدی شد و توحید کافری
\\
بگذاشت آنچه بود هم از هجر و هم ز وصل
&&
برخاست از تصرف و از راه داوری
\\
بیزار شد ز هر چه به جز عشق و باده بود
&&
بست او میان به پیش یکی بت به چاکری
\\
برخیز ای سنایی باده بخواه و چنگ
&&
اینست دین ما و طریق قلندری
\\
مرد آن بود که داند هر جای رای خویش
&&
مردان به کار عشق نباشند سر سری
\\
\end{longtable}
\end{center}
