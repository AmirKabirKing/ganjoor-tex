\begin{center}
\section*{غزل شماره ۱۶۱: زینهار ای یار گلرخ زینهار}
\label{sec:161}
\addcontentsline{toc}{section}{\nameref{sec:161}}
\begin{longtable}{l p{0.5cm} r}
زینهار ای یار گلرخ زینهار
&&
بی گنه بر من مکن تیزی چو خار
\\
لالهٔ خود رویم از فرقت مکن
&&
حجرهٔ من ز اشک خون چون لاله‌زار
\\
چون شکوفه گرد بدعهدی مگرد
&&
تا مگر باقی بمانی چون چنار
\\
چون بنفشه خفته‌ام در خدمتت
&&
پس مدارم چون بنفشه سوگوار
\\
زان که جانها را فراقت چون سمن
&&
یک دو هفته بیش ندهد زینهار
\\
باش با من تازه چون شاه اسپرم
&&
تا نگردم همچو خیری دلفگار
\\
از سر لطف و ظریفی خوش بزی
&&
همچو سوسن تازه‌ای آزاده‌وار
\\
همچو سیسنبر بپژمردم ز غم
&&
یک ره از ابر وفا بر من ببار
\\
چون نخوردم بادهٔ وصلت چو گل
&&
همچو نرگس پس مدارم در خمار
\\
ای همیشه تازه و تر همچو سرو
&&
اشکم از هجران مکن چون گل انار
\\
حوضها کن گلبنان را از عرق
&&
تا چو نیلوفر در او گیرم قرار
\\
زان که از بهر سنایی هر زمان
&&
بر فراز سرو و طرف جویبار
\\
بلبل و قمری همی گویند خوش
&&
زینهار ای یار گلرخ زینهار
\\
\end{longtable}
\end{center}
