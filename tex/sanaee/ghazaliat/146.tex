\begin{center}
\section*{غزل شماره ۱۴۶: دوش ما را در خراباتی شب معراج بود}
\label{sec:146}
\addcontentsline{toc}{section}{\nameref{sec:146}}
\begin{longtable}{l p{0.5cm} r}
دوش ما را در خراباتی شب معراج بود
&&
آنکه مستغنی بد از ما هم به ما محتاج بود
\\
بر امید وصل ما را ملک بود و مال بود
&&
از صفای وقت ما را تخت بود و تاج بود
\\
عشق ما تحقیق بود و شرب ما تسلیم بود
&&
حال ما تصدیق بود و مال ما تاراج بود
\\
چاکر ما چون قباد و بهمن و پرویز بود
&&
خادم ما ایلک و خاقان بد و مهراج بود
\\
از رخ و زلفین او شطرنج بازی کرده‌ام
&&
زان که زلفش ساج بود روی او چون عاج بود
\\
بدرهٔ زر و درم را دست او طیار بود
&&
کعبهٔ محو عدم را جان ما حجاج بود
\\
\end{longtable}
\end{center}
