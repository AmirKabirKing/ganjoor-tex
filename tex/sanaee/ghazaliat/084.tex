\begin{center}
\section*{غزل شماره ۸۴: در دل آن را که روشنایی نیست}
\label{sec:084}
\addcontentsline{toc}{section}{\nameref{sec:084}}
\begin{longtable}{l p{0.5cm} r}
در دل آن را که روشنایی نیست
&&
در خراباتش آشنایی نیست
\\
در خرابات خود به هیچ سبیل
&&
موضع مردم مرایی نیست
\\
پسرا خیز و جام باده بیار
&&
که مرا برگ پارسایی نیست
\\
جرعه‌ای می به جان و دل بخرم
&&
پیش کس می بدین روایی نیست
\\
می خور و علم قیل و قال مگوی
&&
وای تو کاین سخن ملایی نیست
\\
چند گویی تو چون و چند چرا
&&
زین معانی ترا رهایی نیست
\\
در مقام وجود و منزل کشف
&&
چونی و چندی و چرایی نیست
\\
تو یکی گرد دل برآری و ببین
&&
در دل تو غم دوتایی نیست
\\
تو خود از خویش کی رسی به خدای
&&
که ترا خود ز خود جدایی نیست
\\
چون به جایی رسی که جز تو شوی
&&
بعد از آن حال جز خدایی نیست
\\
تو مخوانم سنایی ای غافل
&&
کاین سخنها به خودنمایی نیست
\\
\end{longtable}
\end{center}
