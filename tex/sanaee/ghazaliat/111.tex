\begin{center}
\section*{غزل شماره ۱۱۱: عشق آن معشوق خوش بر عقل و بر ادراک زد}
\label{sec:111}
\addcontentsline{toc}{section}{\nameref{sec:111}}
\begin{longtable}{l p{0.5cm} r}
عشق آن معشوق خوش بر عقل و بر ادراک زد
&&
عشق بازی را بکرد و خاک بر افلاک زد
\\
بر جمال و چهرهٔ او عقلها را پیرهن
&&
نعرهٔ عشق از گریبان تا به دامن چاک زد
\\
حسن او خورشید و ماه و زهره بر فتراک بست
&&
لطف او در چشم آب و باد و آتش خاک زد
\\
آتش عشقش جنیبتهای زر چون در کشید
&&
آب حیوانش به خدمت چنگ در فتراک زد
\\
شاه عشقش چون یکی بر کد خدای روم تاخت
&&
گفتی افریدون در آمد گرز بر ضحاک زد
\\
زهر او آب رخ تریاک برد و پاک برد
&&
درد او بر لشکر درمان زد و بی‌باک زد
\\
درد او دیده چو افسر بر سر درمان نهاد
&&
زهر او چون تیغ دل بر تارک تریاک زد
\\
جادوی استاد پیش خاک پای او بسی
&&
بوسه‌های سرنگون بر پایش از ادراک زد
\\
عقل و جان را همچو شمع و مشعله کرد آنگهی
&&
آتش بی باک را در عقل و جان پاک زد
\\
می سنایی را همو داد و همو زان پس به جرم
&&
سرنگون چون خوشه کرد و حدبه چوب تاک زد
\\
\end{longtable}
\end{center}
