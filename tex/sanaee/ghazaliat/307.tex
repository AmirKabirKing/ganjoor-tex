\begin{center}
\section*{غزل شماره ۳۰۷: نی‌نی به ازین باید با دوست وفا کردن}
\label{sec:307}
\addcontentsline{toc}{section}{\nameref{sec:307}}
\begin{longtable}{l p{0.5cm} r}
نی‌نی به ازین باید با دوست وفا کردن
&&
یا نی کم ازین باید آهنگ جفا کردن
\\
یا زشت بود گویی در کیش نکورویان
&&
یک عهد به سر بردن یک قول وفا کردن
\\
هم گفتن و هم کردن از سوختگان آید
&&
باز از چه شما خامان ناگفتن و ناکردن
\\
باور نکنم قولت زیرا که ترا در دل
&&
یک بادیه ره فرقست از گفتن تاکردن
\\
حاصل نبود کس را از عشق تو در دنیا
&&
جز نامه سیه کردن جز عمر هبا کردن
\\
خود یاد ندارد کس از زلف تو و چشمت
&&
یک تار عطا دادن یک تیر خطا کردن
\\
از بلطمعی تا کی بوسی به رهی دادن
&&
وز بلعجبی تا کی گوشی به ریا کردن
\\
تا چند به طراری ما را به زبان و دل
&&
یک باره بلی گفتن صد باره بلا کردن
\\
تا چند به چالاکی ما را به قبول و رد
&&
یک ماه رهی خواندن یکسال رها کردن
\\
گر فوت شود روزی بد عهدی یک روزه
&&
واجب شمری او را چون فرض قضا کردن
\\
گر بوسه‌ای اندیشم بر خاک سر کویت
&&
صد شهر طمع داری در وقت بها کردن
\\
در مجمع بت رویان تو بوسه دریغی خود
&&
یا رسم بتان نبود از بوسه سخا کردن
\\
یا خوب نباید شد تا هم تو رهی هم ما
&&
ورنه چو شدی باری خوبی به سزا کردن
\\
یا فتنه نباید شد تا کس نشود فتنه
&&
ورنه چو شدی جانا این قاعده نا کردن
\\
هر لحظه یکی دون را صد «طال بقا» گویی
&&
زیشان چه به کف داری زین «طال بقا» کردن
\\
چون هست سنایی را اقبال و سنا از تو
&&
واجب نبود او را مهجور سنا کردن
\\
با این ادب و حرمت حقا که روا نبود
&&
سودای شما پختن صفرای شما کردن
\\
\end{longtable}
\end{center}
