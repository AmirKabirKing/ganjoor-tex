\begin{center}
\section*{غزل شماره ۳۶۲: باد عنبر برد خاک کوی تو}
\label{sec:362}
\addcontentsline{toc}{section}{\nameref{sec:362}}
\begin{longtable}{l p{0.5cm} r}
باد عنبر برد خاک کوی تو
&&
آب آتش ریخت رنگ روی تو
\\
جاودان را نیست اندر کل کون
&&
هیچ دولتخانه چون ابروی تو
\\
کفر و دین را نیست در بازار عشق
&&
گیسه داری چون خم گیسوی تو
\\
چشم و دل ترست و گرم از عشق تو
&&
کام و لب خشک‌ست و سرد از خوی تو
\\
ای بسا خلقا که اندر بند کرد
&&
حلقهاشان حلقه‌های موی تو
\\
گر بهشتی نیست پس جادو چراست
&&
آن دو چشم بلعجب بر روی تو
\\
عالمی را دارویی جز چشم را
&&
بی ضیا چشمست از داروی تو
\\
تا دل ریش مرا دست غمت
&&
بست همچون مهره بر بازوی تو
\\
کافرم چون چشم شوخت گر دهم
&&
دین و دنیا را به تار موی تو
\\
دل چو نار و رخ چو آبی کرده‌ام
&&
از کلوخ امرود و شفتالوی تو
\\
هر کسی محراب دارد هر سویی
&&
هست محراب سنایی سوی تو
\\
ای بسا شرما که برد از چشمها
&&
دیدهٔ شوخ خوش جادوی تو
\\
کی توانم پای در عشقت نهاد
&&
با چنان دست و دل و بازوی تو
\\
سگ به از عقل منست ار عقل من
&&
ناف آهو نشمرد آهوی تو
\\
\end{longtable}
\end{center}
