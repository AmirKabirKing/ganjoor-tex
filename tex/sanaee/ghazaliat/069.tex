\begin{center}
\section*{غزل شماره ۶۹: هر آن روزی که باشم در خرابات}
\label{sec:069}
\addcontentsline{toc}{section}{\nameref{sec:069}}
\begin{longtable}{l p{0.5cm} r}
هر آن روزی که باشم در خرابات
&&
همی نالم چو موسی در مناجات
\\
خوشا روزی که در مستی گذارم
&&
مبارک باشدم ایام و ساعات
\\
مرا بی خویشتن بهتر که باشم
&&
به قرایی فروشم زهد و طاعات
\\
چو از بند خرد آزاد گشتم
&&
نخواهم کرد پس گیتی عمارات
\\
مرا گویی لباسات تو تا کی
&&
خراباتی چه داند جز لباسات
\\
گهی اندر سجودم پیش ساقی
&&
گهی پیش مغنی در تحیات
\\
پدر بر خم خمرم وقف کردست
&&
سبیلم کرد مادر در خرابات
\\
گهی گویم که ای ساقی قدح گیر
&&
گهی گویم که ای مطرب غزل‌هات
\\
گهی باده کشیده تا به مستی
&&
گهی نعره رسیده تا سماوات
\\
مرا موسی نفرماید به تورات
&&
چو کردم حق فرعونی مکافات
\\
چو دانی کاین سنایی ترهاتست
&&
مکن بر روی سلامی خواجه هیهات
\\
\end{longtable}
\end{center}
