\begin{center}
\section*{غزل شماره ۱۹۶: دلم برد آن دلارامی که در چاه زنخدانش}
\label{sec:196}
\addcontentsline{toc}{section}{\nameref{sec:196}}
\begin{longtable}{l p{0.5cm} r}
دلم برد آن دلارامی که در چاه زنخدانش
&&
هزاران یوسف مصرست پیدا در گریبانش
\\
پریرویی که چون دیوست بر رخسار زلفینش
&&
زره مویی که چون تیرست بر عشاق مژگانش
\\
به یک دم می‌کند زنده چو عیسی مرده را زان لب
&&
دم عیسی ست پنداری میان لعل و مرجانش
\\
حلاوت از شکر کم شد چو قیمت آورد نوشش
&&
ازین دو چشم گریانم از آن لبهای خندانش
\\
ندارد لب کس از یاقوت و مروارید تر دندان
&&
گرم باور نمی‌داری بیا بنگر به دندانش
\\
که تا هر گوهری بینی که عکسش در شب تاری
&&
فرو ریزد چو مهر و ماه بر یاقوت گویانش
\\
اگر پیراهن ماهم به مانند فلک آمد
&&
از آن اندر گریبانش بود خورشید تابانش
\\
و یا خورشید پنداری به پیراهن همی هر شب
&&
فرود آید ز گردون و برآید از گریبانش
\\
نشست ما اگر کوهست و او چون ماه بر گردون
&&
چرا هر دو به هم بینیم از آن رخسار رخشانش
\\
بلا و غارت دلهاست آن زلفین او لیکن
&&
هزاران دل چو او جمعست در زلف پریشانش
\\
\end{longtable}
\end{center}
