\begin{center}
\section*{غزل شماره ۲۶۵: بی تو ای آرام جانم زندگانی چون کنم}
\label{sec:265}
\addcontentsline{toc}{section}{\nameref{sec:265}}
\begin{longtable}{l p{0.5cm} r}
بی تو ای آرام جانم زندگانی چون کنم
&&
چون تو پیش من نباشی شادمانی چون کنم
\\
هر زمان گویند دل در مهر دیگر یار بند
&&
پادشاهی کرده باشم پاسبانی چون کنم
\\
داشتی در بر مرا اکنون همان بر در زدی
&&
چون ز من سیر آمدی رفتم گرانی چون کنم
\\
گر بخوانی ور برانی بر منت فرمان رواست
&&
گر بخوانی بنده باشم ور برانی چون کنم
\\
هر شبی گویم که خون خود بریزم در فراق
&&
باز گویم این جهان و آن جهانی چون کنم
\\
بودم اندر وصل تو صاحبقران روزگار
&&
چون فراق آمد کنون صاحبقرانی چون کنم
\\
هست آب زندگانی در لب شیرین تو
&&
بی لب شیرین تو من زندگانی چون کنم
\\
ساختم با عاشقان تا سوختم در عاشقی
&&
پس کنون بی روی خوبت کامرانی چون کنم
\\
هم قضای آسمانی از تو در هجرم فکند
&&
دلبرا من دفع حکم آسمانی چون کنم
\\
بر جهان وصل باری بنده را منشور ده
&&
تات بنمایم که من فرمان روانی چون کنم
\\
من چو موسی مانده‌ام اندر غم دیدار تو
&&
هیچ دانی تا علاج لن ترانی چون کنم
\\
نیستم خضر پیمبر هست این مفخر مرا
&&
چاره و درمان آب زندگانی چون کنم
\\
مر مرا گویی که پیران را نزیبد عاشقی
&&
پیر گشتیم در هوای تو جوانی چون کنم
\\
\end{longtable}
\end{center}
