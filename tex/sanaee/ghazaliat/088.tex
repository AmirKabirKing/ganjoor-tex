\begin{center}
\section*{غزل شماره ۸۸: دوش یارم به بر خویش مرا بار نداد}
\label{sec:088}
\addcontentsline{toc}{section}{\nameref{sec:088}}
\begin{longtable}{l p{0.5cm} r}
دوش یارم به بر خویش مرا بار نداد
&&
قوت جانم زد و یاقوت شکر بار نداد
\\
آن درختی که همه عمر بکشتم به امید
&&
دوش در فرقت او خشک شد و بار نداد
\\
شب تاریک چو من حلقه زدم بر در او
&&
بار چون داد دل او که مرا بار نداد
\\
این چنین کار از آن یار مرا آمد پیش
&&
کم ز یک ماه دل و چشم مرا کار نداد
\\
شربتی ساخته بود از شکر و آب حیات
&&
نه نکو کرد که یک قطره به بیمار نداد
\\
هر که او دل به غم یار دهد خسته شود
&&
رسته آنست که او دل به غم یار نداد
\\
\end{longtable}
\end{center}
