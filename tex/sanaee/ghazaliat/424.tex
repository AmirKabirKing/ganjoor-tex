\begin{center}
\section*{غزل شماره ۴۲۴: ای چشم و چراغ آن جهانی}
\label{sec:424}
\addcontentsline{toc}{section}{\nameref{sec:424}}
\begin{longtable}{l p{0.5cm} r}
ای چشم و چراغ آن جهانی
&&
وی شاهد و شمع آسمانی
\\
خط نو نبشته گرد عارض
&&
منشور جمال جاودانی
\\
بی دیده ز لطف تو بخواند
&&
در جان تو سورهٔ نهانی
\\
با چشم ز تابشت نبیند
&&
بر روی تو صورت عیانی
\\
بخت ازلی و تا قیامت
&&
صافی به طراوت جوانی
\\
حسن تو چو آفتاب آنگه
&&
فارغ ز اشارت نشانی
\\
بوس تو به صد هزار عالم
&&
و آزاد ز زحمت گرانی
\\
دیوانه بسیست آن دو لب را
&&
در سلسله‌های کامرانی
\\
نظاره بسیست آن دو رخ را
&&
از پنجره‌های زندگانی
\\
با فتنهٔ زلف تو که بیند
&&
یک لحظه ز عمر شادمانی
\\
بی آتش عشق تو که یابد
&&
آب خضر و حیات جانی
\\
لطف تو ببست جان و دل را
&&
بر آخور چرب دوستکانی
\\
عشق تو نشاند عقل و دین را
&&
برابرش تیز آنجهانی
\\
با قدر تو پاره میخ بر چرخ
&&
تهمت زدگان باستانی
\\
با قد تو کژ و کوژ در باغ
&&
چالاک و شان بوستانی
\\
از راستی و کژی برونی
&&
آنی که ورای حرف آنی
\\
گویند بگو به ترک ترکت
&&
تا باز دهی ز پاسبانی
\\
ترک چو تو ترک نبود آسان
&&
ترکی تو نه دوغ ترکمانی
\\
حسن تو چو شمس و همچو سایه
&&
پیش و پس تو دوان جوانی
\\
از لفظ تو گوش عاشقانت
&&
نازان به حلاوت معانی
\\
وز چشم تو جسم دوستانت
&&
نازان به حوادث زمانی
\\
در راه تو هیچ دل نشد خوش
&&
تا جانش نگشت کاروانی
\\
بر بام تو پای کس نیاید
&&
تا سرش نکرد نردبانی
\\
در هوش ز تو سماع «ارنی»
&&
در گوش ندای «لن ترانی»
\\
از رد و قبول سیر گشتم
&&
زین بلعجبی چنانکه دانی
\\
یکره بکشم به تیر غمزه
&&
تا سوی عدم برم گردانی
\\
زیرا سر عشق تو ندارد
&&
جز مرد گزاف زندگانی
\\
ور خود تو کشی به دست خویشم
&&
کاری بود آن هزارگانی
\\
فرمان تو هست بر روانها
&&
چون شعر سنایی از روانی
\\
وقتست ترا مراد راندن
&&
کی رانی اگر کنون نرانی
\\
\end{longtable}
\end{center}
