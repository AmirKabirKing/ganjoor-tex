\begin{center}
\section*{غزل شماره ۳۹۸: باز این چه عیاری را شب پوش نهادستی}
\label{sec:398}
\addcontentsline{toc}{section}{\nameref{sec:398}}
\begin{longtable}{l p{0.5cm} r}
باز این چه عیاری را شب پوش نهادستی
&&
آشوب دل ما را بر جوش نهادستی
\\
باز آن چه شگرفی را بر شعلهٔ کافوری
&&
صد کژدم مشکین را بر جوش نهادستی
\\
در حجرهٔ مهجوران چون کلبهٔ زنبوران
&&
هم نیش کشیدستی هم نوش نهادستی
\\
در غارت بی باران چون عادت عیاران
&&
هم چشم گشادستی هم گوش نهادستی
\\
ای روز دو عالم را پوشیده کلاه تو
&&
نامش به چه معنی را شبپوش نهادستی
\\
از جزع تو اقلیمی در شور و تو از شوخی
&&
لعل شکرافشان را خاموش نهادستی
\\
از کشی و چالاکی پیران طریقت را
&&
صد غاشیه از عشقت بر دوش نهادستی
\\
سحرا گه تو کردستی تا نام سنایی را
&&
با آنهمه هوشیاری بی هوش نهادستی
\\
\end{longtable}
\end{center}
