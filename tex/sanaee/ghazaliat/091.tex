\begin{center}
\section*{غزل شماره ۹۱: این نه زلفست آنکه او بر عارض رخشان نهاد}
\label{sec:091}
\addcontentsline{toc}{section}{\nameref{sec:091}}
\begin{longtable}{l p{0.5cm} r}
این نه زلفست آنکه او بر عارض رخشان نهاد
&&
صورت جوریست کو بر عدل نوشروان نهاد
\\
گر زند بر زهر بوسه زهر گردد چون شکر
&&
یارب آن چندین حلاوت در لبی بتوان نهاد
\\
توبه و پرهیز ما را تابش از هم باز کرد
&&
تا به عمدا زلف را بر آن رخ تابان نهاد
\\
از دل من وز سر زلفین او اندازه کرد
&&
آنکه در میدان مدار گوی در چوگان نهاد
\\
دیدمش یک روز شادان و خرامان از کشی
&&
همچو ماهی کش فلک یک روز در دوران نهاد
\\
گفتم ای مست جمال آن وعدهٔ وصل تو کو
&&
خوش بخندید آن صنم انگشت بر دندان نهاد
\\
گفت مستم خوانی و بر وعدهٔ من دل نهی
&&
ساده دل مردا که بر وعدهٔ مستان نهاد
\\
\end{longtable}
\end{center}
