\begin{center}
\section*{غزل شماره ۴۱۳: در ره روش عشق چه میری چه اسیری}
\label{sec:413}
\addcontentsline{toc}{section}{\nameref{sec:413}}
\begin{longtable}{l p{0.5cm} r}
در ره روش عشق چه میری چه اسیری
&&
در مذهب عاشق چه جوانی و چه پیری
\\
آنجا که گذر کرد بناگه سپه عشق
&&
رخها همه زردست و جگرها همه قیری
\\
آزاد کن از تیرگی خویش و غم عشق
&&
تا بندهٔ خال تو بود نور اثیری
\\
عالم همه بی‌رنج حقیری ز غم عشق
&&
ای بی‌خبر از رنج حقیری چه حقیری
\\
میری چه کند مرد که روزی به همه عمر
&&
سودای بتی به که همه عمر امیری
\\
آن سینه که بردی بدل دل غم عشقت
&&
بی غم بود از نعمت گوینده و قیری
\\
این نیمه که عشقست از آن سو همه شادیست
&&
اینجا که تویی تست همه رنج و زحیری
\\
سودای زبان گر چه نشاطیست به ظاهر
&&
خود سود دگر دارد سودای ضمیری
\\
راه و صفت عشق ز اغیار یگانه‌ست
&&
نیکو نبود در ره او جفت پذیری
\\
خواهی که شوی محرم غین غم معشوق
&&
بیوفای فقیهی شو و بی قاف فقیری
\\
تا در چمن صورت خویشی به تماشا
&&
یک میوه ز شاخ چمن دوست نگیری
\\
از پوست برون آی همه دوست شو ایرا
&&
کانگاه همه دوست شوی هیچ نمیری
\\
\end{longtable}
\end{center}
