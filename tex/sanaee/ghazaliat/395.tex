\begin{center}
\section*{غزل شماره ۳۹۵: کودکی داشتم خراباتی}
\label{sec:395}
\addcontentsline{toc}{section}{\nameref{sec:395}}
\begin{longtable}{l p{0.5cm} r}
کودکی داشتم خراباتی
&&
می کش و کمزن و خرافاتی
\\
پارسا شد ز بخت و دولت من
&&
پارسایی شگرف و طاماتی
\\
شیوهٔ خمر و قمر و رمز مدام
&&
صفتی بود مرورا ذاتی
\\
آنکه والتین ز بر ندانستی
&&
همچو بلخیر گشت هیهاتی
\\
خوانده از بر همیشه چو الحمد
&&
عدد سورهٔ لباساتی
\\
گوید امروز بر من از سر زهد
&&
مثل و نکتهٔ اشاراتی
\\
دوش گفتم ورا که ای دل و جان
&&
مر مرا مایهٔ مباهاتی
\\
گر چه مستور و پارسا شده‌ای
&&
واصل هر گونهٔ کراماتی
\\
گر یکی بوسه خواهم از تو دهی؟
&&
گفت: لا والله ای خراباتی
\\
ای سنایی کما ترید خوشست
&&
دل به قسمت بنه کمایاتی
\\
\end{longtable}
\end{center}
