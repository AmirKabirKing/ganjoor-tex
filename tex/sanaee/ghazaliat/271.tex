\begin{center}
\section*{غزل شماره ۲۷۱: خورشید تویی و ذره ماییم}
\label{sec:271}
\addcontentsline{toc}{section}{\nameref{sec:271}}
\begin{longtable}{l p{0.5cm} r}
خورشید تویی و ذره ماییم
&&
بی روی تو روی کی نماییم
\\
تا کی به نقاب و پرده یک ره
&&
از کوی برآی تا برآییم
\\
چون تو صنم و چو ما شمن نیست
&&
شهری و گلی تویی و ماییم
\\
آخر نه ز گلبن تو خاریم
&&
آخر نه ز باغ تو گیاییم
\\
گر دستهٔ گل نیاید از ما
&&
هم هیزم دیگ را بشاییم
\\
بادی داریم در سر ایراک
&&
در پیش سگ تو خاکپاییم
\\
آب رخ ما مبر ازیراک
&&
با خاک در تو آشناییم
\\
از خاک در تو کی شکیبیم
&&
تا عاشق چشم و توتیاییم
\\
یک روز نپرسی از ظریفی
&&
کاخر تو کجا و ما کجاییم
\\
زامد شد ما مکن گرانی
&&
پندار که در هوا هباییم
\\
بل تا کف پای تو ببوسیم
&&
انگار که مهر لالکاییم
\\
برف آب همی دهی تو ما را
&&
ما از تو فقع همی گشاییم
\\
با سینهٔ چاک همچو گندم
&&
گرد تو روان چو آسیاییم
\\
بر در زده‌ای چو حلقه ما را
&&
ما رقص کنان که در سراییم
\\
وندر همه ده جوی نه ما را
&&
ما لاف زنان که ده خداییم
\\
از شیر فلک چه باک داریم
&&
چون با سگ کویت آشناییم
\\
ما را سگ خویش خوان که تا ما
&&
گوییم که شیر چرخ ماییم
\\
پرسند ز ما که‌اید گوییم
&&
ما هیچ کسان پادشاییم
\\
تو بر سر کار خویش می‌باش
&&
تا ماهله خود همی درآییم
\\
کز عشق تو ای نگار چنگی
&&
اکنون نه سناییم ناییم
\\
\end{longtable}
\end{center}
