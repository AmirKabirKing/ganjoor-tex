\begin{center}
\section*{غزل شماره ۱۰: در ده پسرا می مروق را}
\label{sec:010}
\addcontentsline{toc}{section}{\nameref{sec:010}}
\begin{longtable}{l p{0.5cm} r}
در ده پسرا می مروق را
&&
یاران موافق موفق را
\\
زان می که چو آه عاشقان از تف
&&
انگشت کند بر آب زورق را
\\
زان می که کند ز شعله پر آتش
&&
این گنبد خانهٔ معلق را
\\
هین خیز و ز عکس باده گلگون کن
&&
این اسب سوار خوار ابلق را
\\
در زیر لگد بکوب چون مردان
&&
این طارم زرق پوش ازرق را
\\
گه ساقی باش و گه حریفی کن
&&
ترتیب فروگذار و رونق را
\\
یک دم خوش باش تا چه خواهی کرد
&&
این زهد مزور مزیق را
\\
یک ره به دو باده دست کوته کن
&&
این عقل دراز قد احمق را
\\
بنمای به زیرکان دیوانه
&&
از مصحف باطل آیت حق را
\\
بر لاله مزن ز چشم سنبل را
&&
بر پسته منه ز ناز فندق را
\\
بیرون شو ازین دو رنگ و این ساعت
&&
همرنگ حریر کن ستبرق را
\\
مشکن به طمع مرا تو ای ممسک
&&
چونان که جریر مر فرزدق را
\\
گر طمع میان تهی سه حرف آمد
&&
چار است میان تهی مطوق را
\\
در تختهٔ اول ار بنوشتی
&&
بی شکل حروف علم مطلق را
\\
کم زان باری که در دوم تخته
&&
چون نسخ کنی خط محقق را
\\
در موضع خوشدلان و مشتاقان
&&
موضوع فروگذار و مشتق را
\\
شعر تر مطلق سنایی خوان
&&
آتش در زن حدیث مغلق را
\\
\end{longtable}
\end{center}
