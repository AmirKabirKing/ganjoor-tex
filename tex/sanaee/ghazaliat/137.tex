\begin{center}
\section*{غزل شماره ۱۳۷: هر زمان از عشقت ای دلبر دل من خون شود}
\label{sec:137}
\addcontentsline{toc}{section}{\nameref{sec:137}}
\begin{longtable}{l p{0.5cm} r}
هر زمان از عشقت ای دلبر دل من خون شود
&&
قطره‌ها گردد ز راه دیدگان بیرون شود
\\
گر ز بی صبری بگویم راز دل با سنگ و روی
&&
روی را تن آب گردد سنگ را دل خون شود
\\
ز آتش و درد فراقت این نباشد بس عجب
&&
گر دل من چون جحیم و دیده چون جیحون شود
\\
بار اندوهان من گردون کجا داند کشید
&&
خاصه چون فریادم از بیداد بر گردون شود
\\
در غم هجران و تیمار جدایی جان من
&&
گاه چون ذوالکفل گردد گاه چون ذوالنون شود
\\
در دل از مهرت نهالی کشته‌ام کز آب چشم
&&
هر زمانی برگ و شاخ و بیخ او افزون شود
\\
تا تو در حسن و ملاحت همچنان لیلی شدی
&&
عاشق مسکینت ای دلبر همی مجنون شود
\\
خاک درگاه تو ای دلبر اگر گیرد هوا
&&
توتیای حور و چتر شاه سقلاطون شود
\\
ای شده ماه تمام از غایت حسن و جمال
&&
چاکر از هجران رویت «عادکالعرجون» شود
\\
آن دلی کز خلق عالم دارد امیدی به تو
&&
چون ز تو نومید گردد ماهرویا چون شود
\\
چون سنایی مدحتت گوید ز روی تهنیت
&&
لفظ اسرار الاهی در دلش معجون شود
\\
\end{longtable}
\end{center}
