\begin{center}
\section*{غزل شماره ۳۱۹: گر کار بجز مستی اسکندر می من}
\label{sec:319}
\addcontentsline{toc}{section}{\nameref{sec:319}}
\begin{longtable}{l p{0.5cm} r}
گر کار به جز مستی اسکندر می من
&&
ور معجزه شعرستی پیغمبر می من
\\
با اینهمه گر عشق یکی ماه نبودی
&&
اندر دو جهان شاه بلند اختر می من
\\
ماهی و چه ماهی که ز هجرانش برین حال
&&
گر من به غمش نگرومی کافر می من
\\
گر بندهٔ خوی بد خود نیستی آن ماه
&&
حقا که به فردوس همش چاکر می من
\\
گر نیستی آن رنج که او ریش درآورد
&&
وی گه که درین وقت چگوید درمی من
\\
بودیش سر عشق من و برگ مراعات
&&
گر چون دگران فاسق در کون بر می من
\\
گر تیر برویی زندم از سر شنگی
&&
از شادی تیرش به هوا بر پر می من
\\
گادیم بر آنگونه که از جهل و رعونت
&&
از گردن خود بفگنمی گر سرمی من
\\
هر روز دل آید که مگر نیک شود یار
&&
گر خر نیمی عشوهٔ او کی خر می من
\\
گر بلفرج مول خبر یابدی از من
&&
زین روی برین طایقه سر دفتر می من
\\
پس در غم آنکس که ز گل خار نداند
&&
عمر از چه کنم یاد که رشک خور می من
\\
\end{longtable}
\end{center}
