\begin{center}
\section*{غزل شماره ۳۹۲: ای یوسف ایام ز عشق تو سنایی}
\label{sec:392}
\addcontentsline{toc}{section}{\nameref{sec:392}}
\begin{longtable}{l p{0.5cm} r}
ای یوسف ایام ز عشق تو سنایی
&&
مانندهٔ یعقوب شد از درد جدایی
\\
تا چند به سوی دل عشاق چو خورشید
&&
هر روز به رنگ دگر از پرده برآیی
\\
گاهی رخ تو سجده برد مشتی دون را
&&
گه باز کند زلف تو دعوی خدایی
\\
با خوی تو در کوی تو از دیده روانیست
&&
کس را بگذشتن ز سر حد گدایی
\\
در وصل تو با خوی تو از روی خرد نیست
&&
جان را ز خم زلف تو امید رهایی
\\
بس بلعجب آسایی و وین بلعجبی بس
&&
کاندر همه تن کس بنداند که کجایی
\\
بس نادره کرداری وین نادره‌ای بس
&&
کان همه‌ای و همه جویان که کرایی
\\
از ما چه شوی پنهان کاندر ره توحید
&&
ما جمله توایم ای پسر خوب و تو مایی
\\
آنجا که تویی من نتوانم که نباشم
&&
وینجا که منم مانده تو دانم که نیایی
\\
\end{longtable}
\end{center}
