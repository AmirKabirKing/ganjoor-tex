\begin{center}
\section*{غزل شماره ۱۲: مرد بی حاصل نیابد یار با تحصیل را}
\label{sec:012}
\addcontentsline{toc}{section}{\nameref{sec:012}}
\begin{longtable}{l p{0.5cm} r}
مرد بی حاصل نیابد یار با تحصیل را
&&
جان ابراهیم باید عشق اسماعیل را
\\
گر هزاران جان لبش را هدیه آرم گویدم
&&
نزد عیسا تحفه چون آری همی انجیل را
\\
زلف چون پرچین کند خواری نماید مشک را
&&
غمزه چون بر هم زند قیمت فزاید نیل را
\\
چون وصال یار نبود گو دل و جانم مباش
&&
چون شه و فرزین نباشد خاک بر سر فیل را
\\
از دو چشمش تیز گردد ساحری ابلیس را
&&
وز لبانش کند گردد تیغ عزراییل را
\\
گر چه زمزم را پدید آورد هم نامش به پای
&&
او به مویی هم روان کرد از دو چشمم نیل را
\\
جان و دل کردم فدای خاکپایش بهر آنک
&&
از برای کعبه چاکر بود باید میل را
\\
آب خورشید و مه اکنون برده شد کو بر فروخت
&&
در خم زلف از برای عاشقان قندیل را
\\
ای سنایی گر هوای خوبرویان می‌کنی
&&
از نخستت ساخت باید دبه و زنبیل را
\\
\end{longtable}
\end{center}
