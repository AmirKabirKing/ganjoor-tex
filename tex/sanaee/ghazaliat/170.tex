\begin{center}
\section*{غزل شماره ۱۷۰: حلقهٔ زلف تو در گوش ای پسر}
\label{sec:170}
\addcontentsline{toc}{section}{\nameref{sec:170}}
\begin{longtable}{l p{0.5cm} r}
حلقهٔ زلف تو در گوش ای پسر
&&
عالمی افگنده در جوش ای پسر
\\
کیست در عالم که بیند مر ترا
&&
کش بجا ماند دل و هوش ای پسر
\\
هم تویی ماه قدح‌گیر ای غلام
&&
هم تویی سرو قباپوش ای پسر
\\
سرو در بر دارم و مه در کنار
&&
چون ترا دارم در آغوش ای پسر
\\
بر جفا کاری چه کوشی ای غلام
&&
بر وفاداری همی کوش ای پسر
\\
امشب ای دلبر به دام آویختی
&&
کز برم بگریختی دوش ای پسر
\\
بوسهٔ نوشین همی بخش از عقیق
&&
بادهٔ نوشین همی نوش ای پسر
\\
کم کن این آزار و این بدها مجوی
&&
میر داد اینجاست خاموش ای پسر
\\
\end{longtable}
\end{center}
