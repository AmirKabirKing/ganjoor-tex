\begin{center}
\section*{غزل شماره ۳۸۹: از ماه رخی نوش لبی شوخ بلایی}
\label{sec:389}
\addcontentsline{toc}{section}{\nameref{sec:389}}
\begin{longtable}{l p{0.5cm} r}
از ماه رخی نوش لبی شوخ بلایی
&&
هر روز همی بینم رنجی و عنایی
\\
شکرست مر آنرا که نباشد سر و کارش
&&
با پاک‌بری عشوه‌دهی شوخ دغایی
\\
گویی که ندارد به جهان پیشهٔ دیگر
&&
جز آنکه کند با من بیچاره جفایی
\\
تا چند کند جور و جفا با من عاشق
&&
ناکرده به جای من یکروز وفایی
\\
تا چند کشم جورش من بنده به دعوی
&&
یعنی که همی آیم من نیز ز جایی
\\
دانم که خلل ناید در حشمت او را
&&
گر عاشق او باشد بیچاره گدایی
\\
گر جامه کنم پاره و گر بذل کنم دل
&&
گوید که مرا هست درین هر دو ریایی
\\
خورشید رخست او و سنایی را زان چه
&&
چون نیست نصیب او هر روز ضیایی
\\
\end{longtable}
\end{center}
