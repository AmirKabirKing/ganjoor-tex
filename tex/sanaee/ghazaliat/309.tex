\begin{center}
\section*{غزل شماره ۳۰۹: جانا ز لب آموز کنون بنده خریدن}
\label{sec:309}
\addcontentsline{toc}{section}{\nameref{sec:309}}
\begin{longtable}{l p{0.5cm} r}
جانا ز لب آموز کنون بنده خریدن
&&
کز زلف بیاموخته‌ای پرده دریدن
\\
فریادرس او را که به دام تو درافتاد
&&
یا نیست ترامذهب فریاد رسیدن
\\
ما صبر گزیدیم به دام تو که در دام
&&
بیچاره شکاری خبه گردد ز تپیدن
\\
اکنون که رضای تو به اندوه تو جفتست
&&
اندوه تو ما را چو شکر شد به چشیدن
\\
از بیم به یکبار همی خورد نیارم
&&
زیرا که شکر هیچ نماند ز مزیدن
\\
ما رخت غریبانه ز کوی تو کشیدیم
&&
ماندیم به تو آنهمه کشی و چمیدن
\\
رفتیم به یاد تو سوی خانه و بردیم
&&
خاک سر کویت ز پی سرمه کشیدن
\\
در حسرت آن دانهٔ نار تو دل ما
&&
حقا که چو نارست به هنگام کفیدن
\\
یاد آیدت آن آمدن ما به سر کوی
&&
دزدیده در آن دیدهٔ شوخت نگریدن؟
\\
ای راحت آن باد که از نزد تو آید
&&
پیغام تو آرد بر ما وقت بزیدن
\\
وان طیره گری کردن و در راه نشستن
&&
وان سنگدلی کردن و در حجره دویدن
\\
ما را غرض از عشق تو ای ماه رخت بود
&&
خود چیست شمن را غرض از بت گرویدن
\\
ما را فلک از دیده همی خواست جدا کرد
&&
برخیره نبود آن دو سه شب چشم پریدن
\\
زین روی که بر خاک سر کوی تو خسبد
&&
مولای سگ کوی توام وقت گزیدن
\\
زنهار کیانند به زیر خم زلفت
&&
زنهار به هش باش گه زلف بریدن
\\
بشنو سخن ما ز حریفان به ظریفی
&&
کارزد سخن بنده سنایی بشنیدن
\\
پیش و بر ما ز آرزوی چشم چو آهوت
&&
چون پشت پلنگست ز خونابه چکیدن
\\
آرامش و رامش همه در صحبت خلقست
&&
ای آهوک از سر بنه این خوی رمیدن
\\
کوهیست غم عشق تو موییست تن من
&&
هرگز نتوان کوه به یک موی کشیدن
\\
ما بندگی خویش نمودیم ولیکن
&&
خوی بد تو بنده ندانست خریدن
\\
\end{longtable}
\end{center}
