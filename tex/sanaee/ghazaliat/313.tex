\begin{center}
\section*{غزل شماره ۳۱۳: ای هوایی یار یک ره تو هوای یار زن}
\label{sec:313}
\addcontentsline{toc}{section}{\nameref{sec:313}}
\begin{longtable}{l p{0.5cm} r}
ای هوایی یار یک ره تو هوای یار زن
&&
آتشی بفروز و اندر خرمن اغیار زن
\\
طبل از هستی خویش اندر جهان تاکی زنی
&&
بر در هستی یکی از نیستی مسمار زن
\\
با می تلخ مغانه دامن افلاس گیر
&&
آز را بر روی آن قرای دعوی‌دار زن
\\
زاهدان ار تکیه بر زهد و صیام خود کنند
&&
تو چو مردان تکیه بر خمر و در خمار زن
\\
دور شو از صحبت خود بر در صورت پرست
&&
بوسه بر خاک کف پای ز خود بیزار زن
\\
چون خوری می با حریف محرم پر درد خور
&&
چون زنی کم با ندیم زیرک هشیار زن
\\
گر برون هفت چرخ و چار طبع‌ست این سخن
&&
بارگاهش هم برون از هفت و هشت و چار زن
\\
تا تو اندر بند طبع و دهر و چرخ و کوکبی
&&
کی بود جایز که گویی دم قلندوار زن
\\
قیل و قال و دانش و تیمار پندار رهند
&&
خاک بر چشم همه تیمارهٔ پندار زن
\\
\end{longtable}
\end{center}
