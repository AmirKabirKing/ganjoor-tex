\begin{center}
\section*{غزل شماره ۱۱۲: خوبت آراست ای غلام ایزد}
\label{sec:112}
\addcontentsline{toc}{section}{\nameref{sec:112}}
\begin{longtable}{l p{0.5cm} r}
خوبت آراست ای غلام ایزد
&&
چشم بد دورخه به نام ایزد
\\
نافرید و نیاورید به حسن
&&
هیچ صورت چو تو تمام ایزد
\\
در جهان جمالت از رخ و زلف
&&
بهم آورد صبح و شام ایزد
\\
سبب آبروی جانها کرد
&&
خاک کوی تو گام گام ایزد
\\
از پی عزت جمال تو داد
&&
صورت لطف را قوام ایزد
\\
از پی منت وجود تو کرد
&&
گردنانرا به زیر وام ایزد
\\
از پی خدمت رکاب تو داد
&&
آدمی را دم دوام ایزد
\\
کرد گرد سم ستور رهت
&&
سرمهٔ چشم خاص و عام ایزد
\\
برهمن را چو پرسی ایزد کیست
&&
گوید آن رخ نگر کدام ایزد
\\
ای به هر دم شراب آدم خوار
&&
زده بر جام جانت جام ایزد
\\
سر دام خودی نداری هیچ
&&
زان مدامت دهد مدام ایزد
\\
وز برای شکار دلها ساخت
&&
خال تو دانه زلف دام ایزد
\\
آنچنان کعبه‌ای که هست ترا
&&
در و دیوار و صحن و بام ایزد
\\
بده انصاف هیچ وا نگرفت
&&
از تو از نیکویی و کام ایزد
\\
خوبت آراسته‌ست طرفه تر آنک
&&
خود همی گویدت به نام ایزد
\\
تو مقیمی از آن سنایی را
&&
داد بر درگهت مقام ایزد
\\
\end{longtable}
\end{center}
