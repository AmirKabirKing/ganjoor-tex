\begin{center}
\section*{غزل شماره ۳۶۸: ای دل اندر بیم جان از بهر دل بگداخته}
\label{sec:368}
\addcontentsline{toc}{section}{\nameref{sec:368}}
\begin{longtable}{l p{0.5cm} r}
ای دل اندر بیم جان از بهر دل بگداخته
&&
جان شیرین را ز تن در کار دل پرداخته
\\
تا دل و جان درنبازی دل نبیند ناز و عز
&&
کی سر آخور گشت هرگز مرکبی ناتاخته
\\
بند مادرزاد باید همچو مرغابی به پای
&&
طوق ایزد کرد باید در عنق چون فاخته
\\
تا به روی آب چون مرغابیان دانی گذشت
&&
در هوا چون فاخته پری و بال آخته
\\
مرد این ره را گذر بر روی آب و آتشست
&&
آب و آتش آشنا را داند از نشناخته
\\
یاد کن آن مرد را کو پای در دریا نهاد
&&
از پسش دشمن همی آمد علم افراخته
\\
آب رود نیل هر دو مرد را بر سنگ زد
&&
کم عیار آمد یکی زو روح شد پرداخته
\\
آتش نمرود و آن لشکر نمی‌بینم به جای
&&
زر آزر را دگر کن منجنیق انداخته
\\
ایزدش پیرایه چون زر کرد ازین کاتش بدید
&&
هر زری کو دید آتش کار او شد ساخته
\\
\end{longtable}
\end{center}
