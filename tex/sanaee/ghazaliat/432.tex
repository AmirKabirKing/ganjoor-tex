\begin{center}
\section*{غزل شماره ۴۳۲: برخی رویتان من ای رویتان چو ماهی}
\label{sec:432}
\addcontentsline{toc}{section}{\nameref{sec:432}}
\begin{longtable}{l p{0.5cm} r}
برخی رویتان من ای رویتان چو ماهی
&&
وی جان بیدلان را در زلفتان پناهی
\\
با رویتان تنی را باطل نگشت حقی
&&
با زلفتان دلی را مشکل نماند راهی
\\
جز رویتان که سازد جانهای عاشقان را
&&
از ما سجده‌گاهی وز مشک تکیه‌گاهی
\\
جز زلفتان که دارد چون شهد و شمع محفل
&&
از نیش جنگجویی وز نوش عذرخواهی
\\
نگذاشت زلف و رختان اندر مصاف و مجلس
&&
در هیچ پای نعلی در هیچ سر کلاهی
\\
با حد و خد هر یک خورشید کم ز ظلی
&&
با قد و قدر هر یک طوبا کم از گیاهی
\\
از لعل درفشانتان یک خنده و سپهری
&&
ور جزع جانستانتان یک ناوک و سپاهی
\\
چون لعلتان بخندد هر عیسیی و چرخی
&&
چون جزعتان بجنبد هر یوسفی و چاهی
\\
از دام دل شکرتان هر دانه‌ای و شهری
&&
ا زجام جان ستانتان هر قطره‌ای و شاهی
\\
با جام باده هر یک در بزمگه سروشی
&&
با دست و تیغ هر یک در رزمگه سپاهی
\\
جز رویتان که دیدست از روی رنگ رویی
&&
جز چشمتان که دیدست از چشم نور گاهی
\\
زینان سیاه گرتر نشنیده‌ام سپیدی
&&
زینها سپیدگرتر کم دیده‌ام سیاهی
\\
گر چنبر فلکرا ماهیست مر شما را
&&
صد چنبرست هر سو هر چنبری و ماهی
\\
تا باده ده شمایید اندر میان مجلس
&&
از باده توبه کردن نبود مگر گناهی
\\
از روی بی‌نیازی بیجاده که رباید
&&
ورنه چه خیزد آخر بیجاده را ز کاهی
\\
از تیزی سنانتان هر ساعت از سنایی
&&
آهی همی برآید جانی میان آهی
\\
\end{longtable}
\end{center}
