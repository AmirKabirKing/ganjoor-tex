\begin{center}
\section*{غزل شماره ۷۹: شور در شهر فگند آن بت زنارپرست}
\label{sec:079}
\addcontentsline{toc}{section}{\nameref{sec:079}}
\begin{longtable}{l p{0.5cm} r}
شور در شهر فگند آن بت زنارپرست
&&
چون خرامان ز خرابات برون آمد مست
\\
پردهٔ راز دریده قدح می در کف
&&
شربت کفر چشیده علم کفر به دست
\\
شده بیرون ز در نیستی از هستی خویش
&&
نیست حاصل شود آنرا که برون شد از هست
\\
چون بت ست آن بت قلاش دل رهبان کیش
&&
که به شمشیر جفا جز دل عشاق نخست
\\
اندر آن وقت که جاسوس جمال رخ او
&&
از پس پردهٔ پندار و هوا بیرون جست
\\
هیچ ابدال ندیدی که درو در نگریست
&&
که در آن ساعت زنار چهل گردن بست
\\
گاه در خاک خرابات به جان باز نهاد
&&
خاکیی را که ازین خاک شود خاک پرست
\\
بر در کعبهٔ طامات چه لبیک زنیم
&&
که به بتخانه نیابیم همی جای نشست
\\
\end{longtable}
\end{center}
