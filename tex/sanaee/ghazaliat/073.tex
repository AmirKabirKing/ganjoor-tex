\begin{center}
\section*{غزل شماره ۷۳: گل به باغ آمده تقصیر چراست}
\label{sec:073}
\addcontentsline{toc}{section}{\nameref{sec:073}}
\begin{longtable}{l p{0.5cm} r}
گل به باغ آمده تقصیر چراست
&&
ساقیا جام می لعل کجاست
\\
به چنین وقت و چنین فصل عزیز
&&
کاهلی کردن و سستی نه رواست
\\
ای سنایی تو مکن توبه ز می
&&
که ترا توبه درین فصل خطاست
\\
عاشقی خواهی و پس توبه کنی
&&
توبه و عشق بهم ناید راست
\\
روزکی چند بود نوبت گل
&&
روزه و توبه همه روز بجاست
\\
جز از آن نیست که گویند مرا
&&
یار بود آنکه نه از مجمع ماست
\\
شد به بد مردی و میخانه گزید
&&
نیک مردی را با زهد نخواست
\\
من به بد مردی خرسند شدم
&&
هر قضایی که بود خود ز قضاست
\\
ای بدا مرد که امروز منم
&&
ای خوشا عیش که امروز مراست
\\
\end{longtable}
\end{center}
