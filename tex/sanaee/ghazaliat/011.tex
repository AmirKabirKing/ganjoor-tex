\begin{center}
\section*{غزل شماره ۱۱: چند رنجانی نگارا این دل مشتاق را}
\label{sec:011}
\addcontentsline{toc}{section}{\nameref{sec:011}}
\begin{longtable}{l p{0.5cm} r}
چند رنجانی نگارا این دل مشتاق را
&&
یا سلامت خود مسلم نیست مر عشاق را
\\
هر کرا با عشق خوبان اتفاق آمد پدید
&&
مشتری گردد همیشه محنت مخراق را
\\
زآنکه چون سلطان عشق اندر دل ماوا گرفت
&&
محو گرداند ز مردم عادت و اخلاق را
\\
هر که بی اوصاف شد از عشق آن بت برخورد
&&
کان صنم طاقست اندر حسن و خواهد طاق را
\\
ذره‌ای از حسن او در مصر اگر پیدا شدی
&&
دل ربودی یوسف یعقوب بن اسحاق را
\\
گر سر مژگان زند بر هم به عمدا آن نگار
&&
پیکران بی جان کند مر دیلم و قفچاق را
\\
هر که روی او بدید از جان و دل درویش شد
&&
زر سگالی کس ندید آن شهرهٔ آفاق را
\\
\end{longtable}
\end{center}
