\begin{center}
\section*{غزل شماره ۵: باز تابی در ده آن زلفین عالم سوز را}
\label{sec:005}
\addcontentsline{toc}{section}{\nameref{sec:005}}
\begin{longtable}{l p{0.5cm} r}
باز تابی در ده آن زلفین عالم سوز را
&&
باز آبی بر زن آن روی جهان افروز را
\\
باز بر عشاق صوفی طبع صافی جان گمار
&&
آن دو صف جادوی شوخ دلبر جان دوز را
\\
باز بیرون تاز در میدان عقل و عافیت
&&
آن سیه پوشان کفر انگیز ایمان‌سوز را
\\
سر برآوردند مشتی گوشه گشته چون کمان
&&
باز در کار آر نوک ناوک کین توز را
\\
روزها چون عمر بد خواه تو کوتاهی گرفت
&&
پاره‌ای از زلف کم کن مایه‌ای ده روز را
\\
آینه بر گیر و بنگر گر تماشا بایدت
&&
در میان روی نرگس بوستان افروز را
\\
لب ز هم بردار یک دم تا هم اندر تیر ماه
&&
آسمان در پیشت اندر جل کشد نوروز را
\\
نوگرفتان را ببوسی بسته گردان بهر آنک
&&
دانه دادن شرط باشد مرغ نو آموز را
\\
بر شکن دام سنایی ز آن دو تا بادام از آنک
&&
دام را بادام تو چون سنگ باشد گوز را
\\
\end{longtable}
\end{center}
