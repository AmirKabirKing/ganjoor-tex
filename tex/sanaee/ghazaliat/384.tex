\begin{center}
\section*{غزل شماره ۳۸۴: عقل و جانم برد شوخی آفتی عیاره‌ای}
\label{sec:384}
\addcontentsline{toc}{section}{\nameref{sec:384}}
\begin{longtable}{l p{0.5cm} r}
عقل و جانم برد شوخی آفتی عیاره‌ای
&&
باد دستی خاکیی بی آبی آتشپاره‌ای
\\
زین یکی شنگی بلایی فتنه‌ای شکر لبی
&&
پای بازی سر زنی دردی کشی خونخواره‌ای
\\
گه در ایمان از رخ ایمان فزایش حجتی
&&
گاه بر کفر از دو زلف کافرش پتیاره‌ای
\\
کی بدین کفر و بدین ایمان من تن در دهد
&&
هر کرا باشد چنان زلف و چنان رخساره‌ای
\\
هر زمان در زلف جان آویز او گر بنگری
&&
خون خلقی تازه یابی در خم هر تاره‌ای
\\
هر زمان بینی ز شور زلف او برخاسته
&&
در میان عاشقان آوازهٔ آواره‌ای
\\
نقش خود را چینیان از جان همی خدمت کنند
&&
نقش حق را آخر ای مستان کم از نظاره‌ای
\\
\end{longtable}
\end{center}
