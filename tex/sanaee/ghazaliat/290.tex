\begin{center}
\section*{غزل شماره ۲۹۰: باز ماندم در بلایی الغیاث ای دوستان}
\label{sec:290}
\addcontentsline{toc}{section}{\nameref{sec:290}}
\begin{longtable}{l p{0.5cm} r}
باز ماندم در بلایی الغیاث ای دوستان
&&
از هوای بی وفایی الغیاث ای دوستان
\\
باز آتش در زد اندر جانم و آبم ببرد
&&
باد دستی خاکپایی الغیاث ای دوستان
\\
باز دیگر باره چون سنگین دلان بر ساختم
&&
از بت چونین جدایی الغیاث ای دوستان
\\
باز ناگه بلعجب وارم پس چادر نشاند
&&
آفتابی را هبایی الغیاث ای دوستان
\\
باده‌خواران باز رخ دارند زی صحرا و نیست
&&
در همه صحرا گیایی الغیاث ای دوستان
\\
بنگه هادوریان را ماند این دل کز طمع
&&
هر دمش بینم به جایی الغیاث ای دوستان
\\
جادوی فرعونیان در جنبش آمد باز و نیست
&&
در کف موسی عصایی الغیاث ای دوستان
\\
خواهد اندر وی همی از شاخ خشک و مرغ گنگ
&&
هر زمان برگ و نوایی الغیاث ای دوستان
\\
دیدهٔ روشن جز از من در همه عالم که داد
&&
در بهای توتیایی الغیاث ای دوستان
\\
از برای انس جان انس و جان ای سرفراز
&&
مر سنایی را چو نایی الغیاث ای دوستان
\\
\end{longtable}
\end{center}
