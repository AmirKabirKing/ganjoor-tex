\begin{center}
\section*{غزل شماره ۳۹۳: آخر شرمی بدار چند ازین بدخویی}
\label{sec:393}
\addcontentsline{toc}{section}{\nameref{sec:393}}
\begin{longtable}{l p{0.5cm} r}
آخر شرمی بدار چند ازین بدخویی
&&
چون تو من و من توام چند منی و تویی
\\
گلشن گلخن شود چون به ستیزه کنند
&&
در یک خانه دو تن دعوی کدبانویی
\\
نایب عیسی شدی قبله یکی کن چنو
&&
بر دل ترسا نگار رقم دویی و تویی
\\
صدر زمانه تویی پس چو زمانه چرا
&&
گه همه دردی کنی گاه همه دارویی
\\
نازی در سر که چه یعنی من نیکوم
&&
تا تو بدین سیرتی نه تو و نه نیکویی
\\
یک دم و یک رنگ باش چون گهر آفتاب
&&
چند چو چرخ کهن هر دم رسم دویی
\\
روبه بازی مکن در صف عشاق از آنک
&&
زشت بود پیش گرگ شیر کند آهویی
\\
با رخ تو بیهدست بلعجبی چشم تو
&&
با کف موسی کرا دست دهد جادویی
\\
همره درد تو باد دولت بی‌دولتی
&&
هم تک عشق تو باد نیروی بی‌نیرویی
\\
جز ز تویی تو بگو چیست که ملک تو نیست
&&
چشم بدت دور باد چشم بد بدبویی
\\
لولو حسن ترا در ستد و داد عشق
&&
به ز سنایی مباد خود بر تو لولویی
\\
\end{longtable}
\end{center}
