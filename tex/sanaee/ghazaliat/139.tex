\begin{center}
\section*{غزل شماره ۱۳۹: ترا باری چو من گر یار باید}
\label{sec:139}
\addcontentsline{toc}{section}{\nameref{sec:139}}
\begin{longtable}{l p{0.5cm} r}
ترا باری چو من گر یار باید
&&
ازین به مر مرا تیمار باید
\\
اگر بیمار باشد ور نباشد
&&
مر این دل را یکی دلدار باید
\\
اگر ممکن نباشد وصل باری
&&
بسالی در یکی دیدار باید
\\
بیازردی مرا وانگه تو گویی
&&
چه کردی کز منت آزار باید
\\
مرا گویی که بیداری همه شب
&&
دو چشم عاشقان بیدار باید
\\
چو من وصل جمال دوست جویم
&&
مرا دیده پر از زنگار باید
\\
چه کردی بستدی آن دل کز آن دل
&&
مرا در عشق صد خروار باید
\\
مرا طعنه زنی گویی دلیرا
&&
دلی بستان چرا بیکار باید
\\
دل خسته چه قیمت دارد ای دوست
&&
که چندین با منت گفتار باید
\\
طمع برداشتم از دل ولیکن
&&
مر این جان را یکی زنهار باید
\\
همه خون کرد باید در دل خویش
&&
هر آنکس را که چون تو یار باید
\\
ایا نیکوتر از عمر و جوانی
&&
نکو رو را نکو کردار باید
\\
مرا دیدار تو باید ولیکن
&&
ترا یارا همی دینار باید
\\
مرا دینار بی مهرست رخسار
&&
چنین زر مر ترا بسیار باید
\\
اگر خواهی به خون دل کنی نقش
&&
ولیکن نقش را پرگار باید
\\
\end{longtable}
\end{center}
