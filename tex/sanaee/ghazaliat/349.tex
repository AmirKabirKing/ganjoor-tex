\begin{center}
\section*{غزل شماره ۳۴۹: خنده گریند همی لاف زنان بر در تو}
\label{sec:349}
\addcontentsline{toc}{section}{\nameref{sec:349}}
\begin{longtable}{l p{0.5cm} r}
خنده گریند همی لاف زنان بر در تو
&&
گریه خندند همی سوختگان در بر تو
\\
دل آن روح گسسته که ندارد دل تو
&&
سر آن حور بریده که ندارد سر تو
\\
گاه دشنام زدن طاقچهٔ گوش مرا
&&
حقه‌های شکرین گرد دو تا شکر تو
\\
تا خط تو بدمیدست ز بهر خط تو
&&
حرف بوسست چو لبهای قلم چاکر تو
\\
شیر چرخت ز پی آب همی سجده برد
&&
من چه سگ باشم تا خاک بوم بر در تو
\\
نیست در چنبر نه چرخ یکی پروین بیش
&&
هست پروین کده ره چنبری از عنبر تو
\\
عنبر از چنبر زلفت چو خرد یافته‌ام
&&
تا مگر راه دهد سوی خودم چنبر تو
\\
سیم در سنگ بسی باشد لیک اندر کان
&&
سنگ در سیم دل تست پس اندر بر تو
\\
عارم این بس که بوم پیش‌رو دشمن تو
&&
فخرم آن بس که بوم رخت کش لشگر تو
\\
برده شد ز آتش تو پیش سراپردهٔ جان
&&
آب حیوان روان ز آن دو رده گوهر تو
\\
قطب گردم چو بگردم ز پی خدمت تو
&&
پای بر جای چو پرگار به گرد سر تو
\\
شمع نور فلکی خواهد هر لحظه همی
&&
شعله از مشعلهٔ روی ضیاگستر تو
\\
ز آرزوی رخ چون ماه تو هر روز چو صبح
&&
دل همی چاک زند پیش درت کهتر تو
\\
خور گردون چو مه از پیش رخت کاست کند
&&
که ندارد خود گردون فری اندر خور تو
\\
از سنایی به بها هر دم صد جان خواهد
&&
بهر یک بوسه دو تا بسد جان پرور تو
\\
\end{longtable}
\end{center}
