\begin{center}
\section*{غزل شماره ۴۵: تا خیال آن بت قصاب در چشم منست}
\label{sec:045}
\addcontentsline{toc}{section}{\nameref{sec:045}}
\begin{longtable}{l p{0.5cm} r}
تا خیال آن بت قصاب در چشم من است
&&
زین سبب چشمم همیشه همچو رویش روشن است
\\
تا بدیدم دامن پر خونش چشم من ز اشگ
&&
بر گریبان دارم آنچ آن ماه را بر دامن است
\\
جای دارد در دل پر خونم آن دلبر مقیم
&&
جامه پر خون باشد آن کس را که در خون مسکن است
\\
با من از روی طبیعت گر نیامیزد رواست
&&
از برای آنکه من در آب و او در روغن است
\\
گر زبان با من ندارد چرب هم نبود عجب
&&
کانچه او را در زبان بایست در پیراهن است
\\
جان آرامش همی بخشد جهانی را به لطف
&&
گر چه کارش همچو گردون کشتن‌ست و بستن است
\\
از طریق خاصیت بگریزد از آهن پری
&&
آن پریروی از شگرفی روز و شب با آهن است
\\
هر غمی را او ز من جانی به دل خواهد همی
&&
پس بدین قیمت مر او را یک جهان جان بر من است
\\
ترسم آن آرام دل با من نگردد رام از آنک
&&
کودکی بس تند خوی و کره‌ای بس توسن است
\\
بر وصالش دل همی نتوان نهاد از بهر آنک
&&
گر مرا روزی ازو سورست سالی شیون است
\\
هر چه زان خورشید رو آید همه دادست و عدل
&&
جور ما زین گنبد فیروزهٔ بی روزن است
\\
هر زمان هجران نو زاید جهان از بهر من
&&
خود جهان گویی به هجر عاشقان آبستن است
\\
جامه‌های جان همی دوزم ز وصلش تا مرا
&&
تن چو تار ریسمان و دل چو چشم سوزن است
\\
از پس هجران فراوان چون ندیدم در رهش
&&
آن بتی را کافت آفاق و فتنهٔ برزن است
\\
گفتم ای جان از پی یک وصل چندین هجر چیست
&&
گفت من قصابم اینجا گرد ران با گردن است
\\
گر چه باشد با سنایی چون گل رعنا دو روی
&&
در ثنای او سنایی ده زبان چون سوسن است
\\
\end{longtable}
\end{center}
