\begin{center}
\section*{غزل شماره ۱۷۵: زلف چون زنجیر و چون قیر ای پسر}
\label{sec:175}
\addcontentsline{toc}{section}{\nameref{sec:175}}
\begin{longtable}{l p{0.5cm} r}
زلف چون زنجیر و چون قیر ای پسر
&&
یک زمان از دوش برگیر ای پسر
\\
زان که تا در بند و زنجیر توایم
&&
از در بندیم و زنجیر ای پسر
\\
عرصه تا کی کرد خواهی عارضین
&&
چون گل بی خار بر خیز ای پسر
\\
هر زمان آیی به تیر انداختن
&&
هم کمان در دست و هم تیر ای پسر
\\
زان که چشم بد بدان عارض رسد
&&
زود در ده بانگ تکبیر ای پسر
\\
آن لب و دندان و آن شیرین زبان
&&
انگبین‌ست و می و شیر ای پسر
\\
جست نتواند دل از عشق تو هیچ
&&
جست که تواند ز تقدیر ای پسر
\\
پای بفشارد سنایی در غمت
&&
تا به دست آیی به تدبیر ای پسر
\\
\end{longtable}
\end{center}
