\begin{center}
\section*{غزل شماره ۳۲۴: غلاما خیز و ساقی را خبر کن}
\label{sec:324}
\addcontentsline{toc}{section}{\nameref{sec:324}}
\begin{longtable}{l p{0.5cm} r}
غلاما خیز و ساقی را خبر کن
&&
که جیش شب گذشت و باده در کن
\\
چو مستان خفته انداز بادهٔ شام
&&
صبوحی لعلشان صبح و سحر کن
\\
به باغ صبح در هنگام نوروز
&&
صبایی کرد و بر گلبن نظر کن
\\
جهان فردوس‌وش کن از نسیمی
&&
ز بوی گل به باغ اندر اثر کن
\\
ز بهر آبروی عاشقان را
&&
خرد را در جهان عشق خر کن
\\
صفا را خاوری سازش ز رفعت
&&
نشانرا در کسوفش باختر کن
\\
برآی از خاور طاعات عارف
&&
پس اندر اختر همت نظر کن
\\
چو گردون زینت از زنجیر زر ساز
&&
چو جوزا همت از تیغ کمر کن
\\
از آن آغاز آغاز دگر گیر
&&
وز آن انجام انجام دگر کن
\\
چو عشقش بلبلست از باغ جانت
&&
روان و عقل را شاخ شجر کن
\\
اگر خواهی که بر آتش نسوزی
&&
چو ابراهیم قربان از پسر کن
\\
ورت باید که سنگ کعبه سازی
&&
چو اسماعیل فرمان پدر کن
\\
برآمد سایه از دیوار عمرت
&&
سبک چون آفتاب آهنگ در کن
\\
برو تا درگه دیر و خرابات
&&
حریفی گرد و با مستان خطر کن
\\
چو بند و دام دیدی زود آنگه
&&
دف و دفتر بگیر از می حذر کن
\\
اگر اعقاب حسنت ره بگیرد
&&
سبک دفتر سلاح و دف سپر کن
\\
وگر خواهی که پران گردی از روی
&&
ز جان همچون سنایی شاهپر کن
\\
\end{longtable}
\end{center}
