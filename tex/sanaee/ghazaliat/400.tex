\begin{center}
\section*{غزل شماره ۴۰۰: اگر در کوی قلاشی مرا یکبار بارستی}
\label{sec:400}
\addcontentsline{toc}{section}{\nameref{sec:400}}
\begin{longtable}{l p{0.5cm} r}
اگر در کوی قلاشی مرا یکبار بارستی
&&
مرا بر دل درین عالم همه دشخوار خوارستی
\\
ار این ناسازگار ایام با من سازگارستی
&&
سرو کارم همیشه با می و ورد و قمارستی
\\
اگر نه محنت این نامساعد روزگارستی
&&
مرا با زهد و قرایی و مستوری چکارستی
\\
اگر در پارسایی خود مرا او را دوستارستی
&&
سنایی را به ماه نو نسیم نوبهارستی
\\
هرانکو در دلست او را کنون اندر کنارستی
&&
دلش همواره شادستی و کارش چون نگارستی
\\
دلیل صدق او دایم سنایی را بهارستی
&&
نهان وصل او دایم بر او آشکارستی
\\
اگر از غم دل مسکین عاشق را قرارستی
&&
جهنم پیش چشم سر سریر شهریارستی
\\
گل از هجران اقطارش میان کارزارستی
&&
دل از امید دیدارش میان مرغزارستی
\\
مرا هفتم درک با او بدان دارالقرارستی
&&
سماوات العلی بی او حمیم هفت نارستی
\\
چرا گویی سنایی این گر او را خود شکارستی
&&
ز دست سینهٔ کبک دری او را در آرستی
\\
اگر شخص سنایی را جهان سفله یارستی
&&
چو دیگر مدبران دایم به گردون بر سوارستی
\\
\end{longtable}
\end{center}
