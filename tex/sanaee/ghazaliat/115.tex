\begin{center}
\section*{غزل شماره ۱۱۵: چه رنگهاست که آن شوخ دیده نامیزد}
\label{sec:115}
\addcontentsline{toc}{section}{\nameref{sec:115}}
\begin{longtable}{l p{0.5cm} r}
چه رنگهاست که آن شوخ دیده نامیزد
&&
که تا مگر دلم از صحبتش بپرهیزد
\\
گهی ز طیره گری نکته‌ای دراندازد
&&
گهی به بلعجبی فتنه‌ای برانگیزد
\\
به هیچ وقت به بازی کرشمه‌ای نکند
&&
که صد هزار دل از غمزه درنیاویزد
\\
گهی کزو به نفورم بر من آید زود
&&
گهش چو خوانم با من به قصد بستیزد
\\
ز بهر خصم همی سرمه سازد از دیده
&&
چو دود یافت ز بهر سنایی آمیزد
\\
خبر ندارد از آن کز بلاش نگریزم
&&
که هیچ تشنه ز آب فرات نگریزد
\\
هزار شربت زهر ار ز دست او بخورم
&&
ز عشق نعرهٔ «هل من مزید» برخیزد
\\
نه از غمست که چشمم همی ز راه مژه
&&
هزار دریا پالونه‌وار می‌بیزد
\\
به هر که مردم چشمم نگه کند جز از او
&&
جنایتی شمرد آب ازان سبب ریزد
\\
جواب آن غزل خواجه بو سعید است این
&&
«مرا دلیست که با عافیت نیامیزد»
\\
\end{longtable}
\end{center}
