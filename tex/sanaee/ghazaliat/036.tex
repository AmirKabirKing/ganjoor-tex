\begin{center}
\section*{غزل شماره ۳۶: بر دوزخ هم کفر و هم ایمان تراست}
\label{sec:036}
\addcontentsline{toc}{section}{\nameref{sec:036}}
\begin{longtable}{l p{0.5cm} r}
بر دوزخ هم کفر و هم ایمان تراست
&&
بر دو لب هم درد و هم درمان تراست
\\
گر دو صد یعقوب داری زیبدت
&&
کانچه یوسف داشت صد چندان تراست
\\
خندهٔ تو چون دم عیساست کو
&&
هر چه در لب داشت در دندان تراست
\\
چند گویی کان و کان یک ره ببین
&&
کانچه در کانست در ارکان تراست
\\
چند گویی جان و جان یک دم بخند
&&
کانچه در جانست در مرجان تراست
\\
از لطیفی آنت جان خواند از آنک
&&
هر چه آنرا خواند جان بتوان تراست
\\
هر زمان گویی همی چوگان من
&&
گوی از آن کیست گر چوگان تراست
\\
چون همی دانی که میدان آن تست
&&
گوی هم می دان که در میدان تراست
\\
بنده گر خوبست گر زشت آن تست
&&
عاشق ار دانا و گر نادان تراست
\\
صورت ار با تو نباشد گو مباش
&&
خاک بر سر جسم را چون جان تراست
\\
من ترا هرگز بنگذارم ولیک
&&
گر تو بگذاری مرا فرمان تراست
\\
هیچ مرغ آسان سنایی را نیافت
&&
دولتی مرغی که این آسان‌تر است
\\
\end{longtable}
\end{center}
