\begin{center}
\section*{غزل شماره ۴: باز بر عاشق فروش آن سوسن آزاد را}
\label{sec:004}
\addcontentsline{toc}{section}{\nameref{sec:004}}
\begin{longtable}{l p{0.5cm} r}
باز بر عاشق فروش آن سوسن آزاد را
&&
باز بر خورشید پوش آن جوشن شمشاد را
\\
باز چون شاگرد مومن در پس تخته نشان
&&
آن نکو دیدار شوخ کافر استاد را
\\
ناز چون یاقوت گردان خاصگان عشق را
&&
در میان بحر حیرت لولو فریاد را
\\
خویشتن بینان ز حسنت لافگاهی ساختند
&&
هین ببند از غمزه درها کوی عشق آباد را
\\
هر چه بیدادست بر ما ریز کاندر کوی داد
&&
ما به جان پذرفته‌ایم از زلف تو بیداد را
\\
گیرم از راه وفا و بندگی یک سو شویم
&&
چون کنیم ای جان بگو این عشق مادرزاد را
\\
زین توانگر پیشگان چیزی نیفزاید ترا
&&
کز هوس بردند بر سقف فلک بنیاد را
\\
قدر تو درویش داند ز آنکه او بیند مقیم
&&
همچو کرکس در هوا هفتاد در هفتاد را
\\
خوش کن از یک بوسهٔ شیرین‌تر از آب حیات
&&
چو دل و جان سنایی طبع فرخ‌زاد را
\\
\end{longtable}
\end{center}
