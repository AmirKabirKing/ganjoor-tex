\begin{center}
\section*{غزل شماره ۷۴: ای مستان خیزید که هنگام صبوحست}
\label{sec:074}
\addcontentsline{toc}{section}{\nameref{sec:074}}
\begin{longtable}{l p{0.5cm} r}
ای مستان خیزید که هنگام صبوحست
&&
هر دم که درین حال زنی دام فتوحست
\\
آراست همه صومعه مریم که دم صبح
&&
صاحبت خبر گلشن و نزهتگه روحست
\\
یک مطربتان عقل و دگر مطرب عشقست
&&
یک ساقیتان حور و دگر ساقی روحست
\\
طوفان بلا از چپ و از راست برآمد
&&
در باده گریزید که آن کشتی نوحست
\\
باده که درین وقت خوری باده مباحست
&&
تو به که درین وقت کنی توبه نصوحست
\\
خود روز همه نوبت تن خواهد بود
&&
هین راح که این یک دودمک نوبت روحست
\\
وز می خوش خسب گزین صبح سنایی
&&
تا صبح قیامت بدمد مرد صبوحست
\\
\end{longtable}
\end{center}
