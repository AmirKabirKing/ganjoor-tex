\begin{center}
\section*{غزل شماره ۷۷: دوش رفتم به سر کوی به نظارهٔ دوست}
\label{sec:077}
\addcontentsline{toc}{section}{\nameref{sec:077}}
\begin{longtable}{l p{0.5cm} r}
دوش رفتم به سر کوی به نظارهٔ دوست
&&
شب هزیمت شده دیدم ز دو رخسارهٔ دوست
\\
از پی کسب شرف پیش بناگوش و لبش
&&
ماه دیدم رهی و زهره سما کارهٔ دوست
\\
گوشها گشته شکر چین که همی ریخت ز نطق
&&
حرفهای شکرین از دو شکر پارهٔ دوست
\\
چشمهای همه کس گشته تماشاگه جان
&&
نز پی بلعجبی از پی نظارهٔ دوست
\\
پیش یکتا مژهٔ چشم چو آهوش ز ضعف
&&
شده شیران جهان ریشه‌ای از شارهٔ دوست
\\
کرده از شکل عزب خانهٔ زنبور از غم
&&
دل عشاق جهان غمزهٔ خونخوارهٔ دوست
\\
هر زمان مدعی را ز غرور دل خویش
&&
تازه خونی حذر اندر خم هر تارهٔ دوست
\\
چون به سیاره شدی از پی چندین چو فلک
&&
از ستاره شده آراسته سیارهٔ دوست
\\
لب نوشینش بهم کرده بر نظم بقاش
&&
داد نوشروان با چشم ستمگارهٔ دوست
\\
دوش روزیم پدید آمده از تربیتش
&&
بازم امروز شبی از غم بی‌غارهٔ دوست
\\
چه کند قصه سنایی که ز راه لب و زلف
&&
یک جهان دیده پر آوازهٔ آوارهٔ دوست
\\
هست پروارهٔ او را رهی از بام فلک
&&
همت شاه جهان ساکن پروارهٔ دوست
\\
شاه بهرامشه آن شه که همیشه کف او
&&
سبب آفت دشمن بود و چارهٔ دوست
\\
زخم و رحم و بد و نیکش ز ره کون و فساد
&&
تا ابد رخنهٔ دشمن بود و یارهٔ دوست
\\
\end{longtable}
\end{center}
