\begin{center}
\section*{غزل شماره ۴۰۹: دلم بردی و جان بر کار داری}
\label{sec:409}
\addcontentsline{toc}{section}{\nameref{sec:409}}
\begin{longtable}{l p{0.5cm} r}
دلم بردی و جان بر کار داری
&&
تو خود جای دگر بازار داری
\\
نباشد عاشقت هرگز چو من کس
&&
اگر چه عاشق بسیار داری
\\
ز رنج غیرتت بیمار باشم
&&
چو تو با دیگران دیدار داری
\\
عزیزت خوانم ای جان جهانم
&&
از آنست کین چنینم خوار داری
\\
کسی کو عاشق روی تو باشد
&&
سزد او را نزار و زار داری
\\
دو چشمم هر شبی تا بامدادان
&&
ز هجر خویشتن بیدار داری
\\
شدم مهجور و رنجور تو زیراک
&&
تو خوی عالم غدار داری
\\
ترا دارم عزیز ای ماه چون گل
&&
چرا بی‌قیمتم چون خار داری
\\
نگر تا کی مرا از داغ هجران
&&
لبی خشک و دلی پر نار داری
\\
تو خود تنها جهان را می بسوزی
&&
چرا بر خود بلا را یار داری
\\
بکن رحمی بدین عاشق اگر هیچ
&&
امید رحمت جبار داری
\\
سنایی را چنان باید کزین پس
&&
ز وصل خویش بر خوردار داری
\\
\end{longtable}
\end{center}
