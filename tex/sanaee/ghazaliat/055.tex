\begin{center}
\section*{غزل شماره ۵۵: نرگسین چشما به گرد نرگس تو تیر چیست}
\label{sec:055}
\addcontentsline{toc}{section}{\nameref{sec:055}}
\begin{longtable}{l p{0.5cm} r}
نرگسین چشما به گرد نرگس تو تیر چیست
&&
وان سیاهی اندرو پیوسته همچون قیر چیست
\\
گر سیاهی نیست اندر نرگس تو گرد او
&&
آن سیه مژگان زهرآلود همچون تیر چیست
\\
گر شراب و شیر خواهی ریخته بر ارغوان
&&
پنجه‌های دست رنگین پر شراب و شیر چیست
\\
گر مثال دست شاه زنگ دارد زلف تو
&&
پس دو دست شاه زنگی بسته در زنجیر چیست
\\
آیتی بنبشته‌ای گرد لب یاقوت رنگ
&&
اندر آن آیت بگو تا معنی و تفسیر چیست
\\
دل ترا دادم توکل بر خدای دادگر
&&
روی کردم سوی تو تا بر سرم تقدیر چیست
\\
مر مر اگر کشته خواهی پس بکش یکبارگی
&&
من کیم در کشتن من این همه تدبیر چیست
\\
مر مرا چون زیر کردی در فراق روی خویش
&&
وانگهی گویی خروش و نالهٔ چون زیر چیست
\\
ای سنایی در فراقش صابری را پیشه گیر
&&
جز صبوری کردن اندر عاشقی تدبیر چیست
\\
\end{longtable}
\end{center}
