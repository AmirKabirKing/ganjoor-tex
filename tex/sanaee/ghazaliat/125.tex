\begin{center}
\section*{غزل شماره ۱۲۵: وصال حالت اگر عاشقی حلال کند}
\label{sec:125}
\addcontentsline{toc}{section}{\nameref{sec:125}}
\begin{longtable}{l p{0.5cm} r}
وصال حالت اگر عاشقی حلال کند
&&
فراق عشق همه حالها زوال کند
\\
وصال جستن عاشق نشان بی‌خبریست
&&
که نه ره همهٔ عاشقان وصال کند
\\
رهیست عشق کشیده میان درد و دریغ
&&
طلب در او صفت بی خودی مثال کند
\\
نصیب خلق یکی خندقی پر از شهوت
&&
در او مجاز و حقیقت همی جدال کند
\\
چو از نصیب گذشتی روا بود که دلت
&&
حدیث دلبر و دعوی زلف و خال کند
\\
چو آفتاب رخش محترق شود ز جمال
&&
نقاب بندد بعضی ازو هلال کند
\\
نگار من چو شب از گرد مه درآلاید
&&
حرام خون هزاران چو من حلال کند
\\
نگه نیارم کردن به رویش از پی آن
&&
که جان ز تن به ره دیده ارتحال کند
\\
کمال حال ز عشاق خویش نقص کند
&&
بتم چو خوبی بی‌نقص را کمال کند
\\
وصال او به زمانی هزار روز کند
&&
فراق او ز شبی صد هزار سال کند
\\
هزار آیت دل بردنست یار مرا
&&
ز من هر یکیش طبایع دو صد جمال کند
\\
چو او سوار شود سرو را پیاده کند
&&
چو غمزه سازد هاروت را نکال کند
\\
حدیث در دهن او تو گوییی که مگر
&&
وجود با عدم از لذت اتصال کند
\\
گمان بری که سیه زلف او بر آن رخ او
&&
یکی شبست که با روز او جدال کند
\\
زهی بتی که به خوبی خویش در نفسی
&&
هزار عاشق چون من فر و جوال کند
\\
هزار صومعه ویران کند به یک ساعت
&&
چو حلقه‌های سر زلف جیم و دال کند
\\
تبارک‌الله از آن روی پر ملاحت و زیب
&&
که غایت همه عشاق قیل و قال کند
\\
\end{longtable}
\end{center}
