\begin{center}
\section*{غزل شماره ۲۹۶: عاشقی گر خواهد از دیدار معشوقی نشان}
\label{sec:296}
\addcontentsline{toc}{section}{\nameref{sec:296}}
\begin{longtable}{l p{0.5cm} r}
عاشقی گر خواهد از دیدار معشوقی نشان
&&
گر نشان خواهی در آنجا جان و دل بیرون نشان
\\
چون مجرد گشتی و تسلیم کردستی تو دل
&&
بی گمان آنگه تو از معشوق خود یابی نشان
\\
چون ز خود بی‌خود شدی معشوق خود را یافتی
&&
ذات هستی در نشان نیستی دیدن توان
\\
نیستی دیدی که هستی را همیشه طالبست
&&
نیستی جوینده را هستی کم اندر کهکشان
\\
تا همی جویم بیابم چون بیابم گم شوم
&&
گمشده گمکرده را هرگز کجا بیند عیان
\\
چون تو خود جویی مر او را کی توانی یافتن
&&
تا نبازی هر چه داری مال و ملک و جسم و جان
\\
آنگهی چون نفی خود دیدی و گشتی بی‌ثبات
&&
گه فنا و گه بقا و گه یقین و گه گمان
\\
گه تحرک گه سکون و گاه قرب و گاه بعد
&&
گاه گویا گه خموشی گه نشستی گه روان
\\
گه سرور و گه غرور و گه حیات و گه ممات
&&
گه نهان و گه عیان و گه بیان و گه بنان
\\
حیرت اندر حیرتست و آگهی در آگهی
&&
عاجزی در عاجزی و اندهان در اندهان
\\
هر که ما را دوست دارد عاجز و حیران بود
&&
شرط ما اینست اندر دوستی دوستان
\\
\end{longtable}
\end{center}
