\begin{center}
\section*{غزل شماره ۱۸۴: دلبر من عین کمالست و بس}
\label{sec:184}
\addcontentsline{toc}{section}{\nameref{sec:184}}
\begin{longtable}{l p{0.5cm} r}
دلبر من عین کمالست و بس
&&
چهرهٔ او اصل جمالست و بس
\\
بر سر کوی غم او مرد را
&&
هر چه نشانست و بالست و بس
\\
در ره او جستن مقصود از او
&&
هم به سر او که محالست و بس
\\
از همه خوبی که بجویی ز دوست
&&
بوسه ای از دوست حلالست و بس
\\
چند همی پرسی دین تو چیست
&&
دین من امروز سوالست و بس
\\
نزد تو اقبال دوامست و عز
&&
نزد من اقبال زوالست و بس
\\
حالی یابم چو کنم یاد ازو
&&
دین من آن ساعت حالست و بس
\\
پرده منم پیش چو برخاستم
&&
از پس آن پرده وصالست و بس
\\
\end{longtable}
\end{center}
