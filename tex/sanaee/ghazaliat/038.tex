\begin{center}
\section*{غزل شماره ۳۸: ای صنم در دلبری هم دست و هم دستان تراست}
\label{sec:038}
\addcontentsline{toc}{section}{\nameref{sec:038}}
\begin{longtable}{l p{0.5cm} r}
ای صنم در دلبری هم دست و هم دستان تراست
&&
بر دل و جان پادشاهی هم دل و هم جان تراست
\\
هم حیات از لب نمودن هم شفا از رخ چو حور
&&
با دم عیسی و دست موسی عمران تراست
\\
در سر زلف نشان از ظلمت اهریمنست
&&
بر دو رخ از نور یزدان حجت و برهان تراست
\\
ای چراغ دل نمی‌دانی که اندر وصل و هجر
&&
دوزخ بی مالک و فردوس بی رضوان تراست
\\
در میان اهل دین و اهل کفر این شور چیست
&&
گر مسلم بر دو رخ هم کفر و هم ایمان تراست
\\
از جمال و از بهایت خیره گردد سرو و مه
&&
سرو بستانی تو داری ماه بی کیوان تراست
\\
آنچه بت‌گر کرد و جادو دید جانا باطل است
&&
در دو مرجان و دو نرگس کار این و آن تراست
\\
گر من از حواری جنت یاد نارم شایدم
&&
کانچه حورالعین جنت داشت صد چندان تراست
\\
از همه خوبان عالم گوی بردی شاد باش
&&
داوری حاجت نیاید ای صنم فرمان تراست
\\
در همه جایی سنایی چاکر و مولای تست
&&
گر برانی ور بخوانی ای صنم فرمان تراست
\\
این چنین صیدی که در دام تو آمد کس ندید
&&
گوی گردون بس که اکنون نوبت میدان تراست
\\
\end{longtable}
\end{center}
