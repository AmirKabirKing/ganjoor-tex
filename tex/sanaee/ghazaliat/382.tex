\begin{center}
\section*{غزل شماره ۳۸۲: گر بگویی عاشقی با ما هم از یک خانه‌ای}
\label{sec:382}
\addcontentsline{toc}{section}{\nameref{sec:382}}
\begin{longtable}{l p{0.5cm} r}
گر بگویی عاشقی با ما هم از یک خانه‌ای
&&
با همه کس آشنا با ما چرا بیگانه‌ای
\\
ما چو اندر عشق تو یکرویه چون آیینه‌ایم
&&
تو چرا در دوستی با ما دو سر چون شانه‌ای
\\
شمع خود خوانی همی ما را و ما در پیش تو
&&
پس ترا پروای جان از چیست گر پروانه‌ای
\\
جز به عمری در ره ما راست نتوان رفت از آنک
&&
همچو فرزین کجروی در راه نافرزانه‌ای
\\
عاشقی از بند عقل و عافیت جستن بود
&&
گر چنینی عاشقی ور نیستی دیوانه‌ای
\\
زان ز وصل ما نداری یکدم آسایش که تو
&&
روز و شب سودای خود رانی دمی مارا نه‌ای
\\
یارت ای بت صدر دارد زان عزیزست و تو زان
&&
در لگد کوب همه خلقی که در استانه‌ای
\\
هر کجا صحراست گرم و روشنست از آفتاب
&&
تو از آن در سایه ماندستی که اندر خانه‌ای
\\
تو برای ما به گرد دام ما گردی ولیک
&&
دام ما را دانه‌ای هست و تو مرد دانه‌ای
\\
بر خودی عاشق نه بر ما ای سنایی بهر آنک
&&
روز و شب مرد فسون و شعبده و افسانه‌ای
\\
\end{longtable}
\end{center}
