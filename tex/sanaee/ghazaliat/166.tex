\begin{center}
\section*{غزل شماره ۱۶۶: غریبیم چون حسنت ای خوش پسر}
\label{sec:166}
\addcontentsline{toc}{section}{\nameref{sec:166}}
\begin{longtable}{l p{0.5cm} r}
غریبیم چون حسنت ای خوش پسر
&&
یکی از سر لطف بر ما نگر
\\
سفر داد ما را چو تو تحفه‌ای
&&
زهی ما بر تو غلام سفر
\\
نظرمان مباد از خدای ار به تو
&&
جز از روی پاکیست ما را نظر
\\
دل تنگ ما معدن عشق تست
&&
که هم خردی و هم عزیزی چو زر
\\
هنوز از نهالت نرسته‌ست گل
&&
هنوز از درختت نپختست بر
\\
ببندد به عشق تو حورا میان
&&
گشاید ز رشگ تو جوزا کمر
\\
نباشد کم از ناف آهو به بوی
&&
کرا عشق زلف تو سوزد جگر
\\
نگارا ز دشنام چون شکرت
&&
که دارد ز گلبرگ سوری گذر
\\
عجب نیست گر ما قوی دل شدیم
&&
که این خاصیت هست در نیشکر
\\
بینداز چندان که خواهی تو تیر
&&
که ما ساختیم از دل و جان سپر
\\
تو بر ما به نادانی و کودکی
&&
چو متواریان کرده‌ای رهگذر
\\
بدین اتفاقی که ما را فتاد
&&
مکن راز ما پیش یاران سمر
\\
مدر پردهٔ ما که در عشق تو
&&
شدست این سنایی ز پرده به در
\\
که از روی نسبت نیاید نکو
&&
پدر پرده‌دار و پسر پرده‌در
\\
دل و جان و عقل سناییت را
&&
ربودی بدان غمزهٔ دل شکر
\\
\end{longtable}
\end{center}
