\begin{center}
\section*{غزل شماره ۲۱۰: تا به بستانم نشاندی بر بساط انبساط}
\label{sec:210}
\addcontentsline{toc}{section}{\nameref{sec:210}}
\begin{longtable}{l p{0.5cm} r}
تا به بستانم نشاندی بر بساط انبساط
&&
ناگهانم در برآوردی و ماندی در بساط
\\
برگشاد از قهر و لطف لشکر قهرت کمین
&&
تا به دلها درنگون شد رایت انس و نشاط
\\
من ز بهر دوستی را جان و دل کردم سبیل
&&
تا بوم کارم جهاد و تا زیم شغلم رباط
\\
اختلاط عشق تو با جان من باشد همی
&&
تا بود خون مرا با خاک روزی اختلاط
\\
در سرای دوستی آن به که فرشی افگنم
&&
خشت او باشد ز جان و خون او باشد ملاط
\\
تا اگر باری نباشم بر بساط دوستان
&&
خاک باشم زیر پای چاکران اندر سماط
\\
احتیاط و حزم کردم در بلا و درد عشق
&&
تیغ تقدیر آمد و شد پاک حزم و احتیاط
\\
ره ندانم جز به لطفت گر کنی لطفی سزاست
&&
ره نداند جو به پستان طفل خرد اندر قماط
\\
هر که بگذارد صراط آید به درگاه بهشت
&&
من نمی‌بینم بهشت و بیش رفتم صد صراط
\\
از دل آمد بر سنایی کس مباد اندر جهان
&&
گر نماند بر بساط قرب شاهان بی نشاط
\\
\end{longtable}
\end{center}
