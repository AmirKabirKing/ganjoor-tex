\begin{center}
\section*{غزل شماره ۲۹۵: ای وصل تو دستگیر مهجوران}
\label{sec:295}
\addcontentsline{toc}{section}{\nameref{sec:295}}
\begin{longtable}{l p{0.5cm} r}
ای وصل تو دستگیر مهجوران
&&
هجر تو فزود عبرت دوران
\\
هنگام صبوح و تو چنین غافل
&&
حقا که نه‌ای بتا ز معذوران
\\
گر فوت شود همی نماز از تو
&&
بندیش به دل بسوز رنجوران
\\
برخیز و بیار آنچه زو گردد
&&
چون توبهٔ من خمار مخموران
\\
فریاد ز دست آن گران جانان
&&
بی عافیه زاهدان و بی‌نوران
\\
از طلعتها چو روی عفریتان
&&
از سبلتها چو نیش زنبوران
\\
گویند بکوش تا به مستوری
&&
در شهر شوی چو ما ز مشهوران
\\
نزدیکی ما طلب کن ای مسکین
&&
تا روز قضا نباشی از دوران
\\
لا والله اگر من این کنم هرگز
&&
بیزارم از جزای ماجوران
\\
معلوم شما نیست ز نادانی
&&
ای زمرهٔ زاهدان مغروران
\\
آنجا که مصیر ما بود فردا
&&
بی‌رنج دهند مزد مزدوران
\\
\end{longtable}
\end{center}
