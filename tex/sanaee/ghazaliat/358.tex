\begin{center}
\section*{غزل شماره ۳۵۸: ای گشته ز تابش صفای تو}
\label{sec:358}
\addcontentsline{toc}{section}{\nameref{sec:358}}
\begin{longtable}{l p{0.5cm} r}
ای گشته ز تابش صفای تو
&&
آیینهٔ روی ما قفای تو
\\
بادست به دست آب و آتش را
&&
با صفوت و نور خاکپای تو
\\
با تو چه کند رقیب تاریکت
&&
بس نیست رقیب تو ضیای تو
\\
خود قاف ز هم همی فرو ریزد
&&
از سایهٔ کاف کبریای تو
\\
در کوی تو من کدام سگ باشم
&&
تا لاف زنم ز روی و رای تو
\\
هر چند که خوش نیایدت هل تا
&&
لافی بزند ز تو گدای تو
\\
این هژده هزار عالم و آدم
&&
نابوده بهای یک بهای تو
\\
قیمت گر تو حسود بود ای جان
&&
زان هژده قلب شد بهای تو
\\
ای راحت تو همه فنای ما
&&
وی شادی ما همه بقای تو
\\
هم دوست همی کشی و هم دشمن
&&
چه خشک و چه تر در آسیای تو
\\
ایندست که مر تراست در شوخی
&&
اندر دو جهان کراست پای تو
\\
دیریست که هر زمان همی کوبند
&&
این دبدبه بر در سرای تو
\\
من بندهٔ زندگانی خویشم
&&
لیکن نه برای خود برای تو
\\
هر چند نیافت اندرین مدت
&&
یک شعله سنایی از سنای تو
\\
با اینهمه هست بر زبان نونو
&&
شهری و سنایی و ثنای تو
\\
\end{longtable}
\end{center}
