\begin{center}
\section*{غزل شماره ۶۸: عشق ازین معشوقگان بی وفا دل بر گرفت}
\label{sec:068}
\addcontentsline{toc}{section}{\nameref{sec:068}}
\begin{longtable}{l p{0.5cm} r}
عشق ازین معشوقگان بی وفا دل بر گرفت
&&
دست ازین مشتی ریاست جوی دون بر سر گرفت
\\
عالم پر گفتگوی و در میان دردی ندید
&&
از در سلمان در آمد دامن بوذر گرفت
\\
اینت بی همت که در بازار صدق و معرفت
&&
روی از عیسا بگردانید و سم خر گرفت
\\
سامری چون در سرای عافیت بگشاد لب
&&
از برای فتنه را شاگردی آزر گرفت
\\
نان اسکندر خوری و خدمت دارا کنی
&&
خاک سیم از حرص پنداری که آب زر گرفت
\\
بلعجب بازیست در هنگام مستی با فقر
&&
کز میان خشک رودی ماهیان تر گرفت
\\
سالها مجنون طوافی کرد در کهسار دوست
&&
تا شبی معشوقه را در خانه بی مادر گرفت
\\
آنچه از مستی و کوتاهی شبی آهنگ کرد
&&
تا سر زلفش نگیرد زود ازو سر بر گرفت
\\
خواجه از مستی شبی بر پای چاکر بوسه داد
&&
تا نه پنداری که چاکر قیمت دیگر گرفت
\\
زین عجایب تر که چون دزد از خزینت نقد برد
&&
دیده‌بان کور گوش پاسبان کر گرفت
\\
این مرقعها و این سالوسها و رنگها
&&
امر معروفست کز وی جانها آذر گرفت
\\
دیو بد دینست لیکن بر در دین ره زند
&&
زهر ما زهرست لیکن معدنی شکر گرفت
\\
ای سنایی هان که تا نفریبدت دیو لعین
&&
کز فریب دیو عالم جمله شور و شر گرفت
\\
هر دعا گویی که در شش پنج او دادی به خواب
&&
چون سنایی هفت اختر ره ششدر گرفت
\\
\end{longtable}
\end{center}
