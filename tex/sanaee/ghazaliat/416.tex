\begin{center}
\section*{غزل شماره ۴۱۶: چرا ز روی لطافت بدین غریب نسازی}
\label{sec:416}
\addcontentsline{toc}{section}{\nameref{sec:416}}
\begin{longtable}{l p{0.5cm} r}
چرا ز روی لطافت بدین غریب نسازی
&&
که بس غریب نباشد ز تو غریب نوازی
\\
ز بهر یک سخن تو دو گوش ما سوی آن لب
&&
ستیزه بر دل ما و دو چشم تو سوی بازی
\\
چه آفتی تو که شبها میان دیده چو خوابی
&&
چه فتنه‌ای تو که شبها میان روح چو رازی
\\
چو من ز آتش غیرت نهاد کعبه بسوزم
&&
تو از میان دو ابرو هزار قبله بسازی
\\
پس از فراز نباشد جز از نشیب ولیکن
&&
جهان عشق تو دارد پس از فراز فرازی
\\
گداخت مایهٔ صبرم ز بانگ شکر لفظت
&&
گه عتاب نمودن به پارسی و به تازی
\\
نه آن عجب که شنیدم که صبر نوش گدازد
&&
عجبتر آنکه بدیدم ز نوش صبر گدازی
\\
ز بوسهٔ تو نماید زمانه نامهٔ شاهی
&&
ز غمزهٔ تو فزاید جهان کتاب مغازی
\\
چو موی و روی تو بیند خرد چگوید گوید
&&
زهی دو مومن جادو زهی دو کافر غازی
\\
جمال و جاه سعادت چو یافتی ز زمانه
&&
بناز بر همه خوبان که زیبدت که بنازی
\\
بقا و مال و جمالت همیشه باد چو عشقت
&&
که هیچ عمر ندارد چو عمر عشق درازی
\\
چو شد به نزد سنایی یکی جفا و وفایت
&&
رسید کار به جان و گذشت عمر به بازی
\\
\end{longtable}
\end{center}
