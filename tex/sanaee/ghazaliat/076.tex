\begin{center}
\section*{غزل شماره ۷۶: راه فقرست ای برادر فاقه در وی رفتنست}
\label{sec:076}
\addcontentsline{toc}{section}{\nameref{sec:076}}
\begin{longtable}{l p{0.5cm} r}
راه فقرست ای برادر فاقه در وی رفتنست
&&
وندرین ره نفس کش کافر ز بهر کشتنست
\\
نفس اماره و لوامه‌ست و دیگر ملهمه
&&
مطمئنه با سه دشمن در یکی پیراهنست
\\
خاک و باد و آب و آتش در وجود خود بدان
&&
رو درین معنی نظر کن صدهزاران روزنست
\\
چار نفس و چار طبع و پنج حس و شش جهت
&&
هفت سلطان باده و دو جمله با هم دشمنست
\\
نفس را مرکب مساز و با مراد او مرو
&&
همچو خر در گل بماند گر چه اصلش تو سنست
\\
از در دروازهٔ لا تا به دارالملک شاه
&&
هفهزار و هفصد و هفتاد راه و رهزنست
\\
خواجه دارد چار خواهر مختلف اندر وجود
&&
نام خود را مرد کرده پیش ایشان چون زنست
\\
در شریعت کی روا باشد دو خواهر یک نکاح
&&
در طریقت هر دو را از خود مبرا کردنست
\\
سوزنی را پای بند راه عیسی ساختند
&&
حب دنیا پای بندست ار همه یک سوزنست
\\
هیچ دانی از چه باشد قیمت آزاده مرد
&&
بر سر خوان خسیسان دست کوته کردنست
\\
بر سر کوی قناعت حجره‌ای باید گرفت
&&
نیم نانی می‌رسد تا نیم جانی در تنست
\\
گر ز گلشنها براند ما به گلخنها رویم
&&
یار با ما دوست باشد گلخن ما گلشنست
\\
ای سنایی فاقه و فقر و فقیری پیشه کن
&&
فاقه و فقر و فقیری عاشقان را مسکنست
\\
\end{longtable}
\end{center}
