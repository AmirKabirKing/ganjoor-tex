\begin{center}
\section*{غزل شماره ۴۰۵: ای راه ترا دلیل دردی}
\label{sec:405}
\addcontentsline{toc}{section}{\nameref{sec:405}}
\begin{longtable}{l p{0.5cm} r}
ای راه ترا دلیل دردی
&&
فردی تو و آشنات فردی
\\
از دام تو دانه‌ای و مرغی
&&
در جام تو قطره‌ای و مردی
\\
بی روی تو روح چیست بادی
&&
با زلف تو شخص کیست گردی
\\
خارست همه جهان و آنگه
&&
روی تو در آن میانه وردی
\\
در کوی تو نیست تشنگان را
&&
جز خاک در تو آبخوردی
\\
در راه تو نیست عاشقان را
&&
جز داعیهٔ تو ره‌نوردی
\\
در تو که رسد به دستمزدی
&&
تا از تو نبود پایمردی
\\
در عشق تو خود وفا کی آید
&&
از خشک و تری و گرم و سردی
\\
نیک‌ست که آینه نداری
&&
تا هست شفات نیست دردی
\\
از آینه‌ای بدی به دستت
&&
چشم تو ترا به چشم کردی
\\
در شهر تو نیست جز سنایی
&&
بی‌وصل تو جز که یاوه گردی
\\
\end{longtable}
\end{center}
