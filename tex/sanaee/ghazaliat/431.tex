\begin{center}
\section*{غزل شماره ۴۳۱: ربی و ربک‌الله ای ماه تو چه ماهی}
\label{sec:431}
\addcontentsline{toc}{section}{\nameref{sec:431}}
\begin{longtable}{l p{0.5cm} r}
ربی و ربک‌الله ای ماه تو چه ماهی
&&
کافزون شوی ولیکن هرگز چنو نکاهی
\\
مه نیستی که مهری زیرا که هست مه را
&&
گاه از برونش زردی گاه از درون سیاهی
\\
با مایهٔ جمالت ناید ز مهر شمعی
&&
در سایهٔ سلیمان ناید ز دیو شاهی
\\
آنجا که قدت آید ناید ز سر و سروی
&&
آنجا که خدت آید ناید ز ماه ماهی
\\
از جزع عقل عقلی و ز لعل شمع شمعی
&&
از خنده جان جانی وز غمزه جاه جاهی
\\
هر روز صبح صادق از غیرت جمالت
&&
بر خود همی بدرد پیراهن پگاهی
\\
گرد سم سمندت بر گلشن سمایی
&&
در زلف جعد حوران مشکیست جایگاهی
\\
حقا و ثم حقا آنگه که بزم سازی
&&
روح‌الامین نوازد در مجلست ملاهی
\\
خوشخوتر از تو خویی روح‌القدس ندیدست
&&
از قایل الاهی تا قابل گیاهی
\\
آویختی به عمدا از بهر بند دلها
&&
زنجیر بیگناهان از جای بیگناهی
\\
در جنب آبرویت آدم که بود؟ خاکی
&&
با قدر قد و مویت یوسف که بود چاهی
\\
فراش خاک کویت پاکان آسمانی
&&
قلاش آبرویت پیران خانقاهی
\\
در تابهای زلفت بنگر به خط ابرو
&&
ترغیب اگر ندیدی در صورت مناهی
\\
عقلم همی نداند تفسیر خطت آری
&&
نامحرمی چه داند شرح خط الاهی
\\
در ملک خوبرویی بس نادری ولیکن
&&
نادرتر آنکه داری ملکی به بی‌کلاهی
\\
با خنده و کرشمه آنجا که روی آری
&&
هم ماه و هم سپهری هم شاه و هم سپاهی
\\
آهم شکست در بر ز آن دم که دید چشمم
&&
آن حسن بی‌تباهی و آن لطف بی‌تناهی
\\
ز آن آه بر نیارد زیرا که هست پنهان
&&
آه از درون جانش تو در میان آهی
\\
در جل کشید جانرا در خدمتت سنایی
&&
خواهی کنون بر آن را خواه آن زمان که خواهی
\\
\end{longtable}
\end{center}
