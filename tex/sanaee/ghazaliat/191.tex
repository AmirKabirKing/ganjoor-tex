\begin{center}
\section*{غزل شماره ۱۹۱: ای دل اندر نیستی چون دم زنی خمار باش}
\label{sec:191}
\addcontentsline{toc}{section}{\nameref{sec:191}}
\begin{longtable}{l p{0.5cm} r}
ای دل اندر نیستی چون دم زنی خمار باش
&&
شو بری از نام و ننگ و از خودی بیزار باش
\\
دین و دنیا جمله اندر باز و خود مفلس نشین
&&
در صف ناراستان خود جمله مفلس وار باش
\\
تا کی از ناموس و رزق و زهد و تسبیح و نماز
&&
بندهٔ جام شراب و خادم خمار باش
\\
می پرستی پیشه‌گیر اندر خرابات و قمار
&&
کمزن و قلاش و مست و رند و دردی خوار باش
\\
چون همی دانی که باشد شخص هستی خصم خویش
&&
پس به تیغ نیستی با خلق در پیکار باش
\\
طالب عشق و می و عیش و طرب باش و بجوی
&&
چون به کف آمد ترا این روز و شب در کار باش
\\
با سرود و رود و جام باده و جانان بساز
&&
وز میان جان غلام و چاکر هر چار باش
\\
از سر کوی حقیقت بر مگرد و راه عشق
&&
با غرامت همنشین و با ملامت یار باش
\\
\end{longtable}
\end{center}
