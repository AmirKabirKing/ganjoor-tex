\begin{center}
\section*{غزل شماره ۲۵۴: ای ناگزران عقل و جانم}
\label{sec:254}
\addcontentsline{toc}{section}{\nameref{sec:254}}
\begin{longtable}{l p{0.5cm} r}
ای ناگزران عقل و جانم
&&
وی غارت کرده این و آنم
\\
ای نقش خیال تو یقینم
&&
وی خال جمال تو گمانم
\\
تا با خودم از عدم کمم کم
&&
چون با تو بوم همه جهانم
\\
در بازم با تو خویشتن را
&&
تا با تو بمانم ار بمانم
\\
گویی که به دل چه‌ای چو تیرم
&&
پرسی که به تن کئی کمانم
\\
پیش تو به قلب و قالب ای جان
&&
آنم که چو هر دو حرف آنم
\\
ای شکل و دهان تو کم از نیست
&&
کی بود که کنی کم از دهانم
\\
گر با تو به دوزخ اندر آیم
&&
حقا که بود به از جنانم
\\
تا چند چهار میخ داری
&&
در حجرهٔ تنگ کن فکانم
\\
تا چند فسرده روح داری
&&
در سایهٔ دامن زمانم
\\
بی هیچ بخر مرا هم از من
&&
هر چند برایگان گرانم
\\
مانند میان خود کنم نیست
&&
زیرا که هنوز در میانم
\\
با تن چکنم نه از زمینم
&&
با جان چکنم نه آسمانم
\\
من سایه شدم تو آفتابی
&&
یک راه برآی تا نمانم
\\
بگشای نقاب تا ببینم
&&
بنمای جمال تا بدانم
\\
خواننده تو باش سوی خویشم
&&
تا مرکب پی بریده رانم
\\
در دیده به جای دیده بنشین
&&
تا نامهٔ نانبشته خوانم
\\
تو عاشق هست و نیست خواهی
&&
بپذیر مرا که من چنانم
\\
در دیده ز بیم غیرت تو
&&
اکنون نه سناییم سنانم
\\
\end{longtable}
\end{center}
