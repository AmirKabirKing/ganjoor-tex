\begin{center}
\section*{غزل شماره ۱۳۶: هر کرا در دل خمار عشق و برنایی بود}
\label{sec:136}
\addcontentsline{toc}{section}{\nameref{sec:136}}
\begin{longtable}{l p{0.5cm} r}
هر کرا در دل خمار عشق و برنایی بود
&&
کار او در عاشقی زاری و رسوایی بود
\\
این منم زاری که از عشق بتان شیدا شدم
&&
آری اندر عاشقی زاری و شیدایی بود
\\
ای نگارین چند فرمایی شکیبایی مرا
&&
با غم عشقت کجا در دل شکیبایی بود
\\
مر مرا گفتی چرا بر روی من عاشق شدی
&&
عاشقی جانانه خودکامی و خودرایی بود
\\
شد دلم صفرایی از دست فراق این جمال
&&
آنکه صفرایی نشد در عشق سودایی بود
\\
آن که یک ساعت دل آورد و ببرد و باز داد
&&
بر حقیقت دان که او در عشق هر جایی بود
\\
از سخنهای سنایی سیر کی گردند خود
&&
جز کسی کو در ره تحقیق بینایی بود
\\
از جمال یوسفی سیری نیابد جاودان
&&
هر کرا بر جان و دل عشق زلیخایی بود
\\
\end{longtable}
\end{center}
