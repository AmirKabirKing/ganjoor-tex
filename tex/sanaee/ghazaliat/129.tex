\begin{center}
\section*{غزل شماره ۱۲۹: چون دو زلفین تو کمند بود}
\label{sec:129}
\addcontentsline{toc}{section}{\nameref{sec:129}}
\begin{longtable}{l p{0.5cm} r}
چون دو زلفین تو کمند بود
&&
شاید ار دل اسیر بند بود
\\
گوییم صبر کن ز بهر خدا
&&
آخر این صبر نیز چند بود
\\
خواجه انصاف می بباید داد
&&
با چنین رخ چه جای پند بود
\\
سرو را کی رخ چو ماه بود
&&
ما را کی لب چو قند بود
\\
می ندانی که پست گردد زود
&&
هر کرا همت بلند بود
\\
هر که معشوقه‌ای چنین طلبد
&&
همه رنج و غمش پسند بود
\\
\end{longtable}
\end{center}
