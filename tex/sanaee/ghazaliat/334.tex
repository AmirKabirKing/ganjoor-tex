\begin{center}
\section*{غزل شماره ۳۳۴: ای نموده عاشقی بر زلف و چاک پیرهن}
\label{sec:334}
\addcontentsline{toc}{section}{\nameref{sec:334}}
\begin{longtable}{l p{0.5cm} r}
ای نموده عاشقی بر زلف و چاک پیرهن
&&
عاشقی آری ولیکن بر مراد خویشتن
\\
تا ترا در دل چو قارون گنجها باشد ز آز
&&
چند گویی از اویس و چند گویی از قرن
\\
در دیار تو نتابد ز آسمان هرگز سهیل
&&
گر همی باید سهیلت قصد کن سوی یمن
\\
از مراد خویش برخیز ار مریدی عشق را
&&
در یمن ساکن نگردی تا که باشی در ختن
\\
آز را گشتن دگر آن آرزو دیدن دگر
&&
هر دو با هم کرد نتوان یا وثن شو یا شمن
\\
بی جمال یوسف و بی سوز یعقوب از گزاف
&&
توتیایی ناید از هر باد و از هر پیرهن
\\
باده با فرعون خوری از جام عشق موسوی
&&
با علی در بیعت آیی زهر پاشی بر حسن
\\
پای این میدان نداری جامهٔ مردان مپوش
&&
برگ بی‌برگی نداری لاف درویشی مزن
\\
\end{longtable}
\end{center}
