\begin{center}
\section*{غزل شماره ۹: ساقیا می ده که جز می نشکند پرهیز را}
\label{sec:009}
\addcontentsline{toc}{section}{\nameref{sec:009}}
\begin{longtable}{l p{0.5cm} r}
ساقیا می ده که جز می نشکند پرهیز را
&&
تا زمانی کم کنم این زهد رنگ آمیز را
\\
ملکت آل بنی آدم ندارد قیمتی
&&
خاک ره باید شمردن دولت پرویز را
\\
دین زردشتی و آیین قلندر چند چند
&&
توشه باید ساختن مر راه جان آویز را
\\
هر چه اسبابست آتش در زن و خرم نشین
&&
بدرهٔ ناداشتی به روز رستاخیز را
\\
زاهدان و مصلحان مر نزهت فردوس را
&&
وین گروه لاابالی جان عشق‌انگیز را
\\
ساقیا زنجیر مشکین را ز مه بردار زود
&&
بر رخ زردم نه آن یاقوت شکر ریز را
\\
\end{longtable}
\end{center}
