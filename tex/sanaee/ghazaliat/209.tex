\begin{center}
\section*{غزل شماره ۲۰۹: ای بس قدح درد که کردست دلم نوش}
\label{sec:209}
\addcontentsline{toc}{section}{\nameref{sec:209}}
\begin{longtable}{l p{0.5cm} r}
ای بس قدح درد که کردست دلم نوش
&&
دور از لب و دندان شما بی خبران دوش
\\
گه بوسه همی داد بر آن درد لب و چشم
&&
گه رقص همی کرد بر آن حال دل و هوش
\\
گه عقل همی گفت که ای طبع تو کم نال
&&
گه صبر همی گفت که ای آه تو مخروش
\\
درد آمده پاداش که هین ای سر و تن داد
&&
عشق آمده با نیش که هان ای دل و جان نوش
\\
دردی که به افسانه شنیدم همه از خلق
&&
از علم به عین آمد وز گوش به آغوش
\\
در حجرهٔ چشم آمد خورشید خیالش
&&
خورشید که دیدست سیه کرده بناگوش
\\
در حسرت آن دیدهٔ چون دیدهٔ آهو
&&
این دیده نه در خواب و نه بیدار چو خرگوش
\\
حیرت سوی چشم آمده کای چشم تو منگر
&&
غیرت سوی گوش آمده کی گوش تو منیوش
\\
با چشم سرم گفته تراییم تو منگر
&&
در گوش دلم خوانده تراییم تو مخروش
\\
ذوق آمده در چشم که ای چشم چنین چش
&&
شوق آمده در گوش که ای گوش چنین گوش
\\
این خود صفت نقش خیالیست چه چیزست
&&
یارب که ببینم به عیان آن رخ نیکوش
\\
او بلبله بر دست و خرد سلسله در پای
&&
او غالیه بر گوش و رهی غاشیه بر دوش
\\
در عاشقی آنجا که ورا پای مرا سر
&&
در بندگی آنجا که ورا حلقه مرا گوش
\\
صد روح در آویخته از دامن کرته
&&
سی روز برانگیخته از گوشهٔ شب پوش
\\
آوازه در افتاده به هر جا که سنایی
&&
در مکتب او کرد همه تخته فراموش
\\
\end{longtable}
\end{center}
