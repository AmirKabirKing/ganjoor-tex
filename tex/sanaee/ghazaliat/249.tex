\begin{center}
\section*{غزل شماره ۲۴۹: من که باشم که به تن رخت وفای تو کشم}
\label{sec:249}
\addcontentsline{toc}{section}{\nameref{sec:249}}
\begin{longtable}{l p{0.5cm} r}
من که باشم که به تن رخت وفای تو کشم
&&
دیده حمال کنم بار جفای تو کشم
\\
ملک الموت جفای تو ز من جان ببرد
&&
چون به دل بار سرافیل وفای تو کشم
\\
چکند عرش که او غاشیهٔ من نکشد
&&
چون به جان غاشیهٔ حکم و رضای تو کشم
\\
چون زنان رشک برند ایمنی و عافیتی
&&
بر بلایی که به جای تو برای تو کشم
\\
نچشم ور بچشم باده ز دست تو چشم
&&
نکشم ور بکشم طعنه برای تو کشم
\\
گر خورم باده به یاد کف دست تو خورم
&&
ور کشم سرمه ز خاک کف پای تو کشم
\\
جز هوا نسپرم آنگه که هوای تو کنم
&&
جز وفا نشمرم آنگه که جفای تو کشم
\\
بوی جان آیدم آنگه که حدیث تو کنم
&&
شاخ عز رویدم آنگه که بلای تو کشم
\\
به خدای ار تو به دین و خردم قصد کنی
&&
هر دو را گوش گرفته به سرای تو کشم
\\
ور تو با من به تن و جان و دلم حکم کنی
&&
هر سه را رقص کنان پیش هوای تو کشم
\\
من خود از نسبت عشق تو سنایی شده‌ام
&&
کی توانم که خطی گرد ثنای تو کشم
\\
\end{longtable}
\end{center}
