\begin{center}
\section*{غزل شماره ۴۰۴: صنما آن خط مشکین که فراز آوردی}
\label{sec:404}
\addcontentsline{toc}{section}{\nameref{sec:404}}
\begin{longtable}{l p{0.5cm} r}
صنما آن خط مشکین که فراز آوردی
&&
بر گل از غلیه گوی که طراز آوردی
\\
گرچه خوبست به گرد رخ تو زلف دراز
&&
خط بسی خوبتر از زلف دراز آوردی
\\
گر نیازست رهی را به خط خوب تو باز
&&
تو رهی را به خط خویش نیاز آوردی
\\
قبله‌ای ساختی از غالیه بر سیم سپید
&&
تا بدان قبله بتان را به نماز آوردی
\\
پیش خلق از جهت شعبده و بلعجبی
&&
نرگس بلعجب شعبده‌باز آوردی
\\
چند گویی که دلت پیش تو باز آوردم
&&
این سخن بیهده و هزل و مجاز آوردی
\\
دلم افروخته بود از طرب و شادی و ناز
&&
تو دلی سوختهٔ از گرم و گداز آوردی
\\
\end{longtable}
\end{center}
