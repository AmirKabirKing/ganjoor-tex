\begin{center}
\section*{غزل شماره ۳۴۸: ای همه انصاف‌جویان بندهٔ بیداد تو}
\label{sec:348}
\addcontentsline{toc}{section}{\nameref{sec:348}}
\begin{longtable}{l p{0.5cm} r}
ای همه انصاف‌جویان بندهٔ بیداد تو
&&
زاد جان رادمردان حسن مادرزاد تو
\\
حسن را بنیاد افگندی چنان محکم که نیست
&&
جز «و یبقی وجه ربک» نقش بی‌بنیاد تو
\\
بلفضولانرا سوی تو راه نبود تا بود
&&
کبریا در بادبان رایگان آباد تو
\\
آتش اندر خاکپاشان همه عالم زدند
&&
هر کرا بر روی آب تست در سر باد تو
\\
تنگ چشمان را ز تو گردی نخیزد تا بود
&&
«لن تنالوا البر حتی تنفقوا» بر یاد تو
\\
ای بسا در حقهٔ جان غیورانت که هست
&&
نعره‌های سر به مهر از درد بی فریاد تو
\\
فتنه بودی یاسمینت از برگ گل نشکفته بود
&&
فتنه‌تر گشتی چو بررست از سمن شمشاد تو
\\
«فالق الاصباح» بر جانهای ما داد تو خواند
&&
هین که وقت «جاعل اللیل» آمد از بیداد تو
\\
اندر این مجلس به ما شادی و غمگینی ز خصم
&&
چشم بد دور از دل غمگین و طبع شاد تو
\\
روی ما تازست تا تو حاضری از روی تو
&&
جان ما خوش باد چون غایب شوی بر یاد تو
\\
یکزمان خوش باش با ما پیش از آن کز بیم خصم
&&
روز مانا خوش کند گفتار «شب خوش باد» تو
\\
اینهمه سحر حلال آخر کت آموزد همی
&&
گر سنایی نیست اندر ساحری استاد تو
\\
\end{longtable}
\end{center}
