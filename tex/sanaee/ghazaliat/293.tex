\begin{center}
\section*{غزل شماره ۲۹۳: تماشا را یکی بخرام در بستان جان ای جان}
\label{sec:293}
\addcontentsline{toc}{section}{\nameref{sec:293}}
\begin{longtable}{l p{0.5cm} r}
تماشا را یکی بخرام در بستان جان ای جان
&&
ببین در زیر پای خویش جان افشان جان ای جان
\\
نخواهد جان دگر جانی اگر صد جان برافشاند
&&
که بس باشد قبول تو بقای جان جان ای جان
\\
ترا یارست بس در جان ز بهر آنکه نشناسد
&&
ز خوبان جز تو در عالم همی درمان جان ای جان
\\
ز بهر چشم خوب تو برای دفع چشم بد
&&
کمال عافیت باشد همه قربان جان ای جان
\\
از آن تا در دل و دیده گهر جز عشق تو نبود
&&
برون روید گهر هر دم ز بحر و کان جان ای جان
\\
همه عالم چو حرف «ن» از آن در خدمتت مانده
&&
که از کل نکورویان تویی خاص آن جان ای جان
\\
ز بهر سرخ رویی جان چه باشد گر به یک غمزه
&&
ز خوبان جان براندایی تو در میدان جان ای جان
\\
به نور روی تست اکنون همه توحید عقل من
&&
به کفر زلف تست اکنون همه ایمان جان ای جان
\\
سنایی وار در عالم ز بهر آبروی خود
&&
سنایی خاکپای تست سر دیوان جان ای جان
\\
\end{longtable}
\end{center}
