\begin{center}
\section*{غزل شماره ۱۲۷: گر سال عمر من به سر آید روا بود}
\label{sec:127}
\addcontentsline{toc}{section}{\nameref{sec:127}}
\begin{longtable}{l p{0.5cm} r}
گر سال عمر من به سر آید روا بود
&&
اندی که سال عیش همیشه به جا بود
\\
پایان عاشقی نه پدیدست تا ابد
&&
پس سال و ماه و وقت در او از کجا بود
\\
ای وای و حسرتا که اگر عشق یک نفس
&&
در سال و ماه عمر ز جانم جدا بود
\\
ای آمده به طمع وصال نگار خویش
&&
نشنیده‌ای که عشق برای بلا بود
\\
پروانهٔ ضعیف کند جان فدای شمع
&&
تا پیش شمع یک نظرش را سنا بود
\\
دیدار وی همان بود و سوختن همان
&&
گویی فنای وی همه اندر بقا بود
\\
آن را که زندگیش به عشق‌ست مرگ نیست
&&
هرگز گمان مبر که مر او را فنا بود
\\
\end{longtable}
\end{center}
