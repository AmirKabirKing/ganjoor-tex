\begin{center}
\section*{غزل شماره ۱۵۰: هر کو به خرابات مرا راه نماید}
\label{sec:150}
\addcontentsline{toc}{section}{\nameref{sec:150}}
\begin{longtable}{l p{0.5cm} r}
هر کو به خرابات مرا راه نماید
&&
زنگ غم و تیمار ز جانم بزداید
\\
ره کو بگشاید در میخانه به من بر
&&
ایزد در فردوس برو بر بگشاید
\\
ای جمع مسلمانان پیران و جوانان
&&
در شهر شما کس را خود مزد نباید
\\
گویند سنایی را شد شرم به یک بار
&&
رفتن به خرابات ورا شرم نیاید
\\
دایم به خرابات مرا رفتن از آنست
&&
کالا به خرابات مرا دل نگشاید
\\
من می‌روم و رفتن و خواهم رفتن
&&
کمتر غمم اینست که گویند نشاید
\\
\end{longtable}
\end{center}
