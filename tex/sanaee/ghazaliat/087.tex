\begin{center}
\section*{غزل شماره ۸۷: معشوق به سامان شد تا باد چنین باد}
\label{sec:087}
\addcontentsline{toc}{section}{\nameref{sec:087}}
\begin{longtable}{l p{0.5cm} r}
معشوق به سامان شد تا باد چنین باد
&&
کفرش همه ایمان شد تا باد چنین باد
\\
زان لب که همی زهر فشاندی به تکبر
&&
اکنون شکر افشان شد تا باد چنین باد
\\
آن غمزه که بد بودی با مدعی سست
&&
امروز بتر زان شد تا باد چنین باد
\\
آن رخ که شکر بود نهانش به لطافت
&&
اکنون شکرستان شد تا باد چنین باد
\\
حاسد که چو دامنش ببوسید همی پای
&&
بی سر چو گریبان شد تا باد چنین باد
\\
نعلی که بینداخت همی مرکبش از پای
&&
تاج سر سلطان شد تا باد چنین باد
\\
پیداش جفا بودی و پنهانش لطافت
&&
پیداش چو پنهان شد تا باد چنین باد
\\
چون گل همه تن بودی تا بود چنین بود
&&
چون باده همه جان شد تا باد چنین باد
\\
دیوی که بر آن کفر همی داشت مر او را
&&
آن دیو مسلمان شد تا باد چنین باد
\\
تا لاجرم از شکر سنایی چو سنایی
&&
مشهور خراسان شد تا باد چنین باد
\\
\end{longtable}
\end{center}
