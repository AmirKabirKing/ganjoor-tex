\begin{center}
\section*{غزل شماره ۴۲۷: گاه آن آمد بتا کاندر خرابی دم زنی}
\label{sec:427}
\addcontentsline{toc}{section}{\nameref{sec:427}}
\begin{longtable}{l p{0.5cm} r}
گاه آن آمد بتا کاندر خرابی دم زنی
&&
شور در میراث خواران بنی آدم زنی
\\
بارنامهٔ بی‌نیازی برگشایی تا به کی
&&
آتش اندر بار مایهٔ کعبه و زمزم زنی
\\
صدهزاران جان متواری در آری زیر زلف
&&
چون به دو کوکب کمند حلقه‌ها را خم زنی
\\
بر سر آزادگان نه تاج گر گوهر نهی
&&
بر سر سوداییان زن تیغ گر محکم زنی
\\
تیغ خویش از خون هر تر دامنی رنگین مکن
&&
تو چو رستم پیشه‌ای آن به که بر رستم زنی
\\
در خرابات نهاد خود بر آسودست خلق
&&
غمزه بر هم زن یکی تا خلق را بر هم زنی
\\
پاکبازان جهان چون سوختهٔ نفس تواند
&&
خام طمعی باشد ار با خام دستان دم زنی
\\
ما به امیدی هدف کردیم جان چون دیگران
&&
تا چو تیر غمزه سازی بر سنایی هم زنی
\\
\end{longtable}
\end{center}
