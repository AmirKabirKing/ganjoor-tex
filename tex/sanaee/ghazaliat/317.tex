\begin{center}
\section*{غزل شماره ۳۱۷: ای رخ تو بهار و گلشن من}
\label{sec:317}
\addcontentsline{toc}{section}{\nameref{sec:317}}
\begin{longtable}{l p{0.5cm} r}
ای رخ تو بهار و گلشن من
&&
همچو جانست عشق در تن من
\\
راست چون زلف تو بود تاریک
&&
بی رخ تو جهان روشن من
\\
همچو خورشید و ماه در تابد
&&
عشق تو هر شبی ز روزن من
\\
دست تو طوق گردن دگری
&&
عشق تو طوق گردن من
\\
ماه را راه گم شود بر چرخ
&&
هر شبی از خروش و شیون من
\\
گر تو یک ره جمال بنمایی
&&
برزند بابهشت برزن من
\\
خاک پایت برم چو سرمه به کار
&&
گر چه دادی به باد خرمن من
\\
رنجه کن پای خویش و کوته کن
&&
دست جور و بلا ز دامن من
\\
رادمری کنی به در نبری
&&
بنهی بار خلق بر تن من
\\
چون درآیی ز در توام به زمان
&&
بردمد لاله‌زار و سوسن من
\\
تا سنایی ترا همی گوید
&&
ای رخ تو بهار و گلشن من
\\
\end{longtable}
\end{center}
