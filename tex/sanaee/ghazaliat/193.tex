\begin{center}
\section*{غزل شماره ۱۹۳: بامدادان شاه خود را دیده‌ام بر مرکبش}
\label{sec:193}
\addcontentsline{toc}{section}{\nameref{sec:193}}
\begin{longtable}{l p{0.5cm} r}
بامدادان شاه خود را دیده‌ام بر مرکبش
&&
مشک پاشان از دور زلف و بوسه باران از لبش
\\
صد هزاران جسم و جان افشان و حیران از قفاش
&&
از برای بوسه چیدن گرد سایهٔ مرکبش
\\
خنجری در دست و «من یرغب» کنان عیاروار
&&
جسم و جان عاشقان تازان سوی «من برغبش»
\\
بهر دفع چشم زخم مستش را چو من
&&
خیل خیل انجم همی کردند یارب یاربش
\\
سوی دیو و دیو مردم هر زمان چون آسمان
&&
از دو ماه نو شهاب انداز نعل اشهبش
\\
کفر و دین و دیو مردم هر زمان چون آسمان
&&
از دو ماه نو شهاب انداز، نعل اشبهش
\\
دستها بر سر چو عقرب روز و شب از بهر آنک
&&
تا چرا بر می‌خورد پروین ز مشک عقربش
\\
درج یاقوتیش دیدم، پر ز کوکبهای سیم
&&
یارب آن درجش نکوتر بود یا آن کوکبش
\\
جان همی بارید هر ساعت ز سر تا پای او
&&
گوییا بودست آب زندگانی مشربش
\\
آفتابی بود گفتی متصل با شش هلال
&&
چون بدیدم آن دو تا رخسار و شش تو غبغبش
\\
هر زمان از چشم و لعلش، غمزه‌ای و خنده‌ای
&&
جان فزودن کیش دیدم دل ربودن مذهبش
\\
گر چه بودم با سنایی در جهان عافیت
&&
هم بخوردم آخرالامر از پی حبش حبش
\\
\end{longtable}
\end{center}
