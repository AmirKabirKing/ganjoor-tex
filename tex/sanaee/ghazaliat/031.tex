\begin{center}
\section*{غزل شماره ۳۱: ماهرویا در جهان آوازهٔ تست}
\label{sec:031}
\addcontentsline{toc}{section}{\nameref{sec:031}}
\begin{longtable}{l p{0.5cm} r}
ماهرویا در جهان آوازهٔ تست
&&
کارهای عاشقان ناساخته از ساز تست
\\
هر کجا نظمیست شیرین قصه‌های عشق تست
&&
هر کجا نثریست زیبا نامهای ناز تست
\\
باز عشقت جمله باز آن را چو تیهو صید کرد
&&
هست عالی همت آن بازی که صید باز تست
\\
صدهزاران دل فدا بادا دلی را کو ز عشق
&&
سال و ماه و روز و شب مشغول شاهد باز تست
\\
دلبرا دلهای مردان جمله ملک غنج تست
&&
گلرخا جانهای پاکان جمله ملک ناز تست
\\
آسمان تند و سرکش زیر دست و رام تست
&&
روزگار تند و توسن دایهٔ انباز تست
\\
هر کجا چشمیست بینا بارگاه عشق تست
&&
هر کجا گوشیست والا عاشق آواز تست
\\
\end{longtable}
\end{center}
