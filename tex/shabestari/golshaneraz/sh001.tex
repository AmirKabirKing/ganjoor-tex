\begin{center}
\section*{بخش ۱ - دیباچه}
\label{sec:sh001}
\addcontentsline{toc}{section}{\nameref{sec:sh001}}
\begin{longtable}{l p{0.5cm} r}
به نام آن که جان را فکرت آموخت
&&
چراغ دل به نور جان برافروخت
\\
ز فضلش هر دو عالم گشت روشن
&&
ز فیضش خاک آدم گشت گلشن
\\
توانایی که در یک طرفةالعین
&&
ز کاف و نون پدید آورد کونین
\\
چو قاف قدرتش دم بر قلم زد
&&
هزاران نقش بر لوح عدم زد
\\
از آن دم گشت پیدا هر دو عالم
&&
وز آن دم شد هویدا جان آدم
\\
در آدم شد پدید این عقل و تمییز
&&
که تا دانست از آن اصل همه چیز
\\
چو خود را دید یک شخص معین
&&
تفکر کرد تا خود چیستم من
\\
ز جزوی سوی کلی یک سفر کرد
&&
وز آنجا باز بر عالم گذر کرد
\\
جهان را دید امر اعتباری
&&
چو واحد گشته در اعداد ساری
\\
جهان خلق و امر از یک نفس شد
&&
که هم آن دم که آمد باز پس شد
\\
ولی آن جایگه آمد شدن نیست
&&
شدن چون بنگری جز آمدن نیست
\\
به اصل خویش راجع گشت اشیا
&&
همه یک چیز شد پنهان و پیدا
\\
تعالی الله قدیمی کو به یک دم
&&
کند آغاز و انجام دو عالم
\\
جهان خلق و امر اینجا یکی شد
&&
یکی بسیار و بسیار اندکی شد
\\
همه از وهم توست این صورت غیر
&&
که نقطه دایره است از سرعت سیر
\\
یکی خط است از اول تا به آخر
&&
بر او خلق جهان گشته مسافر
\\
در این ره انبیا چون ساربانند
&&
دلیل و رهنمای کاروانند
\\
وز ایشان سید ما گشته سالار
&&
هم او اول هم او آخر در این کار
\\
احد در میم احمد گشت ظاهر
&&
در این دور اول آمد عین آخر
\\
ز احمد تا احد یک میم فرق است
&&
جهانی اندر آن یک میم غرق است
\\
بر او ختم آمده پایان این راه
&&
در او منزل شده «ادعوا الی الله»
\\
مقام دلگشایش جمع جمع است
&&
جمال جانفزایش شمع جمع است
\\
شده او پیش و دلها جمله از پی
&&
گرفته دست دلها دامن وی
\\
در این ره اولیا باز از پس و پیش
&&
نشانی داده‌اند از منزل خویش
\\
به حد خویش چون گشتند واقف
&&
سخن گفتند در معروف و عارف
\\
یکی از بحر وحدت گفت انا الحق
&&
یکی از قرب و بعد و سیر زورق
\\
یکی را علم ظاهر بود حاصل
&&
نشانی داد از خشکی ساحل
\\
یکی گوهر برآورد و هدف شد
&&
یکی بگذاشت آن نزد صدف شد
\\
یکی در جزو و کل گفت این سخن باز
&&
یکی کرد از قدیم و محدث آغاز
\\
یکی از زلف و خال و خط بیان کرد
&&
شراب و شمع و شاهد را عیان کرد
\\
یکی از هستی خود گفت و پندار
&&
یکی مستغرق بت گشت و زنار
\\
سخنها چون به وفق منزل افتاد
&&
در افهام خلایق مشکل افتاد
\\
کسی را کاندر این معنی است حیران
&&
ضرورت می‌شود دانستن آن
\\
\end{longtable}
\end{center}
