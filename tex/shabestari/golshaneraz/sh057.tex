\begin{center}
\section*{بخش ۵۷ - اشارت به خرابات}
\label{sec:sh057}
\addcontentsline{toc}{section}{\nameref{sec:sh057}}
\begin{longtable}{l p{0.5cm} r}
خراباتی شدن از خود رهایی است
&&
خودی کفر است ور خود پارسایی است
\\
نشانی داده‌اندت از خرابات
&&
که «التوحید اسقاط الاضافات»
\\
خرابات از جهان بی‌مثالی است
&&
مقام عاشقان لاابالی است
\\
خرابات آشیان مرغ جان است
&&
خرابات آستان لامکان است
\\
خراباتی خراب اندر خراب است
&&
که در صحرای او عالم سراب است
\\
خراباتی است بی حد و نهایت
&&
نه آغازش کسی دیده نه غایت
\\
اگر صد سال در وی می‌شتابی
&&
نه کس را و نه خود را بازیابی
\\
گروهی اندر او بی پا و بی سر
&&
همه نه مؤمن و نه نیز کافر
\\
شراب بیخودی در سر گرفته
&&
به ترک جمله خیر و شر گرفته
\\
شرابی خورده هر یک بی‌لب و کام
&&
فراغت یافته از ننگ و از نام
\\
حدیث و ماجرای شطح و طامات
&&
خیال خلوت و نور کرامات
\\
به بوی دردیی از دست داده
&&
ز ذوق نیستی مست اوفتاده
\\
عصا و رکوه و تسبیح و مسواک
&&
گرو کرده به دردی جمله را پاک
\\
میان آب و گل افتان و خیزان
&&
به جای اشک خون از دیده ریزان
\\
گهی از سرخوشی در عالم ناز
&&
شده چون شاطران گردن افراز
\\
گهی از روسیاهی رو به دیوار
&&
گهی از سرخ‌رویی بر سر دار
\\
گهی اندر سماع از شوق جانان
&&
شده بی پا و سر چون چرخ گردان
\\
به هر نغمه که از مطرب شنیده
&&
بدو وجدی از آن عالم رسیده
\\
سماع جان نه آخر صوت و حرف است
&&
که در هر پرده‌ای سری شگرف است
\\
ز سر بیرون کشیده دلق ده تو
&&
مجرد گشته از هر رنگ و هر بو
\\
فرو شسته بدان صاف مروق
&&
همه رنگ سیاه و سبز و ازرق
\\
یکی پیمانه خورده از می صاف
&&
شده زان صوفی صافی ز اوصاف
\\
به مژگان خاک مزبل پاک رفته
&&
ز هر چ آن دیده از صد یک نگفته
\\
گرفته دامن رندان خمار
&&
ز شیخی و مریدی گشته بیزار
\\
چه شیخی و مریدی این چه قید است
&&
چه جای زهد و تقوی این چه شید است
\\
اگر روی تو باشد در که و مه
&&
بت و زنار و ترسایی تو را به
\\
\end{longtable}
\end{center}
