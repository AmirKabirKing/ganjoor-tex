\begin{center}
\section*{بخش ۲۸ - تمثیل در بیان نسبت عقل با شهود}
\label{sec:sh028}
\addcontentsline{toc}{section}{\nameref{sec:sh028}}
\begin{longtable}{l p{0.5cm} r}
ندارد باورت اکمه ز الوان
&&
وگر صد سال گویی نقل و برهان
\\
سپید و زرد و سرخ و سبز و کاهی
&&
به نزد وی نباشد جز سیاهی
\\
نگر تا کور مادرزاد بدحال
&&
کجا بینا شود از کحل کحال
\\
خرد از دیدن احوال عقبا
&&
بود چون کور مادرزاد دنیا
\\
ورای عقل طوری دارد انسان
&&
که بشناسد بدان اسرار پنهان
\\
بسان آتش اندر سنگ و آهن
&&
نهاده است ایزد اندر جان و در تن
\\
چو بر هم اوفتاد این سنگ و آهن
&&
ز نورش هر دو عالم گشت روشن
\\
از آن مجموع پیدا گردد این راز
&&
چو دانستی برو خود را برانداز
\\
تویی تو نسخهٔ نقش الهی
&&
بجو از خویش هر چیزی که خواهی
\\
\end{longtable}
\end{center}
