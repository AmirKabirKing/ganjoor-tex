\begin{center}
\section*{بخش ۴ - جواب}
\label{sec:sh004}
\addcontentsline{toc}{section}{\nameref{sec:sh004}}
\begin{longtable}{l p{0.5cm} r}
مرا گفتی بگو چبود تفکر
&&
کز این معنی بماندم در تحیر
\\
تفکر رفتن از باطل سوی حق
&&
به جزو اندر بدیدن کل مطلق
\\
حکیمان کاندر این کردند تصنیف
&&
چنین گفتند در هنگام تعریف
\\
که چون حاصل شود در دل تصور
&&
نخستین نام وی باشد تذکر
\\
وز او چون بگذری هنگام فکرت
&&
بود نام وی اندر عرف عبرت
\\
تصور کان بود بهر تدبر
&&
به نزد اهل عقل آمد تفکر
\\
ز ترتیب تصورهای معلوم
&&
شود تصدیق نامفهوم مفهوم
\\
مقدم چون پدر تالی چو مادر
&&
نتیجه هست فرزند، ای برادر
\\
ولی ترتیب مذکور از چه و چون
&&
بود محتاج استعمال قانون
\\
دگرباره در آن گر نیست تایید
&&
هر آیینه که باشد محض تقلید
\\
ره دور و دراز است آن رها کن
&&
چو موسی یک زمان ترک عصا کن
\\
درآ در وادی ایمن زمانی
&&
شنو «انی انا الله» بی‌گمانی
\\
محقق را که وحدت در شهود است
&&
نخستین نظره بر نور وجود است
\\
دلی کز معرفت نور و صفا دید
&&
ز هر چیزی که دید اول خدا دید
\\
بود فکر نکو را شرط تجرید
&&
پس آنگه لمعه‌ای از برق تایید
\\
هر آنکس را که ایزد راه ننمود
&&
ز استعمال منطق هیچ نگشود
\\
حکیم فلسفی چون هست حیران
&&
نمی‌بیند ز اشیا غیر امکان
\\
از امکان می‌کند اثبات واجب
&&
از این حیران شد اندر ذات واجب
\\
گهی از دور دارد سیر معکوس
&&
گهی اندر تسلسل گشته محبوس
\\
چو عقلش کرد در هستی توغل
&&
فرو پیچید پایش در تسلسل
\\
ظهور جملهٔ اشیا به ضد است
&&
ولی حق را نه مانند و نه ند است
\\
چو نبود ذات حق را ضد و همتا
&&
ندانم تا چگونه دانی او را
\\
ندارد ممکن از واجب نمونه
&&
چگونه دانیش آخر چگونه؟
\\
زهی نادان که او خورشید تابان
&&
به نور شمع جوید در بیابان
\\
\end{longtable}
\end{center}
