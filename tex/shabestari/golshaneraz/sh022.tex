\begin{center}
\section*{بخش ۲۲ - قاعده در حکمت وجود اولیا}
\label{sec:sh022}
\addcontentsline{toc}{section}{\nameref{sec:sh022}}
\begin{longtable}{l p{0.5cm} r}
نبوت را ظهور از آدم آمد
&&
کمالش در وجود خاتم آمد
\\
ولایت بود باقی تا سفر کرد
&&
چو نقطه در جهان دوری دگر کرد
\\
ظهور کل او باشد به خاتم
&&
بدو گردد تمامی دور عالم
\\
وجود اولیا او را چو عضوند
&&
که او کل است و ایشان همچو جزوند
\\
چو او از خواجه یابد نسبت تام
&&
از او با ظاهر آید رحمت عام
\\
شود او مقتدای هر دو عالم
&&
خلیفه گردد از اولاد آدم
\\
\end{longtable}
\end{center}
