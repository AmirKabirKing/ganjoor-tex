\begin{center}
\section*{بخش ۷ - جواب}
\label{sec:sh007}
\addcontentsline{toc}{section}{\nameref{sec:sh007}}
\begin{longtable}{l p{0.5cm} r}
در آلا فکر کردن شرط راه است
&&
ولی در ذات حق محض گناه است
\\
بود در ذات حق اندیشه باطل
&&
محال محض دان تحصیل حاصل
\\
چو آیات است روشن گشته از ذات
&&
نگردد ذات او روشن ز آیات
\\
همه عالم به نور اوست پیدا
&&
کجا او گردد از عالم هویدا
\\
نگنجد نور ذات اندر مظاهر
&&
که سبحات جلالش هست قاهر
\\
رها کن عقل را با حق همی باش
&&
که تاب خور ندارد چشم خفاش
\\
در آن موضع که نور حق دلیل است
&&
چه جای گفتگوی جبرئیل است
\\
فرشته گرچه دارد قرب درگاه
&&
نگنجد در مقام «لی مع الله»
\\
چو نور او ملک را پر بسوزد
&&
خرد را جمله پا و سر بسوزد
\\
بود نور خرد در ذات انور
&&
به سان چشم سر در چشمه خور
\\
چو مبصر با بصر نزدیک گردد
&&
بصر ز ادراک آن تاریک گردد
\\
سیاهی گر بدانی نور ذات است
&&
به تاریکی درون آب حیات است
\\
سیه جز قابض نور بصر نیست
&&
نظر بگذار کین جای نظر نیست
\\
چه نسبت خاک را با عالم پاک
&&
که ادراک است عجز از درک ادراک
\\
سیه رویی ز ممکن در دو عالم
&&
جدا هرگز نشد والله اعلم
\\
سواد الوجه فی الدارین درویش
&&
سواد اعظم آمد بی کم و بیش
\\
چه می‌گویم که هست این نکته باریک
&&
شب روشن میان روز تاریک
\\
در این مشهد که انوار تجلی است
&&
سخن دارم ولی نا گفتن اولی است
\\
\end{longtable}
\end{center}
