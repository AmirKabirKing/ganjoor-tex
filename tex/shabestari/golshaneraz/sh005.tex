\begin{center}
\section*{بخش ۵ - تمثیل در بیان سر پنهانی حق در عین پیدایی}
\label{sec:sh005}
\addcontentsline{toc}{section}{\nameref{sec:sh005}}
\begin{longtable}{l p{0.5cm} r}
اگر خورشید بر یک حال بودی
&&
شعاع او به یک منوال بودی
\\
ندانستی کسی کین پرتو اوست
&&
نبودی هیچ فرق از مغز تا پوست
\\
جهان جمله فروغ نور حق دان
&&
حق اندر وی ز پیدایی است پنهان
\\
چو نور حق ندارد نقل و تحویل
&&
نیاید اندر او تغییر و تبدیل
\\
تو پنداری جهان خود هست قائم
&&
به ذات خویشتن پیوسته دائم
\\
کسی کو عقل دوراندیش دارد
&&
بسی سرگشتگی در پیش دارد
\\
ز دوراندیشی عقل فضولی
&&
یکی شد فلسفی دیگر حلولی
\\
خرد را نیست تاب نور آن روی
&&
برو از بهر او چشم دگر جوی
\\
دو چشم فلسفی چون بود احول
&&
ز وحدت دیدن حق شد معطل
\\
ز نابینایی آمد راه تشبیه
&&
ز یک چشمی است ادراکات تنزیه
\\
تناسخ زان سبب کفر است و باطل
&&
که آن از تنگ چشمی گشت حاصل
\\
چو اکمه بی‌نصیب از هر کمال است
&&
کسی کو را طریق اعتزال است
\\
رمد دارد دو چشم اهل ظاهر
&&
که از ظاهر نبیند جز مظاهر
\\
کلامی کو ندارد ذوق توحید
&&
به تاریکی در است از غیم تقلید
\\
در او هرچ آن بگفتند از کم و بیش
&&
نشانی داده‌اند از دیدهٔ خویش
\\
منزه ذاتش از چند و چه و چون
&&
«تعالی شانه عما یقولون»
\\
\end{longtable}
\end{center}
