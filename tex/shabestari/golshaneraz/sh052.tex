\begin{center}
\section*{بخش ۵۲ - اشارت به زلف}
\label{sec:sh052}
\addcontentsline{toc}{section}{\nameref{sec:sh052}}
\begin{longtable}{l p{0.5cm} r}
حدیث زلف جانان بس دراز است
&&
چه می‌پرسی از او کان جای راز است
\\
مپرس از من حدیث زلف پرچین
&&
مجنبانید زنجیر مجانین
\\
ز قدش راستی گفتم سخن دوش
&&
سر زلفش مرا گفتا فروپوش
\\
کژی بر راستی زو گشت غالب
&&
وز او در پیچش آمد راه طالب
\\
همه دلها از او گشته مسلسل
&&
همه جانها از او بوده مقلقل
\\
معلق صد هزاران دل ز هر سو
&&
نشد یک دل برون از حلقهٔ او
\\
گر او زلفین مشکین برفشاند
&&
به عالم در یکی کافر نماند
\\
وگر بگذاردش پیوسته ساکن
&&
نماند در جهان یک نفس مؤمن
\\
چو دام فتنه می‌شد چنبر او
&&
به شوخی باز کرد از تن سر او
\\
اگر ببریده شد زلفش چه غم بود
&&
که گر شب کم شد اندر روز افزود
\\
چو او بر کاروان عقل ره زد
&&
به دست خویشتن بر وی گره زد
\\
نیابد زلف او یک لحظه آرام
&&
گهی بام آورد گاهی کند شام
\\
ز روی و زلف خود صد روز و شب کرد
&&
بسی بازیچه‌های بوالعجب کرد
\\
گل آدم در آن دم شد مخمر
&&
که دادش بوی آن زلف معطر
\\
دل ما دارد از زلفش نشانی
&&
که خود ساکن نمی‌گردد زمانی
\\
از او هر لحظه کار از سر گرفتم
&&
ز جان خویشتن دل برگرفتم
\\
از آن گردد دل از زلفش مشوش
&&
که از رویش دلی دارد بر آتش
\\
\end{longtable}
\end{center}
