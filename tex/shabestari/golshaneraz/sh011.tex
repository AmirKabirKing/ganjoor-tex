\begin{center}
\section*{بخش ۱۱ - قاعدهٔ تفکر در آفاق}
\label{sec:sh011}
\addcontentsline{toc}{section}{\nameref{sec:sh011}}
\begin{longtable}{l p{0.5cm} r}
مشو محبوس ارکان و طبایع
&&
برون آی و نظر کن در صنایع
\\
تفکر کن تو در خلق سماوات
&&
که تا ممدوح حق گردی در آیات
\\
ببین یک ره که تا خود عرش اعظم
&&
چگونه شد محیط هر دو عالم
\\
چرا کردند نامش عرش رحمان
&&
چه نسبت دارد او با قلب انسان
\\
چرا در جنبشند این هر دو مادام
&&
که یک لحظه نمی‌گیرند آرام
\\
مگر دل مرکز عرش بسیط است
&&
که آن چون نقطه وین دور محیط است
\\
برآید در شبانروزی کم و بیش
&&
سراپای تو عرش ای مرد درویش
\\
از او در جنبش اجسام مدور
&&
چرا گشتند یک ره نیک بنگر
\\
ز مشرق تا به مغرب‌همچو دولاب
&&
همی گردند دائم بی‌خور و خواب
\\
به هر روز و شبی این چرخ اعظم
&&
کند دور تمامی گرد عالم
\\
وز او افلاک دیگر هم بدین سان
&&
به چرخ اندر همی باشند گردان
\\
ولی برعکس دور چرخ اطلس
&&
همی‌گردند این هشت مقوس
\\
معدل کرسی ذات البروج است
&&
که آن را نه تفاوت نه فروج است
\\
حمل با ثور و با جوزا و خرچنگ
&&
بر او بر همچو شیر و خوشه آونگ
\\
دگر میزان عقرب پس کمان است
&&
ز جدی و دلو و حوت آنجا نشان است
\\
ثوابت یک هزار و بیست و چارند
&&
که بر کرسی مقام خویش دارند
\\
به هفتم چرخ کیوان پاسبان است
&&
ششم برجیس را جا و مکان است
\\
بود پنجم فلک مریخ را جای
&&
به چارم آفتاب عالم آرای
\\
سیم زهره دوم جای عطارد
&&
قمر بر چرخ دنیا گشت وارد
\\
زحل را جدی و دلو و مشتری باز
&&
به قوس و حوت کرد انجام و آغاز
\\
حمل با عقرب آمد جای بهرام
&&
اسد خورشید را شد جای آرام
\\
چو زهره ثور و میزان ساخت گوشه
&&
عطارد رفت در جوزا و خوشه
\\
قمر خرچنگ را همجنس خود دید
&&
ذنب چون راس شد یک عقده بگزید
\\
قمر را بیست و هشت آمد منازل
&&
شود با آفتاب آنگه مقابل
\\
پس از وی همچو عرجون قدیم است
&&
ز تقدیر عزیزی کو علیم است
\\
اگر در فکر گردی مرد کامل
&&
هر آیینه که گویی نیست باطل
\\
کلام حق همی ناطق بدین است
&&
که باطل دیدن از ضعف یقین است
\\
وجود پشه دارد حکمت ای خام
&&
نباشد در وجود تیر و بهرام
\\
ولی چون بنگری در اصل این کار
&&
فلک را بینی اندر حکم جبار
\\
منجم چون ز ایمان بی‌نصیب است
&&
اثر گوید که از شکل غریب است
\\
نمی‌بیند مگر کین چرخ اخضر
&&
به حکم و امر حق گشته مسخر
\\
\end{longtable}
\end{center}
