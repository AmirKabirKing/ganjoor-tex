\begin{center}
\section*{بخش ۵۹ - جواب}
\label{sec:sh059}
\addcontentsline{toc}{section}{\nameref{sec:sh059}}
\begin{longtable}{l p{0.5cm} r}
بت اینجا مظهر عشق است و وحدت
&&
بود زنار بستن عقد خدمت
\\
چو کفر و دین بود قائم به هستی
&&
شود توحید عین بت‌پرستی
\\
چو اشیا هست هستی را مظاهر
&&
از آن جمله یکی بت باشد آخر
\\
نکو اندیشه کن ای مرد عاقل
&&
که بت از روی هستی نیست باطل
\\
بدان که ایزد تعالی خالق اوست
&&
ز نیکو هر چه صادر گشت نیکوست
\\
وجود آنجا که باشد محض خیر است
&&
وگر شری است در وی آن ز غیر است
\\
مسلمان گر بدانستی که بت چیست
&&
بدانستی که دین در بت‌پرستی است
\\
وگر مشرک ز بت آگاه گشتی
&&
کجا در دین خود گمراه گشتی
\\
ندید او از بت الا خلق ظاهر
&&
بدین علت شد اندر شرع کافر
\\
تو هم گر زو ببینی حق پنهان
&&
به شرع اندر نخوانندت مسلمان
\\
ز اسلام مجازی گشت بیزار
&&
که را کفر حقیقی شد پدیدار
\\
درون هر بتی جانی است پنهان
&&
به زیر کفر ایمانی است پنهان
\\
همیشه کفر در تسبیح حق است
&&
و «ان من شیء» گفت اینجا چه دق است
\\
چه می‌گویم که دور افتادم از راه
&&
«فذرهم بعد ما جائت قل الله»
\\
بدان خوبی رخ بت را که آراست
&&
که گشتی بت‌پرست ار حق نمی‌خواست
\\
هم او کرد و هم او گفت و هم او بود
&&
نکو کرد و نکو گفت و نکو بود
\\
یکی بین و یکی گوی و یکی دان
&&
بدین ختم آمد اصل و فرع ایمان
\\
نه من می‌گویم این بشنو ز قرآن
&&
تفاوت نیست اندر خلق رحمان
\\
\end{longtable}
\end{center}
