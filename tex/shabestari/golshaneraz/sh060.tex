\begin{center}
\section*{بخش ۶۰ - اشارت به زنار}
\label{sec:sh060}
\addcontentsline{toc}{section}{\nameref{sec:sh060}}
\begin{longtable}{l p{0.5cm} r}
نظر کردم بدیدم اصل هر کار
&&
نشان خدمت آمد عقد زنار
\\
نباشد اهل دانش را مؤول
&&
ز هر چیزی مگر بر وضع اول
\\
میان در بند چون مردان به مردی
&&
درآ در زمرهٔ «اوفوا بعهدی»
\\
به رخش علم و چوگان عبادت
&&
اگر چه خلق بسیار آفریدند
\\
ز میدان در ربا گوی سعادت
&&
تو را از بهر این کار آفریدند
\\
پدر چون علم و مادر هست اعمال
&&
به سان قرةالعین است احوال
\\
نباشد بی‌پدر انسان شکی نیست
&&
مسیح اندر جهان بیش از یکی نیست
\\
رها کن ترهات و شطح و طامات
&&
خیال خلوت و نور کرامات
\\
کرامات تو اندر حق پرستی است
&&
جز این کبر و ریا و عجب و هستی است
\\
در این هر چیز کان نز باب فقر است
&&
همه اسباب استدراج و مکر است
\\
ز ابلیس لعین بی سعادت
&&
شود صادر هزاران خرق عادت
\\
گه از دیوارت آید گاهی از بام
&&
گهی در دل نشیند گه در اندام
\\
همی‌داند ز تو احوال پنهان
&&
در آرد در تو کفر و فسق و عصیان
\\
شد ابلیست امام و در پسی تو
&&
بدو لیکن بدین‌ها کی رسی تو
\\
کرامات تو گر در خودنمایی است
&&
تو فرعونی و این دعوی خدایی است
\\
کسی کو راست با حق آشنایی
&&
نیاید هرگز از وی خودنمایی
\\
همه روی تو در خلق است زنهار
&&
مکن خود را بدین علت گرفتار
\\
چو با عامه نشینی مسخ گردی
&&
چه جای مسخ یک سر نسخ گردی
\\
مبادا هیچ با عامت سر و کار
&&
که از فطرت شوی ناگه نگونسار
\\
تلف کردی به هرزه نازنین عمر
&&
نگویی در چه کاری با چنین عمر
\\
به جمعیت لقب کردند تشویش
&&
خری را پیشوا کردی زهی ریش
\\
فتاده سروری اکنون به جهال
&&
از این گشتند مردم جمله بدحال
\\
نگر دجال اعور تا چگونه
&&
فرستاده است در عالم نمونه
\\
نمونه باز بین ای مرد حساس
&&
خر او را که نامش هست جساس
\\
خران را بین همه در تنگ آن خر
&&
شده از جهل پیش‌آهنگ آن خر
\\
چو خواجه قصهٔ آخر زمان کرد
&&
به چندین جا از این معنی نشان کرد
\\
ببین اکنون که کور و کر شبان شد
&&
علوم دین همه بر آسمان شد
\\
نماند اندر میانه رفق و آزرم
&&
نمی‌دارد کسی از جاهلی شرم
\\
همه احوال عالم باژگون است
&&
اگر تو عاقلی بنگر که چون است
\\
کسی کارباب لعن و طرد و مقت است
&&
پدر نیکو بد، اکنون شیخ وقت است
\\
خضر می‌کشت آن فرزند طالح
&&
که او را بد پدر با جد صالح
\\
کنون با شیخ خود کردی تو ای خر
&&
خری را کز خری هست از تو خرتر
\\
چو او «یعرف الهر من البر»
&&
چگونه پاک گرداند تو را سر
\\
و گر دارد نشان باب خود پور
&&
چه گویم چون بود «نور علی نور»
\\
پسر کو نیک‌رای و نیک‌بخت است
&&
چو میوه زبده و سر درخت است
\\
ولیکن شیخ دین کی گردد آن کو
&&
نداند نیک از بد بد ز نیکو
\\
مریدی علم دین آموختن بود
&&
چراغ دل ز نور افروختن بود
\\
کسی از مرده علم آموخت هرگز
&&
ز خاکستر چراغ افروخت هرگز
\\
مرا در دل همی آید کز این کار
&&
ببندم بر میان خویش زنار
\\
نه زان معنی که من شهرت ندارم
&&
که دارم لیک از وی هست عارم
\\
شریکم چون خسیس آمد در این کار
&&
خمولم بهتر از شهرت به بسیار
\\
دگرباره رسیدالهامم از حق
&&
که بر حکمت مگیر از ابلهی دق
\\
اگر کناس نبود در ممالک
&&
همه خلق اوفتند اندر مهالک
\\
بود جنسیت آخر علت ضم
&&
چنین آمد جهان والله اعلم
\\
ولیک از صحبت نااهل بگریز
&&
عبادت خواهی از عادت بپرهیز
\\
نگردد جمع با عادت عبادت
&&
عبادت می‌کنی بگذر ز عادت
\\
\end{longtable}
\end{center}
