\begin{center}
\section*{بخش ۲۳ - تمثیل در بیان سیر مراتب نبوت و ولایت}
\label{sec:sh023}
\addcontentsline{toc}{section}{\nameref{sec:sh023}}
\begin{longtable}{l p{0.5cm} r}
چه نور آفتاب از شب جدا شد
&&
تو را صبح و طلوع و استوا شد
\\
دگر باره ز دور چرخ دوار
&&
زوال و عصر و مغرب شد پدیدار
\\
بود نور نبی خورشید اعظم
&&
گه از موسی پدید و گه ز آدم
\\
اگر تاریخ عالم را بخوانی
&&
مراتب را یکایک باز دانی
\\
ز خور هر دم ظهور سایه‌ای شد
&&
که آن معراج دین را پایه‌ای شد
\\
زمان خواجه وقت استوا بود
&&
که از هر ظل و ظلمت مصطفا بود
\\
به خط استوا بر قامت راست
&&
ندارد سایه پیش و پس چپ و راست
\\
چو کرد او بر صراط حق اقامت
&&
به امر «فاستقم» می‌داشت قامت
\\
نبودش سایه کان دارد سیاهی
&&
زهی نور خدا ظل الهی
\\
ورا قبله میان غرب و شرق است
&&
ازیرا در میان نور غرق است
\\
به دست او چو شیطان شد مسلمان
&&
به زیر پای او شد سایه پنهان
\\
مراتب جمله زیر پایهٔ اوست
&&
وجود خاکیان از سایهٔ اوست
\\
ز نورش شد ولایت سایه گستر
&&
مشارق با مغارب شد برابر
\\
ز هر سایه که اول گشت حاصل
&&
در آخر شد یکی دیگر مقابل
\\
کنون هر عالمی باشد ز امت
&&
رسولی را مقابل در نبوت
\\
نبی چون در نبوت بود اکمل
&&
بود از هر ولی ناچار افضل
\\
ولایت شد به خاتم جمله ظاهر
&&
بر اول نقطه هم ختم آمد آخر
\\
از او عالم شود پر امن و ایمان
&&
جماد و جانور یابد از او جان
\\
نماند در جهان یک نفس کافر
&&
شود عدل حقیقی جمله ظاهر
\\
بود از سر وحدت واقف حق
&&
در او پیدا نماید وجه مطلق
\\
\end{longtable}
\end{center}
