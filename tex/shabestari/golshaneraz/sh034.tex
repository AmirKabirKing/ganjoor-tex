\begin{center}
\section*{بخش ۳۴ - جواب}
\label{sec:sh034}
\addcontentsline{toc}{section}{\nameref{sec:sh034}}
\begin{longtable}{l p{0.5cm} r}
وصال حق ز خلقیت جدایی است
&&
ز خود بیگانه گشتن آشنایی است
\\
چو ممکن گرد امکان برفشاند
&&
به جز واجب دگر چیزی نماند
\\
وجود هر دو عالم چون خیال است
&&
که در وقت بقا عین زوال است
\\
نه مخلوق است آن کو گشت واصل
&&
نگوید این سخن را مرد کامل
\\
عدم کی راه یابد اندر این باب
&&
چه نسبت خاک را با رب ارباب
\\
عدم چبود که با حق واصل آید
&&
وز او سیر و سلوکی حاصل آید
\\
تو معدوم و عدم پیوسته ساکن
&&
به واجب کی رسد معدوم ممکن
\\
اگر جانت شود زین معنی آگاه
&&
بگویی در زمان استغفرالله
\\
ندارد هیچ جوهر بی‌عرض عین
&&
عرض چبود که لا یبقی زمانین
\\
حکیمی کاندر این فن کرد تصنیف
&&
به طول و عرض و عمقش کرد تعریف
\\
هیولی چیست جز معدوم مطلق
&&
که می‌گردد بدو صورت محقق
\\
چو صورت بی‌هیولی در قدم نیست
&&
هیولی نیز بی او جز عدم نیست
\\
شده اجسام عالم زین دو معدوم
&&
که جز معدوم از ایشان نیست معلوم
\\
ببین ماهیت را بی کم و بیش
&&
نه معدوم و نه موجود است در خویش
\\
نظر کن در حقیقت سوی امکان
&&
که او بی‌هستی آمد عین نقصان
\\
وجود اندر کمال خویش ساری است
&&
تعین‌ها امور اعتباری است
\\
امور اعتباری نیست موجود
&&
عدد بسیار و یک چیز است معدود
\\
جهان را نیست هستی جز مجازی
&&
سراسر کار او لهو است و بازی
\\
\end{longtable}
\end{center}
