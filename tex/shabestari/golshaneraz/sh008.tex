\begin{center}
\section*{بخش ۸ - تمثیل در بیان ظهور خورشید حقیقت در آیینه کائنات}
\label{sec:sh008}
\addcontentsline{toc}{section}{\nameref{sec:sh008}}
\begin{longtable}{l p{0.5cm} r}
اگر خواهی که بینی چشمهٔ خور
&&
تو را حاجت فتد با جسم دیگر
\\
چو چشم سر ندارد طاقت تاب
&&
توان خورشید تابان دید در آب
\\
از او چون روشنی کمتر نماید
&&
در ادراک تو حالی می‌فزاید
\\
عدم آیینهٔ هستی است مطلق
&&
کز او پیداست عکس تابش حق
\\
عدم چون گشت هستی را مقابل
&&
در او عکسی شد اندر حال حاصل
\\
شد آن وحدت از این کثرت پدیدار
&&
یکی را چون شمردی گشت بسیار
\\
عدد گرچه یکی دارد بدایت
&&
ولیکن نبودش هرگز نهایت
\\
عدم در ذات خود چون بود صافی
&&
از او با ظاهر آمد گنج مخفی
\\
حدیث «کنت کنزا» را فرو خوان
&&
که تا پیدا ببینی گنج پنهان
\\
عدم آیینه عالم عکس و انسان
&&
چو چشم عکس در وی شخص پنهان
\\
تو چشم عکسی و او نور دیده است
&&
به دیده دیده را هرگز که دیده است
\\
جهان انسان شد و انسان جهانی
&&
از این پاکیزه‌تر نبود بیانی
\\
چو نیکو بنگری در اصل این کار
&&
هم او بیننده هم دیده است و دیدار
\\
حدیث قدسی این معنی بیان کرد
&&
و بی یسمع و بی یبصر عیان کرد
\\
جهان را سر به سر آیینه‌ای دان
&&
به هر یک ذره در صد مهر تابان
\\
اگر یک قطره را دل بر شکافی
&&
برون آید از آن صد بحر صافی
\\
به هر جزوی ز خاک ار بنگری راست
&&
هزاران آدم اندر وی هویداست
\\
به اعضا پشه‌ای همچند فیل است
&&
در اسما قطره‌ای مانند نیل است
\\
درون حبه‌ای صد خرمن آمد
&&
جهانی در دل یک ارزن آمد
\\
به پر پشه‌ای در جای جانی
&&
درون نقطهٔ چشم آسمانی
\\
بدان خردی که آمد حبهٔ دل
&&
خداوند دو عالم راست منزل
\\
در او در جمع گشته هر دو عالم
&&
گهی ابلیس گردد گاه آدم
\\
ببین عالم همه در هم سرشته
&&
ملک در دیو و دیو اندر فرشته
\\
همه با هم به هم چون دانه و بر
&&
ز کافر مؤمن و مؤمن ز کافر
\\
به هم جمع آمده در نقطهٔ حال
&&
همه دور زمان روز و مه و سال
\\
ازل عین ابد افتاد با هم
&&
نزول عیسی و ایجاد آدم
\\
ز هر یک نقطه زین دور مسلسل
&&
هزاران شکل می‌گردد مشکل
\\
ز هر یک نقطه دوری گشته دایر
&&
هم او مرکز هم او در دور سایر
\\
اگر یک ذره را برگیری از جای
&&
خلل یابد همه عالم سراپای
\\
همه سرگشته و یک جزو از ایشان
&&
برون ننهاده پای از حد امکان
\\
تعین هر یکی را کرده محبوس
&&
به جزویت ز کلی گشته مایوس
\\
تو گویی دائما در سیر و حبسند
&&
که پیوسته میان خلع و لبسند
\\
همه در جنبش و دائم در آرام
&&
نه آغاز یکی پییدا نه انجام
\\
همه از ذات خود پیوسته آگاه
&&
وز آنجا راه برده تا به درگاه
\\
به زیر پردهٔ هر ذره پنهان
&&
جمال جانفزای روی جانان
\\
\end{longtable}
\end{center}
