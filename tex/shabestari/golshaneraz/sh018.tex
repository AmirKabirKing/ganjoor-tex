\begin{center}
\section*{بخش ۱۸ - قاعده در بیان سیر نزول و مراتب صعود آدمی}
\label{sec:sh018}
\addcontentsline{toc}{section}{\nameref{sec:sh018}}
\begin{longtable}{l p{0.5cm} r}
بدان اول که تا چون گشت موجود
&&
کز او انسان کامل گشت مولود
\\
در اطوار جمادی بود پیدا
&&
پس از روح اضافی گشت دانا
\\
پس آنگه جنبشی کرد او ز قدرت
&&
پس از وی شد ز حق صاحب ارادت
\\
به طفلی کرد باز احساس عالم
&&
در او بالفعل شد وسواس عالم
\\
چو جزویات شد بر وی مرتب
&&
به کلیات ره برد از مرکب
\\
غضب شد اندر او پیدا و شهوت
&&
وز ایشان خاست بخل و حرص و نخوت
\\
به فعل آمد صفتهای ذمیمه
&&
بتر شد از دد و دیو و بهیمه
\\
تنزل را بود این نقطه اسفل
&&
که شد با نقطهٔ وحدت مقابل
\\
شد از افعال کثرت بی‌نهایت
&&
مقابل گشت از این رو با بدایت
\\
اگر گردد مقید اندر این دام
&&
به گمراهی بود کمتر ز انعام
\\
وگر نوری رسد از عالم جان
&&
ز فیض جذبه یا از عکس برهان
\\
دلش با لطف حق همراز گردد
&&
از آن راهی که آمد باز گردد
\\
ز جذبه یا ز برهان حقیقی
&&
رهی یابد به ایمان حقیقی
\\
کند یک رجعت از سجین فجار
&&
رخ آرد سوی علیین ابرار
\\
به توبه متصف گردد در آن دم
&&
شود در اصطفی ز اولاد آدم
\\
ز افعال نکوهیده شود پاک
&&
چو ادریس نبی آید بر افلاک
\\
چو یابد از صفات بد نجاتی
&&
شود چون نوح از آن صاحب ثباتی
\\
نماند قدرت جزویش در کل
&&
خلیل آسا شود صاحب توکل
\\
ارادت با رضای حق شود ضم
&&
رود چون موسی اندر باب اعظم
\\
ز علم خویشتن یابد رهائی
&&
چو عیسای نبی گردد سمائی
\\
دهد یکباره هستی را به تاراج
&&
درآید از پی احمد به معراج
\\
رسد چون نقطهٔ آخر به اول
&&
در آنجا نه ملک گنجد نه مرسل
\\
\end{longtable}
\end{center}
