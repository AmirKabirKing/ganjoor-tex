\begin{center}
\section*{بخش ۴۰ - تمثیل در بیان ماهیت صورت و معنی}
\label{sec:sh040}
\addcontentsline{toc}{section}{\nameref{sec:sh040}}
\begin{longtable}{l p{0.5cm} r}
شنیدم من که اندر ماه نیسان
&&
صدف بالا رود از قعر عمان
\\
ز شیب قعر بحر آید برافراز
&&
به روی بحر بنشیند دهن باز
\\
بخاری مرتفع گردد ز دریا
&&
فرو بارد به امر حق تعالی
\\
چکد اندر دهانش قطره‌ای چند
&&
شود بسته دهان او به صد بند
\\
رود با قعر دریا با دلی پر
&&
شود آن قطرهٔ باران یکی در
\\
به قعر اندر رود غواص دریا
&&
از آن آرد برون لؤلؤی لالا
\\
تن تو ساحل و هستی چو دریاست
&&
بخارش فیض و باران علم اسماست
\\
خرد غواص آن بحر عظیم است
&&
که او را صد جواهر در گلیم است
\\
دل آمد علم را مانند یک ظرف
&&
صدف با علم دل صوت است با حرف
\\
نفس گردد روان چون برق لامع
&&
رسد زو حرفها با گوش سامع
\\
صدف بشکن برون کن در شهوار
&&
بیفکن پوست مغز نغز بردار
\\
لغت با اشتقاق و نحو با صرف
&&
همی‌گردد همه پیرامن حرف
\\
هر آن کو جمله عمر خود در این کرد
&&
به هرزه صرف عمر نازنین کرد
\\
ز جوزش قشر سبز افتاد در دست
&&
نیابد مغز هر کو پوست نشکست
\\
بلی بی پوست ناپخته است هر مغز
&&
ز علم ظاهر آمد علم دین نغز
\\
ز من جان برادر پند بنیوش
&&
به جان و دل برو در علم دین کوش
\\
که عالم در دو عالم سروری یافت
&&
اگر کهتر بد از وی مهتری یافت
\\
عمل کان از سر احوال باشد
&&
بسی بهتر ز علم قال باشد
\\
ولی کاری که از آب و گل آید
&&
نه چون علم است کان کار از دل آید
\\
میان جسم و جان بنگر چه فرق است
&&
که این را غرب گیری آن چو شرق است
\\
از اینجا باز دان احوال و اعمال
&&
به نسبت با علوم قال با حال
\\
نه علم است آنکه دارد میل دنیی
&&
که صورت دارد اما نیست معنی
\\
نگردد علم هرگز جمع با آز
&&
ملک خواهی سگ از خود دور انداز
\\
علوم دین ز اخلاق فرشته است
&&
نباشد در دلی کو سگ سرشت است
\\
حدیث مصطفی آخر همین است
&&
نکو بشنو که البته چنین است
\\
درون خانه‌ای چون هست صورت
&&
فرشته ناید اندر وی ضرورت
\\
برو بزدای روی تختهٔ دل
&&
که تا سازد ملک پیش تو منزل
\\
از او تحصیل کن علم وراثت
&&
ز بهر آخرت می‌کن حراثت
\\
کتاب حق بخوان از نفس و آفاق
&&
مزین شو به اصل جمله اخلاق
\\
\end{longtable}
\end{center}
