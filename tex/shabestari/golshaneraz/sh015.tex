\begin{center}
\section*{بخش ۱۵ - جواب}
\label{sec:sh015}
\addcontentsline{toc}{section}{\nameref{sec:sh015}}
\begin{longtable}{l p{0.5cm} r}
دگر کردی سؤال از من که من چیست
&&
مرا از من خبر کن تا که من کیست
\\
چو هست مطلق آید در اشارت
&&
به لفظ من کنند از وی عبارت
\\
حقیقت کز تعین شد معین
&&
تو او را در عبارت گفته‌ای من
\\
من و تو عارض ذات وجودیم
&&
مشبکهای مشکات وجودیم
\\
همه یک نور دان اشباح و ارواح
&&
گه از آیینه پیدا گه ز مصباح
\\
تو گویی لفظ من در هر عبارت
&&
به سوی روح می‌باشد اشارت
\\
چو کردی پیشوای خود خرد را
&&
نمی‌دانی ز جزو خویش خود را
\\
برو ای خواجه خود را نیک بشناس
&&
که نبود فربهی مانند آماس
\\
من تو برتر از جان و تن آمد
&&
که این هر دو ز اجزای من آمد
\\
به لفظ من نه انسان است مخصوص
&&
که تا گویی بدان جان است مخصوص
\\
یکی ره برتر از کون و مکان شو
&&
جهان بگذار و خود در خود جهان شو
\\
ز خط وهمیی‌های هویت
&&
دو چشمی می‌شود در وقت ریت
\\
نماند در میانه رهرو راه
&&
چو های هو شود ملحق به الله
\\
بود هستی بهشت امکان چو دوزخ
&&
من و تو در میان مانند برزخ
\\
چو برخیزد تو را این پرده از پیش
&&
نماند نیز حکم مذهب و کیش
\\
همه حکم شریعت از من توست
&&
که این بربستهٔ جان و تن توست
\\
من تو چون نماند در میانه
&&
چه کعبه چه کنشت چه دیرخانه
\\
تعین نقطهٔ وهمی است بر عین
&&
چو صافی گشت غین تو شود عین
\\
دو خطوه بیش نبود راه سالک
&&
اگر چه دارد آن چندین مهالک
\\
یک از های هویت در گذشتن
&&
دوم صحرای هستی در نوشتن
\\
در این مشهد یکی شد جمع و افراد
&&
چو واحد ساری اندر عین اعداد
\\
تو آن جمعی که عین وحدت آمد
&&
تو آن واحد که عین کثرت آمد
\\
کسی این راه داند کو گذر کرد
&&
ز جز وی سوی کلی یک سفر کرد
\\
\end{longtable}
\end{center}
