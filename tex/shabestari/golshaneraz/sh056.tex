\begin{center}
\section*{بخش ۵۶ - جواب}
\label{sec:sh056}
\addcontentsline{toc}{section}{\nameref{sec:sh056}}
\begin{longtable}{l p{0.5cm} r}
شراب و شمع و شاهد عین معنی است
&&
که در هر صورتی او را تجلی است
\\
شراب و شمع سکر و نور عرفان
&&
ببین شاهد که از کس نیست پنهان
\\
شراب اینجا زجاجه شمع مصباح
&&
بود شاهد فروغ نور ارواح
\\
ز شاهد بر دل موسی شرر شد
&&
شرابش آتش و شمعش شجر شد
\\
شراب و شمع جام و نور اسری است
&&
ولی شاهد همان آیات کبری است
\\
شراب بیخودی در کش زمانی
&&
مگر از دست خود یابی امانی
\\
بخور می تا ز خویشت وارهاند
&&
وجود قطره با دریا رساند
\\
شرابی خور که جامش روی یار است
&&
پیاله چشم مست باده‌خوار است
\\
شرابی را طلب بی‌ساغر و جام
&&
شراب باده خوار و ساقی آشام
\\
شرابی خور ز جام وجه باقی
&&
«سقاهم ربهم» او راست ساقی
\\
طهور آن می بود کز لوث هستی
&&
تو را پاکی دهد در وقت مستی
\\
بخور می وارهان خود را ز سردی
&&
که بد مستی به است از نیک مردی
\\
کسی کو افتد از درگاه حق دور
&&
حجاب ظلمت او را بهتر از نور
\\
که آدم را ز ظلمت صد مدد شد
&&
ز نور ابلیس ملعون ابد شد
\\
اگر آیینهٔ دل را زدوده است
&&
چو خود را بیند اندر وی چه سود است
\\
ز رویش پرتوی چون بر می افتاد
&&
بسی شکل حبابی بر وی افتاد
\\
جهان جان در او شکل حباب است
&&
حبابش اولیائی را قباب است
\\
شده زو عقل کل حیران و مدهوش
&&
فتاده نفس کل را حلقه در گوش
\\
همه عالم چو یک خمخانهٔ اوست
&&
دل هر ذره‌ای پیمانهٔ اوست
\\
خرد مست و ملایک مست و جان مست
&&
هوا مست و زمین مست آسمان مست
\\
فلک سرگشته از وی در تکاپوی
&&
هوا در دل به امید یکی بوی
\\
ملایک خورده صاف از کوزهٔ پاک
&&
به جرعه ریخته دردی بر این خاک
\\
عناصر گشته زان یک جرعه سر خوش
&&
فتاده گه در آب و گه در آتش
\\
ز بوی جرعه‌ای که افتاد بر خاک
&&
برآمد آدمی تا شد بر افلاک
\\
ز عکس او تن پژمرده جان یافت
&&
ز تابش جان افسرده روان یافت
\\
جهانی خلق از او سرگشته دائم
&&
ز خان و مان خود برگشته دائم
\\
یکی از بوی دردش ناقل آمد
&&
یکی از نیم جرعه عاقل آمد
\\
یکی از جرعه‌ای گردیده صادق
&&
یکی از یک صراحی گشته عاشق
\\
یکی دیگر فرو برده به یک بار
&&
می و میخانه و ساقی و میخوار
\\
کشیده جمله و مانده دهن باز
&&
زهی دریا دل رند سرافراز
\\
در آشامیده هستی را به یک بار
&&
فراغت یافته ز اقرار و انکار
\\
شده فارغ ز زهد خشک و طامات
&&
گرفته دامن پیر خرابات
\\
\end{longtable}
\end{center}
