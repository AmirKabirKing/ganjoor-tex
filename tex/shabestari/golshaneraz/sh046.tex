\begin{center}
\section*{بخش ۴۶ - قاعده در بیان معنی حشر}
\label{sec:sh046}
\addcontentsline{toc}{section}{\nameref{sec:sh046}}
\begin{longtable}{l p{0.5cm} r}
ز تو هر فعل که اول گشت صادر
&&
بر آن گردی به باری چند قادر
\\
به هر باری اگر نفع است اگر ضر
&&
شود در نفس تو چیزی مدخر
\\
به عادت حالها با خوی گردد
&&
به مدت میوه‌ها خوش بوی گردد
\\
از آن آموخت انسان پیشه‌ها را
&&
وز آن ترکیب کرد اندیشه‌ها را
\\
همه افعال و اقوال مدخر
&&
هویدا گردد اندر روز محشر
\\
چو عریان گردی از پیراهن تن
&&
شود عیب و هنر یکباره روشن
\\
تنت باشد ولیکن بی‌کدورت
&&
که بنماید از او چون آب صورت
\\
همه پیدا شود آنجا ضمایر
&&
فرو خوان آیت «تبلی السرائر»
\\
دگر باره به وفق عالم خاص
&&
شود اخلاق تو اجسام و اشخاص
\\
چنان کز قوت عنصر در اینجا
&&
موالید سه گانه گشت پیدا
\\
همه اخلاق تو در عالم جان
&&
گهی انوار گردد گاه نیران
\\
تعین مرتفع گردد ز هستی
&&
نماند درنظر بالا و پستی
\\
نماند مرگت اندر دار حیوان
&&
به یک رنگی برآید قالب و جان
\\
بود پا و سر و چشم تو چون دل
&&
شود صافی ز ظلمت صورت گل
\\
کند انوار حق بر تو تجلی
&&
ببینی بی‌جهت حق را تعالی
\\
دو عالم را همه بر هم زنی تو
&&
ندانم تا چه مستی‌ها کنی تو
\\
«سقاهم ربهم» چبود بیندیش
&&
«طهورا» چیست صافی گشتن از خویش
\\
زهی شربت زهی لذت زهی ذوق
&&
زهی حیرت زهی دولت زهی شوق
\\
خوشا آن دم که ما بی‌خویش باشیم
&&
غنی مطلق و درویش باشیم
\\
نه دین نه عقل نه تقوی نه ادراک
&&
فتاده مست و حیران بر سر خاک
\\
بهشت و حور و خلد آنجا چه سنجد
&&
که بیگانه در آن خلوت نگنجد
\\
چو رویت دیدم و خوردم از آن می
&&
ندانم تا چه خواهد شد پس از وی
\\
پی هر مستیی باشد خماری
&&
از این اندیشه دل خون گشت باری
\\
\end{longtable}
\end{center}
