\begin{center}
\section*{بخش ۵۳ - اشارت به رخ و خط}
\label{sec:sh053}
\addcontentsline{toc}{section}{\nameref{sec:sh053}}
\begin{longtable}{l p{0.5cm} r}
رخ اینجا مظهر حسن خدایی است
&&
مراد از خط جناب کبریایی است
\\
رخش خطی کشید اندر نکویی
&&
که از ما نیست بیرون خوبرویی
\\
خط آمد سبزه‌زار عالم جان
&&
از آن کردند نامش دار حیوان
\\
ز تاریکی زلفش روز شب کن
&&
ز خطش چشمهٔ حیوان طلب کن
\\
خضروار از مقام بی‌نشانی
&&
بخور چون خطش آب زندگانی
\\
اگر روی و خطش بینی تو بی‌شک
&&
بدانی کثرت از وحدت یکایک
\\
ز زلفش باز دانی کار عالم
&&
ز خطش باز خوانی سر مبهم
\\
کسی گر خطش از روی نکو دید
&&
دل من روی او در خط او دید
\\
مگر رخسار او سبع المثانی است
&&
که هر حرفی از او بحر معانی است
\\
نهفته زیر هر مویی از او باز
&&
هزاران بحر علم از عالم راز
\\
ببین بر آب قلبت عرش رحمان
&&
ز خط عارض زیبای جانان
\\
\end{longtable}
\end{center}
