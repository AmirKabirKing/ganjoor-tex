\begin{center}
\section*{بخش ۱۲ - تمثیل در بیان وحدت کارخانه عالم}
\label{sec:sh012}
\addcontentsline{toc}{section}{\nameref{sec:sh012}}
\begin{longtable}{l p{0.5cm} r}
تو گویی هست این افلاک دوار
&&
به گردش روز و شب چون چرخ فخار
\\
وز او هر لحظه‌ای دانای داور
&&
ز آب وگل کند یک ظرف دیگر
\\
هر آنچه در مکان و در زمان است
&&
ز یک استاد و از یک کارخانه است
\\
کواکب گر همه اهل کمالند
&&
چرا هر لحظه در نقص و وبالند
\\
همه درجای و سیر و لون و اشکال
&&
چرا گشتند آخر مختلف حال
\\
چرا گه در حضیض و گه در اوجند
&&
گهی تنها فتاده گاه زوجند
\\
دل چرخ از چه شد آخر پر آتش
&&
ز شوق کیست او اندر کشاکش
\\
همه انجم بر او گردان پیاده
&&
گهی بالا و گه شیب اوفتاده
\\
عناصر باد و آب و آتش و خاک
&&
گرفته جای خود در زیر افلاک
\\
ملازم هر یکی در منزل خویش
&&
بننهد پای یک ذره پس و پیش
\\
چهار اضداد در طبع مراکز
&&
به هم جمع آمده، کس دیده هرگز؟
\\
مخالف هر یکی در ذات و صورت
&&
شده یک چیز از حکم ضرورت
\\
موالید سه گانه گشته ز ایشان
&&
جماد آنگه نبات آنگاه حیوان
\\
هیولی را نهاده در میانه
&&
ز صورت گشته صافی صوفیانه
\\
همه از امر وحکم داد داور
&&
به جان استاده و گشته مسخر
\\
جماد از قهر بر خاک اوفتاده
&&
نبات از مهر بر پای ایستاده
\\
نزوع جانور از صدق و اخلاص
&&
پی ابقای جنس و نوع و اشخاص
\\
همه بر حکم داور داده اقرار
&&
مر او را روز و شب گشته طلبکار
\\
\end{longtable}
\end{center}
