\begin{center}
\section*{بخش ۵۴ - اشارت به خال}
\label{sec:sh054}
\addcontentsline{toc}{section}{\nameref{sec:sh054}}
\begin{longtable}{l p{0.5cm} r}
بر آن رخ نقطهٔ خالش بسیط است
&&
که اصل مرکز دور محیط است
\\
از او شد خط دور هر دو عالم
&&
وز او شد خط نفس و قلب آدم
\\
از آن حال دل پرخون تباه است
&&
که عکس نقطهٔ خال سیاه است
\\
ز خالش حال دل جز خون شدن نیست
&&
کز آن منزل ره بیرون شدن نیست
\\
به وحدت در نباشد هیچ کثرت
&&
دو نقطه نبود اندر اصل وحدت
\\
ندانم خال او عکس دل ماست
&&
و یا دل عکس خال روی زیباست
\\
ز عکس خال او دل گشت پیدا
&&
و یا عکس دل آنجا شد هویدا
\\
دل اندر روی او یا اوست در دل
&&
به من پوشیده شد این راز مشکل
\\
اگر هست این دل ما عکس آن خال
&&
چرا می‌باشد آخر مختلف حال
\\
گهی چون چشم مخمورش خراب است
&&
گهی چون زلف او در اضطراب است
\\
گهی روشن چو آن روی چو ماه است
&&
گهی تاریک چون خال سیاه است
\\
گهی مسجد بود گاهی کنشت است
&&
گهی دوزخ بود گاهی بهشت است
\\
گهی برتر شود از هفتم افلاک
&&
گهی افتد به زیر تودهٔ خاک
\\
پس از زهد و ورع گردد دگر بار
&&
شراب و شمع و شاهد را طلبکار
\\
\end{longtable}
\end{center}
