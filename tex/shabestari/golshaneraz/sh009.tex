\begin{center}
\section*{بخش ۹ - قاعده در شناخت عوالم پنهان و شرایط عروج بدان عوالم}
\label{sec:sh009}
\addcontentsline{toc}{section}{\nameref{sec:sh009}}
\begin{longtable}{l p{0.5cm} r}
تو از عالم همین لفظی شنیدی
&&
بیا برگو که از عالم چه دیدی
\\
چه دانستی ز صورت یا ز معنی
&&
چه باشد آخرت چون است دنیی
\\
بگو سیمرغ و کوه قاف چبود
&&
بهشت و دوزخ و اعراف چبود
\\
کدام است آن جهان کان نیست پیدا
&&
که یک روزش بود یک سال اینجا
\\
همین عالم نبود آخر که دیدی
&&
نه «ما لا تبصرون» آخر شنیدی
\\
بیا بنما که جابلقا کدام است
&&
جهان شهر جابلسا کدام است
\\
مشارق با مغارب را بیندیش
&&
چو این عالم ندارد از یکی بیش
\\
بیان «مثلهن» از ابن عباس
&&
شنو پس خویشتن را نیک بشناس
\\
تو در خوابی و این دیدن خیال است
&&
هر آنچه دیده‌ای از وی مثال است
\\
به صبح حشر چون گردی تو بیدار
&&
بدانی کین همه وهم است و پندار
\\
چو برخیزد خیال چشم احول
&&
زمین و آسمان گردد مبدل
\\
چو خورشید نهان بنمایدت چهر
&&
نماند نور ناهید و مه و مهر
\\
فتد یک تاب از او بر سنگ خاره
&&
شود چون پشم رنگین پاره پاره
\\
بکن اکنون که کردن می‌توانی
&&
چون نتوانی چه سود آن را که دانی
\\
چه می‌گویم حدیث عالم دل
&&
تو را ای سرنشیب پای در گل
\\
جهان آن تو و تو مانده عاجز
&&
ز تو محرومتر کس دیده هرگز
\\
چو محبوسان به یک منزل نشسته
&&
به دست عجز پای خویش بسته
\\
نشستی چون زنان در کوی ادبار
&&
نمی‌داری ز جهل خویشتن عار
\\
دلیران جهان آغشته در خون
&&
تو سرپوشیده ننهی پای بیرون
\\
چه کردی فهم از دین العجایز
&&
که بر خود جهل می‌داری تو جایز
\\
زنان چون ناقصات عقل و دینند
&&
چرا مردان ره ایشان گزینند
\\
اگر مردی برون آی و سفر کن
&&
هر آنچ آید به پیشت زان گذر کن
\\
میاسا روز و شب اندر مراحل
&&
مشو موقوف همراه و رواحل
\\
خلیل آسا برو حق را طلب کن
&&
شبی را روز و روزی را به شب کن
\\
ستاره با مه و خورشید اکبر
&&
بود حس و خیال و عقل انور
\\
بگردان زین همه ای راهرو روی
&&
همیشه «لا احب الافلین» گوی
\\
و یا چون موسی عمران در این راه
&&
برو تا بشنوی «انی انا الله»
\\
تو را تا کوه هستی پیش باقی است
&&
صدای لفظ «ارنی» «لن ترانی» است
\\
حقیقت کهربا ذات تو کاه است
&&
اگر کوه تویی نبود چه راه است
\\
تجلی گر رسد بر کوه هستی
&&
شود چون خاک ره هستی ز پستی
\\
گدایی گردد از یک جذبه شاهی
&&
به یک لحظه دهد کوهی به کاهی
\\
برو اندر پی خواجه به اسری
&&
تماشا کن همه آیات کبری
\\
برون آی از سرای «ام هانی»
&&
بگو مطلق حدیث «من رآنی»
\\
گذاری کن ز کاف و نون کونین
&&
نشین بر قاف قرب «قاب قوسین»
\\
دهد حق مر تو را هرچ آن بخواهی
&&
نمایندت همه اشیا کماهی
\\
\end{longtable}
\end{center}
