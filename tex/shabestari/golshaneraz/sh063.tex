\begin{center}
\section*{بخش ۶۳ - اشارت به بت}
\label{sec:sh063}
\addcontentsline{toc}{section}{\nameref{sec:sh063}}
\begin{longtable}{l p{0.5cm} r}
بت ترسا بچه نوری است باهر
&&
که از روی بتان دارد مظاهر
\\
کند او جمله دلها را وشاقی
&&
گهی گردد مغنی گاه ساقی
\\
زهی مطرب که از یک نغمهٔ خوش
&&
زند در خرمن صد زاهد آتش
\\
زهی ساقی که او از یک پیاله
&&
کند بیخود دو صد هفتاد ساله
\\
رود در خانقه مست شبانه
&&
کند افسون صوفی را فسانه
\\
وگر در مسجد آید در سحرگاه
&&
بنگذارد در او یک مرد آگاه
\\
رود در مدرسه چون مست مستور
&&
فقیه از وی شود بیچاره مخمور
\\
ز عشقش زاهدان بیچاره گشته
&&
ز خان و مان خود آواره گشته
\\
یکی مؤمن دگر را کافر او کرد
&&
همه عالم پر از شور و شر او کرد
\\
خرابات از لبش معمور گشته
&&
مساجد از رخش پر نور گشته
\\
همه کار من از وی شد میسر
&&
بدو دیدم خلاص از نفس کافر
\\
دلم از دانش خود صد حجب داشت
&&
ز عجب و نخوت و تلبیس و پنداشت
\\
درآمد از درم آن مه سحرگاه
&&
مرا از خواب غفلت کرد آگاه
\\
ز رویش خلوت جان گشت روشن
&&
بدو دیدم که تا خود چیستم من
\\
چو کردم در رخ خوبش نگاهی
&&
برآمد از میان جانم آهی
\\
مرا گفتا که ای شیاد سالوس
&&
به سر شد عمرت اندر نام و ناموس
\\
ببین تا علم و زهد و کبر و پنداشت
&&
تو را ای نارسیده از که واداشت
\\
نظر کردن به رویم نیم ساعت
&&
همی‌ارزد هزاران ساله طاعت
\\
علی‌الجمله رخ آن عالم آرای
&&
مرا با من نمود آن دم سراپای
\\
سیه شد روی جانم از خجالت
&&
ز فوت عمر و ایام بطالت
\\
چو دید آن ماه کز روی چو خورشید
&&
بریدم من ز جان خویش امید
\\
یکی پیمانه پر کرد و به من داد
&&
که از آب وی آتش در من افتاد
\\
کنون گفت از می بی‌رنگ و بی‌بوی
&&
نقوش تختهٔ هستی فرو شوی
\\
چو آشامیدم آن پیمانه را پاک
&&
در افتادم ز مستی بر سر خاک
\\
کنون نه نیستم در خود نه هستم
&&
نه هشیارم نه مخمورم نه مستم
\\
گهی چون چشم او دارم سری خوش
&&
گهی چون زلف او باشم مشوش
\\
گهی از خوی خود در گلخنم من
&&
گهی از روی او در گلشنم من
\\
\end{longtable}
\end{center}
