\begin{center}
\section*{بخش ۴۵ - تمثیل در بیان اقسام مرگ و ظهور اطوار قیامت در لحظهٔ مرگ}
\label{sec:sh045}
\addcontentsline{toc}{section}{\nameref{sec:sh045}}
\begin{longtable}{l p{0.5cm} r}
اگر خواهی که این معنی بدانی
&&
تو را هم هست مرگ و زندگانی
\\
ز هرچ آن در جهان از زیر و بالاست
&&
مثالش در تن و جان تو پیداست
\\
جهان چون توست یک شخص معین
&&
تو او را گشته چون جان او تو را تن
\\
سه گونه نوع انسان را ممات است
&&
یکی هر لحظه وان بر حسب ذات است
\\
دو دیگر زان ممات اختیاری است
&&
سیم مردن مر او را اضطراری است
\\
چو مرگ و زندگی باشد مقابل
&&
سه نوع آمد حیاتش در سه منزل
\\
جهان را نیست مرگ اختیاری
&&
که آن را از همه عالم تو داری
\\
ولی هر لحظه می‌گردد مبدل
&&
در آخر هم شود مانند اول
\\
هر آنچ آن گردد اندر حشر پیدا
&&
ز تو در نزع می‌گردد هویدا
\\
تن تو چون زمین سر آسمان است
&&
حواست انجم و خورشید جان است
\\
چو کوه است استخوانهایی که سخت است
&&
نباتت موی و اطرافت درخت است
\\
تنت در وقت مردن از ندامت
&&
بلرزد چون زمین روز قیامت
\\
دماغ آشفته و جان تیره گردد
&&
حواست هم چو انجم خیره گردد
\\
مسامت گردد از خوی هم چو دریا
&&
تو در وی غرقه گشته بی سر و پا
\\
شود از جان‌کنش ای مرد مسکین
&&
ز سستی استخوانها پشم رنگین
\\
به هم پیچیده گردد ساق با ساق
&&
همه جفتی شود از جفت خود طاق
\\
چو روح از تن به کلیت جدا شد
&&
زمینت «قاع صف صف لاتری» شد
\\
بدین منوال باشد حال عالم
&&
که تو در خویش می‌بینی در آن دم
\\
بقا حق راست باقی جمله فانی است
&&
بیانش جمله در «سبع المثانی» است
\\
به «کل من علیها فان» بیان کرد
&&
«لفی خلق جدید» هم عیان کرد
\\
بود ایجاد و اعدام دو عالم
&&
چو خلق و بعث نفس ابن آدم
\\
همیشه خلق در خلق جدید است
&&
و گرچه مدت عمرش مدید است
\\
همیشه فیض فضل حق تعالی
&&
بود از شان خود اندر تجلی
\\
از آن جانب بود ایجاد و تکمیل
&&
وز این جانب بود هر لحظه تبدیل
\\
ولیکن چو گذشت این طور دنیی
&&
بقای کل بود در دار عقبی
\\
که هر چیزی که بینی بالضرورت
&&
دو عالم دارد از معنی و صورت
\\
وصال اولین عین فراق است
&&
مر آن دیگر ز «عند الله باق» است
\\
مظاهر چون فتد بر وفق ظاهر
&&
در اول می‌نماید عین آخر
\\
بقا اسم وجود آمد ولیکن
&&
به جایی کان بود سائر چو ساکن
\\
هر آنچ آن هست بالقوه در این دار
&&
به فعل آید در آن عالم به یک بار
\\
\end{longtable}
\end{center}
