\begin{center}
\section*{بخش ۱۰ - قاعده در تشبیه کتاب آفرینش به کتاب وحی}
\label{sec:sh010}
\addcontentsline{toc}{section}{\nameref{sec:sh010}}
\begin{longtable}{l p{0.5cm} r}
به نزد آنکه جانش در تجلی است
&&
همه عالم کتاب حق تعالی است
\\
عرض اعراب و جوهر چون حروف است
&&
مراتب همچو آیات وقوف است
\\
از او هر عالمی چون سوره‌ای خاص
&&
یکی زان فاتحه و آن دیگر اخلاص
\\
نخستین آیتش عقل کل آمد
&&
که در وی همچو باء بسمل آمد
\\
دوم نفس کل آمد آیت نور
&&
که چون مصباح شد از غایت نور
\\
سیم آیت در او شد عرش رحمان
&&
چهارم «آیت الکرسی» همی دان
\\
پس از وی جرمهای آسمانی است
&&
که در وی سورهٔ سبع المثانی است
\\
نظر کن باز در جرم عناصر
&&
که هر یک آیتی هستند باهر
\\
پس از عنصر بود جرم سه مولود
&&
که نتوان کرد این آیات محدود
\\
به آخر گشت نازل نفس انسان
&&
که بر ناس آمد آخر ختم قرآن
\\
\end{longtable}
\end{center}
