\begin{center}
\section*{بخش ۲۵ - جواب}
\label{sec:sh025}
\addcontentsline{toc}{section}{\nameref{sec:sh025}}
\begin{longtable}{l p{0.5cm} r}
کسی بر سر وحدت گشت واقف
&&
که او واقف نشد اندر مواقف
\\
دل عارف شناسای وجود است
&&
وجود مطلق او را در شهود است
\\
به جز هست حقیقی هست نشناخت
&&
از آن رو هستی خود پاک در باخت
\\
وجود تو همه خار است و خاشاک
&&
برون انداز از خود جمله را پاک
\\
برو تو خانهٔ دل را فرو روب
&&
مهیا کن مقام و جای محبوب
\\
چو تو بیرون شدی او اندر آید
&&
به تو بی تو جمال خود نماید
\\
کس کو از نوافل گشت محبوب
&&
به لای نفی کرد او خانه جاروب
\\
درون جان محبوب او مکان یافت
&&
ز «بی یسمع و بی یبصر» نشان یافت
\\
ز هستی تا بود باقی بر او شین
&&
نیابد علم عارف صورت عین
\\
موانع تا نگردانی ز خود دور
&&
درون خانهٔ دل نایدت نور
\\
موانع چون در این عالم چهار است
&&
طهارت کردن از وی هم چهار است
\\
نخستین پاکی از احداث و انجاس
&&
دوم از معصیت وز شر وسواس
\\
سوم پاکی ز اخلاق ذمیمه است
&&
که با وی آدمی همچون بهیمه است
\\
چهارم پاکی سر است از غیر
&&
که اینجا منتهی می‌گرددش سیر
\\
هر آن کو کرد حاصل این طهارات
&&
شود بی شک سزاوار مناجات
\\
تو تا خود را بکلی در نبازی
&&
نمازت کی شود هرگز نمازی
\\
چو ذاتت پاک گردد از همه شین
&&
نمازت گردد آنگه قرةالعین
\\
نماند در میانه هیچ تمییز
&&
شود معروف و عارف جمله یک چیز
\\
\end{longtable}
\end{center}
