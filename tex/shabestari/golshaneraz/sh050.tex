\begin{center}
\section*{بخش ۵۰ - جواب}
\label{sec:sh050}
\addcontentsline{toc}{section}{\nameref{sec:sh050}}
\begin{longtable}{l p{0.5cm} r}
هر آن چیزی که در عالم عیان است
&&
چو عکسی ز آفتاب آن جهان است
\\
جهان چون زلف و خط و خال و ابروست
&&
که هر چیزی به جای خویش نیکوست
\\
تجلی گه جمال و گه جلال است
&&
رخ و زلف آن معانی را مثال است
\\
صفات حق تعالی لطف و قهر است
&&
رخ و زلف بتان را زان دو بهر است
\\
چو محسوس آمد این الفاظ مسموع
&&
نخست از بهر محسوس است موضوع
\\
ندارد عالم معنی نهایت
&&
کجا بیند مر او را لفظ غایت
\\
هر آن معنی که شد از ذوق پیدا
&&
کجا تعبیر لفظی یابد او را
\\
چو اهل دل کند تفسیر معنی
&&
به مانندی کند تعبیر معنی
\\
که محسوسات از آن عالم چو سایه است
&&
که این چون طفل و آن مانند دایه است
\\
به نزد من خود الفاظ ماول
&&
بر آن معنی فتاد از وضع اول
\\
به محسوسات خاص از عرف عام است
&&
چه داند عام کان معنی کدام است
\\
نظر چون در جهان عقل کردند
&&
از آنجا لفظها را نقل کردند
\\
تناسب را رعایت کرد عاقل
&&
چو سوی لفظ معنی گشت نازل
\\
ولی تشبیه کلی نیست ممکن
&&
ز جست و جوی آن می‌باش ساکن
\\
بدین معنی کسی را بر تو دق نیست
&&
که صاحب مذهب اینجا غیر حق نیست
\\
ولی تا با خودی زنهار زنهار
&&
عبارات شریعت را نگه‌دار
\\
که رخصت اهل دل را در سه حال است
&&
فنا و سکر و آن دیگر دلال است
\\
هر آن کس کو شناسد این سه حالت
&&
بداند وضع الفاظ و دلالت
\\
تو را گر نیست احوال مواجید
&&
مشو کافر ز نادانی به تقلید
\\
مجازی نیست احوال حقیقت
&&
نه هر کس یابد اسرار طریقت
\\
گزاف ای دوست ناید ز اهل تحقیق
&&
مر این را کشف باید یا که تصدیق
\\
بگفتم وضع الفاظ و معانی
&&
تو را سربسته گر خواهی بدانی
\\
نظر کن در معانی سوی غایت
&&
لوازم را یکایک کن رعایت
\\
به وجه خاص از آن تشبیه می‌کن
&&
ز دیگر وجه‌ها تنزیه می‌کن
\\
چو شد این قاعده یک سر مقرر
&&
نمایم زان مثالی چند دیگر
\\
\end{longtable}
\end{center}
