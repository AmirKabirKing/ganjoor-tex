\begin{center}
\section*{بخش ۱۳ - قاعده در تفکر در انفس}
\label{sec:sh013}
\addcontentsline{toc}{section}{\nameref{sec:sh013}}
\begin{longtable}{l p{0.5cm} r}
به اصل خویش یک ره نیک بنگر
&&
که مادر را پدر شد باز و مادر
\\
جهان را سر به سر در خویش می‌بین
&&
هر آنچ آمد به آخر پیش می‌بین
\\
در آخر گشت پیدا نفس آدم
&&
طفیل ذات او شد هر دو عالم
\\
نه آخر علت غایی در آخر
&&
همی گردد به ذات خویش ظاهر
\\
ظلومی و جهولی ضد نورند
&&
ولیکن مظهر عین ظهورند
\\
چو پشت آینه باشد مکدر
&&
نماید روی شخص از روی دیگر
\\
شعاع آفتاب از چارم افلاک
&&
نگردد منعکس جز بر سر خاک
\\
تو بودی عکس معبود ملایک
&&
از آن گشتی تو مسجود ملایک
\\
بود از هر تنی پیش تو جانی
&&
وز او در بسته با تو ریسمانی
\\
از آن گشتند امرت را مسخر
&&
که جان هر یکی در توست مضمر
\\
تو مغز عالمی زان در میانی
&&
بدان خود را که تو جان جهانی
\\
تو را ربع شمالی گشت مسکن
&&
که دل در جانب چپ باشد از تن
\\
جهان عقل و جان سرمایهٔ توست
&&
زمین و آسمان پیرایهٔ توست
\\
ببین آن نیستی کو عین هستی است
&&
بلندی را نگر کو ذات پستی است
\\
طبیعی قوت تو ده هزار است
&&
ارادی برتر از حصر و شمار است
\\
وز آن هر یک شده موقوف آلات
&&
ز اعضا و جوارح وز رباطات
\\
پزشکان اندر آن گشتند حیران
&&
فرو ماندند در تشریح انسان
\\
نبرده هیچکس ره سوی این کار
&&
به عجز خویش هر یک کرده اقرار
\\
ز حق با هر یکی حظی و قسمی است
&&
معاد و مبدا هر یک به اسمی است
\\
از آن اسمند موجودات قائم
&&
بدان اسمند در تسبیح دائم
\\
به مبدا هر یکی زان مصدری شد
&&
به وقت بازگشتن چون دری شد
\\
از آن در کامد اول هم بدر شد
&&
اگرچه در معاش از در به در شد
\\
از آن دانسته‌ای تو جمله اسما
&&
که هستی صورت عکس مسما
\\
ظهور قدرت و علم و ارادت
&&
به توست ای بندهٔ صاحب سعادت
\\
سمیعی و بصیری، حی و گویا
&&
بقا داری نه از خود لیک از آنجا
\\
زهی اول که عین آخر آمد
&&
زهی باطن که عین ظاهر آمد
\\
تو از خود روز و شب اندر گمانی
&&
همان بهتر که خود را می‌ندانی
\\
چو انجام تفکر شد تحیر
&&
در اینجا ختم شد بحث تفکر
\\
\end{longtable}
\end{center}
