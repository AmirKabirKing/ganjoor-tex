\begin{center}
\section*{بخش ۴۲ - تمثیل در بیان نکاح معنوی جسم با جان یا صورت با معنی}
\label{sec:sh042}
\addcontentsline{toc}{section}{\nameref{sec:sh042}}
\begin{longtable}{l p{0.5cm} r}
اگرچه خور به چرخ چارمین است
&&
شعاعش نور و تدبیر زمین است
\\
طبیعت های عنصر نزد خور نیست
&&
کواکب گرم و سرد و خشک و تر نیست
\\
عناصر جمله از وی گرم و سرد است
&&
سپید و سرخ و سبز و آل و زرد است
\\
بود حکمش روان چون شاه عادل
&&
که نه خارج توان گفتن نه داخل
\\
چو از تعدیل شد ارکان موافق
&&
ز حسنش نفس گویا گشت عاشق
\\
نکاح معنوی افتاد در دین
&&
جهان را نفس کلی داد کابین
\\
از ایشان می پدید آمد فصاحت
&&
علوم و نطق و اخلاق و صباحت
\\
ملاحت از جهان بی‌مثالی
&&
درآمد همچو رند لاابالی
\\
به شهرستان نیکویی علم زد
&&
همه ترتیب عالم را به هم زد
\\
گهی بر رخش حسن او شهسوار است
&&
گهی با نطق تیغ آبدار است
\\
چو در شخص است خوانندش ملاحت
&&
چو در لفظ است گویندش بلاغت
\\
ولی و شاه و درویش و توانگر
&&
همه در تحت حکم او مسخر
\\
درون حسن روی نیکوان چیست
&&
نه آن حسن است تنها گویی آن چیست
\\
جز از حق می‌نیاید دلربایی
&&
که شرکت نیست کس را در خدایی
\\
کجا شهوت دل مردم رباید
&&
که حق گه گه ز باطل می‌نماید
\\
مؤثر حق شناس اندر همه جای
&&
ز حد خویشتن بیرون منه پای
\\
حق اندر کسوت حق بین و حق دان
&&
حق اندر باطل آمد کار شیطان
\\
\end{longtable}
\end{center}
