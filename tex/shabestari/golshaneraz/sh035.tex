\begin{center}
\section*{بخش ۳۵ - تمثیل در اطوار وجود}
\label{sec:sh035}
\addcontentsline{toc}{section}{\nameref{sec:sh035}}
\begin{longtable}{l p{0.5cm} r}
بخاری مرتفع گردد ز دریا
&&
به امر حق فرو بارد به صحرا
\\
شعاع آفتاب از چرخ چارم
&&
بر او افتد شود ترکیب با هم
\\
کند گرمی دگر ره عزم بالا
&&
در آویزد بدو آن آب دریا
\\
چو با ایشان شود خاک و هوا ضم
&&
برون آید نبات سبز و خرم
\\
غذای جانور گردد ز تبدیل
&&
خورد انسان و یابد باز تحلیل
\\
شود یک نطفه و گردد در اطوار
&&
وز او انسان شود پیدا دگر بار
\\
چو نور نفس گویا بر تن آید
&&
یکی جسم لطیف و روشن آید
\\
شود طفل و جوان و کهل و کمپیر
&&
بیابد علم و رای و فهم و تدبیر
\\
رسد آنگه اجل از حضرت پاک
&&
رود پاکی به پاکی خاک با خاک
\\
هم اجزای عالم چون نباتند
&&
که یک قطره ز دریای حیاتند
\\
زمان چو بگذرد بر وی شود باز
&&
همه انجام ایشان همچو آغاز
\\
رود هر یک از ایشان سوی مرکز
&&
که نگذارد طبیعت خوی مرکز
\\
چو دریایی است وحدت لیک پر خون
&&
کز او خیزد هزاران موج مجنون
\\
نگر تا قطرهٔ باران ز دریا
&&
چگونه یافت چندین شکل و اسما
\\
بخار و ابر و باران و نم و گل
&&
نبات و جانور انسان کامل
\\
همه یک قطره بود آخر در اول
&&
کز او شد این همه اشیا ممثل
\\
جهان از عقل و نفس و چرخ و اجرام
&&
چو آن یک قطره دان ز آغاز و انجام
\\
اجل چون در رسد در چرخ و انجم
&&
شود هستی همه در نیستی گم
\\
چو موجی بر زند گردد جهان طمس
&&
یقین گردد «کان لم تغن بالامس»
\\
خیال از پیش برخیزد به یک بار
&&
نماند غیر حق در دار دیار
\\
تو را قربی شود آن لحظه حاصل
&&
شوی تو بی تویی با دوست واصل
\\
وصال این جایگه رفع خیال است
&&
چو غیر از پیش برخیزد وصال است
\\
مگو ممکن ز حد خویش بگذشت
&&
نه او واجب شد و نه واجب او گشت
\\
هر آن کو در معانی گشت فایق
&&
نگوید کین بود قلب حقایق
\\
هزاران نشاه داری خواجه در پیش
&&
برو آمد شد خود را بیندیش
\\
ز بحث جزو و کل نشئات انسان
&&
بگویم یک به یک پیدا و پنهان
\\
\end{longtable}
\end{center}
