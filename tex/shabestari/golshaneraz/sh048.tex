\begin{center}
\section*{بخش ۴۸ - جواب}
\label{sec:sh048}
\addcontentsline{toc}{section}{\nameref{sec:sh048}}
\begin{longtable}{l p{0.5cm} r}
قدیم و محدث از هم خود جدا نیست
&&
که از هستی است باقی دائما نیست
\\
همه آن است و این مانند عنقاست
&&
جز ازحق جمله اسم بی‌مسماست
\\
عدم موجود گردد این محال است
&&
وجود از روی هستی لایزال است
\\
نه آن این گردد و نه این شود آن
&&
همه اشکال گردد بر تو آسان
\\
جهان خود جمله امر اعتباری است
&&
چو آن یک نقطه که اندر دور ساری است
\\
برو یک نقطهٔ آتش بگردان
&&
که بینی دایره از سرعت آن
\\
یکی گر در شمار آید به ناچار
&&
نگردد واحد از اعداد بسیار
\\
حدیث «ما سوی الله» را رها کن
&&
به عقل خویش این را زان جدا کن
\\
چه شک داری در آن کین چون خیال است
&&
که با وحدت دویی عین محال است
\\
عدم مانند هستی بود یکتا
&&
همه کثرت ز نسبت گشت پیدا
\\
ظهور اختلاف و کثرت شان
&&
شده پیدا ز بوقلمون امکان
\\
وجود هر یکی چون بود واحد
&&
به وحدانیت حق گشت شاهد
\\
\end{longtable}
\end{center}
