\begin{center}
\section*{بخش ۴۴ - جواب}
\label{sec:sh044}
\addcontentsline{toc}{section}{\nameref{sec:sh044}}
\begin{longtable}{l p{0.5cm} r}
وجود آن جزو دان کز کل فزون است
&&
که موجود است کل وین باژگون است
\\
بود موجود را کثرت برونی
&&
که از وحدت ندارد جز درونی
\\
وجود کل ز کثرت گشت ظاهر
&&
که او در وحدت جزو است سائر
\\
ندارد کل وجودی در حقیقت
&&
که او چون عارضی شد بر حقیقت
\\
چو کل از روی ظاهر هست بسیار
&&
بود از جزو خود کمتر به مقدار
\\
نه آخر واجب آمد جزو هستی
&&
که هستی کرد او را زیردستی
\\
وجود کل کثیر واحد آید
&&
کثیر از روی کثرت می‌نماید
\\
عرض شد هستیی کان اجتماعی است
&&
عرض سوی عدم بالذات ساعی است
\\
به هر جزوی ز کل کان نیست گردد
&&
کل اندر دم ز امکان نیست گردد
\\
جهان کل است و در هر طرفةالعین
&&
عدم گردد و لا یبقی زمانین
\\
دگر باره شود پیدا جهانی
&&
به هر لحظه زمین و آسمانی
\\
به هر لحظه جوان و کهنه پیر است
&&
به هر دم اندر او حشر و نشیر است
\\
در آن چیزی دو ساعت می‌نپاید
&&
در آن ساعت که می‌میرد بزاید
\\
ولیکن طامةالکبری نه این است
&&
که این یوم عمل وان یوم دین است
\\
از آن تا این بسی فرق است زنهار
&&
به نادانی مکن خود را گرفتار
\\
نظر بگشای در تفصیل و اجمال
&&
نگر در ساعت و روز و مه و سال
\\
\end{longtable}
\end{center}
