\begin{center}
\section*{بخش ۶۲ - تمثیل در اطوار سیر و سلوک}
\label{sec:sh062}
\addcontentsline{toc}{section}{\nameref{sec:sh062}}
\begin{longtable}{l p{0.5cm} r}
بود محبوس طفل شیرخواره
&&
به نزد مادر اندر گاهواره
\\
چو گشت او بالغ و مرد سفر شد
&&
اگر مرد است همراه پدر شد
\\
عناصر مر تو را چون ام سفلی است
&&
تو فرزند و پدر آبای علوی است
\\
از آن گفته است عیسی گاه اسرا
&&
که آهنگ پدر دارم به بالا
\\
تو هم جان پدر سوی پدر شو
&&
بدر رفتند همراهان بدر شو
\\
اگر خواهی که گردی مرغ پرواز
&&
جهان جیفه پیش کرکس انداز
\\
به دونان ده مر این دنیای غدار
&&
که جز سگ را نشاید داد مردار
\\
نسب چبود تناسب را طلب کن
&&
به حق رو آور و ترک نسب کن
\\
به بحر نیستی هر کو فرو شد
&&
«فلا انساب» نقد وقت او شد
\\
هر آن نسبت که پیدا شد ز شهوت
&&
ندارد حاصلی جز کبر و نخوت
\\
اگر شهوت نبودی در میانه
&&
نسب‌ها جمله می‌گشتی فسانه
\\
چو شهوت در میانه کارگر شد
&&
یکی مادر شد آن دیگر پدر شد
\\
نمی‌گویم که مادر یا پدر کیست
&&
که با ایشان به عزت بایدت زیست
\\
نهاده ناقصی را نام خواهر
&&
حسودی را لقب کرده برادر
\\
عدوی خویش را فرزند خوانی
&&
ز خود بیگانه خویشاوند خوانی
\\
مرا باری بگو تا خال و عم کیست
&&
وز ایشان حاصلی جز درد و غم چیست
\\
رفیقانی که با تو در طریق‌اند
&&
پی هزل ای برادر هم رفیق‌اند
\\
به کوی جد اگر یک دم نشینی
&&
از ایشان من چه گویم تا چه بینی
\\
همه افسانه و افسون و بند است
&&
به جان خواجه که این ها ریشخند است
\\
به مردی وارهان خود را چو مردان
&&
ولیکن حق کس ضایع مگردان
\\
ز شرع ار یک دقیقه ماند مهمل
&&
شوی در هر دو کون از دین معطل
\\
حقوق شرع را زنهار مگذار
&&
ولیکن خویشتن را هم نگهدار
\\
زر و زن نیست الا مایهٔ غم
&&
به جا بگذار چون عیسی مریم
\\
حنیفی شو ز هر قید و مذاهب
&&
درآ در دیر دین مانند راهب
\\
تو را تا در نظر اغیار و غیر است
&&
اگر در مسجدی آن عین دیر است
\\
چو برخیزد ز پیشت کسوت غیر
&&
شود بهر تو مسجد صورت دیر
\\
نمی‌دانم به هر حالی که هستی
&&
خلاف نفس کافر کن که رستی
\\
بت و زنار و ترسایی و ناقوس
&&
اشارت شد همه با ترک ناموس
\\
اگر خواهی که گردی بندهٔ خاص
&&
مهیا شو برای صدق و اخلاص
\\
برو خود را ز راه خویش برگیر
&&
به هر لحظه درآ ایمان ز سر گیر
\\
به باطن نفس ما چون هست کافر
&&
مشو راضی به دین اسلام ظاهر
\\
ز نو هر لحظه ایمان تازه گردان
&&
مسلمان شو مسلمان شو مسلمان
\\
بسا ایمان بود کز کفر زاید
&&
نه کفر است آن کز او ایمان فزاید
\\
ریا و سمعه و ناموس بگذار
&&
بیفکن خرقه و بربند زنار
\\
چو پیر ما شو اندر کفر فردی
&&
اگر مردی بده دل را به مردی
\\
به ترسازاده ده دل را به یک بار
&&
مجرد شود ز هر اقرار و انکار
\\
\end{longtable}
\end{center}
