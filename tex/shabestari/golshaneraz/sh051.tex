\begin{center}
\section*{بخش ۵۱ - اشارت به چشم و لب}
\label{sec:sh051}
\addcontentsline{toc}{section}{\nameref{sec:sh051}}
\begin{longtable}{l p{0.5cm} r}
نگر کز چشم شاهد چیست پیدا
&&
رعایت کن لوازم را بدینجا
\\
ز چشمش خاست بیماری و مستی
&&
ز لعلش گشت پیدا عین هستی
\\
ز چشم اوست دلها مست و مخمور
&&
ز لعل اوست جانها جمله مستور
\\
ز چشم او همه دلها جگرخوار
&&
لب لعلش شفای جان بیمار
\\
به چشمش گرچه عالم در نیاید
&&
لبش هر ساعتی لطفی نماید
\\
دمی از مردمی دلها نوازد
&&
دمی بیچارگان را چاره سازد
\\
به شوخی جان دمد در آب و در خاک
&&
به دم دادن زند آتش بر افلاک
\\
از او هر غمزه دام و دانه‌ای شد
&&
وز او هر گوشه‌ای میخانه‌ای شد
\\
ز غمزه می‌دهد هستی به غارت
&&
به بوسه می‌کند بازش عمارت
\\
ز چشمش خون ما در جوش دائم
&&
ز لعلش جان ما مدهوش دائم
\\
به غمزه چشم او دل می‌رباید
&&
به عشوه لعل او جان می‌فزاید
\\
چو از چشم و لبش جویی کناری
&&
مر این گوید که نه آن گوید آری
\\
ز غمزه عالمی را کار سازد
&&
به بوسه هر زمان جان می‌نوازد
\\
از او یک غمزه و جان دادن از ما
&&
وز او یک بوسه و استادن از ما
\\
ز «لمح بالبصر» شد حشر عالم
&&
ز نفخ روح پیدا گشت آدم
\\
چو از چشم و لبش اندیشه کردند
&&
جهانی می‌پرستی پیشه کردند
\\
نیاید در دو چشمش جمله هستی
&&
در او چون آید آخر خواب و مستی
\\
وجود ما همه مستی است یا خواب
&&
چه نسبت خاک را با رب ارباب
\\
خرد دارد از این صد گونه اشگفت
&&
که «ولتصنع علی عینی» چرا گفت
\\
\end{longtable}
\end{center}
