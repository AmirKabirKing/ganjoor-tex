\begin{center}
\section*{بخش ۴۱ - قاعده در بیان اقسام فضیلت}
\label{sec:sh041}
\addcontentsline{toc}{section}{\nameref{sec:sh041}}
\begin{longtable}{l p{0.5cm} r}
اصول خلق نیک آمد عدالت
&&
پس از وی حکمت وعفت شجاعت
\\
حکیمی راست گفتار است و کردار
&&
کسی کو متصف گردد بدین چار
\\
به حکمت باشدش جان و دل آگه
&&
نه گربز باشد و نه نیز ابله
\\
به عفت شهوت خود کرده مستور
&&
شره همچون خمود از وی شده دور
\\
شجاع و صافی از ذل و تکبر
&&
مبرا ذاتش از جبن و تهور
\\
عدالت چون شعار ذات او شد
&&
ندارد ظلم از آن خلقش نکو شد
\\
همه اخلاق نیکو در میانه است
&&
که از افراط و تفریطش کرانه است
\\
میانه چون صراط مستقیم است
&&
ز هر دو جانبش قعر جحیم است
\\
به باریکی و تیزی موی و شمشیر
&&
نه روی گشتن و بودن بر او دیر
\\
عدالت چون یکی دارد ز اضداد
&&
همی هفت آمد این اضداد ز اعداد
\\
به زیر هر عدد سری نهفت است
&&
از آن درهای دوزخ نیز هفت است
\\
چنان کز ظلم شد دوزخ مهیا
&&
بهشت آمد همیشه عدل را جا
\\
جزای عدل، نور و رحمت آمد
&&
سزای ظلم، لعن و ظلمت آمد
\\
ظهور نیکویی در اعتدال است
&&
عدالت جسم را اقصی کمال است
\\
مرکب چون شود مانند یک چیز
&&
ز اجزا دور گردد فعل و تمییز
\\
بسیط الذات را مانند گردد
&&
میان این و آن پیوند گردد
\\
نه پیوندی که از ترکیب اجزاست
&&
که روح از وصف جسمیت مبراست
\\
چو آب و گل شود یکباره صافی
&&
رسد از حق بدو روح اضافی
\\
چو یابد تسویت اجزای ارکان
&&
در او گیرد فروغ عالم جان
\\
شعاع جان سوی تن وقت تعدیل
&&
چو خورشید و زمین آمد به تمثیل
\\
\end{longtable}
\end{center}
