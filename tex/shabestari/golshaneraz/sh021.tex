\begin{center}
\section*{بخش ۲۱ - تمثیل در بیان رابطهٔ شریعت و طریقت و حقیقت}
\label{sec:sh021}
\addcontentsline{toc}{section}{\nameref{sec:sh021}}
\begin{longtable}{l p{0.5cm} r}
تبه گردد سراسر مغز بادام
&&
گرش از پوست بیرون آوری خام
\\
ولی چون پخته شد بی پوست نیکوست
&&
اگر مغزش بر آری بر کنی پوست
\\
شریعت پوست، مغز آمد حقیقت
&&
میان این و آن باشد طریقت
\\
خلل در راه سالک نقص مغز است
&&
چو مغزش پخته شد بی‌پوست نغز است
\\
چو عارف با یقین خویش پیوست
&&
رسیده گشت مغز و پوست بشکست
\\
وجودش اندر این عالم نپاید
&&
برون رفت و دگر هرگز نیاید
\\
وگر با پوست تابد تابش خور
&&
در این نشات کند یک دور دیگر
\\
درختی گردد او از آب و از خاک
&&
که شاخش بگذرد از جمله افلاک
\\
همان دانه برون آید دگر بار
&&
یکی صد گشته از تقدیر جبار
\\
چو سیر حبه بر خط شجر شد
&&
ز نقطه خط ز خط دوری دگر شد
\\
چو شد در دایره سالک مکمل
&&
رسد هم نقطهٔ آخر به اول
\\
دگر باره شود مانند پرگار
&&
بر آن کاری که اول بود بر کار
\\
تناسخ نبود این کز روی معنی
&&
ظهورات است در عین تجلی
\\
و قد سلوا و قالوا ما النهایة
&&
فقیل هی الرجوع الی البدایة
\\
\end{longtable}
\end{center}
