\begin{center}
\section*{بخش ۲ - سبب نظم کتاب}
\label{sec:sh002}
\addcontentsline{toc}{section}{\nameref{sec:sh002}}
\begin{longtable}{l p{0.5cm} r}
گذشته هفت و ده از هفتصد سال
&&
ز هجرت ناگهان در ماه شوال
\\
رسولی با هزاران لطف و احسان
&&
رسید از خدمت اهل خراسان
\\
بزرگی کاندر آنجا هست مشهور
&&
به انواع هنر چون چشمهٔ هور
\\
جهان را سور و جان را نور اعنی
&&
امام سالکان سید حسینی
\\
همه اهل خراسان از که و مه
&&
در این عصر از همه گفتند او به
\\
نبشته نامه‌ای در باب معنی
&&
فرستاده بر ارباب معنی
\\
در آنجا مشکلی چند از عبارت
&&
ز مشکلهای اصحاب اشارت
\\
به نظم آورده و پرسیده یک یک
&&
جهانی معنی اندر لفظ اندک
\\
ز اهل دانش و ارباب معنی
&&
سؤالی دارم اندر باب معنی
\\
ز اسرار حقیقت مشکلی چند
&&
بگویم در حضور هر خردمند
\\
نخست از فکر خویشم در تحیر
&&
چه چیز است آنکه گویندش تفکر
\\
چه بود آغاز فکرت را نشانی
&&
سرانجام تفکر را چه خوانی
\\
کدامین فکر ما را شرط راه است
&&
چرا گه طاعت و گاهی گناه است
\\
که باشم من مرا از من خبر کن
&&
چه معنی دارد اندر خود سفر کن
\\
مسافر چون بود رهرو کدام است
&&
که را گویم که او مرد تمام است
\\
که شد بر سر وحدت واقف آخر
&&
شناسای چه آمد عارف آخر
\\
اگر معروف و عارف ذات پاک است
&&
چه سودا بر سر این مشت خاک است
\\
کدامین نقطه را جوش است انا الحق
&&
چه گویی، هرزه بود آن یا محقق
\\
چرا مخلوق را گویند واصل
&&
سلوک و سیر او چون گشت حاصل
\\
وصال ممکن و واجب به هم چیست
&&
حدیث قرب و بعد و بیش و کم چیست
\\
چه بحر است آنکه علمش ساحل آمد
&&
ز قعر او چه گوهر حاصل آمد
\\
صدف چون دارد آن معنی بیان کن
&&
کجا زو موج آن دریا نشان کن
\\
چه جزو است آن که او از کل فزون است
&&
طریق جستن آن جزو چون است
\\
قدیم و محدث از هم چون جدا شد
&&
که این عالم شد آن دیگر خدا شد
\\
دو عالم ما سوی الله است بی‌شک
&&
معین شد حقیقت بهر هر یک
\\
دویی ثابت شد آنگه این محال است
&&
چه جای اتصال و انفصال است
\\
اگر عالم ندارد خود وجودی
&&
خیالی گشت هر گفت و شنودی
\\
تو ثابت کن که این و آن چگونه است
&&
وگرنه کار عالم باژگونه است
\\
چه خواهد مرد معنی زان عبارت
&&
که دارد سوی چشم و لب اشارت
\\
چه جوید از سر زلف و خط و خال
&&
کسی کاندر مقامات است و احوال
\\
شراب و شمع و شاهد را چه معنی است
&&
خراباتی شدن آخر چه دعوی است
\\
بت و زنار و ترسایی در این کوی
&&
همه کفر است ورنه چیست بر گوی
\\
چه می‌گویی گزاف این جمله گفتند
&&
که در وی بیخ تحقیقی نهفتند
\\
محقق را مجازی کی بود کار
&&
مدان گفتارشان جز مغز اسرار
\\
کسی کو حل کند این مشکلم را
&&
نثار او کنم جان و دلم را
\\
رسول آن نامه را برخواند ناگاه
&&
فتاد احوال او حالی در افواه
\\
در آن مجلس عزیزان جمله حاضر
&&
بدین درویش هر یک گشته ناظر
\\
یکی کو بود مرد کاردیده
&&
ز ما صد بار این معنی شنیده
\\
مرا گفتا جوابی گوی در دم
&&
کز آنجا نفع گیرند اهل عالم
\\
بدو گفتم چه حاجت کین مسائل
&&
نبشتم بارها اندر رسائل
\\
بلی گفتا ولی بر وفق مسؤول
&&
ز تو منظوم می‌داریم مامول
\\
پس از الحاح ایشان کردم آغاز
&&
جواب نامه در الفاظ ایجاز
\\
به یک لحظه میان جمع بسیار
&&
بگفتم جمله را بی‌فکر و تکرار
\\
کنون از لطف و احسانی که دارند
&&
ز من این خردگیها در گذارند
\\
همه دانند کین کس در همه عمر
&&
نکرده هیچ قصد گفتن شعر
\\
بر آن طبعم اگر چه بود قادر
&&
ولی گفتن نبود الا به نادر
\\
به نثر ارچه کتب بسیار می‌ساخت
&&
به نظم مثنوی هرگز نپرداخت
\\
عروض و قافیه معنی نسنجد
&&
به هر ظرفی درون معنی نگنجد
\\
معانی هرگز اندر حرف ناید
&&
که بحر قلزم اندر ظرف ناید
\\
چو ما از حرف خود در تنگناییم
&&
چرا چیزی دگر بر وی فزاییم
\\
نه فخر است این سخن کز باب شکر است
&&
به نزد اهل دل تمهید عذر است
\\
مرا از شاعری خود عار ناید
&&
که در صد قرن چون عطار ناید
\\
اگرچه زین نمط صد عالم اسرار
&&
بود یک شمه از دکان عطار
\\
ولی این بر سبیل اتفاق است
&&
نه چون دیو از فرشته استراق است
\\
علی الجمله جواب نامه در دم
&&
نبشتم یک به یک نه بیش نه کم
\\
رسول آن نامه را بستد به اعزاز
&&
وز آن راهی که آمد باز شد باز
\\
دگرباره عزیزی کار فرمای
&&
مرا گفتا بر آن چیزی بیفزای
\\
همان معنی که گفتی در بیان آر
&&
ز عین علم با عین عیان آر
\\
نمی‌دیدم در اوقات آن مجالی
&&
که پردازم بدو از ذوق حالی
\\
که وصف آن به گفت و گو محال است
&&
که صاحب حال داند کان چه حال است
\\
ولی بر وفق قول قائل دین
&&
نکردم رد سؤال سائل دین
\\
پی آن تا شود روشن‌تر اسرار
&&
درآمد طوطی طبعم به گفتار
\\
به عون و فضل و توفیق خداوند
&&
بگفتم جمله را در ساعتی چند
\\
دل از حضرت چو نام نامه درخواست
&&
جواب آمد به دل کین گلشن ماست
\\
چو حضرت کرد نام نامه گلشن
&&
شود زان چشم دلها جمله روشن
\\
\end{longtable}
\end{center}
