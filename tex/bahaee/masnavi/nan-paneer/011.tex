\begin{center}
\section*{بخش ۱۱ - باقی سخن در توحید}
\label{sec:011}
\addcontentsline{toc}{section}{\nameref{sec:011}}
\begin{longtable}{l p{0.5cm} r}
می‌برد تا به خدمت ذوالمن
&&
کش کشانش، دوشاخه در گردن
\\
دو نهال است رسته از یک بیخ
&&
میوه‌شان نفس و طبع را توبیخ
\\
کرسی «لا» مثلثی است صغیر
&&
اندر او مضمحل، جهان کبیر
\\
هرکه رو از وجود محدث تافت
&&
ره به کنجی از آن مثلث یافت
\\
عقل داند، ز تنگی هر کنج
&&
که در او نیست ما و من را گنج
\\
«بوحنیفه» چه در معنی سفت
&&
نوعی از باده را مثلث گفت
\\
هست بر رای او به شرح هدی
&&
آن مثلث، مباح و پاک ولی
\\
این مثلث، به کیش اهل فلاح
&&
واجب و مفترض بود نه مباح
\\
زان مثلث، هر آنکه زد جامی
&&
شد ز مستی، زبون هر خامی
\\
زین مثلث، هرآنکه یک جرعه
&&
خورد، بختش به نام زد قرعه
\\
جرعهٔ راحتش، به جام افتاد
&&
قرعهٔ دولتش، به نام افتاد
\\
\end{longtable}
\end{center}
