\begin{center}
\section*{بخش ۹ - فی‌الفطره}
\label{sec:009}
\addcontentsline{toc}{section}{\nameref{sec:009}}
\begin{longtable}{l p{0.5cm} r}
ای لوای اجتهاد افراشته
&&
روزهٔ هر روز، عادت ساخته
\\
اهل وحدت را به شقوت کرده حکم
&&
بسته‌شان در ربقهٔ صم و بکم
\\
هان، مشو مغرور بر افعال خود
&&
هان مشو مسرور بر احوال خود
\\
این عبادتهای تو مقبول نیست
&&
تا ندانی عاقبت، کار تو چیست
\\
ای بسا نعلی که وارون بسته شد
&&
شیشهٔ امن نفوس اشکسته شد
\\
گبر چندین ساله‌ای در حین نزع
&&
کرد بر حقیقت اسلام، قطع
\\
عابدی با شد و مد و کش و فش
&&
بهر ترسا بچه‌ای شد، باده‌کش
\\
کار با انجام کار است و سرشت
&&
ختم کاشف، از سرشت خوب و زشت
\\
ای بسا بدطینت و نیکوخصال
&&
ای بسا خوش طینت و ناخوش فعال
\\
طینت بد، آنکه در علم ازل
&&
رفته از وی ختم بر کفر و دغل
\\
\end{longtable}
\end{center}
