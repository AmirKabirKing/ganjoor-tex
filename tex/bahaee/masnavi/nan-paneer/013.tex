\begin{center}
\section*{بخش ۱۳ - فی ماهیة الذوات}
\label{sec:013}
\addcontentsline{toc}{section}{\nameref{sec:013}}
\begin{longtable}{l p{0.5cm} r}
هر یک از موجود، با طوری وجود
&&
بهر او موجود شد، انسان نمود
\\
بود امر ممکنی از ممکنات
&&
در ازل ممتاز از غیرش به ذات
\\
بود اما بودنی علمی و بس
&&
حد علم ارچه نشد مفهوم کس
\\
مأخذ کل، قدرت بی‌منتهی است
&&
بی‌کم و بی‌کیف و أین و متی است
\\
داشت از حق، بهر حق را هم ظهور
&&
خواهی ار تمثیل وی، چون ظل و نور
\\
ظل، قدرت بود، کل، قبل الوجود
&&
هم ز حق، از بهر حق معلوم بود
\\
چون معانیشان ز یکدیگر جداست
&&
گر تو ماهیاتشان خوانی، رواست
\\
زانکه ماهیت ز ماهو مشتق است
&&
زان به هر یک صدق، تشبیه حق است
\\
آنچه می‌گویم، همه تقریب دان
&&
نیست جز تقریب در وسع بیان
\\
این بیانات و شروح، ای حق شناس
&&
جمله تمثیل و مجاز است و قیاس
\\
وه! چه نیکو گفت دانای حکیم
&&
از پی تمثیل قدوس و قدیم:
\\
ای برون از فکر و قال و قیل من
&&
خاک بر فرق من و تمثیل من
\\
\end{longtable}
\end{center}
