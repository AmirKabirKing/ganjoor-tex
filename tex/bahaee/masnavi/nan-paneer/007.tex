\begin{center}
\section*{بخش ۷ - تمثیل}
\label{sec:007}
\addcontentsline{toc}{section}{\nameref{sec:007}}
\begin{longtable}{l p{0.5cm} r}
بی‌نمازی با یکی از اهل راز
&&
خواست گوید علت ترک نماز
\\
گفت : هر وقتی که کردم قصد آن
&&
آفتی آمد به مالم، ناگهان
\\
و آن دگر گفتش که من کردم نماز
&&
مدتی بسیار و شبهای دراز
\\
تا برون آیم ز فقر و احتیاج
&&
گیرد آن دکان و بازارم رواج
\\
حاصلی از وی توقع داشتم
&&
چون نشد، یکبارگی بگذاشتم
\\
این بود احوال جهال، ای عزیز!
&&
این بودشان پایهٔ قدر و تمیز
\\
واجبی را در خیال، این گمرهان
&&
کرده‌اند از جهل خود، ممکن گمان
\\
داده نسبت بخل یا غفلت به وی
&&
در مقایل، خویش را دانسته شیء
\\
غیر ممکن، کی ز ممکن کرد فرق
&&
آنکه در دریای تشبیه است غرق
\\
تا نشد اوصاف امکانیش فهم
&&
کی تواند دید کوته، دست وهم
\\
ساحت عزت، چه سان داند بری
&&
از خلاء و سطح و بعد جوهری
\\
تا ندانسته است اعراض عدد
&&
بر چه معنی خواهدش گفتی احد
\\
هرچه گوید، در رضا و در غضب
&&
زان منزه‌دان، جناب قدس رب
\\
گرچه تقدیس خداوند صمد
&&
از ره تقلید هم ممکن بود
\\
زان جهت گوییم: جمعی از عوام
&&
یافته در سلک اسلام، انتظام
\\
لیک، این اسلام، حکم ظاهر است
&&
تا برون آید ز گبر و بت‌پرست
\\
گرنه فضل از حق خود دارد قبول
&&
کی شود مقبول تقلید اصول
\\
بلکه آن تقلید هم از مشکلات
&&
اصل مطلب چون بود از غامضات
\\
ز آن، نبی مجمل رساند اول پیام
&&
که در آن منظور بودش خاص و عام
\\
رفته رفته، عقلها چون شد قوی
&&
یافت بسطی مجملات معنوی
\\
آنکه از علم سیر دارد خبر
&&
کرده در اقوال معصومین نظر
\\
دیده اجمالات و تفصیلاتشان
&&
در تکلم، مختلف حالاتشان
\\
سائلی پرسید از تفویض و جبر
&&
تا شناسد، کیست در امت چو گبر
\\
گفت: تفویض، آنکه اعمال تمام
&&
حق مفوض کرده باشد بر آنام
\\
راست گفت؛ این نیز تفویضی بدست
&&
لیک، آن نه کز پیمبر واردست
\\
چون نبودش تاب استعداد و درک
&&
کرد زان تفسیر، این تفیض، درک
\\
\end{longtable}
\end{center}
