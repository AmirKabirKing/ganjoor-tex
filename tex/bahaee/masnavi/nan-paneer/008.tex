\begin{center}
\section*{بخش ۸ - فی‌التحقیق}
\label{sec:008}
\addcontentsline{toc}{section}{\nameref{sec:008}}
\begin{longtable}{l p{0.5cm} r}
ای خوشا نفسی که شد در جستجو
&&
بس تفحص کرد حق را کو به کو
\\
در همه حالات، حق منظور داشت
&&
حق ورا دانست، ناحق را گذاشت
\\
گر چنینی، هر کتابی را بخوان
&&
عاقبت، مأجوری خود را بدان
\\
ورنه حق مقصود داری ای خبیث
&&
بر تو حجت باشد این علم حدیث
\\
رو تتبع کن وجود رأیها
&&
تا شوی واقف مکانهای خطا
\\
این چنین فرموده، شاه علم و دین
&&
هادی عرفان، امیرالمؤمنین
\\
هان، نگویی فلسفه، کل حق بود
&&
آنکه گوید، کافر مطلق بود
\\
آری! از وی می‌کند در دل خطور
&&
بس معانی کز دهانت بوده دور
\\
چون تصور کردش آنکو المعی است
&&
دید دانست آنچه خود را واقعی است
\\
چون تواند کرد عقل اثبات شیء
&&
تا نمی‌فهمند شرح رسم وی
\\
هم برین منوال دان ابطال آن
&&
این بود قانون عقل جاودان
\\
\end{longtable}
\end{center}
