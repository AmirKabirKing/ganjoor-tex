\begin{center}
\section*{بخش ۶ - فی العلم وحده}
\label{sec:006}
\addcontentsline{toc}{section}{\nameref{sec:006}}
\begin{longtable}{l p{0.5cm} r}
ای که هستی، روز و شب، جویای علم
&&
تشنه و غواص، در دریای علم
\\
رفته در حیرت که حد علم چیست؟
&&
از کتب، آیا کدامین خواندنی است؟
\\
هر کسی، نوعی از آن را رو کند
&&
علم بر وفق طبیعت، خو کند
\\
آن یکی گوید: حساب و هندسه
&&
جمله وهم است و خیال و وسوسه
\\
و آن دگر گوید که: هان، علم اصول
&&
فدیه باشد بر خدا و بر رسول
\\
کاش، حد علم را دانستمی
&&
تا از این تشویش و حیرت رستمی
\\
گر تو را مقصود، علم مطلق است
&&
حد آن، نزد قدیم بر حق است
\\
علم مطلق، بی‌حد و بی‌منتهاست
&&
حد بی‌حد باز بی‌حد را سزاست
\\
ور بود مقصود تو ای حق پرست
&&
حد علمی کان کمال انفس است
\\
علم، آن باشد که بنماید رهت
&&
علم، آن باشد که سازد آگهت
\\
علم، آن باشد که بشناسی به وی
&&
لطف و فیض قادر و قیوم و حی
\\
پس بدانی، قدرت بی‌حد او
&&
فیض و جود و نعمت بی‌عد او
\\
آن به تعظیم آردت، بی‌اختیار
&&
وین کند در جمله حال امیدوار
\\
بی‌تصنع، حب خود در دل کند
&&
بی‌تکلف، بر عمل مایل کند
\\
چون ز روی شوق، کردی بندگی
&&
آن زمان، داری نشان زندگی
\\
آنکه در طاعت، دلش افسرده است
&&
گر به ظاهر زنده، باطن مرده است
\\
قوم جهال ار عبادت می‌کنند
&&
بیشتر، از روی عادت می‌کنند
\\
یا عوامی را، به خود داعی بود
&&
یا برای دنیوی، ساعی بود
\\
\end{longtable}
\end{center}
