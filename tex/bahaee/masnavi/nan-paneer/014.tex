\begin{center}
\section*{بخش ۱۴ - فی مجانسة الذوات بالصفات}
\label{sec:014}
\addcontentsline{toc}{section}{\nameref{sec:014}}
\begin{longtable}{l p{0.5cm} r}
داشت هر ذاتی، چو در علم ازل
&&
خواهش خود را به نوعی از عمل
\\
بالسان حال کرد از حق سؤال
&&
تا میسر سازدش در لایزال
\\
گر میسر خیر شد، توفیق دان
&&
گر میسر شر بشد، خذلانش خوان
\\
نی میسر این جز الحان سؤال
&&
گرچه بی‌مسول فعل آمد محال
\\
لوم، پس عائد به اهل شر بود
&&
ذیل عدل حق، از آن اطهر بود
\\
لم این مرموز اسرار خداست
&&
خوض دادن عقل را، در وی خطاست
\\
گر به علم و حکمت حق قائلی
&&
بر تو منحل می‌شود، بی‌مشکلی
\\
ورنه اول رو تتبع کن علوم
&&
خاصه، تشریح و ریاضی و نجوم
\\
بین چه حکمتهاست در دور سپهر
&&
بین چه حکمتهاست در تنویر مهر
\\
بین چه حکمتهاست در خلق جهان
&&
بین چه حکمتهاست در تعلیم جان
\\
بین چه حکمتهاست در خلق نبات
&&
بین چه حکمتهاست در این میوه‌جات
\\
صافی این علمها خواهی اگر
&&
رو به «توحید مفضل» کن نظر
\\
کاندر آن از خان علم اله
&&
بشنوی با حق، بیان ای مرد راه
\\
علم و دانش، جمله ارث انبیاست
&&
انبیا را علم، از نزد خداست
\\
خواندن صوری نشد صورت پذیر
&&
از معانی تنیست دانا را گزیر
\\
نفس چون گردد مهیای قبول
&&
علم از ایشان می‌کند در پی نزول
\\
غایتش، گاهی میانجی حاصل است
&&
مثل عقلی کاو به ایشان واصل است
\\
عقل از بند هوی چون وارهد
&&
روی وجهت سوی علیین کند
\\
انبیا را چیست تعلیم عقول؟
&&
گوش کن گر نیستی ز اهل فضول
\\
کشف سر است آنچه بتوانند دید
&&
نقل ذکر است آنچه باشدشان شنید
\\
\end{longtable}
\end{center}
