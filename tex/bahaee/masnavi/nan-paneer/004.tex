\begin{center}
\section*{بخش ۴ - قال المولوی المعنوی}
\label{sec:004}
\addcontentsline{toc}{section}{\nameref{sec:004}}
\begin{longtable}{l p{0.5cm} r}
« مشورت می‌کرد، شخصی با یکی
&&
تا یقینش رو نماید، بی‌شکی
\\
گفت: ای خوشنام! غیر من بجو
&&
ماجرای مشورت، با من بگو
\\
من عدوم مر تو را، با من مپیچ
&&
نبود از رأی عدو، پیروز هیچ
\\
رو کسی جو که تو را او هست دوست
&&
دوست بهر دوست، لاشک خیر جوست
\\
من عدوم، چاره نبود کز منی
&&
کژ روم، با تو نمایم دشمنی
\\
حارسی از گرگ جستن، شرط نیست
&&
جستن از غیر محل، ناجستنی است
\\
من تو را، بی‌هیچ شکی، دشمنم
&&
من تو را کی ره نمایم؟ ره زنم
\\
هر که باشد همنشین دوستان
&&
هست در گلخن، میان بوستان
\\
هر که با دشمن نشیند، در ز من
&&
هست اندر بوستان، در گولخن
\\
دوست را مازار، از ما و منت
&&
تا نگردد دوست، خصم و دشمنت
\\
خیر کن با خلق، از بهر خدا
&&
یا برای جان خود، ای کدخدا
\\
تا هماره دوست بینی در نظر
&&
در دلت ناید ز کین، ناخوش صور
\\
چون که کردی دشمنی، پرهیز کن
&&
مشورت با یار مهرانگیز کن
\\
گفت: می‌دانم تو را ای بوالحسن
&&
که تویی دیرینه دشمن دار من
\\
لیک مرد عاقلی و معنوی
&&
عقل تو نگذاردت که کج روی
\\
طبع خواهد تا کشد از خصم کین
&&
عقل بر نفس است بند آهنین
\\
آید و منعش کند، واداردش
&&
عقل، چون شحنه است، در نیک و بدش
\\
عقل ایمانی، چو شحنهٔ عادل است
&&
پاسبان و حاکم شهر دل است
\\
همچو گربه باشد او بیدار هوش
&&
دزد در سوراخ ماند، همچو موش
\\
در هر آنجا که برآرد موش دست
&&
نیست گربه، ور بود، آن مرده است
\\
گربهٔ چون شیر، شیرافکن بود
&&
عقل ایمانی که اندر تن بود
\\
غرهٔ او حاکم درندگان
&&
نعرهٔ او، مانع چرندگان
\\
شهر پر دزد است و پر جامه کنی
&&
خواه شحنه باش گو و خواه نی
\\
عقل در تن، حاکم ایمان بود
&&
که ز بیمش، نفس در زندان بود
\\
عقل دو عقل است اول مکسبی
&&
که در آموزی، چو در مکتب صبی
\\
از کتاب و اوستاد و فکر و ذکر
&&
وز معانی و علوم خوب و بکر
\\
عقل تو افزون شود بر دیگران
&&
لیک، تو باشی ز حفظ آن گران
\\
لوح حافظ، تو شوی در دور و گشت
&&
لوح محفوظ است، کاو زین در گذشت
\\
عقل دیگر، بخشش یزدان بود
&&
چشمهٔ آن، در میان جان بود
\\
چون ز سینه، آب دانش، جوش کرد
&&
نی شود گنده، نه دیرینه، نه زرد
\\
ور ره نقبش بود بسته، چه غم؟
&&
کو همی جوشد ز خانه، دم به دم
\\
عقل تحصیلی، مثال جویها
&&
کان رود در خانه‌ای، از کویها
\\
چون که راهش، بسته شد، شد بینوا
&&
تشنه ماند و زار، با صد ابتلا
\\
از درون خویشتن جو چشمه را
&&
تا رهی از منت هر ناسزا
\\
جهد کن تا پیر عقل و دین شوی
&&
تا چو عقل کل، تو باطن بین شوی
\\
از عدم، چون عقل زیبا رو نمود
&&
خلقتش داد و هزاران عز فزود
\\
عقل، چون از عالم غیبی گشاد
&&
رفت افزود و هزاران نام داد
\\
کمترین زان نامهای خوش نفس
&&
این که نبود هیچ او محتاج کس
\\
گر به صورت، وا نماید عقل رو
&&
تیره باشد روز، پیش نور او
\\
ور مثال احمقی، پیدا شود
&&
ظلمت شب، پیش او روشن بود
\\
کاو ز شب مظلم‌تر و تاری‌تر است
&&
لیک، خفاش شقی، ظلمت خر است
\\
اندک اندک، خوی کن با نور روز
&&
ورنه چون خفاش، مانی بی‌فروز
\\
عاشقی هر جا، شکال و مشکلی است
&&
دشمنی هرجا چراغ مقبلی است
\\
ظلمت اشکال، زان جوید دلش
&&
تا که افزونتر نماید حاصلش
\\
تا تو را مشغول آن مشکل کند
&&
وز نهاد زشت خود غافل کند
\\
عقل ضد شهوت است، ای پهلوان
&&
آنکه شهوت می‌تند، عقلش مخوان
\\
وهم خوانش آنکه شهوت را گداست
&&
وهم قلب و نقد، زر عقلهاست
\\
بی‌محک، پیدا نگردد وهم و عقل
&&
هر دو را سوی محک کن زود نقل
\\
این محک، قرآن و حال انبیا
&&
چون محک، هر قلب را گوید: بیا
\\
تا ببینی خویش را ز آسیب من
&&
که نه‌ای اهل فراز و شیب من
\\
عقل را، گر اره‌ای سازد دو نیم
&&
همچو زر باشد در آتش او به سیم»
\\
\end{longtable}
\end{center}
