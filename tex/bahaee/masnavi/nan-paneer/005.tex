\begin{center}
\section*{بخش ۵ - فی اختلاف العقول}
\label{sec:005}
\addcontentsline{toc}{section}{\nameref{sec:005}}
\begin{longtable}{l p{0.5cm} r}
عقلها را داده ایزد اعتداد
&&
مختلف اقدار بر حسب مواد
\\
شعله‌ها هریک به حدی منتهی است
&&
مشعلی از شمع جستن، ابلهی است
\\
پس ز هر نفسی، فروغی ممکن است
&&
چون به فعل آید، توانی گفت هست
\\
سعی می‌کن تا به فعل آید تمام
&&
ورنه خواهی بود ناقص، والسلام
\\
سعی و تحصیل است و فکر اعتبار
&&
ترک شغلی کان تو را نبود به کار
\\
برحذر بودن ز طغیان هوا
&&
زانکه افتد عقل از آن در صعبها
\\
عبرتی گیر از چراغی، ای غنی
&&
در غبار ابر، در کم روغنی
\\
هان، تو بگشا چشم عبرت گیر خود
&&
ساز عبرت رهنمای سیر خود
\\
امتیاز آدمی از گاو و خر
&&
هم به فکر و عبرت آمد، ای پسر!
\\
چون شدی بی‌بهره از فکر ای دغل
&&
دان که «کا لانعام» باشی، بل أضل
\\
فکر یک ساعت تو را در امر دین
&&
افضل آمد از عبادات سنین
\\
ای خوشا نفسی که عبرت گیر شد
&&
در علاج نفس، با تدبیر شد
\\
تقوی قلب و صلاح واقعی
&&
هم به فکر و عبرت است، ای المعی
\\
ای رمیده طبع تو از ذی صلاح
&&
کرده‌ای خود غیبت نیکان مباح
\\
عالمی، گر پیرو سنت شود
&&
مقصدش زان پیروی، غربت شود
\\
چون رسد وقت نماز، از جا جهد
&&
ترک صحبت داده، شغل از کف نهد
\\
گوئیش: مرد ریاکاری بود
&&
اهل مشرب را به دل باری بود
\\
ور ز قید شرع بینی وا شده
&&
لاابالی گشته، بی‌پروا شده
\\
در عبادت کرده عادت، چون صبی
&&
آخر وقت و اقل واجبی
\\
صحبت هر صنف کافتد اتفاق
&&
باشد اندر وسعت خلقش وفاق
\\
نامیش با مشرب و بی‌ساخته
&&
گوئیش: اصلا ریا نشناخته
\\
بس سبکروح و لطیف و بامزه است
&&
گوئیا، نان و پنیر و خربزه است
\\
\end{longtable}
\end{center}
