\begin{center}
\section*{بخش ۱۰ - در توحید}
\label{sec:010}
\addcontentsline{toc}{section}{\nameref{sec:010}}
\begin{longtable}{l p{0.5cm} r}
دست او، طوق گردن جانت
&&
سر برآورده از گریبانت
\\
به تونزدیکتر ز حبل ورید
&&
تو در افتاده در ضلال بعید
\\
چند گردی به گرد هر سر کوی
&&
درد خود را دوا، هم از او جوی
\\
«لا» نهنگی است، کاینات آشام
&&
عرش تا فرش در کشیده به کام
\\
هر کجا کرده آن نهنگ آهنگ
&&
از من و ما نه بوی ماند و نه رنگ
\\
نقطه‌ای زین دوایر پرگار
&&
نیست بیرون ز دور این پرگار
\\
چه مرکب در این فضا، چه بسیط
&&
هست حکم فنا، به جمله محیط
\\
بلکه مقراض قهرمان حق است
&&
قاطع وصل کلمان حق است
\\
هندوی نفس راست غل دو شاخ
&&
تنگ کرده برو جهان فراخ
\\
دارد از «لا» فروغ، نور قدم
&&
گرچه «لا» داشت، تیرگی عدم
\\
چون کند «لا» بساط کثرت طی
&&
دهد «الا» ز جام وحدت، می
\\
\end{longtable}
\end{center}
