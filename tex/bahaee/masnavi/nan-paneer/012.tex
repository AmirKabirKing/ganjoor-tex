\begin{center}
\section*{بخش ۱۲ - فی التکلیف والشوق}
\label{sec:012}
\addcontentsline{toc}{section}{\nameref{sec:012}}
\begin{longtable}{l p{0.5cm} r}
هان، مدان بیگار تکلیفان عام
&&
هان! مدان ضایع رسالات و پیام
\\
باید اول آید از حق نهی و امر
&&
غیر مختص، نه به زید ونه به عمرو
\\
ز استماع آن دو تا بارز شده است
&&
شوق مکنونی که در نیک و بد است
\\
امر و نهی شرع و عقل و دین ز رب
&&
شرط شوق این و آن دان، نه سبب
\\
شرط اصلا محدث مشروط نیست
&&
گرچه از بهر حدوثش، بودنی است
\\
گر نباشد بارش نام از سما
&&
از زمین کی روید اقسام گیا
\\
گل، به فیض عام، روید از زمین
&&
لیک این باشد چنان و آن چنین
\\
این یکی خارست آن یک گل به ذات
&&
هر یکی دارد ز ذات خود صفات
\\
سنبل و گل، بهر روییدن دمید
&&
خار و خس را بهر تون او آفرید
\\
بارش اینها را چنین حالات داد
&&
پس به بارش، حال ذات از وی نزاد
\\
گر نکردی فهم، بگذر زین مقال
&&
خویش را ضایع مکن اندر جلال
\\
\end{longtable}
\end{center}
