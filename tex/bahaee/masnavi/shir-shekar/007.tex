\begin{center}
\section*{بخش ۷ - فی التوبة عن الخطایا و الانابة الی واهب العطایا}
\label{sec:007}
\addcontentsline{toc}{section}{\nameref{sec:007}}
\begin{longtable}{l p{0.5cm} r}
ای داده خلاصهٔ عمر به باد
&&
وی گشته به لهو و لعب، دلشاد
\\
ای مست ز جام هوا و هوس
&&
دیگر ز شراب معاصی بس
\\
تا چند روی به ره عاطل
&&
یک بار بخوان زهق الباطل
\\
زین بیش خطیئه پناه مباش
&&
مرغابی بحر گناه مباش
\\
از توبه بشوی گناه و خطا
&&
وز توبه بجوی نوال و عطا
\\
گر تو برسی به نعیم مقیم
&&
وز توبه رهی، ز عذاب الیم
\\
توبه، در صلح بود بارب
&&
این در می‌کوب، به صد یارب
\\
نومید مباش ز عفوالله
&&
ای مجرم عاصی نامه سیاه
\\
گرچه گنه تو ز عد بیش است
&&
عفو و کرمش از حد بیش است
\\
عفو ازلی که برون ز حد است
&&
خواهان گناه فزون ز عد است
\\
لیکن چندان، در جرم مپیچ
&&
کامکان صلح نماند هیچ
\\
تا چند کنی ای شیخ کبار
&&
توبه تلقین بهائی زار
\\
کو توبهٔ روز به شب شکند
&&
وین توبه به روز دگر فکند
\\
عمرش بگذشت، به لیت و عسی
&&
وز توبهٔ صبح، شکست مسا
\\
ای ساقی دلکش فرخ فال
&&
دارم ز حیات، هزار ملال
\\
در ده قدحی ز شراب طهور
&&
بر دل بگشا در عیش و سرور
\\
که گرفتارم به غم جانکاه
&&
زین توبهٔ سست بتر ز گناه
\\
ای ذاکر خاص بلند مقام!
&&
آزرده دلم ز غم ایام
\\
زین ذکر جدید فرح افزای
&&
غمهای جهان ز دلم بزدای
\\
می‌گو با ذوق و دل آگاه
&&
الله، الله، الله، الله
\\
کاین ذکر رفیع همایون فر
&&
وین نظم بدیع بلند اختر
\\
در بحر خبب، چو جلوه نمود
&&
درهای فرح بر خلق گشود
\\
آن را برخوان به نوای حزین
&&
وز قلهٔ عرش، بشنو تحسین
\\
یارب، به کرامت اهل صفا
&&
به هدایت پیشروان وفا
\\
کاین نامهٔ نامی نیک‌اثر
&&
کاورده ز عالم قدس خبر
\\
پیوسته، خجسته مقامش کن
&&
مقبول خواص و عوامش کن
\\
\end{longtable}
\end{center}
