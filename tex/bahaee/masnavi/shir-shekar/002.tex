\begin{center}
\section*{بخش ۲ - فی المناجات و الالتجاء الی قاضی الحاجات}
\label{sec:002}
\addcontentsline{toc}{section}{\nameref{sec:002}}
\begin{longtable}{l p{0.5cm} r}
زین رنج عظیم، خلاصی جو
&&
دستی به دعا بردار و بگو
\\
یارب، یارب، به کریمی تو
&&
به صفات کمال رحیمی تو
\\
یارب، به نبی و وصی و بتول
&&
یارب، به تقرب سبطین رسول
\\
یارب، به عبادت زین عباد
&&
به زهادت باقر علم و رشاد
\\
یارب، یارب، به حق صادق
&&
به حق موسی، به حق ناطق
\\
یارب، یارب، به رضا، شه دین
&&
آن ثامن من اهل یقین
\\
یارب، به تقی و مقاماتش
&&
یارب، به نقی و کراماتش
\\
یارب، به حسن، شه بحر و بر
&&
به هدایت مهدی دین‌پرور
\\
کاین بندهٔ مجرم عاصی را
&&
وین غرقهٔ بحر معاصی را
\\
از قید علائق جسمانی
&&
از بند وساوس شیطانی
\\
لطف بنما و خلاصش کن
&&
محرم به حریم خواصش کن
\\
یارب، یارب، که بهائی را
&&
این بیهده گرد هوائی را
\\
که به لهو و لعب، شده عمرش صرف
&&
ناخوانده ز لوح وفا یک حرف
\\
زین غم برهان که گرفتارست
&&
در دست هوی و هوس زارست
\\
در شغل ز خارف دنیی دون
&&
مانده به هزار امل، مفتون
\\
رحمی بنما به دل زارش
&&
بگشا به کرم، گره از کارش
\\
زین بیش مران، ز در احسان
&&
به سعادت ساحت قرب رسان
\\
وارسته ز دنیی دونش کن
&&
سر حلقهٔ اهل جنونش کن
\\
\end{longtable}
\end{center}
