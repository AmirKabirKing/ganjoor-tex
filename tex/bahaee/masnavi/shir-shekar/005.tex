\begin{center}
\section*{بخش ۵ - فی العلم النافع فی العماد}
\label{sec:005}
\addcontentsline{toc}{section}{\nameref{sec:005}}
\begin{longtable}{l p{0.5cm} r}
ای مانده ز مقصد اصلی دور!
&&
آکنده دماغ، ز باد غرور!
\\
از علم رسوم چه می‌جویی؟
&&
اندر طلبش، تا کی پویی؟
\\
تا چند زنی ز ریاضی لاف؟
&&
تا کی بافی هزار گزاف؟
\\
ز دوائر عشر و دقایق وی
&&
هرگز نبری، به حقایق پی
\\
وز جبر و مقابله و خطاین
&&
جبر نقصت نشود فی‌البین
\\
در روز پسین، که رسد موعود
&&
نرسد ز عراق و رهاوی سود
\\
زایل نکند ز تو مغبونی
&&
نه «شکل عروس» و نه «مأمونی»
\\
در قبر به وقت سؤال و جواب
&&
نفعی ندهد به تو اسطرلاب
\\
زان ره نبری به در مقصود
&&
فلسش قلب است و فرس نابود
\\
علمی بطلب که تو را فانی
&&
سازد ز علایق جسمانی
\\
علمی بطلب که به دل نور است
&&
سینه ز تجلی آن، طور است
\\
علمی که از آن چو شوی محظوظ
&&
گردد دل تو لوح المحفوظ
\\
علمی بطلب که کتابی نیست
&&
یعنی ذوقی است، خطابی نیست
\\
علمی که نسازدت از دونی
&&
محتاج به آلت قانونی
\\
علمی بطلب که جدالی نیست
&&
حالی است تمام و مقالی نیست
\\
علمی که مجادله را سبب است
&&
نورش ز چراغ ابولهب است
\\
علمی بطلب که گزافی نیست
&&
اجماعیست و خلافی نیست
\\
علمی که دهد به تو جان نو
&&
علم عشق است، ز من بشنو
\\
به علوم غریبه تفاخر چند
&&
زین گفت و شنود، زبان در بند
\\
سهل است نحاس که زر کردی
&&
زر کن مس خویش تو اگر مردی
\\
از جفر و طلسم، به روز پسین
&&
نفعی نرسد به تو ای مسکین
\\
بگذر ز همه، به خودت پرداز
&&
کز پرده برون نرود آواز
\\
آن علم تو را کند آماده
&&
از قید جهان کند آزاده
\\
عشق است کلید خزاین جود
&&
ساری در همه ذرات وجود
\\
غافل، تو نشسته به محنت و رنج
&&
واندر بغل تو کلید گنج
\\
جز حلقهٔ عشق مکن در گوش
&&
از عشق بگو، در عشق بکوش
\\
علم رسمی همه خسران است
&&
در عشق آویز، که علم آن است
\\
آن علم ز تفرقه برهاند
&&
آن علم تو را ز تو بستاند
\\
آن علم تو را ببرد به رهی
&&
کز شرک خفی و جلی برهی
\\
آن علم ز چون و چرا خالیست
&&
سرچشمهٔ آن، علی عالیست
\\
ساقی، قدحی ز شراب الست
&&
که نه خستش پا، نه فشردش دست
\\
در ده به بهائی دلخسته
&&
آن، دل به قیود جهان بسته
\\
تا کندهٔ جاه ز پا شکند
&&
وین تخته کلاه ز سر فکند
\\
\end{longtable}
\end{center}
