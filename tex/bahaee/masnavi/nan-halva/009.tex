\begin{center}
\section*{بخش ۹ - فی تأویل قول النبی صلی الله علیه و آله و سلم: حب الوطن من الایمان}
\label{sec:009}
\addcontentsline{toc}{section}{\nameref{sec:009}}
\begin{longtable}{l p{0.5cm} r}
ایهاالمأثور فی قید الذنوب
&&
ایها المحروم من سر الغیوب
\\
لا تقم فی اسر لذات الجسد
&&
انها فی جید حبل من مسد
\\
قم توجه شطر اقلیم النعیم
&&
و اذکر الاوطان والعهد القدیم
\\
گنج علم «ما ظهر مع ما بطن»
&&
گفت: از ایمان بود حب الوطن
\\
این وطن، مصر و عراق و شام نیست
&&
این وطن، شهریست کان را نام نیست
\\
زانکه از دنیاست، این اوطان تمام
&&
مدح دنیا کی کند «خیر الانام»
\\
حب دنیا هست رأس هر خطا
&&
از خطا کی می‌شود ایمان عطا
\\
ای خوش آنکو یابد از توفیق بهر
&&
کاورد رو سوی آن بی‌نام شهر
\\
تو در این اوطان، غریبی ای پسر!
&&
خو به غربت کرده‌ای، خاکت به سر!
\\
آنقدر در شهر تن ماندی اسیر
&&
کان وطن، یکباره رفتت از ضمیر
\\
رو بتاب از جسم و، جان را شاد کن
&&
موطن اصلی خود را یاد کن
\\
زین جهان تا آن جهان بسیار نیست
&&
در میان، جز یک نفس در کار نیست
\\
تا به چند ای شاهباز پر فتوح
&&
باز مانی دور، از اقلیم روح؟
\\
حیف باشد از تو، ای صاحب هنر!
&&
کاندرین ویرانه ریزی بال و پر
\\
تا به کی ای هدهد شهر سبا
&&
در غریبی مانده باشی، بسته پا؟
\\
جهد کن! این بند از پا باز کن
&&
بر فراز لامکان پرواز کن
\\
تا به کی در چاه طبعی سرنگون؟
&&
یوسفی، یوسف، بیا از چه برون
\\
تا عزیز مصر ربانی شوی
&&
وا رهی از جسم و روحانی شوی
\\
\end{longtable}
\end{center}
