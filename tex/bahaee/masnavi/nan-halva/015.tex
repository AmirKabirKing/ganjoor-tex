\begin{center}
\section*{بخش ۱۵ - فی ذم المتهمین بجمع أسباب الدنیا، المعرضین عن تحصیل أسباب العقبی}
\label{sec:015}
\addcontentsline{toc}{section}{\nameref{sec:015}}
\begin{longtable}{l p{0.5cm} r}
نان و حلوا چیست؟ اسباب جهان
&&
کافت جان کهانست و مهان
\\
آنکه از خوف خدا دورت کند
&&
آنکه از راه هدی دورت کند
\\
آنکه او را بر سر او باختی
&&
وز ره تحقیق، دور انداختی
\\
تلخ کرد این نان و حلوا کام تو
&&
برد آخر، رونق اسلام تو
\\
برکن این اسباب را از بیخ و بن
&&
دل دل، این نارهوس را سرد کن
\\
آتش اندر زن در این حلوا و نان
&&
وارهان خود را از این باد گران
\\
از پی آن می‌دوی از جان و دل
&&
وز پی این مانده‌ای چون خر به گل
\\
الله الله، این چه اسلام است و دین
&&
ترک شد آئین رب العالمین
\\
جمله سعیت، بهر دنیای دنی است
&&
بهر عقبی، می‌ندانی، سعی چیست
\\
در ره آن موشکافی، ای شقی
&&
در ره این، کند فهم و احمقی
\\
\end{longtable}
\end{center}
