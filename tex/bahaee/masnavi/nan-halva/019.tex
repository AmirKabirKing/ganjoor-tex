\begin{center}
\section*{بخش ۱۹ - فی ذم المتمکنین فی المناصب الدنیویة للحظوظ الواهیة الدنیة}
\label{sec:019}
\addcontentsline{toc}{section}{\nameref{sec:019}}
\begin{longtable}{l p{0.5cm} r}
نان و حلوا چیست؟ ای فرزانه مرد
&&
منصب دنیاست، گرد آن مگرد
\\
گر بیالایی از او دست و دهان
&&
روی آسایش نبینی در جهان
\\
منصب دنیا نمی‌دانی که چیست؟
&&
من بگویم با تو، یک ساعت بایست
\\
آنکه بندد از ره حق پای مرد
&&
آنکه سازد کوی حرمان جای مرد
\\
آنکه نامش مایهٔ بدنامی است
&&
آنکه کامش، سر به سر، ناکامی است
\\
آنکه هر ساعت، نهان از خاص و عام
&&
کاسهٔ زهرت فرو ریزد به کام
\\
بر سر این زهر روزان و شبان
&&
چند خواهی بود لرزان و تپان؟
\\
منصب دنیاست، ای نیکونهاد!
&&
آنکه داده خرمن دینت به باد
\\
منصب دنیاست، ای صاحب فنون!
&&
آنکه کردت این چنین، خوار و زبون
\\
ای خوش آن دانا که دنیا را بهشت
&&
رفت همچون شاه مردان در بهشت
\\
مولوی معنوی در مثنوی
&&
نکته‌ای گفته است، هان تا بشنوی:
\\
« ترک دنیا گیر تا سلطان شوی
&&
ورنه گر چرخی تو، سرگردان شوی
\\
زهر دارد در درون، دنیا چو مار
&&
گرچه دارد در برون، نقش و نگار
\\
زهر این مار منقش، قاتل است
&&
می‌گریزد زو هر آن کس عاقل است»
\\
زین سبب، فرمود شاه اولیا
&&
آن گزین اولیا و انبیا:
\\
حب الدنیا، رأس کل خطیة
&&
و ترک الدنیا رأس کل عبادة
\\
\end{longtable}
\end{center}
