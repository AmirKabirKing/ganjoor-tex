\begin{center}
\section*{بخش ۸ - فی الفوائد المتفرقة فیما یتضمن الاشارة الی  قوله تعالی ان الله یأمرکم أن تذبحوا بقرة}
\label{sec:008}
\addcontentsline{toc}{section}{\nameref{sec:008}}
\begin{longtable}{l p{0.5cm} r}
ابذلوا اروا حکم یا عاشقین
&&
ان تکونوا فی هوانا صادقین
\\
داند این را هرکه زین ره آگه است
&&
کاین وجود و هستیش، سنگ ره است
\\
گوی دولت آن سعادتمند برد
&&
کو، به پای دلبر خود، جان سپرد
\\
جان به بوسی می‌خرد آن شهریار
&&
مژده‌ای عشاق، کسان گشت کار
\\
گر همی خواهی حیات و عیش خوش
&&
گاو نفس خویش را اول بکش
\\
در جوانی کن نثار دوست جان
&&
رو «عوان بین ذالک» را بخوان
\\
پیر چون گشتی، گران جانی مکن
&&
گوسفند پیر قربانی مکن
\\
شد همه برباد، ایام شباب
&&
بهر دین، یک ذره ننمودی شتاب
\\
عمرت از پنجه گذشت و یک سجود
&&
کت به کار آید، نکردی ای جهود!
\\
حالیا، ای عندلیب کهنه سال
&&
ساز کن افغان و یک چندی بنال
\\
چون نکردی ناله در فصل بهار
&&
در خزان، باری قضا کن زینهار!
\\
تا که دانستی زیانت را ز سود
&&
توبه‌ات نسیه، گناهت نقد بود
\\
غرق دریای گناهی تا به کی؟
&&
وز معاصی روسیاهی تا به کی؟،
\\
جد تو آدم، بهشتش جای بود
&&
قدسیان کردند پیش او سجود
\\
یک گنه چون کرد، گفتندش: تمام
&&
مذنبی، مذنب، برو بیرون خرام!
\\
تو طمع داری که با چندین گناه
&&
داخل جنت شوی، ای روسیاه!
\\
\end{longtable}
\end{center}
