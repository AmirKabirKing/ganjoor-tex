\begin{center}
\section*{بخش ۱۸ - حکایة العابد الذی کان قوته العلف لیأمن دینه من التلف}
\label{sec:018}
\addcontentsline{toc}{section}{\nameref{sec:018}}
\begin{longtable}{l p{0.5cm} r}
نوجوانی از خواص پادشاه
&&
می‌شدی، با حشمت و تمکین، به راه
\\
دل ز غم خالی و سر پر از هوس
&&
جمله اسباب تنعم پیش و پس
\\
بر یکی عابد، در آن صحرا گذشت
&&
کاو علف می‌خورد، آن آهوی دشت
\\
هر زمان، در ذکر حی لایموت
&&
شکر گویان کش میسر گشت قوت
\\
نوجوان سویش خرامید و بگفت:
&&
کای شده با وحشیان در قوت جفت!
\\
سبز گشته، چون زمرد، رنگ تو
&&
چونکه ناید جز علف در چنگ تو
\\
شد تنت چون عنکبوت، از لاغری
&&
چون گوزنان، چند در صحرا چری؟
\\
گر چو من بودی تو خدمتگار شاه
&&
در علف خوردن نمی‌گشتی تباه
\\
پیر گفتش: کای جوان نامدار
&&
کت بود از خدمت شه افتخار
\\
گر چو من، تو نیز می‌خوردی علف
&&
کی شدی عمرت در این خدمت تلف؟
\\
\end{longtable}
\end{center}
