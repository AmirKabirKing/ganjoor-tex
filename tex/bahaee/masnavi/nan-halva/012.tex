\begin{center}
\section*{بخش ۱۲ - فی الریا و التلبیس بالذین هم أعظم جنود ابلیس}
\label{sec:012}
\addcontentsline{toc}{section}{\nameref{sec:012}}
\begin{longtable}{l p{0.5cm} r}
نان و حلوا چیست ای شوریده سر؟
&&
متقی خود را نمودن بهر زر
\\
دعوی زهد از برای عز و جاه
&&
لاف تقوی، از پی تعظیم شاه
\\
تو نپنداری کزین لاف و دروغ
&&
هرگز افتد نان تلبیست به دوغ؟
\\
خرده بینانند در عالم بسی
&&
واقفند از کار و بار هر کسی
\\
زیرکانند از یسار و از یمین
&&
از پی رد و قبول، اندر کمین
\\
با همه خودبینی و کبر و منی
&&
لاف تقوی و عدالت می‌زنی
\\
سر به سر، کار تو در لیل و نهار
&&
سعی در تحصیل جاه و اعتبار
\\
دین فروشی، از پی مال حرام
&&
مکر و حیله، بهر تسخیر عوام
\\
خوردن مال شهان، با زرق و شید
&&
گاه خبث عمرو، گاهی خبث زید
\\
وین عدالت با وجود این صفات
&&
هست دائم، برقرار و برثبات!
\\
بر سرش، داخل نگردد «لا» و «لیس»
&&
این عدالت هست کوه بوقبیس
\\
می‌نیابد اختلال از هیچ چیز
&&
چون وضوی محکم «بی‌بی تمیز»
\\
\end{longtable}
\end{center}
