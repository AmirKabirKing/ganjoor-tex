\begin{center}
\section*{بخش ۲۰ - فی الترغیب فی حفظ اللسان و هو من احسن صفات الانسان}
\label{sec:020}
\addcontentsline{toc}{section}{\nameref{sec:020}}
\begin{longtable}{l p{0.5cm} r}
نان و حلوا چیست؟ قیل و قال تو
&&
وین زبان پردازی بی‌حال تو
\\
گوش بگشا، لب فرو بند از مقال
&&
هفته هفته، ماه ماه و سال سال
\\
صمت عادت کن که از یک گفتنک
&&
می‌شود تاراج، این تخت الحنک
\\
ای خوش آنکو رفت در حصن سکوت
&&
بسته دل در یاد «حی لایموت»
\\
رو نشین خاموش، چندان ای فلان
&&
که فراموشت شود، نطق و بیان
\\
خامشی باشد، نشان اهل حال
&&
گر بجنبانند لب، گردند لال
\\
چند با این ناکسان بی‌فروغ
&&
باده پیمایی، دروغ اندر دروغ
\\
وارهان خود را از این همصحبتان
&&
جمله مهتابند و دین تو، کتان
\\
صحبت نیکانت ارنبود نصیب
&&
باری از همصحبتان بد شکیب
\\
\end{longtable}
\end{center}
