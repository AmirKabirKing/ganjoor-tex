\begin{center}
\section*{بخش ۲۱ - فی ذم من تشبة بالفقراء لسالکین و هو فی زمرة اشقیاء الهالکین}
\label{sec:021}
\addcontentsline{toc}{section}{\nameref{sec:021}}
\begin{longtable}{l p{0.5cm} r}
نان و حلوا چیست؟ این اعمال تو
&&
جبهٔ پشمین، ردا و شال تو
\\
این مقام فقر خورشید اقتباس
&&
کی شود حاصل کسی را در لباس
\\
زین ردا و جبه‌ات، ای کج نهاد!
&&
این دو بیت از مثنوی آمد به یاد:
\\
«ظاهرت، چون گور کافر پر حلل
&&
وز درون، قهر خدا عز و جل
\\
از برون، طعنه زنی بر بایزید
&&
وز درونت، ننگ می‌دارد یزید»
\\
رو بسوز! این جبهٔ ناپاک را
&&
وین عصا و شانه و مسواک را
\\
ظاهرت، گر هست با باطن یکی
&&
می‌توان ره یافت بر حق، اندکی
\\
ور مخالف شد درونت با برون
&&
رفته باشی در جهنم، سرنگون
\\
ظاهر و باطن، یکی باید، یکی
&&
تابیابی راه حق را، اندکی
\\
\end{longtable}
\end{center}
