\begin{center}
\section*{بخش ۴ - فی التأسف و الندامة علی صرف العمر فیما  لاینفع فی القیامة و تأویل قول النبی صلی الله علیه و آله و سلم: «سر الممن شفاء»}
\label{sec:004}
\addcontentsline{toc}{section}{\nameref{sec:004}}
\begin{longtable}{l p{0.5cm} r}
قد صرفت العمر فی قیل و قال
&&
یا ندیمی قم، فقد ضاق المجال
\\
و اسقنی تلک المدام السلسبیل
&&
انها تهدی الی خیر السبیل
\\
و اخلع النعلین، یا هذا الندیم
&&
انها نار أضائت للکلیم
\\
هاتها صهباء من خمر الجنان
&&
دع کئوسا و اسقنیها بالدنان
\\
ضاق وقت العمر عن آلاتها
&&
هاتها من غیر عصر هاتها
\\
قم ازل عنی بها رسم الهموم
&&
ان عمری ضاع فی علم الرسوم
\\
قل لشیخ قلبه منها نفور
&&
لا تخف، الله تواب غفور
\\
علم رسمی سر به سر قیل است و قال
&&
نه از او کیفیتی حاصل، نه حال
\\
طبع را افسردگی بخشد مدام
&&
مولوی باور ندارد این کلام
\\
وه! چه خوش می‌گفت در راه حجاز
&&
آن عرب، شعری به آهنگ حجاز:
\\
کل من لم یعشق الوجه الحسن
&&
قرب الجل الیه و الرسن
\\
یعنی: «آن کس را که نبود عشق یار
&&
بهر او پالان و افساری بیار»
\\
گر کسی گوید که: از عمرت همین
&&
هفت روزی مانده، وان گردد یقین
\\
تو در این یک هفته، مشغول کدام
&&
علم خواهی گشت، ای مرد تمام؟
\\
فلسفه یا نحو یا طب یا نجوم
&&
هندسه یا رمل یا اعداد شوم
\\
علم نبود غیر علم عاشقی
&&
مابقی تلبیس ابلیس شقی
\\
علم فقه و علم تفسیر و حدیث
&&
هست از تلبیس ابلیس خبیث
\\
زان نگردد بر تو هرگز کشف راز
&&
گر بود شاگر تو صد فخر راز
\\
هر که نبود مبتلای ماهرو
&&
اسم او از لوح انسانی بشو
\\
دل که خالی باشد از مهر بتان
&&
لتهٔ حیض به خون آغشته دان
\\
سینهٔ خالی ز مهر گلرخان
&&
کهنه انبانی بود پر استخوان
\\
سینه، گر خالی ز معشوقی بود
&&
سینه نبود، کهنه صندوقی بود
\\
تا به کی افغان و اشک بی‌شمار؟
&&
از خدا و مصطفی شرمی بدار
\\
از هیولا، تا به کی این گفتگوی؟
&&
رو به معنی آر و از صورت مگوی
\\
دل، که فارغ شد ز مهر آن نگار
&&
سنگ استنجای شیطانش شمار
\\
این علوم و این خیالات و صور
&&
فضلهٔ شیطان بود بر آن حجر
\\
تو، بغیر از علم عشق ار دل نهی
&&
سنگ استنجا به شیطان می‌دهی
\\
شرم بادت، زانکه داری، ای دغل!
&&
سنگ استنجای شیطان در بغل
\\
لوح دل، از فضلهٔ شیطان بشوی
&&
ای مدرس! درس عشقی هم بگوی
\\
چند و چند از حکمت یونانیان؟
&&
حکمت ایمانیان را هم بدان
\\
چند زین فقه و کلام بی‌اصول
&&
مغز را خالی کنی، ای بوالفضول
\\
صرف شد عمرت به بحث نحو و صرف
&&
از اصول عشق هم خوان یک دو حرف
\\
دل منور کن به انوار جلی
&&
چند باشی کاسه لیس بوعلی؟
\\
سرور عالم، شه دنیا و دین
&&
سؤر مؤمن را شفا گفت ای حزین
\\
سؤر رسطالیس و سؤر بوعلی
&&
کی شفا گفته نبی منجلی؟
\\
سینهٔ خود را برو صد چاک کن
&&
دل از این آلودگیها پاک کن
\\
\end{longtable}
\end{center}
