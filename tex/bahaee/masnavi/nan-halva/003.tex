\begin{center}
\section*{بخش ۳ - حکایة فی بعض اللیالی}
\label{sec:003}
\addcontentsline{toc}{section}{\nameref{sec:003}}
\begin{longtable}{l p{0.5cm} r}
شب که بودم با هزاران کوه درد
&&
سر به زانوی غمش، بنشسته فرد
\\
جان به لب، از حسرت گفتار او
&&
دل، پر از نومیدی دیدار او
\\
آن قیامت قامت پیمان شکن
&&
آفت دوران، بلای مرد و زن
\\
فتنهٔ ایام و آشوب جهان
&&
خانه سوز صد چو من، بی‌خانمان
\\
از درم ناگه درآمد، بی‌حجاب
&&
لب گزان، از رخ برافکنده نقاب
\\
کاکل مشکین به دوش انداخته
&&
وز نگاهی، کار عالم ساخته
\\
گفت: ای شیدا دل محزون من!
&&
وی بلاکش عاشق مفتون من
\\
کیف حال القلب فی نار الفراق؟
&&
گفتمش: والله حالی لایطاق
\\
یک دمک، بنشست بر بالین من
&&
رفت و با خود برد عقل و دین من
\\
گفتمش: کی بینمت ای خوش خرام؟
&&
گفت: نصب اللیل لکن فی‌المنام
\\
\end{longtable}
\end{center}
