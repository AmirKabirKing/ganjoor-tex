\begin{center}
\section*{بخش ۱۰ - فی أن البلایا و المحن فی هذا الطریق، وان کانت عسیرة، لکنها علی المحب یسیرة بل  هی الراحة العظمی والنعمة الکبری}
\label{sec:010}
\addcontentsline{toc}{section}{\nameref{sec:010}}
\begin{longtable}{l p{0.5cm} r}
ایها القلب الحزین المبتلا
&&
فی طریق العشق انواع البلا
\\
لیکن القلب العشوق الممتحن
&&
لا یبالی بالبلایا و المحن
\\
سهل باشد در ره فقر و فنا
&&
گر رسد تن را تعب، جان را عنا
\\
رنج راحت دان، چو شد مطلب بزرگ
&&
گرد گله، توتیای چشم گرگ
\\
کی بود در راه عشق آسودگی؟
&&
سر به سر درد است و خون آلودگی
\\
تا نسازی بر خود آسایش حرام
&&
کی توانی زد به راه عشق، گام؟
\\
غیر ناکامی، دراین ره، کام نیست
&&
راه عشق است این، ره حمام نیست
\\
ترککان، چون اسب یغما پی کنند
&&
هرچه باشد، خود به غارت می‌برند
\\
ترک ما، برعکس باشد کار او
&&
حیرتی دارم ز کار و بار او
\\
کافرست و غارت دین می‌کند
&&
من نمی‌دانم چرا این می‌کند؟
\\
نیست جز تقوی، در این ره توشه‌ای
&&
نان و حلوا را بهل در گوشه‌ای
\\
نان و حلوا چیست؟ جاه و مال تو
&&
باغ و راغ و حشمت و اقبال تو
\\
نان و حلوا چیست؟ این طول امل
&&
وین غرور نفس و علم بی‌عمل
\\
نان و حلوا چیست؟ گوید با تو، فاش
&&
این همه سعی تو از بهر معاش
\\
نان و حلوا چیست؟ فرزند و زنت
&&
اوفتاده همچو غل در گردنت
\\
چند باشی بهر این حلوا و نان
&&
زیر منت، از فلان و از فلان؟
\\
برد این حلوا و نان، آرام تو
&&
شست از لوح تو کل نام تو
\\
هیچ بر گوشت نخورده است، ای لیم!
&&
حرف «الرزق علی الله الکریم»
\\
رو قناعت پیشه کن در کنج صبر
&&
پند بپذیر از سگ آن پیر گبر
\\
\end{longtable}
\end{center}
