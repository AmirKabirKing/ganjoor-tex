\begin{center}
\section*{بخش ۱۷ - فی ذم من یتفاخر بتقرب الملوک مع أنه یزعم الانخراط فی سلک أهل السلوک}
\label{sec:017}
\addcontentsline{toc}{section}{\nameref{sec:017}}
\begin{longtable}{l p{0.5cm} r}
نان و حلوا چیست، دانی ای پسر؟
&&
قرب شاهان است، زین قرب، الحذر
\\
می‌برد هوش از سر و از دل قرار
&&
الفرار از قرب شاهان، الفرار
\\
فرخ آنکو رخش همت را بتاخت
&&
کام از این حلوا و نان، شیرین نساخت
\\
قرب شاهان، آفت جان تو شد
&&
پایبند راه ایمان تو شد
\\
جرعه‌ای از نهر قرآن نوش کن
&&
آیهٔ «لا ترکنوا» را گوش کن
\\
لذت تخصیص او وقت خطاب
&&
آن کند که ناید از صد خم شراب
\\
هر زمان که شاه گوید: شیخنا!
&&
شیخنا مدهوش گردد، زین ندا
\\
مست و مدهوش از خطاب شه شود
&&
هر دمی در پیش شه، سجده رود
\\
می‌پرستد گوییا او شاه را
&&
هیچ نارد یاد، آن الله را
\\
الله الله، این چه اسلام است و دین
&&
شرک باشد این، به رب‌العالمین
\\
\end{longtable}
\end{center}
