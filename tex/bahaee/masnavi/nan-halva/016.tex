\begin{center}
\section*{بخش ۱۶ - سال بعض العارفین عن بعض المنعمین عن قدر سعیه فی تحصیل الاسباب الدنیویة و تقصیرة عن اسباب الاخرویة}
\label{sec:016}
\addcontentsline{toc}{section}{\nameref{sec:016}}
\begin{longtable}{l p{0.5cm} r}
عارفی از منعمی کرد این سؤال:
&&
کای تو را دل در پی مال و منال
\\
سعی تو، از بهر دنیای دنی
&&
تا چه مقدار است؟ ای مرد غنی!
\\
گفت: بیرون است از حد شمار
&&
کار من این است در لیل و نهار
\\
عارفش گفت: این که بهرش در تکی
&&
حاصلت زان چیست؟ گفتا: اندکی
\\
آنچه مقصود است، ای روشن ضمیر!
&&
برنیاید زان، مگر عشر عشیر
\\
گفت عارف: آن که هستی روز و شب
&&
از پی تحصیل آن، در تاب و تب
\\
شغل آن را قبلهٔ خود ساختی
&&
عمر خود را بر سر آن باختی
\\
آنچه او می‌خواستی، واصل نشد
&&
مدعای تو از آن، حاصل نشد
\\
دار عقبی، کان ز دنیا برتر است
&&
وز پی آن، سعی خواجه کمتر است
\\
چون شود حاصل تو را چیزی از آن؟
&&
من نگویم، خود بگو، ای نکته‌دان!
\\
\end{longtable}
\end{center}
