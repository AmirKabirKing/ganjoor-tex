\begin{center}
\section*{بخش ۶ - فی قطع العلائق و العزلة عن الخلایق}
\label{sec:006}
\addcontentsline{toc}{section}{\nameref{sec:006}}
\begin{longtable}{l p{0.5cm} r}
هر که را توفیق حق آمد دلیل
&&
عزلتی بگزید و رست از قال و قیل
\\
عزت اندر عزلت آمد، ای فلان
&&
تو چه خواهی ز اختلاط این و آن؟
\\
پا مکش از دامن عزلت به در!
&&
چند گردی چون گدایان در به در؟
\\
گر ز دیو نفس می‌جویی امان
&&
رو نهان شو! چون پری از مردمان
\\
از حقیقت بر تو نگشاید دری
&&
زین مجازی مردمان تا نگذری
\\
گر تو خواهی عزت دنیا و دین
&&
عزلتی از مردم دنیا گزین
\\
گنج خواهی؟ کنج عزلت کن مقام
&&
واستتر واستخف، عن کل الانام
\\
چون شب قدر از همه مستور شد
&&
لاجرم، از پای تا سر نور شد
\\
اسم اعظم، چون که کس نشناسدش
&&
سروری بر کل اسما باشدش
\\
تا تو نیز از خلق پنهانی همی
&&
لیلةالقدری و اسم اعظمی
\\
رو به عزلت آر، ای فرزانه مرد!
&&
وز جمیع ماسوی الله باش فرد
\\
عزلت آمد گنج مقصود ای حزین!
&&
لیک، گر با زهد و علم آید قرین
\\
عزلت بی«زای» زاهد علت است
&&
ور بود بی«عین» علم، آن زلت است
\\
عزلت بی«عین»، عین زلت است
&&
ور بود بی«زای» اصل علت است
\\
زهد و علم ار مجتمع نبود به هم
&&
کی توان زد در ره عزلت قدم؟
\\
علم چبود؟ از همه پرداختن
&&
جمله را در داو اول باختن
\\
این هوسها از سرت بیرون کند
&&
خوف و خشیت، در دلت افزون کند
\\
«خشیة الله» را نشان علم دان!
&&
«انما یخشی»، تو در قرآن بخوان!
\\
سینه را از علم حق آباد کن!
&&
رو حدیث «لو علمتم» یاد کن!
\\
\end{longtable}
\end{center}
