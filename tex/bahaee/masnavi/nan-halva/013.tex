\begin{center}
\section*{بخش ۱۳ - علی سبیل التمثیل}
\label{sec:013}
\addcontentsline{toc}{section}{\nameref{sec:013}}
\begin{longtable}{l p{0.5cm} r}
بود در شهر هری، بیوه زنی
&&
کهنه رندی، حیله‌سازی، پرفنی
\\
نام او، بی‌بی تمیز خالدار
&&
در نمازش، بود رغبت بیشمار
\\
با وضوی صبح، خفتن می‌گزارد
&&
نامرادان را بسی دادی مراد
\\
کم نشد هرگز دواتش از قلم
&&
بر مراد هرکسی، می‌زد رقم
\\
در مهم سازی اوباش و رنود
&&
دائما، طاحونه‌اش در چرخ بود
\\
از ته هر کس که برجستی به ناز
&&
می‌شدی فی‌الحال، مشغول نماز
\\
هرکه آمد، گفت: بر من کن دعا
&&
او به جای دست، برمی‌داشت پا
\\
بابها مفتوحة للداخلین
&&
رجلها، مرفوعة للفاعلین
\\
گفت با او رندکی، کای نیک زن
&&
حیرتی دارم، درین کار تو من
\\
زین جنابتهای پی‌درپی که هست
&&
هیچ ناید در وضوی تو شکست
\\
نیت و آداب این محکم وضو
&&
یک ره از روی کرم، با من بگو
\\
این وضو از سنگ و رو محکمتر است
&&
این وضو نبود، سد اسکندر است
\\
\end{longtable}
\end{center}
