\begin{center}
\section*{بخش ۲۲ - فیما یتضمن الاشارة الی قول سید الاوصیاء صلوات الله علیه و آله: «ما عبدتک خوفا من نارک و لا طمعا فی جنتک، بل وجدتک اهلا للعبادة فعبدتک»}
\label{sec:022}
\addcontentsline{toc}{section}{\nameref{sec:022}}
\begin{longtable}{l p{0.5cm} r}
نان و حلوا چیست؟ ای نیکو سرشت
&&
این عبادتهای تو بهر بهشت
\\
نزد اهل حق، بود دین کاستن
&&
در عبادت، مزد از حق خواستن
\\
رو حدیث ما عبدتک، ای فقیر
&&
از کلام شاه مردان، یاد گیر
\\
چشم بر اجر عمل، از کوری است
&&
طاعت از بهر طمع، مزدوری است
\\
خادمان، بی‌مزد گیرند این گروه
&&
خدمت با مزد، کی دارد شکوه؟
\\
عابدی کاو اجرت طاعات خواست
&&
گر تو ناعابد نهی نامش، رواست
\\
تا به کی بر مزد داری چشم تیز!
&&
مزد از این بهتر چه خواهی، ای عزیز
\\
کاو تو را از فضل و لطف با مزید
&&
از برای خدمت خود آفرید
\\
با همه آلودگی، قدرت نکاست
&&
بر قدت تشریف خدمت کرد راست
\\
\end{longtable}
\end{center}
