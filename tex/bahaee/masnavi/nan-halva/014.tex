\begin{center}
\section*{بخش ۱۴ - فی ذم أصحاب التدریس مقصد هم مجرد أظهار الفضل و التلبیس}
\label{sec:014}
\addcontentsline{toc}{section}{\nameref{sec:014}}
\begin{longtable}{l p{0.5cm} r}
نان و حلوا چیست؟ این تدریس تو
&&
کان بود سرمایهٔ تلبیس تو
\\
بهر اظهار فضیلت، معرکه
&&
ساختی، افتادی اندر مهلکه
\\
تا که عامی چند سازی دام خویش
&&
با صد افسون، آوری در دام خویش
\\
چند بگشایی سر انبان لاف؟
&&
چند پیمایی گزاف اندر گزاف؟
\\
نی فروعت محکم آمد، نی اصول
&&
شرم بادت از خدا و از رسول
\\
اندرین ره چیست دانی غول تو؟
&&
این ریایی درس نامعقول تو
\\
درس اگر قربت نباشد زان غرض
&&
لیس درسا انه بئس المرض
\\
اسب دولت، برفراز عرش تاخت
&&
آنکه خود را زین مرض آزاد ساخت
\\
\end{longtable}
\end{center}
