\begin{center}
\section*{بخش ۲۴ - فی نغمات الجنان من جذبات الرحمان}
\label{sec:024}
\addcontentsline{toc}{section}{\nameref{sec:024}}
\begin{longtable}{l p{0.5cm} r}
اشف قلبی، ایها الساقی الرحیم
&&
بالتی یحیی بها العظم الرمیم
\\
زوج الصهباء بالماء الزلال
&&
واجعلن عقلی لها مهرا حلال
\\
بنت کرم تجعلن الشیخ شاب
&&
من یذق منها عن الکونین غاب
\\
خمرة من نار موسی نورها
&&
دنها قلبی و صدری طورها
\\
قم فلاتمهل، فما فی‌العمر مهل
&&
لا تصعب شربها و الامر سهل
\\
قم فلاتمهل فان الصبح لاح
&&
والثریا غربت والدیک صاح
\\
قل لشیخ قلبه منها نفور
&&
لا تخف، فالله تواب غفور
\\
یا مغنی ان عندی کل غم
&&
قم والق النار فیها بالنغم
\\
یا مغنی قم فان العمر ضاع
&&
لا یطیب العیش الا بالسماع
\\
انت ایضا یا مغنی لا تنم
&&
قم واذهب عن فؤادی کل غم
\\
غن لی دورا، فقد دار القدح
&&
والصبا قد فاح والقمری صدح
\\
واذکرن عندی احادیث الحبیب
&&
ان عیشی من سواها لا یطیب
\\
واذکرن ذکری احادیث الفراق
&&
ان ذکر البعد مما لا یطاق
\\
روحن روحی باشعار العرب
&&
کی یتم الحظ فینا والطرب
\\
وافتحن منها بنظم مستطاب
&&
قلته فی بعض ایام الشباب
\\
قد صرفنا العمر فی قیل و قال
&&
یا ندیمی! قم فقد ضاق المجال
\\
ثم اطربنی باشعار العجم
&&
و اطردن همتا علی قلبی هجم
\\
وابتداء منها ببیت المثنوی
&&
للحکیم المولوی المعنوی
\\
« بشنو از نی، چون حکایت می‌کند
&&
وز جداییها، شکایت می‌کند»
\\
قم و خاطبنی بکل الالسنه
&&
عل قلبی ینتبة من ذی السنه
\\
انه فی غفلة عن حاله
&&
خائض فی قیله مع قاله
\\
کل ان فهو فی قید جدید
&&
قائلا من جهله: هل من مزید
\\
تائه فی الغی قد ضل الطریق
&&
قط من سکرالهوی لا یستفیق
\\
عاکف دهرا، علی اصنامه
&&
تنفر الکفار من اسلامه
\\
کم انادی و هو لایسقی یصغی؟ التناد
&&
وافادی، وافادی، وافاد
\\
یا بهائی اتخذ قلبا سواه
&&
فهو ما معبوده الا هواه
\\
هر چت از حق باز دارد ای پسر
&&
نام کردن، نان و حلوا، سر به سر
\\
گر همی خواهی که باشی تازه‌جان
&&
رو کتاب نان و حلوا را بخوان
\\
\end{longtable}
\end{center}
