\begin{center}
\section*{بخش ۱۱ - حکایة العابد الذی قل الصبر لدیه فتفوق الکلب علیه}
\label{sec:011}
\addcontentsline{toc}{section}{\nameref{sec:011}}
\begin{longtable}{l p{0.5cm} r}
عابدی، در کوه لبنان بد مقیم
&&
در بن غاری، چو اصحاب الرقیم
\\
روی دل، از غیر حق برتافته
&&
گنج عزت را ز عزلت یافته
\\
روزها، می‌بود مشغول صیام
&&
قرص نانی، می‌رسیدش وقت شام
\\
نصف آن شامش بدی، نصفی سحور
&&
وز قناعت، داشت در دل صد سرور
\\
بر همین منوال، حالش می‌گذشت
&&
نامدی زان کوه، هرگز سوی دشت
\\
از قضا، یک شب نیامد آن رغیف
&&
شد ز جوع، آن پارسا زار و نحیف
\\
کرد مغرب را ادا، وآنگه عشاء
&&
دل پر از وسواس، در فکر عشاء
\\
بس که بود از بهر قوتش اضطراب
&&
نه عبادت کرد عابد، شب، نه خواب
\\
صبح چون شد، زان مقام دلپذیر
&&
بهر قوتی آمد آن عابد به زیر
\\
بود یک قریه، به قرب آن جبل
&&
اهل آن قریه، همه گبر و دغل
\\
عابد آمد بر در گبری ستاد
&&
گبر او را یک دو نان جو بداد
\\
بستد آن نان را و شکر او بگفت
&&
وز وصول طعمه‌اش، خاطر شکفت
\\
کرد آهنگ مقام خود دلیر
&&
تا کند افطار زان خبز شعیر
\\
در سرای گبر بد گرگین سگی
&&
مانده از جوع، استخوانی و رگی
\\
پیش او، گر خط پرگاری کشی
&&
شکل نان بیند، بمیرد از خوشی
\\
بر زبان گر بگذرد لفظ خبر
&&
خبز پندار، رود هوشش ز سر
\\
کلب، در دنبال عابد بو گرفت
&&
آمدش دنبال و رخت او گرفت
\\
زان دو نان، عابد یکی پیشش فکند
&&
پس روان شد، تا نیابد زو گزند
\\
سگ بخورد آن نان، وز پی آمدش
&&
تا مگر، بار دگر آزاردش
\\
عابد آن نان دگر، دادش روان
&&
تا که از آزار او یابد امان
\\
کلب خورد آن نان و از دنبال مرد
&&
شد روان و روی خود واپس نکرد
\\
همچو سایه، در پی او می‌دوید
&&
عف عفی می‌کرد و رختش می‌درید
\\
گفت عابد چون بدید آن ماجرا:
&&
من سگی چون تو ندیدم، بی‌حیا
\\
صاحبت، غیر دو نان جو نداد
&&
وان دونان، خود بستدی، ای کج نهاد
\\
دیگرم، از پی دویدن بهر چیست؟
&&
وین همه، رختم دریدن بهر چیست؟
\\
سگ، به نطق آمد که: ای صاحب کمال
&&
بی‌حیا، من نیستم، چشمت بمال
\\
هست، از وقتی که بودم من صغیر
&&
مسکنم، ویرانهٔ این گبر پیر
\\
گوسفندش را شبانی می‌کنم
&&
خانه‌اش را پاسبانی می‌کنم
\\
گاه گاهی، نیم نانم می‌دهد
&&
گاه، مشتی استخوانم می‌دهد
\\
گاه، غافل گردد از اطعام من
&&
وز تغافل، تلخ گردد کام من
\\
بگذرد بسیار، بر من صبح و شام
&&
لا اری خبزا ولا القی الطعام
\\
هفته هفته، بگذرد کاین ناتوان
&&
نی ز نان یابد نشان، نی ز استخوان
\\
گاه هم باشد، که پیر پر محن
&&
نان نیابد بهر خود، چه جای من
\\
چون که بر درگاه او پرورده‌ام
&&
رو به درگاه دگر، ناورده‌ام
\\
هست کارم، بر در این پیر گبر
&&
گاه شکر نعمت او، گاه صبر
\\
تا قمار عشق با او باختم
&&
جز در او، من دری نشناختم
\\
گه به چوبم می‌زند، گه سنگها
&&
از در او، من نمی‌گردم جدا
\\
چون که نامد یک شبی نانت به دست
&&
در بنای صبر تو آمد شکست
\\
از در رزاق رو بر تافتی
&&
بر در گبری روان بشتافتی
\\
بهر نانی، دوست را بگذاشتی
&&
کرده‌ای با دشمن او آشتی
\\
خود بده انصاف، ای مرد گزین!
&&
بی‌حیاتر کیست؟ من یا تو؟ ببین
\\
مرد عابد، زین سخن، مدهوش شد
&&
دست را بر سر زد و از هوش شد
\\
ای سگ نفس بهائی، یاد گیر!
&&
این قناعت، از سگ آن گبر پیر
\\
\end{longtable}
\end{center}
