\begin{center}
\section*{غزل شماره ۲۰: مقصود و مراد کون دیدیم}
\label{sec:020}
\addcontentsline{toc}{section}{\nameref{sec:020}}
\begin{longtable}{l p{0.5cm} r}
مقصود و مراد کون دیدیم
&&
میدان هوس، به پی دویدیم
\\
هر پایه کزان بلندتر بود
&&
از بخشش حق، بدان رسیدیم
\\
چون بوقلمون، به صد طریقت
&&
بر اوج هوای دل، تپیدیم
\\
رخ بر رخ دلبران نهادیم
&&
لحن خوش مطربان شنیدیم
\\
در باغ جمال ماهرویان
&&
ریحان و گل و بنفشه چیدیم
\\
چون ملک بقا نشد میسر
&&
زان جمله، طمع از آن بریدیم
\\
وز دانهٔ شغل باز جستیم
&&
وز دام عمل، برون جهیدیم
\\
رفتیم به کعبهٔ مبارک
&&
در حضرت مصطفی رسیدیم
\\
جستیم هزار گونه تدبیر
&&
تا تیغ اجل، سپر ندیدیم
\\
کردیم به جان و دل تلافی
&&
چون دعوت «ارجعی» شنیدیم
\\
بیهوده صداع خود ندادیم
&&
تسلیم شدیم و وارهیدیم
\\
باشد که چه بعد ما عزیزی
&&
گوید چه به مشهدش رسیدیم:
\\
ایام وفا نکرد با کس
&&
در گنبد او نوشته دیدیم
\\
\end{longtable}
\end{center}
