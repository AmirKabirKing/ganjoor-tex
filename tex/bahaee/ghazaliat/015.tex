\begin{center}
\section*{غزل شماره ۱۵: تا سرو قباپوش تو را دیده‌ام امروز}
\label{sec:015}
\addcontentsline{toc}{section}{\nameref{sec:015}}
\begin{longtable}{l p{0.5cm} r}
تا سرو قباپوش تو را دیده‌ام امروز
&&
در پیرهن از ذوق نگنجیده‌ام امروز
\\
من دانم و دل، غیر چه داند که در این بزم
&&
از طرز نگاه تو چه فهمیده‌ام امروز
\\
تا باد صبا پیچ سر زلف تو وا کرد
&&
بر خود، چو سر زلف تو پیچیده‌ام امروز
\\
هشیاریم افتاد به فردای قیامت
&&
زان باده که از دست تو نوشیده‌ام امروز
\\
صد خنده زند بر حلل قیصر و دارا
&&
این ژندهٔ پر بخیه که پوشیده‌ام امروز
\\
افسوس که برهم زده خواهد شد از آن روی
&&
شیخانه بساطی که فرو چیده‌ام امروز
\\
بر باد دهد توبهٔ صد همچو بهائی
&&
آن طرهٔ طرار که من دیده‌ام امروز
\\
\end{longtable}
\end{center}
