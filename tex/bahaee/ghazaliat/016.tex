\begin{center}
\section*{غزل شماره ۱۶: روی تو گل تازه و خط سبزهٔ نوخیز}
\label{sec:016}
\addcontentsline{toc}{section}{\nameref{sec:016}}
\begin{longtable}{l p{0.5cm} r}
روی تو گل تازه و خط سبزهٔ نوخیز
&&
نشکفته گلی همچو تو در گلشن تبریز
\\
شد هوش دلم غارت آن غمزهٔ خونریز
&&
این بود مرا فایده از دیدن تبریز
\\
ای دل! تو در این ورطه مزن لاف صبوری
&&
وای عقل! تو هم بر سر این واقعه مگریز
\\
فرخنده شبی بود که آن خسرو خوبان
&&
افسوس کنان، لب به تبسم، شکر آمیز
\\
از راه وفا، بر سر بالین من آمد
&&
وز روی کرم گفت که: ای دلشده، برخیز
\\
از دیدهٔ خونبار، نثار قدم او
&&
کردم گهر اشک، من مفلس بی‌چیز
\\
چون رفت دل گمشده‌ام گفت: بهائی!
&&
خوش باش که من رفتم و جان گفت که : من نیز
\\
\end{longtable}
\end{center}
