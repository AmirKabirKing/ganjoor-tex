\begin{center}
\section*{غزل شماره ۲۲: تازه گردید از نسیم صبحگاهی، جان من}
\label{sec:022}
\addcontentsline{toc}{section}{\nameref{sec:022}}
\begin{longtable}{l p{0.5cm} r}
تازه گردید از نسیم صبحگاهی، جان من
&&
شب، مگر بودش گذر بر منزل جانان من
\\
بس که شد گل گل تنم از داغهای آتشین
&&
می‌کند کار سمندر، بلبل بستان من
\\
طفل ابجد خوان عشقم، با وجود آنکه هست
&&
صد چو فرهاد و چو مجنون، طفل ابجد خوان من
\\
گفتمش: از کاو کاو سینه‌ام، مقصود چیست؟
&&
گفت: می‌ترسم که بگذارد در آن پیکان من
\\
بس که بردم آبروی خود به سالوسی و زرق
&&
ننگ می‌دارند اهل کفر، از ایمان من
\\
با خیالت دوش، بزمی داشتم، راحت فزا
&&
از برای مصلحت بود اینهمه افغان من
\\
رفتم و پیش سگ کویت، سپردم جان و دل
&&
ای خوش آن روزی که پیشت، جان سپارد جان من
\\
از دل خود، دارم این محنت، نه از ابنای دهر
&&
کاش بودی این دل سرگشته در فرمان من
\\
چون بهائی، صدهزاران درد دارم جانگداز
&&
صدهزاران، درد دیگر هست سرگردان من
\\
\end{longtable}
\end{center}
