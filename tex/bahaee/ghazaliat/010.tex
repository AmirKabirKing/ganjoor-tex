\begin{center}
\section*{غزل شماره ۱۰: عهد جوانی گذشت، در غم بود و نبود}
\label{sec:010}
\addcontentsline{toc}{section}{\nameref{sec:010}}
\begin{longtable}{l p{0.5cm} r}
عهد جوانی گذشت، در غم بود و نبود
&&
نوبت پیری رسید، صد غم دیگر فزود
\\
کارکنان سپهر، بر سر دعوی شدند
&&
آنچه بدادند دیر، باز گرفتند زود
\\
حاصل ما از جهان نیست به جز درد و غم
&&
هیچ ندانم چراست این همه رشک حسود
\\
نیست عجب گر شدیم شهره به زرق و ریا
&&
پردهٔ تزویر ما، سد سکندر نبود
\\
نام جنون را به خود داد بهائی قرار
&&
نیست به جز راه عشق، زیر سپهر کبود
\\
\end{longtable}
\end{center}
