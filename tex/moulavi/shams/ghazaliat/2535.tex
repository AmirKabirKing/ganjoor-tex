\begin{center}
\section*{غزل شماره ۲۵۳۵: هر آن بیمار مسکین را که از حد رفت بیماری}
\label{sec:2535}
\addcontentsline{toc}{section}{\nameref{sec:2535}}
\begin{longtable}{l p{0.5cm} r}
هر آن بیمار مسکین را که از حد رفت بیماری
&&
نماند مر ورا ناله نباشد مر ورا زاری
\\
نباشد خامشی او را از آن کان درد ساکن شد
&&
چو طاقت طاق شد او را خموش است او ز ناچاری
\\
زمان رقت و رحمت بنالید از برای او
&&
شما یاران دلدارید گرییدش ز دلداری
\\
ازیرا ناله یاران بود تسکین بیماران
&&
نگنجد در چنین حالت به جز ناله شما یاری
\\
بود کاین ناله‌ها درهم شود آن درد را مرهم
&&
درآرد آن پری رو را ز رحمت در کم آزاری
\\
به ناگاهان فرود آید بگوید هی قنق گلدم
&&
شود خرگاه مسکینان طربگاه شکرباری
\\
خمار هجر برخیزد امیر بزم بنشیند
&&
قدح گردان کند در حین به قانون‌های خماری
\\
همه اجزای عشاقان شود رقصان سوی کیوان
&&
هوا را زیر پا آرد شکافد کره ناری
\\
به سوی آسمان جان خرامان گشته آن مستان
&&
همه ره جوی از باده مثال دجله‌ها جاری
\\
زهی کوچ و زهی رحلت زهی بخت و زهی دولت
&&
من این را بی‌خبر گفتم حریفا تو خبر داری
\\
زره کاسد شود آن جا سلح بی‌قیمتی گردد
&&
سیاست‌های شاه ما چو درهم سوخت غداری
\\
چو خوف از خوف او گم شد خجل شد امن از امنش
&&
به پیش شمع علم او فضیحت گشته طراری
\\
فضیحت شد کژی لیکن به زودی دامن لطفش
&&
بر او هم رحمتی کرد و بپوشیدش به ستاری
\\
که تا الطاف مخدومی شمس الحق تبریزی
&&
ببیند دیده دشمن نماند کفر و انکاری
\\
همه اضداد از لطفش بپوشد خلعتی دیگر
&&
ز خجلت جمله محو آمد چو گیرد لطف بسیاری
\\
دگربار از میان محو عجب نومستیی یابند
&&
برویند از میان نفی چون کز خار گلزاری
\\
پس آنگه دیده بگشایند جمال عشق را بینند
&&
همه حکم و همه علم و همه حلم است و غفاری
\\
\end{longtable}
\end{center}
