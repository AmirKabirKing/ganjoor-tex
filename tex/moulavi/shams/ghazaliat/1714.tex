\begin{center}
\section*{غزل شماره ۱۷۱۴: چند روی بی‌خبر آخر بنگر به بام}
\label{sec:1714}
\addcontentsline{toc}{section}{\nameref{sec:1714}}
\begin{longtable}{l p{0.5cm} r}
چند روی بی‌خبر آخر بنگر به بام
&&
بام چه باشد بگو بر فلک سبزفام
\\
تا قمری همچو جان جلوه شود ناگهان
&&
صد مه و صد آفتاب چهره او را غلام
\\
از هوس عشق او چرخ زند نه فلک
&&
وز می او جان و دل نوش کند جام جام
\\
چون به تجلی بتافت جانب جان‌ها شتافت
&&
باده جان شد مباح خوردن و خفتن حرام
\\
گفت جهان سلیم چیست خبر ای نسیم
&&
گفت ندارم ز بیم جز نفسی والسلام
\\
\end{longtable}
\end{center}
