\begin{center}
\section*{غزل شماره ۳۰۸۵: به حق آنک تو جان و جهان جانداری}
\label{sec:3085}
\addcontentsline{toc}{section}{\nameref{sec:3085}}
\begin{longtable}{l p{0.5cm} r}
به حق آنک تو جان و جهان جانداری
&&
مرا چنانک بپرورده‌ای چنان داری
\\
به حق حلقه عزت که دام حلق منست
&&
مرا به حلقه مستان و سرخوشان داری
\\
به حق جان عظیمی که جان نتیجه اوست
&&
چنان کنی که مرا در میان جان داری
\\
به حق گنج نهانی که در خرابه ماست
&&
مرا ز چشم همه مردمان نهان داری
\\
به حق باغی کز چشم خلق پنهانست
&&
رخ نژند مرا همچو ارغوان داری
\\
به حق بام بلندی که صومعه ملکست
&&
مرا به بام برآری چو نردبان داری
\\
دری که هیچ نبستی به روی ما دربند
&&
اگر ز راحت و از سود ما زیان داری
\\
چو از فغان تو نزدیکتر به تو یارست
&&
چه حکمتست که نزدیک را فغان داری
\\
در آفرینش عالم چو حکمت اظهارست
&&
تو نیز ظاهر می‌کن اگر بیان داری
\\
به برج آتش فرمود دیگ پالان کن
&&
برای پختن خامی چو دیگدان داری
\\
به برج آبی فرمود خاک را تر کن
&&
به شکر آنک درون چشمه روان داری
\\
به سعد اکبر فرمود هین هنر بنما
&&
که از گشایش بی‌چون ما نشان داری
\\
به نحس اکبر فرمود رو حسودی کن
&&
دگر بگو چه کنی چون هنر همان داری
\\
چو کرد ظاهر هجده هزار عالم را
&&
برای حکمت اظهار اگر عیان داری
\\
هر آنک او هنری دارد او همی‌کوشد
&&
که شهره گردد در دانش و عنان داری
\\
هنروری که بپوشد هنر غرض آنست
&&
که شهره گردد در ستر و در نهان داری
\\
وگر بستر بپوشد هنر غرض آنست
&&
که شهره گردد در دانش و صوان داری
\\
نه انبیا که رسیدند بهر اظهارند
&&
که ای نتیجه خاک از درونه کان داری
\\
که من به تن بشرمثلکم بدم و اکنون
&&
مقام گنجم و تو حبه‌ای از آن داری
\\
منم دل تو دل از خود مجوی از من جوی
&&
مرید پیر شو ار دولت جوان داری
\\
اگر ز خویش بدانی مرا ندانی خویش
&&
درون خویش بسی رنج و امتحان داری
\\
بیا تو جزو منی جزو را ز کل مسکل
&&
بچفس بر کل زیرا کل کلان داری
\\
گمان که جزو یقینست شد یقین ز یقین
&&
وگر جدا هلیش از یقین گمان داری
\\
دلیل سود ندارد تو را دلیل منم
&&
چو بی‌منی نرهی گر دلیل لان داری
\\
اگر دعا نکنم لطف او همی‌گوید
&&
که سرد و بسته چرایی بگو زبان داری
\\
بگفتمش که چو جانم روان شود از تن
&&
شعار شعر مرا با روان روان داری
\\
جواب داد مرا لطف او که ای طالب
&&
خود این شدست ز اول چه دل طپان داری
\\
دلا بگو تو تمام سخن دهان بستیم
&&
سخن تو گوی که گفتار جاودان داری
\\
بیار معنی اسما تو شمس تبریزی
&&
در آسمان چو نه‌ای تا چه آسمان داری
\\
\end{longtable}
\end{center}
