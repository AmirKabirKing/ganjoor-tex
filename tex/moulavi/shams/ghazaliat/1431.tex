\begin{center}
\section*{غزل شماره ۱۴۳۱: مرا چون کم فرستی غم حزین و تنگ دل باشم}
\label{sec:1431}
\addcontentsline{toc}{section}{\nameref{sec:1431}}
\begin{longtable}{l p{0.5cm} r}
مرا چون کم فرستی غم حزین و تنگ دل باشم
&&
چو غم بر من فروریزی ز لطف غم خجل باشم
\\
غمان تو مرا نگذاشت تا غمگین شوم یک دم
&&
هوای تو مرا نگذاشت تا من آب و گل باشم
\\
همه اجزای عالم را غم تو زنده می دارد
&&
منم کز تو غمی خواهم که در وی مستقل باشم
\\
عجب دردی برانگیزی که دردم را دوا گردد
&&
عجب گردی برانگیزی که از وی مکتحل باشم
\\
فدایی را کفیلی کو که ارزد جان فدا کردن
&&
کسایی را کسایی کو که آن را مشتمل باشم
\\
مرا رنج تو نگذارد که رنجوری به من آید
&&
مرا گنج تو نگذارد که درویش و مقل باشم
\\
صباح تو مرا نگذاشت تا شمعی برافروزم
&&
عیان تو مرا نگذاشت تا من مستدل باشم
\\
خیالی کان به پیش آید خیالت را بپوشاند
&&
اگر خونش بریزم من ز خون او بحل باشم
\\
بسوزانم ز عشق تو خیال هر دو عالم را
&&
بسوزند این دو پروانه چو من شمع چگل باشم
\\
خمش کن نقل کمتر کن ز حال خود به قال خود
&&
چنان نقلی که من دارم چرا من منتقل باشم
\\
\end{longtable}
\end{center}
