\begin{center}
\section*{غزل شماره ۵۸۹: شکایت‌ها همی‌کردی که بهمن برگ ریز آمد}
\label{sec:0589}
\addcontentsline{toc}{section}{\nameref{sec:0589}}
\begin{longtable}{l p{0.5cm} r}
شکایت‌ها همی‌کردی که بهمن برگ ریز آمد
&&
کنون برخیز و گلشن بین که بهمن بر گریز آمد
\\
ز رعد آسمان بشنو تو آواز دهل یعنی
&&
عروسی دارد این عالم که بستان پرجهیز آمد
\\
بیا و بزم سلطان بین ز جرعه خاک خندان بین
&&
که یاغی رفت و از نصرت نسیم مشک بیز آمد
\\
بیا ای پاک مغز من ببو گلزار نغز من
&&
به رغم هر خری کاهل که مشک او کمیز آمد
\\
زمین بشکافت و بیرون شد از آن رو خنجرش خواندم
&&
به یک دم از عدم لشکر به اقلیم حجیز آمد
\\
سپاه گلشن و ریحان بحمدالله مظفر شد
&&
که تیغ و خنجر سوسن در این پیکار تیز آمد
\\
چو حلواهای بی‌آتش رسید از دیگ چوبین خوش
&&
سر هر شاخ پرحلوا به سان کفچلیز آمد
\\
به گوش غنچه نیلوفر همی‌گوید که یا عبهر
&&
باستیز عدو می خور که هنگام ستیز آمد
\\
مفاعیلن مفاعیلن مفاعیلن مفاعیلن
&&
مکن با او تو همراهی که او بس سست و حیز آمد
\\
خمش باش و بجو عصمت سفر کن جانب حضرت
&&
که نبود خواب را لذت چو بانگ خیز خیز آمد
\\
\end{longtable}
\end{center}
