\begin{center}
\section*{غزل شماره ۱۰۸۹: هین که آمد به سر کوی تو مجنون دگر}
\label{sec:1089}
\addcontentsline{toc}{section}{\nameref{sec:1089}}
\begin{longtable}{l p{0.5cm} r}
هین که آمد به سر کوی تو مجنون دگر
&&
هین که آمد به تماشای تو دل خون دگر
\\
عاشق روی تو را گنبد گردون نکشد
&&
مگرش جای دهی بر سر گردون دگر
\\
عاشق تو نخورد حیله و افسون کسی
&&
تو بخوان و تو بدم بر دلش افسون دگر
\\
عشق روی تو به شش سوی جهان دام دلست
&&
که ندیدند چنان رخ رخ گلگون دگر
\\
رحمتی کن تو بر آن مرغ که در دام افتاد
&&
که ندارد چو تو شاهنشه بی‌چون دگر
\\
کو در این خانه یکی سوخته مفتونی
&&
که به شب‌ها شنود ناله مفتون دگر
\\
از پس نیشکرت اشک چو اطلس بارم
&&
چاره‌ام نیست جز این اطلس و اکسون دگر
\\
\end{longtable}
\end{center}
