\begin{center}
\section*{غزل شماره ۱۸۹۵: بفریفتیم دوش و پرندوش به دستان}
\label{sec:1895}
\addcontentsline{toc}{section}{\nameref{sec:1895}}
\begin{longtable}{l p{0.5cm} r}
بفریفتیم دوش و پرندوش به دستان
&&
خوردم دغل گرم تو چون عشوه پرستان
\\
دی عهد نکردی بروم بازبیایم
&&
سوگند نخوردی که بجویم دل مستان
\\
گفتی که به بستان بر من چاشت بیایید
&&
رفتی تو سحرگاه و ببستی در بستان
\\
ای عشوه تو گرمتر از باد تموزی
&&
وی چهره تو خوبتر از روی گلستان
\\
دانی که دغل از چو تو یاری به چه ماند
&&
در عین تموزی بجهد برق زمستان
\\
گر زانک تو را عشوه دهد کس گله کم کن
&&
صد شعبده کردی تو یکی شعبده بستان
\\
بر وعده مکن صبر که گر صبر نبودی
&&
هرگز نرسیدی مدد از نیست بهستان
\\
ور نه بکنم غمز و بگویم که سبب چیست
&&
زان سان که تو اقرار کنی که سبب است آن
\\
\end{longtable}
\end{center}
