\begin{center}
\section*{غزل شماره ۳۰۷۳: بیا بیا که پشیمان شوی از این دوری}
\label{sec:3073}
\addcontentsline{toc}{section}{\nameref{sec:3073}}
\begin{longtable}{l p{0.5cm} r}
بیا بیا که پشیمان شوی از این دوری
&&
بیا به دعوت شیرین ما چه می‌شوری
\\
حیات موج زنان گشته اندر این مجلس
&&
خدای ناصر و هر سو شراب منصوری
\\
به دست طره خوبان به جای دسته گل
&&
به زیر پای بنفشه به جای محفوری
\\
هزار جام سعادت بنوش ای نومید
&&
بگیر صد زر و زور ای غریب زرزوری
\\
هزار گونه زلیخا و یوسفند این جا
&&
شراب روح فزای و سماع طنبوری
\\
جواهر از کف دریای لامکان ز گزاف
&&
به پیش مؤمن و کافر نهاده کافوری
\\
میان بحر عسل بانگ می‌زند هر جان
&&
صلا که بازرهیدم ز شهد زنبوری
\\
فتاده‌اند به هم عاشقان و معشوقان
&&
خراب و مست رهیده ز ناز مستوری
\\
قیامت‌ست همه راز و ماجراها فاش
&&
که مرده زنده کند ناله‌های ناقوری
\\
برآر باز سر ای استخوان پوسیده
&&
اگر چه سخره ماری و طعمه موری
\\
ز مور و مار خریدت امیر کن فیکون
&&
بپوش خلعت میری جزای مأموری
\\
تو راست کان گهر غصه دکان بگذار
&&
ز نور پاک خوری به که نان تنوری
\\
شکوفه‌های شراب خدا شکفت بهل
&&
شکوفه‌ها و خمار شراب انگوری
\\
جمال حور به از بردگان بلغاری
&&
شراب روح به از آش‌های بلغوری
\\
خیال یار به حمام اشک من آمد
&&
نشست مردمک دیده‌ام به ناطوری
\\
دو چشم ترک خطا را چه ننگ از تنگی
&&
چه عار دارد سیاح جان از این عوری
\\
درخت شو هله ای دانه‌ای که پوسیدی
&&
تویی خلیفه و دستور ما به دستوری
\\
کی دیده‌ست چنین روز با چنان روزی
&&
که واخرد همه را از شبی و شب کوری
\\
کرم گشاد چو موسی کنون ید بیضا
&&
جهان شده‌ست چو سینا و سینه نوری
\\
دلا مقیم شو اکنون به مجلس جان‌ها
&&
که کدخدای مقیمان بیت معموری
\\
مباش بسته مستی خراب باش خراب
&&
یقین بدانک خرابیست اصل معموری
\\
خراب و مست خدایی در این چمن امروز
&&
هزار شیشه اگر بشکنی تو معذوری
\\
به دست ساقی تو خاک می‌شود زر سرخ
&&
چو خاک پای ویی خسروی و فغفوری
\\
صلای صحت جان هر کجا که رنجوریست
&&
تو مرده زنده شدن بین چه جای رنجوری
\\
غلام شعر بدانم که شعر گفته توست
&&
که جان جان سرافیل و نفخه صوری
\\
سخن چو تیر و زبان چو کمان خوارزمی است
&&
که دیر و دور دهد دست وای از این دوری
\\
ز حرف و صوت بباید شدن به منطق جان
&&
اگر غفار نباشد بس است مغفوری
\\
کز آن طرف شنوااند بی‌زبان دل‌ها
&&
نه رومیست و نه ترکی و نی نشابوری
\\
بیا که همره موسی شویم تا که طور
&&
که کلم الله آمد مخاطبه طوری
\\
که دامنم بگرفته‌ست و می‌کشد عشقی
&&
چنانک گرسنه گیرد کنار کندوری
\\
ز دست عشق کی جسته‌ست تا جهد دل من
&&
به قبض عشق بود قبضه قلاجوری
\\
\end{longtable}
\end{center}
