\begin{center}
\section*{غزل شماره ۱۳۵۴: تو را سعادت بادا در آن جمال و جلال}
\label{sec:1354}
\addcontentsline{toc}{section}{\nameref{sec:1354}}
\begin{longtable}{l p{0.5cm} r}
تو را سعادت بادا در آن جمال و جلال
&&
هزار عاشق اگر مرد خون مات حلال
\\
به یک دمم بفروزی به یک دمم بکشی
&&
چو آتشیم به پیش تو ای لطیف خصال
\\
دل آب و قالب کوزه‌ست و خوف بر کوزه
&&
چو آب رفت به اصلش شکسته گیر سفال
\\
تو را چگونه فریبم چه در جوال کنم
&&
که اصل مکر تویی و چراغ هر محتال
\\
تو در جوال نگنجی و دام را بدری
&&
که دیده است که شیری رود درون جوال
\\
نه گربه‌ای که روی در جوال و بسته شوی
&&
که شیر پیش تو بر ریگ می‌زند دنبال
\\
هزار صورت زیبا بروید از دل و جان
&&
چو ابر عشق تو بارید در بی‌امثال
\\
مثال آنک ببارد ز آسمان باران
&&
چو قبه قبه شود جوی و حوض و آب زلال
\\
چه قبه قبه کز آن قبه‌ها برون آیند
&&
گل و بنفشه و نسرین و سنبل چو هلال
\\
بگویمت که از این‌ها کیان برون آیند
&&
شنودم از تکشان بانگ ژغرغ خلخال
\\
ردای احمد مرسل بگیر ای عاشق
&&
صلای عشق شنو هر دم از روان بلال
\\
بهل مرا که بگوییم عجایبت ای عشق
&&
دری گشایم در غیب خلق را ز مقال
\\
همه چو کوس و چو طبلیم دل تهی پیشت
&&
برآوریم فغان چون زنی تو زخم دوال
\\
چگونه طبل نپرد بپر کرمنا
&&
که باشدش چو تو سلطان زننده و طبال
\\
خود آفتاب جهانی تو شمس تبریزی
&&
ولی مدام نه آن شمس کو رسد به زوال
\\
\end{longtable}
\end{center}
