\begin{center}
\section*{غزل شماره ۲۵۳۲: کی افسون خواند در گوشت که ابرو پرگره داری}
\label{sec:2532}
\addcontentsline{toc}{section}{\nameref{sec:2532}}
\begin{longtable}{l p{0.5cm} r}
کی افسون خواند در گوشت که ابرو پرگره داری
&&
نگفتم با کسی منشین که باشد از طرب عاری
\\
یکی پرزهر افسونی فروخواند به گوش تو
&&
ز صحن سینه پرغم دهد پیغام بیماری
\\
چو دیدی آن ترش رو را مخلل کرده ابرو را
&&
از او بگریز و بشناسش چرا موقوف گفتاری
\\
چه حاجت آب دریا را چشش چون رنگ او دیدی
&&
که پرزهرت کند آبش اگر چه نوش منقاری
\\
لطیفان و ظریفانی که بودستند در عالم
&&
رمیده و بدگمان بودند همچون کبک کهساری
\\
گر استفراغ می‌خواهی از آن طزغوی گندیده
&&
مفرح بدهمت لیکن مکن دیگر وحل خواری
\\
الا یا صاحب الدار ادر کأسا من النار
&&
فدفینی و صفینی و صفو عینک الجاری
\\
فطفینا و عزینا فان عدنا فجازینا
&&
فانا مسنا ضر فلا ترضی باضراری
\\
ادر کأسا عهدناه فانا ما جحدناه
&&
فعندی منه آثار و انی مدرک ثاری
\\
ادر کأسا باجفانی فدا روحی و ریحانی
&&
و انت المحشر الثانی فاحیینا بمدرار
\\
فاوقد لی مصابیحی و ناولنی مفاتیحی
&&
و غیرنی و سیرنی بجود کفک الساری
\\
چو نامت پارسی گویم کند تازی مرا لابه
&&
چو تازی وصف تو گویم برآرد پارسی زاری
\\
بگه امروز زنجیری دگر در گردنم کردی
&&
زهی طوق و زهی منصب که هست آن سلسله داری
\\
چو زنجیری نهی بر سگ شود شاه همه شیران
&&
چو زنگی را دهی رنگی شود رومی و روم آری
\\
الا یا صاحب الکاس و یا من قلبه قاسی
&&
اتبلینی بافلاسی و تعلینی باکثاری
\\
لسان العرب و الترک هما فی کاسک المر
&&
فناول قهوه تغنی من اعساری و ایساری
\\
مگر شاه عرب را من بدیدم دوش خواب اندر
&&
چه جای خواب می‌بینم جمالش را به بیداری
\\
\end{longtable}
\end{center}
