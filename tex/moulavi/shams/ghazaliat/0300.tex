\begin{center}
\section*{غزل شماره ۳۰۰: کو همه لطف که در روی تو دیدم همه شب}
\label{sec:0300}
\addcontentsline{toc}{section}{\nameref{sec:0300}}
\begin{longtable}{l p{0.5cm} r}
کو همه لطف که در روی تو دیدم همه شب
&&
وان حدیث چو شکر کز تو شنیدم همه شب
\\
گر چه از شمع تو می‌سوخت چو پروانه دلم
&&
گرد شمع رخ خوب تو پریدم همه شب
\\
شب به پیش رخ چون ماه تو چادر می‌بست
&&
من چو مه چادر شب می‌بدریدم همه شب
\\
جان ز ذوق تو چو گربه لب خود می‌لیسد
&&
من چو طفلان سر انگشت گزیدم همه شب
\\
سینه چون خانه زنبور پر از مشغله بود
&&
کز تو ای کان عسل شهد کشیدم همه شب
\\
دام شب آمد جان‌های خلایق بربود
&&
چون دل مرغ در آن دام طپیدم همه شب
\\
آنک جان‌ها چو کبوتر همه در حکم ویند
&&
اندر آن دام مر او را طلبیدم همه شب
\\
\end{longtable}
\end{center}
