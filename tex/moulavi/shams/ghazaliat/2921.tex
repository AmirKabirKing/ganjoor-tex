\begin{center}
\section*{غزل شماره ۲۹۲۱: ساقی این جا هست ای مولا بلی}
\label{sec:2921}
\addcontentsline{toc}{section}{\nameref{sec:2921}}
\begin{longtable}{l p{0.5cm} r}
ساقی این جا هست ای مولا بلی
&&
ره دهد ما را بر آن بالا بلی
\\
پیش آن لب‌های آری گوی او
&&
بنده گردد شکر و حلوا بلی
\\
هست چشمش قلزم مستی نعم
&&
هست جعدش مایه سودا بلی
\\
این همه بگذشت آن سرو سهی
&&
خوش برآید همچو گل با ما بلی
\\
چون بخسبم زیر سایه نخل او
&&
من شوم شیرینتر از خرما بلی
\\
هم عسس هم دزد ای جان هر شبی
&&
سیم دزدد زان قمرسیما بلی
\\
چون برآید آفتاب روی او
&&
دزد گردد عاجز و رسوا بلی
\\
ناشتاب آن کس که او حلوا خورد
&&
در دماغ او کند صفرا بلی
\\
بس کن آن کس کو سری پنهان کند
&&
روید از سر گلشن اخفی بلی
\\
\end{longtable}
\end{center}
