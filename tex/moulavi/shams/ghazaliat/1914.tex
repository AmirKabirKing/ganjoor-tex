\begin{center}
\section*{غزل شماره ۱۹۱۴: اگر خواهی مرا می در هوا کن}
\label{sec:1914}
\addcontentsline{toc}{section}{\nameref{sec:1914}}
\begin{longtable}{l p{0.5cm} r}
اگر خواهی مرا می در هوا کن
&&
وگر سیری ز من رفتم رها کن
\\
نیم قانع به یک جام و به صد جام
&&
دوساله پیش تو دارم قضا کن
\\
بده می گر ننوشم بر سرم ریز
&&
وگر نیکو نگفتم ماجرا کن
\\
من از قندم مرا گویی ترش شو
&&
تو ماشی را بگیر و لوبیا کن
\\
سر خم را به کهگل هین مبندا
&&
دل خم را برآور دلگشا کن
\\
مرا چون نی درآوردی به ناله
&&
چو چنگم خوش بساز و بانوا کن
\\
اگر چه می زنی سیلیم چون دف
&&
که آوازی خوشی داری صدا کن
\\
چو دف تسلیم کردم روی خود را
&&
بزن سیلی و رویم را قفا کن
\\
همی‌زاید ز دف و کف یک آواز
&&
اگر یک نیست از همشان جدا کن
\\
حریف آن لبی ای نی شب و روز
&&
یکی بوسه پی ما اقتضا کن
\\
تو بوسه باره‌ای و جمله خواری
&&
نگیری پند اگر گویم سخا کن
\\
شدی ای نی شکر ز افسون آن لب
&&
ز لب ای نیشکر رو شکرها کن
\\
نه شکر است این نوای خوش که داری
&&
نوای شکرین داری ادا کن
\\
خموش از ذکر نی می باش یکتا
&&
که نی گوید که یکتا را دو تا کن
\\
\end{longtable}
\end{center}
