\begin{center}
\section*{غزل شماره ۱۶۸۷: گر جان منکرانت شد خصم جان مستم}
\label{sec:1687}
\addcontentsline{toc}{section}{\nameref{sec:1687}}
\begin{longtable}{l p{0.5cm} r}
گر جان منکرانت شد خصم جان مستم
&&
اندر جواب ایشان خوبی تو بسستم
\\
در دفع آن خیالش وز بهر گوشمالش
&&
بنمایمش جمالت از دور من برستم
\\
گوید که نیست جوهر وز منش نیست باور
&&
زان نیست ای برادر هستم چنانک هستم
\\
دوش از رخ نگاری دل مست گشت باری
&&
تا پیش شهریاری من ساغری شکستم
\\
من مست روی ماهم من شاد از آن گناهم
&&
من جرم دار شاهم نک بشکنید دستم
\\
بس رندم و قلاشم در دین عشق فاشم
&&
من ملک را چه باشم تا تحفه‌ای فرستم
\\
دل دزد و دزدزاده بر مخزن ایستاده
&&
شه مخزنش گشاده چون دست دزد بستم
\\
ای بی‌خبر ز شاهی گویی که بر چه راهی
&&
من می روم چو ماهی آن سو که برد شستم
\\
شمس الحق است رازم تبریز شد نیازم
&&
او قبله نمازم او نور آب دستم
\\
\end{longtable}
\end{center}
