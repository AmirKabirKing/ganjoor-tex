\begin{center}
\section*{غزل شماره ۲۳۰۵: کی باشد من با تو باده به گرو خورده}
\label{sec:2305}
\addcontentsline{toc}{section}{\nameref{sec:2305}}
\begin{longtable}{l p{0.5cm} r}
کی باشد من با تو باده به گرو خورده
&&
تو برده و من مانده من خرقه گرو کرده
\\
در می شده من غرقه چون ساغر و چون کوزه
&&
با یار درافتاده بی‌حاجب و بی‌پرده
\\
صد نوش تو نوشیده تشریف تو پوشیده
&&
صد جوش بجوشیده این عالم افسرده
\\
از نور تو روشن دل چون ماه ز نور خور
&&
وز بوی گلت خوشدل چون روغن پرورده
\\
تا خود چه فسون گفتی با گل که شد او خندان
&&
تا خود چه جفا گفتی با خارک پژمرده
\\
یک لحظه بخندانی یک لحظه بگریانی
&&
ای نادره صنعت‌ها در صنع درآورده
\\
عاقل ز تو نازارد زان روی که زشت آید
&&
ظلمت ز مه آشفته خاری ز گل آزرده
\\
بس غصه رسول آمد از منعم و می‌گوید
&&
ده مرده شکر خوردی بگذار یکی مرده
\\
پس فکر چو بحر آمد حکمت مثل ماهی
&&
در فکر سخن زنده در گفت سخن مرده
\\
نی فکر چو دام آمد دریا پس این دام است
&&
در دام کجا گنجد جز ماهی بشمرده
\\
پس دل چو بهشتی دان گفتار زبان دوزخ
&&
وین فکر چو اعرافی جای گنه و خرده
\\
\end{longtable}
\end{center}
