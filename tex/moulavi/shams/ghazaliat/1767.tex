\begin{center}
\section*{غزل شماره ۱۷۶۷: تو چه دانی که ما چه مرغانیم}
\label{sec:1767}
\addcontentsline{toc}{section}{\nameref{sec:1767}}
\begin{longtable}{l p{0.5cm} r}
تو چه دانی که ما چه مرغانیم
&&
هر نفس زیر لب چه می خوانیم
\\
چون به دست آورد کسی ما را
&&
ما گهی گنج گاه ویرانیم
\\
چرخ از بهر ماست در گردش
&&
زان سبب همچو چرخ گردانیم
\\
کی بمانیم اندر این خانه
&&
چون در این خانه جمله مهمانیم
\\
گر به صورت گدای این کوییم
&&
به صفت بین که ما چه سلطانیم
\\
چونک فردا شهیم در همه مصر
&&
چه غم امروز اگر به زندانیم
\\
تا در این صورتیم از کس ما
&&
هم نرنجیم و هم نرنجانیم
\\
شمس تبریز چونک شد مهمان
&&
صد هزاران هزار چندانیم
\\
\end{longtable}
\end{center}
