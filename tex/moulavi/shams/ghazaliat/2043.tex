\begin{center}
\section*{غزل شماره ۲۰۴۳: دیدی چه گفت بهمن هیزم بنه چو خرمن}
\label{sec:2043}
\addcontentsline{toc}{section}{\nameref{sec:2043}}
\begin{longtable}{l p{0.5cm} r}
دیدی چه گفت بهمن هیزم بنه چو خرمن
&&
گر دی نکرد سرما سرمای هر دو بر من
\\
سرما چو گشت سرکش هیزم بنه در آتش
&&
هیزم دریغت آید هیزم به است یا تن
\\
نقش فناست هیزم عشق خداست آتش
&&
درسوز نقش‌ها را ای جان پاکدامن
\\
تا نقش را نسوزی جانت فسرده باشد
&&
مانند بت پرستان دور از بهار و مؤمن
\\
در عشق همچو آتش چون نقره باش دلخوش
&&
چون زاده خلیلی آتش تو راست مسکن
\\
آتش به امر یزدان گردد به پیش مردان
&&
لاله و گل و شکوفه ریحان و بید و سوسن
\\
مؤمن فسون بداند بر آتشش بخواند
&&
سوزش در او نماند ماند چو ماه روشن
\\
شاباش ای فسونی کافتد از او سکونی
&&
در آتشی که آهن گردد از او چو سوزن
\\
پروانه زان زند خود بر آتش موقد
&&
کو را همی‌نماید آتش به شکل روزن
\\
تیر و سنان به حمزه چون گلفشان نماید
&&
در گلفشان نپوشد کس خویش را به جوشن
\\
فرعون همچو دوغی در آب غرقه گشته
&&
بر فرق آب موسی بنشسته همچو روغن
\\
اسپان اختیاری حمال شهریاری
&&
پالان کشند و سرگین اسبان کند و کودن
\\
چو لک لک است منطق بر آسیای معنی
&&
طاحون ز آب گردد نه از لکلک مقنن
\\
زان لکلک ای برادر گندم ز دلو بجهد
&&
در آسیا درافتد گردد خوش و مطحن
\\
وز لکلک بیان تو از دلو حرص و غفلت
&&
در آسیا درافتی یعنی رهی مبین
\\
من گرم می‌شوم جان اما ز گفت و گو نی
&&
از شمس دین زرین تبریز همچو معدن
\\
\end{longtable}
\end{center}
