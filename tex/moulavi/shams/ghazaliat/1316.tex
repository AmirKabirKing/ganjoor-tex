\begin{center}
\section*{غزل شماره ۱۳۱۶: رو رو که نه‌ای عاشق ای زلفک و ای خالک}
\label{sec:1316}
\addcontentsline{toc}{section}{\nameref{sec:1316}}
\begin{longtable}{l p{0.5cm} r}
رو رو که نه‌ای عاشق ای زلفک و ای خالک
&&
ای نازک و ای خشمک پابسته به خلخالک
\\
با مرگ کجا پیچد آن زلفک و آن پیچک
&&
بر چرخ کجا پرد آن پرک و آن بالک
\\
ای نازک نازک‌دل دل جو که دلت ماند
&&
روزی که جدا مانی از زرک و از مالک
\\
اشکسته چرا باشی دلتنگ چرا گردی
&&
دل همچو دل میمک قد همچو قد دالک
\\
تو رستم دستانی از زال چه می‌ترسی
&&
یا رب برهان او را از ننگ چنین زالک
\\
من دوش تو را دیدم در خواب و چنان باشد
&&
بر چرخ همی‌گشتی سرمستک و خوش حالک
\\
می‌گشتی و می‌گفتی ای زهره به من بنگر
&&
سرمستم و آزادم ز ادبارک و اقبالک
\\
درویشی وانگه غم از مست نبیذی کم
&&
رو خدمت آن مه کن مردانه یکی سالک
\\
بر هفت فلک بگذر افسون زحل مشنو
&&
بگذار منجم را در اختر و در فالک
\\
من خرقه ز خور دارم چون لعل و گهر دارم
&&
من خرقه کجا پوشم از صوفک و از شالک
\\
با یار عرب گفتم در چشم ترم بنگر
&&
می‌گفت به زیر لب لا تخدعنی والک
\\
می‌گفتم و می‌پختم در سینه دو صد حیلت
&&
می‌گفت مرا خندان کم تکتم احوالک
\\
خامش کن و شه را بین چون باز سپیدی تو
&&
نی بلبل قوالی درمانده در این قالک
\\
\end{longtable}
\end{center}
