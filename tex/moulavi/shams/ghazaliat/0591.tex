\begin{center}
\section*{غزل شماره ۵۹۱: چه بویست این چه بویست این مگر آن یار می‌آید}
\label{sec:0591}
\addcontentsline{toc}{section}{\nameref{sec:0591}}
\begin{longtable}{l p{0.5cm} r}
چه بویست این چه بویست این مگر آن یار می‌آید
&&
مگر آن یار گل رخسار از آن گلزار می‌آید
\\
شبی یا پرده عودی و یا مشک عبرسودی
&&
و یا یوسف بدین زودی از آن بازار می‌آید
\\
چه نورست این چه تابست این چه ماه و آفتابست این
&&
مگر آن یار خلوت جو ز کوه و غار می‌آید
\\
سبوی می چه می‌جویی دهانش را چه می‌بویی
&&
تو پنداری که او چون تو از این خمار می‌آید
\\
چه نقصان آفتابی را اگر تنها رود در ره
&&
چه نقصان حشمت مه را که بی‌دستار می‌آید
\\
چه خورد این دل در آن محفل که همچون مست اندر گل
&&
از آن میخانه چون مستان چه ناهموار می‌آید
\\
مخسب امشب مخسب امشب قوامش گیر و دریابش
&&
که او در حلقه مستان چنین بسیار می‌آید
\\
گلستان می‌شود عالم چو سروش می‌کند سیران
&&
قیامت می‌شود ظاهر چو در اظهار می‌آید
\\
همه چون نقش دیواریم و جنبان می‌شویم آن دم
&&
که نور نقش بند ما بر این دیوار می‌آید
\\
گهی در کوی بیماران چو جالینوس می‌گردد
&&
گهی بر شکل بیماران به حیلت زار می‌آید
\\
خمش کردم خمش کردم که این دیوان شعر من
&&
ز شرم آن پری چهره به استغفار می‌آید
\\
\end{longtable}
\end{center}
