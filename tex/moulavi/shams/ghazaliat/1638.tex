\begin{center}
\section*{غزل شماره ۱۶۳۸: مادرم بخت بده است و پدرم جود و کرم}
\label{sec:1638}
\addcontentsline{toc}{section}{\nameref{sec:1638}}
\begin{longtable}{l p{0.5cm} r}
مادرم بخت بده است و پدرم جود و کرم
&&
فرح ابن الفرح ابن الفرح ابن الفرحم
\\
هین که بکلربک شادی به سعادت برسید
&&
پر شد این شهر و بیابان سپه و طبل و علم
\\
گر به گرگی برسم یوسف مه روی شود
&&
در چهی گر بروم گردد چه باغ ارم
\\
آنک باشد ز بخیلی دل او آهن و سنگ
&&
خاتم وقت شود پیش من از جود و کرم
\\
خاک چون در کف من زر شود و نقره خام
&&
چون مرا راه زند فتنه گر زر و درم
\\
صنمی دارم گر بوی خوشش فاش شود
&&
جان پذیرد ز خوشی گر بود از سنگ صنم
\\
مرد غم در فرحش که جبر الله عزاک
&&
آن چنان تیغ چگونه نزند گردن غم
\\
بستاند به ستم او دل هر کی خواهد
&&
عدل‌ها جمله غلامان چنین ظلم و ستم
\\
آن چه خال است بر آن رخ که اگر جلوه کند
&&
زود بیگانه شود در هوسش خال زعم
\\
گفتم ار بس کنم و قصه فروداشت کنم
&&
تو تمامش کنی و شرح کنی گفت نعم
\\
\end{longtable}
\end{center}
