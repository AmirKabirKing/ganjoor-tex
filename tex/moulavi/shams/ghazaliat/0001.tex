\begin{center}
\section*{غزل شماره ۱: ای رستخیز ناگهان وی رحمت بی‌منتها}
\label{sec:0001}
\addcontentsline{toc}{section}{\nameref{sec:0001}}
\begin{longtable}{l p{0.5cm} r}
ای رستخیز ناگهان وی رحمت بی‌منتها
&&
ای آتشی افروخته در بیشه اندیشه‌ها
\\
امروز خندان آمدی مفتاح زندان آمدی
&&
بر مستمندان آمدی چون بخشش و فضل خدا
\\
خورشید را حاجب تویی اومید را واجب تویی
&&
مطلب تویی طالب تویی هم منتها هم مبتدا
\\
در سینه‌ها برخاسته اندیشه را آراسته
&&
هم خویش حاجت خواسته هم خویشتن کرده روا
\\
ای روح بخش بی‌بدل وی لذت علم و عمل
&&
باقی بهانه‌ست و دغل کاین علت آمد وان دوا
\\
ما زان دغل کژبین شده با بی‌گنه در کین شده
&&
گه مست حورالعین شده گه مست نان و شوربا
\\
این سکر بین هل عقل را وین نقل بین هل نقل را
&&
کز بهر نان و بقل را چندین نشاید ماجرا
\\
تدبیر صدرنگ افکنی بر روم و بر زنگ افکنی
&&
و اندر میان جنگ افکنی فی اصطناع لا یری
\\
می‌مال پنهان گوش جان می‌نه بهانه بر کسان
&&
جان رب خلصنی زنان والله که لاغست ای کیا
\\
خامش که بس مستعجلم رفتم سوی پای علم
&&
کاغذ بنه بشکن قلم ساقی درآمد الصلا
\\
\end{longtable}
\end{center}
