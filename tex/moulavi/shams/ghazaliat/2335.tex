\begin{center}
\section*{غزل شماره ۲۳۳۵: رندان همه جمعند در این دیر مغانه}
\label{sec:2335}
\addcontentsline{toc}{section}{\nameref{sec:2335}}
\begin{longtable}{l p{0.5cm} r}
رندان همه جمعند در این دیر مغانه
&&
درده تو یکی رطل بدان پیر یگانه
\\
خون ریزبک عشق در و بام گرفته‌ست
&&
و آن عقل گریزان شده از خانه به خانه
\\
یک پرده برانداخته آن شاهد اعظم
&&
از پرده برون رفته همه اهل زمانه
\\
آن جنس که عشاق در این بحر فتادند
&&
چه جای امان باشد و چه جای امانه
\\
کی سرد شود عشق ز آواز ملامت
&&
هرگز نرمد شیر ز فریاد زنانه
\\
پر کن تو یکی رطل ز می‌های خدایی
&&
مگذار خدایان طبیعت به میانه
\\
اول بده آن رطل بدان نفس محدث
&&
تا ناطقه‌اش هیچ نگوید ز فسانه
\\
چون بند شود نطق یکی سیل درآید
&&
کز کون و مکان هیچ نبینی تو نشانه
\\
شمس الحق تبریز چه آتش که برافروخت
&&
احسنت زهی آتش و شاباش زبانه
\\
\end{longtable}
\end{center}
