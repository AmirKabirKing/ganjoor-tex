\begin{center}
\section*{غزل شماره ۲۴۹۶: خواجه ترش مرا بگو سرکه به چند می‌دهی}
\label{sec:2496}
\addcontentsline{toc}{section}{\nameref{sec:2496}}
\begin{longtable}{l p{0.5cm} r}
خواجه ترش مرا بگو سرکه به چند می‌دهی
&&
هست شکرلبی اگر سرکه به قند می‌دهی
\\
گر تو نمی‌خری مخر می به هوس همی‌خرم
&&
عاشق و بیخودم مرا هرزه چه پند می‌دهی
\\
پیشتر آ تو ای پری از ترشی تویی بری
&&
تاج و کمر عطا کنی بخت بلند می‌دهی
\\
جان به هزار ولوله بهر تو گشت حامله
&&
کآتش عشق خویش را تو به سپند می‌دهی
\\
چون فرهاد می‌کشی جان مرا به که کنی
&&
ور نه به دست جان من از چه کلند می‌دهی
\\
هر چه که می‌دهی بده بی‌خبر آن کسی که او
&&
بر تو گمان برد که تو بهر گزند می‌دهی
\\
برگ گلی همی‌بری باغ به پیش می‌کشی
&&
لاشه خری همی‌بری بیست سمند می‌دهی
\\
شاکر خدمتی ولی گاه ز لاابالیی
&&
نی به گنه همی‌زنی نی به پسند می‌دهی
\\
چون سر زید بشکند چاره عمرو می‌کنی
&&
چون به دمشق قحط شد آب به جند می‌دهی
\\
چند بگفتمت مگو لیک تو را گناه چیست
&&
ای تو چو آسیا به تو آنچ دهند می‌دهی
\\
\end{longtable}
\end{center}
