\begin{center}
\section*{غزل شماره ۲۶۴۷: دلا چون واقف اسرار گشتی}
\label{sec:2647}
\addcontentsline{toc}{section}{\nameref{sec:2647}}
\begin{longtable}{l p{0.5cm} r}
دلا چون واقف اسرار گشتی
&&
ز جمله کارها بی‌کار گشتی
\\
همان سودایی و دیوانه می‌باش
&&
چرا عاقل شدی هشیار گشتی
\\
تفکر از برای برد باشد
&&
تو سرتاسر همه ایثار گشتی
\\
همان ترتیب مجنون را نگه دار
&&
که از ترتیب‌ها بیزار گشتی
\\
چو تو مستور و عاقل خواستی شد
&&
چرا سرمست در بازار گشتی
\\
نشستن گوشه ای سودت ندارد
&&
چو با رندان این ره یار گشتی
\\
به صحرا رو بدان صحرا که بودی
&&
در این ویرانه‌ها بسیار گشتی
\\
خراباتی است در همسایه تو
&&
که از بوهای می خمار گشتی
\\
بگیر این بو و می‌رو تا خرابات
&&
که همچون بو سبک رفتار گشتی
\\
به کوه قاف رو مانند سیمرغ
&&
چه یار جغد و بوتیمار گشتی
\\
برو در بیشه معنی چو شیران
&&
چه یار روبه و کفتار گشتی
\\
مرو بر بوی پیراهان یوسف
&&
که چون یعقوب ماتم دار گشتی
\\
\end{longtable}
\end{center}
