\begin{center}
\section*{غزل شماره ۱۹۴۰: ای ز تو مه پای کوبان وز تو زهره دف زنان}
\label{sec:1940}
\addcontentsline{toc}{section}{\nameref{sec:1940}}
\begin{longtable}{l p{0.5cm} r}
ای ز تو مه پای کوبان وز تو زهره دف زنان
&&
می زنند ای جان مردان عشق ما بر دف زنان
\\
نقل هر مجلس شده‌ست این عشق ما و حسن تو
&&
شهره شهری شده ما کو چنین بد شد چنان
\\
ای به هر هنگامه دام عشق تو هنگامه گیر
&&
وی چکیده خون ما بر راه ره رو را نشان
\\
صد هزاران زخم بر سینه ز زخم تیر عشق
&&
صد شکار خسته و نی تیر پیدا نی کمان
\\
روی در دیوار کرده در غم تو مرد و زن
&&
ز آب و نان عشق رفته اشتهای آب و نان
\\
خون عاشق اشک شد وز اشک او سبزه برست
&&
سبزه‌ها از عکس روی چون گل تو گلستان
\\
ذوق عشقت چون ز حد شد خلق آتشخوار شد
&&
همچو اشترمرغ آتش می خورد در عشق جان
\\
هجر سرد چون زمستان راه‌ها را بسته بود
&&
در زمین محبوس بود اشکوفه‌های بوستان
\\
چونک راه ایمن شد از داد بهاران آمدند
&&
سبزه را تیغ برهنه غنچه را در کف سنان
\\
خیز بیرون آ به بستان کز ره دور آمدند
&&
خیز کالقادم یزار و رنجه شو مرکب بران
\\
از عدم بستند رخت و جانب بحر آمدند
&&
آنگه از بحر آمدند اندر هوا تا آسمان
\\
برج برج آسمان را گشته و پذرفته اند
&&
از هر استاره بضاعت و آمده تا خاکدان
\\
آب و آتش ز آسمانش می رسد هر دم مدد
&&
چند روزی کاندر این خاکند ایشان میهمان
\\
خوان‌ها بر سر نسیم و کاس‌ها بر کف صبا
&&
با طبق پوشی که پوشیده‌ست جز از اهل خوان
\\
می رسند و هر کسی پرسان که چیست اندر طبق
&&
با زبان حال می گویند با پرسندگان
\\
هر کسی گر محرمستی پس طبق پوشیده چیست
&&
قوت جان چون جان نهان و قوت تن پیدا چو نان
\\
ذوق نان هم گرسنه بیند نبیند هیچ سیر
&&
بر دکان نانبا از نان چه می داند دکان
\\
نانوا گر گرسنه ستی هیچ نان نفروختی
&&
گر بدانستی صبا گل را نکردی گلفشان
\\
هر کش از معشوق ذوقی نیست الا در فروخت
&&
او نباشد عاشق او باشد به معنی قلتبان
\\
عذر عاشق گر فروشد دانک میل دلبر است
&&
از ضرورت تا نبندد در به رویش دلستان
\\
چونک می بیند که میل دلبر اندر شهرگی است
&&
اشک می بارد ز رشک آن صنم از دیدگان
\\
اشک او مر رشک او را ضد و دشمن آمده‌ست
&&
رشک پنهان دارد و اشکش روان و قصه خوان
\\
تخم پنهان کرده خود را نگر باغ و چمن
&&
شهوت پنهان خود را بین یکی شخصی دوان
\\
عین پنهان داشتن شد علت پیدا شدن
&&
بی لسانی می شود بر رغم ما عین لسان
\\
چند فرزندان به هر اندیشه بعد مرگ خویش
&&
گرد جان خویش بینی در لحد باباکنان
\\
زاده از اندیشه‌های خوب تو ولدان و حور
&&
زاده از اندیشه‌های زشت تو دیو کلان
\\
سر اندیشه مهندس بین شده قصر و سرا
&&
سر تقدیر ازل را بین شده چندین جهان
\\
واقفی از سر خود از سر سر واقف نه‌ای
&&
سر سر همچون دل آمد سر تو همچون زبان
\\
گر سر تو هست خوب از سر سر ایمن مباش
&&
باش ناایمن که ناایمن همی‌یابد امان
\\
سربلندی سرو و خنده گل نوای عندلیب
&&
میوه‌های گرم رو سر دم سرد خزان
\\
برگ‌ها لرزان چه می لرزید وقت شادی است
&&
دام‌ها در دانه‌های خوش بود ای باغبان
\\
ما ز سرسبزی به روی زرد چند افتاده‌ایم
&&
در کمین غیب بس تیر است پران از کمان
\\
لاله رخ افروخته وز خشم شد دل سوخته
&&
سنبله پرسود و کژگردن ز اندیشه گران
\\
آن گل سوری ستیزه گل دکانی باز کرد
&&
رنگ‌ها آمیخت اما نیستش بویی از آن
\\
خوشه‌ها از سست پایی رو نهاده بر زمین
&&
غوره‌اش شیرین شد آخر از خطاب یسجدان
\\
نرگس خیره نگر آخر چه می بینی به باغ
&&
گفت غمازی کنم پس من نگنجم در میان
\\
سوسنا افسوس می داری زبان کردی برون
&&
یا زبان درکش چو ما و یا بکن حالی بیان
\\
گفت بی‌گفتن زبان ما بیان حال ماست
&&
گر نه پایان راسخستی سبز کی بودی سران
\\
گفتم ای بید پیاده چون پیاده رسته‌ای
&&
گفت تا لطف تواضع گیرم از آب روان
\\
رنگ معشوق است سیب لعل را طعم ترش
&&
زانک خوبان را ترش بودن بزیبد این بدان
\\
پس درخت و شاخ شفتالو چرا پستی نمود
&&
بهر شفتالو فشاندن پیش شفتالوستان
\\
گفت آری لیک وقتی می دهد شفتالویی
&&
که رسد جان از تن عاشق ز ناخن تا دهان
\\
ای سپیدار این بلندی جستنت رسوایی است
&&
چون نه گل داری نه میوه گفت خامش هان و هان
\\
گر گلم بودی و میوه همچو تو خودبینمی
&&
فارغم از دید خود بر خودپرستان دیدبان
\\
نار آبی را همی‌گفت این رخ زردت ز چیست
&&
گفت زان دردانه‌ها کاندر درون داری نهان
\\
گفت چون دانسته‌ای از سر من گفتا بدانک
&&
می نگنجی در خود و خندان نمایی ناردان
\\
نی تو خندانی همیشه خواه خند و خواه نی
&&
وز تو خندان است عالم چون جنان اندر جنان
\\
لیک آن خنده چون برق او راست کو گرید چو ابر
&&
ابر اگر گریان نباشد برق از او نبود جهان
\\
خاک را دیدم سیاه و تیره و روشن ضمیر
&&
آب روشن آمد از گردون و کردش امتحان
\\
آب روشن را پذیرا شد ضمیر روشنش
&&
زاد چون فردوس و جنت شاخ و کاخ بی‌کران
\\
این خیار و خربزه در راه دور و پای سست
&&
چون پیاده حاج می آیند اندر کاروان
\\
بادیه خون خوار بینی از عدم سوی وجود
&&
بر خطاب کن همه لبیک گو بهر امان
\\
چه پیاده بلک خفته رفته چون اصحاب کهف
&&
خفته پهلو بر زمین و رفته تک تا آسمان
\\
در چنین مجمع کدو آمد رسن بازی گرفت
&&
از کی دید آن زو که دادش آن رسن‌های رسان
\\
این چمن‌ها وین سمن وین میوه‌ها خود رزق ماست
&&
آن گیا و خار و گل کاندر بیابان است آن
\\
آن نصیب و میوه و روزی قومی دیگر است
&&
نفرت و بی‌میلی ما هست آن را پاسبان
\\
صد هزاران مور و مار و صد هزاران رزق خوار
&&
هر یکی جوید نصیبه هر یکی دارد فغان
\\
هر دوا درمان رنجی هر یکی را طالبی
&&
چون عقاقیری که نشناسد به غیر طب دان
\\
بس گیا کان پیش ما زهر و بر ایشان پای زهر
&&
پیش ما خار است و پیش اشتران خرمابنان
\\
جوز و بادام از درون مغز است و بیرون پوست و قشر
&&
اندرون پوست پرورده چو بیضه ماکیان
\\
باز خرما عکس آن بیرون خوش و باطن قشور
&&
باطن و ظاهر تو چون انجیر باش ای مهربان
\\
جذبه شاخ آب را از بیخ تا بالا کشد
&&
همچنانک جذبه جان را برکشد بی‌نردبان
\\
غوصه گشت این باد و آبستن شد آن خاک و درخت
&&
بادها چون گشن تازی شاخه‌ها چون مادیان
\\
می رسد هر جنس مرغی در بهار از گرمسیر
&&
همچو مهمان سرسری می سازد این جا آشیان
\\
صد هزاران غیب می گویند مرغان در ضمیر
&&
کان فلان خواهد گذشتن جای او گیرد فلان
\\
از سلیمان نامه‌ها آورده‌اند این هدهدان
&&
کو زبان مرغ دانی تا شود او ترجمان
\\
عارف مرغان است لک لک لک لکش دانی که چیست
&&
ملک لک و الامر لک و الحمد لک یا مستعان
\\
وقت پیله روح آمد قشلق تن را بهل
&&
آخر از مرغان بیاموزید رسم ترکمان
\\
همچو مرغان پاسبانی خویش کن تسبیح گو
&&
چند گاهی خود شود تسبیح تو تسبیح خوان
\\
بس کنم زین باد پیمودن ولیکن چاره نیست
&&
زانک کشتی مجاهد کی رود بی‌بادبان
\\
بادپیمایی بهار آمد حیات عالمی
&&
بادپیمایی خزان آمد عذاب انس و جان
\\
این بهار و باغ بیرون عکس باغ باطن است
&&
یک قراضه‌ست این همه عالم و باطن هست کان
\\
لاجرم ما هر چه می گوییم اندر نظم هست
&&
نزد عاشق نقد وقت و نزد عاقل داستان
\\
عقل دانایی است و نقلش نقل آمد یا قیاس
&&
عشق کان بینش آمد ز آفتاب کن فکان
\\
آفتابی کو مجرد آمد از برج حمل
&&
آفتابی بی‌نظیر بی‌قرین خوش قران
\\
آنک لاشرقیه بوده‌ست و لاغربیه
&&
زانک شرق و غرب باشد در زمین و در زمان
\\
آفتابی کو نسوزد جز دل عشاق را
&&
مهر جان ره یابد آن جا نی ربیع و مهر جان
\\
چونک ما را از زمین و از زمان بیرون برد
&&
از فنا ایمن شویم از جود او ما جاودان
\\
این زمین و این زمان بیضه‌ست و مرغی کاندر او است
&&
مظلم و اشکسته پر باشد حقیر و مستهان
\\
کفر و ایمان دان در این بیضه سپید و زرده را
&&
واصل و فارق میانشان برزخ لایبغیان
\\
بیضه را چون زیر پر خویش پرورد از کرم
&&
کفر و دین فانی شد و شد مرغ وحدت پرفشان
\\
شمس تبریزی دو عالم بود بی‌رویت عقیم
&&
هر یکی ذره کنون از آفتابت توامان
\\
\end{longtable}
\end{center}
