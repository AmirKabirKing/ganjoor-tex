\begin{center}
\section*{غزل شماره ۱۴۲۶: چه دانی تو که در باطن چه شاهی همنشین دارم}
\label{sec:1426}
\addcontentsline{toc}{section}{\nameref{sec:1426}}
\begin{longtable}{l p{0.5cm} r}
چه دانی تو که در باطن چه شاهی همنشین دارم
&&
رخ زرین من منگر که پای آهنین دارم
\\
بدان شه که مرا آورد کلی روی آوردم
&&
وزان کو آفریدستم هزاران آفرین دارم
\\
گهی خورشید را مانم گهی دریای گوهر را
&&
درون عز فلک دارم برون ذل زمین دارم
\\
درون خمره عالم چو زنبوری همی‌گردم
&&
مبین تو ناله‌ام تنها که خانه انگبین دارم
\\
دلا گر طالب مایی برآ بر چرخ خضرایی
&&
چنان قصری است حصن من که امن الؤمنین دارم
\\
چه باهول است آن آبی که این چرخ است از او گردان
&&
چو من دولاب آن آبم چنین شیرین حنین دارم
\\
چو دیو و آدمی و جن همی‌بینی به فرمانم
&&
نمی‌دانی سلیمانم که در خاتم نگین دارم
\\
چرا پژمرده باشم من که بشکفته‌ست هر جزوم
&&
چرا خربنده باشم من براقی زیر زین دارم
\\
چرا از ماه وامانم نه عقرب کوفت بر پایم
&&
چرا زین چاه برنایم چون من حبل متین دارم
\\
کبوترخانه‌ای کردم کبوترهای جان‌ها را
&&
بپر ای مرغ جان این سو که صد برج حصین دارم
\\
شعاع آفتابم من اگر در خانه‌ها گردم
&&
عقیق و زر و یاقوتم ولادت ز آب و طین دارم
\\
تو هر گوهر که می بینی بجو دری دگر در روی
&&
که هر ذره همی‌گوید که در باطن دفین دارم
\\
تو را هر گوهری گوید مشو قانع به حسن من
&&
که از شمع ضمیر است آن که نوری در جبین دارم
\\
خمش کردم که آن هوشی که دریابد نداری تو
&&
مجنبان گوش و مفریبان که چشمی هوش بین دارم
\\
\end{longtable}
\end{center}
