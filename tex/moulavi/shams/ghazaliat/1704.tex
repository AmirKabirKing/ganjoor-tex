\begin{center}
\section*{غزل شماره ۱۷۰۴: اشکم دهل شده‌ست از این جام دم به دم}
\label{sec:1704}
\addcontentsline{toc}{section}{\nameref{sec:1704}}
\begin{longtable}{l p{0.5cm} r}
اشکم دهل شده‌ست از این جام دم به دم
&&
می زن دهل به شکر دلا لم و لم و لم
\\
هین طبل شکر زن که می طبل یافتی
&&
گه زیر می زن ای دل و گه بم و بم و بم
\\
از بهر من بخر دهلی از دهلزنان
&&
تا برکنم ز باغ جهان شاخ و بیخ غم
\\
لشکر رسید و عشق سپهدار لشکرست
&&
صحرا و کوه پر شد از طبل و از علم
\\
ما پر شدیم تا به گلو ساقی از ستیز
&&
می ریزد آن شراب به اسراف همچو یم
\\
دانی که بحر موج چرا می زند به جوش
&&
از من شنو که بحریم و بحر اندرم
\\
تنگ آمده‌ست و می طلبد موضع فراخ
&&
بر می جهد به سوی هوا آب لاجرم
\\
کان آب از آسمان سفری خوی بوده‌ست
&&
اندر هوا و سیل و که و جوی ای صنم
\\
آب حیات ما کم از آن آب بحر نیست
&&
ما موج می زنیم ز هستی سوی عدم
\\
نی در جهان خاک قرار است روح را
&&
نی در هوای گنبد این چرخ خم به خم
\\
زان باغ کو شکفت همان جاست میل جان
&&
یعنی کنار صنع شهنشاه محتشم
\\
بس بس مکن هنوز تو را باده خوردنی است
&&
ما راضییم خواجه بدین ظلم و این ستم
\\
خاموش باش فتنه درافکنده‌ای به شهر
&&
خاموشیش مجوی که دریاست جان عم
\\
\end{longtable}
\end{center}
