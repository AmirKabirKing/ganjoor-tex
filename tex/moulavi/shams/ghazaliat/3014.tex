\begin{center}
\section*{غزل شماره ۳۰۱۴: رو که به مهمان تو می‌نروم ای اخی}
\label{sec:3014}
\addcontentsline{toc}{section}{\nameref{sec:3014}}
\begin{longtable}{l p{0.5cm} r}
رو که به مهمان تو می‌نروم ای اخی
&&
بست مرا از طعام دود دل مطبخی
\\
رزق جهان می‌دهد خویش نهان می‌کند
&&
گاه وصال او بخیل در زر و مال او سخی
\\
مال و زرش کم ستان جان بده از بهر جان
&&
مذهب سردان مگیر یخ چه کند جز یخی
\\
قسمت آن باردان مایده و نان گرم
&&
قسمت این عاشقان مملکت و فرخی
\\
قسمت قسام بین هیچ مگو و مچخ
&&
کار بتر می‌شود گر تو در این می‌چخی
\\
جنتی دل فروز دوزخیی خوش بسوز
&&
چند میان جهان مانده در برزخی
\\
سوی بتان کم نگر تا نشوی کوردل
&&
کور شود از نظر چشم سگ مسلخی
\\
زلف بتان سلسله‌ست جانب دوزخ کشد
&&
ظاهر او چون بهشت باطن او دوزخی
\\
لیک عنایات حق هست طبق بر طبق
&&
کو برهاند ز دام گر چه اسیر فخی
\\
جانب تبریز رو از جهت شمس دین
&&
چند در این تیرگی همچو خسان می‌زخی
\\
\end{longtable}
\end{center}
