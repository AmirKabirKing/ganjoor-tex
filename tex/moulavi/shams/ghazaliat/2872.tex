\begin{center}
\section*{غزل شماره ۲۸۷۲: مرغ اندیشه که اندر همه دل‌ها بپری}
\label{sec:2872}
\addcontentsline{toc}{section}{\nameref{sec:2872}}
\begin{longtable}{l p{0.5cm} r}
مرغ اندیشه که اندر همه دل‌ها بپری
&&
به خدا کز دل و از دلبر ما بی‌اثری
\\
آفتابی که به هر روزنه‌ای درتابی
&&
از سر روزن آن اصل بصر بی‌بصری
\\
باد شبگیر که چون پیک خبرها آری
&&
ز آنچ دریای خبرهاست چرا بی‌خبری
\\
دیدبانا که تو را عقل و خرد می‌گویند
&&
ساکن سقف دماغی و چراغ نظری
\\
بر سر بام شدستی مه نو می‌جویی
&&
مه نو کو و تو مسکین به کجا می‌نگری
\\
دل ترسنده که از عشق گریزان شده‌ای
&&
ز کف عشق اگر جان ببری جان نبری
\\
رهزنانند به هر گام یکی عشوه دهی
&&
وای بر تو گر از این عشوه دهان عشوه خری
\\
ای مه ار تو عسسی الحذر از جامه کنان
&&
که کلاهت ببرند ار چه که سیمین کمری
\\
به حشر غره مشو این نگر ای مه کز بیم
&&
می‌گریزی همه شب گر چه شه باحشری
\\
می‌گریزی تو ولی جان نبری از کف عشق
&&
تیرت آید سه پری گر چه همه تن سپری
\\
گر همه تن سپری ور ره پنهان سپری
&&
ور دو پر ور سه پری در فخ آن دام وری
\\
مردم چشم که مردم به تو مردم بیند
&&
نظرت نیست به دل گر چه که صاحب نظری
\\
در درون ظلمات سیهی چشمان
&&
همچو آب حیوان ساکنی و مستتری
\\
خانه در دیده گرفتی و تو را یار نشد
&&
آنک از چشمه او جوش کند دیده وری
\\
گر شکر را خبری بودی از لذت عشق
&&
آب گشتی ز خجالت ننمودی شکری
\\
چشم غیرت ز حسد گوش شکر را کر کرد
&&
ترس از آن چشم که در گوش شکر ریخت کری
\\
شیر گردون که همه شیردلان از تو برند
&&
جگر و صف شکنی حمیت و استیزه گری
\\
جگر باجگران آب ظفر از تو خورند
&&
به کمینگاه دل اهل دلان بی‌جگری
\\
شیر ز آتش برمد سخت و دل آتشکده‌ای است
&&
جان پروانه بود بر شرر شمع جری
\\
پر پروانه بسوزد جز پروانه دل
&&
که پرش ده پره گردد ز فروغ شرری
\\
شاه حلمی ز خلاء زیر پر دل می‌رو
&&
تا تو را علم دهد واهب انسان و پری
\\
رو به مریخ بگو که بنگر وصلت دل
&&
تا که خنجر بنهی هیچ سری را نبری
\\
گر توانی عوض سر سر دیگر دادن
&&
سزد ار سر ببری حاکم و وهاب سری
\\
سر ز تو یافت سری پر ز تو دزدید پری
&&
ز تو آموخت تری و ز تو آورد زری
\\
شیشه گر کو به دمی صد قدح و جام کند
&&
قدحی گر شکند زو نتوان گشت بری
\\
مشتری را نرسد لاف که من سیمبرم
&&
که نبود و نبود سیمبری سیم بری
\\
مشتری بود زلیخا مه کنعانی را
&&
سیم بر بود بر سیم بر از زرشمری
\\
زهره زخمه زن آخر بشنو زخمه دل
&&
بتری غره مشو چنگ کنندت بتری
\\
چنگ دل چند از این چنگ و دف و نای شکست
&&
وای بر مادر تو گر نکند دل پدری
\\
ای عطارد بس از این کاغذ و از حبر و قلم
&&
زفتی و لاف و تکبر حیل و پرهنری
\\
گر پلنگی به یکی باد چو موشی گردی
&&
ور تو شیری به یکی برق ز روبه بتری
\\
سر قدم کن چو قلم بر اثر دل می‌رو
&&
که اثرهاست نهان در عدم و بی‌صوری
\\
\end{longtable}
\end{center}
