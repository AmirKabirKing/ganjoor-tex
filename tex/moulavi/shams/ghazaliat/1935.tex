\begin{center}
\section*{غزل شماره ۱۹۳۵: دلبر بیگانه صورت مهر دارد در نهان}
\label{sec:1935}
\addcontentsline{toc}{section}{\nameref{sec:1935}}
\begin{longtable}{l p{0.5cm} r}
دلبر بیگانه صورت مهر دارد در نهان
&&
گر زبانش تلخ گوید قند دارد در دهان
\\
از درون سو آشنا و از برون بیگانه رو
&&
این چنین پرمهر دشمن من ندیدم در جهان
\\
چونک دلبر خشم گیرد عشق او می گویدم
&&
عاشق ناشی مباش و رو مگردان هان و هان
\\
راست ماند تلخی دلبر به تلخی شراب
&&
سازوار اندر مزاج و تلخ تلخ اندر زبان
\\
پیش او مردن به هر دم از شکر شیرینتر است
&&
مرده داند این سخن را تو مپرس از زندگان
\\
شاد روزی کاین غزل را من بخوانم پیش عشق
&&
سجده‌ای آرم بر زمین و جان سپارم در زمان
\\
مرغ جان را عشق گوید میل داری در قفس
&&
مرغ گوید من تو را خواهم قفس را بردران
\\
\end{longtable}
\end{center}
