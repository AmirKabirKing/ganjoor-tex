\begin{center}
\section*{غزل شماره ۲۱۸: ز بهر غیرت آموخت آدم اسما را}
\label{sec:0218}
\addcontentsline{toc}{section}{\nameref{sec:0218}}
\begin{longtable}{l p{0.5cm} r}
ز بهر غیرت آموخت آدم اسما را
&&
ببافت جامع کل پرده‌های اجزا را
\\
برای غیر بود غیرت و چو غیر نبود
&&
چرا نمود دو تا آن یگانه یکتا را
\\
دهان پر است جهان خموش را از راز
&&
چه مانع‌ست فصیحان حرف پیما را
\\
به بوسه‌های پیاپی ره دهان بستند
&&
شکرلبان حقایق دهان گویا را
\\
گهی ز بوسه یار و گهی ز جام عقار
&&
مجال نیست سخن را نه رمز و ایما را
\\
به زخم بوسه سخن را چه خوش همی‌شکنند
&&
به فتنه بسته ره فتنه را و غوغا را
\\
چو فتنه مست شود ناگهان برآشوبند
&&
چه چیز بند کند مست بی‌محابا را
\\
چو موج پست شود کوه‌ها و بحر شود
&&
که بیم آب کند سنگ‌های خارا را
\\
چو سنگ آب شود آب سنگ پس می‌دان
&&
احاطت ملک کامکار بینا را
\\
چو جنگ صلح شود صلح جنگ پس می‌بین
&&
صناعت کف آن کردگار دانا را
\\
بپوش روی که روپوش کار خوبان‌ست
&&
زبون و دستخوش و رام یافتی ما را
\\
حریف بین که فتادی تو شیر با خرگوش
&&
مکن مبند به کلی ره مواسا را
\\
طمع نگر که منت پند می‌دهم که مکن
&&
چنان که پند دهد نیم پشه عنقا را
\\
چنان که جنگ کند روی زرد با صفرا
&&
چنان که راه ببندد حشیش دریا را
\\
اکنت صاعقه یا حبیب او نارا
&&
فما ترکت لنا منزلا و لا دارا
\\
بک الفخار ولکن بهیت من سکر
&&
فلست افهم لی مفخرا و لا عارا
\\
متی اتوب من الذنب توبتی ذنبی
&&
متی اجار اذا العشق صار لی جارا
\\
یقول عقلی لا تبدلن هدی بردی
&&
اما قضیت به فی هلاک اوطارا
\\
\end{longtable}
\end{center}
