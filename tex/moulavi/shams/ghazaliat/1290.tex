\begin{center}
\section*{غزل شماره ۱۲۹۰: مست گشتم ز ذوق دشنامش}
\label{sec:1290}
\addcontentsline{toc}{section}{\nameref{sec:1290}}
\begin{longtable}{l p{0.5cm} r}
مست گشتم ز ذوق دشنامش
&&
یا رب آن می بهست یا جامش
\\
طرب افزاترست از باده
&&
آن سقط‌های تلخ آشامش
\\
بهر دانه نمی‌روم سوی دام
&&
بلک از عشق محنت دامش
\\
آن مهی که نه شرقی و غربیست
&&
نور بخشد شبش چو ایامش
\\
خاک آدم پر از عقیق چراست
&&
تا به معدن کشد به ناکامش
\\
گوهر چشم و دل رسول حقست
&&
حلقه گوش ساز پیغامش
\\
تن از آن سر چو جام جان نوشد
&&
هم از آن سر بود سرانجامش
\\
سرد شد نعمت جهان بر دل
&&
پیش حسن ولی انعامش
\\
شیخ هندو به خانقاه آمد
&&
نی تو ترکی درافکن از بامش
\\
کم او گیر و جمله هندوستان
&&
خاص او را بریز بر عامش
\\
طالع هند خود زحل آمد
&&
گر چه بالاست نحس شد نامش
\\
رفت بالا نرست از نحسی
&&
می بد را چه سود از جامش
\\
بد هندو نمودم آینه‌ام
&&
حسد و کینه نیست اعلامش
\\
نفس هندوست و خانقه دل من
&&
از برون نیست جنگ و آرامش
\\
بس که اصل سخن دو رو دارد
&&
یک سپید و دگر سیه فامش
\\
\end{longtable}
\end{center}
