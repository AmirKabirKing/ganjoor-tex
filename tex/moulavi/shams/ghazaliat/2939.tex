\begin{center}
\section*{غزل شماره ۲۹۳۹: ما را مسلم آمد هم عیش و هم عروسی}
\label{sec:2939}
\addcontentsline{toc}{section}{\nameref{sec:2939}}
\begin{longtable}{l p{0.5cm} r}
ما را مسلم آمد هم عیش و هم عروسی
&&
شادی هر مسلمان کوری هر فسوسی
\\
هر روز خطبه‌ای نو هر شام گردکی نو
&&
هر دم نثار گوهر نی قبضه فلوسی
\\
عشقی است سخت زیبا فقری است پای برجا
&&
بر آسمان نهی پا گر دست این دو بوسی
\\
جانی است چون چراغی در زیر طشت قالب
&&
کرد به پیش نورش خورشید چاپلوسی
\\
صد گونه رخت دارد صد تخت و بخت دارد
&&
تختش ز رفعت آمد نی تخت آبنوسی
\\
رختش ز نور مطلق در تخته جامه حق
&&
نی بارگیر سیسی نی جامه‌های سوسی
\\
از ذوق آتش دل وز سوزش خوش دل
&&
آتش پرست گشتم اما نیم مجوسی
\\
روزی دو همره آمد جان غریب با تن
&&
چون مرغزی و رازی چون مغربی و طوسی
\\
پرویزن است عالم ما همچو آرد در وی
&&
گر بگذری تو صافی ور نگذری سبوسی
\\
هر روز بر دکان‌ها بازار این خسان بین
&&
ای خام پیش ما آ کتان ماست روسی
\\
بشکن سبوی قالب ساغر ستان لبالب
&&
تا چند کاسه لیسی تا کی زبون لوسی
\\
دستور می‌دهی تا گویم تمام این را
&&
تا شرق و غرب گیرد اقبال بی‌نحوسی
\\
\end{longtable}
\end{center}
