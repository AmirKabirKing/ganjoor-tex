\begin{center}
\section*{غزل شماره ۷۱۵: اول نظر ار چه سرسری بود}
\label{sec:0715}
\addcontentsline{toc}{section}{\nameref{sec:0715}}
\begin{longtable}{l p{0.5cm} r}
اول نظر ار چه سرسری بود
&&
سرمایه و اصل دلبری بود
\\
گر عشق وبال و کافری بود
&&
آخر نه به روی آن پری بود
\\
زان رنگ تو گشته‌ایم بی‌رنگ
&&
زان سوی خرد هزار فرسنگ
\\
گر روم گزید جان اگر زنگ
&&
آخر نه به روی آن پری بود
\\
رو کرده به چتر پادشاهی
&&
وز نور مشارقش سپاهی
\\
گر یاوه شد او ز شاهراهی
&&
آخر نه به روی آن پری بود
\\
همچون مه بی‌پری پریدن
&&
چون سایه به رو و سر دویدن
\\
چون سرو ز بادها خمیدن
&&
آخر نه به روی آن پری بود
\\
زان مه که نواخت مشتری را
&&
جان داد بتان آزری را
\\
گر سهو فتاد سامری را
&&
آخر نه به روی آن پری بود
\\
گر هجده هزار عالم ای جان
&&
پر گشت ز قال و قالم ای جان
\\
گر حالم وگر محالم ای جان
&&
آخر نه به روی آن پری بود
\\
چون ماه نزارگشته شادیم
&&
کاندر پی آفتاب رادیم
\\
ور هم به خسوف درفتادیم
&&
آخر نه به روی آن پری بود
\\
ناموس شکسته‌ایم و مستیم
&&
صد توبه و عهد را شکستیم
\\
ور دست و ترنج را بخستیم
&&
آخر نه به روی آن پری بود
\\
زان جام شراب ارغوانی
&&
زان چشمه آب زندگانی
\\
گر داد فضولیی نشانی
&&
آخر نه به روی آن پری بود
\\
فصلی به جز این چهار فصلش
&&
نی فصل ربیع و اصل اصلش
\\
گر لاف زدیم ما ز وصلش
&&
آخر نه به روی آن پری بود
\\
خاموش که گفتنی نتان گفت
&&
رازش باید ز راه جان گفت
\\
ور مست شد این دل و نشان گفت
&&
آخر نه به روی آن پری بود
\\
\end{longtable}
\end{center}
