\begin{center}
\section*{غزل شماره ۷۹۴: هر کی از حلقه ما جای دگر بگریزد}
\label{sec:0794}
\addcontentsline{toc}{section}{\nameref{sec:0794}}
\begin{longtable}{l p{0.5cm} r}
هر کی از حلقه ما جای دگر بگریزد
&&
همچنان باشد کز سمع و بصر بگریزد
\\
زان خورد خون جگر عاشق زیرا شیر است
&&
شیردل کی بود آن کو ز جگر بگریزد
\\
دل چو طوطی بود و جور دلارام شکر
&&
طوطیی دید کسی کو ز شکر بگریزد
\\
پشه باشد که به هر باد مخالف برود
&&
دزد شب باشد کز نور قمر بگریزد
\\
هر سری را که خدا خیره و کالیوه کند
&&
صدر جنت بهلد سوی سقر بگریزد
\\
و آنک واقف بود از مرگ سوی مرگ گریخت
&&
سوی ملک ابد و تاج و کمر بگریزد
\\
چون قضا گفت فلانی به سفر خواهد مرد
&&
آن کس از بیم اجل سوی سفر بگریزد
\\
بس کن و صید مکن آنک نیرزد به شکار
&&
که خیال شب و شب هم ز سحر بگریزد
\\
\end{longtable}
\end{center}
