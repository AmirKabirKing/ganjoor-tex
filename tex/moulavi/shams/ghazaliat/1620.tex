\begin{center}
\section*{غزل شماره ۱۶۲۰: هوسی است در سر من که سر بشر ندارم}
\label{sec:1620}
\addcontentsline{toc}{section}{\nameref{sec:1620}}
\begin{longtable}{l p{0.5cm} r}
هوسی است در سر من که سر بشر ندارم
&&
من از این هوس چنانم که ز خود خبر ندارم
\\
دو هزار ملک بخشد شه عشق هر زمانی
&&
من از او به جز جمالش طمعی دگر ندارم
\\
کمر و کلاه عشقش به دو کون مر مرا بس
&&
چه شد ار کله بیفتد چه غم ار کمر ندارم
\\
سحری ببرد عشقش دل خسته را به جایی
&&
که ز روز و شب گذشتم خبر از سحر ندارم
\\
سفری فتاد جان را به ولایت معانی
&&
که سپهر و ماه گوید که چنین سفر ندارم
\\
ز فراق جان من گر ز دو دیده در فشاند
&&
تو گمان مبر که از وی دل پرگهر ندارم
\\
چه شکرفروش دارم که به من شکر فروشد
&&
که نگفت عذر روزی که برو شکر ندارم
\\
بنمودمی نشانی ز جمال او ولیکن
&&
دو جهان به هم برآید سر شور و شر ندارم
\\
تبریز عهد کردم که چو شمس دین بیاید
&&
بنهم به شکر این سر که به غیر سر ندارم
\\
\end{longtable}
\end{center}
