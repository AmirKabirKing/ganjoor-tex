\begin{center}
\section*{غزل شماره ۱۷۹۲: این کیست این این کیست این این یوسف ثانی است این}
\label{sec:1792}
\addcontentsline{toc}{section}{\nameref{sec:1792}}
\begin{longtable}{l p{0.5cm} r}
این کیست این این کیست این این یوسف ثانی است این
&&
خضر است و الیاس این مگر یا آب حیوانی است این
\\
این باغ روحانی است این یا بزم یزدانی است این
&&
سرمه سپاهانی است این یا نور سبحانی است این
\\
آن جان جان افزاست این یا جنت المأواست این
&&
ساقی خوب ماست این یا باده جانی است این
\\
تنگ شکر را ماند این سودای سر را ماند این
&&
آن سیمبر را ماند این شادی و آسانی است این
\\
امروز مستیم ای پدر توبه شکستیم ای پدر
&&
از قحط رستیم ای پدر امسال ارزانی است این
\\
ای مطرب داووددم آتش بزن در رخت غم
&&
بردار بانگ زیر و بم کاین وقت سرخوانی است این
\\
مست و پریشان توام موقوف فرمان توام
&&
اسحاق قربان توام این عید قربانی است این
\\
رستیم از خوف و رجا عشق از کجا شرم از کجا
&&
ای خاک بر شرم و حیا هنگام پیشانی است این
\\
گل‌های سرخ و زرد بین آشوب و بردابرد بین
&&
در قعر دریا گرد بین موسی عمرانی است این
\\
هر جسم را جان می کند جان را خدادان می کند
&&
داور سلیمان می کند یا حکم دیوانی است این
\\
ای عشق قلماشیت گو از عیش و خوش باشیت گو
&&
کس می نداند حرف تو گویی که سریانی است این
\\
خورشید رخشان می رسد مست و خرامان می رسد
&&
با گوی و چوگان می رسد سلطان میدانی است این
\\
هر جا یکی گویی بود در حکم چوگان می دود
&&
چون گوی شو بی‌دست و پا هنگام وحدانی است این
\\
گویی شوی بی‌دست و پا چوگان او پایت شود
&&
در پیش سلطان می دوی کاین سیر ربانی است این
\\
آن آب بازآمد به جو بر سنگ زن اکنون سبو
&&
سجده کن و چیزی مگو کاین بزم سلطانی است این
\\
\end{longtable}
\end{center}
