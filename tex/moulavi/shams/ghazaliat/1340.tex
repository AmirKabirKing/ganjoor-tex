\begin{center}
\section*{غزل شماره ۱۳۴۰: هر آن کو صبر کرد ای دل ز شهوت‌ها در این منزل}
\label{sec:1340}
\addcontentsline{toc}{section}{\nameref{sec:1340}}
\begin{longtable}{l p{0.5cm} r}
هر آن کو صبر کرد ای دل ز شهوت‌ها در این منزل
&&
عوض دیدست او حاصل به جان زان سوی آب و گل
\\
چو شخصی کو دو زن دارد یکی را دل شکن دارد
&&
بدان دیگر وطن دارد که او خوشتر بدش در دل
\\
تو گویی کاین بدین خوبی زهی صبر وی ایوبی
&&
وزین غبن اندر آشوبی که این کاریست بی‌طایل
\\
و او گوید ز سرمستی که آن را تو بدیدستی
&&
که آن علوست و تو پستی که تو نقصی و آن کامل
\\
بدو گر باز رو آرد و تخم دوستی کارد
&&
حجابی آن دگر دارد کز این سو راند او محمل
\\
چو باز آن خوب کم نازد و با این شخص درسازد
&&
دگربار او نپردازد از این سون رخت دل حاصل
\\
سر رشته صبوری را ببین بگذار کوری را
&&
ببین تو حسن حوری را صبوری نبودت مشکل
\\
همه کدیه از این حضرت به سجده و وقفه و رکعت
&&
برای دید این لذت کز او شهوت شود حامل
\\
بفرما صبر یاران را به پندی حرص داران را
&&
بمشنو نفس زاران را مباش از دست حرص آکل
\\
کسی را چون دهی پندی شود حرص تو را بندی
&&
صبوری گرددت قندی پی آجل در این عاجل
\\
ز بی‌چون بین که چون‌ها شد ز بی‌سون بین که سون‌ها شد
&&
ز حلمی بین که خون‌ها شد ز حقی چند گون باطل
\\
حروف تخته کانی بدین تأویل می‌خوانی
&&
خلاصه صبر می‌دانی بر آن تأویل شو عامل
\\
صبوری کن مکن تیزی ز شمس الدین تبریزی
&&
بشر خسپی ملک خیزی که او شاهیست بس مفضل
\\
\end{longtable}
\end{center}
