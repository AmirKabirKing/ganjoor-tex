\begin{center}
\section*{غزل شماره ۱۶۴۹: وقت آن شد که به زنجیر تو دیوانه شویم}
\label{sec:1649}
\addcontentsline{toc}{section}{\nameref{sec:1649}}
\begin{longtable}{l p{0.5cm} r}
وقت آن شد که به زنجیر تو دیوانه شویم
&&
بند را برگسلیم از همه بیگانه شویم
\\
جان سپاریم دگر ننگ چنین جان نکشیم
&&
خانه سوزیم و چو آتش سوی میخانه شویم
\\
تا نجوشیم از این خنب جهان برناییم
&&
کی حریف لب آن ساغر و پیمانه شویم
\\
سخن راست تو از مردم دیوانه شنو
&&
تا نمیریم مپندار که مردانه شویم
\\
در سر زلف سعادت که شکن در شکن است
&&
واجب آید که نگونتر ز سر شانه شویم
\\
بال و پر باز گشاییم به بستان چو درخت
&&
گر در این راه فنا ریخته چون دانه شویم
\\
گر چه سنگیم پی مهر تو چون موم شویم
&&
گر چه شمعیم پی نور تو پروانه شویم
\\
گر چه شاهیم برای تو چو رخ راست رویم
&&
تا بر این نطع ز فرزین تو فرزانه شویم
\\
در رخ آینه عشق ز خود دم نزنیم
&&
محرم گنج تو گردیم چو پروانه شویم
\\
ما چو افسانه دل بی‌سر و بی‌پایانیم
&&
تا مقیم دل عشاق چو افسانه شویم
\\
گر مریدی کند او ما به مرادی برسیم
&&
ور کلیدی کند او ما همه دندانه شویم
\\
مصطفی در دل ما گر ره و مسند نکند
&&
شاید ار ناله کنیم استن حنانه شویم
\\
نی خمش کن که خموشانه بباید دادن
&&
پاسبان را چو به شب ما سوی کاشانه شویم
\\
\end{longtable}
\end{center}
