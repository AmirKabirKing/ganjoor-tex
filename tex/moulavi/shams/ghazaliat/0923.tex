\begin{center}
\section*{غزل شماره ۹۲۳: رسید ساقی جان ما خمار خواب آلود}
\label{sec:0923}
\addcontentsline{toc}{section}{\nameref{sec:0923}}
\begin{longtable}{l p{0.5cm} r}
رسید ساقی جان ما خمار خواب آلود
&&
گرفت ساغر زرین سر سبو بگشود
\\
صلای باده جان و صلای رطل گران
&&
که می‌دهد به خماران به گاه زودازود
\\
زهی صباح مبارک زهی صبوح عزیز
&&
ز شاه جام شراب و ز ما رکوع و سجود
\\
شراب صافی و سلطان ندیم و دولت یار
&&
دگر نیارم گفتن که در میانه چه بود
\\
هر آنک می نخورد بر سرش فروریزد
&&
بگویدش که برو در جهان کور و کبود
\\
در این جهان که در او مرده می‌خورد مرده
&&
نخورد عاقل و ناسود و یک دمی نغنود
\\
چو پاک داشت شکم را رسید باده پاک
&&
زهی شراب و زهی جام و بزم و گفت و شنود
\\
شراب را تو نبینی و مست را بینی
&&
نبینی آتش دل را و خانه‌ها پردود
\\
دل خسان چو بسوزد چه بوی بد آید
&&
دل شهان چو بسوزد فزود عنبر و عود
\\
نبشته بر رخ هر مست رو که جان بردی
&&
نبشته بر لب ساغر که عاقبت محمود
\\
نبشته بر دف مطرب که زهره بنده تو
&&
نبشته بر کف ساقی که طالعت مسعود
\\
بخند موسی عمران به کوری فرعون
&&
بخور خلیل خدا نوش کوری نمرود
\\
بلیس اگر ز شراب خدای مست بدی
&&
ز صد گنه نشدی هیچ طاعتش مردود
\\
خمش کنم که خمش به پیش هشیاران
&&
که خلق خیره شدند و خیالشان افزود
\\
\end{longtable}
\end{center}
