\begin{center}
\section*{غزل شماره ۵۹۴: امروز جمال تو سیمای دگر دارد}
\label{sec:0594}
\addcontentsline{toc}{section}{\nameref{sec:0594}}
\begin{longtable}{l p{0.5cm} r}
امروز جمال تو سیمای دگر دارد
&&
امروز لب نوشت حلوای دگر دارد
\\
امروز گل لعلت از شاخ دگر رستست
&&
امروز قد سروت بالای دگر دارد
\\
امروز خود آن ماهت در چرخ نمی‌گنجد
&&
وان سکه چون چرخت پهنای دگر دارد
\\
امروز نمی‌دانم فتنه ز چه پهلو خاست
&&
دانم که از او عالم غوغای دگر دارد
\\
آن آهوی شیرافکن پیداست در آن چشمش
&&
کو از دو جهان بیرون صحرای دگر دارد
\\
رفت این دل سودایی گم شد دل و هم سودا
&&
کو برتر از این سودا سودای دگر دارد
\\
گر پا نبود عاشق با پر ازل پرد
&&
ور سر نبود عاشق سرهای دگر دارد
\\
دریای دو چشم او را می‌جست و تهی می‌شد
&&
آگاه نبد کان در دریای دگر دارد
\\
در عشق دو عالم را من زیر و زبر کردم
&&
این جاش چه می‌جستی کو جای دگر دارد
\\
امروز دلم عشقست فردای دلم معشوق
&&
امروز دلم در دل فردای دگر دارد
\\
گر شاه صلاح الدین پنهانست عجب نبود
&&
کز غیرت حق هر دم لالای دگر دارد
\\
\end{longtable}
\end{center}
