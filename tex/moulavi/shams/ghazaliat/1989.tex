\begin{center}
\section*{غزل شماره ۱۹۸۹: جنتی کرد جهان را ز شکر خندیدن}
\label{sec:1989}
\addcontentsline{toc}{section}{\nameref{sec:1989}}
\begin{longtable}{l p{0.5cm} r}
جنتی کرد جهان را ز شکر خندیدن
&&
آنک آموخت مرا همچو شرر خندیدن
\\
گر چه من خود ز عدم دلخوش و خندان زادم
&&
عشق آموخت مرا شکل دگر خندیدن
\\
بی جگر داد مرا شه دل چون خورشیدی
&&
تا نمایم همه را بی‌ز جگر خندیدن
\\
به صدف مانم خندم چو مرا درشکنند
&&
کار خامان بود از فتح و ظفر خندیدن
\\
یک شب آمد به وثاق من و آموخت مرا
&&
جان هر صبح و سحر همچو سحر خندیدن
\\
گر ترش روی چو ابرم ز درون خندانم
&&
عادت برق بود وقت مطر خندیدن
\\
چون به کوره گذری خوش به زر سرخ نگر
&&
تا در آتش تو ببینی ز حجر خندیدن
\\
زر در آتش چو بخندید تو را می گوید
&&
گر نه قلبی بنما وقت ضرر خندیدن
\\
گر تو میر اجلی از اجل آموز کنون
&&
بر شه عاریت و تاج و کمر خندیدن
\\
ور تو عیسی صفتی خواجه درآموز از او
&&
بر غم شهوت و بر ماده و نر خندیدن
\\
ور دمی مدرسه احمد امی دیدی
&&
رو حلالستت بر فضل و هنر خندیدن
\\
ای منجم اگرت شق قمر باور شد
&&
بایدت بر خود و بر شمس و قمر خندیدن
\\
همچو غنچه تو نهان خند و مکن همچو نبات
&&
وقت اشکوفه به بالای شجر خندیدن
\\
\end{longtable}
\end{center}
