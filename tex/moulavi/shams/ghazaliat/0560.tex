\begin{center}
\section*{غزل شماره ۵۶۰: عاشق دلبر مرا شرم و حیا چرا بود}
\label{sec:0560}
\addcontentsline{toc}{section}{\nameref{sec:0560}}
\begin{longtable}{l p{0.5cm} r}
عاشق دلبر مرا شرم و حیا چرا بود
&&
چونک جمال این بود رسم وفا چرا بود
\\
این همه لطف و سرکشی قسمت خلق چون شود
&&
این همه حسن و دلبری بر بت ما چرا بود
\\
درد فراق من کشم ناله به نای چون رسد
&&
آتش عشق من برم چنگ دوتا چرا بود
\\
لذت بی‌کرانه ایست عشق شدست نام او
&&
قاعده خود شکایتست ور نه جفا چرا بود
\\
از سر ناز و غنج خود روی چنان ترش کند
&&
آن ترشی روی او روح فزا چرا بود
\\
آن ترشی روی او ابرصفت همی‌شود
&&
ور نه حیات و خرمی باغ و گیا چرا بود
\\
\end{longtable}
\end{center}
