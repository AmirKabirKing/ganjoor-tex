\begin{center}
\section*{غزل شماره ۲۵۶۷: افتاد دل و جانم در فتنه طراری}
\label{sec:2567}
\addcontentsline{toc}{section}{\nameref{sec:2567}}
\begin{longtable}{l p{0.5cm} r}
افتاد دل و جانم در فتنه طراری
&&
سنگینک جنگینک سر بسته چو بیماری
\\
آید سوی بی‌خوابی خواهد ز درش آبی
&&
آب چه که می‌خواهد تا درفکند ناری
\\
گوید که به اجرت ده این خانه مرا چندی
&&
هین تا چه کنی سازم از آتشش انباری
\\
گه گوید این عرصه کاین خانه برآوردی
&&
بوده‌ست از آن من تو دانی و دیواری
\\
دیوار ببر زین جا این عرصه به ما واده
&&
در عرصه جان باشد دیوار تو مرداری
\\
آن دلبر سروین قد در قصد کسی باشد
&&
در کوی همی‌گردد چون مشتغل کاری
\\
ناگه بکند چاهی ناگه بزند راهی
&&
ناگه شنوی آهی از کوچه و بازاری
\\
جان نقش همی‌خواند می‌داند و می‌راند
&&
چون رخت نمی‌ماند در غارت او باری
\\
ای شاه شکرخنده‌ای شادی هر زنده
&&
دل کیست تو را بنده جان کیست گرفتاری
\\
ای ذوق دل از نوشت وی شوق دل از جوشت
&&
پیش آر به من گوشت تا نشنود اغیاری
\\
از باغ تو جان و تن پر کرده ز گل دامن
&&
آموخت خرامیدن با تو به سمن زاری
\\
زان گوش همی‌خارد کاومید چنین دارد
&&
و آن گاه یقین دارد این از کرمت آری
\\
تا از تو شدم دانا چون چنگ شدم جانا
&&
بشنو هله مولانا زاری چنین زاری
\\
تا عشق حمیاخد این مهر همی‌کارد
&&
خامش که دلم دارد بی‌مشغله گفتاری
\\
\end{longtable}
\end{center}
