\begin{center}
\section*{غزل شماره ۲۶۳۷: امروز سماع است و شراب است و صراحی}
\label{sec:2637}
\addcontentsline{toc}{section}{\nameref{sec:2637}}
\begin{longtable}{l p{0.5cm} r}
امروز سماع است و شراب است و صراحی
&&
یک ساقی بدمست یکی جمع مباحی
\\
زان جنس مباحی که از آن سوی وجود است
&&
نی اباحتی گیج حشیشی مزاحی
\\
روحی است مباحی که از آن روح چشیده‌ست
&&
کو روح قدیمی و کجا روح ریاحی
\\
در پیش چنین فتنه و در دست چنین می
&&
یا رب چه شود جان مسلمان صلاحی
\\
زین باده کسی را جگر تشنه خنک شد
&&
کو خون جگر ریخت در این ره به سفاحی
\\
جاوید شود عمر بدین کاس صبوحی
&&
ایمن شود از مرگ و ز افغان نیاحی
\\
این صورت غیب است که سرخیش ز خون نیست
&&
اسپید ز نور است نه کافور رباحی
\\
شمعی است برافروخته وز عرش گذشته
&&
پروانه او سینه دل‌های فلاحی
\\
سوزیده ز نورش حجب سبع سماوات
&&
پران شده جان‌ها و روان‌ها ز نواحی
\\
این حلقه مستان خرابات خراب است
&&
دور از لب و دندان تو ای خواجه صاحی
\\
شاباش زهی حال که از حال رهیدیت
&&
شاباش زهی عیش صبوحی و صباحی
\\
با خود ملک الموت بگوید هله واگرد
&&
کاین جا نکند هیچ سلاح تو سلاحی
\\
ما را خبری نی که خبر نیز چه باشد
&&
خود مغفرت این باشد و آمرزش ماحی
\\
از غیب شنو نعره مستان و خمش کن
&&
یک غلغله پاک ز آواز صیاحی
\\
ور نه بدو نان بنده دونان و خسان باش
&&
می‌خور پی سه نان ز سنان زخم رماحی
\\
فارس شده شمس الحق تبریز همیشه
&&
بر شمس شموس و نکند شمس جماحی
\\
\end{longtable}
\end{center}
