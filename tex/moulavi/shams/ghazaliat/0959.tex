\begin{center}
\section*{غزل شماره ۹۵۹: اگر دل از غم دنیا جدا توانی کرد}
\label{sec:0959}
\addcontentsline{toc}{section}{\nameref{sec:0959}}
\begin{longtable}{l p{0.5cm} r}
اگر دل از غم دنیا جدا توانی کرد
&&
نشاط و عیش به باغ بقا توانی کرد
\\
اگر به آب ریاضت برآوری غسلی
&&
همه کدورت دل را صفا توانی کرد
\\
ز منزل هوسات ار دو گام پیش نهی
&&
نزول در حرم کبریا توانی کرد
\\
درون بحر معانی لا نه آن گهری
&&
که قدر و قیمت خود را بها توانی کرد
\\
به همت ار نشوی در مقام خاک مقیم
&&
مقام خویش بر اوج علا توانی کرد
\\
اگر به جیب تفکر فروبری سر خویش
&&
گذشته‌های قضا را ادا توانی کرد
\\
ولیکن این صفت ره روان چالاکست
&&
تو نازنین جهانی کجا توانی کرد
\\
نه دست و پای اجل را فرو توانی بست
&&
نه رنگ و بوی جهان را رها توانی کرد
\\
تو رستم دل و جانی و سرور مردان
&&
اگر به نفس لئیمت غزا توانی کرد
\\
مگر که درد غم عشق سر زند در تو
&&
به درد او غم دل را روا توانی کرد
\\
ز خار چون و چرا این زمان چو درگذری
&&
به باغ جنت وصلش چرا توانی کرد
\\
اگر تو جنس همایی و جنس زاغ نه‌ای
&&
ز جان تو میل به سوی هما توانی کرد
\\
همای سایه دولت چو شمس تبریزیست
&&
نگر که در دل آن شاه جا توانی کرد
\\
\end{longtable}
\end{center}
