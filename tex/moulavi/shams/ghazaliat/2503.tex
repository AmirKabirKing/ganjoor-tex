\begin{center}
\section*{غزل شماره ۲۵۰۳: چو سرمست منی ای جان ز خیر و شر چه اندیشی}
\label{sec:2503}
\addcontentsline{toc}{section}{\nameref{sec:2503}}
\begin{longtable}{l p{0.5cm} r}
چو سرمست منی ای جان ز خیر و شر چه اندیشی
&&
براق عشق جان داری ز مرگ خر چه اندیشی
\\
چو من با تو چنین گرمم چه آه سرد می‌آری
&&
چو بر بام فلک رفتی ز بحر و بر چه اندیشی
\\
خوش آوازی من دیدی دواسازی من دیدی
&&
رسن بازی من دیدی از این چنبر چه اندیشی
\\
بر این صورت چه می‌چفسی ز بی‌معنی چه می‌ترسی
&&
چو گوهر در بغل داری ز بدگوهر چه اندیشی
\\
تویی گوهر ز دست تو که بجهد یا ز شست تو
&&
همه مصرند مست تو ز کور و کر چه اندیشی
\\
چو با دل یار غاری تو چراغ چار یاری تو
&&
فقیر ذوالفقاری تو از آن خنجر چه اندیشی
\\
چو مد و جر خود دیدی چو بال و پر خود دیدی
&&
چو کر و فر خود دیدی ز هر بی‌فر چه اندیشی
\\
بیا ای خاصه جانان پناه جان مهمانان
&&
تویی سلطان سلطانان ز بوالفنجر چه اندیشی
\\
خمش کن همچو ماهی شو در این دریای خوش دررو
&&
چو در قعر چنین آبی از آن آذر چه اندیشی
\\
\end{longtable}
\end{center}
