\begin{center}
\section*{غزل شماره ۲۴۷۳: آب تو ده گسسته را در دو جهان سقا تویی}
\label{sec:2473}
\addcontentsline{toc}{section}{\nameref{sec:2473}}
\begin{longtable}{l p{0.5cm} r}
آب تو ده گسسته را در دو جهان سقا تویی
&&
بار تو ده شکسته را بارگه وفا تویی
\\
برج نشاط رخنه شد لشکر دل برهنه شد
&&
میمنه را کله تویی میسره را قبا تویی
\\
می زده مییم ما کوفته دییم ما
&&
چشم نهاده‌ایم ما در تو که توتیا تویی
\\
روی متاب از وفا خاک مریز بر صفا
&&
آب حیاتی و حیا پشت دل و بقا تویی
\\
چرخ تو را ندا کند بهر تو جان فدا کند
&&
هر چه ز تو زیان کند آن همه را دوا تویی
\\
خیز بیار باده‌ای مرکب هر پیاده‌ای
&&
بهر زکات جان خود ساقی جان ما تویی
\\
این خبر و مجادلی نیست نشان یک دلی
&&
گردن این خبر بزن شحنه کبریا تویی
\\
گردن عربده بزن وسوسه را ز بن بکن
&&
باده خاص درفکن خاصبک خدا تویی
\\
وقت لقای یوسفان مست بدند کف بران
&&
ما نه کمیم از زنان یوسف خوش لقا تویی
\\
از رخ دوست باخبر وز کف خویش بی‌خبر
&&
این خبری است معتبر پیش تو کاوستا تویی
\\
پر کن زان می نهان تا بخوریم بی‌دهان
&&
تا که بداند این جهان باز که کیمیا تویی
\\
باده کهنه خدا روز الست ره نما
&&
گشته به دست انبیا وارث انبیا تویی
\\
\end{longtable}
\end{center}
