\begin{center}
\section*{غزل شماره ۴۰۷: چشم پرنور که مست نظر جانانست}
\label{sec:0407}
\addcontentsline{toc}{section}{\nameref{sec:0407}}
\begin{longtable}{l p{0.5cm} r}
چشم پرنور که مست نظر جانانست
&&
ماه از او چشم گرفتست و فلک لرزانست
\\
خاصه آن لحظه که از حضرت حق نور کشد
&&
سجده گاه ملک و قبله هر انسانست
\\
هر که او سر ننهد بر کف پایش آن دم
&&
بهر ناموس منی آن نفس او شیطانست
\\
و آنک آن لحظه نبیند اثر نور برو
&&
او کم از دیو بود زانک تن بی‌جانست
\\
دل به جا دار در آن طلعت باهیبت او
&&
گر تو مردی که رخش قبله گه مردانست
\\
دست بردار ز سینه چه نگه می‌داری
&&
جان در آن لحظه بده شاد که مقصود آنست
\\
جمله را آب درانداز و در آن آتش شو
&&
کآتش چهره او چشمه گه حیوانست
\\
سر برآور ز میان دل شمس تبریز
&&
کو خدیو ابد و خسرو هر فرمانست
\\
\end{longtable}
\end{center}
