\begin{center}
\section*{غزل شماره ۲۵۷۰: پنهان به میان ما می‌گردد سلطانی}
\label{sec:2570}
\addcontentsline{toc}{section}{\nameref{sec:2570}}
\begin{longtable}{l p{0.5cm} r}
پنهان به میان ما می‌گردد سلطانی
&&
و اندر حشر موران افتاده سلیمانی
\\
می‌بیند و می‌داند یک یک سر یاران را
&&
امروز در این مجمع شاهنشه سردانی
\\
اسرار بر او ظاهر همچون طبق حلوا
&&
گر مکر کند دزدی ور راست رود جانی
\\
نیک و بد هر کس را از تخته پیشانی
&&
می‌بیند و می‌خواند با تجربه خط خوانی
\\
در مطبخ ما آمد یک بی‌من و بی‌مایی
&&
تا شور دراندازد بر ما ز نمکدانی
\\
امروز سماع ما چون دل سبکی دارد
&&
یا رب تو نگهدارش ز آسیب گران جانی
\\
آن شیشه دلی کو دی بگریخت چو نامردان
&&
امروز همی‌آید پرشرم و پشیمانی
\\
صد سال اگر جایی بگریزد و بستیزد
&&
پرگریه و غم باشد بی‌دولت خندانی
\\
خورشید چه غم دارد ار خشم کند گازر
&&
خاموش که بازآید بلبل به گلستانی
\\
\end{longtable}
\end{center}
