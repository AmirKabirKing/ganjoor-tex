\begin{center}
\section*{غزل شماره ۱۵۳۱: بیا کامروز شه را ما شکاریم}
\label{sec:1531}
\addcontentsline{toc}{section}{\nameref{sec:1531}}
\begin{longtable}{l p{0.5cm} r}
بیا کامروز شه را ما شکاریم
&&
سر خویش و سر عالم نداریم
\\
بیا کامروز چون موسی عمران
&&
به مردی گرد از دریا برآریم
\\
همه شب چون عصا افتاده بودیم
&&
چو روز آمد چو ثعبان بی‌قراریم
\\
چو گرد سینه خود طوف کردیم
&&
ید بیضا ز جیب جان برآریم
\\
بدان قدرت که ماری شد عصایی
&&
به هر شب چون عصا و روز ماریم
\\
پی فرعون سرکش اژدهاییم
&&
پی موسی عصا و بردباریم
\\
به همت خون نمرودان بریزیم
&&
تو این منگر که چون پشه نزاریم
\\
برافزاییم بر شیران و پیلان
&&
اگر چه در کف آن شیر زاریم
\\
اگر چه همچو اشتر کژنهادیم
&&
چو اشتر سوی کعبه راهواریم
\\
به اقبال دوروزه دل نبندیم
&&
که در اقبال باقی کامکاریم
\\
چو خورشید و قمر نزدیک و دوریم
&&
چو عشق و دل نهان و آشکاریم
\\
برای عشق خون آشام خون خوار
&&
سگانش را چو خون اندر تغاریم
\\
چو ماهی وقت خاموشی خموشیم
&&
به وقت گفت ماه بی‌غباریم
\\
\end{longtable}
\end{center}
