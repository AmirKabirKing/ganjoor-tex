\begin{center}
\section*{غزل شماره ۹۸۶: عشق را جان بی‌قرار بود}
\label{sec:0986}
\addcontentsline{toc}{section}{\nameref{sec:0986}}
\begin{longtable}{l p{0.5cm} r}
عشق را جان بی‌قرار بود
&&
یاد جان پیش عشق عار بود
\\
سر و جان پیش او حقیر بود
&&
هر که را در سر این خمار بود
\\
همه بر قلب می‌زند عاشق
&&
اندر آن صف که کارزار بود
\\
نکند جانب گریز نظر
&&
گر چه شمشیر صد هزار بود
\\
عشق خود مرغزار شیرانست
&&
کی سگی شیر مرغزار بود
\\
عشق جان‌ها در آستین دارد
&&
در ره عشق جان نثار بود
\\
نام و ناموس و شرم و اندیشه
&&
پیش جاروبشان غبار بود
\\
همه کس را شکار کرد بلا
&&
عاشقان را بلا شکار بود
\\
مر بلا را چنان به جان بخرند
&&
کان بلا نیز شرمسار بود
\\
جان عشق است شه صلاح الدین
&&
کو ز اسرار کردگار بود
\\
\end{longtable}
\end{center}
