\begin{center}
\section*{غزل شماره ۷۵۴: می‌خرامد آفتاب خوبرویان ره کنید}
\label{sec:0754}
\addcontentsline{toc}{section}{\nameref{sec:0754}}
\begin{longtable}{l p{0.5cm} r}
می‌خرامد آفتاب خوبرویان ره کنید
&&
روی‌ها را از جمال خوب او چون مه کنید
\\
مردگان کهنه را رویش دو صد جان می‌دهد
&&
عاشقان رفته را از روی او آگه کنید
\\
از کف آن هر دو ساقی چشم او و لعل او
&&
هر زمانی می خورید و هر زمانی خه کنید
\\
جانب صحرای رویش طرفه چاهی گفته‌اند
&&
قصد آن صحرا کنید و نیت آن چه کنید
\\
نک نشان روشنی در خیمه‌ها تابان شدست
&&
گوش اسبان را به سوی خیمه و خرگه کنید
\\
آستان خرگهش شد کهربای عاشقان
&&
عاشقان لاغر تن خود را چو برگ که کنید
\\
در خمار چشم مستش چشم‌ها روشن کنید
&&
وز برای چشم بد را ناله و آوه کنید
\\
شاه جان‌ها شمس تبریزیست و این دم آن اوست
&&
رخ بدو آرید و خود را جمله مات شه کنید
\\
\end{longtable}
\end{center}
