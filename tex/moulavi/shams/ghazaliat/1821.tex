\begin{center}
\section*{غزل شماره ۱۸۲۱: آب حیات عشق را در رگ ما روانه کن}
\label{sec:1821}
\addcontentsline{toc}{section}{\nameref{sec:1821}}
\begin{longtable}{l p{0.5cm} r}
آب حیات عشق را در رگ ما روانه کن
&&
آینه صبوح را ترجمه شبانه کن
\\
ای پدر نشاط نو بر رگ جان ما برو
&&
جام فلک نمای شو وز دو جهان کرانه کن
\\
ای خردم شکار تو تیر زدن شعار تو
&&
شست دلم به دست کن جان مرا نشانه کن
\\
گر عسس خرد تو را منع کند از این روش
&&
حیله کن و ازو بجه دفع دهش بهانه کن
\\
در مثل است کاشقران دور بوند از کرم
&&
ز اشقر می کرم نگر با همگان فسانه کن
\\
ای که ز لعب اختران مات و پیاده گشته‌ای
&&
اسپ گزین فروز رخ جانب شه دوانه کن
\\
خیز کلاه کژ بنه وز همه دام‌ها بجه
&&
بر رخ روح بوسه ده زلف نشاط شانه کن
\\
خیز بر آسمان برآ با ملکان شو آشنا
&&
مقعد صدق اندرآ خدمت آن ستانه کن
\\
چونک خیال خوب او خانه گرفت در دلت
&&
چون تو خیال گشته‌ای در دل و عقل خانه کن
\\
هست دو طشت در یکی آتش و آن دگر ز زر
&&
آتش اختیار کن دست در آن میانه کن
\\
شو چو کلیم هین نظر تا نکنی به طشت زر
&&
آتش گیر در دهان لب وطن زبانه کن
\\
حمله شیر یاسه کن کله خصم خاصه کن
&&
جرعه خون خصم را نام می مغانه کن
\\
کار تو است ساقیا دفع دوی بیا بیا
&&
ده به کفم یگانه‌ای تفرقه را یگانه کن
\\
شش جهت است این وطن قبله در او یکی مجو
&&
بی وطنی است قبله گه در عدم آشیانه کن
\\
کهنه گر است این زمان عمر ابد مجو در آن
&&
مرتع عمر خلد را خارج این زمانه کن
\\
ای تو چو خوشه جان تو گندم و کاه قالبت
&&
گر نه خری چه که خوری روی به مغز و دانه کن
\\
هست زبان برون در حلقه در چه می شوی
&&
در بشکن به جان تو سوی روان روانه کن
\\
\end{longtable}
\end{center}
