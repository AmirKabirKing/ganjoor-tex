\begin{center}
\section*{غزل شماره ۷۸۱: در دلم چون غمت ای سرو روان برخیزد}
\label{sec:0781}
\addcontentsline{toc}{section}{\nameref{sec:0781}}
\begin{longtable}{l p{0.5cm} r}
در دلم چون غمت ای سرو روان برخیزد
&&
همچو سرو این تن من بی‌دل و جان برخیزد
\\
من گمانم تو عیان پیش تو من محو به هم
&&
چون عیان جلوه کند چهره گمان برخیزد
\\
چون رسد سنجق تو در ستمستان جهان
&&
ظلم کوته شود و کوچ و قلان برخیزد
\\
بر حصار فلک ار خوبی تو جمله برد
&&
از مقیمان فلک بانگ امان برخیزد
\\
بگذر از باغ جهان یک سحر ای رشک بهار
&&
تا ز گلزار چمن رسم خزان برخیزد
\\
پشت افلاک خمیدست از این بار گران
&&
ز سبک روحی تو بار گران برخیزد
\\
من چو از تیر توم بال و پرم ده بپران
&&
خوش پرد تیر زمانی که کمان برخیزد
\\
رمه خفتست و همی‌گردد گرگ از چپ و راست
&&
سگ ما بانگ زند تا که شبان برخیزد
\\
هین خمش دل پنهانست چو رگ زیر زبان
&&
آشکارا شود آن رگ چو زبان برخیزد
\\
این مجابات مجیرست در آن قطعه که گفت
&&
بر سر کوی تو عقل از سر جان برخیزد
\\
\end{longtable}
\end{center}
