\begin{center}
\section*{غزل شماره ۲۵۴۳: بیا ای شاه خودکامه نشین بر تخت خودکامی}
\label{sec:2543}
\addcontentsline{toc}{section}{\nameref{sec:2543}}
\begin{longtable}{l p{0.5cm} r}
بیا ای شاه خودکامه نشین بر تخت خودکامی
&&
بیا بر قلب رندان زن که صاحب قرن ایامی
\\
برآور دودها از دل به جز در خون مکن منزل
&&
فلک را از فلک بگسل که جان آتش اندامی
\\
در آن دریا که خون است آن ز خشک و تر برون است آن
&&
بیا بنما که چون است آن که حوت موج آشامی
\\
اشارت کن بدان سرده که رندانند اندر ده
&&
سبک رطل گران درده که تو ساقی آن جامی
\\
قدح در کار شیران کن ز زرشان چشم سیران کن
&&
به جامی عقل ویران کن که عقل آن جا بود خامی
\\
بسوز از حسن ای خاقان تو نام و ننگ مشتاقان
&&
که سرد آید ز عشاقان حذر کردن ز بدنامی
\\
بدیدم عقل کل را من نهاده ذبح بر گردن
&&
بگفتم پیش این پرفن چو اسماعیل چون رامی
\\
بگفت از عشق شمس الدین که تبریز است از او چون چین
&&
چو مه رویان نوآیین به گرد مجلس سامی
\\
\end{longtable}
\end{center}
