\begin{center}
\section*{غزل شماره ۱۶۹۵: پیش چنین جمال جان بخش چون نمیرم}
\label{sec:1695}
\addcontentsline{toc}{section}{\nameref{sec:1695}}
\begin{longtable}{l p{0.5cm} r}
پیش چنین جمال جان بخش چون نمیرم
&&
دیوانه چون نگردم زنجیر چون نگیرم
\\
چون باده تو خوردم من محو چون نگردم
&&
تو چون میی من آبم تو شهد و من چو شیرم
\\
بگشا دهان خود را آن قند بی‌عدد را
&&
عذر ار نمی‌پذیری من عشوه می پذیرم
\\
دانی که از چه خندم از همت بلندم
&&
زیرا به شهر عشقت بر عاشقان امیرم
\\
با عشق لایزالی از یک شکم بزادم
&&
نوعشق می نمایم والله که سخت پیرم
\\
آن چشم اگر گشایی جز خویش را نشایی
&&
ور این نظر گشایی دانی که بی‌نظیرم
\\
اندر تنور سردان آتش زنم چو مردان
&&
و اندر تنور گرمان من پخته‌تر خمیرم
\\
در لطف همچو شیرم اندر گلو نگیرم
&&
تا در غلط نیفتی گر شور چون پنیرم
\\
در عشق شمس تبریز سلطان تاجدارم
&&
چون او به تخت آید من پیش او وزیرم
\\
\end{longtable}
\end{center}
