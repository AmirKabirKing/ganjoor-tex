\begin{center}
\section*{غزل شماره ۹۷۹: شاهدی بین که در زمانه بزاد}
\label{sec:0979}
\addcontentsline{toc}{section}{\nameref{sec:0979}}
\begin{longtable}{l p{0.5cm} r}
شاهدی بین که در زمانه بزاد
&&
بت و بتخانه را به باد بداد
\\
شاهدانی که در جهان سمرند
&&
کس از ایشان دگر نیارد یاد
\\
از رخ ماه او چو ابر گشود
&&
هفت گردون ز همدگر بگشاد
\\
همچو مهتاب شاخ شاخ آن نور
&&
سوی هر روزنی درون افتاد
\\
تابشش چون بتافت بیشترک
&&
جان‌ها را بخورد از بنیاد
\\
جان‌ها ذره ذره رقصان گشت
&&
پیش خورشید جان‌ها دلشاد
\\
همچو پرواز شمس تبریزی
&&
جمله پران که هر چه بادا باد
\\
\end{longtable}
\end{center}
