\begin{center}
\section*{غزل شماره ۳۵۹: در این جو دل چو دولاب خرابست}
\label{sec:0359}
\addcontentsline{toc}{section}{\nameref{sec:0359}}
\begin{longtable}{l p{0.5cm} r}
در این جو دل چو دولاب خرابست
&&
که هر سویی که گردد پیشش آبست
\\
وگر تو پشت سوی آب داری
&&
به پیش روت آب اندر شتابست
\\
چگونه جان برد سایه ز خورشید
&&
که جان او به دست آفتابست
\\
اگر سایه کند گردن درازی
&&
رخ خورشید آن دم در نقابست
\\
زهی خورشید کاین خورشید پیشش
&&
چو سیماب از خطر در اضطرابست
\\
چو سیماب‌ست مه بر کف مفلوج
&&
بجز یک شب دگر در انسکابست
\\
به هر سی شب دو شب جمع‌ست و لاغر
&&
دگر فرقت کشد فرقت عذابست
\\
اگر چه زار گردد تازه روی‌ست
&&
ضحوکی عاشقان را خوی و دابست
\\
زید خندان بمیرد نیز خندان
&&
که سوی بخت خندانش ایابست
\\
خمش کن زانک آفات بصیرت
&&
همیشه از سؤالست و جوابست
\\
\end{longtable}
\end{center}
