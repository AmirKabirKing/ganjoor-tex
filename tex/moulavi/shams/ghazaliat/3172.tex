\begin{center}
\section*{غزل شماره ۳۱۷۲: ای که ازین تنگ قفص می‌پری}
\label{sec:3172}
\addcontentsline{toc}{section}{\nameref{sec:3172}}
\begin{longtable}{l p{0.5cm} r}
ای که ازین تنگ قفس می‌پری
&&
رخت به بالای فلک می‌بری
\\
زندگی تازه ببین بعد ازین
&&
چند ازین زندگی سرسری؟!
\\
در هوس مشتریت عمر رفت
&&
ماه ببین و بره از مشتری
\\
دلق شپشناک درانداختی
&&
جان برهنه شده خود خوشتری
\\
در عوض دلق تن چار میخ
&&
بافته‌اند از صفتت ششتری
\\
جامهٔ این جسم، غلامانه بود
&&
گیر کنون پیرهن مهتری
\\
مرگ حیاتست و حیاتست مرگ
&&
عکس نماید نظر کافری
\\
جملهٔ جانها که ازین تن شدند
&&
حی و نهانند کنون چون پری
\\
گشت سوار فرس غیب، جان
&&
باز رهید از خر و از خرخری
\\
سوخت درین آخر دنیا دلت
&&
بهر وجوه جو این لاغری
\\
پرده چو برخاست اگر این خرت
&&
گردد زرین، تو درو ننگری
\\
بر سر دریاست چو کشتی روان
&&
روح، که بود از تن خود لنگری
\\
گر چه جدا گشت ز دست و ز پا
&&
فضل حقش داد پر جعفری
\\
خانهٔ تن گر شکند، هین منال
&&
خواجه! یقین دان که به زندان دری
\\
چونک ز زندان و چه آیی برون
&&
یوسف مصری و شه و سروری
\\
چون برهی از چه و از آب شور
&&
ماهیی و معتکف کوثری
\\
باقی این را تو بگو، زانک خلق
&&
از تو کنند ای شه من، باوری
\\
\end{longtable}
\end{center}
