\begin{center}
\section*{غزل شماره ۳۱۳۳: حدی نداری در خوش لقایی}
\label{sec:3133}
\addcontentsline{toc}{section}{\nameref{sec:3133}}
\begin{longtable}{l p{0.5cm} r}
حدی نداری در خوش لقایی
&&
مثلی نداری در جان فزایی
\\
بر وعده تو بر نجده تو
&&
که م دوش گفتی هی تو کجایی
\\
کردم کرانه ز اهل زمانه
&&
رفتم به خانه تا تو بیایی
\\
نزلت چشیدم رویت ندیدم
&&
آن قرص مه را کی می‌نمایی
\\
ماهی کمالی آب زلالی
&&
جاه و جلالی کان عطایی
\\
امروز مستم مجنون پرستم
&&
بگرفت دستم دست خدایی
\\
ای ساقی شه هین الله الله
&&
افزون ده آن می چون مرتضایی
\\
یک گوشه جان ماندست پیچان
&&
و آن پیچش از تو یابد رهایی
\\
جنگ است نیمم با نیم دیگر
&&
هین صلح شان ده تا چند پایی
\\
زاغی و بازی در یک قفس شد
&&
و از زخم هر دو در ابتلایی
\\
بگشا قفس را تا ره شودشان
&&
جنگی نماند چون در گشایی
\\
نفسی و عقلی در سینه ما
&&
در جنگ و محنت مست خدایی
\\
گر جنگ خواهی درشان فروبند
&&
ور نی بکن شان یک دم سقایی
\\
در آب افکن چون مهد موسی
&&
این جان ما را چون جان مایی
\\
تا کش نیاید فرعون ملعون
&&
نی آن عوانان اندر دغایی
\\
در آب رقصان مهد لطیفش
&&
از خوف رسته وز بی‌نوایی
\\
فرعون اکنون بشناسد او را
&&
کز راه آب او کرد ارتقایی
\\
تو میر آبی و آن آب قایم
&&
داد و دهش را دایم سزایی
\\
در خانه موسی در خوف جان بد
&&
در آب بودش امن بقایی
\\
هر چیز زنده از آب باشد
&&
کب است ما را نقل سمایی
\\
تو آب آبی تو تاب تابی
&&
آب از تو یابد لطف و روایی
\\
قارون نعمت طماع گردد
&&
در بخشش تو گیرد گدایی
\\
جز در گدایی کس این نیابد
&&
ناموس کم کن با کبریایی
\\
گیرنده خواهد جوینده خواهد
&&
ناموس آرد جان را جدایی
\\
خاموش کردم لیکن روانم
&&
در اندرونم گشته‌ست نایی
\\
\end{longtable}
\end{center}
