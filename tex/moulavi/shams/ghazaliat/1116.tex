\begin{center}
\section*{غزل شماره ۱۱۱۶: هر کس به جنس خویش درآمیخت ای نگار}
\label{sec:1116}
\addcontentsline{toc}{section}{\nameref{sec:1116}}
\begin{longtable}{l p{0.5cm} r}
هر کس به جنس خویش درآمیخت ای نگار
&&
هر کس به لایق گهر خود گرفت یار
\\
او را که داغ توست نیارد کسی خرید
&&
آن کو شکار توست کسی چون کند شکار
\\
ما را چو لطف روی تو بی‌خویشتن کند
&&
ما را ز روی لطف تو بی‌خویشتن مدار
\\
چون جنس همدگر بگرفتند جنس جنس
&&
هر جنس جنس گوهر خود کرد اختیار
\\
با غیر جنس اگر بنشیند بود نفاق
&&
مانند آب و روغن و مانند قیر و قار
\\
تا چون به جنس خویش رود از خلاف جنس
&&
زین سوی تشنه‌تر شده باشد بدان کنار
\\
هرکه از تو می‌گریزد با دیگری خوشست
&&
و آنک از تو می‌رمد به کسی دارد او قرار
\\
و آن کو ترش نشست به پیش تو همچو ابر
&&
خندان دلست پیش دگر کس چو نوبهار
\\
گویی که نیست از مه غیبم به جز دریغ
&&
وز جام و خمر روح مرا نیست جز خمار
\\
آن نای و نوش یاد نمی‌آیدت که تو
&&
خوش می‌خوری ز دست یکی دیو سنگسار
\\
صد جام درکشی ز کف دیو آنگهی
&&
بینی ترش کنی بخور ای خام پخته خوار
\\
این جا سرک فکنده و رویک ترش ولیک
&&
آن جا چو اژدهای سیه فام کوهسار
\\
با جنس همچو سوسن و با غیر جنس گنگ
&&
با جنس خویش چون گل و با غیر جنس خار
\\
رو رو به جمله خلق نتانی تو جنس بود
&&
شاخی ز صد درخت نشد حامل ثمار
\\
چون شاخ یک درخت شدی زان دگر ببر
&&
جویای وصل این شده‌ای دست از آن بدار
\\
گر زانک جنس مفخر تبریز گشت جان
&&
احسنت ای ولایت و شاباش کار و بار
\\
\end{longtable}
\end{center}
