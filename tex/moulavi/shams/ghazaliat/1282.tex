\begin{center}
\section*{غزل شماره ۱۲۸۲: تمام اوست که فانی شدست آثارش}
\label{sec:1282}
\addcontentsline{toc}{section}{\nameref{sec:1282}}
\begin{longtable}{l p{0.5cm} r}
تمام اوست که فانی شدست آثارش
&&
به دوستگانی اول تمام شد کارش
\\
مرا دلیست خراب خراب در ره عشق
&&
خراب کرده خراباتیی به یک بارش
\\
بگو به عشق بیا گر فتاده می‌خواهی
&&
چنان فتاد که خواهی بیا و بردارش
\\
میا به پیش ز درش ببین که می‌ترسم
&&
ز شعله‌ها که بسوزی ز سوز اسرارش
\\
وگر بگیردت آتش به سوی چشم من آ
&&
که سیل سیل روانست اشک دربارش
\\
حدیث موسی و سنگ و عصا و چشمه آب
&&
ز اشک بنده ببینی به وقت رفتارش
\\
برآر بانگ و بگو هر کجا که بیماریست
&&
صلای صحت و دولت ز چشم بیمارش
\\
برآ به کوه و بگو هر کجا که خفته دلیست
&&
صلای بینش و دانش ز بخت بیدارش
\\
که نور من شرح الله صدره شمعیست
&&
که در دو کون نگنجد فروغ انوارش
\\
\end{longtable}
\end{center}
