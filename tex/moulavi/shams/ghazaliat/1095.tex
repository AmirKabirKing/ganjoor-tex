\begin{center}
\section*{غزل شماره ۱۰۹۵: داد جاروبی به دستم آن نگار}
\label{sec:1095}
\addcontentsline{toc}{section}{\nameref{sec:1095}}
\begin{longtable}{l p{0.5cm} r}
داد جاروبی به دستم آن نگار
&&
گفت کز دریا برانگیزان غبار
\\
باز آن جاروب را ز آتش بسوخت
&&
گفت کز آتش تو جاروبی برآر
\\
کردم از حیرت سجودی پیش او
&&
گفت بی‌ساجد سجودی خوش بیار
\\
آه بی‌ساجد سجودی چون بود
&&
گفت بی‌چون باشد و بی‌خارخار
\\
گردنک را پیش کردم گفتمش
&&
ساجدی را سر ببر از ذوالفقار
\\
تیغ تا او بیش زد سر بیش شد
&&
تا برست از گردنم سر صد هزار
\\
من چراغ و هر سرم همچون فتیل
&&
هر طرف اندر گرفته از شرار
\\
شمع‌ها می‌ورشد از سرهای من
&&
شرق تا مغرب گرفته از قطار
\\
شرق و مغرب چیست اندر لامکان
&&
گلخنی تاریک و حمامی به کار
\\
ای مزاجت سرد کو تاسه دلت
&&
اندر این گرمابه تا کی این قرار
\\
برشو از گرمابه و گلخن مرو
&&
جامه کن دربنگر آن نقش و نگار
\\
تا ببینی نقش‌های دلربا
&&
تا ببینی رنگ‌های لاله زار
\\
چون بدیدی سوی روزن درنگر
&&
کان نگار از عکس روزن شد نگار
\\
شش جهت حمام و روزن لامکان
&&
بر سر روزن جمال شهریار
\\
خاک و آب از عکس او رنگین شده
&&
جان بباریده به ترک و زنگبار
\\
روز رفت و قصه‌ام کوته نشد
&&
ای شب و روز از حدیثش شرمسار
\\
شاه شمس الدین تبریزی مرا
&&
مست می‌دارد خمار اندر خمار
\\
\end{longtable}
\end{center}
