\begin{center}
\section*{غزل شماره ۵۵۳: بی همگان به سر شود بی‌تو به سر نمی‌شود}
\label{sec:0553}
\addcontentsline{toc}{section}{\nameref{sec:0553}}
\begin{longtable}{l p{0.5cm} r}
بی همگان به سر شود بی‌تو به سر نمی‌شود
&&
داغ تو دارد این دلم جای دگر نمی‌شود
\\
دیده عقل مست تو چرخه چرخ پست تو
&&
گوش طرب به دست تو بی‌تو به سر نمی‌شود
\\
جان ز تو جوش می‌کند دل ز تو نوش می‌کند
&&
عقل خروش می‌کند بی‌تو به سر نمی‌شود
\\
خمر من و خمار من باغ من و بهار من
&&
خواب من و قرار من بی‌تو به سر نمی‌شود
\\
جاه و جلال من تویی ملکت و مال من تویی
&&
آب زلال من تویی بی‌تو به سر نمی‌شود
\\
گاه سوی وفا روی گاه سوی جفا روی
&&
آن منی کجا روی بی‌تو به سر نمی‌شود
\\
دل بنهند برکنی توبه کنند بشکنی
&&
این همه خود تو می‌کنی بی‌تو به سر نمی‌شود
\\
بی تو اگر به سر شدی زیر جهان زبر شدی
&&
باغ ارم سقر شدی بی‌تو به سر نمی‌شود
\\
گر تو سری قدم شوم ور تو کفی علم شوم
&&
ور بروی عدم شوم بی‌تو به سر نمی‌شود
\\
خواب مرا ببسته‌ای نقش مرا بشسته‌ای
&&
وز همه‌ام گسسته‌ای بی‌تو به سر نمی‌شود
\\
گر تو نباشی یار من گشت خراب کار من
&&
مونس و غمگسار من بی‌تو به سر نمی‌شود
\\
بی تو نه زندگی خوشم بی‌تو نه مردگی خوشم
&&
سر ز غم تو چون کشم بی‌تو به سر نمی‌شود
\\
هر چه بگویم ای سند نیست جدا ز نیک و بد
&&
هم تو بگو به لطف خود بی‌تو به سر نمی‌شود
\\
\end{longtable}
\end{center}
