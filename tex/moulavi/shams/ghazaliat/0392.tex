\begin{center}
\section*{غزل شماره ۳۹۲: گر ندید آن شادجان این گلستان را شاد چیست}
\label{sec:0392}
\addcontentsline{toc}{section}{\nameref{sec:0392}}
\begin{longtable}{l p{0.5cm} r}
گر ندید آن شادجان این گلستان را شاد چیست
&&
گر نه لطف او بود پس عیش را بنیاد چیست
\\
گر خرابات ازل از تاب رویش پر نگشت
&&
پس هزاران صومعه در محو جان آباد چیست
\\
جان ما با عشق او گر نی ز یک جا رسته‌اند
&&
جان بااقبال ما با عشق او همزاد چیست
\\
گر نه پرتوهای آن رخسار داد حسن داد
&&
پس به دیوان سرای عاشقان بیداد چیست
\\
ساکنان آب و گل گر عشق ما را محرمند
&&
پس درون گنبد دل غلغله و فریاد چیست
\\
گر نه آتش می‌زند آتش رخی در جان نهان
&&
پس دماغ عاشقان پرآتش و پرباد چیست
\\
گر نه آتش رنگ گشتی جان‌ها در لامکان
&&
صد هزاران مشعله همچون شب میلاد چیست
\\
گر نه تقصیر است از جان در فدا گشتن در او
&&
لطف نقد اولین و وعده و میعاد چیست
\\
گر نه شمس الدین تبریزی قباد جان‌ها است
&&
صد هزاران جان قدسی هر دمش منقاد چیست
\\
\end{longtable}
\end{center}
