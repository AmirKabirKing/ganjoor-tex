\begin{center}
\section*{غزل شماره ۱۲۲۳: قرین مه دو مریخند و آن دو چشمت ای دلکش}
\label{sec:1223}
\addcontentsline{toc}{section}{\nameref{sec:1223}}
\begin{longtable}{l p{0.5cm} r}
قرین مه دو مریخند و آن دو چشمت ای دلکش
&&
بدان هاروت و ماروتت لجوجان را به بابل کش
\\
سلیمانا بدان خاتم که ختم جمله خوبانی
&&
همه دیوان و پریان را به قهر اندر سلاسل کش
\\
برای جن و انسان را گشادی گنج احسان را
&&
مثال نحن اعطیناک بر محروم سائل کش
\\
جسد را کن به جان روشن حسد را بیخ و بن برکن
&&
نظر را بر مشارق زن خرد را در مسائل کش
\\
چو لب الحمد برخواند دهش نقل و می بی‌حد
&&
چو برخواند و لا الضالین تو او را در دلایل کش
\\
سوی تو جان چو بشتابد دهش شمعی که ره یابد
&&
چو خورشید تو را جوید چو ماهش در منازل کش
\\
شراب کاس کیکاووس ده مخمور عاشق را
&&
دقیقه دانی و فن را به پیش فکر عاقل کش
\\
به اقبال عنایاتت بکش جان را و قابل کن
&&
قبول و خلعت خود را به سوی نفس قابل کش
\\
اسیر درد و حسرت را بده پیغام لاتأسوا
&&
قتول عشق حسنت را از این مقتل به قاتل کش
\\
اگر کافردلست این تن شهادت عرضه کن بر وی
&&
وگر بی‌حاصلست این جان چه باشد توش به حاصل کش
\\
کنش زنده وگر نکنی مسیحا را تو نایب کن
&&
تو وصلش ده وگر ندهی به فضلش سوی فاضل کش
\\
زمین لرزید ای خاکی چو دید آن قدس و آن پاکی
&&
اذا ما زلزلت برخوان نظر را در زلازل کش
\\
تمامش کن هلا حالی که شاه حالی و قالی
&&
کسی که قول پیش آرد خطی بر قول و قایل کش
\\
\end{longtable}
\end{center}
