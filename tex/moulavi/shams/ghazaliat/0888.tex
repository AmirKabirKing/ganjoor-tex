\begin{center}
\section*{غزل شماره ۸۸۸: زهره من بر فلک شکل دگر می‌رود}
\label{sec:0888}
\addcontentsline{toc}{section}{\nameref{sec:0888}}
\begin{longtable}{l p{0.5cm} r}
زهره من بر فلک شکل دگر می‌رود
&&
در دل و در دیده‌ها همچو نظر می‌رود
\\
چشم چو مریخ او مست ز تاریخ او
&&
جان به سوی ناوکش همچو سپر می‌رود
\\
ابروی چون سنبله بی‌خبرست از مهش
&&
گر خبرستش چرا فوق قمر می‌رود
\\
ذره چرا شد سوار بر سر کره هوا
&&
چون سوی تو آفتاب جمله به سر می‌رود
\\
آن زحل از ابلهی جست زبردستیی
&&
غافل از آن کاین فلک زیر و زبر می‌رود
\\
دل ز شب زلف تو دید رخ همچو روز
&&
زین شب و روز او نهان همچو سحر می‌رود
\\
ترک فلک گاو را بر سر گردون ببست
&&
کرد ندا در جهان کی به سفر می‌رود
\\
جامه کبود آسمان کرد ز دست قضا
&&
این قدرش فهم نی کو به قدر می‌رود
\\
خاک دهان خشک را رعد بشارت دهد
&&
کابر چو مشک سقا بهر مطر می‌رود
\\
اختر و ابر و فلک جنی و دیو و ملک
&&
آخر ای بی‌یقین بهر بشر می‌رود
\\
پنبه برون کن ز گوش عقل و بصر را مپوش
&&
کان صنم حله پوش سوی بصر می‌رود
\\
نای و دف و چنگ را از پی گوشی زنند
&&
نقش جهان جانب نقش نگر می‌رود
\\
آن نظری جو که آن هست ز نور قدیم
&&
کاین نظر ناریت همچو شرر می‌رود
\\
جنس رود سوی جنس بس بود این امتحان
&&
شه سوی شه می‌رود خر سوی خر می‌رود
\\
هر چه نهال ترست جانب بستان برند
&&
خشک چو هیزم شود زیر تبر می‌رود
\\
آب معانی بخور هر دم چون شاخ تر
&&
شکر که در باغ عشق جوی شکر می‌رود
\\
بس کن از این امر و نهی بین که تو نفس حرون
&&
چونش بگویی مرو لنگ بتر می‌رود
\\
جان سوی تبریز شد در هوس شمس دین
&&
جان صدفست و سوی بحر گهر می‌رود
\\
\end{longtable}
\end{center}
