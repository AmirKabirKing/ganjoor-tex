\begin{center}
\section*{غزل شماره ۱۲۵۴: من توام تو منی ای دوست مرو از بر خویش}
\label{sec:1254}
\addcontentsline{toc}{section}{\nameref{sec:1254}}
\begin{longtable}{l p{0.5cm} r}
من توام تو منی ای دوست مرو از بر خویش
&&
خویش را غیر مینگار و مران از در خویش
\\
سر و پا گم مکن از فتنه بی‌پایانت
&&
تا چو حیران بزنم پای جفا بر سر خویش
\\
آن که چون سایه ز شخص تو جدا نیست منم
&&
مکش ای دوست تو بر سایه خود خنجر خویش
\\
ای درختی که به هر سوت هزاران سایه‌ست
&&
سایه‌ها را بنواز و مبر از گوهر خویش
\\
سایه‌ها را همه پنهان کن و فانی در نور
&&
برگشا طلعت خورشیدرخ انور خویش
\\
ملک دل از دودلی تو مخبط گشتست
&&
بر سر تخت برآ پا مکش از منبر خویش
\\
عقل تاجست چنین گفت به تثمیل علی
&&
تاج را گوهر نو بخش تو از گوهر خویش
\\
\end{longtable}
\end{center}
