\begin{center}
\section*{غزل شماره ۷۰۳: آن کس که ز جان خود نترسد}
\label{sec:0703}
\addcontentsline{toc}{section}{\nameref{sec:0703}}
\begin{longtable}{l p{0.5cm} r}
آن کس که ز جان خود نترسد
&&
از کشتن نیک و بد نترسد
\\
وان کس که بدید حسن یوسف
&&
از حاسد و از حسد نترسد
\\
آن کس که هوای شاه دارد
&&
از لشکر بی‌عدد نترسد
\\
آخر حیوان ز ذوق صحبت
&&
از جفته و از لگد نترسد
\\
آن کس که سعادت ازل دید
&&
از عاقبت ابد نترسد
\\
چون کوه احد دلی بباید
&&
تا او ز جز احد نترسد
\\
مرغی که ز دام نفس خود رست
&&
هر جای که برپرد نترسد
\\
هر جای که هست گنج گنجست
&&
کشته احد از لحد نترسد
\\
هر جانوری کز اصل آبست
&&
گر غرقه شود عمد نترسد
\\
هر تن که سرشته بهشتست
&&
بر دوزخ برزند نترسد
\\
وان را که مدد از اندرونست
&&
زین عالم بی‌مدد نترسد
\\
از ابلهیست نی شجاعت
&&
گر جاهل از خرد نترسد
\\
خود سر نبدست آن خسی را
&&
کز عشق تو پا کشد نترسد
\\
این مایه لعنتست کابله
&&
دل‌های شهان خلد نترسد
\\
هم پرده خویش می‌درد کو
&&
پرده من و تو درد نترسد
\\
پازهر چو نیستش چرا او
&&
زهر دنیا خورد نترسد
\\
در حضرت آن چنان رقیبی
&&
در شاهد بنگرد نترسد
\\
زنهار به سر برو بدان ره
&&
کان جا دلت از رصد نترسد
\\
صراف کمین درست و آن دزد
&&
از کیسه درم برد نترسد
\\
آن جا گرگان همه شبانند
&&
آن جا مردی ز صد نترسد
\\
آن جا من و تو و او نباشد
&&
چون وام ز خود ستد نترسد
\\
هرگز دل تو ز تو نرنجد
&&
هرگز ذقنت ز خد نترسد
\\
گلشن ز بهار و باغ سوسن
&&
وز سرو لطیف قد نترسد
\\
چون گل بشکفت و روی خود دید
&&
زان پس ز قبول و رد نترسد
\\
بس کن هر چند تا قیامت
&&
این بحر گهر دهد نترسد
\\
\end{longtable}
\end{center}
