\begin{center}
\section*{غزل شماره ۲۳۸۷: پیغام زاهدان را کمد بلای توبه}
\label{sec:2387}
\addcontentsline{toc}{section}{\nameref{sec:2387}}
\begin{longtable}{l p{0.5cm} r}
پیغام زاهدان را کآمد بلای توبه
&&
با آن جمال و خوبی آخر چه جای توبه
\\
هم زهد برشکسته هم توبه توبه کرده
&&
چون هست عاشقان را کاری ورای توبه
\\
چون از جهان رمیدی در نور جان رسیدی
&&
چون شمع سر بریدی بشکن تو پای توبه
\\
شرط است بی‌قراری با آهوی تتاری
&&
ترک خطا چو آمد ای بس خطای توبه
\\
در صید چون درآید بس جان که او رباید
&&
یک تیر غمزه او صد خونبهای توبه
\\
چون هر سحر خیالش بر عاشقان بتازد
&&
گرد غبار اسبش صد توتیای توبه
\\
از باده لب او مخمور گشته جان‌ها
&&
و آن چشم پرخمارش داده سزای توبه
\\
تا باغ عاشقان را سرسبز و تازه کردی
&&
حسنت خراب کرده بام و سرای توبه
\\
ای توبه برگشاده بی‌شمس حق تبریز
&&
روزی که ره نماید ای وای وای توبه
\\
\end{longtable}
\end{center}
