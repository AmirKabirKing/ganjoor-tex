\begin{center}
\section*{غزل شماره ۲۱۵۰: عید نمی‌دهد فرح بی‌نظر هلال تو}
\label{sec:2150}
\addcontentsline{toc}{section}{\nameref{sec:2150}}
\begin{longtable}{l p{0.5cm} r}
عید نمی‌دهد فرح بی‌نظر هلال تو
&&
کوس و دهل نمی‌چخد بی‌شرف دوال تو
\\
من به تو مایل و تویی هر نفسی ملولتر
&&
وه که خجل نمی‌شود میل من از ملال تو
\\
ناز کن ای حیات جان کبر کن و بکش عنان
&&
شمس و قمر دلیل تو شهد و شکر دلال تو
\\
آیت هر ملاحتی ماه تو خواند بر جهان
&&
مایه هر خجستگی ماه تو است و سال تو
\\
آب زلال ملک تو باغ و نهال ملک تو
&&
جز ز زلال صافیت می‌نخورد نهال تو
\\
ملک تو است تخت‌ها باغ و سرا و رخت‌ها
&&
رقص کند درخت‌ها چونک رسد شمال تو
\\
مطبخ توست آسمان مطبخیانت اختران
&&
آتش و آب ملک تو خلق همه عیال تو
\\
عشق کمینه نام تو چرخ کمینه بام تو
&&
رونق آفتاب‌ها از مه بی‌زوال تو
\\
خشک لبند عالمی از لمع سراب تو
&&
لطف سراب این بود تا چه بود زلال تو
\\
ای ز خیال‌های تو گشته خیال عاشقان
&&
خیل خیال این بود تا چه بود جمال تو
\\
وصل کنی درخت را حالت او بدل شود
&&
چون نشود مها بدل جان و دل از وصال تو
\\
زهر بود شکر شود سنگ بود گهر شود
&&
شام بود سحر شود از کرم خصال تو
\\
بس سخن است در دلم بسته‌ام و نمی‌هلم
&&
گوش گشاده‌ام که تا نوش کنم مقال تو
\\
\end{longtable}
\end{center}
