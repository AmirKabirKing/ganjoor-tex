\begin{center}
\section*{غزل شماره ۱۷۱۰: با روی تو ز سبزه و گلزار فارغیم}
\label{sec:1710}
\addcontentsline{toc}{section}{\nameref{sec:1710}}
\begin{longtable}{l p{0.5cm} r}
با روی تو ز سبزه و گلزار فارغیم
&&
با چشم تو ز باده و خمار فارغیم
\\
خانه گرو نهاده و در کوی تو مقیم
&&
دکان خراب کرده و از کار فارغیم
\\
رختی که داشتیم به یغما ببرد عشق
&&
از سود و از زیان و ز بازار فارغیم
\\
دعوی عشق وانگه ناموس و نام و ننگ
&&
ما ننگ را خریده و از عار فارغیم
\\
غم را چه زهره باشد تا نام ما برد
&&
دستی بزن که از غم و غمخواره فارغیم
\\
ای روترش که کاله گران است چون خرم
&&
بگذر مخر که ما ز خریدار فارغیم
\\
ما را مسلم آمد شادی و خوشدلی
&&
کز باد و بود اندک و بسیار فارغیم
\\
بررفت و برگذشت سر ما ز آسمان
&&
کز ذوق عشق از سر و دستار فارغیم
\\
ما لاف می زنیم و تو انکار می کنی
&&
ز اقرار هر دو عالم و ز انکار فارغیم
\\
مشتی سگان نگر که به هم درفتاده‌اند
&&
ما سگ نزاده‌ایم و ز مردار فارغیم
\\
اسرار تو خدای همی‌داند و بس است
&&
ما از دغا و حیلت و مکار فارغیم
\\
درسی که عشق داد فراموش کی شود
&&
از بحث و از جدال و ز تکرار فارغیم
\\
پنهان تو هر چه کاری پیدا بروید آن
&&
هر تخم را که خواهی می کار فارغیم
\\
آهن ربای جذب رفیقان کشید حرف
&&
ور نی در این طریق ز گفتار فارغیم
\\
با نور روی مفخر تبریز شمس دین
&&
از شمس چرخ گنبد دوار فارغیم
\\
\end{longtable}
\end{center}
