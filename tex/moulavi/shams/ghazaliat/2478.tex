\begin{center}
\section*{غزل شماره ۲۴۷۸: باز چه شد تو را دلا باز چه مکر اندری}
\label{sec:2478}
\addcontentsline{toc}{section}{\nameref{sec:2478}}
\begin{longtable}{l p{0.5cm} r}
باز چه شد تو را دلا باز چه مکر اندری
&&
یک نفسی چو بازی و یک نفسی کبوتری
\\
همچو دعای صالحان دی سوی اوج می‌شدی
&&
باز چو نور اختران سوی حضیض می‌پری
\\
کشت مرا به جان تو حیله و داستان تو
&&
سیل تو می‌کشد مرا تا به کجام می‌بری
\\
از رحموت گشته‌ای در رهبوت رفته‌ای
&&
تا دم مهر نشنوی تا سوی دوست ننگری
\\
گر سبکی کند دلم خنده زنی که هین بپر
&&
چونک به خود فروروم طعنه زنی که لنگری
\\
خنده کنم تو گوییم چون سر پخته خنده زن
&&
گریه کنم تو گوییم چون بن کوزه می‌گری
\\
ترک تویی ز هندوان چهره ترک کم طلب
&&
ز آنک نداد هند را صورت ترک تنگری
\\
خنده نصیب ماه شد گریه نصیب ابر شد
&&
بخت بداد خاک را تابش زر جعفری
\\
حسن ز دلبران طلب درد ز عاشقان طلب
&&
چهره زرد جو ز من وز رخ خویش احمری
\\
من چو کمینه بنده‌ام خاک شوم ستم کشم
&&
تو ملکی و زیبدت سرکشی و ستمگری
\\
مست و خوشم کن آنگهی رقص و خوشی طلب ز من
&&
در دهنم بنه شکر چون ترشی نمی‌خوری
\\
دیگ توام خوشی دهم چونک ابای خوش پزی
&&
ور ترشی پزی ز من هم ترشی برآوری
\\
دیو شود فرشته‌ای چون نگری در او تو خوش
&&
ای پرییی که از رخت بوی نمی‌برد پری
\\
سحر چرا حرام شد ز آنک به عهد حسن تو
&&
حیف بود که هر خسی لاف زند ز ساحری
\\
ای دل چون عتاب و غم هست نشان مهر او
&&
ترک عتاب اگر کند دانک بود ز تو بری
\\
ای تبریز شمس دین خسرو شمس مشرقت
&&
پرتو نور آن سری عاریتی است ای سری
\\
\end{longtable}
\end{center}
