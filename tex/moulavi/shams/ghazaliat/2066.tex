\begin{center}
\section*{غزل شماره ۲۰۶۶: باز برآمد ز کوه خسرو شیرین من}
\label{sec:2066}
\addcontentsline{toc}{section}{\nameref{sec:2066}}
\begin{longtable}{l p{0.5cm} r}
باز برآمد ز کوه خسرو شیرین من
&&
باز مرا یاد کرد جان و دل و دین من
\\
سوره یاسین بسی خواندم از عشق و ذوق
&&
زان که مرا خوانده بود سوره یاسین من
\\
عقل همه عاقلان خبره شود چون رسد
&&
لیلی و مجنون من ویسه و رامین من
\\
در حسد افتاده‌ایم دل به جفا داده‌ایم
&&
جنگ که می‌افکند یار سخن چین من
\\
او نگذارد که خلق صلح کنند و وفا
&&
تازه کند دم به دم کین تو و کین من
\\
گوید کای عاشقان رحم میارید هیچ
&&
در کشش همدگر از پی آیین من
\\
یا رب و آمین بسی کردم و جستم امان
&&
آه که می‌نشنود یارب و آمین من
\\
گوید تو کار خویش می‌کن و من کار خویش
&&
این بده‌ست از ازل یاسه پیشین من
\\
کار من آن کت زنم کار تو افغان گری
&&
عید منم طبل تو سخره تکوین من
\\
بنده این زاریم عاشق بیماریم
&&
کو نرود آن زمان از سر بالین من
\\
راست رود سوی شه جان و دلم همچو رخ
&&
گر چه کند کژروی طبع چو فرزین من
\\
درگذر از تنگ من ای من من ننگ من
&&
دیده شدی آن من گر نبدی این من
\\
بس کن ای شهسوار کز حجب گفت تو
&&
نقد عجب می‌برد دزد ز خرجین من
\\
\end{longtable}
\end{center}
