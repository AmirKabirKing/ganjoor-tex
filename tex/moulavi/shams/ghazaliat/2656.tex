\begin{center}
\section*{غزل شماره ۲۶۵۶: نه آتش‌های ما را ترجمانی}
\label{sec:2656}
\addcontentsline{toc}{section}{\nameref{sec:2656}}
\begin{longtable}{l p{0.5cm} r}
نه آتش‌های ما را ترجمانی
&&
نه اسرار دل ما را زبانی
\\
برهنه شد ز صد پرده دل و عشق
&&
نشسته دو به دو جانی و جانی
\\
میان هر دو گر جبریل آید
&&
نباشد ز آتشش یک دم امانی
\\
به هر لحظه وصال اندر وصالی
&&
به هر سویی عیان اندر عیانی
\\
ببینی تو چه سلطانان معنی
&&
به گوشه بامشان چون پاسبانی
\\
سرشته وصل یزدان کوه طور است
&&
در آن کان تاب نارد یک زمانی
\\
اگر صد عقل کل بر هم ببندی
&&
نگردد بامشان را نردبانی
\\
نشانی‌های مردان سجده آرد
&&
اگر زان بی‌نشان گویم نشانی
\\
از آن نوری که حرف آن جا نگنجد
&&
تو را این حرف گشته ارمغانی
\\
کمر شد حرف‌ها از شمس تبریز
&&
بیا بربند اگر داری میانی
\\
\end{longtable}
\end{center}
