\begin{center}
\section*{غزل شماره ۲۷۷۷: شاد آن صبحی که جان را چاره آموزی کنی}
\label{sec:2777}
\addcontentsline{toc}{section}{\nameref{sec:2777}}
\begin{longtable}{l p{0.5cm} r}
شاد آن صبحی که جان را چاره آموزی کنی
&&
چاره او یابد که تش بیچارگی روزی کنی
\\
عشق جامه می‌دراند عقل بخیه می‌زند
&&
هر دو را زهره بدرد چون تو دلدوزی کنی
\\
خوش بسوزم همچو عود و نیست گردم همچو دود
&&
خوشتر از سوزش چه باشد چون تو دلسوزی کنی
\\
گه لباس قهر درپوشی و راه دل زنی
&&
گه بگردانی لباس آیی قلاوزی کنی
\\
خوش بچر ای گاو عنبربخش نفس مطمئن
&&
در چنین ساحل حلال است ار تو خوش پوزی کنی
\\
طوطیی که طمع اسب و مرکب تازی کنی
&&
ماهیی که میل شعر و جامه توزی کنی
\\
شیر مستی و شکارت آهوان شیرمست
&&
با پنیر گنده فانی کجا یوزی کنی
\\
چند گویم قبله کامشب هر یکی را قبله‌ای است
&&
قبله‌ها گردد یکی گر تو شب افروزی کنی
\\
گر ز لعل شمس تبریزی بیابی مایه‌ای
&&
کمترین پایه فراز چرخ پیروزی کنی
\\
\end{longtable}
\end{center}
