\begin{center}
\section*{غزل شماره ۱۰۲: سلیمانا بیار انگشتری را}
\label{sec:0102}
\addcontentsline{toc}{section}{\nameref{sec:0102}}
\begin{longtable}{l p{0.5cm} r}
سلیمانا بیار انگشتری را
&&
مطیع و بنده کن دیو و پری را
\\
برآر آواز ردوها علی
&&
منور کن سرای شش دری را
\\
برآوردن ز مغرب آفتابی
&&
مسلم شد ضمیر آن سری را
\\
بدین سان مهتری یابد هر آن کس
&&
که بهر حق گذارد مهتری را
\\
بنه بر خوان جفان کالجوابی
&&
مکرم کن نیاز مشتری را
\\
به کاسی کاسه سر را طرب ده
&&
تو کن مخمور چشم عبهری را
\\
ز صورت‌های غیبی پرده بردار
&&
کسادی ده نقوش آزری را
\\
ز چاه و آب چه رنجور گشتیم
&&
روان کن چشمه‌های کوثری را
\\
دلا در بزم شاهنشاه دررو
&&
پذیرا شو شراب احمری را
\\
زر و زن را به جان مپرست زیرا
&&
بر این دو دوخت یزدان کافری را
\\
جهاد نفس کن زیرا که اجری
&&
برای این دهد شه لشکری را
\\
دل سیمین بری کز عشق رویش
&&
ز حیرت گم کند زر هم زری را
\\
بدان دریادلی کز جوش و نوشش
&&
به دست آورد گوهر گوهری را
\\
که باقی غزل را تو بگویی
&&
به رشک آری تو سحر سامری را
\\
خمش کردم که پایم گل فرورفت
&&
تو بگشا پر نطق جعفری را
\\
\end{longtable}
\end{center}
