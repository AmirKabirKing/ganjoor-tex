\begin{center}
\section*{غزل شماره ۱۵۸۲: هرچ گویی از بهانه لا نسلم لا نسلم}
\label{sec:1582}
\addcontentsline{toc}{section}{\nameref{sec:1582}}
\begin{longtable}{l p{0.5cm} r}
هرچ گویی از بهانه لا نسلم لا نسلم
&&
کار دارم من به خانه لا نسلم لا نسلم
\\
گفته‌ای فردا بیایم لطف و نیکویی نمایم
&&
وعده‌ست این بی‌نشانه لا نسلم لا نسلم
\\
گفته‌ای رنجور دارم دل ز غم پرشور دارم
&&
این فریب است و بهانه لا نسلم لا نسلم
\\
گفت مادر مادرانه چون ببینی دام و دانه
&&
این چنین گو ره روانه لا نسلم لا نسلم
\\
گوییم امروز زارم نیت حمام دارم
&&
می نمایی سنگ و شانه لا نسلم لا نسلم
\\
هر کجا خوانند ما را تا فریبانند ما را
&&
غیر این عالی ستانه لا نسلم لا نسلم
\\
بر سر مستان بیایی هر دمی زحمت نمایی
&&
کاین فلان است آن فلانه لا نسلم لا نسلم
\\
گوییم من خواجه تاشم عاقبت اندیش باشم
&&
تا درافتی در میانه لا نسلم لا نسلم
\\
رو ترش کرد آن مبرسم تا ز شکل او بترسم
&&
ای عجوزه بامثانه لا نسلم لا نسلم
\\
دست از خشمم گزیدی گویی از عشقت گزیدم
&&
مغلطه است این ای یگانه لا نسلم لا نسلم
\\
جمله را نتوان شمردن شرح یک یک حیله کردن
&&
نیست مکرت را کرانه لا نسلم لا نسلم
\\
\end{longtable}
\end{center}
