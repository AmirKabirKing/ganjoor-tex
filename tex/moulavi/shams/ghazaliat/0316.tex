\begin{center}
\section*{غزل شماره ۳۱۶: چونک درآییم به غوغای شب}
\label{sec:0316}
\addcontentsline{toc}{section}{\nameref{sec:0316}}
\begin{longtable}{l p{0.5cm} r}
چونک درآییم به غوغای شب
&&
گرد برآریم ز دریای شب
\\
خواب نخواهد بگریزد ز خواب
&&
آنک بدیدست تماشای شب
\\
بس دل پرنور و بسی جان پاک
&&
مشتغل و بنده و مولای شب
\\
شب تتق شاهد غیبی بود
&&
روز کجا باشد همتای شب
\\
پیش تو شب هست چو دیگ سیاه
&&
چون نچشیدی تو ز حلوای شب
\\
دست مرا بست شب از کسب و کار
&&
تا به سحر دست من و پای شب
\\
راه درازست برانیم تیز
&&
ما به درازا و به پهنای شب
\\
روز اگر مکسب و سوداگریست
&&
ذوق دگر دارد سودای شب
\\
مفخر تبریز توی شمس دین
&&
حسرت روزی و تمنای شب
\\
\end{longtable}
\end{center}
