\begin{center}
\section*{غزل شماره ۴۳۲: این چنین پابند جان میدان کیست}
\label{sec:0432}
\addcontentsline{toc}{section}{\nameref{sec:0432}}
\begin{longtable}{l p{0.5cm} r}
این چنین پابند جان میدان کیست
&&
ما شدیم از دست این دستان کیست
\\
می‌دود چون گوی زرین آفتاب
&&
ای عجب اندر خم چوگان کیست
\\
آفتابا راه زن راهت نزد
&&
چون زند داند که این ره آن کیست
\\
سیب را بو کرد موسی جان بداد
&&
بازجو آن بو ز سیبستان کیست
\\
چشم یعقوبی از این بو باز شد
&&
ای خدا این بوی از کنعان کیست
\\
خاک بودیم این چنین موزون شدیم
&&
خاک ما زر گشت در میزان کیست
\\
بر زر ما هر زمان مهر نوست
&&
تا بداند زر که او از کان کیست
\\
جمله حیرانند و سرگردان عشق
&&
ای عجب این عشق سرگردان کیست
\\
جمله مهمانند در عالم ولیک
&&
کم کسی داند که او مهمان کیست
\\
نرگس چشم بتان ره می‌زند
&&
آب این نرگس ز نرگسدان کیست
\\
جسم‌ها شب خالی از ما روز پر
&&
ما و من چون گربه در انبان کیست
\\
هر کسی دستک زنان کای جان من
&&
و آنک دستک زن کند او جان کیست
\\
شمس تبریزی که نور اولیاست
&&
با چنان عز و شرف سلطان کیست
\\
\end{longtable}
\end{center}
