\begin{center}
\section*{غزل شماره ۲۶۸۲: به بخت و طالع ما ای افندی}
\label{sec:2682}
\addcontentsline{toc}{section}{\nameref{sec:2682}}
\begin{longtable}{l p{0.5cm} r}
به بخت و طالع ما ای افندی
&&
سفر کردی از این جا ای افندی
\\
چراغم مرد و دودم رفت بالا
&&
دو چشمم ماند بالا ای افندی
\\
زمین تا آسمان دود سیاه‌ست
&&
سیه پوشید سودا ای افندی
\\
در این عالم مرا تنها تو بودی
&&
بماندم بی‌تو تنها ای افندی
\\
کجا بختی که اندر آتش تو
&&
ببیند حال ما را ای افندی
\\
همی‌گویم افندی ای افندی
&&
جوابم گوی و بازآ ای افندی
\\
چه بازآیم چه گویم من که رفتم
&&
ورای هفت دریا ای افندی
\\
چه حیران و چه دشمن کام گشتم
&&
تو رحمت کن خدایا ای افندی
\\
همی‌ترسم که تا آن رحمت آید
&&
نماند بنده برجا ای افندی
\\
تتیپایش افندی این چه کردی
&&
تتیپا ثا تتیپا ای افندی
\\
\end{longtable}
\end{center}
