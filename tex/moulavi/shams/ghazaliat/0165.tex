\begin{center}
\section*{غزل شماره ۱۶۵: اگر آن میی که خوردی به سحر نبود گیرا}
\label{sec:0165}
\addcontentsline{toc}{section}{\nameref{sec:0165}}
\begin{longtable}{l p{0.5cm} r}
اگر آن میی که خوردی به سحر نبود گیرا
&&
بستان ز من شرابی که قیامتست حقا
\\
چه تفرج و تماشا که رسد ز جام اول
&&
دومش نعوذبالله چه کنم صفت سوم را
\\
غم و مصلحت نماند همه را فرود راند
&&
پس از آن خدای داند که کجا کشد تماشا
\\
تو اسیر بو و رنگی به مثال نقش سنگی
&&
بجهی چو آب چشمه ز درون سنگ خارا
\\
بده آن می رواقی هله ای کریم ساقی
&&
چو چنان شوم بگویم سخن تو بی‌محابا
\\
قدحی گران به من ده به غلام خویشتن ده
&&
بنگر که از خمارت نگران شدم به بالا
\\
نگران شدم بدان سو که تو کرده‌ای مرا خو
&&
که روانه باد آن جو که روانه شد ز دریا
\\
\end{longtable}
\end{center}
