\begin{center}
\section*{غزل شماره ۴۰۸: آن شنیدی که خضر تخته کشتی بشکست}
\label{sec:0408}
\addcontentsline{toc}{section}{\nameref{sec:0408}}
\begin{longtable}{l p{0.5cm} r}
آن شنیدی که خضر تخته کشتی بشکست
&&
تا که کشتی ز کف ظالم جبار برست
\\
خضر وقت تو عشق است که صوفی ز شکست
&&
صافیست و مثل درد به پستی بنشست
\\
لذت فقر چو باده‌ست که پستی جوید
&&
که همه عاشق سجده‌ست و تواضع سرمست
\\
تا بدانی که تکبر همه از بی‌مزگیست
&&
پس سزای متکبر سر بی‌ذوق بس است
\\
گریه شمع همه شب نه که از درد سرست
&&
چون ز سر رست همه نور شد از گریه برست
\\
کف هستی ز سر خم مدمغ برود
&&
چون بگیرد قدح باده جان بر کف دست
\\
ماهیا هر چه تو را کام دل از بحر بجو
&&
طمع خام مکن تا نخلد کام ز شست
\\
بحر می‌غرد و می‌گوید کای امت آب
&&
راست گویید بر این مایده کس را گله هست
\\
دم به دم بحر دل و امت او در خوش و نوش
&&
در خطابات و مجابات بلی‌اند و الست
\\
نی در آن بزم کس از درد دلی سر بگرفت
&&
نی در آن باغ و چمن پای کس از خار بخست
\\
هله خامش به خموشیت اسیران برهند
&&
ز خموشانه تو ناطق و خاموش بجست
\\
لب فروبند چو دیدی که لب بسته یار
&&
دست شمشیرزنان را به چه تدبیر ببست
\\
\end{longtable}
\end{center}
