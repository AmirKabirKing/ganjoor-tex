\begin{center}
\section*{غزل شماره ۱۴۲۸: همه بازان عجب ماندند در آهنگ پروازم}
\label{sec:1428}
\addcontentsline{toc}{section}{\nameref{sec:1428}}
\begin{longtable}{l p{0.5cm} r}
همه بازان عجب ماندند در آهنگ پروازم
&&
کبوتر همچو من دیدی که من در جستن بازم
\\
به هر هنگام هر مرغی به هر پری همی‌پرد
&&
مگر من سنگ پولادم که در پرواز آغازم
\\
دهان مگشای بی‌هنگام و می ترس از زبان من
&&
زبانت گر بود زرین زبان درکش که من گازم
\\
به دنبل دنبه می گوید مرا نیشی است در باطن
&&
تو را بشکافم ای دنبل گر از آغاز بنوازم
\\
بمالم بر تو من خود را به نرمی تا شوی ایمن
&&
به ناگاهانت بشکافم که تا دانی چه فن سازم
\\
دهان مگشای این ساعت ازیرا دنبل خامی
&&
چو وقت آید شوی پخته به کار تو بپردازم
\\
کدامین شوخ برد از ما که دیده شوخ کردستی
&&
چه خوانی دیده پیهی را که پس فرداش بگدازم
\\
کمان نطق من بستان که تیر قهر می پرد
&&
که از مستی مبادا تیر سوی خویش اندازم
\\
یکی سوزی است سازنده عتاب شمس تبریزی
&&
رهم از عالم ناری چو با این سوز درسازم
\\
\end{longtable}
\end{center}
