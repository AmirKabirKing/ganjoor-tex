\begin{center}
\section*{غزل شماره ۲۲۲۰: هله ای شاه مپیچان سر و دستار مرو}
\label{sec:2220}
\addcontentsline{toc}{section}{\nameref{sec:2220}}
\begin{longtable}{l p{0.5cm} r}
هله ای شاه مپیچان سر و دستار مرو
&&
هله ای ماه که نغزت رخ و رخسار مرو
\\
در همه روی زمین چشم و دل باز که راست
&&
مکن آزار مکن جانب اغیار مرو
\\
مبر از یار مبر خانه اسرار مسوز
&&
گل و گلزار مکن جانب هر خار مرو
\\
مکن ای یار ستیزه دغل و جنگ مجوی
&&
هله آن بار برفتی مکن این بار مرو
\\
بنده و چاکر و پرورده و مولای توایم
&&
ای دل و دین و حیات خوش ناچار مرو
\\
هله سرنای توام مست نواهای توام
&&
مشکن چنگ طرب را مسکل تار مرو
\\
هله مخمور چه نالی بر مخمور دگر
&&
پهلوی خم بنشین از بر خمار مرو
\\
هله جان بخش بیا ای صدقات تو حیات
&&
به از این خیر نباشد به جز این کار مرو
\\
خاتم حسن و جمالی هله ای یوسف دهر
&&
سوی مکاری اخوان ستمکار مرو
\\
هله دیدار مهل برمگزین فکر و خیال
&&
از عیان سر مکشان در پی آثار مرو
\\
هله موسی زمان گرد برآر از دریا
&&
دل فرعون مجو جانب انکار مرو
\\
هله عیسی قران صحت رنجور گران
&&
از برای دو سه ترسا سوی زنار مرو
\\
هله ای شاهد جان خواجه جان‌های شهان
&&
شیوه کن لب بگز و غبغبه افشار مرو
\\
هله صدیق زمانی به تو ختم است وفا
&&
جز سوی احمد بگزیده مختار مرو
\\
جبرئیل کرمی سدره مقام و وطنت
&&
همچو مرغان زمین بر سر شخسار مرو
\\
تو یقین دار که بی‌تو نفسی جان نزید
&&
در احسان بگشا و پس دیوار مرو
\\
همه رندان و حریفان و بتان جمع شدند
&&
وقت کار است بیا کار کن از کار مرو
\\
هله باقی غزل را ز شهنشاه بجوی
&&
همگی گوش شو اکنون سوی گفتار مرو
\\
\end{longtable}
\end{center}
