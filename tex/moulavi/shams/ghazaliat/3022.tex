\begin{center}
\section*{غزل شماره ۳۰۲۲: گفت مرا آن طبیب رو ترشی خورده‌ای}
\label{sec:3022}
\addcontentsline{toc}{section}{\nameref{sec:3022}}
\begin{longtable}{l p{0.5cm} r}
گفت مرا آن طبیب رو ترشی خورده‌ای
&&
گفتم نی گفت نک رنگ ترش کرده‌ای
\\
دل چو سیاهی دهد رنگ گواهی دهد
&&
عکس برون می‌زند گر چه تو در پرده‌ای
\\
خاک تو گر آب خوش یابد چون روضه‌ایست
&&
ور خورد او آب شور شوره برآورده‌ای
\\
سبز شوند از بهار زرد شوند از خزان
&&
گر نه خزان دیده‌ای پس ز چه روزرده‌ای
\\
گفتمش ای غیب دان از تو چه دارم نهان
&&
پرورش جان تویی جان چو تو پرورده‌ای
\\
کیست که زنده کند آنک تواش کشته‌ای
&&
کیست که گرمش کند چون تواش افسرده‌ای
\\
شربت صحت فرست هم ز شرابات خاص
&&
زانک تو جوشیده‌ای زانک تو افشرده‌ای
\\
داد شراب خطیر گفت هلا این بگیر
&&
شاد شو ار پرغمی زنده شو ار مرده‌ای
\\
چشمه بجوشد ز تو چون ارس از خاره‌ای
&&
نور بتابد ز تو گر چه سیه چرده‌ای
\\
خضر بقایی شوی گر عرض فانیی
&&
شادی دل‌ها شوی گر چه دل آزرده‌ای
\\
کی بشود این وجود پاک ز بیگانگان
&&
تا نرسد خلعتی دولت صدمرده‌ای
\\
گفت درختی به باد چند وزی باد گفت
&&
باد بهاری کند گر چه تو پژمرده‌ای
\\
\end{longtable}
\end{center}
