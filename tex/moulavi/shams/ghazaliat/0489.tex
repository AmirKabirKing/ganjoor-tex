\begin{center}
\section*{غزل شماره ۴۸۹: اگر تو مست وصالی رخ تو ترش چراست}
\label{sec:0489}
\addcontentsline{toc}{section}{\nameref{sec:0489}}
\begin{longtable}{l p{0.5cm} r}
اگر تو مست وصالی رخ تو ترش چراست
&&
برون شیشه ز حال درون شیشه گواست
\\
پدید باشد مستی میان صد هشیار
&&
ز بوی رنگ و ز چشم و فتادن از چپ و راست
\\
علی الخصوص شرابی که اولیا نوشند
&&
که جوش و نوش و قوامش ز خم لطف خداست
\\
خم شراب میان هزار خم دگر
&&
به کف و تف و به جوش و به غلغله پیداست
\\
چو جوش دیدی می‌دان که آتش‌ست ز جان
&&
خروش دیدی می‌دانک شعله سوداست
\\
بدانک سرکه فروشی شراب کی دهدت
&&
که جرعه‌اش را صد من شکر به نقد بهاست
\\
بهای باده من المؤمنین انفسهم
&&
هوای نفس بمان گر هوات بیع و شراست
\\
هوای نفس رها کردی و عوض نرسید
&&
مگو چنین که بر آن مکرم این دروغ خطاست
\\
کسی که شب به خرابات قاب قوسینست
&&
درون دیده پرنور او خمار لقاست
\\
طهارتی‌ست ز غم باده شراب طهور
&&
در آن دماغ که باده‌ست باد غم ز کجاست
\\
ابیت عند ربی نام آن خراباتست
&&
نشان یطعم و یسقن هم از پیمبر ماست
\\
\end{longtable}
\end{center}
