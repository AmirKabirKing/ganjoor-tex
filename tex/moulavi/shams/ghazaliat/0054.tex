\begin{center}
\section*{غزل شماره ۵۴: از این اقبالگاه خوش مشو یک دم دلا تنها}
\label{sec:0054}
\addcontentsline{toc}{section}{\nameref{sec:0054}}
\begin{longtable}{l p{0.5cm} r}
از این اقبالگاه خوش مشو یک دم دلا تنها
&&
دمی می نوش باده جان و یک لحظه شکر می‌خا
\\
به باطن همچو عقل کل به ظاهر همچو تنگ گل
&&
دمی الهام امر قل دمی تشریف اعطینا
\\
تصورهای روحانی خوشی بی‌پشیمانی
&&
ز رزم و بزم پنهانی ز سر سر او اخفی
\\
ملاحت‌های هر چهره از آن دریاست یک قطره
&&
به قطره سیر کی گردد کسی کش هست استسقا
\\
دلا زین تنگ زندان‌ها رهی داری به میدان‌ها
&&
مگر خفته‌ست پای تو تو پنداری نداری پا
\\
چه روزی‌هاست پنهانی جز این روزی که می‌جویی
&&
چه نان‌ها پخته‌اند ای جان برون از صنعت نانبا
\\
تو دو دیده فروبندی و گویی روز روشن کو
&&
زند خورشید بر چشمت که اینک من تو در بگشا
\\
از این سو می‌کشانندت و زان سو می‌کشانندت
&&
مرو ای ناب با دردی بپر زین درد رو بالا
\\
هر اندیشه که می‌پوشی درون خلوت سینه
&&
نشان و رنگ اندیشه ز دل پیداست بر سیما
\\
ضمیر هر درخت ای جان ز هر دانه که می‌نوشد
&&
شود بر شاخ و برگ او نتیجه شرب او پیدا
\\
ز دانه سیب اگر نوشد بروید برگ سیب از وی
&&
ز دانه تمر اگر نوشد بروید بر سرش خرما
\\
چنانک از رنگ رنجوران طبیب از علت آگه شد
&&
ز رنگ و روی چشم تو به دینت پی برد بینا
\\
ببیند حال دین تو بداند مهر و کین تو
&&
ز رنگت لیک پوشاند نگرداند تو را رسوا
\\
نظر در نامه می‌دارد ولی با لب نمی‌خواند
&&
همی‌داند کز این حامل چه صورت زایدش فردا
\\
وگر برگوید از دیده بگوید رمز و پوشیده
&&
اگر درد طلب داری بدانی نکته و ایما
\\
وگر درد طلب نبود صریحا گفته گیر این را
&&
فسانه دیگران دانی حواله می‌کنی هر جا
\\
\end{longtable}
\end{center}
