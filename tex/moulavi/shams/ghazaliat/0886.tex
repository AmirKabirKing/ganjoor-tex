\begin{center}
\section*{غزل شماره ۸۸۶: از رسن زلف تو خلق به جان آمدند}
\label{sec:0886}
\addcontentsline{toc}{section}{\nameref{sec:0886}}
\begin{longtable}{l p{0.5cm} r}
از رسن زلف تو خلق به جان آمدند
&&
بهر رسن بازیش لولیکان آمدند
\\
در دل هر لولیی عشق چو استاره‌ای
&&
رقص کنان گرد ماه نورفشان آمدند
\\
در هوس این سماع از پس بستان عشق
&&
سروقدان چون چنار دست زنان آمدند
\\
بین که چه ریسیده‌ایم دست که لیسیده‌ایم
&&
تا که چنین لقمه‌ها سوی دهان آمدند
\\
لولیکان قنق در کف گوشه تتق
&&
وز تتق آن عروس شاه جهان آمدند
\\
شاه که در دولتش هر طرفی شاهدی
&&
سینه گشاده به ما بهر امان آمدند
\\
شیوه ابرو کند هر نفسی پیش ما
&&
گر چه که از تیر غمز سخته کمان آمدند
\\
شب رو و عیار باش بر سر هر کوی از آنک
&&
زیر لحاف ازل نیک نهان آمدند
\\
جانب تبریز در شمس حقم دیده‌اند
&&
ترک دکان خواندند چونک به کان آمدند
\\
\end{longtable}
\end{center}
