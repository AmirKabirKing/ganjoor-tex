\begin{center}
\section*{غزل شماره ۲۶۹۷: بدید این دل درون دل بهاری}
\label{sec:2697}
\addcontentsline{toc}{section}{\nameref{sec:2697}}
\begin{longtable}{l p{0.5cm} r}
بدید این دل درون دل بهاری
&&
سحرگه دید طرفه مرغزاری
\\
در او آرامگاه جان عاشق
&&
در او بوس و کنار بی‌کناری
\\
که فردوسش غلام آن گلستان
&&
بهشت از سبزه زارش شرمساری
\\
به هر جانب یکی حلقه سماعی
&&
به زیر هر درختی خوش نگاری
\\
اگر پیری درآید همچو کافور
&&
شود گل عارضی مشکین عذاری
\\
چو شیر اسکست جان زنجیرها را
&&
رمید آن سو چو مجنون بی‌قراری
\\
برفتم در پی جان تا کجا شد
&&
در آن رفتن مرا بگشاد کاری
\\
بدیدم طرفه منزل‌های دلکش
&&
ولیک از جان ندیدم من غباری
\\
بگو راز مرا تا بازآید
&&
وگر ناید بیا واپس تو باری
\\
نشانی‌ها بیاور ارمغانی
&&
که تا تن را کنم من دارداری
\\
کیست آن مه خداوند شمس تبریز
&&
خداخلقی عجیبی نامداری
\\
\end{longtable}
\end{center}
