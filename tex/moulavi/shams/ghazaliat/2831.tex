\begin{center}
\section*{غزل شماره ۲۸۳۱: چو نماز شام هر کس بنهد چراغ و خوانی}
\label{sec:2831}
\addcontentsline{toc}{section}{\nameref{sec:2831}}
\begin{longtable}{l p{0.5cm} r}
چو نماز شام هر کس بنهد چراغ و خوانی
&&
منم و خیال یاری غم و نوحه و فغانی
\\
چو وضو ز اشک سازم بود آتشین نمازم
&&
در مسجدم بسوزد چو بدو رسد اذانی
\\
رخ قبله‌ام کجا شد که نماز من قضا شد
&&
ز قضا رسد هماره به من و تو امتحانی
\\
عجبا نماز مستان تو بگو درست هست آن
&&
که نداند او زمانی نشناسد او مکانی
\\
عجبا دو رکعت است این عجبا که هشتمین است
&&
عجبا چه سوره خواندم چو نداشتم زبانی
\\
در حق چگونه کوبم که نه دست ماند و نه دل
&&
دل و دست چون تو بردی بده ای خدا امانی
\\
به خدا خبر ندارم چو نماز می‌گزارم
&&
که تمام شد رکوعی که امام شد فلانی
\\
پس از این چو سایه باشم پس و پیش هر امامی
&&
که بکاهم و فزایم ز حراک سایه بانی
\\
به رکوع سایه منگر به قیام سایه منگر
&&
مطلب ز سایه قصدی مطلب ز سایه جانی
\\
ز حساب رست سایه که به جان غیر جنبد
&&
که همی‌زند دو دستک که کجاست سایه دانی
\\
چو شه است سایه بانم چو روان شود روانم
&&
چو نشیند او نشستم به کرانه دکانی
\\
چو مرا نماند مایه منم و حدیث سایه
&&
چه کند دهان سایه تبعیت دهانی
\\
نکنی خمش برادر چو پری ز آب و آذر
&&
ز سبو همان تلابد که در او کنند یا نی
\\
\end{longtable}
\end{center}
