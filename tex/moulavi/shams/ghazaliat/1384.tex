\begin{center}
\section*{غزل شماره ۱۳۸۴: عشقا تو را قاضی برم کاشکستیم همچون صنم}
\label{sec:1384}
\addcontentsline{toc}{section}{\nameref{sec:1384}}
\begin{longtable}{l p{0.5cm} r}
عشقا تو را قاضی برم کاشکستیم همچون صنم
&&
از من نخواهد کس گوا که شاهدم نی ضامنم
\\
مقضی تویی قاضی تویی مستقبل و ماضی تویی
&&
خشمین تویی راضی تویی تا چون نمایی دم به دم
\\
ای عشق زیبای منی هم من توام هم تو منی
&&
هم سیلی و هم خرمنی هم شادیی هم درد و غم
\\
آن‌ها تویی وین‌ها تویی وزین و آن تنها تویی
&&
وان دشت باپهنا تویی وان کوه و صحرای کرم
\\
شیرینی خویشان تویی سرمستی ایشان تویی
&&
دریای درافشان تویی کان‌های پرزر و درم
\\
عشق سخن کوشی تویی سودای خاموشی تویی
&&
ادراک و بی‌هوشی تویی کفر و هدی عدل و ستم
\\
ای خسرو شاهنشهان ای تختگاهت عقل و جان
&&
ای بی‌نشان با صد نشان ای مخزنت بحر عدم
\\
پیش تو خوبان و بتان چون پیش سوزن لعبتان
&&
زشتش کنی نغزش کنی بردری از مرگ و سقم
\\
هر نقش با نقشی دگر چون شیر بودی و شکر
&&
گر واقفندی نقش‌ها که آمدند از یک قلم
\\
آن کس که آمد سوی تو تا جان دهد در کوی تو
&&
رشک تو گوید که برو لطف تو خواند که نعم
\\
لطف تو سابق می شود جذاب عاشق می شود
&&
بر قهر سابق می شود چون روشنایی بر ظلم
\\
هر زنده‌ای را می کشد وهم خیالی سو به سو
&&
کرده خیالی را کفت لشکرکش و صاحب علم
\\
دیگر خیالی آوری ز اول رباید سروری
&&
آن را اسیر این کنی ای مالک الملک و حشم
\\
هر دم خیالی نو رسد از سوی جان اندر جسد
&&
چون کودکان قلعه بزم گوید ز قسام القسم
\\
خامش کنم بندم دهان تا برنشورد این جهان
&&
چون می نگنجی در بیان دیگر نگویم بیش و کم
\\
\end{longtable}
\end{center}
