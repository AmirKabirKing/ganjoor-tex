\begin{center}
\section*{غزل شماره ۲۸۶۵: در رخ عشق نگر تا به صفت مرد شوی}
\label{sec:2865}
\addcontentsline{toc}{section}{\nameref{sec:2865}}
\begin{longtable}{l p{0.5cm} r}
در رخ عشق نگر تا به صفت مرد شوی
&&
نزد سردان منشین کز دمشان سرد شوی
\\
از رخ عشق بجو چیز دگر جز صورت
&&
کار آن است که با عشق تو هم درد شوی
\\
چون کلوخی به صفت تو به هوا برنپری
&&
به هوا برشوی ار بشکنی و گرد شوی
\\
تو اگر نشکنی آن کت به سرشت او شکند
&&
چونک مرگت شکند کی گهر فرد شوی
\\
برگ چون زرد شود بیخ ترش سبز کند
&&
تو چرا قانعی از عشق کز او زرد شوی
\\
\end{longtable}
\end{center}
