\begin{center}
\section*{غزل شماره ۲۶۵۲: نگارا تو در اندیشه درازی}
\label{sec:2652}
\addcontentsline{toc}{section}{\nameref{sec:2652}}
\begin{longtable}{l p{0.5cm} r}
نگارا تو در اندیشه درازی
&&
بیاوردی که با یاران نسازی
\\
نه عاشق بر سر آتش نشیند
&&
مگر که عاشقی باشد مجازی
\\
به من بنگر که بودم پیش از این عشق
&&
ز عالم فارغ اندر بی‌نیازی
\\
قضا آمد بدیدم ماه رویی
&&
گرفتم من سر زلفش به بازی
\\
گناه این بود افتادم به عشقی
&&
چو صد روز قیامت در درازی
\\
ز خونم بوی مشک آید چو ریزد
&&
شهید شرمسارم من ز غازی
\\
نصیحت داد شمس الدین تبریز
&&
که چون معشوق ای عاشق ننازی
\\
\end{longtable}
\end{center}
