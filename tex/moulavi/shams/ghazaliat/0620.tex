\begin{center}
\section*{غزل شماره ۶۲۰: از سرو مرا بوی بالای تو می‌آید}
\label{sec:0620}
\addcontentsline{toc}{section}{\nameref{sec:0620}}
\begin{longtable}{l p{0.5cm} r}
از سرو مرا بوی بالای تو می‌آید
&&
وز ماه مرا رنگ و سیمای تو می‌آید
\\
هر نی کمر خدمت در پیش تو می‌بندد
&&
شکر به غلامی حلوای تو می‌آید
\\
هر نور که آید او از نور تو زاید او
&&
می مژده دهد یعنی فردای تو می‌آید
\\
گل خواجه سوسن شد آرایش گلشن شد
&&
زیرا که از آن خنده رعنای تو می‌آید
\\
هر گه ز تو بگریزم با عشق تو بستیزم
&&
اندر سرم از شش سو سودای تو می‌آید
\\
چون برروم از پستی بیرون شوم از هستی
&&
در گوش من آن جا هم هیهای تو می‌آید
\\
اندر دل آوازی پرشورش و غمازی
&&
آن ناله چنین دانم کز نای تو می‌آید
\\
روزست شبم از تو خشکست لبم از تو
&&
غم نیست اگر خشکست دریای تو می‌آید
\\
زیر فلک اطلس هشیار نماند کس
&&
زیرا که ز بیش و پس می‌های تو می‌آید
\\
از جور تو اندیشم جور آید در پیشم
&&
بینم که چنان تلخی از رای تو می‌آید
\\
شمس الحق تبریزی اندیشه چو باد خوش
&&
جان تازه کند زیرا صحرای تو می‌آید
\\
\end{longtable}
\end{center}
