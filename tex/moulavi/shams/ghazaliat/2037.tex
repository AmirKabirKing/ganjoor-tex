\begin{center}
\section*{غزل شماره ۲۰۳۷: چون جان تو می‌ستانی چون شکر است مردن}
\label{sec:2037}
\addcontentsline{toc}{section}{\nameref{sec:2037}}
\begin{longtable}{l p{0.5cm} r}
چون جان تو می‌ستانی چون شکر است مردن
&&
با تو ز جان شیرین شیرینتر است مردن
\\
بردار این طبق را زیرا خلیل حق را
&&
باغ است و آب حیوان گر آذر است مردن
\\
این سر نشان مردن و آن سر نشان زادن
&&
زان سر کسی نمیرد نی زین سر است مردن
\\
بگذار جسم و جان شو رقصان بدان جهان شو
&&
مگریز اگر چه حالی شور و شر است مردن
\\
والله به ذات پاکش نه چرخ گشت خاکش
&&
با قند وصل همچون حلواگر است مردن
\\
از جان چرا گریزیم جان است جان سپردن
&&
وز کان چرا گریزیم کان زر است مردن
\\
چون زین قفس برستی در گلشن است مسکن
&&
چون این صدف شکستی چون گوهر است مردن
\\
چون حق تو را بخواند سوی خودت کشاند
&&
چون جنت است رفتن چون کوثر است مردن
\\
مرگ آینه‌ست و حسنت در آینه درآمد
&&
آیینه بربگوید خوش منظر است مردن
\\
گر مؤمنی و شیرین هم مؤمن است مرگت
&&
ور کافری و تلخی هم کافر است مردن
\\
گر یوسفی و خوبی آیینه‌ات چنان است
&&
ور نی در آن نمایش هم مضطر است مردن
\\
خامش که خوش زبانی چون خضر جاودانی
&&
کز آب زندگانی کور و کر است مردن
\\
\end{longtable}
\end{center}
