\begin{center}
\section*{غزل شماره ۲۴۳۷: ای یوسف خوش نام هی در ره میا بی‌همرهی}
\label{sec:2437}
\addcontentsline{toc}{section}{\nameref{sec:2437}}
\begin{longtable}{l p{0.5cm} r}
ای یوسف خوش نام هی در ره میا بی‌همرهی
&&
مسکل ز یعقوب خرد تا درنیفتی در چهی
\\
آن سگ بود کو بیهده خسپد به پیش هر دری
&&
و آن خر بود کز ماندگی آید سوی هر خرگهی
\\
در سینه این عشق و حسد بین کز چه جانب می‌رسد
&&
دل را کی آگاهی دهد جز دلنوازی آگهی
\\
مانند مرغی باش هان بر بیضه همچو پاسبان
&&
کز بیضه دل زایدت مستی و وصل و قهقهی
\\
دامن ندارد غیر او جمله گدااند ای عمو
&&
درزن دو دست خویش را در دامن شاهنشهی
\\
مانند خورشید از غمش می‌رو در آتش تا به شب
&&
چون شب شود می‌گرد خوش بر بام او همچون مهی
\\
بر بام او این اختران تا صبحدم چوبک زنان
&&
والله مبارک حضرتی والله همایون درگهی
\\
آن انبیا کاندر جهان کردند رو در آسمان
&&
رستند از دام زمین وز شرکت هر ابلهی
\\
بربوده گشتند آن طرف چون آهن از آهن ربا
&&
زان سان که سوی کهربا بی‌پر و پا پرد کهی
\\
می‌دانک بی‌انزال او نزلی نروید در زمین
&&
بی‌صحبت تصویر او یک مایه را نبود زهی
\\
ارواح همچون اشتران ز آواز سیروا مستیان
&&
همچون عرابی می‌کند آن اشتران را نهنهی
\\
بر لوح دل رمال جان رمل حقایق می‌زند
&&
تا از رقومش رمل شد زر لطیف ده دهی
\\
خوشتر روید ای همرهان کآمد طبیبی در جهان
&&
زنده کن هر مرده‌ای بیناکن هر اکمهی
\\
این‌ها همه باشد ولی چون پرده بردارد رخش
&&
نی زهره ماند نی نوا نی نوحه گر را وه وهی
\\
خاموش کن گر بلبلی رو سوی گلشن بازپر
&&
بلبل به خارستان رود اما به نادر گه گهی
\\
\end{longtable}
\end{center}
