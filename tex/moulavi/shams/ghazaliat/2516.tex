\begin{center}
\section*{غزل شماره ۲۵۱۶: اگر الطاف شمس الدین بدیده برفتادستی}
\label{sec:2516}
\addcontentsline{toc}{section}{\nameref{sec:2516}}
\begin{longtable}{l p{0.5cm} r}
اگر الطاف شمس الدین بدیده برفتادستی
&&
سوی افلاک روحانی دو دیده برگشادستی
\\
گشادستی دو دیده پرقدم را نیز از مستی
&&
ولی پرسعادت او در آن عالم نزادستی
\\
چو بنهادی قدم آن جا برفتی جسم از یادش
&&
که پنداری ز مادر او در آن عالم نزادستی
\\
میان خوبرویان جان شده چون ذره‌ها رقصان
&&
گهی مست جمالستی گهی سرمست باده ستی
\\
رخ خوبان روحانی که هر شاهی که دید آن را
&&
ز فرزین بند سوداها ز اسب خود پیادستی
\\
چو از مخدوم شمس الدین زدی لطفی به روی دل
&&
از این‌ها جمله روی دل شدی بی رنگ و سادستی
\\
بدیدی جمله شاهان را و خوبان را و ماهان را
&&
کمربسته به پیش او نشسته بر وسادستی
\\
اگر نه غیرت حضرت گرفتی دامن جاهش
&&
سزای جمله کردستی و داد حسن دادستی
\\
نه نفسی رهزنی کردی نه آوازه فنا بودی
&&
دل ذرات خاک از جان و جان از شاه شادستی
\\
اگر در آب می‌دیدی خیال روی چون آتش
&&
همه اجزای جرم خاک رقصان همچو بادستی
\\
ایا تبریز اگر سرت شدی محسوس هر حسی
&&
غلام خاک تو سنجر اسیرت کیقبادستی
\\
\end{longtable}
\end{center}
