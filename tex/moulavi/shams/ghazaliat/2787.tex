\begin{center}
\section*{غزل شماره ۲۷۸۷: ای مهی کاندر نکویی از صفت افزوده‌ای}
\label{sec:2787}
\addcontentsline{toc}{section}{\nameref{sec:2787}}
\begin{longtable}{l p{0.5cm} r}
ای مهی کاندر نکویی از صفت افزوده‌ای
&&
تا بسی درهای دولت بر فلک بگشوده‌ای
\\
ای بسا کوه احد کز راه دل برکنده‌ای
&&
ای بسا وصف احد کاندر نظر بنموده‌ای
\\
جان‌ها زنبوروار از عشق تو پران شده
&&
تا دهان خاکیان را زان عسل آلوده‌ای
\\
ای سبک عقلی که از خویشش گرانی داده‌ای
&&
وی گران جانی که سوی خویشتن بربوده‌ای
\\
شاد با گوش مقیم اندر مقالات الست
&&
چون ز بی‌چشمان مقالات خطا بشنوده‌ای
\\
در رخ پرزهر دونان کمترک خندیده‌ای
&&
هر خسی را از ضرورت در جهان بستوده‌ای
\\
فارغی از چرب و شیرین در حلاوت‌های خود
&&
چرب و شیرین باش از خود ز آنک خوش پالوده‌ای
\\
ای همه دعویت معنی ای ز معنی بیشتر
&&
ای دو صد چندانک دعوی کرده‌ای بنموده‌ای
\\
ای که می‌جویی مثال شمس تبریزی تو هم
&&
روزگاری می‌بری و اندر غم بیهوده‌ای
\\
\end{longtable}
\end{center}
