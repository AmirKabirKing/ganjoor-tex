\begin{center}
\section*{غزل شماره ۲۸۲۴: مه ما نیست منور تو مگر چرخ درآیی}
\label{sec:2824}
\addcontentsline{toc}{section}{\nameref{sec:2824}}
\begin{longtable}{l p{0.5cm} r}
مه ما نیست منور تو مگر چرخ درآیی
&&
ز تو پرماه شود چرخ چو بر چرخ برآیی
\\
کی بود چرخ و ثریا که بشاید قدمت را
&&
و اگر نیز بشاید ز تو یابند سزایی
\\
همه بی‌خدمت و رشوت رسد از لطف تو خلعت
&&
نه عدم بود من و ما که بدادی من و مایی
\\
ز من و ماست که جانی بگشاده‌ست دکانی
&&
و اگر نه به چه بازو کشد او قوس خدایی
\\
غلطی جان غلطی جان همه خود را بمرنجان
&&
نه مسیحی که به افسون به دمی چشم گشایی
\\
به سحرگاه و مشارق که شود تیره رخ مه
&&
کی بود نیم چراغی که کند نورفزایی
\\
چه کشیمش چه کشیمش تو بیا تا که کشیمش
&&
که چراغ خلق است این بر آن شمع سمایی
\\
مشکی را مشکی را مشکی پرهوسی را
&&
چه کشانی چه کشانی به مطارات همایی
\\
چو رخ روز ببیند ز بن گوش بمیرد
&&
ز چه رفتی ز چه مردی تو چنین سست چرایی
\\
زر و مال تو کجا شد پر و بال تو کجا شد
&&
عم و خال تو کجا شد و تو ادبار کجایی
\\
هله بازآ هله بازآ به سوی نعمت و ناز آ
&&
که منت بازفرستم ز پس مرگ و جدایی
\\
پر و بال تو بریدم غم و آه تو شنیدم
&&
هله بازت بخریدم که نه درخورد جفایی
\\
ز پس مرگ برون پر خبر رحمت من بر
&&
که نگویند چو رفتی به عدم بازنیایی
\\
کتب الله تعالی کرم الله توالی
&&
فتدلی و تجلی بعث العشق دوایی
\\
فعلاتن فعلاتن فعلاتن فعلاتن
&&
خمش و آب فرورو سمک بحر وفایی
\\
\end{longtable}
\end{center}
