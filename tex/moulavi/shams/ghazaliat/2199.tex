\begin{center}
\section*{غزل شماره ۲۱۹۹: ای سنایی عاشقان را درد باید درد کو}
\label{sec:2199}
\addcontentsline{toc}{section}{\nameref{sec:2199}}
\begin{longtable}{l p{0.5cm} r}
ای سنایی عاشقان را درد باید درد کو
&&
بار جور نیکوان را مرد باید مرد کو
\\
بار جور نیکوان از دی و فردا برتر است
&&
وانما جان کسی از دی و فردا فرد کو
\\
ور خیال آید تو را کز دی و فردا برتری
&&
برتری را کار و بار و ملک و بردابرد کو
\\
در میان هفت دریا دامن تو خشک کو
&&
در میان هفت دوزخ عنصر تو سرد کو
\\
این نداری خود ولیکن گر تو این را طالبی
&&
آه سرد و اشک گرم و چهره‌های زرد کو
\\
هر نفس بوی دل آید از صراط المستقیم
&&
تا نگویی عشق ره رو را که راه آورد کو
\\
گرد از آن دریا برآمد گرد جسم اولیاست
&&
تا نگویی قوم موسی را در این یم گرد کو
\\
\end{longtable}
\end{center}
