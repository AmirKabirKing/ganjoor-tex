\begin{center}
\section*{غزل شماره ۱۸۳۸: چند گریزی ای قمر هر طرفی ز کوی من}
\label{sec:1838}
\addcontentsline{toc}{section}{\nameref{sec:1838}}
\begin{longtable}{l p{0.5cm} r}
چند گریزی ای قمر هر طرفی ز کوی من
&&
صید توایم و ملک تو گر صنمیم وگر شمن
\\
هر نفس از کرانه‌ای ساز کنی بهانه‌ای
&&
هر نفسی برون کشی از عدمی هزار فن
\\
گر چه کثیف منزلم شد وطن تو این دلم
&&
رحمت مؤمنی بود میل و محبت وطن
\\
دشمن جاه تو نیم گر چه که بس مقصرم
&&
هیچ کسی بود شها دشمن جان خویشتن
\\
مطرب جمع عاشقان برجه و کاهلی مکن
&&
قصه حسن او بگو پرده عاشقان بزن
\\
همچو چهی است هجر او چون رسنی است ذکر او
&&
در تک چاه یوسفی دست زنان در آن رسن
\\
ذوق ز نیشکر بجو آن نی خشک را مخا
&&
چاره ز حسن او طلب چاره مجو ز بوالحسن
\\
گر تو مرید و طالبی هست مراد مطلق او
&&
ور تو ادیم طایفی هست سهیل در یمن
\\
آن دم کآفتاب او روزی و نور می دهد
&&
ذره به ذره را نگر نور گرفته در دهن
\\
گر چه که گل لطیفتر رزق گرفت بیشتر
&&
لیک رسید اندکی هم به دهان یاسمن
\\
عمر و ذکا و زیرکی داد به هندوان اگر
&&
حسن و جمال و دلبری داد به شاهد ختن
\\
ملک نصیب مهتران عشق نصیب کهتران
&&
قهر نصیب تیغ شد لطف نصیبه مجن
\\
شهد خدای هر شبی هست نصیبه لبی
&&
همچو کسی که باشدش بسته به عقد چار زن
\\
تا که بود حیات من عشق بود نبات من
&&
چونک بر آن جهان روم عشق بود مرا کفن
\\
مدمن خمرم و مرا مستی باده کم مکن
&&
نازک و شیرخواره‌ام دوره مکن ز من لبن
\\
چونک حزین غم شوم عشق ندیمیم کند
&&
عشق زمردی بود باشد اژدها حزن
\\
گفتم من به دل اگر بست رهت خمار غم
&&
باده و نقل آرمت شمع و ندیم خوش ذقن
\\
گفت دلم اگر جز او سازی شمع و ساقیم
&&
بر سر مام و باب زن جام و کباب بابزن
\\
گفتم ساقی او است و بس لیک به صورت دگر
&&
نیک ببین غلط مکن ای دل مست ممتحن
\\
بس کن از این بهانه‌ها وام هوای او بده
&&
تا نبود قماش جان پیش فراق مرتهن
\\
\end{longtable}
\end{center}
