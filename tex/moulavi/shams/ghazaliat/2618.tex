\begin{center}
\section*{غزل شماره ۲۶۱۸: جانا نظری فرما چون جان نظرهایی}
\label{sec:2618}
\addcontentsline{toc}{section}{\nameref{sec:2618}}
\begin{longtable}{l p{0.5cm} r}
جانا نظری فرما چون جان نظرهایی
&&
چون گویم دل بردی چون عین دل مایی
\\
جان‌ها همه پا کوبند آن لحظه که دل کوبی
&&
دل نیز شکر خاید آن دم که جگر خایی
\\
تن روح برافشاند چون دست برافشانی
&&
مرده ز تو حال آرد چون شعبده بنمایی
\\
گر جور و جفا این است پس گشت وفا کاسد
&&
ای دل به جفای او جان باز چه می‌پایی
\\
امروز چنان مستم کز خویش برون جستم
&&
ای یار بکش دستم آن جا که تو آن جایی
\\
چیزی که تو را باید افلاک همان زاید
&&
گوهر چه کمت آید چون در تک دریایی
\\
مردم ز تو شد ای جان هر مردمک دیده
&&
بی‌تو چه بود دیده‌ای گوهر بینایی
\\
ای روح بزن دستی در دولت سرمستی
&&
هستی و چه خوش هستی در وحدت یکتایی
\\
ای روح چه می‌ترسی روحی نه تن و نفسی
&&
تن معدن ترس آمد تو عیش و تماشایی
\\
ای روز چه خوش روزی شمع طرب افروزی
&&
او را برسان روزی جان را و پذیرایی
\\
صبحا نفسی داری سرمایه بیداری
&&
بر خفته دلان بردم انفاس مسیحایی
\\
شمس الحق تبریزی خورشید چو استاره
&&
در نور تو گم گردد چون شرق برآرایی
\\
\end{longtable}
\end{center}
