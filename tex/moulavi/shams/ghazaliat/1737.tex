\begin{center}
\section*{غزل شماره ۱۷۳۷: بیار باده که اندر خمار خمارم}
\label{sec:1737}
\addcontentsline{toc}{section}{\nameref{sec:1737}}
\begin{longtable}{l p{0.5cm} r}
بیار باده که اندر خمار خمارم
&&
خدا گرفت مرا زان چنین گرفتارم
\\
بیار جام شرابی که رشک خورشید است
&&
به جان عشق که از غیر عشق بیزارم
\\
بیار آنک اگر جان بخوانمش حیف است
&&
بدان سبب که ز جان دردهای سر دارم
\\
بیار آنک نگنجد در این دهان نامش
&&
که می شکافد از او شقه‌های گفتارم
\\
بیار آنک چو او نیست گولم و نادان
&&
چو با ویم ملک گربزان و طرارم
\\
بیار آنک دمی کز سرم شود خالی
&&
سیاه و تیره شوم گوییا ز کفارم
\\
بیار آنک رهاند از این بیار و میار
&&
بیار زود و مگو دفع کز کجا آرم
\\
بیار و بازرهان سقف آسمان‌ها را
&&
شب دراز ز دود و فغان بسیارم
\\
بیار آنک پس مرگ من هم از خاکم
&&
به شکر و گفت درآرد مثال نجارم
\\
بیار می که امین میم مثال قدح
&&
که هر چه در شکمم رفت پاک بسپارم
\\
نجار گفت پس مرگ کاشکی قومم
&&
گشاده دیده بدندی ز ذوق اسرارم
\\
به استخوان و به خونم نظر نکردندی
&&
به روح شاه عزیزم اگر به تن خوارم
\\
چه نردبان که تراشیده‌ام من نجار
&&
به بام هفتم گردون رسید رفتارم
\\
مسیح وار شدم من خرم بماند به زیر
&&
نه در غم خرم و نی به گوش خروارم
\\
بلیس وار ز آدم مبین تو آب و گلی
&&
ببین که در پس گل صد هزار گلزارم
\\
طلوع کرد از این لحم شمس تبریزی
&&
که آفتابم و سر زین وحل برون آرم
\\
غلط مشو چو وحل در رویم دیگربار
&&
که برقرارم و زین روی پوش در عارم
\\
به هر صبوح درآیم به کوری کوران
&&
برای کور طلوع و غروب نگذارم
\\
\end{longtable}
\end{center}
