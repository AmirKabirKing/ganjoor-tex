\begin{center}
\section*{غزل شماره ۲۹۳۸: آن مه چو در دل آید او را عجب شناسی}
\label{sec:2938}
\addcontentsline{toc}{section}{\nameref{sec:2938}}
\begin{longtable}{l p{0.5cm} r}
آن مه چو در دل آید او را عجب شناسی
&&
در دل چگونه آید از راه بی‌قیاسی
\\
گر گویی می‌شناسم لاف بزرگ و دعوی
&&
ور گویی من چه دانم کفر است و ناسپاسی
\\
بردانم و ندانم گردان شده‌ست خلقی
&&
گردان و چشم بسته چون استر خراسی
\\
می‌گرد چون خراسی خواهی و گر نخواهی
&&
گردن مپیچ زیرا دربند احتباسی
\\
یوسف خرید کوری با هیجده قلب آری
&&
از کوری خرنده وز حاسدی نخاسی
\\
تو هم ز یوسفانی در چاه تن فتاده
&&
اینک رسن برون آ تا در زمین نتاسی
\\
ای نفس مطمئنه اندر صفات حق رو
&&
اینک قبای اطلس تا کی در این پلاسی
\\
گر من غزل نخوانم بشکافد او دهانم
&&
گوید طرب بیفزا آخر حریف کاسی
\\
از بانگ طاس ماه بگرفته می‌گشاید
&&
ماهت منم گرفته بانگی زن ار تو طاسی
\\
آدم ز سنبلی خورد کان عاقبت بریزد
&&
تو سنبل وصالی ایمن ز زخم داسی
\\
\end{longtable}
\end{center}
