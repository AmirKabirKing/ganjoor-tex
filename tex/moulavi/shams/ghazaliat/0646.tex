\begin{center}
\section*{غزل شماره ۶۴۶: بار دگر آن مست به بازار درآمد}
\label{sec:0646}
\addcontentsline{toc}{section}{\nameref{sec:0646}}
\begin{longtable}{l p{0.5cm} r}
بار دگر آن مست به بازار درآمد
&&
وان سرده مخمور به خمار درآمد
\\
سرهای درختان همه پربار چرا شد
&&
کان بلبل خوش لحن به تکرار درآمد
\\
یک حمله دیگر همه در رقص درآییم
&&
مستانه و یارانه که آن یار درآمد
\\
یک حمله دیگر همه دامن بگشاییم
&&
کز بهر نثار آن شه دربار درآمد
\\
یک حمله دیگر به شکرخانه درآییم
&&
کز مصر چنین قند به خروار درآمد
\\
یک حمله دیگر بنه خواب بسوزیم
&&
زیرا که چنین دولت بیدار درآمد
\\
یک حمله دیگر به شب این پاس بداریم
&&
کان لولی شب دزد به اقرار درآمد
\\
یک حمله دیگر برسان باده که مستی
&&
در عربده ویران شده دستار درآمد
\\
یک حمله دیگر به سلیمان بگراییم
&&
کان هدهد پرخون شده منقار درآمد
\\
این شربت جان پرور جان بخش چه ساقیست
&&
از دست مسیحی که به بیمار درآمد
\\
اکنون بزند گردن غم‌های جهان را
&&
کاقبال تو چون حیدر کرار درآمد
\\
دارالحرج امروز چو دارالفرجی شد
&&
کان شادی و آن مستی بسیار درآمد
\\
بربند لب اکنون که سخن گستر بی‌لب
&&
بی حرف سیه روی به گفتار درآمد
\\
\end{longtable}
\end{center}
