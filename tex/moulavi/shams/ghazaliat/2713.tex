\begin{center}
\section*{غزل شماره ۲۷۱۳: چنان گشتم ز مستی و خرابی}
\label{sec:2713}
\addcontentsline{toc}{section}{\nameref{sec:2713}}
\begin{longtable}{l p{0.5cm} r}
چنان گشتم ز مستی و خرابی
&&
که خاکی را نمی‌دانم ز آبی
\\
در این خانه نمی‌یابم کسی را
&&
تو هشیاری بیا باشد بیابی
\\
همین دانم که مجلس از تو برپاست
&&
نمی‌دانم شرابی یا کبابی
\\
به باطن جان جان جان جانی
&&
به ظاهر آفتاب آفتابی
\\
از آن رو خوش فسونی که مسیحی
&&
از آن رو دیوسوزی که شهابی
\\
مرا خوش خوی کن زیرا شرابی
&&
مرا خوش بوی کن زیرا گلابی
\\
صبایی که بخندانی چمن را
&&
اگر چه تشنگان را تو عذابی
\\
بیا مستان بی‌حد بین به بازار
&&
اگر تو محتسب در احتسابی
\\
چو نان خواهان گهی اندر سؤالی
&&
چو رنجوران گهی اندر جوابی
\\
مثال برق کوته خنده تو
&&
از آن محبوس ظلمات سحابی
\\
درآ در مجلس سلطان باقی
&&
ببین گردان جفان کالجوابی
\\
تو خوش لعلی ولیکن زیر کانی
&&
تو بس خوبی ولیکن در نقابی
\\
به سوی شه پری باز سپیدی
&&
وگر پری به گورستان غرابی
\\
جوان بختا بزن دستی و می‌گو
&&
شبابی یا شبابی یا شبابی
\\
مگو با کس سخن ور سخت گیرد
&&
بگو والله اعلم بالصواب
\\
\end{longtable}
\end{center}
