\begin{center}
\section*{غزل شماره ۸: جز وی چه باشد کز اجل اندررباید کل ما}
\label{sec:0008}
\addcontentsline{toc}{section}{\nameref{sec:0008}}
\begin{longtable}{l p{0.5cm} r}
جز وی چه باشد کز اجل اندررباید کل ما
&&
صد جان برافشانم بر او گویم هنییا مرحبا
\\
رقصان سوی گردون شوم زان جا سوی بی‌چون شوم
&&
صبر و قرارم برده‌ای ای میزبان زوتر بیا
\\
از مه ستاره می‌بری تو پاره پاره می‌بری
&&
گه شیرخواره می‌بری گه می‌کشانی دایه را
\\
دارم دلی همچون جهان تا می‌کشد کوه گران
&&
من که کشم که کی کشم زین کاهدان واخر مرا
\\
گر موی من چون شیر شد از شوق مردن پیر شد
&&
من آردم گندم نیم چون آمدم در آسیا
\\
در آسیا گندم رود کز سنبله زادست او
&&
زاده مهم نی سنبله در آسیا باشم چرا
\\
نی نی فتد در آسیا هم نور مه از روزنی
&&
زان جا به سوی مه رود نی در دکان نانبا
\\
با عقل خود گر جفتمی من گفتنی‌ها گفتمی
&&
خاموش کن تا نشنود این قصه را باد هوا
\\
\end{longtable}
\end{center}
