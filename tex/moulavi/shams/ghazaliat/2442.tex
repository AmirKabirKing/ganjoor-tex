\begin{center}
\section*{غزل شماره ۲۴۴۲: بانکی عجب از آسمان در می‌رسد هر ساعتی}
\label{sec:2442}
\addcontentsline{toc}{section}{\nameref{sec:2442}}
\begin{longtable}{l p{0.5cm} r}
بانکی عجب از آسمان در می‌رسد هر ساعتی
&&
می‌نشنود آن بانگ را الا که صاحب حالتی
\\
ای سر فروبرده چو خر زین آب و سبزه بس مچر
&&
یک لحظه‌ای بالا نگر تا بوک بینی آیتی
\\
ساقی در این آخرزمان بگشاد خم آسمان
&&
از روح او را لشکری وز راح او را رایتی
\\
کو شیرمردی در جهان تا شیرگیر او شود
&&
شاه و فتی باید شدن تا باده نوشی یا فتی
\\
بیچاره گوش مشترک کو نشنود بانگ فلک
&&
بیچاره جان بی‌مزه کز حق ندارد راحتی
\\
آخر چه باشد گر شبی از جان برآری یاربی
&&
بیرون جهی از گور تن و اندرروی در ساحتی
\\
از پا گشایی ریسمان تا برپری بر آسمان
&&
چون آسمان ایمن شوی از هر شکست و آفتی
\\
از جان برآری یک سری ایمن ز شمشیر اجل
&&
باغی درآیی کاندر او نبود خزان را غارتی
\\
خامش کنم خامش کنم تا عشق گوید شرح خود
&&
شرحی خوشی جان پروری کان را نباشد غایتی
\\
\end{longtable}
\end{center}
