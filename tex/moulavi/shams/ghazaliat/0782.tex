\begin{center}
\section*{غزل شماره ۷۸۲: خبرت هست که در شهر شکر ارزان شد}
\label{sec:0782}
\addcontentsline{toc}{section}{\nameref{sec:0782}}
\begin{longtable}{l p{0.5cm} r}
خبرت هست که در شهر شکر ارزان شد
&&
خبرت هست که دی گم شد و تابستان شد
\\
خبرت هست که ریحان و قرنفل در باغ
&&
زیر لب خنده زنانند که کار آسان شد
\\
خبرت هست که بلبل ز سفر بازرسید
&&
در سماع آمد و استاد همه مرغان شد
\\
خبرت هست که در باغ کنون شاخ درخت
&&
مژده نو بشنید از گل و دست افشان شد
\\
خبرت هست که جان مست شد از جام بهار
&&
سرخوش و رقص کنان در حرم سلطان شد
\\
خبرت هست که لاله رخ پرخون آمد
&&
خبرت هست که گل خاصبک دیوان شد
\\
خبرت هست ز دزدی دی دیوانه
&&
شحنه عدل بهار آمد او پنهان شد
\\
بستدند آن صنمان خط عبور از دیوان
&&
تا زمین سبز شد و باسر و باسامان شد
\\
شاهدان چمن ار پار قیامت کردند
&&
هر یک امسال به زیبایی صد چندان شد
\\
گلرخانی ز عدم چرخ زنان آمده‌اند
&&
کانجم چرخ نثار قدم ایشان شد
\\
ناظر ملک شد آن نرگس معزول شده
&&
غنچه طفل چو عیسی فطن و خط خوان شد
\\
بزم آن عشرتیان بار دگر زیب گرفت
&&
باز آن باد صبا باده ده بستان شد
\\
نقش‌ها بود پس پرده دل پنهانی
&&
باغ‌ها آینه سر دل ایشان شد
\\
آنچ بینی تو ز دل جوی ز آیینه مجوی
&&
آینه نقش شود لیک نتاند جان شد
\\
مردگان چمن از دعوت حق زنده شدند
&&
کفرهاشان همه از رحمت حق ایمان شد
\\
باقیان در لحدند و همه جنبان شده‌اند
&&
زانک زنده نتواند گرو زندان شد
\\
گفت بس کن که من این را به از این شرح کنم
&&
من دهان بستم کو آمد و پایندان شد
\\
هم لب شاه بگوید صفت جمله تمام
&&
گر خلاصه ز شما در کنف کتمان شد
\\
\end{longtable}
\end{center}
