\begin{center}
\section*{غزل شماره ۱۸۹۲: هر شب که بود قاعده سفره نهادن}
\label{sec:1892}
\addcontentsline{toc}{section}{\nameref{sec:1892}}
\begin{longtable}{l p{0.5cm} r}
هر شب که بود قاعده سفره نهادن
&&
ما را ز خیال تو بود روزه گشادن
\\
ای لطف تو را قاعده بر روزه گشایان
&&
مانند مسیحا ز فلک مایده دادن
\\
چون قوت دل از مطبخ سودای تو باشد
&&
باید به میان رفتن و در لوت فتادن
\\
ما را هم از آن آتش دل آب حیات است
&&
بر آتش دل شاد بسوزیم چو لادن
\\
کار حیوان است نه کار دل و جان است
&&
در خاک بپوسیدن و از خاک بزادن
\\
\end{longtable}
\end{center}
