\begin{center}
\section*{غزل شماره ۲۷۵: ابصرت روحی ملیحا زلزلت زلزالها}
\label{sec:0275}
\addcontentsline{toc}{section}{\nameref{sec:0275}}
\begin{longtable}{l p{0.5cm} r}
ابصرت روحی ملیحا زلزلت زلزالها
&&
انعطش روحی فقلت ویح روحی مالها
\\
ذاق من شعشاع خمر العشق روحی جرعه
&&
طار فی جو الهوی و استقلعت اثقالها
\\
صار روحی فی هواه غارقا حتی دری
&&
لو تلقاه ضریر تائه احوالها
\\
فی الهوی من لیس فی الکونین بدر مثله
&&
ان روحی فی الهوی من لا تری امثالها
\\
لم تمل روحی الی مال الی ان اعشقت
&&
رامت الاموال کی تنثر له اموالها
\\
لم تزل سفن الهوی تجری بها مذ اصبحت
&&
فی بحار العز و الاقبال یوما یالها
\\
عین روحی قد اصابتها فاردتها بها
&&
حین عدت فضلها و استکثرت اعمالها
\\
افلحت من بعد هلک ان اعوان الهوی
&&
اعتنوا فی امرها ان خففوا حمالها
\\
آه روحی من هوی صدر کبیر فائق
&&
کل مدح قالها فیه ازدرت اقوالها
\\
ییاس النفس اللقاء من وصال فائت
&&
حین تتلو فی کتاب الغیب من افعالها
\\
حبذا احسان مولی عاد روحا اذ نفث
&&
ناولتها شربه صفی لها احوالها
\\
ان روحی تقشع اللقیات فی الماضی مدا
&&
ثم لا تبصر مضی اذ تفکر استقبالها
\\
اختفی العشق الثقیل فی ضمیری دره
&&
ان روحی اثقلت من دره قد شالها
\\
مثله ان اثقل الیوم المخاض حره
&&
اوقعتها فی ردی لم تغنها احجالها
\\
غیر ان سیدا جادت لها الطافه
&&
ان روحی ربوه و استنزلت اطلالها
\\
سیدا مولی عزیزا کاملا فی امره
&&
شمس دین مالک اوفت لها آمالها
\\
صادف المولی بروحی و هی فی ذاک الردی
&&
من زمان اکرمته ما رات اذلالها
\\
جاء من تبریز سربال نسیج بالهوی
&&
اکتست روحی صباحا انزعت سربالها
\\
قالت الروح افتخارا اصطفانا فضله
&&
ثم غارت بعد حین من مقال نالها
\\
\end{longtable}
\end{center}
