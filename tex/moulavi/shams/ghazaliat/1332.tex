\begin{center}
\section*{غزل شماره ۱۳۳۲: توبه سفر گیرد با پای لنگ}
\label{sec:1332}
\addcontentsline{toc}{section}{\nameref{sec:1332}}
\begin{longtable}{l p{0.5cm} r}
توبه سفر گیرد با پای لنگ
&&
صبر فروافتد در چاه تنگ
\\
جز من و ساقی بنماند کسی
&&
چون کند آن چنگ ترنگاترنگ
\\
عقل چو این دید برون جست و رفت
&&
با دل دیوانه که کردست جنگ
\\
صدر خرابات کسی را بود
&&
کو رهد از صدر و ز نام و ز ننگ
\\
هر کی ز اندیشه دلارام ساخت
&&
کشتی برساخت ز پشت نهنگ
\\
و آنک در اندیشه یک جو زر است
&&
او خر پالان بود و پالهنگ
\\
یار منی زود فروجه ز خر
&&
خر بفروش و برهان بی‌درنگ
\\
کون خری دنب خری گیر و رو
&&
رو که کلیدی نبود در مدنگ
\\
راز مگو پیش خران ای مسیح
&&
باده ستان از کف ساقی شنگ
\\
\end{longtable}
\end{center}
