\begin{center}
\section*{غزل شماره ۱۷۶۹: ای دل صافی دم ثابت قدم}
\label{sec:1769}
\addcontentsline{toc}{section}{\nameref{sec:1769}}
\begin{longtable}{l p{0.5cm} r}
ای دل صافی دم ثابت قدم
&&
جئت لکی تنذر خیر الامم
\\
سر ننهی جز به اشارات دل
&&
بر ورق عشق ازل چون قلم
\\
از طرب باد تو و داد تو
&&
رقص کنانیم چو شقه علم
\\
رقص کنان خواجه کجا می روی
&&
سوی گشایشگه عرصه عدم
\\
خواجه کدامین عدم است این بگو
&&
گوش قدم داند حرف قدم
\\
عشق غریب است و زبانش غریب
&&
همچو غریب عربی در عجم
\\
خیز که آورده امت قصه‌ای
&&
بشنو از بنده نه بیش و نه کم
\\
بشنو این حرف غریبانه را
&&
قصه غریب آمد و گوینده هم
\\
از رخ آن یوسف شد قعر چاه
&&
روشن و فرخنده چو باغ ارم
\\
قصر شد آن حبس و در او باغ و راغ
&&
جنت و ایوان شد و صفه حرم
\\
همچو کلوخی که در آب افکنی
&&
باز شود آب در آن دم ز هم
\\
همچو شب ابر که خورشید صبح
&&
ناگه سر برزند از چاه غم
\\
همچو شرابی که عرب خورد و گفت
&&
صل علی دنتها و ارتسم
\\
از طرب این حبس به خواری و نقص
&&
می نگرد بر فلک محتشم
\\
ای خرد از رشک دهانم مگیر
&&
قد شهد الله و عد النعم
\\
گر چه درخت آب نهان می خورد
&&
بان علی شعبته ما کتم
\\
هر چه بدزدید زمین ز آسمان
&&
فصل بهاران بدهد دم به دم
\\
گر شبه دزدیده‌ای وگر گهر
&&
ور علم افراشتی وگر قلم
\\
رفت شب و روز تو اینک رسید
&&
سوف یری النائم ماذا احتلم
\\
\end{longtable}
\end{center}
