\begin{center}
\section*{غزل شماره ۲۲۷۸: این کیست این این کیست این شیرین و زیبا آمده}
\label{sec:2278}
\addcontentsline{toc}{section}{\nameref{sec:2278}}
\begin{longtable}{l p{0.5cm} r}
این کیست این این کیست این شیرین و زیبا آمده
&&
سرمست و نعلین در بغل در خانه ما آمده
\\
خانه در او حیران شده اندیشه سرگردان شده
&&
صد عقل و جان اندر پیش بی‌دست و بی‌پا آمده
\\
آمد به مکر آن لعل لب کفچه به کف آتش طلب
&&
تا خود که را سوزد عجب آن یار تنها آمده
\\
ای معدن آتش بیا آتش چه می‌جویی ز ما
&&
والله که مکر است و دغا ای ناگه این جا آمده
\\
روپوش چون پوشد تو را ای روی تو شمس الضحی
&&
ای کنج و خانه از رخت چون دشت و صحرا آمده
\\
ای یوسف از بالای چه بر آب چه زد عکس تو
&&
آن آب چه از عشق تو جوشیده بالا آمده
\\
شاد آمدی شاد آمدی جادو و استاد آمدی
&&
چون هدهد پیغامبری از پیش عنقا آمده
\\
ای آب حیوان در جگر هر جور تو صد من شکر
&&
هر لحظه‌ای شکلی دگر از رب اعلا آمده
\\
ای دلنواز و دلبری کاندرنگنجی در بری
&&
ای چشم ما از گوهرت افزون ز دریا آمده
\\
چرخ و زمین آیینه‌ای وز عکس ماه روی تو
&&
آن آینه زنده شده و اندر تماشا آمده
\\
خاموش کن خاموش کن از راه دیگر جوش کن
&&
ای دود آتش‌های تو سودای سرها آمده
\\
\end{longtable}
\end{center}
