\begin{center}
\section*{غزل شماره ۳۱۳: رباب مشرب عشقست و مونس اصحاب}
\label{sec:0313}
\addcontentsline{toc}{section}{\nameref{sec:0313}}
\begin{longtable}{l p{0.5cm} r}
رباب مشرب عشقست و مونس اصحاب
&&
که ابر را عربان نام کرده‌اند رباب
\\
چنانک ابر سقای گل و گلستانست
&&
رباب قوت ضمیرست و ساقی الباب
\\
در آتشی بدمی شعله‌ها برافزود
&&
بجز غبار نخیزد چو دردمی به تراب
\\
رباب دعوت بازست سوی شه بازآ
&&
به طبل باز نیاید به سوی شاه غراب
\\
گشایش گره مشکلات عشاقست
&&
چو مشکلیش نباشد چه درخورست جواب
\\
جواب مشکل حیوان گیاه آمد و کاه
&&
که تخم شهوت او شد خمیرمایه خواب
\\
خر از کجا و دم عشق عیسوی ز کجا
&&
که این گشاد ندادش مفتح الابواب
\\
که عشق خلعت جانست و طوق کرمنا
&&
برای ملک وصال و برای رفع حجاب
\\
به بانگ او همه دل‌ها به یک مهم آیند
&&
ندای رب برهاند ز تفرقه ارباب
\\
ز عشق کم گو با جسمیان که ایشان را
&&
وظیفه خوف و رجا آمد و ثواب و عقاب
\\
\end{longtable}
\end{center}
