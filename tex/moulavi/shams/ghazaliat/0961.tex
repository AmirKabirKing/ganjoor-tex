\begin{center}
\section*{غزل شماره ۹۶۱: جهان را بدیدم وفایی ندارد}
\label{sec:0961}
\addcontentsline{toc}{section}{\nameref{sec:0961}}
\begin{longtable}{l p{0.5cm} r}
جهان را بدیدم وفایی ندارد
&&
جهان در جهان آشنایی ندارد
\\
در این قرص زرین بالا تو منگر
&&
که در اندرون بوریایی ندارد
\\
بس ابله شتابان شده سوی دامش
&&
چو کوری که در کف عصایی ندارد
\\
بر او گشته ترسان بر او گشته لرزان
&&
زهی علتی کان دوایی ندارد
\\
نموده جمالی ولی زیر چادر
&&
عجوزی قبیحی لقایی ندارد
\\
کسی سر نهد بر فسونش که چون مار
&&
ز عقل و ز دین دست و پایی ندارد
\\
کسی جان دهد در رهش کز شقاوت
&&
ز جانان ره جان فزایی ندارد
\\
چه مردار مسی که مرد او ز مسی
&&
که پنداشت کو کیمیایی ندارد
\\
برای خیالی شده چون خیالی
&&
بجز درد و رنج و عنایی ندارد
\\
چرا جان نکارد به درگاه معشوق
&&
عجب عشق خود اصطفایی ندارد
\\
چه شاهان که از عشق صد ملک بردند
&&
که آن سلطنت منتهایی ندارد
\\
چه تقصیر کردست این عشق با تو
&&
که منکر شدی کو عطایی ندارد
\\
به یک دردسر زو تو پا را کشیدی
&&
چه ره دیده‌ای کان بلایی ندارد
\\
خمش کن نثارست بر عاشقانش
&&
گهرها که هر یک بهایی ندارد
\\
\end{longtable}
\end{center}
