\begin{center}
\section*{غزل شماره ۲۲۱۴: خنک آن دم که نشینیم در ایوان من و تو}
\label{sec:2214}
\addcontentsline{toc}{section}{\nameref{sec:2214}}
\begin{longtable}{l p{0.5cm} r}
خنک آن دم که نشینیم در ایوان من و تو
&&
به دو نقش و به دو صورت به یکی جان من و تو
\\
داد باغ و دم مرغان بدهد آب حیات
&&
آن زمانی که درآییم به بستان من و تو
\\
اختران فلک آیند به نظاره ما
&&
مه خود را بنماییم بدیشان من و تو
\\
من و تو بی‌من و تو جمع شویم از سر ذوق
&&
خوش و فارغ ز خرافات پریشان من و تو
\\
طوطیان فلکی جمله شکرخوار شوند
&&
در مقامی که بخندیم بدان سان من و تو
\\
این عجبتر که من و تو به یکی کنج این جا
&&
هم در این دم به عراقیم و خراسان من و تو
\\
به یکی نقش بر این خاک و بر آن نقش دگر
&&
در بهشت ابدی و شکرستان من و تو
\\
\end{longtable}
\end{center}
