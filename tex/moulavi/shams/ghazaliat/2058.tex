\begin{center}
\section*{غزل شماره ۲۰۵۸: با رخ چون مشعله بر در ما کیست آن}
\label{sec:2058}
\addcontentsline{toc}{section}{\nameref{sec:2058}}
\begin{longtable}{l p{0.5cm} r}
با رخ چون مشعله بر در ما کیست آن
&&
هر طرفی موج خون نیم شبان چیست آن
\\
در کفن خویشتن رقص کنان مردگان
&&
نفخه صور است یا عیسی ثانی است آن
\\
سینه خود باز کن روزن دل درنگر
&&
کآتش تو شعله زد نی خبر دی است آن
\\
آتش نو را ببین زود درآ چون خلیل
&&
گر چه به شکل آتش است باده صافی است آن
\\
یونس قدسی تویی در تن چون ماهیی
&&
بازشکاف و ببین کاین تن ماهی است آن
\\
دلق تن خویش را بر گرو می‌بنه
&&
پاک شوی پاکباز نوبت پاکی است آن
\\
باده کشیدی ولیک در قدحت باقی است
&&
حمله دیگر که اصل جرعه باقی است آن
\\
دشنه تیز ار خلیل بنهد بر گردنت
&&
رو بمگردان که آن شیوه شاهی است آن
\\
حکم به هم درشکست هست قضا در خطر
&&
فتنه حکم است این آفت قاضی است آن
\\
نفس تو امروز اگر وعده فردا دهد
&&
بر دهنش زن از آنک مردک لافی است آن
\\
باده فروشد ولیک باده دهد جمله باد
&&
خم نماید ولیک حق نمک نیست آن
\\
ما ز زمستان نفس برف تن آورده‌ایم
&&
بهر تقاضای لطف نکته کاجی است آن
\\
مفخر تبریزیان شمس حق ای پیش تو
&&
طاق و طرنب دو کون طفلی و بازی است آن
\\
\end{longtable}
\end{center}
