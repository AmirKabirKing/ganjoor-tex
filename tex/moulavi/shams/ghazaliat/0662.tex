\begin{center}
\section*{غزل شماره ۶۶۲: اگر عالم همه پرخار باشد}
\label{sec:0662}
\addcontentsline{toc}{section}{\nameref{sec:0662}}
\begin{longtable}{l p{0.5cm} r}
اگر عالم همه پرخار باشد
&&
دل عاشق همه گلزار باشد
\\
وگر بی‌کار گردد چرخ گردون
&&
جهان عاشقان بر کار باشد
\\
همه غمگین شوند و جان عاشق
&&
لطیف و خرم و عیار باشد
\\
به عاشق ده تو هر جا شمع مرده‌ست
&&
که او را صد هزار انوار باشد
\\
وگر تنهاست عاشق نیست تنها
&&
که با معشوق پنهان یار باشد
\\
شراب عاشقان از سینه جوشد
&&
حریف عشق در اسرار باشد
\\
به صد وعده نباشد عشق خرسند
&&
که مکر دلبران بسیار باشد
\\
وگر بیمار بینی عاشقی را
&&
نه شاهد بر سر بیمار باشد
\\
سوار عشق شو وز ره میندیش
&&
که اسب عشق بس رهوار باشد
\\
به یک حمله تو را منزل رساند
&&
اگر چه راه ناهموار باشد
\\
علف خواری نداند جان عاشق
&&
که جان عاشقان خمار باشد
\\
ز شمس الدین تبریزی بیابی
&&
دلی کو مست و بس هشیار باشد
\\
\end{longtable}
\end{center}
