\begin{center}
\section*{غزل شماره ۹۴: زهی عشق زهی عشق که ما راست خدایا}
\label{sec:0094}
\addcontentsline{toc}{section}{\nameref{sec:0094}}
\begin{longtable}{l p{0.5cm} r}
زهی عشق زهی عشق که ما راست خدایا
&&
چه نغزست و چه خوبست چه زیباست خدایا
\\
از آن آب حیاتست که ما چرخ زنانیم
&&
نه از کف و نه از نای نه دف‌هاست خدایا
\\
یقین گشت که آن شاه در این عرس نهانست
&&
که اسباب شکرریز مهیاست خدایا
\\
به هر مغز و دماغی که درافتاد خیالش
&&
چه مغزست و چه نغزست چه بیناست خدایا
\\
تن ار کرد فغانی ز غم سود و زیانی
&&
ز تست آنک دمیدن نه ز سرناست خدایا
\\
نی تن را همه سوراخ چنان کرد کف تو
&&
که شب و روز در این ناله و غوغاست خدایا
\\
نی بیچاره چه داند که ره پرده چه باشد
&&
دم ناییست که بیننده و داناست خدایا
\\
که در باغ و گلستان ز کر و فر مستان
&&
چه نورست و چه شورست چه سوداست خدایا
\\
ز تیه خوش موسی و ز مایده عیسی
&&
چه لوتست و چه قوتست و چه حلواست خدایا
\\
از این لوت و زین قوت چه مستیم و چه مبهوت
&&
که از دخل زمین نیست ز بالاست خدایا
\\
ز عکس رخ آن یار در این گلشن و گلزار
&&
به هر سو مه و خورشید و ثریاست خدایا
\\
چو سیلیم و چو جوییم همه سوی تو پوییم
&&
که منزلگه هر سیل به دریاست خدایا
\\
بسی خوردم سوگند که خاموش کنم لیک
&&
مگر هر در دریای تو گویاست خدایا
\\
خمش ای دل که تو مستی مبادا به جهانی
&&
نگهش دار ز آفت که برجاست خدایا
\\
ز شمس الحق تبریز دل و جان و دو دیده
&&
سراسیمه و آشفته سوداست خدایا
\\
\end{longtable}
\end{center}
