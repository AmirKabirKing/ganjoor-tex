\begin{center}
\section*{غزل شماره ۷۸۴: عید بگذشت و همه خلق سوی کار شدند}
\label{sec:0784}
\addcontentsline{toc}{section}{\nameref{sec:0784}}
\begin{longtable}{l p{0.5cm} r}
عید بگذشت و همه خلق سوی کار شدند
&&
زیرکان از پی سرمایه به بازار شدند
\\
عاشقان را چو همه پیشه و بازار تویی
&&
عاشقان از جز بازار تو بیزار شدند
\\
سفها سوی مجالس گرو فرج و گلو
&&
فقها سوی مدارس پی تکرار شدند
\\
همه از سلسله عشق تو دیوانه شدند
&&
همه از نرگس مخمور تو خمار شدند
\\
دست و پاشان تو شکستی چو نه پا ماند و نه دست
&&
پر گشادند و همه جعفر طیار شدند
\\
صدقات شه ما حصه درویشانست
&&
عاشقان حصه بر آن رخ و رخسار شدند
\\
ما چو خورشیدپرستان همه صحرا کوبیم
&&
سایه جویان چو زنان در پس دیوار شدند
\\
تو که در سایه مخلوقی و او دیواریست
&&
ور نه ز آسیب اجل چون همه مردار شدند
\\
جان چه کار آید اگر پیش تو قربان نشود
&&
جان کنون شد که چو منصور سوی دار شدند
\\
همه سوگند بخورده که دگر دم نزنند
&&
مست گشتند صبوحی سوی گفتار شدند
\\
\end{longtable}
\end{center}
