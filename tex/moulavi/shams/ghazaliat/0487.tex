\begin{center}
\section*{غزل شماره ۴۸۷: چو عید و چون عرفه عارفان این عرفات}
\label{sec:0487}
\addcontentsline{toc}{section}{\nameref{sec:0487}}
\begin{longtable}{l p{0.5cm} r}
چو عید و چون عرفه عارفان این عرفات
&&
به هر که قدر تو دانست می‌دهند برات
\\
هلال وار ز راه دراز می‌آیند
&&
برای کارگزاری ز قاضی الحاجات
\\
به مفلسان که ز بازارشان نصیبی نیست
&&
ز مخزن زر سلطان همی‌کشند زکات
\\
پی گشادن درهای بسته می‌آیند
&&
گرفته زیر بغل‌ها کلیدهای نجات
\\
به دست هر جان زنبیل زفت می‌آید
&&
شنیده بانگ تعالو لتأخذوا الصدقات
\\
بیا بیا گذری کن ببین زکات ملک
&&
به طور موسی عمران و غلغل میقات
\\
دریده پهلوی همیان از آن زر بسیار
&&
دریده قوصره‌هاشان ز بار قند و نبات
\\
ز خرمن دو جهان مور خود چه تاند برد
&&
خمش کن و بنشین دور و می‌شنو صلوات
\\
\end{longtable}
\end{center}
