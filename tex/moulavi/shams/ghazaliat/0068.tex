\begin{center}
\section*{غزل شماره ۶۸: چو شست عشق در جانم شناسا گشت شستش را}
\label{sec:0068}
\addcontentsline{toc}{section}{\nameref{sec:0068}}
\begin{longtable}{l p{0.5cm} r}
چو شست عشق در جانم شناسا گشت شستش را
&&
به شست عشق دست آورد جان بت پرستش را
\\
به گوش دل بگفت اقبال رست آن جان به عشق ما
&&
بکرد این دل هزاران جان نثار آن گفت رستش را
\\
ز غیرت چونک جان افتاد گفت اقبال هم نجهد
&&
نشستست این دل و جانم همی‌پاید نجستش را
\\
چو اندر نیستی هستست و در هستی نباشد هست
&&
بیامد آتشی در جان بسوزانید هستش را
\\
برات عمر جان اقبال چون برخواند پنجه شصت
&&
تراشید و ابد بنوشت بر طومار شصتش را
\\
خدیو روح شمس الدین که از بسیاری رفعت
&&
نداند جبرئیل وحی خود جای نشستش را
\\
چو جامش دید این عقلم چو قرابه شد اشکسته
&&
درستی‌های بی‌پایان ببخشید آن شکستش را
\\
چو عشقش دید جانم را به بالای‌یست از این هستی
&&
بلندی داد از اقبال او بالا و پستش را
\\
اگر چه شیرگیری تو دلا می‌ترس از آن آهو
&&
که شیرانند بیچاره مر آن آهوی مستش را
\\
چو از تیغ حیات انگیز زد مر مرگ را گردن
&&
فروآمد ز اسپ اقبال و می‌بوسید دستش را
\\
در آن روزی که در عالم الست آمد ندا از حق
&&
بده تبریز از اول بلی گویان الستش را
\\
\end{longtable}
\end{center}
