\begin{center}
\section*{غزل شماره ۲۰۵۱: می‌بینمت که عزم جفا می‌کنی مکن}
\label{sec:2051}
\addcontentsline{toc}{section}{\nameref{sec:2051}}
\begin{longtable}{l p{0.5cm} r}
می‌بینمت که عزم جفا می‌کنی مکن
&&
عزم عتاب و فرقت ما می‌کنی مکن
\\
در مرغزار غیرت چون شیر خشمگین
&&
در خونم ای دو دیده چرا می‌کنی مکن
\\
بخت مرا چو کلک نگون می‌کنی مکن
&&
پشت مرا چو دال دوتا می‌کنی مکن
\\
ای تو تمام لطف خدا و عطای او
&&
خود را نکال و قهر خدا می‌کنی مکن
\\
پیوند کرده‌ای کرم و لطف با دلم
&&
پیوند کرده را چه جدا می‌کنی مکن
\\
آن بیذقی که شاه شده‌ست از رخ خوشت
&&
بازش به مات غم چه گدا می‌کنی مکن
\\
آن بنده‌ای که بدر شد از پرتو رخت
&&
چون ماه نو ز غصه دوتا می‌کنی مکن
\\
گر گبر و مؤمن است چو کشته هوای توست
&&
بر گبر کشته تو چه غزا می‌کنی مکن
\\
بی‌هوش شو چو موسی و همچون عصا خموش
&&
مانند طور تو چه صدا می‌کنی مکن
\\
\end{longtable}
\end{center}
