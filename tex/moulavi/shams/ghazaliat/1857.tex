\begin{center}
\section*{غزل شماره ۱۸۵۷: توقع دارم از لطف تو ای صدر نکوآیین}
\label{sec:1857}
\addcontentsline{toc}{section}{\nameref{sec:1857}}
\begin{longtable}{l p{0.5cm} r}
توقع دارم از لطف تو ای صدر نکوآیین
&&
درون مدرسه حجره به پهلوی شهاب الدین
\\
پیاده قاضیم می خوان درون محکمه قاصد
&&
و یا خود داعی سلطان دعاها را کنم آمین
\\
بدین حیله بگنجانی در آن خانه ربابی را
&&
که نامم را بگردانی نهی نامم فلان الدین
\\
که خلقان صورت و نامند مثال میوه خامند
&&
کی از جانشان خبر باشد که آن تلخ است یا شیرین
\\
وگر حال آورد قاضی سماعش آرزو آید
&&
رباب خوب بنوازم سماعی آرمش شیرین
\\
ز آواز سماع من اقنجی هم شود زنده
&&
سر از تربت برون آرد بکوبد پا کند تحسین
\\
کفن را اندراندازد قوال انداز مستانه
&&
از آن پس مردگان یک یک برون آیند هم در حین
\\
عجب نبود که صورت‌ها بدین آواز برخیزند
&&
که صورت‌های عشق تو درونت زنده شد می بین
\\
ز مردم آن به کار آید کی زنده می شود در تو
&&
و باقی تن غباری دان که پیدا می شود از طین
\\
دلت را هر زمان نقشی تنت یک نقش افسرده
&&
از آن افسرده‌ای که تو بر آنی نه‌ای با این
\\
مرا گوید یکی صورت منم اصل غزل واگو
&&
خمش کردم نشاید داد این خاتم به هر گرگین
\\
\end{longtable}
\end{center}
