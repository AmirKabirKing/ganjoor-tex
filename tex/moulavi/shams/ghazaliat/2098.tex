\begin{center}
\section*{غزل شماره ۲۰۹۸: شب محنت که بد طبیب و تو افکار یاد کن}
\label{sec:2098}
\addcontentsline{toc}{section}{\nameref{sec:2098}}
\begin{longtable}{l p{0.5cm} r}
شب محنت که بد طبیب و تو افکار یاد کن
&&
که ز پای دلت بکند چنان خار یاد کن
\\
چو فتادی به چاه و گو که ببخشید جان نو
&&
به سوی او بیا مرو مکن انکار یاد کن
\\
مکن اندک نبود آن به خدا شک نبود آن
&&
نه به خویش آی اندکی و تو بسیار یاد کن
\\
تو به هنگام یاد کن که چو هنگام بگذرد
&&
تو خوه از گل سخن تراش و خوه از خار یاد کن
\\
چو رسیدی به صدر او تو بدان حق قدر او
&&
چو بدیدی تو بدر او تو ز دیدار یاد کن
\\
تو بدان قدر سوز او برسد باز روز او
&&
ور از آن روز ایمنی تو ز اغیار یاد کن
\\
چه سپاس ار دو نان دهد به طبیبی که جان دهد
&&
چو بزارد که ای طبیب ز بیمار یاد کن
\\
چو طبیبت نمود خرد دل تو آن زمان بمرد
&&
پس از آن بانگ می‌زنی که ز مردار یاد کن
\\
مکن ار چه شدی چنین چو خزان دانه در زمین
&&
ز بهارم حسام دین و ز گلزار یاد کن
\\
اگرت کار چون زر است نه گرو پیش گازر است
&&
گرت امسال گوهر است نه تو از پار یاد کن
\\
چو بدیدی رحیل گل پس اقبال چیست ذل
&&
نه که زنهار او است بس هله زنهار یاد کن
\\
\end{longtable}
\end{center}
