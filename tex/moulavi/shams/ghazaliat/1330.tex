\begin{center}
\section*{غزل شماره ۱۳۳۰: بگردان شراب ای صنم بی‌درنگ}
\label{sec:1330}
\addcontentsline{toc}{section}{\nameref{sec:1330}}
\begin{longtable}{l p{0.5cm} r}
بگردان شراب ای صنم بی‌درنگ
&&
که بزمست و چنگ و ترنگاترنگ
\\
ولی بزم روحست و ساقی غیب
&&
ببویید بوی و نبینید رنگ
\\
تو صحرای دل بین در آن قطره خون
&&
زهی دشت بی‌حد در آن کنج تنگ
\\
در آن بزم قدسند ابدال مست
&&
نه قدسی که افتد به دست فرنگ
\\
چه افرنگ عقلی که بود اصل دین
&&
چو حلقه‌ست بر در در آن کوی و دنگ
\\
ز خشکیست این عقل و دریاست آن
&&
بمانده است بیرون ز بیم نهنگ
\\
بده می گزافه به مستان حق
&&
که نی عربده بینی آن جا نه جنگ
\\
یکی جام بنمودشان در الست
&&
که از جام خورشید دارند ننگ
\\
تو گویی که بی‌دست و شیشه که دید
&&
شراب دلارام و بکنی و بنگ
\\
ببین نیم شب خلق را جمله مست
&&
ز سغراق خواب و ز ساقی زنگ
\\
قطار شتر بین که گشتند مست
&&
ندانند افسار از پالهنگ
\\
خمش کن که اغلب همه باخودند
&&
همه شهر لنگند تو هم بلنگ
\\
ره سیرت شمس تبریز گیر
&&
به جرات چو شیر و به حمله پلنگ
\\
\end{longtable}
\end{center}
