\begin{center}
\section*{غزل شماره ۷۳۲: دی میان عاشقان ساقی و مطرب میر بود}
\label{sec:0732}
\addcontentsline{toc}{section}{\nameref{sec:0732}}
\begin{longtable}{l p{0.5cm} r}
دی میان عاشقان ساقی و مطرب میر بود
&&
در هم افتادیم زیرا زور گیراگیر بود
\\
عقل باتدبیر آمد در میان جوش ما
&&
در چنان آتش چه جای عقل یا تدبیر بود
\\
در شکار بی‌دلان صد دیده جان دام بود
&&
وز کمان عشق پران صد هزاران تیر بود
\\
آهوی می‌تاخت آن جا بر مثال اژدها
&&
بر شمار خاک شیران پیش او نخجیر بود
\\
دیدم آن جا پیرمردی طرفه‌ای روحانیی
&&
چشم او چون طشت خون و موی او چون شیر بود
\\
دیدم آن آهو به ناگه جانب آن پیر تاخت
&&
چرخ‌ها از هم جدا شد گوییا تزویر بود
\\
کاسه خورشید و مه از عربده درهم شکست
&&
چونک ساغرهای مستان نیک باتوفیر بود
\\
روح قدسی را بپرسیدم از آن احوال گفت
&&
بیخودم من می‌ندانم فتنه آن پیر بود
\\
شمس تبریزی تو دانی حالت مستان خویش
&&
بی دل و دستم خداوندا اگر تقصیر بود
\\
\end{longtable}
\end{center}
