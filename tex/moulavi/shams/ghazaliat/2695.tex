\begin{center}
\section*{غزل شماره ۲۶۹۵: متاز ای دل سوی دریای ناری}
\label{sec:2695}
\addcontentsline{toc}{section}{\nameref{sec:2695}}
\begin{longtable}{l p{0.5cm} r}
متاز ای دل سوی دریای ناری
&&
که می‌ترسم که تاب نار ناری
\\
وجودت از نی و دارد نوایی
&&
ز نی هر دم نوایی نو برآری
\\
نیستانت ندارد تاب آتش
&&
وگر چه تو ز نی شهری برآری
\\
میان شهر نی منشین بر آذر
&&
که هر سو شعله اندر شعله داری
\\
اگر نی سوی آتش میل دارد
&&
چو میل رزق سوی رزق خواری
\\
نیاز آتش است آن میل تنها
&&
که آتش رزق می‌خواهد به زاری
\\
به هر چت نی بفرماید تو نی کن
&&
خلاف نی بکن از شهریاری
\\
خلافش کردی و نی در کمین است
&&
چو نی کم شد سر دیگر نخاری
\\
پدید آید تو را ناگه وجودی
&&
نه نی دارد نه شکر آنچ داری
\\
یکی نوری لطیفی جان فزایی
&&
در او می‌های گوناگون کاری
\\
گشایی پر و بالی کز حلاوت
&&
نمایی لطف‌های لاله زاری
\\
میان این چنین نوری نماید
&&
دگر خورشید و جان‌ها چون ذراری
\\
به نور او بسوزی پر خود را
&&
ز شیرینی نورش گردی عاری
\\
ز ناله واشکافد قرص خورشید
&&
که گل گل وادهد هم خار خاری
\\
زبان واماند زین پس از بیانش
&&
زبان را کار نقش است و نگاری
\\
نگار و نقش چون گلبرگ باشد
&&
گدازیده شود چون آب واری
\\
بر آن ساحل که‌ای‌ن گل‌ها گدازید
&&
اگر خواهی تو مستی و خماری
\\
همی‌گو نام شمس الدین تبریز
&&
کز او این کارها را برگزاری
\\
\end{longtable}
\end{center}
