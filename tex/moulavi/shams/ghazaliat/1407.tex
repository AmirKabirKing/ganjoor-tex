\begin{center}
\section*{غزل شماره ۱۴۰۷: دوش چه خورده‌ای بگو ای بت همچو شکرم}
\label{sec:1407}
\addcontentsline{toc}{section}{\nameref{sec:1407}}
\begin{longtable}{l p{0.5cm} r}
دوش چه خورده‌ای بگو ای بت همچو شکرم
&&
تا همه سال روز و شب باقی عمر از آن خورم
\\
گر تو غلط دهی مرا رنگ تو غمز می کند
&&
رنگ تو تا بدیده‌ام دنگ شده‌ست این سرم
\\
یک نفسی عنان بکش تیز مرو ز پیش من
&&
تا بفروزد این دلم تا به تو سیر بنگرم
\\
سخت دلم همی‌تپد یک نفسی قرار کن
&&
خون ز دو دیده می چکد تیز مرو ز منظرم
\\
چون ز تو دور می شوم عبرت خاک تیره‌ام
&&
چونک ببینمت دمی رونق چرخ اخضرم
\\
چون رخ آفتاب شد دور ز دیده زمین
&&
جامه سیاه می کند شب ز فراق لاجرم
\\
خور چو به صبح سر زند جامه سپید می کند
&&
ای رخت آفتاب جان دور مشو ز محضرم
\\
خیره کشی مکن بتا خیره مریز خون من
&&
تنگ دلی مکن بتا درمشکن تو گوهرم
\\
ساغر می خیال تو بر کف من نهاد دی
&&
تا بندیدمت در او میل نشد به ساغرم
\\
داروی فربهی ز تو یافت زمین و آسمان
&&
تربیتی نما مرا از بر خود که لاغرم
\\
ای صنم ستیزه گر مست ستیزه‌ات شکر
&&
جان تو است جان من اختر توست اخترم
\\
چند به دل بگفته‌ام خون بخور و خموش کن
&&
دل کتفک همی‌زند که تو خموش من کرم
\\
\end{longtable}
\end{center}
