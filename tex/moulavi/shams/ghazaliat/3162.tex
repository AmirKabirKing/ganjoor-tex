\begin{center}
\section*{غزل شماره ۳۱۶۲: ای که مستک شدی و می‌گویی}
\label{sec:3162}
\addcontentsline{toc}{section}{\nameref{sec:3162}}
\begin{longtable}{l p{0.5cm} r}
ای که مستک شدی و می‌گویی
&&
تو غریبی و یا از این کویی
\\
مست و بی‌خویش می‌روی چپ و راست
&&
بی چپ و راست را همی‌جویی
\\
نی چپست و نه راست در جانست
&&
آن که جان خسته از پی اویی
\\
ز آن شکر روی اگر بگردانی
&&
اگر نباتی بدانک بدخویی
\\
ور تو دیوی و رو بدو آری
&&
الله الله چه خوب مه رویی
\\
دلم از جا رود چو گویم او
&&
می‌برد جان و دل زهی اویی
\\
هین ز خوهای او یکی بشنو
&&
گاه شیری کند گه آهویی
\\
در ره او نماند پای مرا
&&
زانوم را نماند زانویی
\\
جز به چوگان او مغلطان سر
&&
گر به میدان او یکی گویی
\\
هین خمش کن در این حدیث بازمپیچ
&&
آسمان وار اگر یکی تویی
\\
\end{longtable}
\end{center}
