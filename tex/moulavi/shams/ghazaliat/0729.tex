\begin{center}
\section*{غزل شماره ۷۲۹: اینک آن جویی که چرخ سبز را گردان کند}
\label{sec:0729}
\addcontentsline{toc}{section}{\nameref{sec:0729}}
\begin{longtable}{l p{0.5cm} r}
اینک آن جویی که چرخ سبز را گردان کند
&&
اینک آن رویی که ماه و زهره را حیران کند
\\
اینک آن چوگان سلطانی که در میدان روح
&&
هر یکی گو را به وحدت سالک میدان کند
\\
اینک آن نوحی که لوح معرفت کشتی اوست
&&
هر که در کشتیش ناید غرقه طوفان کند
\\
هر که از وی خرقه پوشد برکشد خرقه فلک
&&
هر که از وی لقمه یابد حکمتش لقمان کند
\\
نیست ترتیب زمستان و بهارت با شهی
&&
بر من این دم را کند دی بر تو تابستان کند
\\
خار و گل پیشش یکی آمد که او از نوک خار
&&
بر یکی کس خار و بر دیگر کسی بستان کند
\\
هر که در آبی گریزد ز امر او آتش شود
&&
هر که در آتش شود از بهر او ریحان کند
\\
من بر این برهان بگویم زانک آن برهان من
&&
گر همه شبهه‌ست او آن شبهه را برهان کند
\\
چه نگری در دیو مردم این نگر کو دم به دم
&&
آدمی را دیو سازد دیو را انسان کند
\\
اینک آن خضری که میرآب حیوان گشته بود
&&
زنده را بخشد بقا و مرده را حیوان کند
\\
گر چه نامش فلسفی خود علت اولی نهد
&&
علت آن فلسفی را از کرم درمان کند
\\
گوهر آیینه کلست با او دم مزن
&&
کو از این دم بشکند چون بشکند تاوان کند
\\
دم مزن با آینه تا با تو او همدم بود
&&
گر تو با او دم زنی او روی خود پنهان کند
\\
کفر و ایمان تو و غیر تو در فرمان اوست
&&
سر مکش از وی که چشمش غارت ایمان کند
\\
هر که نادان ساخت خود را پیش او دانا شود
&&
ور بر او دانش فروشد غیرتش نادان کند
\\
دام نان آمد تو را این دانش تقلید و ظن
&&
صورت عین الیقین را علم القرآن کند
\\
پس ز نومیدی بود کان کور بر درها رود
&&
داروی دیده نجوید جمله ذکر نان کند
\\
این سخن آبیست از دریای بی‌پایان عشق
&&
تا جهان را آب بخشد جسم‌ها را جان کند
\\
هر که چون ماهی نباشد جوید او پایان آب
&&
هر که او ماهی بود کی فکرت پایان کند
\\
گر به فقر و صدق پیش آیی به راه عاشقان
&&
شمس تبریزی تو را همصحبت مردان کند
\\
\end{longtable}
\end{center}
