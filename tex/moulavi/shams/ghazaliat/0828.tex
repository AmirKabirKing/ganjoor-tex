\begin{center}
\section*{غزل شماره ۸۲۸: هر که را اسرار عشق اظهار شد}
\label{sec:0828}
\addcontentsline{toc}{section}{\nameref{sec:0828}}
\begin{longtable}{l p{0.5cm} r}
هر که را اسرار عشق اظهار شد
&&
رفت یاری زانک محو یار شد
\\
شمع افروزان بنه در آفتاب
&&
بنگرش چون محو آن انوار شد
\\
نیست نور شمع هست آن نور شمع
&&
هم نشد آثار و هم آثار شد
\\
همچنان در نور روح این نار تن
&&
هم نشد این نار و هم این نار شد
\\
جوی جویانست و پویان سوی بحر
&&
گم شود چون غرق دریابار شد
\\
تا طلب جنبان بود مطلوب نیست
&&
مطلب آمد آن طلب بی‌کار شد
\\
پس طلب تا هست ناقص بد طلب
&&
چون نماند آگهی سالار شد
\\
هر تن بی‌عشق کو جوید کله
&&
سر ندارد جملگی دستار شد
\\
تا ببیند ناگهانی گلرخی
&&
بر وی آن دستار و سر چون خار شد
\\
همچو من شد در هوای شمس دین
&&
آنک او را در سر این اسرار شد
\\
\end{longtable}
\end{center}
