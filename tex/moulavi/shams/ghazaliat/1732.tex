\begin{center}
\section*{غزل شماره ۱۷۳۲: به حق آنک بخواندی مرا ز گوشه بام}
\label{sec:1732}
\addcontentsline{toc}{section}{\nameref{sec:1732}}
\begin{longtable}{l p{0.5cm} r}
به حق آنک بخواندی مرا ز گوشه بام
&&
اشارتی که بکردی به سر به جای سلام
\\
به حق آنک گشادی کمر که می نروم
&&
که شد قمر کمرت را چو من کمینه غلام
\\
به حق آنک نداند دل خیال اندیش
&&
مثال‌های خیال مرا به وقت پیام
\\
به حق آنک به فراش گفته‌ای که بروب
&&
ز چند گنده بغل خانه را برای کرام
\\
به حق آنک گزیدی دو لب که جام بگیر
&&
بنوش جام رها کن حدیث پخته و خام
\\
به حق آنک تو را دیدم و قلم افتاد
&&
ز دست عشق نویسم به پیش تو ناکام
\\
به حق آنک گمان‌های بد فرستی تو
&&
به هدهدی که بخواهی که جان ببر زین دام
\\
به حق حلقه رندان که باده می نوشند
&&
به پیش خلق هویدا میان روز صیام
\\
هزار شیشه شکستند و روزه شان نشکست
&&
از آنک شیشه گر عشق ساخته‌ست آن جام
\\
به ماه روزه جهودانه می مخور تو به شب
&&
بیا به بزم محمد مدام نوش مدام
\\
میان گفت بدم من که سست خندیدی
&&
که ای سلیم دل آخر کشیده دار لگام
\\
بگفتمش چو دهان مرا نمی‌دوزی
&&
بدوز گوش کسی را که نیست یار تمام
\\
به حق آنک حلال است خون من بر تو
&&
که بر عدو سخنم را حرام دار حرام
\\
خیال من ز ملاقات شمس تبریزی
&&
هزار صورت بیند عجب پی اعلام
\\
\end{longtable}
\end{center}
