\begin{center}
\section*{غزل شماره ۲۶۷۶: سبک بنواز ای مطرب ربایی}
\label{sec:2676}
\addcontentsline{toc}{section}{\nameref{sec:2676}}
\begin{longtable}{l p{0.5cm} r}
سبک بنواز ای مطرب ربایی
&&
بگردان زوتر ای ساقی شرابی
\\
که آورد آن پری رو رنگ دیگر
&&
ز چشمه زندگی جوشید آبی
\\
چه آتش زد نهان دلبر به دل‌ها
&&
که مجلس پر شد از بوی کبابی
\\
چرا ای پیر مجلس چنگ پرفن
&&
نگویی ناله نی را جوابی
\\
نی نه چشم زان چشمان چه گوید
&&
چنین بیدار باشد مست خوابی
\\
دل سنگین چو یابد تاب آن چشم
&&
شود در حال او در خوشابی
\\
گدازد هر دو عالم بحر گیرد
&&
چون آن مه رو براندازد نقابی
\\
ایا ساقی به اصحاب سعادت
&&
بده حالی تو باری خمر نابی
\\
قدم تا فرق پر دارید از این می
&&
که بوی شمس تبریزی بیابی
\\
\end{longtable}
\end{center}
