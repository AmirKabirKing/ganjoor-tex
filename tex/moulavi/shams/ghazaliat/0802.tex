\begin{center}
\section*{غزل شماره ۸۰۲: می‌رسد یوسف مصری همه اقرار دهید}
\label{sec:0802}
\addcontentsline{toc}{section}{\nameref{sec:0802}}
\begin{longtable}{l p{0.5cm} r}
می‌رسد یوسف مصری همه اقرار دهید
&&
می‌خرامد چو دو صد تنگ شکر بار دهید
\\
جان بدان عشق سپارید و همه روح شوید
&&
وز پی صدقه از آن رنگ به گلزار دهید
\\
جمع رندان و حریفان همه یک رنگ شدیم
&&
گروی‌ها بستانید و به بازار دهید
\\
تا که از کفر و ز ایمان بنماند اثری
&&
این قدح را ز می‌شرع به کفار دهید
\\
اول این سوختگان را به قدح دریابید
&&
و آخرالامر بدان خواجه هشیار دهید
\\
در کمینست خرد می‌نگرد از چپ و راست
&&
قدح زفت بدان پیرک طرار دهید
\\
هر کی جنس است بر این آتش عشاق نهید
&&
هر چه نقدست به سرفتنه اسرار دهید
\\
کار و بار از سر مستی و خرابی ببرید
&&
خویش را زود به یک بار بدین کار دهید
\\
آتش عشق و جنون چون بزند بر ناموس
&&
سر و دستار به یک ریشه دستار دهید
\\
جان‌ها را بگذارید و در آن حلقه روید
&&
جامه‌ها را بفروشید و به خمار دهید
\\
می فروشیست سیه کار و همه عور شدیم
&&
پیرهن نیست کسی را مگر ایزار دهید
\\
حاش لله که به تن جامه طمع کرده بود
&&
آن بهانه‌ست دل پاک به دلدار دهید
\\
طالب جان صفا جامه چرا می‌خواهد
&&
و آنک برده‌ست تن و جامه به ایثار دهید
\\
عنکبوتیست ز شهوت که تو را پرده کشد
&&
جامه و تن زر و سر جمله به یک بار دهید
\\
تا ببینید پس پرده یکی خورشیدی
&&
شمس تبریز کز او دیده به دیدار دهید
\\
\end{longtable}
\end{center}
