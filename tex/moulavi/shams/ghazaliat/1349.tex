\begin{center}
\section*{غزل شماره ۱۳۴۹: تا نزند آفتاب خیمه نور جلال}
\label{sec:1349}
\addcontentsline{toc}{section}{\nameref{sec:1349}}
\begin{longtable}{l p{0.5cm} r}
تا نزند آفتاب خیمه نور جلال
&&
حلقه مرغان روز کی بزند پر و بال
\\
از نظر آفتاب گشت زمین لاله زار
&&
خانه نشستن کنون هست وبال وبال
\\
تیغ کشید آفتاب خون شفق را بریخت
&&
خون هزاران شفق طلعت او را حلال
\\
چشم گشا عاشقا بر فلک جان ببین
&&
صورت او چون قمر قامت من چون هلال
\\
عرضه کند هر دمی ساغر جام بقا
&&
شیشه شده من ز لطف ساغر او مال مال
\\
چشم پر از خواب بود گفتم شاها شبست
&&
گفت که با روی من شب بود اینک محال
\\
تا که کبود است صبح روز بود در گمان
&&
چونک بشد نیم روز نیست دگر قیل و قال
\\
تیز نظر کن تو نیز در رخ خورشید جان
&&
وز نظر من نگر تا تو ببینی جمال
\\
در لمع قرص او صورت شه شمس دین
&&
زینت تبریز کوست سعد مبارک به فال
\\
\end{longtable}
\end{center}
