\begin{center}
\section*{غزل شماره ۲۴۰: تو جان و جهانی کریما مرا}
\label{sec:0240}
\addcontentsline{toc}{section}{\nameref{sec:0240}}
\begin{longtable}{l p{0.5cm} r}
تو جان و جهانی کریما مرا
&&
چه جان و جهان از کجا تا کجا
\\
که جان خود چه باشد بر عاشقان
&&
جهان خود چه باشد بر اولیا
\\
نه بر پشت گاویست جمله زمین
&&
که در مرغزار تو دارد چرا
\\
در آن کاروانی که کل زمین
&&
یکی گاوبارست و تو ره نما
\\
در انبار فضل تو بس دانه‌هاست
&&
که آن نشکند زیر هفت آسیا
\\
تو در چشم نقاش و پنهان ز چشم
&&
زهی چشم بند و زهی سیمیا
\\
تو را عالمی غیر هجده هزار
&&
زهی کیمیا و زهی کبریا
\\
یکی بیت دیگر بر این قافیه
&&
بگویم بلی وام دارم تو را
\\
که نگزارد این وام را جز فقیر
&&
که فقرست دریای در وفا
\\
غنی از بخیلی غنی مانده‌ست
&&
فقیر از سخاوت فقیر از سخا
\\
\end{longtable}
\end{center}
