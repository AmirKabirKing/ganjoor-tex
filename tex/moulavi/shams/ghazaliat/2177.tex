\begin{center}
\section*{غزل شماره ۲۱۷۷: خزان عاشقان را نوبهار او}
\label{sec:2177}
\addcontentsline{toc}{section}{\nameref{sec:2177}}
\begin{longtable}{l p{0.5cm} r}
خزان عاشقان را نوبهار او
&&
روان ره روان را افتخار او
\\
همه گردن کشان شیردل را
&&
کشیده سوی خود بی‌اختیار او
\\
قطار شیر می‌بینم چو اشتر
&&
به بینیشان درآورده مهار او
\\
مهارش آنک حاجتمندشان کرد
&&
ز خوف و حرصشان کرده نزار او
\\
گران جانتر ز عنصرها نه خاک است
&&
سبک کرد و ببرد از وی قرار او
\\
از آب و آتش و از باد این خاک
&&
سبکتر شد چو برد از وی وقار او
\\
به خاک آن هر سه عنصر را کند صید
&&
به گردون می‌کند آهو شکار او
\\
یکی کاهل نخواهد رست از وی
&&
که یک یک را کند دربند کار او
\\
ز خاک تیره کاهلتر نباشی
&&
به زیر دم او بنهاد خار او
\\
عصا زد بر سر دریا که برجه
&&
برآورد از دل دریا غبار او
\\
عصا را گفت بگذار این عصایی
&&
همی‌پیچد بر خود همچو مار او
\\
برآرد مطبخ معده بخاری
&&
بسازد جان و حسی زان بخار او
\\
ز تف دل دگر جانی بسازد
&&
که تا دارد از آن جان ننگ و عار او
\\
زهی غیرت که بر خود دارد آن شه
&&
که سلطان هم وی است و پرده دار او
\\
زهی عشقی که دارد بر کفی خاک
&&
که گاهش گل کند گه لاله زار او
\\
کند با او به هر دم یک صفت یار
&&
ز جمله بسکلد در اضطرار او
\\
که تا داند که آن‌ها بی‌وفااند
&&
بداند قدر این بگزیده یار او
\\
عجایب یار غاری گردد او را
&&
که یار او باشد و هم یار غار او
\\
زبان بربند و بگشا چشم عبرت
&&
که بگشاده‌ست راه اعتبار او
\\
\end{longtable}
\end{center}
