\begin{center}
\section*{غزل شماره ۱۴۲۹: نه آن بی‌بهره دلدارم که از دلدار بگریزم}
\label{sec:1429}
\addcontentsline{toc}{section}{\nameref{sec:1429}}
\begin{longtable}{l p{0.5cm} r}
نه آن بی‌بهره دلدارم که از دلدار بگریزم
&&
نه آن خنجر به کف دارم کز این پیکار بگریزم
\\
منم آن تخته که با من دروگر کارها دارد
&&
نه از تیشه زبون گردم نه از مسمار بگریزم
\\
مثال تخته بی‌خویشم خلاف تیشه نندیشم
&&
نشایم جز که آتش را گر از نجار بگریزم
\\
چو سنگم خوار و سرد ار من به لعلی کم سفر سازم
&&
چو غارم تنگ و تاری گر ز یار غار بگریزم
\\
نیابم بوس شفتالو چو بگریزم ز بی‌برگی
&&
نبویم مشک تاتاری گر از تاتار بگریزم
\\
از آن از خود همی‌رنجم که منهم در نمی‌گنجم
&&
سزد چون سر نمی‌گنجد گر از دستار بگریزم
\\
هزاران قرن می باید که این دولت به پیش آید
&&
کجا یابم دگربارش اگر این بار بگریزم
\\
نه رنجورم نه نامردم که از خوبان بپرهیزم
&&
نه فاسد معده‌ای دارم که از خمار بگریزم
\\
نیم بر پشت پالانی که در میدان سپس مانم
&&
نیم فلاح این ده من که از سالار بگریزم
\\
همی‌گویم دلا بس کن دلم گوید جواب من
&&
که من در کان زر غرقم چرا ز ایثار بگریزم
\\
\end{longtable}
\end{center}
