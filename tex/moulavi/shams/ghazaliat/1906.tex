\begin{center}
\section*{غزل شماره ۱۹۰۶: تو نقد قلب را از زر برون کن}
\label{sec:1906}
\addcontentsline{toc}{section}{\nameref{sec:1906}}
\begin{longtable}{l p{0.5cm} r}
تو نقد قلب را از زر برون کن
&&
وگر گوید زرم زوتر برون کن
\\
که بیگانه چو سیلاب است دشمن
&&
ز بامش تو بران وز در برون کن
\\
مگس‌ها را ز غیرت ای برادر
&&
از این بزم پر از شکر برون کن
\\
دو چشم خاین نامحرمان را
&&
از آن زیب و جمال فر برون کن
\\
اگر کر نشنود آواز آن چنگ
&&
اگر تانی کری از کر برون کن
\\
چو مستان شیشه اندر دست دارند
&&
دلی کو هست چون مرمر برون کن
\\
نران راه معنی عاشقانند
&&
نر شهوت بود چون خر برون کن
\\
بر یزید است شهوت پر و بالش
&&
از این مرغان نیکو پر برون کن
\\
چو بنده شمس تبریزی نباشد
&&
تو او را آدمی مشمر برون کن
\\
\end{longtable}
\end{center}
