\begin{center}
\section*{غزل شماره ۱۲۸۴: سری برآر که تا ما رویم بر سر عیش}
\label{sec:1284}
\addcontentsline{toc}{section}{\nameref{sec:1284}}
\begin{longtable}{l p{0.5cm} r}
سری برآر که تا ما رویم بر سر عیش
&&
دمی چو جان مجرد رویم در بر عیش
\\
ز مرگ خویش شنیدم پیام عیش ابد
&&
زهی خدا که کند مرگ را پیمبر عیش
\\
به نام عیش بریدند ناف هستی ما
&&
به روز عید بزادیم ما ز مادر عیش
\\
بپرس عیش چه باشد برون شدن زین عیش
&&
که عیش صورت چون حلقه ایست بر در عیش
\\
درون پرده ز ارواح عیش صورت‌هاست
&&
ز عکس ایشان این پرده شد مصور عیش
\\
وجود چون زر خود را به عیش ده نه به غم
&&
که خاک بر سر آن زر که نیست درخور عیش
\\
بگویمت که چرا چرخ می‌زند گردون
&&
کیش به چرخ درآورد تاب اختر عیش
\\
بگویمت که چرا بحر موج در موجست
&&
کیش به رقص درآورد نور گوهر عیش
\\
بگویمت که چرا خاک حور و ولدان زاد
&&
که داد بوی بهشتش نسیم عنبر عیش
\\
بگویمت که چرا باد حرف حرف شدست
&&
که تا ورق ورق آیی سبک ز دفتر عیش
\\
بگویمت که چرا شب تتق فروآویخت
&&
که گرد کست و عروسی بگیرد جا در عیش
\\
بگفتمی سر پنج و چهار و هفت ولیک
&&
به یک دو لعب فرومانده‌ام به شش در عیش
\\
\end{longtable}
\end{center}
