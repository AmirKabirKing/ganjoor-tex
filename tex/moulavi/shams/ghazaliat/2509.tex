\begin{center}
\section*{غزل شماره ۲۵۰۹: بر دیوانگان امروز آمد شاه پنهانی}
\label{sec:2509}
\addcontentsline{toc}{section}{\nameref{sec:2509}}
\begin{longtable}{l p{0.5cm} r}
بر دیوانگان امروز آمد شاه پنهانی
&&
فغان برخاست از جان‌های مجنونان روحانی
\\
میان نعره‌ها بشناخت آواز مرا آن شه
&&
که صافی گشته بود آوازم از انفاس حیوانی
\\
اشارت کرد شاهانه که جست از بند دیوانه
&&
اگر دیوانه‌ام شاها تو دیوان را سلیمانی
\\
شها همراز مرغابی و هم افسون دیوانی
&&
بر این دیوانه هم شاید که افسونی فروخوانی
\\
به پیش شاه شد پیری که بربندش به زنجیری
&&
کز این دیوانه در دیوان بس آشوب است و ویرانی
\\
شه من گفت کاین مجنون به جز زنجیر زلف من
&&
دگر زنجیر نپذیرد تو خوی او نمی‌دانی
\\
هزاران بند بردرد به سوی دست ما پرد
&&
الیناراجعون گردد که او بازی است سلطانی
\\
\end{longtable}
\end{center}
