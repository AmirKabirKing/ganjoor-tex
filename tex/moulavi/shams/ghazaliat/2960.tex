\begin{center}
\section*{غزل شماره ۲۹۶۰: رقصان شو ای قراضه کز اصل اصل کانی}
\label{sec:2960}
\addcontentsline{toc}{section}{\nameref{sec:2960}}
\begin{longtable}{l p{0.5cm} r}
رقصان شو ای قراضه کز اصل اصل کانی
&&
جویای هر چه هستی می‌دانک عین آنی
\\
خورشید رو نماید وز ذره رقص خواهد
&&
آن به که رقص آری دامن همی‌کشانی
\\
روزی کنار گیری ای ذره آفتابی
&&
سر بر برش نهاده این نکته را بدانی
\\
پیش آردت شرابی کای ذره درکش این را
&&
خوردی و محو گشتی در آفتاب جانی
\\
شد ذره آفتابی از خوردن شرابی
&&
در دولت تجلی از طعن لن ترانی
\\
ما میوه‌های خامیم در تاب آفتابت
&&
رقصی کنیم رقصی زیرا تو می‌پزانی
\\
احسنت ای پزیدن شاباش ای مزیدن
&&
از آفتاب جانی کو را نبود ثانی
\\
مخدوم شمس دینم شاهنشهی ز تبریز
&&
تسلیم توست جان‌ها ای جان و دل تو دانی
\\
\end{longtable}
\end{center}
