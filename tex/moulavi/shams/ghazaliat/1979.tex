\begin{center}
\section*{غزل شماره ۱۹۷۹: عاشقان را مژده‌ای از سرفراز راستین}
\label{sec:1979}
\addcontentsline{toc}{section}{\nameref{sec:1979}}
\begin{longtable}{l p{0.5cm} r}
عاشقان را مژده‌ای از سرفراز راستین
&&
مژده مر دل را هزار از دلنواز راستین
\\
مژده مر کان‌های زر را از برای خالصیش
&&
هست نقاد بصیر و هست گاز راستین
\\
مژده مر کسوه بقا را کز پی عمر ابد
&&
هستش از اقبال و دولت‌ها طراز راستین
\\
فرخا زاغی که در زاغی نماند بعد از این
&&
پیش شمس الدین درآید گشت باز راستین
\\
حبذا دستی که او بستم درازی کم کند
&&
دست در فتراک او زد شد دراز راستین
\\
شد دراز آن دست او تا بگذرید او را ختن
&&
تا گرفت از جیب معشوقی طراز راستین
\\
بعد از آن خوب طرازی چون شود همدست او
&&
دو به دو چون مست گشته گفته راز راستین
\\
چشم بگشاید ببیند از ورای وهم و روح
&&
آنک بر ترک طرازی کرد ناز راستین
\\
شاه تبریزی کریمی روح بخشی کاملی
&&
در فرازی در وصال و ملک باز راستین
\\
ملک جانی‌ها نه ملک فانیی جسمانیی
&&
تا شود جان‌ها ز ملکش چشم باز راستین
\\
مرحبا ای شاه جان‌ها مرحبا ای فر و حسن
&&
ملک بخش بندگان و کارساز راستین
\\
\end{longtable}
\end{center}
