\begin{center}
\section*{غزل شماره ۹۳۳: میان باغ گل سرخ‌های و هو دارد}
\label{sec:0933}
\addcontentsline{toc}{section}{\nameref{sec:0933}}
\begin{longtable}{l p{0.5cm} r}
میان باغ گل سرخ‌های و هو دارد
&&
که بو کنید دهان مرا چه بو دارد
\\
به باغ خود همه مستند لیک نی چون گل
&&
که هر یکی به قدح خورد و او سبو دارد
\\
چو سال سال نشاطست و روز روز طرب
&&
خنک مرا و کسی را که عیش خو دارد
\\
چرا مقیم نباشد چو ما به مجلس گل
&&
کسی که ساقی باقی ماه رو دارد
\\
به باغ جمله شراب خدای می‌نوشند
&&
در آن میانه کسی نیست کو گلو دارد
\\
عجایبند درختانش بکر و آبستن
&&
چو مریمی که نه معشوقه و نه شو دارد
\\
هزار بار چمن را بسوخت و بازآراست
&&
چه عشق دارد با ما چه جست و جو دارد
\\
وجود ما و وجود چمن بدو زنده‌ست
&&
زهی وجود لطیف و ظریف کو دارد
\\
چراست خار سلحدار و ابر روی ترش
&&
ز رشک آن که گل سرخ صد عدو دارد
\\
چو آینه‌ست و ترازو خموش و گویا یار
&&
ز من رمیده که او خوی گفت و گو دارد
\\
\end{longtable}
\end{center}
