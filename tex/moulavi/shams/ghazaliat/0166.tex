\begin{center}
\section*{غزل شماره ۱۶۶: چمنی که تا قیامت گل او به بار بادا}
\label{sec:0166}
\addcontentsline{toc}{section}{\nameref{sec:0166}}
\begin{longtable}{l p{0.5cm} r}
چمنی که تا قیامت گل او به بار بادا
&&
صنمی که بر جمالش دو جهان نثار بادا
\\
ز بگاه میر خوبان به شکار می‌خرامد
&&
که به تیر غمزه او دل ما شکار بادا
\\
به دو چشم من ز چشمش چه پیام‌هاست هر دم
&&
که دو چشم از پیامش خوش و پرخمار بادا
\\
در زاهدی شکستم به دعا نمود نفرین
&&
که برو که روزگارت همه بی‌قرار بادا
\\
نه قرار ماند و نی دل به دعای او ز یاری
&&
که به خون ماست تشنه که خداش یار بادا
\\
تن ما به ماه ماند که ز عشق می‌گدازد
&&
دل ما چو چنگ زهره که گسسته تار بادا
\\
به گداز ماه منگر به گسستگی زهره
&&
تو حلاوت غمش بین که یکش هزار بادا
\\
چه عروسیست در جان که جهان ز عکس رویش
&&
چو دو دست نوعروسان تر و پرنگار بادا
\\
به عذار جسم منگر که بپوسد و بریزد
&&
به عذار جان نگر که خوش و خوش عذار بادا
\\
تن تیره همچو زاغی و جهان تن زمستان
&&
که به رغم این دو ناخوش ابدا بهار بادا
\\
که قوام این دو ناخوش به چهار عنصر آمد
&&
که قوام بندگانت به جز این چهار بادا
\\
\end{longtable}
\end{center}
