\begin{center}
\section*{غزل شماره ۲۲۷۵: امروز مستان را نگر در مست ما آویخته}
\label{sec:2275}
\addcontentsline{toc}{section}{\nameref{sec:2275}}
\begin{longtable}{l p{0.5cm} r}
امروز مستان را نگر در مست ما آویخته
&&
افکنده عقل و عافیت و اندر بلا آویخته
\\
گفتم که ای مستان جان می‌خورده از دستان جان
&&
ای صد هزاران جان و دل اندر شما آویخته
\\
گفتند شکر الله را کو جلوه کرد این ماه را
&&
افتاده بودیم از بقا در قعر لا آویخته
\\
بگریختیم از جور او یک مدتی وز دور او
&&
چون دشمنان بودیم ما اندر جفا آویخته
\\
جام وفا برداشته کار و دکان بگذاشته
&&
و افسردگان بی‌مزه در کارها آویخته
\\
بنشسته عقل سرمه کش با هر کی با چشمی است خوش
&&
بنشسته زاغ دیده کش بر هر کجا آویخته
\\
زین خنب‌های تلخ و خوش گر چاشنی داری بچش
&&
ترک هوا خوشتر بود یا در هوا آویخته
\\
عمری دل من در غمش آواره شد می‌جستمش
&&
دیدم دل بیچاره را خوش در خدا آویخته
\\
بر دار دنیا ای فتی گر ایمنی برخیز تا
&&
بنمایم آزادانت را و هم تو را آویخته
\\
بر دار ملک جاودان بین کشتگان زنده جان
&&
مانند منصور جوان در ارتضا آویخته
\\
عشقا تویی سلطان من از بهر من داری بزن
&&
روشن ندارد خانه را قندیل ناآویخته
\\
من خاک پای آن کسم کو دست در مردان زند
&&
جانم غلام آن مسی در کیمیا آویخته
\\
برجه طرب را ساز کن عیش و سماع آغاز کن
&&
خوش نیست آن دف سرنگون نی بی‌نوا آویخته
\\
دف دل گشاید بسته را نی جان فزاید خسته را
&&
این دلگشا چون بسته شد و آن جان فزا آویخته
\\
امروز دستی برگشا ایثار کن جان در سخا
&&
با کفر حاتم رست چون بد در سخا آویخته
\\
هست آن سخا چون دام نان اما صفا چون دام جان
&&
کو در سخا آویخته کو در صفا آویخته
\\
باشد سخی چون خایفی در غار ایثاری شده
&&
صوفی چو بوبکری بود در مصطفی آویخته
\\
این دل دهد در دلبری جان هم سپارد بر سری
&&
و آن صرفه جو چون مشتری اندر بها آویخته
\\
آن چون نهنگ آیان شده دریا در او حیران شده
&&
وین بحری نوآشنا در آشنا آویخته
\\
گویی که این کار و کیا یا صدق باشد یا ریا
&&
آن جا که عشاقند و ما صدق و ریا آویخته
\\
شب گشت ای شاه جهان چشم و چراغ شب روان
&&
ای پیش روی چون مهت ماه سما آویخته
\\
من شادمان چون ماه نو تو جان فزا چون جاه نو
&&
وی در غم تو ماه نو چون من دوتا آویخته
\\
کوه است جان در معرفت تن برگ کاهی در صفت
&&
بر برگ کی دیده است کس یک کوه را آویخته
\\
از ره روان گردی روان صحبت ببر از دیگران
&&
ور نی بمانی مبتلا در مبتلا آویخته
\\
جان عزیزان گشته خون تا عاقبت چون است چون
&&
از بدگمانی سرنگون در انتها آویخته
\\
چون دید جان پاکشان آن تخم کاول کاشت جان
&&
واگشت فکر از انتها در ابتدا آویخته
\\
اصل ندا از دل بود در کوه تن افتد صدا
&&
خاموش رو در اصل کن ای در صدا آویخته
\\
گفت زبان کبر آورد کبرت نیازت را خورد
&&
شو تو ز کبر خود جدا در کبریا آویخته
\\
ای شمس تبریزی برآ از سوی شرق کبریا
&&
جان‌ها ز تو چون ذره‌ها اندر ضیا آویخته
\\
\end{longtable}
\end{center}
