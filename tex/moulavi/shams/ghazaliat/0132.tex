\begin{center}
\section*{غزل شماره ۱۳۲: در میان پرده خون عشق را گلزارها}
\label{sec:0132}
\addcontentsline{toc}{section}{\nameref{sec:0132}}
\begin{longtable}{l p{0.5cm} r}
در میان پرده خون عشق را گلزارها
&&
عاشقان را با جمال عشق بی‌چون کارها
\\
عقل گوید شش جهت حدست و بیرون راه نیست
&&
عشق گوید راه هست و رفته‌ام من بارها
\\
عقل بازاری بدید و تاجری آغاز کرد
&&
عشق دیده زان سوی بازار او بازارها
\\
ای بسا منصور پنهان ز اعتماد جان عشق
&&
ترک منبرها بگفته برشده بر دارها
\\
عاشقان دردکش را در درونه ذوق‌ها
&&
عاقلان تیره دل را در درون انکارها
\\
عقل گوید پا منه کاندر فنا جز خار نیست
&&
عشق گوید عقل را کاندر توست آن خارها
\\
هین خمش کن خار هستی را ز پای دل بکن
&&
تا ببینی در درون خویشتن گلزارها
\\
شمس تبریزی تویی خورشید اندر ابر حرف
&&
چون برآمد آفتابت محو شد گفتارها
\\
\end{longtable}
\end{center}
