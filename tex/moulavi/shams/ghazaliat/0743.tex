\begin{center}
\section*{غزل شماره ۷۴۳: آن زمانی را که چشم از چشم او مخمور بود}
\label{sec:0743}
\addcontentsline{toc}{section}{\nameref{sec:0743}}
\begin{longtable}{l p{0.5cm} r}
آن زمانی را که چشم از چشم او مخمور بود
&&
چون رسیدش چشم بد کز چشم‌ها مستور بود
\\
شادی شب‌های ما کز مشک و عنبر پرده داشت
&&
شادی آن صبح‌ها کز یار پرکافور بود
\\
از فراز عرش و کرسی بانگ تحسین می‌رسید
&&
تا به پشت گاو و ماهی از رخش پرنور بود
\\
هر طرف از حسن او بدلیلیی کاسد شده
&&
ذره ذره همچو مجنون عاشق مشهور بود
\\
دل به پیش روی او چون بایزید اندر مزید
&&
جان در آویزان ز زلفش شیوه منصور بود
\\
شمع عشق افروز را یک بار دیگر اندرآر
&&
کوری آن کس که او از عشرت ما دور بود
\\
ساقیی با رطل آمد مر مرا از کار برد
&&
تا ز مستی من ندانستم که رشک حور بود
\\
نقش شمس الدین تبریزیست جان جان عشق
&&
کاین به دفترهای عشق اندر ازل مسطور بود
\\
\end{longtable}
\end{center}
