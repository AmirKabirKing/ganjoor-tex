\begin{center}
\section*{غزل شماره ۸۱: ای ساقی جان پر کن آن ساغر پیشین را}
\label{sec:0081}
\addcontentsline{toc}{section}{\nameref{sec:0081}}
\begin{longtable}{l p{0.5cm} r}
ای ساقی جان پر کن آن ساغر پیشین را
&&
آن راه زن دل را آن راه بر دین را
\\
زان می که ز دل خیزد با روح درآمیزد
&&
مخمور کند جوشش مر چشم خدابین را
\\
آن باده انگوری مر امت عیسی را
&&
و این باده منصوری مر امت یاسین را
\\
خم‌ها است از آن باده خم‌ها است از این باده
&&
تا نشکنی آن خم را هرگز نچشی این را
\\
آن باده به جز یک دم دل را نکند بی‌غم
&&
هرگز نکشد غم را هرگز نکند کین را
\\
یک قطره از این ساغر کار تو کند چون زر
&&
جانم به فدا باشد این ساغر زرین را
\\
این حالت اگر باشد اغلب به سحر باشد
&&
آن را که براندازد او بستر و بالین را
\\
زنهار که یار بد از وسوسه نفریبد
&&
تا نشکنی از سستی مر عهد سلاطین را
\\
گر زخم خوری بر رو رو زخم دگر می‌جو
&&
رستم چه کند در صف دسته گل و نسرین را
\\
\end{longtable}
\end{center}
