\begin{center}
\section*{غزل شماره ۳۰۶۲: اگر تو مست شرابی چرا حشر نکنی}
\label{sec:3062}
\addcontentsline{toc}{section}{\nameref{sec:3062}}
\begin{longtable}{l p{0.5cm} r}
اگر تو مست شرابی چرا حشر نکنی
&&
وگر شراب نداری چرا خبر نکنی
\\
وگر سه چار قدح از مسیح جان خوردی
&&
ز آسمان چهارم چرا گذر نکنی
\\
از آن کسی که تو مستی چرا جدا باشی
&&
وز آن کسی که خماری چرا حذر نکنی
\\
چو آفتاب چرا تو کلاه کژ ننهی
&&
ز نور خود چو مه نو چرا کمر نکنی
\\
چو آفتاب جمال قدیم تیغ زند
&&
چو کان لعل چرا جان و دل سپر نکنی
\\
وگر چو نای چشیدی ز لعل خوش دم او
&&
چرا چو نی تو جهان را پر از شکر نکنی
\\
وگر چو ابر تو حامل شدی از آن دریا
&&
چرا چو ابر زمین را پر از گهر نکنی
\\
ز گلشن رخ تو گلرخان همی‌جوشند
&&
چرا چو حیز و محنث نه‌ای نظر نکنی
\\
نگر به سبزقبایان باغ کآمده‌اند
&&
به سوی شاه قبابخش چون سفر نکنی
\\
چو خرقه و شجره داری از بهار حیات
&&
چرا سر دل خود جلوه چون شجر نکنی
\\
چو اعتبار ندارد جهان بر درویش
&&
به بزم فقر چرا عیش معتبر نکنی
\\
\end{longtable}
\end{center}
