\begin{center}
\section*{غزل شماره ۶۸۳: ز خاک من اگر گندم برآید}
\label{sec:0683}
\addcontentsline{toc}{section}{\nameref{sec:0683}}
\begin{longtable}{l p{0.5cm} r}
ز خاک من اگر گندم برآید
&&
از آن گر نان پزی مستی فزاید
\\
خمیر و نانبا دیوانه گردد
&&
تنورش بیت مستانه سراید
\\
اگر بر گور من آیی زیارت
&&
تو را خرپشته‌ام رقصان نماید
\\
میا بی‌دف به گور من برادر
&&
که در بزم خدا غمگین نشاید
\\
زنخ بربسته و در گور خفته
&&
دهان افیون و نقل یار خاید
\\
بدری زان کفن بر سینه بندی
&&
خراباتی ز جانت درگشاید
\\
ز هر سو بانگ جنگ و چنگ مستان
&&
ز هر کاری به لابد کار زاید
\\
مرا حق از می عشق آفریده‌ست
&&
همان عشقم اگر مرگم بساید
\\
منم مستی و اصل من می عشق
&&
بگو از می به جز مستی چه آید
\\
به برج روح شمس الدین تبریز
&&
بپرد روح من یک دم نپاید
\\
\end{longtable}
\end{center}
