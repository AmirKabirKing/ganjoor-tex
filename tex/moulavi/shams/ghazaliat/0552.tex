\begin{center}
\section*{غزل شماره ۵۵۲: چیست صلای چاشتگه خواجه به گور می‌رود}
\label{sec:0552}
\addcontentsline{toc}{section}{\nameref{sec:0552}}
\begin{longtable}{l p{0.5cm} r}
چیست صلای چاشتگه خواجه به گور می‌رود
&&
دیر به خانه وارسد منزل دور می‌رود
\\
در عوض بت گزین کزدم و مار همنشین
&&
وز تتق بریشمین سوی قبور می‌رود
\\
شد می و نقل خوردنش عشرت و عیش کردنش
&&
سخت شکست گردنش سخت صبور می‌رود
\\
زهره نداشت هیچ کس تا بر او زند نفس
&&
پخته شود از این سپس چون به تنور می‌رود
\\
صاف صفا نمی‌رود راه وفا نمی‌رود
&&
مست خدا نمی‌رود مست غرور می‌رود
\\
ای خنک آن که پیش شد بنده دین و کیش شد
&&
موسی وقت خویش شد جانب طور می‌رود
\\
چند برید جامه‌ها بست بسی عمامه‌ها
&&
چون که نداشت ستر حق ناکس و عور می‌رود
\\
آنک ز روم زاده بد جانب روم وارود
&&
وان که ز غور زاده بد هم سوی غور می‌رود
\\
آن که ز نار زاده بد همچو بلیس نار شد
&&
وان که ز نور زاده بد هم سوی نور می‌رود
\\
آن که ز دیو زاده بد دست جفا گشاده بد
&&
هیچ گمان مبر که او در بر حور می‌رود
\\
بانمکان و چابکان جانب خوان حق شده
&&
وان دل خام بی‌نمک در شر و شور می‌رود
\\
طبل سیاستی ببین کز فزع نهیب او
&&
شیر چو گربه می‌شود میر چو مور می‌رود
\\
بس که بیان سر تو گر چه به لب نیاوری
&&
همچو خیال نیکوان سوی صدور می‌رود
\\
\end{longtable}
\end{center}
