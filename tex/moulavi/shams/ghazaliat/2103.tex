\begin{center}
\section*{غزل شماره ۲۱۰۳: گر چه اندر فغان و نالیدن}
\label{sec:2103}
\addcontentsline{toc}{section}{\nameref{sec:2103}}
\begin{longtable}{l p{0.5cm} r}
گر چه اندر فغان و نالیدن
&&
اندکی هست خویشتن دیدن
\\
آن نباشد مرا چو در عشقت
&&
خوگرم من به خویش دزدیدن
\\
به خدا و به پاکی ذاتش
&&
پاکم از خویشتن پسندیدن
\\
دیده کی از رخ تو برگردد
&&
به که آید به وقت گردیدن
\\
در چنین دولت و چنین میدان
&&
ننگ باشد ز مرگ لنگیدن
\\
عاشقان تو را مسلم شد
&&
بر همه مرگ‌ها بخندیدن
\\
فرع‌های درخت لرزانند
&&
اصل را نیست خوف لرزیدن
\\
باغبانان عشق را باشد
&&
از دل خویش میوه برچیدن
\\
جان عاشق نواله‌ها می‌پیچ
&&
در مکافات رنج پیچیدن
\\
زهد و دانش بورز ای خواجه
&&
نتوان عشق را بورزیدن
\\
پیش از این گفت شمس تبریزی
&&
لیک کو گوش بهر بشنیدن
\\
\end{longtable}
\end{center}
