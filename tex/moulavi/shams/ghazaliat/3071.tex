\begin{center}
\section*{غزل شماره ۳۰۷۱: دلا همای وصالی بپر چرا نپری}
\label{sec:3071}
\addcontentsline{toc}{section}{\nameref{sec:3071}}
\begin{longtable}{l p{0.5cm} r}
دلا همای وصالی بپر چرا نپری
&&
تو را کسی نشناسد نه آدمی نه پری
\\
تو دلبری نه دلی لیک به هر حیله و مکر
&&
به شکل دل شده‌ای تا هزار دل ببری
\\
دمی به خاک درآمیزی از وفا و دمی
&&
ز عرش و فرش و حدود دو کون برگذری
\\
روان چرات نیابد چو پر و بال ویی
&&
نظر چرات نبیند چو مایه نظری
\\
چه زهره دارد توبه که با تو توبه کند
&&
خبر کی باشد تا با تو ماندش خبری
\\
چه باشد آن مس مسکین چو کیمیا آید
&&
که او فنا نشود از مسی به وصف زری
\\
کیست دانه مسکین چو نوبهار آید
&&
که دانگیش نگردد فنا پی شجری
\\
کیست هیزم مسکین که چون فتد در نار
&&
بدل نگردد هیزم به شعله شرری
\\
ستاره‌هاست همه عقل‌ها و دانش‌ها
&&
تو آفتاب جهانی که پرده شان بدری
\\
جهان چو برف و یخی آمد و تو فصل تموز
&&
اثر نماند از او چون تو شاه بر اثری
\\
کیم بگو من مسکین که با تو من مانم
&&
فنا شوم من و صد من چو سوی من نگری
\\
کمال وصف خداوند شمس تبریزی
&&
گذشته‌ست ز اوهام جبری و قدری
\\
\end{longtable}
\end{center}
