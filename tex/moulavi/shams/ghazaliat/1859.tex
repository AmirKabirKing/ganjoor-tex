\begin{center}
\section*{غزل شماره ۱۸۵۹: منم آن حلقه در گوش و نشسته گوش شمس الدین}
\label{sec:1859}
\addcontentsline{toc}{section}{\nameref{sec:1859}}
\begin{longtable}{l p{0.5cm} r}
منم آن حلقه در گوش و نشسته گوش شمس الدین
&&
دلم پرنیش هجران است بهر نوش شمس الدین
\\
چو آتش‌های عشق او ز عرش و فرش بگذشته‌ست
&&
در این آتش ندانم کرد من روپوش شمس الدین
\\
در آغوشم ببینی تو ز آتش تنگ‌ها لیکن
&&
شود آن آب حیوان از پی آغوش شمس الدین
\\
چو دیکی پخت عقل من چشیدم بود ناپخته
&&
زدم آن دیک در رویش ز بهر جوش شمس الدین
\\
در این خانه تنم بینی یکی را دست بر سر زن
&&
یکی رنجور در نزع و یکی مدهوش شمس الدین
\\
زبان ذوالفقار عقل کاین دریا پر از در کرد
&&
زبانش بازبگرفت و شد او خاموش شمس الدین
\\
\end{longtable}
\end{center}
