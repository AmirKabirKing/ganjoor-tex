\begin{center}
\section*{غزل شماره ۱۸۸۷: گرت هست سر ما سر و ریش بجنبان}
\label{sec:1887}
\addcontentsline{toc}{section}{\nameref{sec:1887}}
\begin{longtable}{l p{0.5cm} r}
گرت هست سر ما سر و ریش بجنبان
&&
وگر عاشق شاهی روان باش به میدان
\\
صلا روز وصال است همه جاه و جمال است
&&
همه لطف و کمال است زهی نادره سلطان
\\
کجایی تو کجایی نه از حلقه مایی
&&
وگر خود به بهشتی چه خوش باشد بی‌جان
\\
یکی چرب زبانی یکی جان و جهانی
&&
از او بوسه به جانی زهی کاله ارزان
\\
اگر شیر اگر پیل چنانش کند این عشق
&&
چو بینیش بگوییش زهی گربه در انبان
\\
چه تلخ است و چه شیرین پر از مهر و پر از کین
&&
زهی لذت نوشین زهی لقمه دندان
\\
بیا پیش و مپرهیز و زین فتنه بمگریز
&&
بمستیز بمستیز هلا ای شه مردان
\\
زهی روز زهی روز زهی عید دل افروز
&&
از آن چشم کرشمه وزان لب شکرافشان
\\
بجو باده گلگون از آن دلبر موزون
&&
که این دم مه گردون روان گشت به میزان
\\
بنوش از می بالا لب و ریش میالا
&&
شنو بانگ و علالا ز هر اختر و کیوان
\\
بیندیش و خمش باش چنین راز مگو فاش
&&
دریغ است بر اوباش چنین گوهر و مرجان
\\
\end{longtable}
\end{center}
