\begin{center}
\section*{غزل شماره ۲۶۰: چرخ فلک با همه کار و کیا}
\label{sec:0260}
\addcontentsline{toc}{section}{\nameref{sec:0260}}
\begin{longtable}{l p{0.5cm} r}
چرخ فلک با همه کار و کیا
&&
گرد خدا گردد چون آسیا
\\
گرد چنین کعبه کن ای جان طواف
&&
گرد چنین مایده گرد ای گدا
\\
بر مثل گوی به میدانش گرد
&&
چونک شدی سرخوش بی‌دست و پا
\\
اسب و رخت راست بر این شه طواف
&&
گر چه بر این نطع روی جا به جا
\\
خاتم شاهیت در انگشت کرد
&&
تا که شوی حاکم و فرمانروا
\\
هر که به گرد دل آرد طواف
&&
جان جهانی شود و دلربا
\\
همره پروانه شود دلشده
&&
گردد بر گرد سر شمع‌ها
\\
زانک تنش خاکی و دل آتشی‌ست
&&
میل سوی جنس بود جنس را
\\
گرد فلک گردد هر اختری
&&
زانک بود جنس صفا با صفا
\\
گرد فنا گردد جان فقیر
&&
بر مثل آهن و آهن ربا
\\
زانک وجودست فنا پیش او
&&
شسته نظر از حول و از خطا
\\
مست همی‌کرد وضو از کمیز
&&
کز حدثم بازرهان ربنا
\\
گفت نخستین تو حدث را بدان
&&
کژمژ و مقلوب نباید دعا
\\
زانک کلیدست و چو کژ شد کلید
&&
وا شدن قفل نیابی عطا
\\
خامش کردم همگان برجهید
&&
قامت چون سرو بتم زد صلا
\\
خسرو تبریز شهم شمس دین
&&
بستم لب را تو بیا برگشا
\\
\end{longtable}
\end{center}
