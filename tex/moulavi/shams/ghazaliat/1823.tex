\begin{center}
\section*{غزل شماره ۱۸۲۳: سیر نمی‌شوم ز تو نیست جز این گناه من}
\label{sec:1823}
\addcontentsline{toc}{section}{\nameref{sec:1823}}
\begin{longtable}{l p{0.5cm} r}
سیر نمی‌شوم ز تو نیست جز این گناه من
&&
سیر مشو ز رحمتم ای دو جهان پناه من
\\
سیر و ملول شد ز من خنب و سقا و مشک او
&&
تشنه‌تر است هر زمان ماهی آب خواه من
\\
درشکنید کوزه را پاره کنید مشک را
&&
جانب بحر می روم پاک کنید راه من
\\
چند شود زمین وحل از قطرات اشک من
&&
چند شود فلک سیه از غم و دود آه من
\\
چند بزارد این دلم وای دلم خراب دل
&&
چند بنالد این لبم پیش خیال شاه من
\\
جانب بحر رو کز او موج صفا همی‌رسد
&&
غرقه نگر ز موج او خانه و خانقاه من
\\
آب حیات موج زد دوش ز صحن خانه‌ام
&&
یوسف من فتاد دی همچو قمر به چاه من
\\
سیل رسید ناگهان جمله ببرد خرمنم
&&
دود برآمد از دلم دانه بسوخت و کاه من
\\
خرمن من اگر بشد غم نخورم چه غم خورم
&&
صد چو مرا بس است و بس خرمن نور ماه من
\\
در دل من درآمد او بود خیالش آتشین
&&
آتش رفت بر سرم سوخته شد کلاه من
\\
گفت که از سماع‌ها حرمت و جاه کم شود
&&
جاه تو را که عشق او بخت من است و جاه من
\\
عقل نخواهم و خرد دانش او مرا بس است
&&
نور رخش به نیم شب غره صبحگاه من
\\
لشکر غم حشر کند غم نخورم ز لشکرش
&&
زانک گرفت طلب طلب تا به فلک سپاه من
\\
از پی هر غزل دلم توبه کند ز گفت و گو
&&
راه زند دل مرا داعیه اله من
\\
\end{longtable}
\end{center}
