\begin{center}
\section*{غزل شماره ۱۴۱: جمله یاران تو سنگند و توی مرجان چرا}
\label{sec:0141}
\addcontentsline{toc}{section}{\nameref{sec:0141}}
\begin{longtable}{l p{0.5cm} r}
جمله یاران تو سنگند و توی مرجان چرا
&&
آسمان با جملگان جسمست و با تو جان چرا
\\
چون تو آیی جزو جزوم جمله دستک می‌زنند
&&
چون تو رفتی جمله افتادند در افغان چرا
\\
با خیالت جزو جزوم می‌شود خندان لبی
&&
می‌شود با دشمن تو مو به مو دندان چرا
\\
بی خط و بی‌خال تو این عقل امی می‌بود
&&
چون ببیند آن خطت را می‌شود خط خوان چرا
\\
تن همی‌گوید به جان پرهیز کن از عشق او
&&
جانش می‌گوید حذر از چشمه حیوان چرا
\\
روی تو پیغامبر خوبی و حسن ایزدست
&&
جان به تو ایمان نیارد با چنین برهان چرا
\\
کو یکی برهان که آن از روی تو روشنترست
&&
کف نبرد کفرها زین یوسف کنعان چرا
\\
هر کجا تخمی بکاری آن بروید عاقبت
&&
برنروید هیچ از شه دانه احسان چرا
\\
هر کجا ویران بود آن جا امید گنج هست
&&
گنج حق را می‌نجویی در دل ویران چرا
\\
بی ترازو هیچ بازاری ندیدم در جهان
&&
جمله موزونند عالم نبودش میزان چرا
\\
گیرم این خربندگان خود بار سرگین می‌کشند
&&
این سواران باز می‌مانند از میدان چرا
\\
هر ترانه اولی دارد دلا و آخری
&&
بس کن آخر این ترانه نیستش پایان چرا
\\
\end{longtable}
\end{center}
