\begin{center}
\section*{غزل شماره ۲۸۴۷: دل بی‌قرار را گو که چو مستقر نداری}
\label{sec:2847}
\addcontentsline{toc}{section}{\nameref{sec:2847}}
\begin{longtable}{l p{0.5cm} r}
دل بی‌قرار را گو که چو مستقر نداری
&&
سوی مستقر اصلی ز چه رو سفر نداری
\\
به دم خوش سحرگه همه خلق زنده گردد
&&
تو چگونه دلستانی که دم سحر نداری
\\
تو چگونه گلستانی که گلی ز تو نروید
&&
تو چگونه باغ و راغی که یکی شجر نداری
\\
تو دلا چنان شدستی ز خرابی و ز مستی
&&
سخن پدر نگویی هوس پسر نداری
\\
به مثال آفتابی نروی مگر که تنها
&&
به مثال ماه شب رو حشم و حشر نداری
\\
تو در این سرا چو مرغی چو هوات آرزو شد
&&
بپری ز راه روزن هله گیر در نداری
\\
و اگر گرفته جانی که نه روزن است و نی در
&&
چو عرق ز تن برون رو که جز این گذر نداری
\\
تو چو جعد موی داری چه غم ار کله بیفتد
&&
تو چو کوه پای داری چه غم ار کمر نداری
\\
چو فرشتگان گردون به تو تشنه‌اند و عاشق
&&
رسدت ز نازنینی که سر بشر نداری
\\
نظرت ز چیست روشن اگر آن نظر ندیدی
&&
رخ تو ز چیست تابان اگر آن گهر نداری
\\
تو بگو مر آن ترش را ترشی ببر از این جا
&&
ور از آن شراب خوردی ز چه رو بطر نداری
\\
وگر از درونه مستی و به قاصدی ترش رو
&&
بدر اندر آب و آتش که دگر خطر نداری
\\
بدهد خدا به دریا خبری که رام او شو
&&
بنهد خبر در آتش که در او اثر نداری
\\
\end{longtable}
\end{center}
