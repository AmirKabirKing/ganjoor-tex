\begin{center}
\section*{غزل شماره ۱۹۳۲: ای دشمن عقل و جان شیرین}
\label{sec:1932}
\addcontentsline{toc}{section}{\nameref{sec:1932}}
\begin{longtable}{l p{0.5cm} r}
ای دشمن عقل و جان شیرین
&&
نور موسی و طور سینین
\\
ای دوست که زهره نیست جان را
&&
تا از تو نشان دهد به تعیین
\\
ای هر چه بگویم و نویسم
&&
برخوانده نانبشته پیشین
\\
ای آنک طبیب دردهایی
&&
بی قرص بنفشه و فسنتین
\\
ای باعث رزق مستمندان
&&
بی قوصره و جوال و خرجین
\\
هر ذوق که غیر حضرت توست
&&
نوش تین است و نیش تنین
\\
دو پاره کلوخ را بگیری
&&
ویسی سازی از آن و رامین
\\
وان نقش از آن فروتراشی
&&
طینی باشد میانه طین
\\
پس در کف صنع نقش بندت
&&
لعبت‌هااند این سلاطین
\\
بر هم زنشان چو دو سبو تو
&&
تا بشکند آن یکی به توهین
\\
تا لاف زند که من شکستم
&&
تو بشکسته به دست تکوین
\\
چون بادی را کنی مصور
&&
طاووس شوند و باز و شاهین
\\
شب خواب مسافری ببندی
&&
یعنی که مخسب خیز بنشین
\\
بنشین به خیال خانه دل
&&
هر نقش که می کنیم می بین
\\
نقشی دگری همی‌فرستیم
&&
تا لقمه او شود نخستین
\\
تا صورت راست را بدانی
&&
در سینه ز صورت دروغین
\\
من از پی اینت نقش کردم
&&
تا کلک مرا کنی تو تحسین
\\
امشب همه نقش‌ها شکارند
&&
از اسب فرومگیر تو زین
\\
تا روز سوار باش بر صید
&&
مندیش ز بالش و نهالین
\\
می گرد به گرد لیل لیلی
&&
گر مجنونی ز پای منشین
\\
امشب صدقات می دهد شاه
&&
ان الصدقات للمساکین
\\
صاع سلطان اگر بجویی
&&
یابی به جوال ابن یامین
\\
بس کن که دعا بسی بکردی
&&
گوش آر از این سپس به آمین
\\
\end{longtable}
\end{center}
