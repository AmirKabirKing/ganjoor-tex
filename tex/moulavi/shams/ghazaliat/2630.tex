\begin{center}
\section*{غزل شماره ۲۶۳۰: یا ساقی شرف بشراباتک زندی}
\label{sec:2630}
\addcontentsline{toc}{section}{\nameref{sec:2630}}
\begin{longtable}{l p{0.5cm} r}
یا ساقی شرف بشراباتک زندی
&&
فالراح مع الروح من افضالک عندی
\\
برخیز که شورید خرابات افندی
&&
مستان نگر و نقل و شرابات افندی
\\
هر مست درآویخته با مست ز مستی
&&
گردان شده ساقی به مساقات افندی
\\
یک موی نمی‌گنجد در حلقه مستان
&&
جز رقص و هیاهوی و مراعات افندی
\\
بسم الله ساقی ولی نعمت برخیز
&&
تا جان بدهیمت به مکافات افندی
\\
در هر دو جهان است و نبوده‌ست و نباشد
&&
جز دیدن روی تو کرامات افندی
\\
چون تنگ شکر میر خرابات درآمد
&&
یا رب چه لطیف است ملاقات افندی
\\
می‌خندد و می‌گوید من خفته بدم مست
&&
هیهای شنیدم من و هیهات افندی
\\
زان خنده و زان گفتن و زان شیوه شیرین
&&
صد غلغله در سقف سماوات افندی
\\
خورشید ز برق رخ تو چشم ببندد
&&
کافزون ز زجاجه‌ست و ز مشکات افندی
\\
در خانه خمار و خرابات کی دیده‌ست
&&
معراج و تجلی و مقامات افندی
\\
با مست خرابات خدا تا بنپیچی
&&
تا وا ننماید همه رگ‌هات افندی
\\
در خانه دل کژ مکن آن چانه به افسوس
&&
کامروز عیان است خفیات افندی
\\
روزی که روم جانب دریای معانی
&&
یاد آیدت این جمله مقالات افندی
\\
شاد آمدی ای کان شکر عیب مفرما
&&
گر بوسه دهد بنده بر آن پات افندی
\\
واجب کند ای دوست که آرم به صد اخلاص
&&
در سایه زلف تو مناجات افندی
\\
از مصحف آن روی چو ماه تو بخوانیم
&&
سوره قصص و نادره آیات افندی
\\
مستیم ز جام تو و زان نرگس مخمور
&&
رستیم به شاهیت ز شهمات افندی
\\
عالم همه پرغصه و آن نرگس مخمور
&&
فارغ ز بدایات و نهایات افندی
\\
چون سایه فناییم به خورشید جمالت
&&
ایمن شده از جمله آفات افندی
\\
سرمست بیا جانب بازار نظر کن
&&
تا راست شود جمله مهمات افندی
\\
تا روز اجل هر چه بگوییم ز اشعار
&&
این است و دگر جمله خرافات افندی
\\
سلطان غزل‌هاست و همه بنده اینند
&&
هر بیتش مفتاح مرادات افندی
\\
من کردم خاموش تو باقیش بفرما
&&
ای جان اشارات و عبارات افندی
\\
شمس الحق تبریز تویی موسی ایام
&&
بر طور دلم رفته به میقات افندی
\\
\end{longtable}
\end{center}
