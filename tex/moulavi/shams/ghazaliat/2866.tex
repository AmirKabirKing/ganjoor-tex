\begin{center}
\section*{غزل شماره ۲۸۶۶: گر گریزی به ملولی ز من سودایی}
\label{sec:2866}
\addcontentsline{toc}{section}{\nameref{sec:2866}}
\begin{longtable}{l p{0.5cm} r}
گر گریزی به ملولی ز من سودایی
&&
روکشان دست گزان جانب جان بازآیی
\\
زین خیالی که کشان کرد تو را دست بکش
&&
دست از او گر نکشی دست پشیمان خایی
\\
رو بدو آر و بگو خواجه کجا می‌کشیم
&&
کآسمان ماه ندیده‌ست بدین زیبایی
\\
رایگان روی نموده‌ست غلط افتادی
&&
باش تا در طلب و پویه جهان پیمایی
\\
گنده پیر است جهان چادر نو پوشیده
&&
از برون شیوه و غنج و ز درون رسوایی
\\
چو بدان پیر روی بخت جوانت گوید
&&
سرخر معده سگ رو که همان را شایی
\\
لا یغرنک سد هوس عن رایی
&&
کم قصور هدمت من عوج الا رآ
\\
اشتهی انصح لکن لسانی قفلت
&&
اننی انصح بالصمت علی الاخفا
\\
این همه ترس و نفاق و دودلی باری چیست
&&
نه که در سایه و در دولت این مولایی
\\
بیم از آن می‌کندت تا برود بیم از تو
&&
یار از آن می‌گزدت تا همه شکر خایی
\\
شمس تبریز نه شمعی است که غایب گردد
&&
شب چو شد روز چرا منتظر فردایی
\\
\end{longtable}
\end{center}
