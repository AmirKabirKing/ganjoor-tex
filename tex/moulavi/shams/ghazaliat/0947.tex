\begin{center}
\section*{غزل شماره ۹۴۷: مخسب شب که شبی صد هزار جان ارزد}
\label{sec:0947}
\addcontentsline{toc}{section}{\nameref{sec:0947}}
\begin{longtable}{l p{0.5cm} r}
مخسب شب که شبی صد هزار جان ارزد
&&
که شب ببخشد آن بدر بدره بی‌حد
\\
به آسمان جهان هر شبی فرود آید
&&
برای هر متظلم سپاه فضل احد
\\
خدای گفت قم اللیل و از گزاف نگفت
&&
ز شب رویست فرو قد زهره و فرقد
\\
ز دود شب پزی ای خام ز آتش موسی
&&
مداد شب دهد آن خامه را ز علم مدد
\\
بگیر لیلی شب را کنار ای مجنون
&&
شبست خلوت توحید و روز شرک و عدد
\\
شبست لیلی و روزست در پیش مجنون
&&
که نور عقل سحر را به جعد خویش کشد
\\
بدانک آب حیات اندرون تاریکیست
&&
چه ماهیی که ره آب بسته‌ای بر خود
\\
به دیبه سیه این کعبه را لباسی ساخت
&&
که اوست پشت مطیعان و اوستشان مسند
\\
درون کعبه شب یک نماز صد باشد
&&
ز بهر خواب ندارد کسی چنین معبد
\\
شکست جمله بتان را شب و بماند خدا
&&
که نیست در کرم او را قرین و کفو احد
\\
خمش که شعر کسادست و جهل از آن اکسد
&&
چه زاهدی تو در این علم و در تو علم ازهد
\\
\end{longtable}
\end{center}
