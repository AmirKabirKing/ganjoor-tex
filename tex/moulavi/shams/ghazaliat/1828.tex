\begin{center}
\section*{غزل شماره ۱۸۲۸: باز نگار می کشد چون شتران مهار من}
\label{sec:1828}
\addcontentsline{toc}{section}{\nameref{sec:1828}}
\begin{longtable}{l p{0.5cm} r}
باز نگار می کشد چون شتران مهار من
&&
یارکشی است کار او بارکشی است کار من
\\
پیش رو قطارها کرد مرا و می کشد
&&
آن شتران مست را جمله در این قطار من
\\
اشتر مست او منم خارپرست او منم
&&
گاه کشد مهار من گاه شود سوار من
\\
اشتر مست کف کند هر چه بود تلف کند
&&
لیک نداند اشتری لذت نوشخوار من
\\
راست چو کف برآورم بر کف او کف افکنم
&&
کف چو به کف او رسد جوش کند بخار من
\\
کار کنم چو کهتران بار کشم چو اشتران
&&
بار کی می کشم ببین عزت کار و بار من
\\
نرگس او ز خون من چون شکند خمار خود
&&
صبر و قرار او برد صبر من و قرار من
\\
گشته خیال روی او قبله نور چشم من
&&
وان سخنان چون زرش حلقه گوشوار من
\\
باغ و بهار را بگو لاف خوشی چه می زنی
&&
من بنمایمت خوشی چون برسد بهار من
\\
می چو خوری بگو به می بر سر من چه می زنی
&&
در سر خود ندیده‌ای باده بی‌خمار من
\\
باز سپیدی و برو میر شکار را بگو
&&
هر دو مرا تویی بلی میر من و شکار من
\\
مطلع این غزل شتر بود از آن دراز شد
&&
ز اشتر کوتهی مجو ای شه هوشیار من
\\
\end{longtable}
\end{center}
