\begin{center}
\section*{غزل شماره ۲۰۶۲: سیر نشد چشم و دل از نظر شاه من}
\label{sec:2062}
\addcontentsline{toc}{section}{\nameref{sec:2062}}
\begin{longtable}{l p{0.5cm} r}
سیر نشد چشم و دل از نظر شاه من
&&
سیر مشو هم تو نیز زین دل آگاه من
\\
مشک و سقا سیر شد از جگر گرم من
&&
هیچ به جز آب نیست لذت و دلخواه من
\\
درشکنم کوزه را پاره کنم مشک را
&&
روی به دریا نهم نیست جز این راه من
\\
چند شود تر زمین از مدد اشک من
&&
چند بسوزد فلک از تبش و آه من
\\
چند بگوید دلم وای دلم وای دل
&&
چند بگوید لبم راز شهنشاه من
\\
رو سوی بحری کز او هر نفسی موج موج
&&
آمد و اندرربود خیمه و خرگاه من
\\
آب خوشی جوش کرد نیم شب از خانه‌ام
&&
یوسف حسن اوفتاد ناگه در چاه من
\\
ز آب رخ یوسفی خرمن من سیل برد
&&
دود برآمد ز دل سوخته شد کاه من
\\
خرمن من گر بسوخت باک ندارم خوشم
&&
صد چو مرا بس بود خرمن آن ماه من
\\
عقل نخواهم بس است دانش و علمش مرا
&&
شمع رخ او بس است در شب بی‌گاه من
\\
گفت کسی کاین سماع جاه و ادب کم کند
&&
جاه نخواهم که عشق در دو جهان جاه من
\\
در پی هر بیت من گویم پایان رسید
&&
چون ز سرم می‌برد آن شه آگاه من
\\
\end{longtable}
\end{center}
