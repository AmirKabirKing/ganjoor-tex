\begin{center}
\section*{غزل شماره ۲۶۳: به شکرخنده اگر می‌ببرد جان مرا}
\label{sec:0263}
\addcontentsline{toc}{section}{\nameref{sec:0263}}
\begin{longtable}{l p{0.5cm} r}
به شکرخنده اگر می‌ببرد جان مرا
&&
متع الله فوادی بحبیبی ابدا
\\
جانم آن لحظه بخندد که ویش قبض کند
&&
انما یوم اجزای اذا اسکرها
\\
مغز هر ذره چو از روزن او مست شود
&&
سبحت راقصه عز حبیبی و علا
\\
چونک از خوردن باده همگی باده شوم
&&
انا نقل و مدام فاشربانی و کلا
\\
هله ای روز چه روزی تو که عمر تو دراز
&&
یوم وصل و رحیق و نعیم و رضا
\\
تن همچون خم ما را پی آن باده سرشت
&&
نعم ما قدر ربی لفوادی و قضا
\\
خم سرکه دگرست و خم دوشاب دگر
&&
کان فی خابیه الروح نبیذ فغلی
\\
چون بخسپد خم باده پی آن می‌جوشد
&&
انما القهوه تغلی لشرور و دما
\\
می منم خود که نمی‌گنجم در خم جهان
&&
برنتابد خم نه چرخ کف و جوش مرا
\\
می مرده چه خوری هین تو مرا خور که میم
&&
انا زق ملئت فیه شراب و سقا
\\
وگرت رزق نباشد من و یاران بخوریم
&&
فانصتوا و اعترفوا معشرا اخوان صفا
\\
\end{longtable}
\end{center}
