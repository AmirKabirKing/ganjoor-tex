\begin{center}
\section*{غزل شماره ۸۵۳: مر بحر را ز ماهی دایم گزیر باشد}
\label{sec:0853}
\addcontentsline{toc}{section}{\nameref{sec:0853}}
\begin{longtable}{l p{0.5cm} r}
مر بحر را ز ماهی دایم گزیر باشد
&&
زیرا به پیش دریا ماهی حقیر باشد
\\
مانند بحر قلزم ماهی نیابی ای جان
&&
در بحر قلزم حق ماهی کثیر باشد
\\
بحرست همچو دایه ماهی چو شیرخواره
&&
پیوسته طفل مسکین گریان شیر باشد
\\
با این همه فراغت گر بحر را به ماهی
&&
میلی بود به رحمت فضل کبیر باشد
\\
وان ماهیی که داند کان بحر طالب اوست
&&
پایش ز روی نخوت فوق اثیر باشد
\\
آن ماهیی که دریا کار کسی نسازد
&&
الا که رای ماهی آن را مشیر باشد
\\
گویی ز بس عنایت آن ماهیست سلطان
&&
وان بحر بی‌نهایت او را وزیر باشد
\\
گر هیچ کس ز جرات ماهیش خواند او را
&&
هر قطره‌ای به قهرش مانند تیر باشد
\\
تا چند رمز گویی رمزت تحیر آرد
&&
روشنترک بیان کن تا دل بصیر باشد
\\
مخدوم شمس دینست هم سید و خداوند
&&
کز وی زمین تبریز مشک و عبیر باشد
\\
گر خارهای عالم الطاف او ببینند
&&
در نرمی و لطافت همچون حریر باشد
\\
جانم مباد هرگز گر جانم از شرابش
&&
وز مستی جمالش از خود خبیر باشد
\\
\end{longtable}
\end{center}
