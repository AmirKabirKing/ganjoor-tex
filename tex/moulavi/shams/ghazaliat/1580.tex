\begin{center}
\section*{غزل شماره ۱۵۸۰: تا دلبر خویش را نبینیم}
\label{sec:1580}
\addcontentsline{toc}{section}{\nameref{sec:1580}}
\begin{longtable}{l p{0.5cm} r}
تا دلبر خویش را نبینیم
&&
جز در تک خون دل نشینیم
\\
ما به نشویم از نصیحت
&&
چون گمره عشق آن بهینیم
\\
اندر دل درد خانه داریم
&&
درمان نبود چو همچنینیم
\\
در حلقه عاشقان قدسی
&&
سرحلقه چو گوهر نگینیم
\\
حاشا که ز عقل و روح لافیم
&&
آتش در ما اگر همینیم
\\
گر از عقبات روح جستی
&&
مستانه مرو که در کمینیم
\\
چون فتنه نشان آسمانیم
&&
چون است که فتنه زمینیم
\\
چون ساده‌تر از روان پاکیم
&&
پرنقش چرا مثال چینیم
\\
پژمرده شود هزار دولت
&&
ما تازه و تر چو یاسمینیم
\\
گر متهمیم پیش هستی
&&
اندر تتق فنا امینیم
\\
ما پشت بدین وجود داریم
&&
کاندر شکم فنا جنینیم
\\
تبریز ببین چه تاجداریم
&&
زان سر که غلام شمس دینیم
\\
\end{longtable}
\end{center}
