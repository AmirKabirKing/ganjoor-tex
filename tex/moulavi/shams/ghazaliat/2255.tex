\begin{center}
\section*{غزل شماره ۲۲۵۵: بنشسته به گوشه‌ای دو سه مست ترانه گو}
\label{sec:2255}
\addcontentsline{toc}{section}{\nameref{sec:2255}}
\begin{longtable}{l p{0.5cm} r}
بنشسته به گوشه‌ای دو سه مست ترانه گو
&&
ز دل و جان لطیفتر شده مهمان عنده
\\
ز طرب چون حشر شود سرشان مستتر شود
&&
فتد از جنگ و عربده سر مستان میان کو
\\
ز اشارات روحشان ز صباح و صبوحشان
&&
عسل و می روان شود به چپ و راست جوی جو
\\
نفسیشان معانقه نفسیشان معاشقه
&&
نفسی سجده طرب نفسی جنگ و گفت و گو
\\
نفسی یار قندلب شکرین شکرنسب
&&
به چنین حال بوالعجب تو از ایشان ادب مجو
\\
به خدا خوب ساقیی که وفادار و باقیی
&&
به حلیمی گناه جو به طبیعت نشاط خو
\\
قدحی دو ز دست خود بده ای جان به مست خود
&&
هله تا راز آسمان شنوی جمله مو به مو
\\
تو بر او ریز جام می که حجاب وی است وی
&&
هله تا از سعادتت برهد اوی او ز او
\\
چو خرد غرق باده شد در دولت گشاده شد
&&
سر هر کیسه کرم بگشاید که انفقوا
\\
بهل آن پوست مغز بین صنم خوب نغز بین
&&
هله بردار ابر را ز رخ ماه تو به تو
\\
پس از این جمله آب‌ها نرود جز بجوی ما
&&
من سرمست می‌کشم ز فراتش سبو سبو
\\
من و دلدار نازنین خوش و سرمست همچنین
&&
به گلستان جان روان ز گلستان رنگ و بو
\\
نظری کن به چشم او به جمال و کرشم او
&&
نظری کن به خال او به حق صحبت ای عمو
\\
تو اگر در فرح نه‌ای که حریف قدح نه‌ای
&&
چه برد طفل از لبش چو بود مست لبلبو
\\
چو شدی محرم فلک سبک ای یار بانمک
&&
بنگر ذره ذره را زده زیر بغل کدو
\\
چو تف آفتاب زد ره ذرات بی‌عدد
&&
بشکافید پرده شان نپذیرد دگر رفو
\\
به لبانت ز دست شد سر او باز مست شد
&&
زند او باز این زمان چو کبوتر بقوبقو
\\
تو بخسپی و عشق و دل گذران بی ز غش و غل
&&
ز ره خواب بر فلک خوش و سرمست دو به دو
\\
بخورند از نخیل جان که ندیده‌ست انس و جان
&&
رطب و تمر نادری که نگنجد در این گلو
\\
که ابیت بمهجتی شرفا عند سیدی
&&
ز طعام و شراب حق بخورم اندر آن غلو
\\
هله امشب به خانه رو که دل مست شد گرو
&&
چو شود روز خوش بیا شنو این را تمام تو
\\
تو بگو باقی غزل که کند در همه عمل
&&
که تویی عشق و عشق را نبود هیچ کس عدو
\\
تو بگو کآب کوثری خوش و نوش و معطری
&&
همه را سبز کن طری و ز پژمردگی بشو
\\
\end{longtable}
\end{center}
