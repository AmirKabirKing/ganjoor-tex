\begin{center}
\section*{غزل شماره ۹۷۵: آتش افکند در جهان جمشید}
\label{sec:0975}
\addcontentsline{toc}{section}{\nameref{sec:0975}}
\begin{longtable}{l p{0.5cm} r}
آتش افکند در جهان جمشید
&&
از پس چار پرده چون خورشید
\\
خنک او را که شد برهنه ز بود
&&
وای آن را که جست سایه بید
\\
دل سپیدست و عشق را رو سرخ
&&
زان سپیدی که نیست سرخ و سپید
\\
عشق ایمن ولایتیست چنانک
&&
ترس را نیست اندر او امید
\\
هر حیاتی که یک دمش عمرست
&&
چون برآید ز عشق شد جاوید
\\
یک عروسیست بر فلک که مپرس
&&
ور بپرسی بپرس از ناهید
\\
زین عروسی خبر نداشت کسی
&&
آمدند انبیا به رسم نوید
\\
شمس تبریز خسرو عهدست
&&
خسروان را هله به جان بخرید
\\
\end{longtable}
\end{center}
