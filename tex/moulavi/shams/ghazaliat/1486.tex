\begin{center}
\section*{غزل شماره ۱۴۸۶: چون آینه رازنما باشد جانم}
\label{sec:1486}
\addcontentsline{toc}{section}{\nameref{sec:1486}}
\begin{longtable}{l p{0.5cm} r}
چون آینه رازنما باشد جانم
&&
تانم که نگویم نتوانم که ندانم
\\
از جسم گریزان شدم از روح بپرهیز
&&
سوگند ندانم نه از اینم نه از آنم
\\
ای طالب بو بردن شرط است به مردن
&&
زنده منگر در من زیرا نه چنانم
\\
اندر کژیم منگر وین راست سخن بین
&&
تیر است حدیث من و من همچو کمانم
\\
این سر چو کدو بر سر وین دلق تن من
&&
بازار جهان در به کی مانم به کی مانم
\\
وان گاه کدو بر سر من پر ز شرابی
&&
دارمش نگوسار از او من نچکانم
\\
ور زان که چکانم تو ببین قدرت حق را
&&
کز بحر بدان قطره جواهر بستانم
\\
چون ابر دو چشمم بستد جوهر آن بحر
&&
بر چرخ وفا آید این ابر روانم
\\
در حضرت شمس الحق تبریز ببارم
&&
تا سوسن‌ها روید بر شکل زبانم
\\
\end{longtable}
\end{center}
