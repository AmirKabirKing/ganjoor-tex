\begin{center}
\section*{غزل شماره ۶۶۰: نثرنا فی ربیع الوصل بالورد}
\label{sec:0660}
\addcontentsline{toc}{section}{\nameref{sec:0660}}
\begin{longtable}{l p{0.5cm} r}
نثرنا فی ربیع الوصل بالورد
&&
حنانینا فنعم الزوج و الفرد
\\
ز رویت باغ و عبهر می‌توان کرد
&&
ز زلفت مشک و عنبر می‌توان کرد
\\
ز روی زرد همچون زعفرانم
&&
جهانی را مزعفر می‌توان کرد
\\
به یک دانه ز خرمنگاه ماهت
&&
فلک‌ها را مسخر می‌توان کرد
\\
تو آن خضری که از آب حیاتت
&&
گدایان را سکندر می‌توان کرد
\\
در آن حالی که حالم بازجویی
&&
محالی را میسر می‌توان کرد
\\
نخاف العین ترمینا بسو
&&
فیا داود قدر حلقه السرد
\\
به خود واگرد ای دل زانک از دل
&&
ره پنهان به دلبر می‌توان کرد
\\
جهان شش جهت را گر دری نیست
&&
چو در دل آمدی در می‌توان کرد
\\
درآ در دل که منظرگاه حقست
&&
وگر هم نیست منظر می‌توان کرد
\\
چو دردی ماند جان ما در این زیر
&&
اگر زیرست از بر می‌توان کرد
\\
ز گولی در جوال نفس رفتی
&&
وگر نی ترک این خر می‌توان کرد
\\
الا یا ساقیا هات الحمیا
&&
لتکفینا عناء الحر و البرد
\\
دل سنگین عشق ار نرم گردد
&&
دل ار سنگست جوهر می‌توان کرد
\\
بیار آن باده حمرا و درده
&&
کز احمر عالم اخضر می‌توان کرد
\\
از آن باده که پر و بال عیش است
&&
ز هر جزوم کبوتر می‌توان کرد
\\
از آن جرعه که از دریای فضل است
&&
بهشت و حور و کوثر می‌توان کرد
\\
چو تیرانداز گردد باده در خم
&&
ز تیر باده اسپر می‌توان کرد
\\
و اسکرنا به کاسات عظام
&&
فان السکر دفع الهم و الحرد
\\
چو باده در من آتش زد بدیدم
&&
که از هر آب آذر می‌توان کرد
\\
بیا ای مادر عشرت به خانه
&&
که جان را فرش مادر می‌توان کرد
\\
وگر در راه تو نامحرمانند
&&
تو را از جام چادر می‌توان کرد
\\
چو گشتی شیرگیر و شیرآشام
&&
سزای شیر صفدر می‌توان کرد
\\
بزن گردن امل‌ها را به باده
&&
کز آن هر قطره خنجر می‌توان کرد
\\
سقاهم ربهم برخوان و می نوش
&&
که هر دم عیش دیگر می‌توان کرد
\\
وگر ساغر نداری می بیاور
&&
دهان را همچو ساغر می‌توان کرد
\\
و اعتقنا به خمر من هموم
&&
و جازی همنا بالدفع و الطرد
\\
\end{longtable}
\end{center}
