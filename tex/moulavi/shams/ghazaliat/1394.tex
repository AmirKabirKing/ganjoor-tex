\begin{center}
\section*{غزل شماره ۱۳۹۴: دفع مده دفع مده من نروم تا نخورم}
\label{sec:1394}
\addcontentsline{toc}{section}{\nameref{sec:1394}}
\begin{longtable}{l p{0.5cm} r}
دفع مده دفع مده من نروم تا نخورم
&&
عشوه مده عشوه مده عشوه مستان نخرم
\\
وعده مکن وعده مکن مشتری وعده نیم
&&
یا بدهی یا ز دکان تو گروگان ببرم
\\
گر تو بهایی بنهی تا که مرا دفع کنی
&&
رو که به جز حق نبری گر چه چنین بی‌خبرم
\\
پرده مکن پرده مدر در سپس پرده مرو
&&
راه بده راه بده یا تو برون آ ز حرم
\\
ای دل و جان بنده تو بند شکرخنده تو
&&
خنده تو چیست بگو جوشش دریای کرم
\\
طالع استیز مرا از مه و مریخ بجو
&&
همچو قضاهای فلک خیره و استیزه گرم
\\
چرخ ز استیزه من خیره و سرگشته شود
&&
زانک دو چندان که ویم گر چه چنین مختصرم
\\
گر تو ز من صرفه بری من ز تو صد صرفه برم
&&
کیسه برم کاسه برم زانک دورو همچو زرم
\\
گر چه دورو همچو زرم مهر تو دارد نظرم
&&
از مه و از مهر فلک مه‌تر و افلاک ترم
\\
لاف زنم لاف که تو راست کنی لاف مرا
&&
ناز کنم ناز که من در نظرت معتبرم
\\
چه عجب ار خوش خبرم چونک تو کردی خبرم
&&
چه عجب ار خوش نظرم چونک تویی در نظرم
\\
بر همگان گر ز فلک زهر ببارد همه شب
&&
من شکر اندر شکر اندر شکر اندر شکرم
\\
هر کسکی را کسکی هر جگری را هوسی
&&
لیک کجا تا به کجا من ز هوایی دگرم
\\
من طلب اندر طلبم تو طرب اندر طربی
&&
آن طربت در طلبم پا زد و برگشت سرم
\\
تیر تراشنده تویی دوک تراشنده منم
&&
ماه درخشنده تویی من چو شب تیره برم
\\
میر شکار فلکی تیر بزن در دل من
&&
ور بزنی تیر جفا همچو زمین پی سپرم
\\
جمله سپرهای جهان باخلل از زخم بود
&&
بی‌خطر آن گاه بوم کز پی زخمت سپرم
\\
گیج شد از تو سر من این سر سرگشته من
&&
تا که ندانم پسرا که پسرم یا پدرم
\\
آن دل آواره من گر ز سفر بازرسد
&&
خانه تهی یابد او هیچ نبیند اثرم
\\
سرکه فشانی چه کنی کآتش ما را بکشی
&&
کآتشم از سرکه‌ات افزون شود افزون شررم
\\
عشق چو قربان کندم عید من آن روز بود
&&
ور نبود عید من آن مرد نیم بلک غرم
\\
چون عرفه و عید تویی غره ذی الحجه منم
&&
هیچ به تو درنرسم وز پی تو هم نبرم
\\
باز توام باز توام چون شنوم طبل تو را
&&
ای شه و شاهنشه من باز شود بال و پرم
\\
گر بدهی می بچشم ور ندهی نیز خوشم
&&
سر بنهم پا بکشم بی‌سر و پا می نگرم
\\
\end{longtable}
\end{center}
