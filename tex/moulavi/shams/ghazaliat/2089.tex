\begin{center}
\section*{غزل شماره ۲۰۸۹: تنت زین جهان است و دل زان جهان}
\label{sec:2089}
\addcontentsline{toc}{section}{\nameref{sec:2089}}
\begin{longtable}{l p{0.5cm} r}
تنت زین جهان است و دل زان جهان
&&
هوا یار این و خدا یار آن
\\
دل تو غریب و غم او غریب
&&
نیند از زمین و نه از آسمان
\\
اگر یار جانی و یار خرد
&&
رسیدی بیار و ببردی تو جان
\\
وگر یار جسمی و یار هوا
&&
تو با این دو ماندی در این خاکدان
\\
مگر ناگهان آن عنایت رسد
&&
که ای من غلام چنان ناگهان
\\
که یک جذب حق به ز صد کوشش است
&&
نشان‌ها چه باشد بر بی‌نشان
\\
نشان چون کف و بی‌نشان بحر دان
&&
نشان چون بیان بی‌نشان چون عیان
\\
ز خورشید یک جو چو ظاهر شود
&&
بروبد ز گردون ره کهکشان
\\
خمش کن خمش کن که در خامشی است
&&
هزاران زبان و هزاران بیان
\\
\end{longtable}
\end{center}
