\begin{center}
\section*{غزل شماره ۱۸۱۰: من دزد دیدم کو برد مال و متاع مردمان}
\label{sec:1810}
\addcontentsline{toc}{section}{\nameref{sec:1810}}
\begin{longtable}{l p{0.5cm} r}
من دزد دیدم کو برد مال و متاع مردمان
&&
این دزد ما خود دزد را چون می بدزدد از میان
\\
خواهند از سلطان امان چون دزد افزونی کند
&&
دزدی چو سلطان می کند پس از کجا خواهند امان
\\
عشق است آن سلطان که او از جمله دزدان دل برد
&&
تا پیش آن سرکش برد حق سرکشان را موکشان
\\
عشق است آن دزدی که او از شحنگان دل می برد
&&
در خدمت آن دزد بین تو شحنگان بی‌کران
\\
آواز دادم دوش من کای خفتگان دزد آمده‌ست
&&
دزدید او از چابکی در حین زبانم از دهان
\\
گفتم ببندم دست او خود بست او دستان من
&&
گفتم به زندانش کنم او می نگنجد در جهان
\\
از لذت دزدی او هر پاسبان دزدی شده
&&
از حیله و دستان او هر زیرکی گشته نهان
\\
خلقی ببینی نیم شب جمع آمده کان دزد کو
&&
او نیز می پرسد که کو آن دزد او خود در میان
\\
ای مایه هر گفت و گو ای دشمن و ای دوست رو
&&
ای هم حیات جاودان ای هم بلای ناگهان
\\
ای رفته اندر خون دل ای دل تو را کرده بحل
&&
بر من بزن زخم و مهل حقا نمی‌خواهم امان
\\
سخته کمانی خوش بکش بر من بزن آن تیر خوش
&&
ای من فدای تیر تو ای من غلام آن کمان
\\
زخم تو در رگ‌های من جان است و جان افزای من
&&
شمشیر تو بر نای من حیف است ای شاه جهان
\\
کو حلق اسماعیل تا از خنجرت شکری کند
&&
جرجیس کو کز زخم تو جانی سپارد هر زمان
\\
شه شمس تبریزی مگر چون بازآید از سفر
&&
یک چند بود اندر بشر شد همچو عنقا بی‌نشان
\\
\end{longtable}
\end{center}
