\begin{center}
\section*{غزل شماره ۲۹۴۲: اندر مصاف ما را در پیش رو سپر نی}
\label{sec:2942}
\addcontentsline{toc}{section}{\nameref{sec:2942}}
\begin{longtable}{l p{0.5cm} r}
اندر مصاف ما را در پیش رو سپر نی
&&
و اندر سماع ما را از نای و دف خبر نی
\\
ما خود فنای عشقش ما خاک پای عشقش
&&
عشقیم توی بر تو عشقیم کل دگر نی
\\
خود را چو درنوردیم ما جمله عشق گردیم
&&
سرمه چو سوده گردد جز مایه نظر نی
\\
هر جسم کو عرض شد جان و دل غرض شد
&&
بگداز کز مرض‌ها ز افسردگی بتر نی
\\
از حرص آن گدازش وز عشق آن نوازش
&&
باری جگر درونم خون شد مرا جگر نی
\\
صدپاره شد دل من و آواره شد دل من
&&
امروز اگر بجویی در من ز دل اثر نی
\\
در قرص مه نگه کن هر روز می‌گدازد
&&
تا در محاق گویی کاندر فلک قمر نی
\\
لاغرتری آن مه از قرب شمس باشد
&&
در بعد زفت باشد لیکن چنان هنر نی
\\
شاها ز بهر جان‌ها زهره فرست مطرب
&&
کفو سماع جان‌ها این نای و دف تر نی
\\
نی نی که زهره چه بود چون شمس عاجز آمد
&&
درخورد این حراره در هیچ چنگ و خور نی
\\
\end{longtable}
\end{center}
