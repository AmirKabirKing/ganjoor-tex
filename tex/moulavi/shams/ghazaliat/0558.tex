\begin{center}
\section*{غزل شماره ۵۵۸: یار مرا چو اشتران باز مهار می‌کشد}
\label{sec:0558}
\addcontentsline{toc}{section}{\nameref{sec:0558}}
\begin{longtable}{l p{0.5cm} r}
یار مرا چو اشتران باز مهار می‌کشد
&&
اشتر مست خویش را در چه قطار می‌کشد
\\
جان و تنم بخست او شیشه من شکست او
&&
گردن من به بست او تا به چه کار می‌کشد
\\
شست ویم چو ماهیان جانب خشک می‌برد
&&
دام دلم به جانب میر شکار می‌کشد
\\
آنک قطار ابر را زیر فلک چو اشتران
&&
ساقی دشت می‌کند برکه و غار می‌کشد
\\
رعد همی‌زند دهل زنده شدست جزو و کل
&&
در دل شاخ و مغز گل بوی بهار می‌کشد
\\
آنک ضمیر دانه را علت میوه می‌کند
&&
راز دل درخت را بر سر دار می‌کشد
\\
لطف بهار بشکند رنج خمار باغ را
&&
گر چه جفای دی کنون سوی خمار می‌کشد
\\
\end{longtable}
\end{center}
