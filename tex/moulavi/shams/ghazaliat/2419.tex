\begin{center}
\section*{غزل شماره ۲۴۱۹: خوش بود فرش تن نور دیده}
\label{sec:2419}
\addcontentsline{toc}{section}{\nameref{sec:2419}}
\begin{longtable}{l p{0.5cm} r}
خوش بود فرش تن نور دیده
&&
خوش بود مرغ جان بپریده
\\
جان نادیده خسیس شده
&&
جان دیده رسیده در دیده
\\
جان زرین و جان سنگین را
&&
چون کلوخ از برنج بگزیده
\\
سر کاغذ گشاده دست اجل
&&
نقد در کاغذ است پیچیده
\\
خمره پرعسل سرش بسته
&&
پشت و پهلوش را تو لیسیده
\\
خمره را بر زمین زن و بشکن
&&
دیده نبود چنانک بشنیده
\\
شمس تبریز بشکند خم را
&&
که ز نامش فلک بلرزیده
\\
\end{longtable}
\end{center}
