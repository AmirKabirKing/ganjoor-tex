\begin{center}
\section*{غزل شماره ۷۵۶: علتی باشد که آن اندر بهاران بد شود}
\label{sec:0756}
\addcontentsline{toc}{section}{\nameref{sec:0756}}
\begin{longtable}{l p{0.5cm} r}
علتی باشد که آن اندر بهاران بد شود
&&
گر زمستان بد بود اندر بهاران صد شود
\\
بر بهار جان فزا زنهار تو جرمی منه
&&
علت ناصور تو گر زانک گرگ و دد شود
\\
هر درخت و باغ را داده بهاران بخششی
&&
هر درخت تلخ و شیرین آنچ می‌ارزد شود
\\
ای برادر از رهی این یک سخن را گوش دار
&&
هر نباتی این نیرزد آنک چون سر زد شود
\\
از هزاران آب شهوت ناگهان آبی بود
&&
کز خمیرش صورت حسن و جمال و خد شود
\\
وانگه آن حسن و جمالان خرج گردد صد هزار
&&
تا یکی را خود از آن‌ها دولتی باشد شود
\\
نیکبختان در جهان بسیار آیند و روند
&&
لیک بر درگاه شمس الدین نباید رد شود
\\
هر که او یک سجده کردش گر چه کردش از نفاق
&&
در دو عالم عاقبت او خاصه ایزد شود
\\
از جفاها یاد ماور ای حریف باوفا
&&
زانک یاد آن جفاها در ره تو سد شود
\\
\end{longtable}
\end{center}
