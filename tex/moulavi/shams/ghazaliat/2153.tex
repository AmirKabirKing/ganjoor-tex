\begin{center}
\section*{غزل شماره ۲۱۵۳: ای تو امان هر بلا ما همه در امان تو}
\label{sec:2153}
\addcontentsline{toc}{section}{\nameref{sec:2153}}
\begin{longtable}{l p{0.5cm} r}
ای تو امان هر بلا ما همه در امان تو
&&
جان همه خوش است در سایه لطف جان تو
\\
شاه همه جهان تویی اصل همه کسان تویی
&&
چونک تو هستی آن ما نیست غم از کسان تو
\\
ابر غم تو ای قمر آمد دوش بر جگر
&&
گفت مرا ز بام و در صد سقط از زبان تو
\\
جست دلم ز قال او رفت بر خیال او
&&
شاید ای نبات خو این همه در زمان تو
\\
جان مرا در این جهان آتش توست در دهان
&&
از هوس وصال تو وز طلب جهان تو
\\
نیست مرا ز جسم و جان در ره عشق تو نشان
&&
ز آنک نغول می‌روم در طلب نشان تو
\\
بنده بدید جوهرت لنگ شده‌ست بر درت
&&
مانده‌ام ای جواهری بر طرف دکان تو
\\
شاد شود دل و جگر چون بگشایی آن کمر
&&
بازگشا تو خوش قبا آن کمر از میان تو
\\
تا نظری به جان کنی جان مرا چو کان کنی
&&
در تبریز شمس دین نقد رسم به کان تو
\\
\end{longtable}
\end{center}
