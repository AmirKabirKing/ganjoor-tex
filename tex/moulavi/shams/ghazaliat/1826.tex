\begin{center}
\section*{غزل شماره ۱۸۲۶: هر کی ز حور پرسدت رخ بنما که همچنین}
\label{sec:1826}
\addcontentsline{toc}{section}{\nameref{sec:1826}}
\begin{longtable}{l p{0.5cm} r}
هر کی ز حور پرسدت رخ بنما که همچنین
&&
هر کی ز ماه گویدت بام برآ که همچنین
\\
هر کی پری طلب کند چهره خود بدو نما
&&
هر کی ز مشک دم زند زلف گشا که همچنین
\\
هر کی بگویدت ز مه ابر چگونه وا شود
&&
باز گشا گره گره بند قبا که همچنین
\\
گر ز مسیح پرسدت مرده چگونه زنده کرد
&&
بوسه بده به پیش او جان مرا که همچنین
\\
هر کی بگویدت بگو کشته عشق چون بود
&&
عرضه بده به پیش او جان مرا که همچنین
\\
هر کی ز روی مرحمت از قد من بپرسدت
&&
ابروی خویش عرضه ده گشته دوتا که همچنین
\\
جان ز بدن جدا شود باز درآید اندرون
&&
هین بنما به منکران خانه درآ که همچنین
\\
هر طرفی که بشنوی ناله عاشقانه‌ای
&&
قصه ماست آن همه حق خدا که همچنین
\\
خانه هر فرشته‌ام سینه کبود گشته‌ام
&&
چشم برآر و خوش نگر سوی سما که همچنین
\\
سر وصال دوست را جز به صبا نگفته‌ام
&&
تا به صفای سر خود گفت صبا که همچنین
\\
کوری آنک گوید او بنده به حق کجا رسد
&&
در کف هر یکی بنه شمع صفا که همچنین
\\
گفتم بوی یوسفی شهر به شهر کی رود
&&
بوی حق از جهان هو داد هوا که همچنین
\\
گفتم بوی یوسفی چشم چگونه وادهد
&&
چشم مرا نسیم تو داد ضیا که همچنین
\\
از تبریز شمس دین بوک مگر کرم کند
&&
وز سر لطف برزند سر ز وفا که همچنین
\\
\end{longtable}
\end{center}
