\begin{center}
\section*{غزل شماره ۳۹: آه که آن صدر سرا می‌ندهد بار مرا}
\label{sec:0039}
\addcontentsline{toc}{section}{\nameref{sec:0039}}
\begin{longtable}{l p{0.5cm} r}
آه که آن صدر سرا می‌ندهد بار مرا
&&
می‌نکند محرم جان محرم اسرار مرا
\\
نغزی و خوبی و فرش آتش تیز نظرش
&&
پرسش همچون شکرش کرد گرفتار مرا
\\
گفت مرا مهر تو کو رنگ تو کو فر تو کو
&&
رنگ کجا ماند و بو ساعت دیدار مرا
\\
غرقه جوی کرمم بنده آن صبحدمم
&&
کان گل خوش بوی کشد جانب گلزار مرا
\\
هر که به جوبار بود جامه بر او بار بود
&&
چند زیانست و گران خرقه و دستار مرا
\\
ملکت و اسباب کز این ماه رخان شکرین
&&
هست به معنی چو بود یار وفادار مرا
\\
دستگه و پیشه تو را دانش و اندیشه تو را
&&
شیر تو را بیشه تو را آهوی تاتار مرا
\\
نیست کند هست کند بی‌دل و بی‌دست کند
&&
باده دهد مست کند ساقی خمار مرا
\\
ای دل قلاش مکن فتنه و پرخاش مکن
&&
شهره مکن فاش مکن بر سر بازار مرا
\\
گر شکند پند مرا زفت کند بند مرا
&&
بر طمع ساختن یار خریدار مرا
\\
بیش مزن دم ز دوی دو دو مگو چون ثنوی
&&
اصل سبب را بطلب بس شد از آثار مرا
\\
\end{longtable}
\end{center}
