\begin{center}
\section*{غزل شماره ۱۶۶۷: من ز وصلت چون به هجران می روم}
\label{sec:1667}
\addcontentsline{toc}{section}{\nameref{sec:1667}}
\begin{longtable}{l p{0.5cm} r}
من ز وصلت چون به هجران می روم
&&
در بیابان مغیلان می روم
\\
من به خود کی رفتمی او می کشد
&&
تا نپنداری که خواهان می روم
\\
چشم نرگس خیره در من مانده‌ست
&&
کز میان باغ و بستان می روم
\\
عقل هم انگشت خود را می گزد
&&
زانک جان این جاست و بی‌جان می روم
\\
دست ناپیدا گریبان می کشد
&&
من پی دست و گریبان می روم
\\
این چنین پیدا و پنهان دست کیست
&&
تا که من پیدا و پنهان می روم
\\
این همان دست است کاول او مرا
&&
جمع کرد و من پریشان می روم
\\
در تماشای چنین دست عجب
&&
من شدم از دست و حیران می روم
\\
من چو از دریای عمان قطره‌ام
&&
قطره قطره سوی عمان می روم
\\
من چو از کان معانی یک جوم
&&
همچنین جو جو بدان کان می روم
\\
من چو از خورشید کیوان ذره‌ام
&&
ذره ذره سوی کیوان می روم
\\
این سخن پایان ندارد لیک من
&&
آمدم زان سر به پایان می روم
\\
\end{longtable}
\end{center}
