\begin{center}
\section*{غزل شماره ۶۳۲: عید آمد و عید آمد وان بخت سعید آمد}
\label{sec:0632}
\addcontentsline{toc}{section}{\nameref{sec:0632}}
\begin{longtable}{l p{0.5cm} r}
عید آمد و عید آمد وان بخت سعید آمد
&&
برگیر و دهل می‌زن کان ماه پدید آمد
\\
عید آمد ای مجنون غلغل شنو از گردون
&&
کان معتمد سدره از عرش مجید آمد
\\
عید آمد ره جویان رقصان و غزل گویان
&&
کان قیصر مه رویان زان قصر مشید آمد
\\
صد معدن دانایی مجنون شد و سودایی
&&
کان خوبی و زیبایی بی‌مثل و ندید آمد
\\
زان قدرت پیوستش داوود نبی مستش
&&
تا موم کند دستش گر سنگ و حدید آمد
\\
عید آمد و ما بی‌او عیدیم بیا تا ما
&&
بر عید زنیم این دم کان خوان و ثرید آمد
\\
زو زهر شکر گردد زو ابر قمر گردد
&&
زو تازه و تر گردد هر جا که قدید آمد
\\
برخیز به میدان رو در حلقه رندان رو
&&
رو جانب مهمان رو کز راه بعید آمد
\\
غم‌هاش همه شادی بندش همه آزادی
&&
یک دانه بدو دادی صد باغ مزید آمد
\\
من بنده آن شرقم در نعمت آن غرقم
&&
جز نعمت پاک او منحوس و پلید آمد
\\
بربند لب و تن زن چون غنچه و چون سوسن
&&
رو صبر کن از گفتن چون صبر کلید آمد
\\
\end{longtable}
\end{center}
