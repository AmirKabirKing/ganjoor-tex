\begin{center}
\section*{غزل شماره ۴۸۳: هر آنک از سبب وحشت غمی تنهاست}
\label{sec:0483}
\addcontentsline{toc}{section}{\nameref{sec:0483}}
\begin{longtable}{l p{0.5cm} r}
هر آنک از سبب وحشت غمی تنهاست
&&
بدانک خصم دلست و مراقب تن‌هاست
\\
به چنگ و تنتن این تن نهاده‌ای گوشی
&&
تن تو توده خاکست و دمدمه ش چو هواست
\\
هوای نفس تو همچون هوای گردانگیز
&&
عدو دیده و بیناییست و خصم ضیاست
\\
تویی مگر مگس این مطاعم عسلین
&&
که زامقلو تو را درد و زانقلوه عناست
\\
در آن زمان که در این دوغ می‌فتی چو مگس
&&
عجب که توبه و عقل و رأیت تو کجاست
\\
به عهد و توبه چرا چون فتیله می‌پیچی
&&
که عهد تو چو چراغی رهین هر نکباست
\\
بگو به یوسف یعقوب هجر را دریاب
&&
که بی ز پیرهن نصرت تو حبس عماست
\\
چو گوشت پاره ضریریست مانده بر جایی
&&
چو مرده‌ای‌ست ضریر و عقیله احیاست
\\
به جای دارو او خاک می‌زند در چشم
&&
بدان گمان که مگر سرمه است و خاک و دواست
\\
چو لا تعاف من الکافرین دیارا
&&
دعای نوح نبیست و او مجاب دعاست
\\
همیشه کشتی احمق غریق طوفان‌ست
&&
که زشت صنعت و مبغوض گوهر و رسواست
\\
اگر چه بحر کرم موج می‌زند هر سو
&&
به حکم عدل خبیثات مر خبیثین راست
\\
قفا همی‌خور و اندرمکش کلا گردن
&&
چنان گلو که تو داری سزای صفع و قفاست
\\
گلو گشاده چو فرج فراخ ماده خران
&&
که کیر خر نرهد زو چو پیش او برخاست
\\
بخور تو ای سگ گرگین شکنبه و سرگین
&&
شکمبه و دهن سگ بلی سزا به سزاست
\\
بیا بخور خر مرده سگ شکار نه‌ای
&&
ز پوز و ز شکم و طلعت تو خود پیداست
\\
سگ محله و بازار صید کی گیرد
&&
مقام صید سر کوه و بیشه و صحراست
\\
رها کن این همه را نام یار و دلبر گو
&&
که زشت‌ها که بدو دررسد همه زیباست
\\
که کیمیاست پناه وی و تعلق او
&&
مصرف همه ذرات اسفل و اعلاست
\\
نهان کند دو جهان را درون یک ذره
&&
که از تصرف او عقل گول و نابیناست
\\
بدانک زیرکی عقل جمله دهلیزیست
&&
اگر به علم فلاطون بود برون سراست
\\
جنون عشق به از صد هزار گردون عقل
&&
که عقل دعوی سر کرد و عشق بی‌سر و پاست
\\
هر آنک سر بودش بیم سر همش باشد
&&
حریف بیم نباشد هر آنک شیر وغاست
\\
رود درونه سم الخیاط رشته عشق
&&
که سر ندارد و بی‌سر مجرد و یکتاست
\\
قلاوزی کندش سوزن و روان کندش
&&
که تا وصال ببخشد به پاره‌ها که جداست
\\
حدیث سوزن و رشته بهل که باریکست
&&
حدیث موسی جان کن که با ید بیضاست
\\
حدیث قصه آن بحر خوشدلی‌ها گو
&&
که قطره قطره او مایه دو صد دریاست
\\
چو کاسه بر سر بحری و بی‌خبر از بحر
&&
ببین ز موج تو را هر نفس چه گردشهاست
\\
\end{longtable}
\end{center}
