\begin{center}
\section*{غزل شماره ۳۰۵۰: خدایگان جمال و خلاصه خوبی}
\label{sec:3050}
\addcontentsline{toc}{section}{\nameref{sec:3050}}
\begin{longtable}{l p{0.5cm} r}
خدایگان جمال و خلاصه خوبی
&&
به جان و عقل درآمد به رسم گل کوبی
\\
بیا بیا که حیات و نجات خلق تویی
&&
بیا بیا که تو چشم و چراغ یعقوبی
\\
قدم بنه تو بر آب و گلم که از قدمت
&&
ز آب و گل برود تیرگی و محجوبی
\\
ز تاب تو برسد سنگ‌ها به یاقوتی
&&
ز طالبیت رسد طالبی به مطلوبی
\\
بیا بیا که جمال و جلال می‌بخشی
&&
بیا بیا که دوای هزار ایوبی
\\
بیا بیا تو اگر چه نرفته‌ای هرگز
&&
ولیک هر سخنی گویمت به مرغوبی
\\
به جای جان تو نشین که هزار چون جانی
&&
محب و عاشق خود را تو کش که محبوبی
\\
اگر نه شاه جهان اوست ای جهان دژم
&&
به جان او که بگویی چرا در آشوبی
\\
گهی ز رایت سبزش لطیف و سرسبزی
&&
ز قلب لشکر هیجاش گاه مقلوبی
\\
دمی چو فکرت نقاش نقش‌ها سازی
&&
گهی چو دسته فراش فرش‌ها روبی
\\
چو نقش را تو بروبی خلاصه آن را
&&
فرشتگی دهی و پر و بال کروبی
\\
خموش آب نگهدار همچو مشک درست
&&
ور از شکاف بریزی بدانک معیوبی
\\
به شمس مفخر تبریز از آن رسید دلت
&&
که چست دلدل دل می‌نمود مرکوبی
\\
\end{longtable}
\end{center}
