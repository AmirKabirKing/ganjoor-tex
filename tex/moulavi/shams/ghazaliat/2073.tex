\begin{center}
\section*{غزل شماره ۲۰۷۳: دلا تو شهد منه در دهان رنجوران}
\label{sec:2073}
\addcontentsline{toc}{section}{\nameref{sec:2073}}
\begin{longtable}{l p{0.5cm} r}
دلا تو شهد منه در دهان رنجوران
&&
حدیث چشم مگو با جماعت کوران
\\
اگر چه از رگ گردن به بنده نزدیک است
&&
خدای دور بود از بر خدادوران
\\
درون خویش بپرداز تا برون آیند
&&
ز پرده‌ها به تجلی چو ماه مستوران
\\
اگر چه گم شوی از خویش و از جهان این جا
&&
برون خویش و جهان گشته‌ای ز مشهوران
\\
اگر تو ماه وصالی نشان بده از وصل
&&
ز ساعد و بر سیمین و چهره حوران
\\
وگر چو زر ز فراقی کجاست داغ فراق
&&
چنین فسرده بود سکه‌های مهجوران
\\
چو نیست عشق تو را بندگی به جا می‌آر
&&
که حق فرونهلد مزدهای مزدوران
\\
بدانک عشق خدا خاتم سلیمانی است
&&
کجاست دخل سلیمان و مکسب موران
\\
لباس فکرت و اندیشه‌ها برون انداز
&&
که آفتاب نتابد مگر که بر عوران
\\
پناه گیر تو در زلف شمس تبریزی
&&
که مشک بارد تا وارهی ز کافوران
\\
\end{longtable}
\end{center}
