\begin{center}
\section*{غزل شماره ۳۲۰۴: یا ملک المبعث والمحشر}
\label{sec:3204}
\addcontentsline{toc}{section}{\nameref{sec:3204}}
\begin{longtable}{l p{0.5cm} r}
یا ملک المبعث والمحشر
&&
لیس سوی صدرک من مصدر
\\
سر نبری ای سر، اگر سر بری
&&
آن ز خری دان که تو سر واخری
\\
مقلة عینی لک یا ناظری
&&
نظرة قلبی لک یا منظری
\\
همچو پری، باش ز خلقان نهان
&&
بر نپری تا نشنوی چون پری
\\
غاب فؤادی لم غیبته
&&
بعد حضوری لک، یا محضری
\\
بر سر خشکی چو ثقیلان مران
&&
برتر از آنی که روی برتری
\\
منزلناالعرش و ما فوقه
&&
عمرک یا نفس قمی، سافری
\\
جمله چو دردند به پایان خم
&&
سرور از آنی تو، که تو سروری
\\
قلت الا بدلنا سلما
&&
اسلمک الصبر قفی واصبری
\\
چند پس پرده و از در برون
&&
بر در این پرده، اگر بر دری
\\
قالت هل صبری الا به
&&
هل عقدالبیع بلا مشتری
\\
می مفروش از جهت حرص زر
&&
جوهر می خود بنماید زری
\\
اذ حضرالراح فما فاتنا
&&
افتح عینیک به وابصری
\\
می بفروشی، چه خری؟! جز که غم
&&
دین بفروشی چه بری؟! کافری
\\
قر به‌العین کلی واشربی
&&
قد قرب‌امنزل فاستبشری
\\
وصلت فانی ننماید بقا
&&
زن نشود حامله از سعتری
\\
\end{longtable}
\end{center}
