\begin{center}
\section*{غزل شماره ۲۵۶۳: من پای همی‌کوبم ای جان و جهان دستی}
\label{sec:2563}
\addcontentsline{toc}{section}{\nameref{sec:2563}}
\begin{longtable}{l p{0.5cm} r}
من پای همی‌کوبم ای جان و جهان دستی
&&
ای جان و جهان برجه از بهر دل مستی
\\
ای مست مکش محشر بازآی ز شور و شر
&&
آن دست بر آن دل نه ای کاش دلی هستی
\\
ترک دل و جان کردم تا بی‌دل و جان گردم
&&
یک دل چه محل دارد صد دلکده بایستی
\\
بنگر به درخت ای جان در رقص و سراندازی
&&
اشکوفه چرا کردی گر باده نخوردستی
\\
آن باد بهاری بین آمیزش و یاری بین
&&
گر نی همه لطفستی با خاک نپیوستی
\\
از یار مکن افغان بی‌جور نیامد عشق
&&
گر نی ره عشق این است او کی دل ما خستی
\\
صد لطف و عطا دارد صد مهر و وفا دارد
&&
گر غیرت بگذارد دل بر دل ما بستی
\\
با جمله جفاکاری پشتی کند و یاری
&&
گر پشتی او نبود پشت همه بشکستی
\\
دامی که در او عنقا بی‌پر شود و بی‌پا
&&
بی‌رحمت او صعوه زین دام کجا خستی
\\
خامش کن و ساکن شو ای باد سخن گر چه
&&
در جنبش باد دل صد مروحه بایستی
\\
شمس الحق تبریزی ماییم و شب وحشت
&&
گر شمس نبودی شب از خویش کجا رستی
\\
\end{longtable}
\end{center}
