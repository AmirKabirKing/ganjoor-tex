\begin{center}
\section*{غزل شماره ۵۲۹: خامی سوی پالیز جان آمد که تا خربز خورد}
\label{sec:0529}
\addcontentsline{toc}{section}{\nameref{sec:0529}}
\begin{longtable}{l p{0.5cm} r}
خامی سوی پالیز جان آمد که تا خربز خورد
&&
دیدی تو یا خود دید کس کاندر جهان خر بز خورد
\\
ترونده پالیز جان هر گاو و خر را کی رسد
&&
زان میوه‌های نادره زیرک دل و گربز خورد
\\
آن کس که در مغرب بود یابد خورش از اندلس
&&
وان کس که در مشرق بود او نعمت هرمز خورد
\\
چون خدمت قیصر کند او راتبه قیصر خورد
&&
چون چاکر اربز بود از مطبخ اربز خورد
\\
آن کو به غصب و دزدیی آهنگ پالیزی کند
&&
از داد و داور عاقبت اشکنجه‌های غز خورد
\\
ترک آن بود کز بیم او دیه از خراج ایمن بود
&&
ترک آن نباشد کز طمع سیلی هر قنسز خورد
\\
وان عقل پرمغزی که او در نوبهاری دررسد
&&
از پوست‌ها فارغ شود کی غصه قندز خورد
\\
صفراییی کز طبع بد از نار شیرین می‌رمد
&&
نار ترش خواهد ولی آن به که نار مز خورد
\\
خامش نخواهد خورد خود این راح‌های روح را
&&
آن کس که از جوع البقر ده مرده ماش و رز خورد
\\
\end{longtable}
\end{center}
