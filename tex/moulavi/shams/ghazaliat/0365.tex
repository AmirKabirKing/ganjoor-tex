\begin{center}
\section*{غزل شماره ۳۶۵: می‌دان که زمانه نقش سوداست}
\label{sec:0365}
\addcontentsline{toc}{section}{\nameref{sec:0365}}
\begin{longtable}{l p{0.5cm} r}
می‌دان که زمانه نقش سوداست
&&
بیرون ز زمانه صورت ماست
\\
زیرا قفسی‌ست این زمانه
&&
بیرون همه کوه قاف و عنقاست
\\
جویی‌ست جهان و ما برونیم
&&
بر جوی فتاده سایه ماست
\\
این جا سر نکته‌ای‌ست مشکل
&&
این جا نبود ولیکن این جاست
\\
جز در رخ جان مخند ای دل
&&
بی او همه خنده گریه افزاست
\\
آن دل نبود که باشد او تنگ
&&
زان روی که دل فراخ پهناست
\\
دل غم نخورد غذاش غم نیست
&&
طوطی‌ست دل و عجب شکرخاست
\\
مانند درخت سر قدم ساز
&&
زیرا که ره تو زیر و بالاست
\\
شاخ ار چه نظر به بیخ دارد
&&
کان قوت مغز او هم از پاست
\\
\end{longtable}
\end{center}
