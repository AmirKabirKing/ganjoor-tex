\begin{center}
\section*{غزل شماره ۲۲۲: رویم و خانه بگیریم پهلوی دریا}
\label{sec:0222}
\addcontentsline{toc}{section}{\nameref{sec:0222}}
\begin{longtable}{l p{0.5cm} r}
رویم و خانه بگیریم پهلوی دریا
&&
که داد اوست جواهر که خوی اوست سخا
\\
بدان که صحبت جان را همی‌کند همرنگ
&&
ز صحبت فلک آمد ستاره خوش سیما
\\
نه تن به صحبت جان خوبروی و خوش فعل‌ست
&&
چه می‌شود تن مسکین چو شد ز جان عذرا
\\
چو دست متصل توست بس هنر دارد
&&
چو شد ز جسم جدا اوفتاد اندر پا
\\
کجاست آن هنر تو نه که همان دستی
&&
نه این زمان فراق‌ست و آن زمان لقا
\\
پس الله الله زنهار ناز یار بکش
&&
که ناز یار بود صد هزار من حلوا
\\
فراق را بندیدی خدات منما یاد
&&
که این دعاگو به زین نداشت هیچ دعا
\\
ز نفس کلی چون نفس جزو ما ببرید
&&
به اهبطوا و فرود آمد از چنان بالا
\\
مثال دست بریده ز کار خویش بماند
&&
که گشت طعمه گربه زهی ذلیل و بلا
\\
ز دست او همه شیران شکسته پنجه بدند
&&
که گربه می‌کشدش سو به سو ز دست قضا
\\
امید وصل بود تا رگیش می‌جنبد
&&
که یافت دولت وصلت هزار دست جدا
\\
مدار این عجب از شهریار خوش پیوند
&&
که پاره پاره دود از کفش شدست سما
\\
شه جهانی و هم پاره دوز استادی
&&
بکن نظر سوی اجزای پاره پاره ما
\\
چو چنگ ما بشکستی بساز و کش سوی خود
&&
ز الست زخمه همی‌زن همی‌پذیر بلا
\\
بلا کنیم ولیکن بلی اول کو
&&
که آن چو نعره روحست وین ز کوه صدا
\\
چو نای ما بشکستی شکسته را بربند
&&
نیاز این نی ما را ببین بدان دم‌ها
\\
که نای پاره ما پاره می‌دهد صد جان
&&
که کی دمم دهد او تا شوم لطیف ادا
\\
\end{longtable}
\end{center}
