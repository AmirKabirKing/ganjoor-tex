\begin{center}
\section*{غزل شماره ۱۹۹۰: جان حیوان که ندیده است بجز کاه و عطن}
\label{sec:1990}
\addcontentsline{toc}{section}{\nameref{sec:1990}}
\begin{longtable}{l p{0.5cm} r}
جان حیوان که ندیده است به جز کاه و عطن
&&
شد ز تبدیل خدا لایق گلزار فطن
\\
نوبهاری است خدا را جز از این فصل بهار
&&
که در او مرده نماند وثنی و نه وثن
\\
ز نسیمش شود آن جغد به از باز سپید
&&
بهتر از شیر شود از دم او ماده زغن
\\
زنده گشتند و پی شکر دهان بگشادند
&&
بوسه‌ها مست شدند از طرب بوی دهن
\\
دست دستان صبا لخلخه را شورانید
&&
تا بیاموخت به طفلان چمن خلق حسن
\\
جبرئیل است مگر باد و درختان مریم
&&
دست بازی نگر آن سان که کند شوهر و زن
\\
ابر چون دید که در زیر تتق خوبانند
&&
برفشانید نثار گهر و در عدن
\\
چون گل سرخ گریبان ز طرب بدرانید
&&
وقت آن شد که به یعقوب رسد پیراهن
\\
چون عقیق یمنی لب دلبر خندید
&&
بوی رحمان به محمد رسد از سوی یمن
\\
چند گفتیم پراکنده دل آرام نیافت
&&
جز بدان جعد پراکنده آن خوب زمن
\\
شمس تبریز برآ تیغ بزن چون خورشید
&&
تیغ خورشید دهد نور به جان چو مجن
\\
\end{longtable}
\end{center}
