\begin{center}
\section*{غزل شماره ۱۸۶۳: ای سنجق نصرالله وی مشعله یاسین}
\label{sec:1863}
\addcontentsline{toc}{section}{\nameref{sec:1863}}
\begin{longtable}{l p{0.5cm} r}
ای سنجق نصرالله وی مشعله یاسین
&&
یا رب چه سبک روحی بر چشم و سرم بنشین
\\
ای تاج هنرمندی معراج خردمندی
&&
تعریف چه می باید چون جمله تویی تعیین
\\
هر ذره که می جنبد هر برگ که می خنبد
&&
بی کام و زبان گفتی در گوش فلک بنشین
\\
جان همه جانا ای دولت مولانا
&&
جان را برهانیدی از ناز فلان الدین
\\
از نفخ تو می روید پر ملاء الاعلی
&&
وز شرق تو می تفسد پشت فلک عنین
\\
از عشق جهان سوزت وز شوق جگردوزت
&&
بی هیچ دعاگویی عالم شده پرآمین
\\
ناگاه سحرگاهی بی‌رخنه و بیراهی
&&
آورد طبیب جان یک خمره پرافسنتین
\\
تا این تن بیمارم وین کشته دل زارم
&&
زنده شد و چابک شد برداشت سر از بالین
\\
گفتم که ملیحی تو مانا که مسیحی تو
&&
شاد آمدی ای سلطان ای چاره هر مسکین
\\
پیغامبر بیماران نافعتری از باران
&&
در خمره چه داری گفت داروی دل غمگین
\\
حرز دل یعقوبم سرچشمه ایوبم
&&
هم چستم و هم خوبم هم خسرو و هم شیرین
\\
گفتم که چنان دریا در خمره کجا گنجد
&&
گفتا که چه دانی تو این شیوه و این آیین
\\
کی داند چون آخر استادی بی‌چون را
&&
گنجاند در سجین او عالم علیین
\\
یوسف به بن چاهی بر هفت فلک ناظر
&&
و اندر شکم ماهی یونس زبر پروین
\\
گر فوقی وگر پستی هستی طلب و مستی
&&
نی بر زبرین وقف است این بخت نه بر زیرین
\\
خامش که نمی‌گنجد این حصه در این قصه
&&
رو چشم به بالا کن روی چو مهش می بین
\\
\end{longtable}
\end{center}
