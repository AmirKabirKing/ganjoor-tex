\begin{center}
\section*{غزل شماره ۳۱۱۵: در لطف اگر بروی شاه همه چمنی}
\label{sec:3115}
\addcontentsline{toc}{section}{\nameref{sec:3115}}
\begin{longtable}{l p{0.5cm} r}
در لطف اگر بروی شاه همه چمنی
&&
در قهر اگر بروی که را ز بن بکنی
\\
دانی که بر گل تو بلبل چه ناله کند
&&
املی الهوی اسقا یوم النوی بدنی
\\
عقل از تو تازه بود جان از تو زنده بود
&&
تو عقل عقل منی تو جان جان منی
\\
من مست نعمت تو دانم ز رحمت تو
&&
کز من به هر گنهی دل را تو برنکنی
\\
تاج تو بر سر ما نور تو در بر ما
&&
بوی تو رهبر ما گر راه ما نزنی
\\
حارس تویی رمه را ایمن کنی همه را
&&
اهوی الهوا امنو فی ظل ذو المننی
\\
آن دم که دم بزنم با تو ز خود بروم
&&
لو لا مخاطبتی ایاک لم ترنی
\\
ای جان اسیر تنی وی تن حجاب منی
&&
وی سر تو در رسنی وی دل تو در وطنی
\\
ای دل چو در وطنی یاد آر صحبت ما
&&
آخر رفیق بدی در راه ممتحنی
\\
\end{longtable}
\end{center}
