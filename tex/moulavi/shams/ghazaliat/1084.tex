\begin{center}
\section*{غزل شماره ۱۰۸۴: مه روزه اندرآمد هله ای بت چو شکر}
\label{sec:1084}
\addcontentsline{toc}{section}{\nameref{sec:1084}}
\begin{longtable}{l p{0.5cm} r}
مه روزه اندرآمد هله ای بت چو شکر
&&
گه بوسه است تنها نه کنار و چیز دیگر
\\
بنشین نظاره می‌کن ز خورش کناره می‌کن
&&
دو هزار خشک لب بین به کنار حوض کوثر
\\
اگر آتش است روزه تو زلال بین نه کوزه
&&
تری دماغت آرد چو شراب همچو آذر
\\
چو عجوزه گشت گریان شه روزه گشت خندان
&&
دل نور گشت فربه تن موم گشت لاغر
\\
رخ عاشقان مزعفر رخ جان و عقل احمر
&&
منگر برون شیشه بنگر درون ساغر
\\
همه مست و خوش شکفته رمضان ز یاد رفته
&&
به وثاق ساقی خود بزدیم حلقه بر در
\\
چو بدید مست ما را بگزید دست‌ها را
&&
سر خود چنین چنین کرد و بتافت روز معشر
\\
ز میانه گفت مستی خوش و شوخ و می‌پرستی
&&
که کی گوید اینک روزه شکند ز قند و شکر
\\
شکر از لبان عیسی که بود حیات موتی
&&
که ز ذوق باز ماند دهن نکیر و منکر
\\
تو اگر خراب و مستی به من آ که از منستی
&&
و اگر خمار یاری سخنی شنو مخمر
\\
چو خوشی چه خوش نهادی به کدام روز زادی
&&
به کدام دست کردت قلم قضا مصور
\\
تن تو حجاب عزت پس او هزار جنت
&&
شکران و ماه رویان همه همچو مه مطهر
\\
هله مطرب شکرلب برسان صدا به کوکب
&&
که ز صید بازآمد شه ما خوش و مظفر
\\
ز تو هر صباح عیدی ز تو هر شبست قدری
&&
نه چو قدر عامیانه که شبی بود مقدر
\\
تو بگو سخن که جانی قصصات آسمانی
&&
که کلام تست صافی و حدیث من مکدر
\\
\end{longtable}
\end{center}
