\begin{center}
\section*{غزل شماره ۵۸۷: صلا جان‌های مشتاقان که نک دلدار خوب آمد}
\label{sec:0587}
\addcontentsline{toc}{section}{\nameref{sec:0587}}
\begin{longtable}{l p{0.5cm} r}
صلا جان‌های مشتاقان که نک دلدار خوب آمد
&&
چو زرکوبست آن دلبر رخ من سیم کوب آمد
\\
از او کو حسن مه دارد هر آن کو دل نگه دارد
&&
به خاک پای آن دلبر که آن کس سنگ و چوب آمد
\\
هر آنک از عشق بگریزد حقیقت خون خود ریزد
&&
کجا خورشید را هرگز ز مرغ شب غروب آمد
\\
بروب از خویش این خانه ببین آن حس شاهانه
&&
برو جاروب لا بستان که لا بس خانه روب آمد
\\
تن تو همچو خاک آمد دم تو تخم پاک آمد
&&
هوس‌ها چون ملخ‌ها شد نفس‌ها چون حبوب آمد
\\
ز بینایی بگردیدی مگر خواب دگر دیدی
&&
چه خوردی تو که قاروره پر از خلط رسوب آمد
\\
تو چه شنیدی تو چه گفتی بگو تا شب کجا خفتی
&&
حکایت می‌کند رنگت که جاسوس القلوب آمد
\\
صلاح الدین یعقوبان جواهربخش زرکوبان
&&
که او خورشید اسرارست و علام الغیوب آمد
\\
\end{longtable}
\end{center}
