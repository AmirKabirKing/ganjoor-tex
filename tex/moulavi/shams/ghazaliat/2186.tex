\begin{center}
\section*{غزل شماره ۲۱۸۶: خداوندا چو تو صاحب قران کو}
\label{sec:2186}
\addcontentsline{toc}{section}{\nameref{sec:2186}}
\begin{longtable}{l p{0.5cm} r}
خداوندا چو تو صاحب قران کو
&&
برابر با مکان تو مکان کو
\\
زمان محتاج و مسکین تو باشد
&&
تو را حاجت به دوران و زمان کو
\\
کسی کو گفت دیدم شمس دین را
&&
سؤالش کن که راه آسمان کو
\\
در آن دریا مرو بی‌امر دریا
&&
نمی‌ترسی برای تو ضمان کو
\\
مگر بی‌قصد افتی کو کریم است
&&
خطاکن را ز عفو او غمان کو
\\
چو سجده کرد آیینه مر او را
&&
بر آن آیینه زنگار گمان کو
\\
همو تیر است همو اسپر همو قوس
&&
چه گفتم آن طرف تیر و کمان کو
\\
هر آن جسمی که از لطفش نظر یافت
&&
نظیرش در ولایت‌های جان کو
\\
بجز از روی عجز و فقر و تسلیم
&&
ببرده سر از او از انس و جان کو
\\
ز غیرت حق شد حارس و گر نی
&&
مر او را از کی بیم است پاسبان کو
\\
به پیشانی جانا داغ مهرش
&&
کسی بی‌داغ مهرش در قران کو
\\
به نوبتگاه او بین صف کشیده
&&
به خدمت گر همی‌جویی مهان کو
\\
نباشد خنده جز از زعفرانش
&&
بجز از عشق رویش شادمان کو
\\
بجز از هجر آن مخدوم جانی
&&
دل و جان را به عالم اندهان کو
\\
خداوند شمس دین از بهر الله
&&
که لایق در ثنای او دهان کو
\\
زبان و جان من با وصل او رفت
&&
به شرح خاک تبریزم زبان کو
\\
همه کان هست محتاج خریدار
&&
بدان حد بی‌نیازی هیچ کان کو
\\
\end{longtable}
\end{center}
