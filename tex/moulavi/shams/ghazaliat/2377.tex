\begin{center}
\section*{غزل شماره ۲۳۷۷: ای خداوند یکی یار جفاکارش ده}
\label{sec:2377}
\addcontentsline{toc}{section}{\nameref{sec:2377}}
\begin{longtable}{l p{0.5cm} r}
ای خداوند یکی یار جفاکارش ده
&&
دلبری عشوه ده سرکش خون خوارش ده
\\
تا بداند که شب ما به چه سان می‌گذرد
&&
غم عشقش ده و عشقش ده و بسیارش ده
\\
چند روزی جهت تجربه بیمارش کن
&&
با طبیبی دغلی پیشه سر و کارش ده
\\
ببرش سوی بیابان و کن او را تشنه
&&
یک سقایی حجری سینه سبکسارش ده
\\
گمرهش کن که ره راست نداند سوی شهر
&&
پس قلاوز کژ بیهده رفتارش ده
\\
عالم از سرکشی آن مه سرگشته شدند
&&
مدتی گردش این گنبد دوارش ده
\\
کو صیادی که همی‌کرد دل ما را پار
&&
زو ببر سنگ دلی و دل پیرارش ده
\\
منکر پار شده‌ست او که مرا یاد نماند
&&
ببر انکار از او و دم اقرارش ده
\\
گفتم آخر به نشانی که به دربان گفتی
&&
که فلانی چو بیاید بر ما بارش ده
\\
گفت آمد که مرا خواجه ز بالا گیرد
&&
رو بجو همچو خودی ابله و آچارش ده
\\
بس کن ای ساقی و کس را چو رهی مست مکن
&&
ور کنی مست بدین حد ره هموارش ده
\\
\end{longtable}
\end{center}
