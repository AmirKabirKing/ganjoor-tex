\begin{center}
\section*{غزل شماره ۲۴۰۱: ای صد هزار خرمن‌ها را بسوخته}
\label{sec:2401}
\addcontentsline{toc}{section}{\nameref{sec:2401}}
\begin{longtable}{l p{0.5cm} r}
ای صد هزار خرمن‌ها را بسوخته
&&
زین پس مدار خرمن ما را بسوخته
\\
از عشق سنگ خارا بر آهنی زده
&&
برقی بجسته ز آهن و خارا بسوخته
\\
از سر قدم بساختم ای آفتاب حسن
&&
هم سر به جوش آمده هم پا بسوخته
\\
سرنای این دلم ز تو بنواخت پرده‌ای
&&
هم پرده‌اش دریده و سرنا بسوخته
\\
در اصل زمهریر گر افتد ز آتشت
&&
تا روز حشر بینی سرما بسوخته
\\
از عالم نه جای ندا کرد عشق تو
&&
هر جان که گوش داشته برجا بسوخته
\\
ای لطف سوزشی که شرار جمال تو
&&
جان را کشیده پیش و به عمدا بسوخته
\\
آن روی سرخ را می احمر دمی بدید
&&
صفرای عشق او می حمرا بسوخته
\\
آن خد احمر ار بنمایی دمی دگر
&&
سودای تو برآید و صفرا بسوخته
\\
طبعی که لاف زلف مطرا همی‌زدی
&&
از جعد طره تو مطرا بسوخته
\\
در وا شدم به جستن تو جانب فلک
&&
در وا نگشت ماندم دروا بسوخته
\\
کی بینم از شعاع وصال تو آتشی
&&
راه دراز هجر ز پهنا بسوخته
\\
من چون سپند رقص کنان اندر او شده
&&
شعر تر و قصیده غرا بسوخته
\\
اندرفتاده برق به دکان عاشقان
&&
بازار و نقد و ناقد و کالا بسوخته
\\
زر گشته مس جسم ز اکسیر جان چنانک
&&
ز اکسیر مس‌ها را استا بسوخته
\\
ایمان و مؤمنان همه حیران شده ز عشق
&&
زنار پیر راهب ترسا بسوخته
\\
برقی ز شمس دین و ز تبریز آمده
&&
ابری که پرده گشت ز بالا بسوخته
\\
\end{longtable}
\end{center}
