\begin{center}
\section*{غزل شماره ۷۱۸: آخر گهر وفا ببارید}
\label{sec:0718}
\addcontentsline{toc}{section}{\nameref{sec:0718}}
\begin{longtable}{l p{0.5cm} r}
آخر گهر وفا ببارید
&&
آخر سر عاشقان بخارید
\\
ما خاک شما شدیم در خاک
&&
تخم ستم و جفا مکارید
\\
بر مظلومان راه هجران
&&
این ظلم دگر روا مدارید
\\
ای زهره ییان به بام این مه
&&
بر پرده زیر و بم بزارید
\\
یا نیز شما ز درد دوری
&&
همچون من خسته دلفکارید
\\
محروم نماند کس از این در
&&
ما را به کسی نمی‌شمارید
\\
آن درد که کوه از او چو ذرست
&&
بر ذرگکی چه می‌گمارید
\\
ای قوم که شیرگیر بودیت
&&
آن آهو را کنون شکارید
\\
زان نرگس مست شیرگیرش
&&
بی خمر وصال در خمارید
\\
زان دلبر گلعذار اکنون
&&
بس بی‌دل و زعفران عذارید
\\
با این همه گنج نیست بی‌رنج
&&
بر صبر و وفا قدم فشارید
\\
مردانه و مردرنگ باشید
&&
گر در ره عشق مرد کارید
\\
چون عاشق را هزار جانست
&&
بی صرفه و ترس جان سپارید
\\
جان کم ناید ز جان مترسید
&&
کاندر پی جان کامکارید
\\
عشقست حریف حیله آموز
&&
گرد از دغل و حیل برآرید
\\
در عشق حلال گشت حیله
&&
در عشق رهین صد قمارید
\\
حقست اگر ز عشق آن سرو
&&
با جمله گلرخان چو خارید
\\
حقست اگر ز عشق موسی
&&
بر فرعونان نفس مارید
\\
جان را سپر بلاش سازید
&&
کاندر کف عشق ذوالفقارید
\\
در صبر و ثبات کوه قافید
&&
چون کوه حلیم و باوقارید
\\
چون بحر نهان به مظهر آید
&&
ماننده موج بی‌قرارید
\\
هنگام نثار و درفشانی
&&
چون ابر به وقت نوبهارید
\\
در تیر شهیت اگر شهیدیت
&&
در پیش مهیت اگر غبارید
\\
پاینده و تازه همچو سروید
&&
چون شاخ بلند میوه دارید
\\
ز آسیب درخت او چو سیبید
&&
چون سیب درخت سنگسارید
\\
گر سنگ دلان زنندتان سنگ
&&
با گوهر خویش یار غارید
\\
چون دامن در پیش دوانید
&&
گر همچو سجاف بر کنارید
\\
چون همسفرید با مه خویش
&&
پیوسته چو چرخ در دوارید
\\
هم عشق شما و هم شما عشق
&&
با اشتر عشق هم مهارید
\\
گر نقب زنست نفس و دزدست
&&
آخر نه در این حصین حصارید
\\
از عشق خورید باده و نقل
&&
گر مقبل وگر حلال خوارید
\\
دیدیت که تان همی‌نگارد
&&
دیگر چه خیال می‌نگارید
\\
اوتان به خود اختیار کردست
&&
چه در پی جبر و اختیارید
\\
محکوم یک اختیار باشید
&&
گر عاشق و اهل اعتبارید
\\
خاموش کنم اگر چه با من
&&
در نطق و سکوت سازوارید
\\
\end{longtable}
\end{center}
