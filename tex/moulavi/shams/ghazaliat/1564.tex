\begin{center}
\section*{غزل شماره ۱۵۶۴: روزی که گذر کنی به گورم}
\label{sec:1564}
\addcontentsline{toc}{section}{\nameref{sec:1564}}
\begin{longtable}{l p{0.5cm} r}
روزی که گذر کنی به گورم
&&
یاد آور از این نفیر و شورم
\\
پرنور کن آن تک لحد را
&&
ای دیده و ای چراغ نورم
\\
تا از تو سجود شکر آرد
&&
اندر لحد این تن صبورم
\\
ای خرمن گل شتاب مگذار
&&
خوش کن نفسی بدان بخورم
\\
وان گاه که بگذری مینگار
&&
کز روزن و درگه تو دورم
\\
گر سنگ لحد ببست راهم
&&
از راه خیال بی‌فتورم
\\
گر صد کفنم بود ز اطلس
&&
بی‌خلعت صورت تو عورم
\\
از صحن سرای تو برآیم
&&
در نقب زنی مگر که مورم
\\
من مور توام تویی سلیمان
&&
یک دم مگذار بی‌حضورم
\\
خامش کردم بگو تو باقی
&&
کز گفت و شنود خود نفورم
\\
شمس تبریز دعوتم کن
&&
چون دعوت توست نفخ صورم
\\
\end{longtable}
\end{center}
