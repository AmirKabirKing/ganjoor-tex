\begin{center}
\section*{غزل شماره ۲۴۶۶: ای که به لطف و دلبری از دو جهان زیاده‌ای}
\label{sec:2466}
\addcontentsline{toc}{section}{\nameref{sec:2466}}
\begin{longtable}{l p{0.5cm} r}
ای که به لطف و دلبری از دو جهان زیاده‌ای
&&
ای که چو آفتاب و مه دست کرم گشاده‌ای
\\
صبح که آفتاب خود سر نزده‌ست از زمین
&&
جام جهان نمای را بر کف جان نهاده‌ای
\\
مهدی و مهتدی تویی رحمت ایزدی تویی
&&
روی زمین گرفته‌ای داد زمانه داده‌ای
\\
مایه صد ملامتی شورش صد قیامتی
&&
چشمه مشک دیده‌ای جوشش خنب باده‌ای
\\
سر نبرد هر آنک او سر کشد از هوای تو
&&
ز آنک به گردن همه بسته‌تر از قلاده‌ای
\\
خیز دلا و خلق را سوی صبوح بانگ زن
&&
گر چه ز دوش بیخودی بی‌سر و پا فتاده‌ای
\\
هر سحری خیال تو دارد میل سردهی
&&
دشمن عقل و دانشی فتنه مرد ساده‌ای
\\
همچو بهار ساقیی همچو بهشت باقیی
&&
همچو کباب قوتی همچو شراب شاده‌ای
\\
خیز دلا کشان کشان رو سوی بزم بی‌نشان
&&
عشق سواره‌ات کند گر چه چنین پیاده‌ای
\\
ذره به ذره ای جهان جانب تو نظرکنان
&&
گوهر آب و آتشی مونس نر و ماده‌ای
\\
این تن همچو غرقه را تا نکنی ز سر برون
&&
بند ردا و خرقه‌ای مرد سر سجاده‌ای
\\
باده خامشانه خور تا برهی ز گفت و گو
&&
یا حیوان ناطقی جمله ز نطق زاده‌ای
\\
لطف نمای ساقیا دست بگیر مست را
&&
جانب بزم خویش کش شاه طریق جاده‌ای
\\
\end{longtable}
\end{center}
