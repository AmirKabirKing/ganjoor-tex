\begin{center}
\section*{غزل شماره ۱۷۰۵: از ما مشو ملول که ما سخت شاهدیم}
\label{sec:1705}
\addcontentsline{toc}{section}{\nameref{sec:1705}}
\begin{longtable}{l p{0.5cm} r}
از ما مشو ملول که ما سخت شاهدیم
&&
از رشک و غیرت است که در چادری شدیم
\\
روزی که افکنیم ز جان چادر بدن
&&
بینی که رشک و حسرت ماهیم و فرقدیم
\\
رو را بشو و پاک شو از بهر دید ما
&&
ور نی تو دور باش که ما شاهد خودیم
\\
آن شاهدی نه‌ایم که فردا شود عجوز
&&
ما تا ابد جوان و دلارام و خوش قدیم
\\
آن چادر ار خلق شد شاهد کهن نشد
&&
فانی است عمر چادر و ما عمر بی‌حدیم
\\
چادر چو دید از آدم ابلیس کرد رد
&&
آدم نداش کرد تو ردی نه ما ردیم
\\
باقی فرشتگان به سجود اندرآمدند
&&
گفتند در سجود که بر شاهدی زدیم
\\
در زیر چادر است بتی کز صفات او
&&
ما را ز عقل برد و سجود اندرآمدیم
\\
اشکال گنده پیر ز اشکال شاهدان
&&
گر عقل ما نداند در عشق مرتدیم
\\
چه جای شاهد است که شیر خداست او
&&
طفلانه دم زدیم که با طفل ابجدیم
\\
با جوز و با مویز فریبند طفل را
&&
ور نی که ما چه لایق جوزیم و کنجدیم
\\
در خود و در زره چو نهان شد عجوزه‌ای
&&
گوید که رستم صف پیکار امجدیم
\\
از کر و فر او همه دانند کو زن است
&&
ما چون غلط کنیم که در نور احمدیم
\\
مؤمن ممیز است چنین گفت مصطفی
&&
اکنون دهان ببند که بی‌گفت مرشدیم
\\
بشنو ز شمس مفخر تبریز باقیش
&&
زیرا تمام قصه از آن شاه نستدیم
\\
\end{longtable}
\end{center}
