\begin{center}
\section*{غزل شماره ۲۲۵۶: به قرار تو او رسد که بود بی‌قرار تو}
\label{sec:2256}
\addcontentsline{toc}{section}{\nameref{sec:2256}}
\begin{longtable}{l p{0.5cm} r}
به قرار تو او رسد که بود بی‌قرار تو
&&
که به گلزار تو رسد دل خسته به خار تو
\\
گل و سوسن از آن تو همه گلشن از آن تو
&&
تلفش از خزان تو طربش از بهار تو
\\
ز زمین تا به آسمان همه گویان و خامشان
&&
چو دل و جان عاشقان به درون بی‌قرار تو
\\
همه سوداپرست تو همه عالم به دست تو
&&
نفسی پست و مست تو نفسی در خمار تو
\\
همه زیر و زبر ز تو همگان بی‌خبر ز تو
&&
چه غریب است نظر به تو چه خوش است انتظار تو
\\
چه کند سرو و باغ را چو نظر نیست زاغ را
&&
تو ز بلبل فغان شنو که وی است اختیار تو
\\
منم از کار مانده‌ای ز خریدار مانده‌ای
&&
به فراغت نظرکنان به سوی کار و بار تو
\\
بگذارم ز بحر و پل بگریزم ز جزو و کل
&&
چه کنم من عذار گل که ندارد عذار تو
\\
چه کنم عمر مرده را تن و جان فسرده را
&&
دو سه روز شمرده را چو منم در شمار تو
\\
چو دل و چشم و گوش‌ها ز تو نوشند نوش‌ها
&&
همه هر دم شکوفه‌ها شکفد در نثار تو
\\
پس از این جان که دارمش به خموشی سپارمش
&&
ز کجا خامشم هلد هوس جان سپار تو
\\
به خموشی نهان شدن چو شکارم نتان شدن
&&
که شکار و شکاریان نجهند از شکار تو
\\
همه فربه ز بوی تو همه لاغر ز هجر تو
&&
همه شادی و گریه شان اثر و یادگار تو
\\
\end{longtable}
\end{center}
