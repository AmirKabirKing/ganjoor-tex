\begin{center}
\section*{غزل شماره ۸۱۷: چون مرا جمعی خریدار آمدند}
\label{sec:0817}
\addcontentsline{toc}{section}{\nameref{sec:0817}}
\begin{longtable}{l p{0.5cm} r}
چون مرا جمعی خریدار آمدند
&&
کهنه دوزان جمله در کار آمدند
\\
از ستیزه ریش را صابون زدند
&&
وز حسد ناشسته رخسار آمدند
\\
همچو نغزان روز شیوه می‌کنند
&&
همچو چغزان شب به تکرار آمدند
\\
شکر کز آواز من این خفتگان
&&
خواب را هشتند و بیدار آمدند
\\
کاش بیداری برای حق بدی
&&
اینک بهر سیم و زر زار آمدند
\\
چون شود بیمار از ایشان سرخ رو
&&
چون به زردی همچو دینار آمدند
\\
خلق را پس چون رهانند از حسد
&&
کز حسد این قوم بیمار آمدند
\\
در دل خلقند چون دیده منیر
&&
آن شهان کز بهر دیدار آمدند
\\
همچو هفت استاره یک نور آمدند
&&
همچو پنج انگشت یک کار آمدند
\\
تا نگردی ریش گاو مردمی
&&
سر به سر خود ریش و دستار آمدند
\\
اهل دل خورشید و اهل گل غبار
&&
اهل دل گل اهل گل خار آمدند
\\
غم مخور ای میر عالم زین گروه
&&
کاهل دل دل بخش و دلدار آمدند
\\
\end{longtable}
\end{center}
