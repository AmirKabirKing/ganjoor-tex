\begin{center}
\section*{غزل شماره ۱۹۸۵: صنما بیار باده بنشان خمار مستان}
\label{sec:1985}
\addcontentsline{toc}{section}{\nameref{sec:1985}}
\begin{longtable}{l p{0.5cm} r}
صنما بیار باده بنشان خمار مستان
&&
که ببرد عشق رویت همگی قرار مستان
\\
می کهنه را کشان کن به صبوح گلستان کن
&&
که به جوش اندرآمد فلک از عقار مستان
\\
بده آن قرار جان را گل و لاله زار جان را
&&
ز نبات و قند پر کن دهن و کنار مستان
\\
قدحی به دست برنه به کف شکرلبان ده
&&
بنشان به آب رحمت به کرم غبار مستان
\\
صنما به چشم مستت دل و جان غلام دستت
&&
به می خوشی که هستت ببر اختیار مستان
\\
چو شراب لاله رنگت به دماغ‌ها برآید
&&
گل سرخ شرم دارد ز رخ و عذار مستان
\\
چو جناح و قلب مجلس ز شراب یافت مونس
&&
ببرد گلوی غم را سر ذوالفقار مستان
\\
صنما تو روز مایی غم و غصه سوز مایی
&&
ز تو است ای معلا همه کار و بار مستان
\\
بکشان تو گوش شیران چو شتر قطارشان کن
&&
که تو شیرگیر حقی به کفت مهار مستان
\\
ز عقیق جام داری نمکی تمام داری
&&
چه غریب دام داری جهت شکار مستان
\\
سخنی بماند جانی که تو بی‌بیان بدانی
&&
که تو رشک ساقیانی سر و افتخار مستان
\\
\end{longtable}
\end{center}
