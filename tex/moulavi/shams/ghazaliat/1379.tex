\begin{center}
\section*{غزل شماره ۱۳۷۹: آمد خیال خوش که من از گلشن یار آمدم}
\label{sec:1379}
\addcontentsline{toc}{section}{\nameref{sec:1379}}
\begin{longtable}{l p{0.5cm} r}
آمد خیال خوش که من از گلشن یار آمدم
&&
در چشم مست من نگر کز کوی خمار آمدم
\\
سرمایه مستی منم هم دایه هستی منم
&&
بالا منم پستی منم چون چرخ دوار آمدم
\\
آنم کز آغاز آمدم با روح دمساز آمدم
&&
برگشتم و بازآمدم بر نقطه پرگار آمدم
\\
گفتم بیا شاد آمدی دادم بده داد آمدی
&&
گفتا بدید و داد من کز بهر این کار آمدم
\\
هم من مه و مهتاب تو هم گلشن و هم آب تو
&&
چندین ره از اشتاب تو بی‌کفش و دستار آمدم
\\
فرخنده نامی ای پسر گر چه که خامی ای پسر
&&
تلخی مکن زیرا که من از لطف بسیار آمدم
\\
خندان درآ تلخی بکش شاباش ای تلخی خوش
&&
گل‌ها دهم گر چه که من اول همه خار آمدم
\\
گل سر برون کرد از درج کالصبر مفتاح الفرج
&&
هر شاخ گوید لاحرج کز صبر دربار آمدم
\\
\end{longtable}
\end{center}
