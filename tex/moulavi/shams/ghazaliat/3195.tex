\begin{center}
\section*{غزل شماره ۳۱۹۵: درهم شکن چو شیشه خود را، چو مست جامی}
\label{sec:3195}
\addcontentsline{toc}{section}{\nameref{sec:3195}}
\begin{longtable}{l p{0.5cm} r}
درهم شکن چو شیشه خود را، چو مست جامی
&&
بد نام عشق جان شو، اینست نیکنامی
\\
پرذوق، چون صراحی بنشین، اگر نشینی
&&
کن کالقدح مذیقا للقوم فی‌القیام
\\
عقل تو پای‌بندی، عشق تو سربلندی
&&
العقل فی‌الملام والعشق فی‌المدام
\\
الدیک فی صیاح، واللیل فی انهزام
&&
والصبح قد تبدی فی مهجةالضلام
\\
معشوق غیر ما، نی، جز که خون ما، نی
&&
هم جان کند رئیسی، هم جان کند غلامی
\\
دل را کباب کردی، خون را شراب کردی
&&
یا من فداک روحی یا سیدالانام
\\
ز اندیشه شو پیاده، تا بر خوری ز باده
&&
من راوق قدیم، مستکمل‌القوام
\\
مستفعلن فعولن، آتش مکن مجوشان
&&
زیرا کمال آمد، دیگر نماند خامی
\\
می‌گو تو هرچه خواهی، فرمان‌روا و شاهی
&&
سلمت یا عزیزی، یا صاحب‌السلام
\\
باده چو با خیزان، چون پشه غم‌گریزان
&&
لا تعذلوا السکارا افدیکم کرامی
\\
تبریز شاد بادا، ز اشرق شمس دینم
&&
فالشمس حیث تجری للمشرقین حامی
\\
\end{longtable}
\end{center}
