\begin{center}
\section*{غزل شماره ۱۴۶۷: این شکل که من دارم ای خواجه که را مانم}
\label{sec:1467}
\addcontentsline{toc}{section}{\nameref{sec:1467}}
\begin{longtable}{l p{0.5cm} r}
این شکل که من دارم ای خواجه که را مانم
&&
یک لحظه پری شکلم یک لحظه پری خوانم
\\
در آتش مشتاقی هم جمعم و هم شمعم
&&
هم دودم و هم نورم هم جمع و پریشانم
\\
جز گوش رباب دل از خشم نمالم من
&&
جز چنگ سعادت را از زخمه نرنجانم
\\
چون شکر و چون شیرم با خود زنم و گیرم
&&
طبعم چو جنون آرد زنجیر بجنبانم
\\
ای خواجه چه مرغم من نی کبکم و نی بازم
&&
نی خوبم و نی زشتم نی اینم و نی آنم
\\
نی خواجه بازارم نی بلبل گلزارم
&&
ای خواجه تو نامم نه تا خویش بدان خوانم
\\
نی بنده نی آزادم نی موم نه پولادم
&&
نی دل به کسی دادم نی دلبر ایشانم
\\
گر در شرم و خیرم از خود نه‌ام از غیرم
&&
آن سو که کشد آن کس ناچار چنان رانم
\\
\end{longtable}
\end{center}
