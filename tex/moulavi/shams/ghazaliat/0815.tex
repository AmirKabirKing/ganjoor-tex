\begin{center}
\section*{غزل شماره ۸۱۵: شهر پر شد لولیان عقل دزد}
\label{sec:0815}
\addcontentsline{toc}{section}{\nameref{sec:0815}}
\begin{longtable}{l p{0.5cm} r}
شهر پر شد لولیان عقل دزد
&&
هم بدزدد هم بخواهد دستمزد
\\
هر که بتواند نگه دارد خرد
&&
من نتانستم مرا باری ببرد
\\
گرد من می‌گشت یک لولی پریر
&&
همچنینم برد کلی کرد و مرد
\\
کرد لولی دست خود در خون من
&&
خون من در دست آن لولی فسرد
\\
تا که می‌شد خون من انگوروار
&&
سال‌ها انگور دل را می‌فشرد
\\
کرد دیدم کو کند دزدی ولیک
&&
کرد ما را بین که او دزدید کرد
\\
کی گمان دارد که او دزدی کند
&&
خاصه شه صوفی شد آمد مو سترد
\\
دزد خونی بین که هر کس را که کشت
&&
خضر و الیاسی شد و هرگز نمرد
\\
رخت برد و بخت داد آنگه چه بخت
&&
سیم برد و دامن پرزر شمرد
\\
دردها و دردها را صاف کرد
&&
پیش او آرید هر جا هست درد
\\
این جهان چشمست و او چون مردمک
&&
تنگ می‌آید جهان زین مرد خرد
\\
باز رشک حق دهانم قفل کرد
&&
شد کلید و قفل را جایی سپرد
\\
\end{longtable}
\end{center}
