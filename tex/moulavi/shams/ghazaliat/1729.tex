\begin{center}
\section*{غزل شماره ۱۷۲۹: اگر چه شرط نهادیم و امتحان کردیم}
\label{sec:1729}
\addcontentsline{toc}{section}{\nameref{sec:1729}}
\begin{longtable}{l p{0.5cm} r}
اگر چه شرط نهادیم و امتحان کردیم
&&
ز شرط‌ها بگذشتیم و رایگان کردیم
\\
اگر چه یک طرف از آسمان زمینی شد
&&
نه پاره پاره زمین را هم آسمان کردیم
\\
اگر چه بام بلندست آسمان مگریز
&&
چه غم خوری ز بلندی چو نردبان کردیم
\\
پرت دهیم که چون تیر بر فلک بپری
&&
اگر ز غم تن بیچاره را کمان کردیم
\\
اگر چه جان مدد جسم شد کثیفی یافت
&&
لطافتش بنمودیم و باز جان کردیم
\\
اگر تو دیوی ما دیو را فرشته کنیم
&&
وگر تو گرگی ما گرگ را شبان کردیم
\\
تو ماهیی که به بحر عسل بخواهی تاخت
&&
هزار بارت از آن شهد در دهان کردیم
\\
اگر چه مرغ ضعیفی بجوی شاخ بلند
&&
بر این درخت سعادت که آشیان کردیم
\\
بگیر ملک دو عالم که مالک الملکیم
&&
بیا به بزم که شمشیر در میان کردیم
\\
هزار ذره از این قطب آفتابی یافت
&&
بسا قراضه قلبی که ماش کان کردیم
\\
بسا یخی بفسرده کز آفتاب کرم
&&
فسردگیش ببردیم و خوش روان کردیم
\\
گر آب روح مکدر شد اندر این گرداب
&&
ز سیل‌ها و مددهاش خوش عنان کردیم
\\
چرا شکفته نباشی چو برگ می لرزی
&&
چه ناامیدی از ما که را زیان کردیم
\\
بسا دلی که چو برگ درخت می لرزید
&&
به آخرش بگزیدیم و باغبان کردیم
\\
الست گفتیم از غیب و تو بلی گفتی
&&
چه شد بلی تو چون غیب را عیان کردیم
\\
پنیر صدق بگیر و به باغ روح بیا
&&
که ما بلی تو را باغ و بوستان کردیم
\\
خموش باش که تا سر به سر زبان گردی
&&
زبان نبود زبان تو ما زبان کردیم
\\
\end{longtable}
\end{center}
