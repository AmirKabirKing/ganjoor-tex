\begin{center}
\section*{غزل شماره ۲۱۶۰: ندیدم در جهان کس را که تا سر پر نبوده‌ست او}
\label{sec:2160}
\addcontentsline{toc}{section}{\nameref{sec:2160}}
\begin{longtable}{l p{0.5cm} r}
ندیدم در جهان کس را که تا سر پر نبوده‌ست او
&&
همه جوشان و پرآتش کمین اندر بهانه جو
\\
همه از عشق بررسته جگرها خسته لب بسته
&&
ولی در گلشن جانشان شقایق‌های تو بر تو
\\
حقایق‌های نیک و بد به شیر خفته می‌ماند
&&
که عالم را زند برهم چو دستی برنهی بر او
\\
بسی خورشید افلاکی نهان در جسم هر خاکی
&&
بسی شیران غرنده نهان در صورت آهو
\\
به مثل خلقت مردم نزاد از خاک و از انجم
&&
وگر چه زاد بس نادر از این داماد و کدبانو
\\
ضمیرت بس محل دارد قدم فوق زحل دارد
&&
اگر چه اندر آب و گل فروشد پاش تا زانو
\\
روان گشته‌ست از بالا زلال لطف تا این جا
&&
که ای جان گل آلوده از این گل خویش را واشو
\\
نمی‌بینی تو این زمزم فروتر می‌روی هر دم
&&
اگر ایوبی و محرم به زیر پای جو دارو
\\
چو شستن گیرد او خود را رباید آب جو او را
&&
چو سیبش می‌برد غلطان به باغ خرم بی‌سو
\\
به سیبستان رسد سیبش رهد از سنگ آسیبش
&&
نبیند اندر آن گلشن به جز آسیب شفتالو
\\
دل ویس و دل رامین ببیند جنت وحدت
&&
گل سرخ و گل خیری نشیند مست رو با رو
\\
از آن سو در کف حوری شراب صاف انگوری
&&
از این سو کرده رو بانو به خنده سوی روبانو
\\
در آن باغ خوش اعلوفه سپی پوشان چو اشکوفه
&&
که رستیم از سیه کاری ز مازو رفت آن ما زو
\\
بصیرت‌ها گشاده هر نظر حیران در آن منظر
&&
دهان پرقند و پرشکر تو خود باقیش را برگو
\\
\end{longtable}
\end{center}
