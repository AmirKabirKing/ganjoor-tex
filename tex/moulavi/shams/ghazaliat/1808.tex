\begin{center}
\section*{غزل شماره ۱۸۰۸: با آنک از پیوستگی من عشق گشتم عشق من}
\label{sec:1808}
\addcontentsline{toc}{section}{\nameref{sec:1808}}
\begin{longtable}{l p{0.5cm} r}
با آنک از پیوستگی من عشق گشتم عشق من
&&
بیگانه می باشم چنین با عشق از دست فتن
\\
از غایت پیوستگی بیگانه باشد کس بلی
&&
این مشکلات ار حل شود دشمن نماند در زمن
\\
بحری است از ما دور نی ظاهر نه و مستور نی
&&
هم دم زدن دستور نی هم کفر از او خامش شدن
\\
گفتن از او تشبیه شد خاموشیت تعطیل شد
&&
این درد بی‌درمان بود فرج لنا یا ذا المنن
\\
نقش جهان رنگ و بو هر دم مدد خواهد از او
&&
هم بی‌خبر هم لقمه جو چون طفل بگشاده دهن
\\
خفته‌ست و برجسته‌ست دل در جوش پیوسته‌ست دل
&&
چون دیگ سربسته‌ست دل در آتشش کرده وطن
\\
ای داده خاموشانه‌ای ما را تو از پیمانه‌ای
&&
هر لحظه نوافسانه‌ای در خامشی شد نعره زن
\\
در قهر او صد مرحمت در بخل او صد مکرمت
&&
در جهل او صد معرفت در خامشی گویا چو ظن
\\
الفاظ خاموشان تو بشنوده بی‌هوشان تو
&&
خاموشم و جوشان تو مانند دریای عدن
\\
لطفت خدایی می کند حاجت روایی می کند
&&
وان کو جدایی می کند یا رب تو از بیخش بکن
\\
ای خوشدلی و ناز ما ای اصل و ای آغاز ما
&&
آخر چه داند راز ما عقل حسن یا بوالحسن
\\
ای عشق تو بخریده ما وز غیر تو ببریده ما
&&
ای جامه‌ها بدریده ما بر چاک ما بخیه مزن
\\
ای خون عقلم ریخته صبر از دلم بگریخته
&&
ای جان من آمیخته با جان هر صورت شکن
\\
آن جا که شد عاشق تلف مرغی نپرد آن طرف
&&
ور مرده یابد زان علف بیخود بدراند کفن
\\
\end{longtable}
\end{center}
