\begin{center}
\section*{غزل شماره ۲۹۸۹: رویش ندیده پس مکنیدم ملامتی}
\label{sec:2989}
\addcontentsline{toc}{section}{\nameref{sec:2989}}
\begin{longtable}{l p{0.5cm} r}
رویش ندیده پس مکنیدم ملامتی
&&
نادیده حکم کردن باشد غرامتی
\\
پروانه چون نسوزد چون شمع او بود
&&
چون خم نیاورم ز چنان سروقامتی
\\
آن مه اگر برآید در روز رستخیز
&&
برخیزد از میان قیامت قیامتی
\\
زان رو که زهره نیست فلک را که دم زند
&&
در خود همی‌بسوزد دارد علامتی
\\
گر حسن حسن او است کجا عافیت کجا
&&
با غمزه‌های آتش او کو سلامتی
\\
هر دم دلم به عشق وی اندر حریصتر
&&
هر دم ز عشق او دل من با سمتی
\\
یا هجر لم تقل لی بالله ربنا
&&
هذا الصدود منک علینا الی متی
\\
می‌ترسم از فراق دراز تو سنگ دل
&&
تا نشکند سبوی امیدم ز آفتی
\\
ای آنک جبرئیل ز تو راه گم کند
&&
با صبر تو ندارد این چرخ طاقتی
\\
دل را ببرد عشق که تا سود دل کند
&&
حاشا که او کند طمعی یا تجارتی
\\
عشق آن توانگری است که از بس توانگری
&&
داردهمی ز ریش فراغت فراغتی
\\
از من مپرس این و ز عقل کمال پرس
&&
کو راست در عیار گهرها مهارتی
\\
او نیز خود چه گوید لیکن به قدر خویش
&&
کو در قدم بود حدثی نوطهارتی
\\
عقل از امید وصل چو مجنون روان شود
&&
در عشق می‌رود به امید زیارتی
\\
ور ز آنک درنیابد در ره کمال عشق
&&
از پرتو شرارش یابد حرارتی
\\
بادا ز نور عشق من و عقل کل را
&&
زان شکر شگرف شفای مرارتی
\\
تا طعم آن حلاوت بر عاشقان زند
&&
وز عاشقان برآید مستانه حالتی
\\
تبریز شمس دین که بصیرت از او بود
&&
چون بر دلم رسید سپاهش به غارتی
\\
\end{longtable}
\end{center}
