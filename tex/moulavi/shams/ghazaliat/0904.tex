\begin{center}
\section*{غزل شماره ۹۰۴: ز باد حضرت قدسی بنفشه زار چه می‌شد}
\label{sec:0904}
\addcontentsline{toc}{section}{\nameref{sec:0904}}
\begin{longtable}{l p{0.5cm} r}
ز باد حضرت قدسی بنفشه زار چه می‌شد
&&
درخت‌های حقایق از آن بهار چه می‌شد
\\
دل از دیار خلایق بشد به شهر حقایق
&&
خدای داند کاین دل در آن دیار چه می‌شد
\\
ز های و هوی حریفان ز نای و نوش ظریفان
&&
هوای نور صبوح و شراب نار چه می‌شد
\\
هزار بلبل مست و هزار عاشق بی‌دل
&&
در آن مقام تحیر ز روی یار چه می‌شد
\\
چو عشق در بر سیمین کشید عاشق خود را
&&
ز بوسه‌های چو شکر در آن کنار چه می‌شد
\\
در آن طرف که ز مستی تو گل ز خار ندانی
&&
عجب که گل چه چشید و عجب که خار چه می‌شد
\\
میان خلعت جانان قبول عشق خرامان
&&
به بارگاه تجلی ز کار و بار چه می‌شد
\\
به باد و آتش و آب و به خاک عشق درآمد
&&
به نور یک نظر عشق هر چهار چه می‌شد
\\
چو شمس مفخر تبریز زد آتشی به درختی
&&
ز شعله‌های لطیفش درخت و بار چه می‌شد
\\
\end{longtable}
\end{center}
