\begin{center}
\section*{غزل شماره ۲۳۱۷: ای خاک کف پایت رشک فلکی بوده}
\label{sec:2317}
\addcontentsline{toc}{section}{\nameref{sec:2317}}
\begin{longtable}{l p{0.5cm} r}
ای خاک کف پایت رشک فلکی بوده
&&
جان من و جان تو در اصل یکی بوده
\\
در خانه نقشینی دیدم صنم چینی
&&
خون خواره صد آدم جان ملکی بوده
\\
صد ماه یقینم شد اندر دل شب پنهان
&&
صد نور یقین دیدم مشتاق شکی بوده
\\
گفتم به ایاز ای حر محمود شدی آخر
&&
در شاه چه جا کردی ای آیبکی بوده
\\
ای سگ که ز اصحابی در کهف تو در خوابی
&&
چون شیر خدا گشتی اول سگکی بوده
\\
ای ماهی در آتش تو جانب دریا کش
&&
ای پیشتر از عالم در وی سمکی بوده
\\
شمس الحق تبریزم همرنگ تو می‌خیزم
&&
من مرده تو گرد من بحر نمکی بوده
\\
\end{longtable}
\end{center}
