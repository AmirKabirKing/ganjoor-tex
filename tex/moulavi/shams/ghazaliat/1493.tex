\begin{center}
\section*{غزل شماره ۱۴۹۳: ما عاشق و سرگشته و شیدای دمشقیم}
\label{sec:1493}
\addcontentsline{toc}{section}{\nameref{sec:1493}}
\begin{longtable}{l p{0.5cm} r}
ما عاشق و سرگشته و شیدای دمشقیم
&&
جان داده و دل بسته سودای دمشقیم
\\
زان صبح سعادت که بتابید از آن سو
&&
هر شام و سحر مست سحرهای دمشقیم
\\
بر باب بریدیم که از یار بریدیم
&&
زان جامع عشاق به خضرای دمشقیم
\\
از چشمه بونواس مگر آب نخوردی
&&
ما عاشق آن ساعد سقای دمشقیم
\\
بر مصحف عثمان بنهم دست به سوگند
&&
کز لولوی آن دلبر لالای دمشقیم
\\
از باب فرج دوری و از باب فرادیس
&&
کی داند کاندر چه تماشای دمشقیم
\\
بر ربوه برآییم چو در مهد مسیحیم
&&
چون راهب سرمست ز حمرای دمشقیم
\\
در نیرب شاهانه بدیدیم درختی
&&
در سایه آن شسته و دروای دمشقیم
\\
اخضر شده میدان و بغلطیم چو گویی
&&
از زلف چو چوگان که به صحرای دمشقیم
\\
کی بی‌مزه مانیم چو در مزه درآییم
&&
دروازه شرقی سویدای دمشقیم
\\
اندر جبل صالح کانی است ز گوهر
&&
زان گوهر ما غرقه دریای دمشقیم
\\
چون جنت دنیاست دمشق از پی دیدار
&&
ما منتظر رأیت حسنای دمشقیم
\\
از روم بتازیم سوم بار سوی شام
&&
کز طره چون شام مطرای دمشقیم
\\
مخدومی شمس الحق تبریز گر آن جاست
&&
مولای دمشقیم و چه مولای دمشقیم
\\
\end{longtable}
\end{center}
