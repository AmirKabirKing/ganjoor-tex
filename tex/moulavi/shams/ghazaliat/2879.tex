\begin{center}
\section*{غزل شماره ۲۸۷۹: ننگ هر قافله در شش دره ابلیسی}
\label{sec:2879}
\addcontentsline{toc}{section}{\nameref{sec:2879}}
\begin{longtable}{l p{0.5cm} r}
ننگ هر قافله در شش دره ابلیسی
&&
تو به هر نیت خود مسخره ابلیسی
\\
از برای علف دیو تو قربان تنی
&&
بز دیوی تو مگر یا بره ابلیسی
\\
سره مردا چه پشیمان شده‌ای گردن نه
&&
که در این خوردن سیلی سره ابلیسی
\\
شلغم پخته تو امید ببر زان تره زار
&&
ز آنک در خدمت نان چون تره ابلیسی
\\
نان ببینی تو و حیزانه درافتی در رو
&&
عاشق نطفه دیو و نره ابلیسی
\\
نیت روزه کنی توبره گوید کای خر
&&
سر فروکن خر باتوبره ابلیسی
\\
از حقیقت خبرت نیست که چون خواهد بود
&&
تو بدان علم و هنر قوصره ابلیسی
\\
در غم فربهی گوشت تو لاغر گشتی
&&
ناله برداشته چون حنجره ابلیسی
\\
کفر و ایمان چه می‌خور چو سگان قی می‌کن
&&
ز آنک تو مؤمنه و کافره ابلیسی
\\
تا دم مرگ و دم غرغره چون سرکه بد
&&
ترش و گنده تو چون غرغره ابلیسی
\\
گرد آن دایره گرده و خوان پر چو مگس
&&
تا قیامت تو که از دایره ابلیسی
\\
\end{longtable}
\end{center}
