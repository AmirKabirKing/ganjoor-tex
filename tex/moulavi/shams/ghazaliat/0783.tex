\begin{center}
\section*{غزل شماره ۷۸۳: ای دریغا که حریفان همه سر بنهادند}
\label{sec:0783}
\addcontentsline{toc}{section}{\nameref{sec:0783}}
\begin{longtable}{l p{0.5cm} r}
ای دریغا که حریفان همه سر بنهادند
&&
باده عشق عمل کرد و همه افتادند
\\
همه را از تبش عشق قبا تنگ آمد
&&
کله از سر بنهادند و کمر بگشادند
\\
این همه عربده و تندی و ناسازی چیست
&&
نه همه همره و هم قافله و هم زادند
\\
ساقیا دست من و دامن تو مخمورم
&&
تو بده داد دل من دگران بیدادند
\\
من عمارت نپذیرم که خرابم کردی
&&
ای خراب از می تو هر کی در این بنیادند
\\
ای خدا رحم کن آن را که مرا رحم نکرد
&&
به صفات تو که در کشتن من استادند
\\
بیخودم کن که از آن حالتم آزادیهاست
&&
بنده آن نفرم کز خود خود آزادند
\\
دختران دارم چون ماه پس پرده دل
&&
ماه رویان سماوات مرا دامادند
\\
دخترانم چو شکر سرتاسر شیرینند
&&
خسروان فلک اندر پیشان فرهادند
\\
چون همه باز نظر از جز شه دوخته‌اند
&&
گرد مردار نگردند نه ایشان خادند
\\
همه لب بر لب معشوق چو نی نالانند
&&
دل ندارند و عجب این که همه دلشادند
\\
گر فقیرند همه شیردل و زربخش اند
&&
این فقیران تراشنده همه خرادند
\\
خود از آن کس که تراشیده تو را زو بتراش
&&
دگران حیله گر و ظالم و بی‌فریادند
\\
رو ترش کرده چرایی که خریدارم نیست
&&
عاشقانند تو را منتظر میعادند
\\
تن زدم لیک دلم نعره زنان می‌گوید
&&
باده عشق تو خواهم که دگرها بادند
\\
شمس تبریز به نور تو که ذرات وجود
&&
همه در عشق تو موم‌اند اگر پولادند
\\
\end{longtable}
\end{center}
