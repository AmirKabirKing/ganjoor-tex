\begin{center}
\section*{غزل شماره ۳۰۵۴: منم که کار ندارم به غیر بی‌کاری}
\label{sec:3054}
\addcontentsline{toc}{section}{\nameref{sec:3054}}
\begin{longtable}{l p{0.5cm} r}
منم که کار ندارم به غیر بی‌کاری
&&
دلم ز کار زمانه گرفت بیزاری
\\
ز خاک تیره ندیدم به غیر تاریکی
&&
ز پیر چرخ ندیدم به غیر مکاری
\\
فروگذاشته‌ای شست دل در این دریا
&&
نه ماهیی بگرفتی نه دست می‌داری
\\
تو را چه شصت و چه هفتاد چون نخواهی پخت
&&
گلی به دست نداری چه خار می‌خاری
\\
کلاه کژ بنهی همچو ماه و نورت نیست
&&
برو برو که گرفتار ریش و دستاری
\\
چگونه برقی آخر که کشت می‌سوزی
&&
چگونه ابری آخر که سنگ می‌باری
\\
چو صید دام خودی پس چگونه صیادی
&&
چو دزد خانه خویشی چگونه عیاری
\\
اگر چه این همه باشد ولی اگر روزی
&&
خیال یار مرا دیده‌ای نکو یاری
\\
به ذات پاک خدایی که کارساز همه‌ست
&&
چو مست کار امیر منی نکوکاری
\\
اگر دو گام پیاده دویدی از پی او
&&
تو یک سواره نه‌ای تو سپاه سالاری
\\
بگیر دامن عشقی که دامنش گرمست
&&
که غیر او نرهاند تو را ز اغیاری
\\
به یاد عشق شب تیره را به روز آور
&&
چو عشق یاد بود شب کجا بود تاری
\\
تو خفته باشی و آن عشق بر سر بالین
&&
برآوریده دو کف در دعا و در زاری
\\
اگر بگویم باقی بسوزد این عالم
&&
هلا قناعت کردم بس است گفتاری
\\
\end{longtable}
\end{center}
