\begin{center}
\section*{غزل شماره ۲۳۸۹: در خانه دل ای جان آن کیست ایستاده}
\label{sec:2389}
\addcontentsline{toc}{section}{\nameref{sec:2389}}
\begin{longtable}{l p{0.5cm} r}
در خانه دل ای جان آن کیست ایستاده
&&
بر تخت شه کی باشد جز شاه و شاه زاده
\\
کرده به دست اشارت کز من بگو چه خواهی
&&
مخمور می چه خواهد جز نقل و جام و باده
\\
نقلی ز دل معلق جامی ز نور مطلق
&&
در خلوت هوالحق بزم ابد نهاده
\\
ای بس دغل فروشان در بزم باده نوشان
&&
هش دار تا نیفتی ای مرد نرم و ساده
\\
در حلقه قلاشی زنهار تا نباشی
&&
چون غنچه چشم بسته چون گل دهان گشاده
\\
چون آینه است عالم نقش کمال عشق است
&&
ای مردمان کی دیده است جزوی ز کل زیاده
\\
چون سبزه شو پیاده زیرا در این گلستان
&&
دلبر چو گل سوار است باقی همه پیاده
\\
هم تیغ و هم کشنده هم کشته هم کشنده
&&
هم جمله عقل گشته هم عقل باده داده
\\
آن شه صلاح دین است کو پایدار بادا
&&
دست عطاش دایم در گردنم قلاده
\\
\end{longtable}
\end{center}
