\begin{center}
\section*{غزل شماره ۱۴۶۴: سر برمزن از هستی تا راه نگردد گم}
\label{sec:1464}
\addcontentsline{toc}{section}{\nameref{sec:1464}}
\begin{longtable}{l p{0.5cm} r}
سر برمزن از هستی تا راه نگردد گم
&&
در بادیه مردان محوست تو را جم جم
\\
در عالم پرآتش در محو سر اندرکش
&&
در عالم هستی بین نیلین سر چون قاقم
\\
زیر فلک ناری در حلقه بیداری
&&
هر چند که سر داری نه سر هلدت نی دم
\\
هر رنج که دیده‌ست او در رنج شدیدست او
&&
محو است که عید است او باقی دهل و لم لم
\\
سرگشتگی حالم تو فهم کن از قالم
&&
کای هیزم از آن آتش برخوان که و ان منکم
\\
کی روید از این صحرا جز لقمه پرصفرا
&&
کی تازد بر بالا این مرکب پشمین سم
\\
ور پرد چون کرکس خاکش بکشد واپس
&&
هر چیز به اصل خود بازآید می دانم
\\
رو آر گر انسانی در جوهر پنهانی
&&
کو آب حیات آمد در قالب همچون خم
\\
شمس الحق تبریزی ما بیضه مرغ تو
&&
در زیر پرت جوشان تا آید وقت قم
\\
\end{longtable}
\end{center}
