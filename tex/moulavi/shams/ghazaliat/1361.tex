\begin{center}
\section*{غزل شماره ۱۳۶۱: عمرک یا واحدا فی درجات الکمال}
\label{sec:1361}
\addcontentsline{toc}{section}{\nameref{sec:1361}}
\begin{longtable}{l p{0.5cm} r}
عمرک یا واحدا فی درجات الکمال
&&
قد نزل الهم بی یا سندی قم تعال
\\
چند از این قیل و قال عشق پرست و ببال
&&
تا تو بمانی چو عشق در دو جهان بی‌زوال
\\
یا فرجی مونسی یا قمر المجلس
&&
وجهک بدر تمام ریقک خمر حلال
\\
چند کشی بار هجر غصه و تیمار هجر
&&
خاصه که منقار هجر کند تو را پر و بال
\\
روحک بحر الوفا لونک لمع الصفا
&&
عمرک لو لا التقی قلت ایا ذا الجلال
\\
آه ز نفس فضول آه ز ضعف عقول
&&
آه ز یار ملول چند نماید ملال
\\
تطرب قلب الوری تسکرهم بالهوی
&&
تدرک ما لا یری انت لطیف الخیال
\\
آنک همی‌خوانمش عجز نمی‌دانمش
&&
تا که بترسانمش از ستم و از وبال
\\
تدخل ارواحهم تسکر اشباحهم
&&
تجلسهم مجلسا فیه کؤوس ثقال
\\
جمله سؤال و جواب زوست و منم چون رباب
&&
می زندم او شتاب زخمه که یعنی بنال
\\
تصلح میزاننا تحسن الحاننا
&&
تذهب احزاننا انت شدید المحال
\\
یک دم آواز مات یک دم بانگ نجات
&&
می زند آن خوش صفات بر من و بر وصف حال
\\
\end{longtable}
\end{center}
