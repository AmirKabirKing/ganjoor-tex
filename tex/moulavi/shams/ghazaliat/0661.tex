\begin{center}
\section*{غزل شماره ۶۶۱: بیا ای زیرک و بر گول می‌خند}
\label{sec:0661}
\addcontentsline{toc}{section}{\nameref{sec:0661}}
\begin{longtable}{l p{0.5cm} r}
بیا ای زیرک و بر گول می‌خند
&&
بیا ای راه دان بر غول می‌خند
\\
چو در سلطان بی‌علت رسیدی
&&
هلا بر علت و معلول می‌خند
\\
اگر بر نفس نحسی دیو شد چیر
&&
برو بر خاذل و مخذول می‌خند
\\
چو مرده مرده‌ای را کرد معزول
&&
تو خوش بر عازل و معزول می‌خند
\\
مثال محتلم پندار عزلش
&&
تو هم بر فاعل و مفعول می‌خند
\\
یکی در خواب حاصل کرد ملکی
&&
برو بر حاصل و محصول می‌خند
\\
سؤالی گفت کوری پیش کری
&&
دلا بر سائل و مسول می‌خند
\\
وگر گوید فروشستم فلان را
&&
هلا بر غاسل و مغسول می‌خند
\\
چو نقدت دست داد از نقل بس کن
&&
خمش بر ناقل و منقول می‌خند
\\
\end{longtable}
\end{center}
