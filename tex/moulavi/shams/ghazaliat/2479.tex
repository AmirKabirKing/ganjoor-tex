\begin{center}
\section*{غزل شماره ۲۴۷۹: پیش از آنک از عدم کرد وجودها سری}
\label{sec:2479}
\addcontentsline{toc}{section}{\nameref{sec:2479}}
\begin{longtable}{l p{0.5cm} r}
پیش از آنک از عدم کرد وجودها سری
&&
بی ز وجود وز عدم باز شدم یکی دری
\\
بی‌مه و سال سال‌ها روح زده‌ست بال‌ها
&&
نقطه روح لم یزل پاک روی قلندری
\\
آتش عشق لامکان سوخته پاک جسم و جان
&&
گوهر فقر در میان بر مثل سمندری
\\
خود خورد و فزون شود آنک ز خود برون شود
&&
سیمبری که خون شود از بر خود خورد بری
\\
کوره دل درآ ببین زان سوی کافری و دین
&&
زر شده جان عاشقان عشق دکان زرگری
\\
چهره فقر را فدا فقر منزه از ردا
&&
کز رخ فقر نور شد جمله ز عرش تا ثری
\\
مست ز جام شمس دین میکده الست بین
&&
صد تبریز را ضمین از غم آب و آذری
\\
\end{longtable}
\end{center}
