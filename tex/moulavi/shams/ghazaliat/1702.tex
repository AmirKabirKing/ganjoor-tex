\begin{center}
\section*{غزل شماره ۱۷۰۲: درده شراب یک سان تا جمله جمع باشیم}
\label{sec:1702}
\addcontentsline{toc}{section}{\nameref{sec:1702}}
\begin{longtable}{l p{0.5cm} r}
درده شراب یک سان تا جمله جمع باشیم
&&
تا نقش‌های خود را یک یک فروتراشیم
\\
از خویش خواب گردیم همرنگ آب گردیم
&&
ما شاخ یک درختیم ما جمله خواجه تاشیم
\\
ما طبع عشق داریم پنهان آشکاریم
&&
در شهر عشق پنهان در کوی عشق فاشیم
\\
خود را چو مرده بینیم بر گور خود نشینیم
&&
خود را چو زنده بینیم در نوحه رو خراشیم
\\
هر صورتی که روید بر آینه دل ما
&&
رنگ قلاش دارد زیرا که ما قلاشیم
\\
ما جمع ماهیانیم بر روی آب رانیم
&&
این خاک بوالهوس را بر روی خاک پاشیم
\\
تا ملک عشق دیدیم سرخیل مفلسانیم
&&
تا نقد عشق دیدیم تجار بی‌قماشیم
\\
\end{longtable}
\end{center}
