\begin{center}
\section*{غزل شماره ۱۹۰۹: دل معشوق سوزیده است بر من}
\label{sec:1909}
\addcontentsline{toc}{section}{\nameref{sec:1909}}
\begin{longtable}{l p{0.5cm} r}
دل معشوق سوزیده است بر من
&&
وزان سوزش جهان را سوخت خرمن
\\
بزد آتش به جان بنده شمعی
&&
کز او شد موم جان سنگ و آهن
\\
بدید آمد از آن آتش به ناگه
&&
میان شب هزاران صبح روشن
\\
به کوی عشق آوازه درافتاد
&&
که شد در خانه دل شکل روزن
\\
چه روزن کآفتاب نو برآمد
&&
که سایه نیست آن جا قدر سوزن
\\
از آن نوری که از لطفش برسته‌ست
&&
ز آتش گلبن و نسرین و سوسن
\\
از آن سو بازگرد ای یار بدخو
&&
بدین سو آ که این سوی است مؤمن
\\
به سوی بی‌سوی جمله بهار است
&&
به هر سو غیر این سرمای بهمن
\\
چو شمس الدین جان آمد ز تبریز
&&
تو جان کندن همی‌خواهی همی‌کن
\\
\end{longtable}
\end{center}
