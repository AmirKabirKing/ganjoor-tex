\begin{center}
\section*{غزل شماره ۱۲۵۳: بر ملک نیست نهان حال دل و نیک و بدش}
\label{sec:1253}
\addcontentsline{toc}{section}{\nameref{sec:1253}}
\begin{longtable}{l p{0.5cm} r}
بر ملک نیست نهان حال دل و نیک و بدش
&&
نفس اگر سر بکشد گوش کشان می‌کشدش
\\
جان دل اصل دل و اصل دلت فصل دلست
&&
وگرش او ندهد جان ز کی باشد مددش
\\
دل ز دردش چه خوشی‌ها و طرب‌ها دارد
&&
تو مگیر آن کرم وان دهش بی‌عددش
\\
ملک الموت برید از دلم آن روز طمع
&&
که مشرف شدم از طوق حیات ابدش
\\
برد سود دو جهان و آنچ نیاید به زبان
&&
کاروانی که غم عشق خدا راه زدش
\\
سوسن استایش او کرد کز او یافت زبان
&&
سرو آزادی او کرد که بخشید قدش
\\
بلبل آن را بستاید که زبانش آموخت
&&
گل از او جامه دراند که برافروخت خدش
\\
کیست کو دانه اومید در این خاک بکاشت
&&
که بهار کرمش بازنبخشید صدش
\\
میوه تلخ و ترش خام طمع بود ولی
&&
آفتاب کرم تو به کرم می‌پزدش
\\
آفتاب از پی آن سجده که هر شام کند
&&
چه زیان کرد از آن شاه که جان شد جسدش
\\
همه شب سجده کنان می‌رود و وقت سحر
&&
روش بخشد که بمیرد مه چرخ از حسدش
\\
هر که امروز کند شهوت خود را در گور
&&
هر یکی حور شود مونس گور و الحدش
\\
هر کی او اسب دواند به سوی گمراهی
&&
کند آن اسب لگدکوب نکال از لگدش
\\
بهل ابتر تو غزل را به ازل حیران باش
&&
که تمامش کند و شرح دهد هم صمدش
\\
\end{longtable}
\end{center}
