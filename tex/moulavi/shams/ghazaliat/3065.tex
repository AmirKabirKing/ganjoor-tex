\begin{center}
\section*{غزل شماره ۳۰۶۵: شدم به سوی چه آب همچو سقایی}
\label{sec:3065}
\addcontentsline{toc}{section}{\nameref{sec:3065}}
\begin{longtable}{l p{0.5cm} r}
شدم به سوی چه آب همچو سقایی
&&
برآمد از تک چه یوسفی معلایی
\\
سبک به دامن پیراهنش زدم من دست
&&
ز بوی پیرهنش دیده گشت بینایی
\\
به چاه در نظری کردم از تعجب من
&&
چه از ملاحت او گشته بود صحرایی
\\
کلیم روح به هر جا رسید میقاتش
&&
اگر چه کور بود گشت طور سینایی
\\
زنخ ز دست رقیبی که گفت از چه دور
&&
از این سپس منم و چاه و چون تو زیبایی
\\
کسی که زنده شود صد هزار مرده از او
&&
عجب نباشد اگر پیر گشت برنایی
\\
هزار گنج گدای چنین عجب کانی
&&
هزار سیم نثار لطیف سیمایی
\\
جهان چو آینه پرنقش توست اما کو
&&
به روی خوب تو بی‌آینه تماشایی
\\
سخن تو گو که مرا از حلاوت لب تو
&&
نه عقل ماند و نه اندیشه‌ای و نی رایی
\\
\end{longtable}
\end{center}
