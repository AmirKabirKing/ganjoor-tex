\begin{center}
\section*{غزل شماره ۱۰۹۹: ای نهاده بر سر زانو تو سر}
\label{sec:1099}
\addcontentsline{toc}{section}{\nameref{sec:1099}}
\begin{longtable}{l p{0.5cm} r}
ای نهاده بر سر زانو تو سر
&&
وز درون جان جمله باخبر
\\
پیش چشمت سرکش روپوش نیست
&&
آفرین‌ها بر صفای آن بصر
\\
بحر خونست ای صنم آن چشم نیست
&&
الحذر ای دل ز زخم آن نظر
\\
در مژه او گر چه دل را مژده‌هاست
&&
الحذر ای عاشقان از وی حذر
\\
او به زیر کاه آب خفته‌ست
&&
پا منه گستاخ ور نی رفت سر
\\
خفته شکلی اصل هر بیدادیی
&&
تا ز خوابش تو نخسپی ای پسر
\\
پاره خواهم کرد من جامه ز تو
&&
ای برادر پاره‌ای زین گرمتر
\\
سرکه آشامی و گویی شهد کو
&&
دست تو در زهر و گویی کو شکر
\\
روح را عمریست صابون می‌زنی
&&
یا تو را خود جان نبودست ای مگر
\\
تا به کی صیقل زنی آیینه را
&&
شرم بادت آخر از آیینه گر
\\
سوی بحر شمس تبریزی گریز
&&
تا برآرد ز آینه جانت گهر
\\
\end{longtable}
\end{center}
