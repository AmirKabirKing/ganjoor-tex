\begin{center}
\section*{غزل شماره ۲۲۸۶: ای تو برای آبرو آب حیات ریخته}
\label{sec:2286}
\addcontentsline{toc}{section}{\nameref{sec:2286}}
\begin{longtable}{l p{0.5cm} r}
ای تو برای آبرو آب حیات ریخته
&&
زهر گرفته در دهان قند و نبات ریخته
\\
مست و خراب این چنین چرخ ندانی از زمین
&&
از پی آب پارگین آب فرات ریخته
\\
همچو خران به کاه و جو نیست روا چنین مرو
&&
بر فقرا تو درنگر زر صدقات ریخته
\\
روح شو و جهت مجو ذات شو و صفت مگو
&&
زان شه بی‌جهت نگر جمله جهات ریخته
\\
آه دریغ مغز تو در ره پوست باخته
&&
آه دریغ شاه تو در غم مات ریخته
\\
از غم مات شاه دل خانه به خانه می‌دود
&&
رنگ رخ و پیاده‌ها بهر نجات ریخته
\\
جسته برات جان از او باز چو دیده روی او
&&
کیسه دریده پیش او جمله برات ریخته
\\
از صفتش صفات ما خارشناس گل شده
&&
باز صفات ما چو گل در ره ذات ریخته
\\
بال و پری که او تو را برد و اسیر دام کرد
&&
بال و پری است عاریت روز وفات ریخته
\\
\end{longtable}
\end{center}
