\begin{center}
\section*{غزل شماره ۲۴۷۱: سوخت یکی جهان به غم آتش غم پدید نی}
\label{sec:2471}
\addcontentsline{toc}{section}{\nameref{sec:2471}}
\begin{longtable}{l p{0.5cm} r}
سوخت یکی جهان به غم آتش غم پدید نی
&&
صورت این طلسم را هیچ کسی بدید نی
\\
می‌کشدم به هر طرف قوت کهربای او
&&
ای عجبا بدید کس آنک مرا کشید نی
\\
هست سماع چنگ نی هست شراب رنگ نی
&&
صد قدح است بر قدح آنک قدح چشید نی
\\
عشق قرابه باز و من در کف او چو شیشه‌ای
&&
شیشه شکست زیر پا پای کسی خلید نی
\\
در قدم روندگان شیخ و مرید بی‌عدد
&&
در نفس یگانگی شیخ نه و مرید نی
\\
آنک میان مردمان شهره شد و حدیث شد
&&
سایه بایزید بد مایه بایزید نی
\\
مژده دهید عاشقان عید وصال می رسد
&&
ز آنک ندید هیچ کس خود رمضان و عید نی
\\
\end{longtable}
\end{center}
