\begin{center}
\section*{غزل شماره ۲۵۵۴: اگر بی‌من خوشی یارا به صد دامم چه می‌بندی}
\label{sec:2554}
\addcontentsline{toc}{section}{\nameref{sec:2554}}
\begin{longtable}{l p{0.5cm} r}
اگر بی‌من خوشی یارا به صد دامم چه می‌بندی
&&
وگر ما را همی‌خواهی چرا تندی نمی‌خندی
\\
کسی کو در شکرخانه شکر نوشد به پیمانه
&&
بدین سرکای نه ساله نداند کرد خرسندی
\\
بخند ای دوست چون گلشن مبادا خاطر دشمن
&&
کند شادی و پندارد که دل زین بنده برکندی
\\
چو رشک ماه و گل گشتی چو در دل‌ها طمع کشتی
&&
نباشد لایق از حسنت که برگردی ز پیوندی
\\
خوشا آن حالت مستی که با ما عهد می‌بستی
&&
مرا مستانه می‌گفتی که ما را خویش و فرزندی
\\
پیاپی باده می‌دادی به صد لطف و به صد شادی
&&
که گیر این جام بی‌خویشی که باخویشی و هشمندی
\\
سلام علیک ای خواجه بهانه چیست این ساعت
&&
نه دریایی و دریادل نه ساقی و خداوندی
\\
نه یاقوتی نه مرجانی نه آرام دل و جانی
&&
نه بستان و گلستانی نه کان شکر و قندی
\\
خمش باشم بدان شرطی که بدهی می خموشانه
&&
من از گولی دهم پندت نه ز آنک قابل پندی
\\
\end{longtable}
\end{center}
