\begin{center}
\section*{غزل شماره ۱۷۵۸: در وصالت چرا بیاموزم}
\label{sec:1758}
\addcontentsline{toc}{section}{\nameref{sec:1758}}
\begin{longtable}{l p{0.5cm} r}
در وصالت چرا بیاموزم
&&
در فراقت چرا بیاموزم
\\
یا تو با درد من بیامیزی
&&
یا من از تو دوا بیاموزم
\\
می گریزی ز من که نادانم
&&
یا بیامیزی یا بیاموزم
\\
پیش از این ناز و خشم می کردم
&&
تا من از تو جدا بیاموزم
\\
چون خدا با تو است در شب و روز
&&
بعد از این از خدا بیاموزم
\\
در فراقت سزای خود دیدم
&&
چون بدیدم سزا بیاموزم
\\
خاک پای تو را به دست آرم
&&
تا از او کیمیا بیاموزم
\\
آفتاب تو را شوم ذره
&&
معنی والضحی بیاموزم
\\
کهربای تو را شوم کاهی
&&
جذبه کهربا بیاموزم
\\
از دو عالم دو دیده بردوزم
&&
این من از مصطفی بیاموزم
\\
سر مازاغ و ماطغی را من
&&
جز از او از کجا بیاموزم
\\
در هوایش طواف سازم تا
&&
چون فلک در هوا بیاموزم
\\
بند هستی فروگشادم تا
&&
همچو مه بی‌قبا بیاموزم
\\
همچو ماهی زره ز خود سازم
&&
تا به بحر آشنا بیاموزم
\\
همچو دل خون خورم که تا چون دل
&&
سیر بی‌دست و پا بیاموزم
\\
در وفا نیست کس تمام استاد
&&
پس وفا از وفا بیاموزم
\\
ختمش این شد که خوش لقای منی
&&
از تو خوش خوش لقا بیاموزم
\\
\end{longtable}
\end{center}
