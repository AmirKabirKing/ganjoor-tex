\begin{center}
\section*{غزل شماره ۱۹۸۴: بده آن مرد ترش را قدحی ای شه شیرین}
\label{sec:1984}
\addcontentsline{toc}{section}{\nameref{sec:1984}}
\begin{longtable}{l p{0.5cm} r}
بده آن مرد ترش را قدحی ای شه شیرین
&&
صدقات تو روان است به هر بیوه و مسکین
\\
صدقات تو لطیف است توان خورد دو صد من
&&
که نداند لب بالا و نجنبد لب زیرین
\\
هله ای باغ نگویی به چه لب باده کشیدی
&&
مگر اشکوفه بگوید پنهان با گل و نسرین
\\
چه شراب است کز آن بو گل‌تر آهوی ناف است
&&
به زمستان نه که دیدی همه را چون سگ گرگین
\\
هله تا جمع رسیدن بده آن می به کف من
&&
پس من زهره بنوشد قدح از ساعد پروین
\\
وگر آن مست نهد سر که رباید ز تو ساغر
&&
مده او را تو مرا ده که منم بر در تحسین
\\
چه کند باده حق را جگر باطل فانی
&&
چه شناسد مه جان را نظر و غمزه عنین
\\
هنر و زر چو فزون شد خطر و خوف کنون شد
&&
ملکان را تب لرز است و حریر است نهالین
\\
چو مه توبه درآمد مه توبه شکن آمد
&&
شکنش باد همیشه تو بگو نیز که آمین
\\
\end{longtable}
\end{center}
