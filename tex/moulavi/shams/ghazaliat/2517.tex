\begin{center}
\section*{غزل شماره ۲۵۱۷: ز رنگ روی شمس الدین گرم خود بو و رنگستی}
\label{sec:2517}
\addcontentsline{toc}{section}{\nameref{sec:2517}}
\begin{longtable}{l p{0.5cm} r}
ز رنگ روی شمس الدین گرم خود بو و رنگستی
&&
مرا از روی این خورشید عارستی و ننگستی
\\
قرابه دل ز اشکستن شدی ایمن اگر از لطف
&&
شراب وصل آن شه را دمی در وی درنگستی
\\
به بزمش جان‌های ما ندانستی سر از پایان
&&
اگر نه هجر بدمستش به بدمستی و جنگستی
\\
الا ای ساقی بزمش بگردان جام باقی را
&&
چرا بر من دلت رحمی نیارد گویی سنگستی
\\
از آن می کو ز بهر شه دهان خویش بگشادی
&&
همه هستی فروبردی تو پنداری نهنگستی
\\
ز بانگ رعد آن دریا تو بنگر چون به جوش آید
&&
ولیک آن بحر می‌بودی و رعدش بانگ چنگستی
\\
روان گشته میش چون خون درون دل به هر سویی
&&
تو گویی دل چو قدسستی و می همچون فرنگستی
\\
که لشکرهای اسلام شه ما را درون قدس
&&
ز نصرت‌های یزدانی بر آن افرنگ هنگستی
\\
به یک ساغر نگردم مست تو ساقی بیشتر گردان
&&
خرابی گشتمی گر می ز جام شاه شنگستی
\\
ایا تبریز عقلم را خیال تو بشوراند
&&
تو گویی باده صافی خیالت گویی بنگستی
\\
ترنگ چنگ وصل او بپراندهمی جان را
&&
تو گویی عیسی خوش دم درون آن ترنگستی
\\
پیاپی گردد از وصلش قدح‌ها بر مثال آن
&&
که اندر جنگ سلطانی قدح تیر خدنگستی
\\
چنین عقلی که از تزویر مو در موی می‌بیند
&&
شمار موی عقل آن جا تو بینی گویی دنگستی
\\
ز تیزی‌های آن جامش که برق از وی فغان آید
&&
قدح در رو همی‌آید بریزش گویی لنگستی
\\
چه بالایی همی‌جوید می اندر مغز مستانش
&&
چو گردند شیرگیر از وی مگر گویی پلنگستی
\\
فراوان ریز در جانم از آن می‌های ربانی
&&
ز بحر صدر شمس الدین که کان خمر تنگستی
\\
\end{longtable}
\end{center}
