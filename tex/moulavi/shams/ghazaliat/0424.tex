\begin{center}
\section*{غزل شماره ۴۲۴: دلبری و بی‌دلی اسرار ماست}
\label{sec:0424}
\addcontentsline{toc}{section}{\nameref{sec:0424}}
\begin{longtable}{l p{0.5cm} r}
دلبری و بی‌دلی اسرار ماست
&&
کار کار ماست چون او یار ماست
\\
نوبت کهنه فروشان درگذشت
&&
نوفروشانیم و این بازار ماست
\\
نوبهاری کو جهان را نو کند
&&
جان گلزارست اما زار ماست
\\
عقل اگر سلطان این اقلیم شد
&&
همچو دزد آویخته بر دار ماست
\\
آنک افلاطون و جالینوس ماست
&&
پرفنا و علت و بیمار ماست
\\
گاو و ماهی ثری قربان ماست
&&
شیر گردونی به زیر بار ماست
\\
هر چه اول زهر بد تریاق شد
&&
هر چه آن غم بد کنون غمخوار ماست
\\
دعوی شیری کند هر شیرگیر
&&
شیرگیر و شیر او کفتار ماست
\\
ترک خویش و ترک خویشان می‌کنیم
&&
هر چه خویش ما کنون اغیار ماست
\\
خودپرستی نامبارک حالتی‌ست
&&
کاندر او ایمان ما انکار ماست
\\
هر غزل کان بی‌من آید خوش بود
&&
کاین نوا بی‌فر ز چنگ و تار ماست
\\
شمس تبریزی به نور ذوالجلال
&&
در دو عالم مایه اقرار ماست
\\
\end{longtable}
\end{center}
