\begin{center}
\section*{غزل شماره ۲۲۹۷: چو در دل پای بنهادی بشد از دست اندیشه}
\label{sec:2297}
\addcontentsline{toc}{section}{\nameref{sec:2297}}
\begin{longtable}{l p{0.5cm} r}
چو در دل پای بنهادی بشد از دست اندیشه
&&
میان بگشاد اسرار و میان بربست اندیشه
\\
به پیش جان درآمد دل که اندر خود مکن منزل
&&
گران جان دید مر جان را سبک برجست اندیشه
\\
رسید از عشق جاسوسش که بسم الله زمین بوسش
&&
در این اندیشه بیخود شد به حق پیوست اندیشه
\\
خرابات بتان درشد حریف رطل و ساغر شد
&&
همه غیبش مصور شد زهی سرمست اندیشه
\\
برست او از خوداندیشی چنان آمد ز بی‌خویشی
&&
که از هر کس همی‌پرسد عجب خود هست اندیشه
\\
فلک از خوف دل کم زد دو دست خویش بر هم زد
&&
که از من کس نرست آخر چگونه رست اندیشه
\\
چنین اندیشه را هر کس نهد دامی به پیش و پس
&&
گمان دارد که درگنجد به دام و شست اندیشه
\\
چو هر نقشی که می‌جوید ز اندیشه همی‌روید
&&
تو مر هر نقش را مپرست و خود بپرست اندیشه
\\
جواهر جمله ساکن بد همه همچون اماکن بد
&&
شکافید این جواهر را و بیرون جست اندیشه
\\
جهان کهنه را بنگر گهی فربه گهی لاغر
&&
که درد کهنه زان دارد که نوزاد است اندیشه
\\
که درد زه ازان دارد که تا شه زاده‌ای زاید
&&
نتیجه سربلند آمد چو شد سربست اندیشه
\\
چو دل از غم رسول آمد بر دل جبرئیل آمد
&&
چو مریم از دو صد عیسی شده‌ست آبست اندیشه
\\
چو شهد شمس تبریزی فزاید در مزاجم خون
&&
از آن چون زخم فصادی رگ دل خست اندیشه
\\
\end{longtable}
\end{center}
