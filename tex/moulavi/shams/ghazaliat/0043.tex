\begin{center}
\section*{غزل شماره ۴۳: کاهل و ناداشت بدم کام درآورد مرا}
\label{sec:0043}
\addcontentsline{toc}{section}{\nameref{sec:0043}}
\begin{longtable}{l p{0.5cm} r}
کاهل و ناداشت بدم کام درآورد مرا
&&
طوطی اندیشه او همچو شکر خورد مرا
\\
تابش خورشید ازل پرورش جان و جهان
&&
بر صفت گلبشکر پخت و بپرورد مرا
\\
گفتم ای چرخ فلک مرد جفای تو نیم
&&
گفت زبون یافت مگر ای سره این مرد مرا
\\
ای شه شطرنج فلک مات مرا برد تو را
&&
ای ملک آن تخت تو را تخته این نرد مرا
\\
تشنه و مستسقی تو گشته‌ام ای بحر چنانک
&&
بحر محیط ار بخورم باشد درخورد مرا
\\
حسن غریب تو مرا کرد غریب دو جهان
&&
فردی تو چون نکند از همگان فرد مرا
\\
رفتم هنگام خزان سوی رزان دست گزان
&&
نوحه گر هجر تو شد هر ورق زرد مرا
\\
فتنه عشاق کند آن رخ چون روز تو را
&&
شهره آفاق کند این دل شب گرد مرا
\\
راست چو شقه علمت رقص کنانم ز هوا
&&
بال مرا بازگشا خوش خوش و منورد مرا
\\
صبح دم سرد زند از پی خورشید زند
&&
از پی خورشید تو است این نفس سرد مرا
\\
جزو ز جزوی چو برید از تن تو درد کند
&&
جزو من از کل ببرد چون نبود درد مرا
\\
بنده آنم که مرا بی‌گنه آزرده کند
&&
چون صفتی دارد از آن مه که بیازرد مرا
\\
هر کسکی را هوسی قسم قضا و قدر است
&&
عشق وی آورد قضا هدیه ره آورد مرا
\\
اسب سخن بیش مران در ره جان گرد مکن
&&
گر چه که خود سرمه جان آمد آن گرد مرا
\\
\end{longtable}
\end{center}
