\begin{center}
\section*{غزل شماره ۲۰۵۹: گفت لبم ناگهان نام گل و گلستان}
\label{sec:2059}
\addcontentsline{toc}{section}{\nameref{sec:2059}}
\begin{longtable}{l p{0.5cm} r}
گفت لبم ناگهان نام گل و گلستان
&&
آمد آن گلعذار کوفت مرا بر دهان
\\
گفت که سلطان منم جان گلستان منم
&&
حضرت چون من شهی وآنگه یاد فلان
\\
دف منی هین مخور سیلی هر ناکسی
&&
نای منی هین مکن از دم هر کس فغان
\\
پیش چو من کیقباد چشم بدم دور باد
&&
شرم ندارد کسی یاد کند از کهان
\\
جغد بود کو به باغ یاد خرابه کند
&&
زاغ بود کو بهار یاد کند از خزان
\\
چنگ به من درزدی چنگ منی در کنار
&&
تار که در زخمه‌ام سست شود بگسلان
\\
پشت جهان دیده‌ای روی جهان را ببین
&&
پشت به خود کن که تا روی نماید جهان
\\
ای قمر زیر میغ خویش ندیدی دریغ
&&
چند چو سایه دوی در پی این دیگران
\\
بس که مرا دام شعر از دغلی بند کرد
&&
تا که ز دستم شکار جست سوی گلستان
\\
در پی دزدی بدم دزد دگر بانگ کرد
&&
هشتم بازآمدم گفتم و هین چیست آن
\\
گفت که اینک نشان دزد تو این سوی رفت
&&
دزد مرا باد داد آن دغل کژنشان
\\
\end{longtable}
\end{center}
