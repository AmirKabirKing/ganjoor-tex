\begin{center}
\section*{غزل شماره ۱۶۸۴: من از این خانه به در می نروم}
\label{sec:1684}
\addcontentsline{toc}{section}{\nameref{sec:1684}}
\begin{longtable}{l p{0.5cm} r}
من از این خانه به در می نروم
&&
من از این شهر سفر می نروم
\\
منم و این صنم و باقی عمر
&&
من از او جای دگر می نروم
\\
خاکیان رو به اثر آوردند
&&
من ز اثیرم به اثر می نروم
\\
ای دو دیده ز نظر دورم کن
&&
من چو دیده به نظر می نروم
\\
بخت من زیر و زبر کرد غمش
&&
چون فلک زیر و زبر می نروم
\\
خانه چرخ و زمین تاریک است
&&
من ز خرگاه قمر می نروم
\\
گر چو خورشید مرا تیغ زند
&&
من ز تیغش به سپر می نروم
\\
بس بود عشق شهم تاج و کمر
&&
من سوی تاج و کمر می نروم
\\
گم کنم خویش در اوصاف ملک
&&
من در اوصاف بشر می نروم
\\
عشق او چون شجر و من موسی
&&
من گزافه به شجر می نروم
\\
زان شجر خواند یکی نور مرا
&&
ور نه من بهر خضر می نروم
\\
چون شجر خوش بکشم آب حیات
&&
من چو هیزم به سفر می نروم
\\
شمس تبریز که نور سحر است
&&
جز به نورش به سحر می نروم
\\
\end{longtable}
\end{center}
