\begin{center}
\section*{غزل شماره ۲۰۵۵: مست شدی عاقبت آمدی اندر میان}
\label{sec:2055}
\addcontentsline{toc}{section}{\nameref{sec:2055}}
\begin{longtable}{l p{0.5cm} r}
مست شدی عاقبت آمدی اندر میان
&&
مست ز خود می‌شوی کیست دگر در جهان
\\
عاقبت امر رست مرغ فلک از قفس
&&
عاقبت امر جست تیر مراد از کمان
\\
چند زنیم ای کریم طبل تو زیر گلیم
&&
چند کنیم ای ندیم مستی خود را نهان
\\
بازرسید از الست کار برون شد ز دست
&&
فاش بود فاش مست خاصه ز بوی دهان
\\
دارد طامات ما بوی خرابات ما
&&
هست شرابات ما از کف شاهنشهان
\\
جمله اجزای خاک روح شد و جان پاک
&&
عالم خاکش مخوان مایه اکسیر خوان
\\
تو کمری ما میان یا تو میان ما کمر
&&
گر کمری گر میان بی‌تو مبا گر میان
\\
گاه به دزدی درآ کیسه دل را ببر
&&
گاه مرا دزد گیر گو که منم پاسبان
\\
گه بربا همچون گرگ بره درویش را
&&
گه سگ بر من گمار های کنان چون شبان
\\
چون تو ندیده‌ست کس کس تویی ای جان و بس
&&
نادره ای در جهان اسب وفا درجهان
\\
گر چه جهان است عشق جان و جهان است عشق
&&
گر چه نهان است یار هست سر سر نهان
\\
چشم تو با چشم من گفت چه مطمع کسی
&&
هم بخوری قند ما هم ببری ارمغان
\\
هر تن و هر جان که هست خاک تو بوده‌ست مست
&&
غافلشان کرده‌ای زان هوس بی‌نشان
\\
باز چو ناگه کنی سلسله جنبانیی
&&
شور برآرد به کبر از جهت امتحان
\\
کافر و مؤمن مگو فاسق و محسن مجو
&&
جمله خراب تواند بر همه افسون بخوان
\\
کیست که مست تو نیست عشوه پرست تو نیست
&&
مهره دست تو نیست دست کرم برفشان
\\
سختتر از کوه چیست چونک به تو بنگریست
&&
زنده شد از عشق زیست شهره شد اندر زمان
\\
\end{longtable}
\end{center}
