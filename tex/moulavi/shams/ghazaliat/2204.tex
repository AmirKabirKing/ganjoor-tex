\begin{center}
\section*{غزل شماره ۲۲۰۴: عاشقی بر من پریشانت کنم نیکو شنو}
\label{sec:2204}
\addcontentsline{toc}{section}{\nameref{sec:2204}}
\begin{longtable}{l p{0.5cm} r}
عاشقی بر من پریشانت کنم نیکو شنو
&&
کم عمارت کن که ویرانت کنم نیکو شنو
\\
گر دو صد خانه کنی زنبوروار و موروار
&&
بی‌کس و بی‌خان و بی‌مانت کنم نیکو شنو
\\
تو بر آنک خلق مست تو شوند از مرد و زن
&&
من بر آنک مست و حیرانت کنم نیکو شنو
\\
چون خلیلی هیچ از آتش مترس ایمن برو
&&
من ز آتش صد گلستانت کنم نیکو شنو
\\
گر که قافی تو را چون آسیای تیزگرد
&&
آورم در چرخ و گردانت کنم نیکو شنو
\\
ور تو افلاطون و لقمانی به علم و کر و فر
&&
من به یک دیدار نادانت کنم نیکو شنو
\\
تو به دست من چو مرغی مرده‌ای وقت شکار
&&
من صیادم دام مرغانت کنم نیکو شنو
\\
بر سر گنجی چو ماری خفته‌ای ای پاسبان
&&
همچو مار خسته پیچانت کنم نیکو شنو
\\
ای صدف چون آمدی در بحر ما غمگین مباش
&&
چون صدف‌ها گوهرافشانت کنم نیکو شنو
\\
بر گلویت تیغ‌ها را دست نی و زخم نی
&&
گر چو اسماعیل قربانت کنم نیکو شنو
\\
دامن ما گیر اگر تردامنی تردامنی
&&
تا چو مه از نور دامانت کنم نیکو شنو
\\
من همایم سایه کردم بر سرت از فضل خود
&&
تا که افریدون و سلطانت کنم نیکو شنو
\\
هین قرائت کم کن و خاموش باش و صبر کن
&&
تا بخوانم عین قرآنت کنم نیکو شنو
\\
\end{longtable}
\end{center}
