\begin{center}
\section*{غزل شماره ۱۵۵: از فراق شمس دین افتاده‌ام در تنگنا}
\label{sec:0155}
\addcontentsline{toc}{section}{\nameref{sec:0155}}
\begin{longtable}{l p{0.5cm} r}
از فراق شمس دین افتاده‌ام در تنگنا
&&
او مسیح روزگار و درد چشمم بی‌دوا
\\
گر چه درد عشق او خود راحت جان منست
&&
خون جانم گر بریزد او بود صد خونبها
\\
عقل آواره شده دوش آمد و حلقه بزد
&&
من بگفتم کیست بر در باز کن در اندرآ
\\
گفت آخر چون درآید خانه تا سر آتشست
&&
می‌بسوزد هر دو عالم را ز آتش‌های لا
\\
گفتمش تو غم مخور پا اندرون نه مردوار
&&
تا کند پاکت ز هستی هست گردی ز اجتبا
\\
عاقبت بینی مکن تا عاقبت بینی شوی
&&
تا چو شیر حق باشی در شجاعت لافتی
\\
تا ببینی هستیت چون از عدم سر برزند
&&
روح مطلق کامکار و شهسوار هل اتی
\\
جمله عشق و جمله لطف و جمله قدرت جمله دید
&&
گشته در هستی شهید و در عدم او مرتضی
\\
آن عدم نامی که هستی موج‌ها دارد از او
&&
کز نهیب و موج او گردان شد صد آسیا
\\
اندر آن موج اندرآیی چون بپرسندت از این
&&
تو بگویی صوفیم صوفی بخواند مامضی
\\
از میان شمع بینی برفروزد شمع تو
&&
نور شمعت اندرآمیزد به نور اولیا
\\
مر تو را جایی برد آن موج دریا در فنا
&&
دررباید جانت را او از سزا و ناسزا
\\
لیک از آسیب جانت وز صفای سینه‌ات
&&
بی تو داده باغ هستی را بسی نشو و نما
\\
در جهان محو باشی هست مطلق کامران
&&
در حریم محو باشی پیشوا و مقتدا
\\
دیده‌های کون در رویت نیارد بنگرید
&&
تا که نجهد دیده‌اش از شعشعه آن کبریا
\\
ناگهان گردی بخیزد زان سوی محو فنا
&&
که تو را وهمی نبوده زان طریق ماورا
\\
شعله‌های نور بینی از میان گردها
&&
محو گردد نور تو از پرتو آن شعله‌ها
\\
زو فروآ تو ز تخت و سجده‌ای کن زانک هست
&&
آن شعاع شمس دین شهریار اصفیا
\\
ور کسی منکر شود اندر جبین او نگر
&&
تا ببینی داغ فرعونی بر آن جا قد طغی
\\
تا نیارد سجده‌ای بر خاک تبریز صفا
&&
کم نگردد از جبینش داغ نفرین خدا
\\
\end{longtable}
\end{center}
