\begin{center}
\section*{غزل شماره ۳۶۱: اگر حوا بدانستی ز رنگت}
\label{sec:0361}
\addcontentsline{toc}{section}{\nameref{sec:0361}}
\begin{longtable}{l p{0.5cm} r}
اگر حوا بدانستی ز رنگت
&&
سترون ساختی خود را ز ننگت
\\
سیاهی جانت ار محسوس گشتی
&&
همه عالم شدی زنگی ز زنگت
\\
تو آن ماری که سنگ از تو دریغ است
&&
سرت را کس نکوبد جز به سنگت
\\
اگر دریا درافتی ای منافق
&&
ز زشتی کی خورد مار و نهنگت
\\
مرا گویی که از معنی نظر کن
&&
رها کن صورت نقش و پلنگت
\\
چه گویم با تو ای نقش مزور
&&
چه معنی گنجد اندر جان تنگت
\\
هوای شمس تبریزی چو قدس است
&&
تو آن خوکی که نپذیرد فرنگت
\\
\end{longtable}
\end{center}
