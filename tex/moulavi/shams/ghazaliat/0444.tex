\begin{center}
\section*{غزل شماره ۴۴۴: ساقی بیار باده که ایام بس خوشست}
\label{sec:0444}
\addcontentsline{toc}{section}{\nameref{sec:0444}}
\begin{longtable}{l p{0.5cm} r}
ساقی بیار باده که ایام بس خوشست
&&
امروز روز باده و خرگاه و آتش است
\\
ساقی ظریف و باده لطیف و زمان شریف
&&
مجلس چو چرخ روشن و دلدار مه وشست
\\
بشنو نوای نای کز آن نفخه بانواست
&&
درکش شراب لعل که غم در کشاکش است
\\
امروز غیر توبه نبینی شکسته‌ای
&&
امروز زلف دوست بود کان مشوش است
\\
هفتاد بار توبه کند شب رسول حق
&&
توبه شکن حق است که توبه مخمش است
\\
آن صورت نهان که جهان در هوای او است
&&
بر آب و گل به قدرت یزدان منقش است
\\
امروز جان بیابد هر جا که مرده‌ای است
&&
چشمی دگر گشاید چشمی که اعمش است
\\
شاخی که خشک نیست ز آتش مسلم است
&&
از تیر غم ندارد سغری که ترکش است
\\
در عاشقی نگر که رخش بوسه گاه او است
&&
منگر بدانک زرد و ضعیف و مکرمش است
\\
بس تن اسیر خاک و دلش بر فلک امیر
&&
بس دانه زیر خاک درختش منعش است
\\
در خاک کی بود که دلش گنج گوهر است
&&
دلتنگ کی بود که دلارام در کش است
\\
ای مرده شوی من زنخم را ببند سخت
&&
زیرا که بی‌دهان دل و جانم شکرچش است
\\
خامش زنخ مزن که تو را مرده شوی نیست
&&
ذات تو را مقام نه پنج است و نی شش است
\\
\end{longtable}
\end{center}
