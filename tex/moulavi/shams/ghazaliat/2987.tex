\begin{center}
\section*{غزل شماره ۲۹۸۷: ای جان و ای دو دیده بینا چگونه‌ای}
\label{sec:2987}
\addcontentsline{toc}{section}{\nameref{sec:2987}}
\begin{longtable}{l p{0.5cm} r}
ای جان و ای دو دیده بینا چگونه‌ای
&&
وی رشک ماه و گنبد مینا چگونه‌ای
\\
ای ما و صد چو ما ز پی تو خراب و مست
&&
ما بی‌تو خسته‌ایم تو بی‌ما چگونه‌ای
\\
آن جا که با تو نیست چو سوراخ کژدم است
&&
و آن جا که جز تو نیست تو آن جا چگونه‌ای
\\
ای جان تو در گزینش جان‌ها چه می‌کنی
&&
وی گوهری فزوده ز دریا چگونه‌ای
\\
ای مرغ عرش آمده در دام آب و گل
&&
در خون و خلط و بلغم و صفرا چگونه‌ای
\\
زان گلشن لطیف به گلخن فتاده‌ای
&&
با اهل گولخن به مواسا چگونه‌ای
\\
ای کوه قاف صبر و سکینه چه صابری
&&
وی عزلتی گرفته چو عنقا چگونه‌ای
\\
عالم به توست قایم تو در چه عالمی
&&
تن‌ها به توست زنده تو تنها چگونه‌ای
\\
ای آفتاب از تو خجل در چه مشرقی
&&
وی زهر ناب با تو چو حلوا چگونه‌ای
\\
زیر و زبر شدیمت بی‌زیر و بی‌زبر
&&
ای درفکنده فتنه و غوغا چگونه‌ای
\\
گر غایبی ز دل تو در این دل چه می‌کنی
&&
ور در دلی ز دوده سودا چگونه‌ای
\\
ای شاه شمس مفخر تبریز بی‌نظیر
&&
در قاب قوس قرب و در ادنی چگونه‌ای
\\
\end{longtable}
\end{center}
