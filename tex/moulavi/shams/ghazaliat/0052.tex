\begin{center}
\section*{غزل شماره ۵۲: چون همه عشق روی تست جمله رضای نفس ما}
\label{sec:0052}
\addcontentsline{toc}{section}{\nameref{sec:0052}}
\begin{longtable}{l p{0.5cm} r}
چون همه عشق روی تست جمله رضای نفس ما
&&
کفر شدست لاجرم ترک هوای نفس ما
\\
چونک به عشق زنده شد قصد غزاش چون کنم
&&
غمزه خونی تو شد حج و غزای نفس ما
\\
نیست ز نفس ما مگر نقش و نشان سایه‌ای
&&
چون به خم دو زلف تست مسکن و جای نفس ما
\\
عشق فروخت آتشی کآب حیات از او خجل
&&
پرس که از برای که آن ز برای نفس ما
\\
هژده هزار عالم عیش و مراد عرضه شد
&&
جز به جمال تو نبود جوشش و رای نفس ما
\\
دوزخ جای کافران جنت جای مؤمنان
&&
عشق برای عاشقان محو سزای نفس ما
\\
اصل حقیقت وفا سر خلاصه رضا
&&
خواجه روح شمس دین بود صفای نفس ما
\\
در عوض عبیر جان در بدن هزار سنگ
&&
از تبریز خاک را کحل ضیای نفس ما
\\
\end{longtable}
\end{center}
