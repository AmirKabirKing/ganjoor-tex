\begin{center}
\section*{غزل شماره ۲۰۷۸: چهار روز ببودم به پیش تو مهمان}
\label{sec:2078}
\addcontentsline{toc}{section}{\nameref{sec:2078}}
\begin{longtable}{l p{0.5cm} r}
چهار روز ببودم به پیش تو مهمان
&&
سه روز دیگر خواهم بدن یقین می‌دان
\\
به حق این سه و آن چار رو ترش نکنی
&&
که تا نیفتد این دل به صد هزار گمان
\\
به هر طعام خوشم من جز این یکی ترشی
&&
که سخت این ترشی کند می‌کند دندان
\\
که جمله ترشی‌ها بدان گوار شود
&&
که تو ترش نکنی روی ای گل خندان
\\
گشای آن لب خندان که آن گوارش ماست
&&
که تعبیه‌ست دو صد گلشکر در آن احسان
\\
ترش مکن که نخواهد ترش شدن آن رو
&&
که می‌دهد مدد قند هر دمش رحمان
\\
چه جای این که اگر صد هزار تلخ و ترش
&&
به نزد روی تو افتد شود خوش و شادان
\\
مگر به روز قیامت نهان شود رویت
&&
وگر نه دوزخ خوشتر شود ز صدر جنان
\\
اگر میان زمستان بهار نو خواهی
&&
درآ به باغ جمالت درخت‌ها بفشان
\\
به روز جمعه چو خواهی که عیدها بینند
&&
برآی بر سر منبر صفات خود برخوان
\\
غلط شدم که تو گر برروی به منبر بر
&&
پری برآرد منبر چو دل شود پران
\\
مرا به قند و شکرهای خویش مهمان کن
&&
علف میاور پیشم منه نیم حیوان
\\
فرشته از چه خورد از جمال حضرت حق
&&
غذای ماه و ستاره ز آفتاب جهان
\\
غذای خلق در آن قحط حسن یوسف بود
&&
که اهل مصر رهیده بدند از غم نان
\\
خمش کنم که دگربار یار می‌خواهد
&&
که درروم به سخن او برون جهد ز میان
\\
غلط که او چو بخواهد که از خرم فکند
&&
حذر چه سود کند یا گرفتن پالان
\\
مگر همو بنماید ره حذر کردن
&&
همو بدوزد انبان همو درد انبان
\\
مرا سخن همه با او است گر چه در ظاهر
&&
عتاب و صلح کنم گرم با فلان و فلان
\\
خمش که تا نزند بر چنین حدیث هوا
&&
از آنک باد هوا نیست محرم ایشان
\\
\end{longtable}
\end{center}
