\begin{center}
\section*{غزل شماره ۱۲۰۷: نیم شب از عشق تا دانی چه می‌گوید خروس}
\label{sec:1207}
\addcontentsline{toc}{section}{\nameref{sec:1207}}
\begin{longtable}{l p{0.5cm} r}
نیم شب از عشق تا دانی چه می‌گوید خروس
&&
خیز شب را زنده دار و روز روشن نستکوس
\\
پرها بر هم زند یعنی دریغا خواجه‌ام
&&
روزگار نازنین را می‌دهد بر آنموس
\\
در خروش است آن خروس و تو همی در خواب خوش
&&
نام او را طیر خوانی نام خود را اثربوس
\\
آن خروسی که تو را دعوت کند سوی خدا
&&
او به صورت مرغ باشد در حقیقات انگلوس
\\
من غلام آن خروسم کو چنین پندی دهد
&&
خاک پای او به آید از سر واسیلیوس
\\
گرد کفش خاک پای مصطفی را سرمه ساز
&&
تا نباشی روز حشر از جمله کالویروس
\\
رو شریعت را گزین و امر حق را پاس دار
&&
گر عرب باشی وگر ترک وگر سراکنوس
\\
\end{longtable}
\end{center}
