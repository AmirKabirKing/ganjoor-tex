\begin{center}
\section*{غزل شماره ۳۲۵: که دید ای عاشقان شهری که شهر نیکبختانست}
\label{sec:0325}
\addcontentsline{toc}{section}{\nameref{sec:0325}}
\begin{longtable}{l p{0.5cm} r}
که دید ای عاشقان شهری که شهر نیکبختانست
&&
که آن جا کم رسد عاشق و معشوق فراوانست
\\
که تا نازی کنیم آن جا و بازاری نهیم آن جا
&&
که تا دل‌ها خنک گردد که دل‌ها سخت بریانست
\\
نباشد این چنین شهری ولی باری کم از شهری
&&
که در وی عدل و انصافست و معشوق مسلمانست
\\
که این سو عاشقان باری چو عود کهنه می‌سوزد
&&
وان معشوق نادرتر کز او آتش فروزانست
\\
خداوندا به احسانت به حق نور تابانت
&&
مگیر آشفته می‌گویم که دل بی‌تو پریشانست
\\
تو مستان را نمی‌گیری پریشان را نمی‌گیری
&&
خنک آن را که می‌گیری که جانم مست ایشانست
\\
اگر گیری ور اندازی چه غم داری چه کم داری
&&
که عاشق چون گیا این جا بیابان در بیابانست
\\
بخندد چشم مریخش مرا گوید نمی‌ترسی
&&
نگارا بوی خون آید اگر مریخ خندانست
\\
دلم با خویشتن آمد شکایت را رها کردم
&&
هزاران جان همی‌بخشد چه شد گر خصم یک جانست
\\
منم قاضی خشم آلود و هر دو خصم خشنودند
&&
که جانان طالب جانست و جان جویای جانانست
\\
که جان ذره‌ست و او کیوان که جان میوه‌ست و او بستان
&&
که جان قطره‌ست و او عمان که جان حبه‌ست و او کانست
\\
سخن در پوست می‌گویم که جان این سخن غیبست
&&
نه در اندیشه می‌گنجد نه آن را گفتن امکانست
\\
خمش کن همچو عالم باش خموش و مست و سرگردان
&&
وگر او نیست مست مست چرا افتان و خیزانست
\\
\end{longtable}
\end{center}
