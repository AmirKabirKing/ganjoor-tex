\begin{center}
\section*{غزل شماره ۱۵۲۶: از آن باده ندانم چون فنایم}
\label{sec:1526}
\addcontentsline{toc}{section}{\nameref{sec:1526}}
\begin{longtable}{l p{0.5cm} r}
از آن باده ندانم چون فنایم
&&
از آن بی‌جا نمی‌دانم کجایم
\\
زمانی قعر دریایی درافتم
&&
دمی دیگر چو خورشیدی برآیم
\\
زمانی از من آبستن جهانی
&&
زمانی چون جهان خلقی بزایم
\\
چو طوطی جان شکر خاید به ناگه
&&
شوم سرمست و طوطی را بخایم
\\
به جایی درنگنجیدم به عالم
&&
بجز آن یار بی‌جا را نشایم
\\
منم آن رند مست سخت شیدا
&&
میان جمله رندان‌های هایم
\\
مرا گویی چرا با خود نیایی
&&
تو بنما خود که تا با خود بیایم
\\
مرا سایه هما چندان نوازد
&&
که گویی سایه او شد من همایم
\\
بدیدم حسن را سرمست می گفت
&&
بلایم من بلایم من بلایم
\\
جوابش آمد از هر سو ز صد جان
&&
ترایم من ترایم من ترایم
\\
تو آن نوری که با موسی همی‌گفت
&&
خدایم من خدایم من خدایم
\\
بگفتم شمس تبریزی کیی گفت
&&
شمایم من شمایم من شمایم
\\
\end{longtable}
\end{center}
