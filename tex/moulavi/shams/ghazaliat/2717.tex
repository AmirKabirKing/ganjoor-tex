\begin{center}
\section*{غزل شماره ۲۷۱۷: تو جانا بی‌وصالش در چه کاری}
\label{sec:2717}
\addcontentsline{toc}{section}{\nameref{sec:2717}}
\begin{longtable}{l p{0.5cm} r}
تو جانا بی‌وصالش در چه کاری
&&
به دست خویش بی‌وصلش چه داری
\\
همه لافت که زاری‌ها کنم من
&&
به نزد او نیرزد خاک زاری
\\
اگر سنگت ببیند بر تو گرید
&&
که از وصل چه کس گشتی تو عاری
\\
به وصلش مر سما را فخر بودی
&&
به هجرش خاک را اکنون تو عاری
\\
چنان مغرور و سرکش گشته بودی
&&
زمان وصل یعنی یار غاری
\\
از آن می‌ها ز وصلش مست بودی
&&
نک آمد مر تو را دور خماری
\\
ولیکن مرغ دولت مژده آورد
&&
کز آن اقبال می‌آید بهاری
\\
ز لطف و حلم او بوده‌ست آن وصل
&&
نبود از عقل و فرهنگ و عیاری
\\
به پیر هندوی بگذشت لطفش
&&
چو ماهی گشت پیر از خوش عذاری
\\
چنین‌ها دیده‌ای از لطف و حسنش
&&
تو جانا کز پی او بی‌قراری
\\
چه سودم دارد ار صد ملک دارم
&&
که تو که جان آنی در فراری
\\
خداوندی ز تو دور است ای دل
&&
که بی‌او یاوه گشته و بی‌مهاری
\\
هزاران زخم دارد از تو ای هجر
&&
که این دم بر سر گنجش تو ماری
\\
ایا روز فراقم همچو قیری
&&
ایا روز وصالم همچو قاری
\\
تو بودی در وصالش در قماری
&&
کنون تو با خیالش در قماری
\\
به هجر فخر ما شمس الحق و دین
&&
ایا صبرا نکردی هیچ یاری
\\
مگر صبری که رست از خاک تبریز
&&
خورم یابم دمی زو بردباری
\\
ببینا این فراق من فراقی
&&
ببینا بخت لنگم راهواری
\\
\end{longtable}
\end{center}
