\begin{center}
\section*{غزل شماره ۱۱۴۴: ندا رسید به جان‌ها ز خسرو منصور}
\label{sec:1144}
\addcontentsline{toc}{section}{\nameref{sec:1144}}
\begin{longtable}{l p{0.5cm} r}
ندا رسید به جان‌ها ز خسرو منصور
&&
نظر به حلقه مردان چه می‌کنید از دور
\\
چو آفتاب برآمد چه خفته‌اند این خلق
&&
نه روح عاشق روزست و چشم عاشق نور
\\
درون چاه ز خورشید روح روشن شد
&&
ز نور خارش پذرفت نیز دیده کور
\\
بجنب بر خود آخر که چاشتگاه شدست
&&
از آنک خفته چو جنبید خواب شد مهجور
\\
مگو که خفته نیم ناظرم به صنع خدا
&&
نظر به صنع حجابست از چنان منظور
\\
روان خفته اگر داندی که در خوابست
&&
از آنچ دیدی نی خوش شدی و نی رنجور
\\
چنانک روزی در خواب رفت گلخن تاب
&&
به خواب دید که سلطان شدست و شد مغرور
\\
بدید خود را بر تخت ملک وز چپ و راست
&&
هزار صف ز امیر و ز حاجب و دستور
\\
چنان نشسته بر آن تخت او که پنداری
&&
در امر و نهی خداوند بد سنین و شهور
\\
میان غلغله و دار و گیر و بردابرد
&&
میان آن لمن الملک و عزت و شر و شور
\\
درآمد از در گلخن به خشم حمامی
&&
زدش به پای که برجه نه مرده‌ای در گور
\\
بجست و پهلوی خود نی خزینه دید و نه ملک
&&
ولی خزینه حمام سرد دید و نفور
\\
بخوان ز آخر یاسین که صیحه فاذا
&&
تو هم به بانگی حاضر شوی ز خواب غرور
\\
چه خفته‌ایم ولیکن ز خفته تا خفته
&&
هزار مرتبه فرقست ظاهر و مستور
\\
شهی که خفت ز شاهی خود بود غافل
&&
خسی که خفت ز ادبیر خود بود معذور
\\
چو هر دو باز از این خواب خویش بازآیند
&&
به تخت آید شاه و به تخته آن مقهور
\\
لباب قصه بماندست و گفت فرمان نیست
&&
نگر به دانش داوود و کوتهی زبور
\\
مگر که لطف کند باز شمس تبریزی
&&
وگر نه ماند سخن در دهن چنین مقصور
\\
\end{longtable}
\end{center}
