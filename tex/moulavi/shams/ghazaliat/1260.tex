\begin{center}
\section*{غزل شماره ۱۲۶۰: اندرآمد شاه شیرینان ترش}
\label{sec:1260}
\addcontentsline{toc}{section}{\nameref{sec:1260}}
\begin{longtable}{l p{0.5cm} r}
اندرآمد شاه شیرینان ترش
&&
جان شیرینم فدای آن ترش
\\
چشم کژبین را بگفتم کژ مبین
&&
کس کند باور گل خندان ترش
\\
در هر آن زندان که درتابد رخش
&&
کس نماند در همه زندان ترش
\\
گرد باغش گشتم و والله نبود
&&
میوه‌ای اندر همه بستان ترش
\\
در حرم خندان بود سلطان ولیک
&&
می‌نماید خویش در دیوان ترش
\\
گر تو مرد مؤمنی باور مکن
&&
انگبین و شکر و ایمان ترش
\\
منکر ار باشد ترش نبود عجب
&&
نسبتی دارد به بادنجان ترش
\\
\end{longtable}
\end{center}
