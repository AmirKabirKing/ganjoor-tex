\begin{center}
\section*{غزل شماره ۱۶۱۱: مکن ای دوست غریبم سر سودای تو دارم}
\label{sec:1611}
\addcontentsline{toc}{section}{\nameref{sec:1611}}
\begin{longtable}{l p{0.5cm} r}
مکن ای دوست غریبم سر سودای تو دارم
&&
من و بالای مناره که تمنای تو دارم
\\
ز تو سرمست و خمارم خبر از خویش ندارم
&&
سر خود نیز نخارم که تقاضای تو دارم
\\
دل من روشن و مقبل ز چه شد با تو بگویم
&&
که در این آینه دل رخ زیبای تو دارم
\\
مکن ای دوست ملامت بنگر روز قیامت
&&
همه موجم همه جوشم در دریای تو دارم
\\
مشنو قول طبیبان که شکر زاید صفرا
&&
به شکر داروی من کن چه که صفرای تو دارم
\\
هله ای گنبد گردون بشنو قصه‌ام اکنون
&&
که چو تو همره ماهم بر و پهنای تو دارم
\\
بر دربان تو آیم ندهد راه و براند
&&
خبرش نیست که پنهان چه تماشای تو دارم
\\
ز درم راه نباشد ز سر بام و دریچه
&&
ستر الله علینا چه علالای تو دارم
\\
هله دربان عوان خو مدهم راه و سقط گو
&&
چو دفم می زن بر رو دف و سرنای تو دارم
\\
چو دف از سیلی مطرب هنرم بیش نماید
&&
بزن و تجربه می کن همه هیهای تو دارم
\\
هله زین پس نخروشم نکنم فتنه نجوشم
&&
به دلم حکم کی دارد دل گویای تو دارم
\\
\end{longtable}
\end{center}
