\begin{center}
\section*{غزل شماره ۱۴۴۱: بنه ای سبز خنگ من فراز آسمان‌ها سم}
\label{sec:1441}
\addcontentsline{toc}{section}{\nameref{sec:1441}}
\begin{longtable}{l p{0.5cm} r}
بنه ای سبز خنگ من فراز آسمان‌ها سم
&&
که بنوشت آن مه بی‌کیف دعوت نامه‌ای پیشم
\\
روان شد سوی ما کوثر که گنجا نیست ظرف اندر
&&
بدران مشک سقا را بزن سنگی و بشکن خم
\\
یکی آهوی چون جانی برآمد از بیابانی
&&
که شیر نر ز بیم او زند بر ریگ سوزان دم
\\
همه مستیم ای خواجه به روز عید می ماند
&&
دهل مست و دهلزن مست و بیخود می زند لم لم
\\
درآمد عقل در میدان سر انگشت در دندان
&&
که بر سرمست و با حیران چه برخوانیم الهاکم
\\
یکی عاقل میان ما به دار وهم نمی‌یابند
&&
در این زنجیر مجنونان چه مجنون می شود مردم
\\
بر مخمور یک ساغر به از صد خانه پرزر
&&
بریزم بر تن لاغر از آن باده یکی قمقم
\\
میان روزه داران خوش شراب عشق در می کش
&&
نه آن مستی که شب آیی ز شرم خلق چون کزدم
\\
بخور بی‌رطل و بی‌کوزه میی کو نشکند روزه
&&
نه ز انگور است و نه از شیره نه از بکنی نه از گندم
\\
شرابی نی که درریزی سر مخمور برخیزی
&&
دروغین است آن باده از آن افتاد کوته دم
\\
رسید از باده خانه پر به زیر مشک می اشتر
&&
رها کن خواب خراخر که قمقم بانگ زد قم قم
\\
دهان بربند و محرم شو به کعبه خامشان می رو
&&
پیاپی اندر این مستی نه اشتر جو و نی جم جم
\\
\end{longtable}
\end{center}
