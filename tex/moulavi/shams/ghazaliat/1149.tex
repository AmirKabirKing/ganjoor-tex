\begin{center}
\section*{غزل شماره ۱۱۴۹: فغان فغان که ببست آن نگار بار سفر}
\label{sec:1149}
\addcontentsline{toc}{section}{\nameref{sec:1149}}
\begin{longtable}{l p{0.5cm} r}
فغان فغان که ببست آن نگار بار سفر
&&
فغان که بنده مر او را نبود یار سفر
\\
فغان که کار سفر نیست سخره دستم
&&
که تا ز هم بدرم جمله پود و تار سفر
\\
ولیک طالع خورشید و مه سفر باشد
&&
که تاز گردششان سایه شد سوار سفر
\\
سفر بیامد وزان هجر عذرها می‌خواست
&&
بدان زبان که شد این بنده شرمسار سفر
\\
بگفتمش که ز روباه شانگی بگذر
&&
که شیر کرد شکارم به مرغزار سفر
\\
مراست جان مسافر چو آب و من چون جوی
&&
روانه جانب دریا که شد مدار سفر
\\
دود به لب لب این جوی تا لب دریا
&&
دلی که خست در این راه‌ها ز خار سفر
\\
به روی آینه بنگر که از سفر آمد
&&
صفا نگر تو به رویش از آن غبار سفر
\\
سفر سفر چو چنان یار غار در سفرست
&&
تو بخت بخت سفر دان و کار کار سفر
\\
همیشه چشم گشایم چو غنچه بر سر راه
&&
چو سرو روح روانست در بهار سفر
\\
چو شمس مفخر تبریز در سفر افتاد
&&
چه مملکت که بگسترد در دوار سفر
\\
\end{longtable}
\end{center}
