\begin{center}
\section*{غزل شماره ۲۸۶۸: در دلت چیست عجب که چو شکر می‌خندی}
\label{sec:2868}
\addcontentsline{toc}{section}{\nameref{sec:2868}}
\begin{longtable}{l p{0.5cm} r}
در دلت چیست عجب که چو شکر می‌خندی
&&
دوش شب با کی بدی که چو سحر می‌خندی
\\
ای بهاری که جهان از دم تو خندان است
&&
در سمن زار شکفتی چو شجر می‌خندی
\\
آتشی از رخ خود در بت و بتخانه زدی
&&
و اندر آتش بنشستی و چو زر می‌خندی
\\
مست و خندان ز خرابات خدا می‌آیی
&&
بر شر و خیر جهان همچو شرر می‌خندی
\\
همچو گل ناف تو بر خنده بریده‌ست خدا
&&
لیک امروز مها نوع دگر می‌خندی
\\
باغ با جمله درختان ز خزان خشک شدند
&&
ز چه باغی تو که همچون گل‌تر می‌خندی
\\
تو چو ماهی و عدو سوی تو گر تیر کشد
&&
چو مه از چرخ بر آن تیر و سپر می‌خندی
\\
بوی مشکی تو که بر خنگ هوا می‌تازی
&&
آفتابی تو که بر قرص قمر می‌خندی
\\
تو یقینی و عیان بر ظن و تقلید بخند
&&
نظری جمله و بر نقل و خبر می‌خندی
\\
در حضور ابدی شاهد و مشهود تویی
&&
بر ره و ره رو و بر کوچ و سفر می‌خندی
\\
از میان عدم و محو برآوردی سر
&&
بر سر و افسر و بر تاج و کمر می‌خندی
\\
چون سگ گرسنه هر خلق دهان بگشاده‌ست
&&
تویی آن شیر که بر جوع بقر می‌خندی
\\
آهوان را ز دمت خون جگر مشک شده‌ست
&&
رحمت است آنک تو بر خون جگر می‌خندی
\\
آهوان را به گه صید به گردون گیری
&&
ای که بر دام و دم شعبده گر می‌خندی
\\
دو سه بیتی که بمانده‌ست بگو مستانه
&&
ای که تو بر دل بی‌زیر و زبر می‌خندی
\\
\end{longtable}
\end{center}
