\begin{center}
\section*{غزل شماره ۲۸۵۹: برو ای عشق که تا شحنه خوبان شده‌ای}
\label{sec:2859}
\addcontentsline{toc}{section}{\nameref{sec:2859}}
\begin{longtable}{l p{0.5cm} r}
برو ای عشق که تا شحنه خوبان شده‌ای
&&
توبه و توبه کنان را همه گردن زده‌ای
\\
کی شود با تو معول که چنین صاعقه‌ای
&&
کی کند با تو حریفی که همه عربده‌ای
\\
نی زمین و نه فلک را قدم و طاقت توست
&&
نه در این شش جهتی پس ز کجا آمده‌ای
\\
هشت جنت به تو عاشق تو چه زیبا رویی
&&
هفت دوزخ ز تو لرزان تو چه آتشکده‌ای
\\
دوزخت گوید بگذر که مرا تاب تو نیست
&&
جنت جنتی و دوزخ دوزخ بده‌ای
\\
چشم عشاق ز چشم خوش تو تردامن
&&
فتنه و رهزن هر زاهد و هر زاهده‌ای
\\
بی تو در صومعه بودن به جز از سودا نیست
&&
ز آنک تو زندگی صومعه و معبده‌ای
\\
دل ویران مرا داد ده ای قاضی عشق
&&
که خراج از ده ویران دلم بستده‌ای
\\
ای دل ساده من داد ز کی می‌خواهی
&&
خون مباح است بر عشق اگر زین رده‌ای
\\
داد عشاق ز اندازه جان بیرون است
&&
تو در اندیشه و در وسوسه بیهده‌ای
\\
جز صفات ملکی نیست یقین محرم عشق
&&
تو گرفتار صفات خر و دیو و دده‌ای
\\
بس کن و سحر مکن اول خود را برهان
&&
که اسیر هوس جادویی و شعبده‌ای
\\
\end{longtable}
\end{center}
