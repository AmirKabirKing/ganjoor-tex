\begin{center}
\section*{غزل شماره ۲۵۱۵: یکی گنجی پدید آمد در آن دکان زرکوبی}
\label{sec:2515}
\addcontentsline{toc}{section}{\nameref{sec:2515}}
\begin{longtable}{l p{0.5cm} r}
یکی گنجی پدید آمد در آن دکان زرکوبی
&&
زهی صورت زهی معنی زهی خوبی زهی خوبی
\\
زهی بازار زرکوبان زهی اسرار یعقوبان
&&
که جان یوسف از عشقش برآرد شور یعقوبی
\\
ز عشق او دو صد لیلی چو مجنون بند می‌درد
&&
کز این آتش زبون آید صبوری‌های ایوبی
\\
شده زرکوب و حق مانده تنش چون زرورق مانده
&&
جواهر بر طبق مانده چو زرکوبی کروبی
\\
بیا بنواز عاشق را که تو جانی حقایق را
&&
بزن گردن منافق را اگر از وی بیاشوبی
\\
\end{longtable}
\end{center}
