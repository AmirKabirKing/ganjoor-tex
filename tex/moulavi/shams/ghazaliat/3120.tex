\begin{center}
\section*{غزل شماره ۳۱۲۰: اگر چه لطیفی و زیبالقایی}
\label{sec:3120}
\addcontentsline{toc}{section}{\nameref{sec:3120}}
\begin{longtable}{l p{0.5cm} r}
اگر چه لطیفی و زیبالقایی
&&
به جان بقا رو ز جان هوایی
\\
هوا گاه سردست و گه گرم و سوزان
&&
وفا زو چه جویی ببین بی‌وفایی
\\
بدن را قفس دان و جان مرغ پران
&&
قفس حاضر آمد تو جانا کجایی
\\
در آفاق گردون زمانی پریدی
&&
گذشتی بدان شه که او را سزایی
\\
جهان چون تو مرغی ندید و نبیند
&&
که هم فوق بامی و هم در سرایی
\\
گهی پا زنی بر سر تاجداران
&&
گهی درروی در پلاس گدایی
\\
گهی آفتابی بتابی جهان را
&&
گهی همچو برقی زمانی نپایی
\\
تو کان نباتی و دل‌ها چو طوطی
&&
تو صحرای سبزی و جان‌ها چرایی
\\
از این‌ها گذشتم مبر سایه از ما
&&
که در باغ دولت گل و سرو مایی
\\
اگر بر دل ما دو صد قفل باشد
&&
کلیدی فرستی و در را گشایی
\\
درآ در دل ما که روشن چراغی
&&
درآ در دو دیده که خوش توتیایی
\\
اگر لشکر غم سیاهی درآرد
&&
تو خورشید رزمی و صاحب لوایی
\\
شدم در گلستان و با گل بگفتم
&&
جهاز از کی داری که لعلین قبایی
\\
مرا گفت بو کن به بو خود شناسی
&&
چو مجنون عشقی و صاحب صفایی
\\
چو مجنون بیامد به وادی لیلی
&&
که یابد نسیمش ز باد صبایی
\\
بگفتند لیلی شما را بقا باد
&&
ببین بر تبارش لباس عزایی
\\
پس آن تلخکامه بدرید جامه
&&
بغلطید در خون ز بی‌دست و پایی
\\
همی‌کوفت سر را به هر سنگ و هر در
&&
بسی کرد نوحه بسی دست خایی
\\
همی‌کوفت بر سر که تاجت کجا شد
&&
همی‌کوفت بر دل که صید بلایی
\\
درازست قصه تو خود این بدانی
&&
تپش‌های ماهی ز بی‌استقایی
\\
چو با خویش آمد بپرسید مجنون
&&
که گورش نشان ده که بادش فضایی
\\
بگفتند شب بود و تاریک و گم شد
&&
بس افتد از این‌ها ز س القضایی
\\
ندا کرد مجنون قلاوز دارم
&&
مرا بوی لیلی کند ره نمایی
\\
چو یعقوب وقتم یقین بوی یوسف
&&
ز صدساله راهم رساند دوایی
\\
مشام محمد به ما داد صله
&&
کشیم از یمن خوش نسیم خدایی
\\
ز هر گور کف کف همی‌برد خاکی
&&
به بینی و می‌جست از آن مشک سایی
\\
مثال مریدی که او شیخ جوید
&&
کشد از دهان‌ها دم اولیایی
\\
بجو بوی حق از دهان قلندر
&&
به جد چون بجویی یقین محرم آیی
\\
ز جرعه‌ست آن بو نه از خاک تیره
&&
که در خاک افتاد جرعه ولایی
\\
به مجنون تو بازآ و این را رها کن
&&
که شد خیره چشمم ز شمس ضیایی
\\
ضعیفست در قرص خورشید چشمم
&&
ولی مه دهد بر شعاعش گوایی
\\
کجا عشق ذوالنون کجا عشق مجنون
&&
ولی این نشانست از کبریایی
\\
چو موسی که نگرفت پستان دایه
&&
که با شیر مادر بدش آشنایی
\\
ز صد گور بو کرد مجنون و بگذشت
&&
که در بوشناسی بدش اوستایی
\\
چراغیست تمییز در سینه روشن
&&
رهاند تو را از فریب و دغایی
\\
بیاورد بویش سوی گور لیلی
&&
بزد نعره و اوفتاد آن فنایی
\\
همان بو شکفتش همان بو بکشتش
&&
به یک نفخه حشری به یک نفخه لایی
\\
به لیلی رسید او به مولی رسد جان
&&
زمین شد زمینی سما شد سمایی
\\
شما را هوای خدای است لیکن
&&
خدا کی گذارد شما را شمایی
\\
گروهی ز پشه که جویند صرصر
&&
بود جذب صرصر که کرد اقتضایی
\\
که صرصر به پشه دل شیر بخشد
&&
رهاند ز خویشش به حسن الجزایی
\\
بیان کردمی رونق لاله زارش
&&
ولی برنتابد دل لالکایی
\\
چمن خود بگوید تو را بی‌زبانی
&&
صلا در چمن رو که اصل صلایی
\\
\end{longtable}
\end{center}
