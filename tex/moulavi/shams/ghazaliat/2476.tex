\begin{center}
\section*{غزل شماره ۲۴۷۶: هین که خروس بانگ زد وقت صبوح یافتی}
\label{sec:2476}
\addcontentsline{toc}{section}{\nameref{sec:2476}}
\begin{longtable}{l p{0.5cm} r}
هین که خروس بانگ زد وقت صبوح یافتی
&&
شرح نمی‌کنم که بس عاقل را اشارتی
\\
فهم کنی تو خود که تو زیرک و پاک خاطری
&&
باده بیار و دل ببر زود بکن تجارتی
\\
نای بنه دهان همی‌آرد صبح ناله‌ای
&&
چنگ ز چنگ هجر تو کرد حزین شکایتی
\\
درده بی‌دریغ از آن شیره و شیر رایگان
&&
شیر و نبید خلد را نیست حدی و غایتی
\\
درده باده‌ای چو زر پاک ز خویشمان ببر
&&
نیست بتر ز باخودی مذهب ما جنایتی
\\
باده شاد جان فزا تحفه بیار از سما
&&
تا غم و غصه را کند اشقر می سیاستی
\\
عقل ز نقل تو شود منتقل از عقیله‌ها
&&
دانش غیب یابد و تبصره و فراستی
\\
جام تو را چو دل بود در سر و سینه شعله‌ای
&&
مست تو را چه کم بود تجربه یا کفایتی
\\
دست که یافت مشربی ماند ز حرص و مکسبی
&&
سر که بیافت آن طرب کی طلبد ریاستی
\\
شست تو ماهی مرا چله نشاند مدتی
&&
دام تو کرکس مرا داد به غم ریاضتی
\\
قطره ز بحر فضل تو یافت عجب تبدلی
&&
پاکدلی و صفوتی توسعه و احاطتی
\\
نفس خسیس حرص خو عاشق مال و گفت و گو
&&
یافت به گنج رحمتت از دو جهان فراغتی
\\
ترک زیارتت شها دان ز خری نه بی‌خری
&&
ز آنک به جان است متصل حج تو بی‌مسافتی
\\
هیچ مگو دلا هلا طاقت رنج نیستم
&&
طاق شو از فضول خود حاجت نیست طاقتی
\\
طاقت رنج هر کسی داری و می‌کشی بسی
&&
طاقت گنج نیستت این چه بود خساستی
\\
سر دل تو جز ولا تا نبود که بی‌گمان
&&
بر سر بینیت کند سر دلت علامتی
\\
حشر شود ضمیر تو در سخن و صفیر تو
&&
نقد شود در این جهان عرض تو را قیامتی
\\
از بد و نیک مجرمان کند نشد وفای تو
&&
ز آنک تو راست در کرم ثابتی و مهارتی
\\
جان و دل مرید را از شهوات ما و من
&&
جز ز زلال بحر تو نیست یقین طهارتی
\\
متقیان به بادیه رفته عشا و غادیه
&&
کعبه روان شده به تو تا که کند زیارتی
\\
روح سجود می‌کند شکر وجود می‌کند
&&
یافت ز بندگی تو سروری و سیادتی
\\
بر کرم و کرامت خنده آفتاب تو
&&
ذره به ذره را بود نوع دگر شهادتی
\\
جمله به جست و جوی تو معتکفان کوی تو
&&
روی به کعبه کرم مشتغل عبادتی
\\
پنج حس از مصاحف نور و حیات جامعت
&&
یاد گرفته ز اوستا ظاهر پنج آیتی
\\
گاه چو چنگ می‌کند پیش درت رکوع خوش
&&
گاه چو نای می‌کند بهر دم تو قامتی
\\
بس کن ای خرد از این ناله و قصه حزین
&&
بوی برد به خامشی هر دل باشهامتی
\\
\end{longtable}
\end{center}
