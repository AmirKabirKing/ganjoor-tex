\begin{center}
\section*{غزل شماره ۶۴۴: تا نقش تو در سینه ما خانه نشین شد}
\label{sec:0644}
\addcontentsline{toc}{section}{\nameref{sec:0644}}
\begin{longtable}{l p{0.5cm} r}
تا نقش تو در سینه ما خانه نشین شد
&&
هر جا که نشینیم چو فردوس برین شد
\\
آن فکر و خیالات چو یأجوج و چو مأجوج
&&
هر یک چو رخ حوری و چون لعبت چین شد
\\
آن نقش که مرد و زن از او نوحه کنانند
&&
گر بئس قرین بود کنون نعم قرین شد
\\
بالا همه باغ آمد و پستی همگی گنج
&&
آخر تو چه چیزی که جهان از تو چنین شد
\\
زان روز که دیدیمش ما روزفزونیم
&&
خاری که ورا جست گلستان یقین شد
\\
هر غوره ز خورشید شد انگور و شکر بست
&&
وان سنگ سیه نیز از او لعل ثمین شد
\\
بسیار زمین‌ها که به تفصیل فلک شد
&&
بسیار یسار از کف اقبال یمین شد
\\
گر ظلمت دل بود کنون روزن دل شد
&&
ور رهزن دین بود کنون قدوه دین شد
\\
گر چاه بلا بود که بد محبس یوسف
&&
از بهر برون آمدنش حبل متین شد
\\
هر جزو چو جندالله محکوم خداییست
&&
بر بنده امان آمد و بر گبر کمین شد
\\
خاموش که گفتار تو ماننده نیلست
&&
بر قبط چو خون آمد و بر سبط معین شد
\\
خاموش که گفتار تو انجیر رسیدست
&&
اما نه همه مرغ هوا درخور تین شد
\\
\end{longtable}
\end{center}
