\begin{center}
\section*{غزل شماره ۹۰۷: مده به دست فراقت دل مرا که نشاید}
\label{sec:0907}
\addcontentsline{toc}{section}{\nameref{sec:0907}}
\begin{longtable}{l p{0.5cm} r}
مده به دست فراقت دل مرا که نشاید
&&
مکش تو کشته خود را مکن بتا که نشاید
\\
مرا به لطف گزیدی چرا ز من برمیدی
&&
ایا نموده وفاها مکن جفا که نشاید
\\
بداد خازن لطفت مرا قبای سعادت
&&
برون مکن ز تن من چنین قبا که نشاید
\\
مثال دل همه رویی قفا نباشد دل را
&&
ز ما تو روی مگردان مده قفا که نشاید
\\
حدیث وصل تو گفتم بگفت لطف تو کری
&&
ز بعد گفتن آری مگو چرا که نشاید
\\
تو کان قند و نباتی نبات تلخ نگوید
&&
مگوی تلخ سخن‌ها به روی ما که نشاید
\\
بیار آن سخنانی که هر یکیست چو جانی
&&
نهان مکن تو در این شب چراغ را که نشاید
\\
غمت که کاهش تن شد نه در تنست نه بیرون
&&
غم آتشیست نه در جا مگو کجا که نشاید
\\
دلم ز عالم بی‌چون خیالت از دل از آن سو
&&
میان این دو مسافر مکن جدا که نشاید
\\
مبند آن در خانه به صوفیان نظری کن
&&
مخور به رنج به تنها بگو صلا که نشاید
\\
دلا بخسب ز فکرت که فکر دام دل آمد
&&
مرو به جز که مجرد بر خدا که نشاید
\\
\end{longtable}
\end{center}
