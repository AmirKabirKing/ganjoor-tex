\begin{center}
\section*{غزل شماره ۱۵۷۰: آن عشرت نو که برگرفتیم}
\label{sec:1570}
\addcontentsline{toc}{section}{\nameref{sec:1570}}
\begin{longtable}{l p{0.5cm} r}
آن عشرت نو که برگرفتیم
&&
پا دار که ما ز سر گرفتیم
\\
آن دلبر خوب باخبر را
&&
مست و خوش و بی‌خبر گرفتیم
\\
هر لحظه ز حسن یوسف خود
&&
صد مصر پر از شکر گرفتیم
\\
در خانه حسن بود ماهی
&&
رفتیمش و بام و در گرفتیم
\\
آن آب حیات سرمدی را
&&
چون آب در این جگر گرفتیم
\\
چون گوشه تاج او بدیدیم
&&
مستانه‌اش از کمر گرفتیم
\\
هر نقش که بی‌وی است مرده‌ست
&&
از بهر تو جانور گرفتیم
\\
هر جانوری که آن ندارد
&&
او را علف سقر گرفتیم
\\
هر کس گهری گرفت از کان
&&
از کان همه سیمبر گرفتیم
\\
از تابش نور آفتابی
&&
چون ماه جمال و فر گرفتیم
\\
شمس تبریز چون سفر کرد
&&
چون ماه از آن سفر گرفتیم
\\
\end{longtable}
\end{center}
