\begin{center}
\section*{غزل شماره ۶۱۴: آن بنده آواره بازآمد و بازآمد}
\label{sec:0614}
\addcontentsline{toc}{section}{\nameref{sec:0614}}
\begin{longtable}{l p{0.5cm} r}
آن بنده آواره بازآمد و بازآمد
&&
چون شمع به پیش تو در سوز و گداز آمد
\\
چون عبهر و قند ای جان در روش بخند ای جان
&&
در را بمبند ای جان زیرا به نیاز آمد
\\
ور زانک ببندی در بر حکم تو بنهد سر
&&
بر بنده نیاز آمد شه را همه ناز آمد
\\
هر شمع گدازیده شد روشنی دیده
&&
کان را که گداز آمد او محرم راز آمد
\\
زهراب ز دست وی گر فرق کنم از می
&&
پس در ره جان جانم والله به مجاز آمد
\\
آب حیوانش را حیوان ز کجا نوشد
&&
کی بیند رویش را چشمی که فرازآمد
\\
من ترک سفر کردم با یار شدم ساکن
&&
وز مرگ شدم ایمن کان عمر دراز آمد
\\
ای دل چو در این جویی پس آب چه می‌جویی
&&
تا چند صلا گویی هنگام نماز آمد
\\
\end{longtable}
\end{center}
