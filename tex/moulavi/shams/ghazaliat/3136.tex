\begin{center}
\section*{غزل شماره ۳۱۳۶: خواهی ز جنون بویی ببری}
\label{sec:3136}
\addcontentsline{toc}{section}{\nameref{sec:3136}}
\begin{longtable}{l p{0.5cm} r}
خواهی ز جنون بویی ببری
&&
ز اندیشه و غم می‌باش بری
\\
تا تنگ دلی از بهر قبا
&&
جانت نکند زرین کمری
\\
کی عشق تو را محرم شمرد
&&
تا همچو خسان زر می‌شمری
\\
فوق همه‌ای چون نور شوی
&&
تا نور نه‌ای در زیر دری
\\
هیزم بود آن چوبی که نسوخت
&&
چون سوخته شد باشد شرری
\\
وانگه شررش وا اصل رود
&&
همچون شرر جان بشری
\\
سرمه بود آن کز چشم جداست
&&
در چشم رود گردد نظری
\\
یک قطره بود در ابر گران
&&
در بحر فتد یابد گهری
\\
خار سیهی بد سوختنی
&&
گردش گل تر باد سحری
\\
یک لقمه نان چون کوفته شد
&&
جان گشت و کند نان جانوری
\\
خون گشت غذا در پیشه وری
&&
آن لقمه کند هم پیشه وری
\\
گر زانک بلا کوبد دل تو
&&
از عین بلانوشی بچری
\\
ور زانک اجل کوبد سر تو
&&
دانی پس از آن که جمله سری
\\
در بیضه تن مرغ عجبی
&&
در بیضه دری ز آن می‌نپری
\\
گر بیضه تن سوراخ شود
&&
هم پر بزنی هم جان ببری
\\
سودای سفر از ذکر بود
&&
از ذکر شود مردم سفری
\\
تو در حضری وین وهم سفر
&&
پنداشت توست از بی‌هنری
\\
یا رب برهان زین وهم کژش
&&
تو وهم نهی در دیو و پری
\\
چون در حضری بربند دهان
&&
در ذکر مرو چون در حضری
\\
\end{longtable}
\end{center}
