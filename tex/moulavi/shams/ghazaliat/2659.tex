\begin{center}
\section*{غزل شماره ۲۶۵۹: کسی کو را بود در طبع سستی}
\label{sec:2659}
\addcontentsline{toc}{section}{\nameref{sec:2659}}
\begin{longtable}{l p{0.5cm} r}
کسی کو را بود در طبع سستی
&&
نخواهد هیچ کس را تندرستی
\\
مده دامن به دستان حسودان
&&
که ایشان می‌کشندت سوی پستی
\\
زیانتر خویش را و دیگران را
&&
نباشد چون حسد در جمله هستی
\\
هلا بشکن دل و دام حسودان
&&
وگر نی پشت بخت خود شکستی
\\
از این اخوان چو ببریدی چو یوسف
&&
عزیز مصری و از گرگ رستی
\\
اگر حاسد دو پایت را ببوسد
&&
به باطن می‌زند خنجر دودستی
\\
ندارد مهر مهره او چه گشتی
&&
ندارد دل دل اندر وی چه بستی
\\
اگر در حصن تقوا راه یابی
&&
ز حاسد وز حسد جاوید رستی
\\
اگر چه شیرگیری ترک او کن
&&
نه آن شیر است کش گیری به مستی
\\
\end{longtable}
\end{center}
