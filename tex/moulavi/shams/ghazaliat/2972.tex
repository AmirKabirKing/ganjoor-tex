\begin{center}
\section*{غزل شماره ۲۹۷۲: ای آن که مر مرا تو به از جان و دیده‌ای}
\label{sec:2972}
\addcontentsline{toc}{section}{\nameref{sec:2972}}
\begin{longtable}{l p{0.5cm} r}
ای آن که مر مرا تو به از جان و دیده‌ای
&&
در جان من هر آنچ ندیدم تو دیده‌ای
\\
بگزیده‌ام ز هجر تو تابوت آتشین
&&
آری به حق آنک مرا تو گزیده‌ای
\\
گر از بریده خون چکد اینک ز چشم من
&&
خون می‌چکد که بی‌سبب از من بریده‌ای
\\
از چشم من بپرس چرا چشمه گشته‌ای
&&
وز قد من بپرس که از کی خمیده‌ای
\\
از جان من بپرس که با کفش آهنین
&&
اندر ره فراق کجاها رسیده‌ای
\\
این هم بپرس از او که تو در حسن و در جمال
&&
مانند او ز هیچ زبانی شنیده‌ای
\\
این هم بگو که گر رخ او آفتاب نیست
&&
چون ابر پاره پاره ز هم چون دریده‌ای
\\
پیداست در دم تو که از ناف مشک خاست
&&
کاندر کدام سبزه و صحرا چریده‌ای
\\
آنی که دیده‌ای تو دلا آسمانیی
&&
زیرا ز دلبران زمینی رمیده‌ای
\\
دانم که دیده‌ای تو بدین چشم یوسفی
&&
تا تو ترنج و دست ز مستی بریده‌ای
\\
تبریز و شمس دین و دگرها بهانه‌هاست
&&
کز وی دو کون را تو خطی درکشیده‌ای
\\
\end{longtable}
\end{center}
