\begin{center}
\section*{غزل شماره ۲۸۹۲: ای شه جاودانی وی مه آسمانی}
\label{sec:2892}
\addcontentsline{toc}{section}{\nameref{sec:2892}}
\begin{longtable}{l p{0.5cm} r}
ای شه جاودانی وی مه آسمانی
&&
چشمه زندگانی گلشن لامکانی
\\
تا زلال تو دیدم قصه جان شنیدم
&&
همچو جان ناپدیدم در تک بی‌نشانی
\\
عاشق مشک خوش بو می‌کند صید آهو
&&
می‌رود مست هر سو یا تواش می‌دوانی
\\
ای شکر بنده تو زان شکرخنده تو
&&
ای جهان زنده از تو غرقه زندگانی
\\
روز شد های مستان بشنوید از گلستان
&&
می‌کند مرغ دستان شیوه دلستانی
\\
شیوه یاسمین کن سر بجنبان چنین کن
&&
خانه پرانگبین کن چون شکر می‌فشانی
\\
نرگست مست گشته جنیی یا فرشته
&&
با شکر درسرشته غنچه گلستانی
\\
با چنین ساقی حق با خودی کفر مطلق
&&
می‌زند جان معلق با می رایگانی
\\
روز و شب ای برادر مست و بی‌خویش خوشتر
&&
مست الله اکبر کش نبوده است ثانی
\\
نام او جان جان‌ها یاد او لعل کان‌ها
&&
عشق او در روان‌ها هم امان هم امانی
\\
چون برم نام او را دررسد بخت خضرا
&&
اسم شد پس مسما بی‌دوی بی‌توانی
\\
چند مستند پنهان اندر این سبز میدان
&&
می‌روم سوی ایشان با تو گفتم تو دانی
\\
جان ویسند و رامین سخت شیرین شیرین
&&
مفخر آل یاسین وز خدا ارمغانی
\\
تو اگر می‌شتابی سوی مرغان آبی
&&
آب حیوان بیابی قلزم شادمانی
\\
چرب و شیرین بخوردی عیش و عشرت بکردی
&&
سوی عشق آی یک شب هم ببین میزبانی
\\
ما هم از بامدادان بیخود و مست و شادان
&&
ای شه بامرادان مستمان می‌کشانی
\\
با ظریفان و خوبان تا به شب پای کوبان
&&
وز می پیر رهبان هر دمی دوستگانی
\\
این قدح می شتابد تا شما را بیابد
&&
در دل و جان بتابد از ره بی‌دهانی
\\
ای که داری تو فهمی قبض کن قبض اعمی
&&
غیر این نیست چیزی تو مباش امتحانی
\\
غیر این نیست راهی غیر این نیست شاهی
&&
غیر این نیست ماهی غیر این جمله فانی
\\
نی خمش کن خمش کن رو به قاصد ترش کن
&&
ترک اصحاب هش کن باده خور در نهانی
\\
\end{longtable}
\end{center}
