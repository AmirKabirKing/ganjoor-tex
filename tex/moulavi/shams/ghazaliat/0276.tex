\begin{center}
\section*{غزل شماره ۲۷۶: یا خفی الحسن بین الناس یا نور الدجی}
\label{sec:0276}
\addcontentsline{toc}{section}{\nameref{sec:0276}}
\begin{longtable}{l p{0.5cm} r}
یا خفی الحسن بین الناس یا نور الدجی
&&
انت شمس الحق تخفی بین شعشاع الضحی
\\
کاد رب العرش یخفی حسنه من نفسه
&&
غیره منه علی ذاک الکمال المنتهی
\\
لیتنی یوما اخر میتا فی فیه
&&
ان فی موتی هناک دوله لا ترتجی
\\
فی غبار نعله کحل یجلی عن عمی
&&
فی عیون فضله الوافی زلال للظما
\\
غیر ان السیر و النقلان فی ذاک الهوی
&&
مشکل صعب مخوف فیه اهراق الدما
\\
نوره یهدی الی قصر رفیع آمن
&&
لا ابالی من ضلال فیه لی هذا الهدی
\\
ابشری یا عین من اشراق نور شامل
&&
ما علیک من ضریر سرمدی لا یری
\\
اصبحت تبریز عندی قبله او مشرقا
&&
ساعه اضحی لنور ساعه ابغی الصلا
\\
ایها الساقی ادر کاس البقا من حبه
&&
طال ما بتنا مریضا نبتغی هذا الشفا
\\
لا نبالی من لیال شیبتنا برهه
&&
بعد ما صرنا شبابا من رحیق دائما
\\
ایها الصاحون فی ایامه تعسا لکم
&&
اشربوا اخواننا من کاسه طوبی لنا
\\
حصحص الحق الحقیق المستضی من فضله
&&
سوف یهدی الناس من ظلماتهم نحو الفضا
\\
یا لها من سؤ حظ معرض عن فضله
&&
منکر مستکبر حیران فی وادی الردی
\\
معرض عن عین هدل مستدیم للبقا
&&
طالب للماء فی وسواس یوم للکری
\\
عین بحر فجرت من ارض تبریز لها
&&
ارض تبریز فداک روحنا نعم الثری
\\
\end{longtable}
\end{center}
