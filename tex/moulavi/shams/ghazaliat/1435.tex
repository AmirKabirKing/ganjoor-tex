\begin{center}
\section*{غزل شماره ۱۴۳۵: به گرد دل همی‌گردی چه خواهی کرد می دانم}
\label{sec:1435}
\addcontentsline{toc}{section}{\nameref{sec:1435}}
\begin{longtable}{l p{0.5cm} r}
به گرد دل همی‌گردی چه خواهی کرد می دانم
&&
چه خواهی کرد دل را خون و رخ را زرد می دانم
\\
یکی بازی برآوردی که رخت دل همه بردی
&&
چه خواهی بعد از این بازی دگر آورد می دانم
\\
به یک غمزه جگر خستی پس آتش اندر او بستی
&&
بخواهی پخت می بینم بخواهی خورد می دانم
\\
به حق اشک گرم من به حق آه سرد من
&&
که گرمم پرس چون بینی که گرم از سرد می دانم
\\
مرا دل سوزد و سینه تو را دامن ولی فرق است
&&
که سوز از سوز و دود از دود و درد از درد می دانم
\\
به دل گویم که چون مردان صبوری کن دلم گوید
&&
نه مردم نی زن ار از غم ز زن تا مرد می دانم
\\
دلا چون گرد برخیزی ز هر بادی نمی‌گفتی
&&
که از مردی برآوردن ز دریا گرد می دانم
\\
جوابم داد دل کان مه چو جفت و طاق می بازد
&&
چو ترسا جفت گویم گر ز جفت و فرد می دانم
\\
چو در شطرنج شد قایم بریزد نرد شش پنجی
&&
بگویم مات غم باشم اگر این نرد می دانم
\\
\end{longtable}
\end{center}
