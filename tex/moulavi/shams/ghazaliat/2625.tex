\begin{center}
\section*{غزل شماره ۲۶۲۵: گر علم خرابات تو را همنفسستی}
\label{sec:2625}
\addcontentsline{toc}{section}{\nameref{sec:2625}}
\begin{longtable}{l p{0.5cm} r}
گر علم خرابات تو را همنفسستی
&&
این علم و هنر پیش تو باد و هوسستی
\\
ور طایر غیبی به تو بر سایه فکندی
&&
سیمرغ جهان در نظر تو مگسستی
\\
گر کوکبه شاه حقیقت بنمودی
&&
این کوس سلاطین بر تو چون جرسستی
\\
گر صبح سعادت به تو اقبال نمودی
&&
کی دامن و ریش تو به دست عسسستی
\\
گر پیش روان بر تو عنایت فکنندی
&&
فکری که به پیش دل توست آن سپسستی
\\
معکوس شنو گر نبدی گوش دل تو
&&
از دفتر عشاق یکی حرف بسستی
\\
گوید همه مردند یکی بازنیامد
&&
بازآمده دیدی اگر آن گیج کسستی
\\
لرزان لهب جان تو از صرصر مرگ است
&&
لرزان نبدی گر ز بقا مقتبسستی
\\
همراه خسان گر نبدی طبع خسیست
&&
در حلق تو این شربت فانی چو خسستی
\\
طفل خرد تو به تبارک برسیدی
&&
در مکتب شادی ز کجا در عبسستی
\\
خاموش که این‌ها همه موقوف به وقت است
&&
گر وقت بدی داعیه فریادرسستی
\\
\end{longtable}
\end{center}
