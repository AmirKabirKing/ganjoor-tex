\begin{center}
\section*{غزل شماره ۲۹۸۵: مه طلعتی و شهره قبایی بدیده‌ای}
\label{sec:2985}
\addcontentsline{toc}{section}{\nameref{sec:2985}}
\begin{longtable}{l p{0.5cm} r}
مه طلعتی و شهره قبایی بدیده‌ای
&&
خوبی و آتشی و بلایی بدیده‌ای
\\
چشمی که مستتر کند از صد هزار می
&&
چشمی لطیفتر ز صبایی بدیده‌ای
\\
دولت شفاست مر همه را وز هوای او
&&
دولت پیش دوان که شفایی بدیده‌ای
\\
سایه هماست فتنه شاهان و این هما
&&
جویای شاه تا که همایی بدیده‌ای
\\
ای چرخ راست گو که در این گردش آن چنان
&&
خورشیدرو و ماه لقایی بدیده‌ای
\\
ای دل فنا شدی تو در این عشق یا مگر
&&
در عین این فنا تو بقایی بدیده‌ای
\\
هر گریه خنده جوید و امروز خنده‌ها
&&
با چشم لابه گر که بکایی بدیده‌ای
\\
جان را وباست هجر تو سوزان آن لطف
&&
مهلکتر از فراق وبایی بدیده‌ای
\\
تو خاک آن جفا شده‌ای وین گزاف نیست
&&
در زیر این جفا تو وفایی بدیده‌ای
\\
شاهی شنیده‌ای چو خداوند شمس دین
&&
تبریز مثل شاه تو جایی بدیده‌ای
\\
\end{longtable}
\end{center}
