\begin{center}
\section*{غزل شماره ۱۷۰۸: ای گوش من گرفته تویی چشم روشنم}
\label{sec:1708}
\addcontentsline{toc}{section}{\nameref{sec:1708}}
\begin{longtable}{l p{0.5cm} r}
ای گوش من گرفته تویی چشم روشنم
&&
باغم چه می بری چو تویی باغ و گلشنم
\\
عمری است کز عطای تو من طبل می خورم
&&
در سایه لوای کرم طبل می زنم
\\
می مالم این دو چشم که خواب است یا خیال
&&
باور نمی‌کنم عجب ای دوست کاین منم
\\
آری منم ولیک برون رفته از منی
&&
چون ماه نو ز بدر تو باریک می تنم
\\
در تاج خسروان به حقارت نظر کنم
&&
تا شوق روی توست مها طوق گردنم
\\
با ماهیان ز بحر تو من نزل می خورم
&&
با خاکیان ز رشک تو چون آب و روغنم
\\
گر چه ز بحر صنعت من آب خوردنی است
&&
چون ماهیم نبیند کس آب خوردنم
\\
گر ناخن جفا بخراشد رگ مرا
&&
من خوش صدا چو چنگ ز آسیب ناخنم
\\
خود پی ببرده‌ای تو که رگ دار نیستم
&&
گر می جهد رگی بنما تاش برکنم
\\
گفتی چه کار داری بر نیست کار نیست
&&
گر نیست نیستم ز چه شد نیست مسکنم
\\
نفخ قیامتی تو و من شخص مرده‌ام
&&
تا جان نوبهاری و من سرو و سوسنم
\\
من نیم کاره گفتم باقیش تو بگو
&&
تو عقل عقل عقلی و من سخت کودنم
\\
من صورتی کشیدم جان بخشی آن توست
&&
تو جان جان جانی و من قالب تنم
\\
\end{longtable}
\end{center}
