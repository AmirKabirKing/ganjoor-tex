\begin{center}
\section*{غزل شماره ۲۰۳: آمد بهار خرم آمد نگار ما}
\label{sec:0203}
\addcontentsline{toc}{section}{\nameref{sec:0203}}
\begin{longtable}{l p{0.5cm} r}
آمد بهار خرم آمد نگار ما
&&
چون صد هزار تنگ شکر در کنار ما
\\
آمد مهی که مجلس جان زو منورست
&&
تا بشکند ز باده گلگون خمار ما
\\
شاد آمدی بیا و ملوکانه آمدی
&&
ای سرو گلستان چمن و لاله زار ما
\\
پاینده باش ای مه و پاینده عمر باش
&&
در بیشه جهان ز برای شکار ما
\\
دریا به جوش از تو که بی‌مثل گوهری
&&
کهسار در خروش که ای یار غار ما
\\
در روز بزم ساقی دریاعطای ما
&&
در روز رزم شیر نر و ذوالفقار ما
\\
چونی در این غریبی و چونی در این سفر
&&
برخیز تا رویم به سوی دیار ما
\\
ما را به مشک و خم و سبوها قرار نیست
&&
ما را کشان کنید سوی جویبار ما
\\
سوی پری رخی که بر آن چشم‌ها نشست
&&
آرام عقل مست و دل بی‌قرار ما
\\
شد ماه در گدازش سوداش همچو ما
&&
شد آفتاب از رخ او یادگار ما
\\
ای رونق صباح و صبوح ظریف ما
&&
وی دولت پیاپی بیش از شمار ما
\\
هر چند سخت مستی سستی مکن بگیر
&&
کارزد به هر چه گویی خمر و خمار ما
\\
جامی چو آفتاب پرآتش بگیر زود
&&
درکش به روی چون قمر شهریار ما
\\
این نیم کاره ماند و دل من ز کار شد
&&
کار او کند که هست خداوندگار ما
\\
\end{longtable}
\end{center}
