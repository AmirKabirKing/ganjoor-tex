\begin{center}
\section*{غزل شماره ۶۲۶: هر کتش من دارد او خرقه ز من دارد}
\label{sec:0626}
\addcontentsline{toc}{section}{\nameref{sec:0626}}
\begin{longtable}{l p{0.5cm} r}
هر کآتش من دارد او خرقه ز من دارد
&&
زخمی چو حسینستش جامی چو حسن دارد
\\
نفس ار چه که زاهد شد او راست نخواهد شد
&&
ور راستیی خواهی آن سرو چمن دارد
\\
جانیست تو را ساده نقش تو از آن زاده
&&
در ساده جان بنگر کان ساده چه تن دارد
\\
آیینه جان را بین هم ساده و هم نقشین
&&
هر دم بت نو سازد گویی که شمن دارد
\\
گه جانب دل باشد گه در غم گل باشد
&&
ماننده آن مردی کز حرص دو زن دارد
\\
کی شاد شود آن شه کز جان نبود آگه
&&
کی ناز کند مرده کز شعر کفن دارد
\\
می‌خاید چون اشتر یعنی که دهانم پر
&&
خاییدن بی‌لقمه تصدیق ذقن دارد
\\
مردانه تو مجنون شو و اندر لگن خون شو
&&
گه ماده و گه نر نی کان شیوه زغن دارد
\\
چون موسی رخ زردش توبه مکن از دردش
&&
تا یار نعم گوید کر گفتن لن دارد
\\
چون مست نعم گشتی بی‌غصه و غم گشتی
&&
پس مست کجا داند کاین چرخ سخن دارد
\\
گر چشمه بود دلکش دارد دهنت را خوش
&&
لیکن همه گوهرها دریای عدن دارد
\\
\end{longtable}
\end{center}
