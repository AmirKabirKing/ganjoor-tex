\begin{center}
\section*{غزل شماره ۱۳۱۸: هر اول روز ای جان صد بار سلام علیک}
\label{sec:1318}
\addcontentsline{toc}{section}{\nameref{sec:1318}}
\begin{longtable}{l p{0.5cm} r}
هر اول روز ای جان صد بار سلام علیک
&&
در گفتن و خاموشی ای یار سلام علیک
\\
از جان همه قدوسی وز تن همه سالوسی
&&
وز گل همه جباری وز خار سلام علیک
\\
من ترکم و سرمستم ترکانه سلح بستم
&&
در ده شدم و گفتم سالار سلام علیک
\\
بنهاد یکی صهبا بر کف من و گفتا
&&
این شهره امانت را هشدار سلام علیک
\\
گفتم من دیوانه پیوسته خلیلانه
&&
بر مالک خود گویم در نار سلام علیک
\\
آن لحظه که بیرونم عالم ز سلامم پر
&&
وان لحظه که در غارم با یار سلام علیک
\\
چون صنع و نشان او دارد همه صورت‌ها
&&
ای مور شبت خوش باد ای مار سلام علیک
\\
داوود تو را گوید بر تخت فدیناکم
&&
منصور تو را گوید بر دار سلام علیک
\\
مشتاق تو را گوید بی‌طمع سلام از جان
&&
محتاج همت گوید ناچار سلام علیک
\\
شاهان چو سلام تو با طبل و علم گویند
&&
در زیر زبان گوید بیمار سلام علیک
\\
چون باده جان خوردم ایزار گرو کردم
&&
تا مست مرا گوید ای زار سلام علیک
\\
امسال ز ماه تو چندان خوش و خرم شد
&&
کز کبر نمی‌گوید بر پار سلام علیک
\\
از لذت زخمه تو این چنگ فلک بیخود
&&
سر زیر کند هر دم کای تار سلام علیک
\\
مرغان خلیلی هم سررفته و پرکنده
&&
آورده از آن عالم هر چار سلام علیک
\\
بس سیل سخن راندم بس قارعه برخواندم
&&
از کار فروماندم ای کار سلام علیک
\\
\end{longtable}
\end{center}
