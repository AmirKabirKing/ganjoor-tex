\begin{center}
\section*{غزل شماره ۸۴۵: مرغی که ناگهانی در دام ما درآمد}
\label{sec:0845}
\addcontentsline{toc}{section}{\nameref{sec:0845}}
\begin{longtable}{l p{0.5cm} r}
مرغی که ناگهانی در دام ما درآمد
&&
بشکست دام‌ها را بر لامکان برآمد
\\
از باده گزافی شد صاف صاف صافی
&&
وز درد هر دو عالم جوشید و بر سر آمد
\\
جان را چو شست از گل معراج برشد آن دل
&&
آن جا چو کرد منزل آن جاش خوشتر آمد
\\
در عالم طراوت او یافت بس حلاوت
&&
وز وصف لاله رویان رویش مزعفر آمد
\\
زان ماه هر که ماند وین نقش را نخواند
&&
در نقش دین بماند والله که کافر آمد
\\
ز اوصاف خود گذشتم وز خود برهنه گشتم
&&
زیرا برهنگان را خورشید زیور آمد
\\
الله اکبر تو خوش نیست با سر تو
&&
این سر چو گشت قربان الله اکبر آمد
\\
هر جان باملالت دورست از این جلالت
&&
چون عشق با ملولی کشتی و لنگر آمد
\\
ای شمس حق تبریز دل پیش آفتابت
&&
در کم زنی مطلق از ذره کمتر آمد
\\
\end{longtable}
\end{center}
