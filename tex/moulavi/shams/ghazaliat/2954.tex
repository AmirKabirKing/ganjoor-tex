\begin{center}
\section*{غزل شماره ۲۹۵۴: گر روشنی تو یارا یا خود سیه ضمیری}
\label{sec:2954}
\addcontentsline{toc}{section}{\nameref{sec:2954}}
\begin{longtable}{l p{0.5cm} r}
گر روشنی تو یارا یا خود سیه ضمیری
&&
در هر دو حال خود را از یار وانگیری
\\
پا واگرفتن تو هر دو ز حال کفر است
&&
صد کفر بیش باشد در عاشقان نفیری
\\
پاکت شود پلیدی چون از صنم بریدی
&&
گردد پلید پاکی چون غرقه در غدیری
\\
دنبال شیر گیری کی بی‌کباب مانی
&&
کی بی‌نوا نشینی چون صاحب امیری
\\
بگذار سر بد را پنهان مکن تو خود را
&&
در زیرکی چو مویی پیدا میان شیری
\\
خوردی تو زهر و گفتی حق را از این چه نقصان
&&
حق بی‌نیاز باشد وز زهر تو بمیری
\\
زیر درخت خرما انداز همچو مریم
&&
گر کاهلی به غایت ور نیز سست پیری
\\
از سایه‌های خرما شیرین شوی چو خرما
&&
وز پختگی خرما تو پختگی پذیری
\\
\end{longtable}
\end{center}
