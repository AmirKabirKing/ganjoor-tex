\begin{center}
\section*{غزل شماره ۲۰۷۷: به من نگر به دو رخسار زعفرانی من}
\label{sec:2077}
\addcontentsline{toc}{section}{\nameref{sec:2077}}
\begin{longtable}{l p{0.5cm} r}
به من نگر به دو رخسار زعفرانی من
&&
به گونه گونه علامات آن جهانی من
\\
به جان پیر قدیمی که در نهاد من است
&&
که باد خاک قدم‌هاش این جوانی من
\\
تو چشم تیز کن آخر به چشم من بنگر
&&
مدزد این دل خود را ز دلستانی من
\\
بر این لبم چو از آن بخت بوسه‌ای برسید
&&
شکر کساد شد از قند خوش زبانی من
\\
به گوش‌ها برسد حرف‌های ظاهر من
&&
به هیچ کس نرسد نعره‌های جانی من
\\
بس آتشی که فروزد از این نفس به جهان
&&
بسی بقا که بجوشد ز حرف فانی من
\\
ز شمس مفخر تبریز تا چه دیدستم
&&
که بی‌قرار شدستند این معانی من
\\
\end{longtable}
\end{center}
