\begin{center}
\section*{غزل شماره ۲۶۷۲: دلا در روزه مهمان خدایی}
\label{sec:2672}
\addcontentsline{toc}{section}{\nameref{sec:2672}}
\begin{longtable}{l p{0.5cm} r}
دلا در روزه مهمان خدایی
&&
طعام آسمانی را سرایی
\\
در این مه چون در دوزخ ببندی
&&
هزاران در ز جنت برگشایی
\\
نخواهد ماند این یخ زود بفروش
&&
بیاموز از خدا این کدخدایی
\\
برون کن خرقه کان زین چار رقعه‌ست
&&
ترابی آتشی آبی هوایی
\\
برهنه کن تو جزو جان و بنما
&&
ز خرقه گر به کل بیرون نیایی
\\
بیامد جان که عذر عشق خواهد
&&
که عفوم کن که جان عذرهایی
\\
در این مه عذر ما بپذیر ای عشق
&&
خطا کردیم ای ترک خطایی
\\
به خنده گوید او دستت گرفتم
&&
که می‌دانم که بس بی‌دست و پایی
\\
تو را پرهیز فرمودم طبیبم
&&
که تو رنجور این خوف و رجایی
\\
بکن پرهیز تا شربت بسازم
&&
که تا دور ابد باخود نیایی
\\
خمش کردم که شرحش عشق گوید
&&
که گفت او است جان را جان فزایی
\\
\end{longtable}
\end{center}
