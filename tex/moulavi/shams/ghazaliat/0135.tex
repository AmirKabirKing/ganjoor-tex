\begin{center}
\section*{غزل شماره ۱۳۵: ساقیا گردان کن آخر آن شراب صاف را}
\label{sec:0135}
\addcontentsline{toc}{section}{\nameref{sec:0135}}
\begin{longtable}{l p{0.5cm} r}
ساقیا گردان کن آخر آن شراب صاف را
&&
محو کن هست و عدم را بردران این لاف را
\\
آن میی کز قوت و لطف و رواقی و طرب
&&
برکند از بیخ هستی چو کوه قاف را
\\
در دماغ اندرببافد خمر صافی تا دماغ
&&
در زمان بیرون کند جولاه هستی باف را
\\
آن میی کز ظلم و جور و کافری‌های خوشش
&&
شرم آید عدل و داد و دین باانصاف را
\\
عقل و تدبیر و صفات تست چون استارگان
&&
زان می خورشیدوش تو محو کن اوصاف را
\\
جام جان پر کن از آن می بنگر اندر لطف او
&&
تا گشاید چشم جانت بیند آن الطاف را
\\
تن چو کفشی جان حیوانی در او چون کفشگر
&&
رازدار شاه کی خوانند هر اسکاف را
\\
روح ناری از کجا دارد ز نور می خبر
&&
آتش غیرت کجا باشد دل خزاف را
\\
سیف حق گشتست شمس الدین ما در دست حق
&&
آفرین آن سیف را و مرحبا سیاف را
\\
اسب حاجت‌های مشتاقان بدو اندررساد
&&
ای خدا ضایع مکن این سیر و این الحاف را
\\
شهر تبریزست آنک از شوق او مستی بود
&&
گر خبر گردد ز سر سر او اسلاف را
\\
\end{longtable}
\end{center}
