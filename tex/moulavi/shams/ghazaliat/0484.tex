\begin{center}
\section*{غزل شماره ۴۸۴: هر آنچ دور کند مر تو را ز دوست بدست}
\label{sec:0484}
\addcontentsline{toc}{section}{\nameref{sec:0484}}
\begin{longtable}{l p{0.5cm} r}
هر آنچ دور کند مر تو را ز دوست بدست
&&
به هر چه روی نهی بی‌وی ار نکوست بدست
\\
چو مغز خام بود در درون پوست نکوست
&&
چو پخته گشت از این پس بدانک پوست بدست
\\
درون بیضه چو آن مرغ پر و بال گرفت
&&
بدانک بیضه از این پس حجاب اوست بدست
\\
به خلق خوب اگر با جهان بسازد کس
&&
چو خلق حق نشناسد نه نیک خوست بدست
\\
فراق دوست اگر اندک‌ست اندک نیست
&&
درون چشم اگر نیم تای موست بدست
\\
در این فراق چو عمری به جست و جو بگذشت
&&
به وقت مرگ اگر نیز جست و جوست بدست
\\
غزل رها کن از این پس صلاح دین را بین
&&
از آنک خلعت نو را غزل رفوست بدست
\\
\end{longtable}
\end{center}
