\begin{center}
\section*{غزل شماره ۷۳۸: مطربم سرمست شد انگشت بر رق می‌زند}
\label{sec:0738}
\addcontentsline{toc}{section}{\nameref{sec:0738}}
\begin{longtable}{l p{0.5cm} r}
مطربم سرمست شد انگشت بر رق می‌زند
&&
پرده عشاق را از دل به رونق می‌زند
\\
رخت بربندید ای یاران که سلطان دو کون
&&
ایستاده بر فراز عرش سنجق می‌زند
\\
اولیا و انبیا حیران شده در حضرتش
&&
یحیی و داوود و یوسف خوش معلق می‌زند
\\
عیسی و موسی که باشد چاوشان درگهش
&&
جبرئیل اندر فسونش سحر مطلق می‌زند
\\
جان ابراهیم مجنون گشت اندر شوق او
&&
تیغ را بر حلق اسماعیل و اسحق می‌زند
\\
احمدش گوید که واشوقا لقا اخواننا
&&
در هوای عشق او صدیق صدق می‌زند
\\
لیلی و مجنون به فاقه آه حسرت می‌خورند
&&
خسرو و شیرین به عشرت جام راوق می‌زند
\\
شمس تبریز ایستاده مست در دستش کمان
&&
تیر زهرآلود را بر جان احمق می‌زند
\\
رستم و حمزه فکنده تیغ و اسپر پیش او
&&
او چو حیدر گردن هشام و اربق می‌زند
\\
کیست آن کس کو چنین مردی کند اندر جهان
&&
شمس تبریزی که ماه بدر را شق می‌زند
\\
هر که نام شمس تبریزی شنید و سجده کرد
&&
روح او مقبول حضرت شد اناالحق می‌زند
\\
ای حسام الدین تو بنویس مدح آن سلطان عشق
&&
گر چه منکر در هوای عشق او دق می‌زند
\\
منکرست و روسیه ملعون و مردود ابد
&&
از حسد همچون سگان از دور بق بق می‌زند
\\
\end{longtable}
\end{center}
