\begin{center}
\section*{غزل شماره ۶۵: ببین ذرات روحانی که شد تابان از این صحرا}
\label{sec:0065}
\addcontentsline{toc}{section}{\nameref{sec:0065}}
\begin{longtable}{l p{0.5cm} r}
ببین ذرات روحانی که شد تابان از این صحرا
&&
ببین این بحر و کشتی‌ها که بر هم می‌زنند این جا
\\
ببین عذرا و وامق را در آن آتش خلایق را
&&
ببین معشوق و عاشق را ببین آن شاه و آن طغرا
\\
چو جوهر قلزم اندر شد نه پنهان گشت و نی تر شد
&&
ز قلزم آتشی برشد در او هم لا و هم الا
\\
چو بی‌گاهست آهسته چو چشمت هست بربسته
&&
مزن لاف و مشو خسته مگو زیر و مگو بالا
\\
که سوی عقل کژبینی درآمد از قضا کینی
&&
چو مفلوجی چو مسکینی بماند آن عقل هم برجا
\\
اگر هستی تو از آدم در این دریا فروکش دم
&&
که اینت واجبست ای عم اگر امروز اگر فردا
\\
ز بحر این در خجل باشد چه جای آب و گل باشد
&&
چه جان و عقل و دل باشد که نبود او کف دریا
\\
چه سودا می‌پزد این دل چه صفرا می‌کند این جان
&&
چه سرگردان همی‌دارد تو را این عقل کارافزا
\\
زهی ابر گهربیزی ز شمس الدین تبریزی
&&
زهی امن و شکرریزی میان عالم غوغا
\\
\end{longtable}
\end{center}
