\begin{center}
\section*{غزل شماره ۱۴۷۲: دگربار دگربار ز زنجیر بجستم}
\label{sec:1472}
\addcontentsline{toc}{section}{\nameref{sec:1472}}
\begin{longtable}{l p{0.5cm} r}
دگربار دگربار ز زنجیر بجستم
&&
از این بند و از این دام زبون گیر بجستم
\\
فلک پیر دوتایی پر از سحر و دغایی
&&
به اقبال جوان تو از این پیر بجستم
\\
شب و روز دویدم ز شب و روز بریدم
&&
و زین چرخ بپرسید که چون تیر بجستم
\\
من از غصه چه ترسم چو با مرگ حریفم
&&
ز سرهنگ چه ترسم چو از میر بجستم
\\
به اندیشه فروبرد مرا عقل چهل سال
&&
به شصت و دو شدم صید و ز تدبیر بجستم
\\
ز تقدیر همه خلق کر و کور شدستند
&&
ز کر و فر تقدیر و ز تقدیر بجستم
\\
برون پوست درون دانه بود میوه گرفتار
&&
ازان پوست وزان دانه چو انجیر بجستم
\\
ز تأخیر بود آفت و تعجیل ز شیطان
&&
ز تعجیل دلم رست و ز تأخیر بجستم
\\
ز خون بود غذا اول و آخر شد خون شیر
&&
چو دندان خرد رست از آن شیر بجستم
\\
پی نان بدویدم یکی چند به تزویر
&&
خدا داد غذایی که ز تزویر بجستم
\\
خمش باش خمش باش به تفصیل مگو بیش
&&
ز تفسیر بگویم ز تف سیر بجستم
\\
\end{longtable}
\end{center}
