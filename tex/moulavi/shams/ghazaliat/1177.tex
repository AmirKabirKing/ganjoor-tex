\begin{center}
\section*{غزل شماره ۱۱۷۷: آفتابی برآمد از اسرار}
\label{sec:1177}
\addcontentsline{toc}{section}{\nameref{sec:1177}}
\begin{longtable}{l p{0.5cm} r}
آفتابی برآمد از اسرار
&&
جامه شویی کنیم صوفی وار
\\
تن ما خرقه ایست پرتضریب
&&
جان ما صوفییست معنی دار
\\
خرقه پر ز بند روزی چند
&&
جان و عشق است تا ابد بر کار
\\
به سر توست شاه را سوگند
&&
با چنین سر چه می‌کنی دستار
\\
چون رخ توست ماه را قبله
&&
با چنین رخ چه می‌کنی گلزار
\\
تو بها کرده بودی ای نادان
&&
گشته بودی ز عاشقی بیزار
\\
عشق ناگه جمال خود بنمود
&&
توبه سودت نکرد و استغفار
\\
این جهان همچو موم رنگارنگ
&&
عشق چون آتشی عظیم شرار
\\
موم و آتش چو گشت همسایه
&&
نقش و رنگش فنا شود ناچار
\\
گر بگویم دگر فنا گردی
&&
ور نگویم نمی‌گذارد یار
\\
جنه الروح عشق خالقها
&&
منه تجری جمیعه الانهار
\\
منه تصفر خضره الاوراق
&&
منه تخضر اغصن الاشجار
\\
منه تحمر و جنه المعشوق
&&
منه تصفر و جنه الاحرار
\\
منه تهتز صوره المسرور
&&
منه یبکی الکئیب بالاسحار
\\
ان فی العشق فسحه الارواح
&&
ان فی ذاک عبره الابصار
\\
ذبت فی العشق کی اعاینه
&&
ما کفی ان اراه باثار
\\
ان اثار تعجب اثار
&&
ان الاسرار تستر الاسرار
\\
کثره الحجب لا تحجبنی
&&
ان ذکراک تخرق الاستار
\\
\end{longtable}
\end{center}
