\begin{center}
\section*{غزل شماره ۲۷۹۱: هر دلی را گر سوی گلزار جانان خاستی}
\label{sec:2791}
\addcontentsline{toc}{section}{\nameref{sec:2791}}
\begin{longtable}{l p{0.5cm} r}
هر دلی را گر سوی گلزار جانان خاستی
&&
در دل هر خار غم گلزار جان افزاستی
\\
گر نه جوشاجوش غیرت کف برون انداختی
&&
نقش بند جان آتش رنگ او با ماستی
\\
ور نبودی پرده دار برق سوزان ماه را
&&
این زمین خاک همچون آسمان درواستی
\\
در ره معشوق جان گر پا و پر کار آمدی
&&
ذره ذره در طریقش باپر و باپاستی
\\
دیده نامحرمان گردیده بودی عشق را
&&
خود طناب خیمه‌های جمله بر دریاستی
\\
گر نه خون آمیز بودی آب چشم عاشقان
&&
بر سر هر آب چشمی نقش آن میناستی
\\
روز و شب گر دیده بودی آتش عشق مرا
&&
گرم رو بودی زمانه دی ز من فرداستی
\\
خاک باشی خواهد آن معشوق ما ور نی از او
&&
جای هر عاشق ورای گنبد خضراستی
\\
حسن شمس الدین تبریزی برافکندی نقاب
&&
گر نه اندر پیش او فراش لا لالاستی
\\
\end{longtable}
\end{center}
