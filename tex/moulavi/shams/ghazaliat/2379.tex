\begin{center}
\section*{غزل شماره ۲۳۷۹: بده آن باده جانی که چنانیم همه}
\label{sec:2379}
\addcontentsline{toc}{section}{\nameref{sec:2379}}
\begin{longtable}{l p{0.5cm} r}
بده آن باده جانی که چنانیم همه
&&
که می از جام و سر از پای ندانیم همه
\\
همه سرسبزتر از سوسن و از شاخ گلیم
&&
روح مطلق شده و تابش جانیم همه
\\
همه دربند هوااند و هوا بنده ماست
&&
که برون رفته از این دور زمانیم همه
\\
همچو سرنا بخروشیم به شکر لب یار
&&
همه دکان بفروشیم که کانیم همه
\\
تاب مشرق تن ما را مثل سایه بخورد
&&
که به صورت مثل کون و مکانیم همه
\\
زعفران رخ ما از حذر چشم بد است
&&
ما حریف چمن و لاله ستانیم همه
\\
مصحف آریم و به ساقی همه سوگند خوریم
&&
که جز از دست و کفت می‌نستانیم همه
\\
هر کی جان دارد از گلشن جان بوی برد
&&
هر کی آن دارد دریافت که آنیم همه
\\
دل ما چون دل مرغ است ز اندیشه برون
&&
که سبک دل شده زان رطل گرانیم همه
\\
ملکان تاج زر از عشق ره ما بدهند
&&
که کمربخشتر از بخت جوانیم همه
\\
جان ما را به صف اول پیکار طلب
&&
ز آنک در پیش روی تیر و سنانیم همه
\\
در پس پرده ظلمات بشر ننشینیم
&&
ز آنک چون نور سحر پرده درانیم همه
\\
شام بودیم ز خورشید جهان صبح شدیم
&&
گرگ بودیم کنون شهره شبانیم همه
\\
شمس تبریز چو بنمود رخ جان آرای
&&
سوی او با دل و جان همچو روانیم همه
\\
\end{longtable}
\end{center}
