\begin{center}
\section*{غزل شماره ۲۹۷: ای در غم تو به سوز و یارب}
\label{sec:0297}
\addcontentsline{toc}{section}{\nameref{sec:0297}}
\begin{longtable}{l p{0.5cm} r}
ای در غم تو به سوز و یارب
&&
بگریسته آسمان همه شب
\\
گر چرخ بگرید و بخندد
&&
آن جذبه خاک باشد اغلب
\\
از بس که بریخت اشک بر خاک
&&
شد خاک ز اشک او مطیب
\\
از گریه آسمان درآمد
&&
صد باغ به خنده مذهب
\\
من بودم و چرخ دوش گریان
&&
او را و مرا یکی‌ست مذهب
\\
از گریه آسمان چه روید
&&
گل‌ها و بنفشه مرطب
\\
وز گریه عاشقان چه روید
&&
صد مهر درون آن شکرلب
\\
آن چشم به گریه می‌فشارد
&&
تا بفشارد نگار غبغب
\\
این گریه ابر و خنده خاک
&&
از بهر من و تو شد مرکب
\\
وین گریه ما و خنده ما
&&
از بهر نتیجه شد مرتب
\\
خاموش کن و نظاره می‌کن
&&
اندر طلب جهان و مطلب
\\
\end{longtable}
\end{center}
