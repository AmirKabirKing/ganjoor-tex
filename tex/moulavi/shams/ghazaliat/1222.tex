\begin{center}
\section*{غزل شماره ۱۲۲۲: چه دارد در دل آن خواجه که می‌تابد ز رخسارش}
\label{sec:1222}
\addcontentsline{toc}{section}{\nameref{sec:1222}}
\begin{longtable}{l p{0.5cm} r}
چه دارد در دل آن خواجه که می‌تابد ز رخسارش
&&
چه خوردست او که می‌پیچد دو نرگسدان خمارش
\\
چه باشد در چنان دریا به غیر گوهر گویا
&&
چه باتابست آن گردون ز عکس بحر دربارش
\\
به کار خویش می‌رفتم به درویشی خود ناگه
&&
مرا پیش آمد آن خواجه بدیدم پیچ دستارش
\\
اگر چه مرغ استادم به دام خواجه افتادم
&&
دل و دیده بدو دادم شدم مست و سبکسارش
\\
بگفت ابروش تکبیری بزد چشمش یکی تیری
&&
دلم از تیر تقدیری شد آن لحظه گرفتارش
\\
مگر آن خواب دوشینه که من شوریده می‌دیدم
&&
چنین بودست تعبیرش که دیدم روز بیدارش
\\
شب تیره اگر دیدی همان خوابی که من دیدم
&&
ز نور روز بگذشتی شعاع و فر انوارش
\\
چه خواجست این چه خواجست این بنامیزد بنامیزد
&&
هزاران خواجه می‌زیبد اسیر و بند دیدارش
\\
کجا خواجه جهان باشد کسی کو بند جان باشد
&&
چو او بنده جهان باشد نباشد خواجگی یارش
\\
\end{longtable}
\end{center}
