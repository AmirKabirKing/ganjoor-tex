\begin{center}
\section*{غزل شماره ۳۹۱: مطربا این پرده زن کان یار ما مست آمدست}
\label{sec:0391}
\addcontentsline{toc}{section}{\nameref{sec:0391}}
\begin{longtable}{l p{0.5cm} r}
مطربا این پرده زن کان یار ما مست آمدست
&&
وان حیات باصفای باوفا مست آمدست
\\
گر لباس قهر پوشد چون شرر بشناسمش
&&
کو بدین شیوه بر ما بارها مست آمدست
\\
آب ما را گر بریزد ور سبو را بشکند
&&
ای برادر دم مزن کاین دم سقا مست آمدست
\\
می‌فریبم مست خود را او تبسم می‌کند
&&
کاین سلیم القلب را بین کز کجا مست آمدست
\\
آن کسی را می‌فریبی کز کمینه حرف او
&&
آب و آتش بیخود و خاک و هوا مست آمدست
\\
گفتمش گر من بمیرم تو رسی بر گور من
&&
برجهم از گور خود کان خوش لقا مست آمدست
\\
گفت آن کاین دم پذیرد کی بمیرد جان او
&&
با خدا باقی بود آن کز خدا مست آمدست
\\
عشق بی‌چون بین که جان را چون قدح پر می‌کند
&&
روی ساقی بین که خندان از بقا مست آمدست
\\
یار ما عشق است و هر کس در جهان یاری گزید
&&
کز الست این عشق بی‌ما و شما مست آمدست
\\
\end{longtable}
\end{center}
