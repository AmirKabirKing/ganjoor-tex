\begin{center}
\section*{غزل شماره ۲۹۷۹: هر روز بامداد طلبکار ما تویی}
\label{sec:2979}
\addcontentsline{toc}{section}{\nameref{sec:2979}}
\begin{longtable}{l p{0.5cm} r}
هر روز بامداد طلبکار ما تویی
&&
ما خوابناک و دولت بیدار ما تویی
\\
هر روز زان برآری ما را ز کسب و کار
&&
زیرا دکان و مکسبه و کار ما تویی
\\
دکان چرا رویم که کان و دکان تویی
&&
بازار چون رویم که بازار ما تویی
\\
زان دلخوشیم و شاد که جان بخش ما تویی
&&
زان سرخوشیم و مست که دستار ما تویی
\\
ما خمره کی نهیم پر از سیم چون بخیل
&&
ما خمره بشکنیم چو خمار ما تویی
\\
طوطی غذا شدیم که تو کان شکری
&&
بلبل نوا شدیم که گلزار ما تویی
\\
زان همچو گلشنیم که داری تو صد بهار
&&
زان سینه روشنیم که دلدار ما تویی
\\
در بحر تو ز کشتی بی‌دست و پاتریم
&&
آواز و رقص و جنبش و رفتار ما تویی
\\
هر چاره گر که هست نه سرمایه دار توست
&&
از جمله چاره باشد ناچار ما تویی
\\
دل را هر آنچ بود از آن‌ها دلش گرفت
&&
تا گفته‌ای به دل که گرفتار ما تویی
\\
گه گه گمان بریم که این جمله فعل ماست
&&
این هم ز توست مایه پندار ما تویی
\\
چیزی نمی‌کشیم که ما را تو می‌کشی
&&
چیزی نمی‌خریم خریدار ما تویی
\\
از گفت توبه کردم ای شه گواه باش
&&
بی گفت و ناله عالم اسرار ما تویی
\\
ای شمس حق مفخر تبریز شمس دین
&&
خود آفتاب گنبد دوار ما تویی
\\
\end{longtable}
\end{center}
