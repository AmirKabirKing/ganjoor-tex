\begin{center}
\section*{غزل شماره ۱۸۸۲: آن کس که تو را بیند وانگه نظرش بر تن}
\label{sec:1882}
\addcontentsline{toc}{section}{\nameref{sec:1882}}
\begin{longtable}{l p{0.5cm} r}
آن کس که تو را بیند وانگه نظرش بر تن
&&
ز آیینه ندیده‌ست او الا سیهی آهن
\\
از آب حیات تو دور است به ذات تو
&&
کز کبر برآید او بالا مثل روغن
\\
پای تو چو جان بوسد تا حشر لبان لیسد
&&
از لذت آن بوسه ای روت مه روشن
\\
گفتم به دلم چونی گفتا که در افزونی
&&
زیرا که خیالش را هستم به خدا مسکن
\\
در سینه خیال او وان گاه غم و غصه
&&
در آب حیات او وانگه خطر مردن
\\
\end{longtable}
\end{center}
