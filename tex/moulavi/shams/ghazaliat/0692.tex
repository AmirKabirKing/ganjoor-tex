\begin{center}
\section*{غزل شماره ۶۹۲: رفتیم بقیه را بقا باد}
\label{sec:0692}
\addcontentsline{toc}{section}{\nameref{sec:0692}}
\begin{longtable}{l p{0.5cm} r}
رفتیم بقیه را بقا باد
&&
لابد برود هر آنک او زاد
\\
پنگان فلک ندید هرگز
&&
طشتی که ز بام درنیفتاد
\\
چندین مدوید کاندر این خاک
&&
شاگرد همان شدست کاستاد
\\
ای خوب مناز کاندر آن گور
&&
بس شیرینست لا چو فرهاد
\\
آخر چه وفا کند بنایی
&&
کاستون ویست پاره‌ای باد
\\
گر بد بودیم بد ببردیم
&&
ور نیک بدیم یادتان باد
\\
گر اوحد دهر خویش باشی
&&
امروز روان شوی چو آحاد
\\
تنها ماندن اگر نخواهی
&&
از طاعت و خیر ساز اولاد
\\
آن رشته نور غیب باقیست
&&
کانست لباب روح اوتاد
\\
آن جوهر عشق کان خلاصه‌ست
&&
آن باقی ماند تا به آباد
\\
این ریگ روان چو بی‌قرارست
&&
شکل دگر افکنند بنیاد
\\
چون کشتی نوحم اندر این خشک
&&
کان طوفانست ختم میعاد
\\
زان خانه نوح کشتیی بود
&&
کز غیب بدید موج مرصاد
\\
خفتیم میانه خموشان
&&
کز حد بردیم بانگ و فریاد
\\
\end{longtable}
\end{center}
