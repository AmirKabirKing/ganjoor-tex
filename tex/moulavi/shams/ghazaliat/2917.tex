\begin{center}
\section*{غزل شماره ۲۹۱۷: هیچ خمری بی‌خماری دیده‌ای}
\label{sec:2917}
\addcontentsline{toc}{section}{\nameref{sec:2917}}
\begin{longtable}{l p{0.5cm} r}
هیچ خمری بی‌خماری دیده‌ای
&&
هیچ گل بی‌زخم خاری دیده‌ای
\\
در گلستان جهان آب و گل
&&
بی خزانی نوبهاری دیده‌ای
\\
چونک غم پیش آیدت در حق گریز
&&
هیچ چون حق غمگساری دیده‌ای
\\
کار حق کن بار حق کش جز ز حق
&&
هیچ کس را کار و باری دیده‌ای
\\
هیچ دل را بی‌صقال لطف او
&&
در تجلی بی‌غباری دیده‌ای
\\
بی جمال خوب دلدار قدیم
&&
جز خیالی دل فشاری دیده‌ای
\\
از نشاط صرف ناآمیخته
&&
شرح ده ای دل تو باری دیده‌ای
\\
در جهان صاف بی‌درد و دغل
&&
بی خطر ایمن مطاری دیده‌ای
\\
چون سگ کهف آی در غار وفا
&&
ای شکاری چون شکاری دیده‌ای
\\
لب ببند و چشم عبرت برگشا
&&
چونک دیده اعتباری دیده‌ای
\\
شمس تبریزی بگیرد دست تو
&&
گر ز چشم بد عثاری دیده‌ای
\\
\end{longtable}
\end{center}
