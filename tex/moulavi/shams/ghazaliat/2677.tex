\begin{center}
\section*{غزل شماره ۲۶۷۷: سلام علیک ای مقصود هستی}
\label{sec:2677}
\addcontentsline{toc}{section}{\nameref{sec:2677}}
\begin{longtable}{l p{0.5cm} r}
سلام علیک ای مقصود هستی
&&
هم از آغاز روز امروز مستی
\\
تویی می واجب آید باده خوردن
&&
تویی بت واجب آید بت پرستی
\\
به دوران تو منسوخ است شیشه
&&
بگردان آن سبوهای دودستی
\\
بیا بشنو حدیث پوست کنده
&&
همه مغزم چو در مغزم نشستی
\\
هلا ای یوسف خوبان به مصر آ
&&
ز قعر چه به حبل الله رستی
\\
بگیر ای چرخ پیر چنبری پشت
&&
رسن را سخت کز چنبر بجستی
\\
منم لولی و سرنا خوش نوازم
&&
بده شکر نیم را چون شکستی
\\
به دو بوسه مخا از خشم لب را
&&
تو ده نان چون دکان‌ها را ببستی
\\
بلی گو نی مگو ای صورت عشق
&&
که سلطان بلی شاه الستی
\\
بلی تو برآردمان به بالا
&&
بلی ما فرود آرد به پستی
\\
خمش کن عشق خود مجنون خویش است
&&
نه لیلی گنجد و نی فاطمستی
\\
\end{longtable}
\end{center}
