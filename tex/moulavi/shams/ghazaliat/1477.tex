\begin{center}
\section*{غزل شماره ۱۴۷۷: از اول امروز چو آشفته و مستیم}
\label{sec:1477}
\addcontentsline{toc}{section}{\nameref{sec:1477}}
\begin{longtable}{l p{0.5cm} r}
از اول امروز چو آشفته و مستیم
&&
آشفته بگوییم که آشفته شدستیم
\\
آن ساقی بدمست که امروز درآمد
&&
صد عذر بگفتیم و زان مست نرستیم
\\
آن باده که دادی تو و این عقل که ما راست
&&
معذور همی‌دار اگر جام شکستیم
\\
امروز سر زلف تو مستانه گرفتیم
&&
صد بار گشادیمش و صد بار ببستیم
\\
رندان خرابات بخوردند و برفتند
&&
ماییم که جاوید بخوردیم و نشستیم
\\
وقت است که خوبان همه در رقص درآیند
&&
انگشت زنان گشته که از پرده بجستیم
\\
یک لحظه بلانوش ره عشق قدیمیم
&&
یک لحظه بلی گوی مناجات الستیم
\\
از گفت بلی صبر نداریم ازیرا
&&
بسرشته و بر رسته سغراق الستیم
\\
بالا همه باغ آمد و پستی همگی گنج
&&
ما بوالعجبانیم نه بالا و نه پستیم
\\
خاموش که تا هستی او کرد تجلی
&&
هستیم بدان سان که ندانیم که هستیم
\\
تو دست بنه بر رگ ما خواجه حکیما
&&
کز دست شدستیم ببین تا ز چه دستیم
\\
هر چند پرستیدن بت مایه کفر است
&&
ما کافر عشقیم گر این بت نپرستیم
\\
جز قصه شمس حق تبریز مگویید
&&
از ماه مگویید که خورشیدپرستیم
\\
\end{longtable}
\end{center}
