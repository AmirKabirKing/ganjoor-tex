\begin{center}
\section*{غزل شماره ۱۸۵۶: مرا هر دم همی‌گویی که برگو قطعه شیرین}
\label{sec:1856}
\addcontentsline{toc}{section}{\nameref{sec:1856}}
\begin{longtable}{l p{0.5cm} r}
مرا هر دم همی‌گویی که برگو قطعه شیرین
&&
به هر بیتی یکی بوسه بده پهلوی من بنشین
\\
زهی بوسه زهی بوسه زهی حلوا و سنبوسه
&&
برآرد شیر از سنگی که عاجز گشت از او میتین
\\
تو بوسه عشق را دیدی مگر ای دل که پریدی
&&
که هر جزوت شده‌ست ای دل چو لب نالان و بوسه چین
\\
چو تلقین گفت پیغامبر شهیدان ره حق را
&&
تو هم مر کشته خود را بیا برخوان یکی تلقین
\\
به تلقین گر کنی نیت بپرد مرده در ساعت
&&
کفن گردد بر او اطلس ز گورش بردمد نسرین
\\
بکن پی مرکب تن را دلا چون تو نیاسایی
&&
چه آسایی از آن مرکب که لنگ است او ز علیین
\\
بکن پی اشتری را کو نیاید در پیت هرگز
&&
به خارستان همی‌گردد که خار افتاد او را تین
\\
چو او را پی کنی در دم چو کشتی ره رود بی‌پا
&&
ز موج بحر بی‌پایان نبرد بادبان دین
\\
\end{longtable}
\end{center}
