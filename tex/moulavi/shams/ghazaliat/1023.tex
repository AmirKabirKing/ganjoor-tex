\begin{center}
\section*{غزل شماره ۱۰۲۳: اگر باده خوری باری ز دست دلبر ما خور}
\label{sec:1023}
\addcontentsline{toc}{section}{\nameref{sec:1023}}
\begin{longtable}{l p{0.5cm} r}
اگر باده خوری باری ز دست دلبر ما خور
&&
ز دست یار آتشروی عالم سوز زیبا خور
\\
نمی‌شاید که چون برقی به هر دم خرمنی سوزی
&&
مثال کشت کوهستان همه شربت ز بالا خور
\\
اگر خواهی که چون مجنون حجاب عقل بردری
&&
ز دست عشق پابرجا شراب آن جا ز بی‌جا خور
\\
اگر دلتنگ و بدرنگی به زیر گلبنش بنشین
&&
وگر مخمور و مغموری از این بگزیده صهبا خور
\\
گریزانست این ساقی از این مستان ناموسی
&&
اگر اوباش و قلاشی مخور پنهان و پیدا خور
\\
حریفان گر همی‌خواهی چو بسطامی و چون کرخی
&&
مخور باده در این گلخن بر آن سقف معلا خور
\\
برو گر کارکی داری به کار خویشتن بنشین
&&
چو بر یوسف نه‌ای مجنون غم نان زلیخا خور
\\
کسی دکان کند ویران که بطال جهان باشد
&&
چو نربودست سیلابت تو آب از مشک سقا خور
\\
بگرد دیگ این دنیا چو کفلیز ار همی‌گردی
&&
برون رو ای سیه کاسه مخور حمرا و حلوا خور
\\
در این بازار ای مجنون چو منبل گرد تن پرخون
&&
چو در شاهد طمع کردی برو شمشیر لالا خور
\\
اگر مشتاق اشراقات شمس الدین تبریزی
&&
شراب صبر و تقوا را تو بی‌اکراه و صفرا خور
\\
\end{longtable}
\end{center}
