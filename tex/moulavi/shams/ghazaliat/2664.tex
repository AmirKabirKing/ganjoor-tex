\begin{center}
\section*{غزل شماره ۲۶۶۴: از این تنگین قفص جانا پریدی}
\label{sec:2664}
\addcontentsline{toc}{section}{\nameref{sec:2664}}
\begin{longtable}{l p{0.5cm} r}
از این تنگین قفس جانا پریدی
&&
وزین زندان طراران رهیدی
\\
ز روی آینه گل دور کردی
&&
در آیینه بدیدی آنچ دیدی
\\
خبرها می‌شنیدی زیر و بالا
&&
بر آن بالا ببین آنچ شنیدی
\\
چو آب و گل به آب و گل سپردی
&&
قماش روح بر گردون کشیدی
\\
ز گردش‌های جسمانی بجستی
&&
به گردش‌های روحانی رسیدی
\\
بجستی ز اشکم مادر که دنیاست
&&
سوی بابای عقلانی دویدی
\\
بخور هر دم می شیرینتر از جان
&&
به هر تلخی که بهر ما چشیدی
\\
گزین کن هر چه می‌خواهی و بستان
&&
چو ما را بر همه عالم گزیدی
\\
از این دیگ جهان رفتی چو حلوا
&&
به خوان آن جهان زیرا پزیدی
\\
اگر چه بیضه خالی شد ز مرغت
&&
برون بیضه عالم پریدی
\\
در این عالم نگنجی زین سپس تو
&&
همان سو پر که هر دم در مزیدی
\\
خمش کن رو که قفل تو گشادند
&&
اجل بنمود قفلت را کلیدی
\\
\end{longtable}
\end{center}
