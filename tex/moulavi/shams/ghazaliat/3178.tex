\begin{center}
\section*{غزل شماره ۳۱۷۸: در دل من پردهٔ نو می‌زنی}
\label{sec:3178}
\addcontentsline{toc}{section}{\nameref{sec:3178}}
\begin{longtable}{l p{0.5cm} r}
در دل من پردهٔ نو می‌زنی
&&
ای دل و ای دیده و ای روشنی
\\
پرده توی وز پس پرده توی
&&
هر نفسی شکل دگر می‌کنی
\\
پرده چنان زن که بهر زخمهٔ
&&
پردهٔ غفلت ز نظر برکنی
\\
شب منم و خلوت و قندیل جان
&&
خیره که تو آتشی یا روغنی
\\
بی‌من و تو، هر دو توی، هر دو من
&&
جان منی، آن منی، یا منی
\\
نکتهٔ چون جان شنوم من ز چنگ
&&
تنتن تنتن، که تو یعنی تنی
\\
گر تنم و گر دلم و گر روان
&&
شاد بدانم که توم می‌تنی
\\
از تو چرا تازه نباشم؟! که تو
&&
تازگی سرو و گل و سوسنی
\\
از تو چرا نور نگیرم؟! که تو
&&
تابش هر خانه و هر روزنی
\\
از تو چرا زور نیابم؟! که تو
&&
قوت هر صخره و هر آهنی
\\
\end{longtable}
\end{center}
