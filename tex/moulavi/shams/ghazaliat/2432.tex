\begin{center}
\section*{غزل شماره ۲۴۳۲: ای رونق هر گلشنی وی روزن هر خانه‌ای}
\label{sec:2432}
\addcontentsline{toc}{section}{\nameref{sec:2432}}
\begin{longtable}{l p{0.5cm} r}
ای رونق هر گلشنی وی روزن هر خانه‌ای
&&
هر ذره از خورشید تو تابنده چون دردانه‌ای
\\
ای غوث هر بیچاره‌ای واگشت هر آواره‌ای
&&
اصلاح هر مکاره‌ای مقصود هر افسانه‌ای
\\
ای حسرت سرو سهی ای رونق شاهنشهی
&&
خواهم که یاران را دهی یک یاریی یارانه‌ای
\\
در هر سری سودای تو در هر لبی هیهای تو
&&
بی‌فیض شربت‌های تو عالم تهی پیمانه‌ای
\\
هر خسروی مسکین تو صید کمین شاهین تو
&&
وی سلسله تقلیب تو زنجیر هر دیوانه‌ای
\\
هر نور را ناری بود با هر گلی خاری بود
&&
بهر حرس ماری بود بر گنج هر ویرانه‌ای
\\
ای گلشنت را خار نی با نور پاکت نار نی
&&
بر گرد گنجت مار نی نی زخم و نی دندانه‌ای
\\
یک عشرتی افراشتی صد تخم فتنه کاشتی
&&
در شهر ما نگذاشتی یک عاقلی فرزانه‌ای
\\
اندیشه و فرهنگ‌ها دارد ز عشقت رنگ‌ها
&&
شب تا سحرگه چنگ‌ها ماه تو را حنانه‌ای
\\
عقل و جنون آمیخته صد نعل در ره ریخته
&&
در جعد تو آویخته اندیشه همچون شانه‌ای
\\
ای چشم تو چون نرگسی شد خواب در چشمم خسی
&&
بیدار می‌بینم بسی لیک از پی دانگانه‌ای
\\
بقال با دوغ ترش جانش مراقب لب خمش
&&
تا روز بیدار و به هش بر گوشه دکانه‌ای
\\
چون روز گردد می‌دود از بهر کسب و بهر کد
&&
تا خشک نانه او شود مشتری ترنانه‌ای
\\
ای مزرعه بگذاشته در شوره گندم کاشته
&&
ای شعله را پنداشته روزن تو چون پروانه‌ای
\\
امروز تشریفت دهد تفهیم و تشریفت دهد
&&
ترکیب و تألیفت دهد با عقل کل جانانه‌ای
\\
خامش که تو زین رسته‌ای زین دام‌ها برجسته‌ای
&&
جان و دل اندربسته‌ای در دلبری فتانه‌ای
\\
\end{longtable}
\end{center}
