\begin{center}
\section*{غزل شماره ۱۳۸۱: هرگز ندانم راندن مستی که افتد بر درم}
\label{sec:1381}
\addcontentsline{toc}{section}{\nameref{sec:1381}}
\begin{longtable}{l p{0.5cm} r}
هرگز ندانم راندن مستی که افتد بر درم
&&
در خانه گر می باشدم پیشش نهم با وی خورم
\\
مستی که شد مهمان من جان منست و آن من
&&
تاج من و سلطان من تا برنشیند بر سرم
\\
ای یار من وی خویش من مستی بیاور پیش من
&&
روزی که مستی کم کنم از عمر خویشش نشمرم
\\
چون وقف کردستم پدر بر باده‌های همچو زر
&&
در غیر ساقی ننگرم وز امر ساقی نگذرم
\\
چند آزمایم خویش را وین جان عقل اندیش را
&&
روزی که مستم کشتیم روزی که عاقل لنگرم
\\
کو خمر تن کو خمر جان کو آسمان کو ریسمان
&&
تو مست جام ابتری من مست حوض کوثرم
\\
مستی بیاید قی کند مستی زمین را طی کند
&&
این خوار و زار اندر زمین وان آسمان بر محترم
\\
گر مستی و روشن روان امشب مخسب ای ساربان
&&
خاموش کن خاموش کن زین باده نوش ای بوالکرم
\\
\end{longtable}
\end{center}
