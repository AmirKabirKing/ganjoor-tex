\begin{center}
\section*{غزل شماره ۹۰۰: بگیر دامن لطفش که ناگهان بگریزد}
\label{sec:0900}
\addcontentsline{toc}{section}{\nameref{sec:0900}}
\begin{longtable}{l p{0.5cm} r}
بگیر دامن لطفش که ناگهان بگریزد
&&
ولی مکش تو چو تیرش که از کمان بگریزد
\\
چه نقش‌ها که ببازد چه حیله‌ها که بسازد
&&
به نقش حاضر باشد ز راه جان بگریزد
\\
بر آسمانش بجویی چو مه ز آب بتابد
&&
در آب چونک درآیی بر آسمان بگریزد
\\
ز لامکانش بخوانی نشان دهد به مکانت
&&
چو در مکانش بجویی به لامکان بگریزد
\\
نه پیک تیزرو اندر وجود مرغ گمانست
&&
یقین بدان که یقین وار از گمان بگریزد
\\
از این و آن بگریزم ز ترس نی ز ملولی
&&
که آن نگار لطیفم از این و آن بگریزد
\\
گریزپای چو بادم ز عشق گل نه گلی که
&&
ز بیم باد خزانی ز بوستان بگریزد
\\
چنان گریزد نامش چو قصد گفتن بیند
&&
که گفت نیز نتانی که آن فلان بگریزد
\\
چنان گریزد از تو که گر نویسی نقشش
&&
ز لوح نقش بپرد ز دل نشان بگریزد
\\
\end{longtable}
\end{center}
