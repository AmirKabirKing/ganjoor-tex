\begin{center}
\section*{غزل شماره ۴۲۸: این چنین پابند جان میدان کیست}
\label{sec:0428}
\addcontentsline{toc}{section}{\nameref{sec:0428}}
\begin{longtable}{l p{0.5cm} r}
این چنین پابند جان میدان کیست
&&
ما شدیم از دست این دستان کیست
\\
عشق گردان کرد ساغرهای خاص
&&
عشق می‌داند که او گردان کیست
\\
جان حیاتی داد کوه و دشت را
&&
ای خدایا ای خدایا جان کیست
\\
این چه باغست این که جنت مست اوست
&&
وین بنفشه و سوسن و ریحان کیست
\\
شاخ گل از بلبلان گویاترست
&&
سرو رقصان گشته کاین بستان کیست
\\
یاسمن گفتا نگویی با سمن
&&
کاین چنین نرگس ز نرگسدان کیست
\\
چون بگفتم یاسمن خندید و گفت
&&
بیخودم من می‌ندانم کان کیست
\\
می‌دود چون گوی زرین آفتاب
&&
ای عجب اندر خم چوگان کیست
\\
ماه همچون عاشقان اندر پیش
&&
فربه و لاغر شده حیران کیست
\\
ابر غمگین در غم و اندیشه است
&&
سر پرآتش عجب گریان کیست
\\
چرخ ازرق پوش روشن دل عجب
&&
روز و شب سرمست و سرگردان کیست
\\
درد هم از درد او پرسان شده
&&
کای عجب این درد بی‌درمان کیست
\\
شمس تبریزی گشاده‌ست این گره
&&
ای عجب این قدرت و امکان کیست
\\
\end{longtable}
\end{center}
