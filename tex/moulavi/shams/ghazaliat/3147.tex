\begin{center}
\section*{غزل شماره ۳۱۴۷: ساقیا ساقیا روا داری}
\label{sec:3147}
\addcontentsline{toc}{section}{\nameref{sec:3147}}
\begin{longtable}{l p{0.5cm} r}
ساقیا ساقیا روا داری
&&
که رود روز ما به هشیاری
\\
گر بریزی تو نقل‌ها در پیش
&&
عقل‌ها را ز پیش برداری
\\
عوض باده نکته می‌گویی
&&
تا بری وقت ما به طراری
\\
درد دل را اگر نمی‌بینی
&&
بشنو از چنگ ناله و زاری
\\
ناله نای و چنگ حال دلست
&&
حال دل را تو بین که دلداری
\\
دست بر حرف بی‌دلی چه نهی
&&
حرف را در میان چه می‌آری
\\
طوق گردن تویی و حلقه گوش
&&
گردن و گوش را چه می‌خاری
\\
گفته را دانه‌های دام مساز
&&
که ز گفتست این گرفتاری
\\
گه کلیدست گفت و گه قفلست
&&
گاه از او روشنیم و گه تاری
\\
گفت بادست گر در او بوییست
&&
هدیه تو بود که گلزاری
\\
گفت جامست گر بر او نوریست
&&
از رخ تو بود که انواری
\\
مشک بربند کوزه‌ها پر شد
&&
مشک هم می‌درد ز بسیاری
\\
\end{longtable}
\end{center}
