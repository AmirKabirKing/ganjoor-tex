\begin{center}
\section*{غزل شماره ۸۷۵: گر عید وصل تست منم خود غلام عید}
\label{sec:0875}
\addcontentsline{toc}{section}{\nameref{sec:0875}}
\begin{longtable}{l p{0.5cm} r}
گر عید وصل تست منم خود غلام عید
&&
بهر تست خدمت و سجده و سلام عید
\\
تا نام تو شنیدم شد سرد بر دلم
&&
از غایت حلاوت نام تو نام عید
\\
ای شاد آن زمان که درآید وصال تو
&&
تا ما ز گنج وصل تو بدهیم وام عید
\\
تا آفتاب چهره زیبات دررسید
&&
صبحی شود ز صبح جمال تو شام عید
\\
در یمن و در سعادت و در بخت و در صفا
&&
ای پرتو خیال تو بوده امام عید
\\
ای سجده‌ها به پیش درت واجبات عید
&&
وی دیده خویشتن ز تو قایم خرام عید
\\
جام شراب وصل تو پر کن ز فضل خود
&&
تا کام جان روا شود از جام و کام عید
\\
اندر رکاب تو چو روان‌ها روا شوند
&&
در وی کجا رسد به دو صد سال گام عید
\\
آمد ز گرد راه تو این عید و مژده داد
&&
جانم دوید پیش و گرفته لگام عید
\\
دانست کز خدیو اجل شمس دین بود
&&
این فرو این جلالت و این لطف عام عید
\\
لیکن کجاست فر و جمال تو بی‌نظیر
&&
خود کی شوند دلشدگان تو رام عید
\\
تبریز با شراب چنان صدر نامدار
&&
بر تو حرام باشد بی‌شبهه تو جام عید
\\
\end{longtable}
\end{center}
