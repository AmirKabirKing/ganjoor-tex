\begin{center}
\section*{غزل شماره ۴۵۶: ما را کنار گیر تو را خود کنار نیست}
\label{sec:0456}
\addcontentsline{toc}{section}{\nameref{sec:0456}}
\begin{longtable}{l p{0.5cm} r}
ما را کنار گیر تو را خود کنار نیست
&&
عاشق نواختن به خدا هیچ عار نیست
\\
بی حد و بی‌کناری نایی تو در کنار
&&
ای بحر بی‌امان که تو را زینهار نیست
\\
زان شب که ماه خویش نمودی به عاشقان
&&
چون چرخ بی‌قرار کسی را قرار نیست
\\
جز فیض بحر فضل تو ما را امید نیست
&&
جز گوهر ثنای تو ما را نثار نیست
\\
تا کار و بار عشق هوای تو دیده‌ام
&&
ما را تحیریست که با کار کار نیست
\\
یک میر وانما که تو را او اسیر نیست
&&
یک شیر وانما که تو را او شکار نیست
\\
مرغان جسته‌ایم ز صد دام مردوار
&&
دامیست دام تو که از این سو مطار نیست
\\
آمد رسول عشق تو چون ساقی صبوح
&&
با جام باده‌ای که مر آن را خمار نیست
\\
گفتم که ناتوانم و رنجورم از فراق
&&
گفتا بگیر هین که گه اعتذار نیست
\\
گفتم بهانه نیست تو خود حال من ببین
&&
مپذیر عذر بنده اگر زار زار نیست
\\
کارم به یک دم آمد از دمدمه جفا
&&
هنگام مردنست زمان عقار نیست
\\
گفتا که حال خویش فراموش کن بگیر
&&
زیرا که عاشقان را هیچ اختیار نیست
\\
تا نگذری ز راحت و رنج و ز یاد خویش
&&
سوی مقربان وصالت گذار نیست
\\
آبی بزن از این می و بنشان غبار هوش
&&
جز ماه عشق هر چه بود جز غبار نیست
\\
\end{longtable}
\end{center}
