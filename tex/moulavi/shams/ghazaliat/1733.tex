\begin{center}
\section*{غزل شماره ۱۷۳۳: به جان عشق که از بهر عشق دانه و دام}
\label{sec:1733}
\addcontentsline{toc}{section}{\nameref{sec:1733}}
\begin{longtable}{l p{0.5cm} r}
به جان عشق که از بهر عشق دانه و دام
&&
که عزم صد سفرستم ز روم تا سوی شام
\\
نمی‌خورم به حلال و حرام من سوگند
&&
به جان عشق که بالاست از حلال و حرام
\\
به جان عشق که از جان جان لطیفتر است
&&
که عاشقان را عشق است هم شراب و طعام
\\
فتاده ولوله در شهر از ضمیر حسود
&&
که بازگشت فلان کس ز دوست دشمن کام
\\
نه عشق آتش و جان من است سامندر
&&
نه عشق کوره و نقد من است زر تمام
\\
نه عشق ساقی و مخمور اوست جان شب و روز
&&
نه آن شراب ازل را شده‌ست جسمم جام
\\
نهاده بر کف جامی بر من آمد عشق
&&
که ای هزار چو من عشق را غلام غلام
\\
هزار رمز به هم گفته جان من با عشق
&&
در آن رموز نگنجیده نظم حرف و کلام
\\
بیار باده خامی که خالی است وطن
&&
که عاشق زر پخته ز عشق باشد خام
\\
ورای وهم حریفی کنیم خوش با عشق
&&
نه عقل گنجد آن جا نه زحمت اجسام
\\
چو گم کنیم من و عشق خویشتن در می
&&
بیاید آن شه تبریز شمس دین که سلام
\\
\end{longtable}
\end{center}
