\begin{center}
\section*{غزل شماره ۲۶۶۶: به تن این جا به باطن در چه کاری}
\label{sec:2666}
\addcontentsline{toc}{section}{\nameref{sec:2666}}
\begin{longtable}{l p{0.5cm} r}
به تن این جا به باطن در چه کاری
&&
شکاری می‌کنی یا تو شکاری
\\
کز او در آینه ساعت به ساعت
&&
همی‌تابد عجب نقش و نگاری
\\
مثال باز سلطان است هر نقش
&&
شکار است او و می‌جوید شکاری
\\
چه ساکن می‌نماید صورت تو
&&
درون پرده تو بس بی‌قراری
\\
لباست بر لب جوی و تو غرقه
&&
از این غرقه عجب سر چون برآری
\\
حریفت حاضر است آن جا که هستی
&&
ولیکن گر بگوید شرم داری
\\
به هر شیوه که گردد شاخ رقصان
&&
نباشد غایب از باد بهاری
\\
مجه تو سو به سو ای شاخ از این باد
&&
نمی‌دانی کز این با دست یاری
\\
به صد دستان به کار توست این باد
&&
تو را خود نیست خوی حق گزاری
\\
از او یابی به آخر هر مرادی
&&
همو مستی دهد هم هوشیاری
\\
بپرس او کیست شمس الدین تبریز
&&
بجز در عشق او تا سر نخاری
\\
\end{longtable}
\end{center}
