\begin{center}
\section*{غزل شماره ۲۲۶۴: بوسیسی افندیمو هم محسن و هم مه رو}
\label{sec:2264}
\addcontentsline{toc}{section}{\nameref{sec:2264}}
\begin{longtable}{l p{0.5cm} r}
بوسیسی افندیمو هم محسن و هم مه رو
&&
نیپو سر کینیکا چونم من و چونی تو
\\
یا نعم صباح ای جان مستند همه رندان
&&
تا شب همگان عریان با یار در آب جو
\\
یا قوم اتیناکم فی الحب فدیناکم
&&
مذ نحن رایناکم امنیتنا تصفوا
\\
گر جام دهی شادم دشنام دهی شادم
&&
افندی اوتی تیلس ثیلو که براکالو
\\
چون مست شد این بنده بشنو تو پراکنده
&&
قویثز می کناکیمو سیمیر ابرالالو
\\
یا سیدتی هاتی من قهوه کاساتی
&&
من زارک من صحو ایاک و ایاه
\\
ای فارس این میدان می‌گرد تو سرگردان
&&
آخر نه کم از چرخی در خدمت آن مه رو
\\
پویسی چلبی پویسی ای پوسه اغا پوسی
&&
بی‌نخوت و ناموسی این دم دل ما را جو
\\
ای دل چو بیاسودی در خواب کجا بودی
&&
اسکرت کما تدری من سکرک لا تصحو
\\
واها سندی واها لما فتحت فاها
&&
ما اطیب سقیاها تحلوا ابدا تحلو
\\
ای چون نمکستانی اندر دل هر جانی
&&
هر صورت را ملحی از حسن تو ای مرجو
\\
چیزی به تو می‌ماند هر صورت خوب ار نی
&&
از دیدن مرد و زن خالی کنمی پهلو
\\
گر خلق بخندندم ور دست ببندندم
&&
ور زجر پسندندم من می‌نروم زین کو
\\
از مردم پژمرده دل می‌شود افسرده
&&
دارد سیهی در جان گر زرد بود مازو
\\
بانگ تو کبوتر را در برج وصال آرد
&&
گر هست حجاب او صد برج و دو صد بارو
\\
قوم خلقو بورا قالو شططا زورا
&&
فی وصفک یا مولی لا نسمع ما قالوا
\\
این نفس ستیزه رو چون بزبچه بالاجو
&&
جز ریش ندارد او نامش چه کنم ریشو
\\
خامش کن خامش کن از گفته فرامش کن
&&
هین بازمیا این سو آن سو پر چون تیهو
\\
\end{longtable}
\end{center}
