\begin{center}
\section*{غزل شماره ۲۵۸۲: خواهم که روم زین جا پایم بگرفتستی}
\label{sec:2582}
\addcontentsline{toc}{section}{\nameref{sec:2582}}
\begin{longtable}{l p{0.5cm} r}
خواهم که روم زین جا پایم بگرفتستی
&&
دل را بربودستی در دل بنشستستی
\\
سر سخره سودا شد دل بی‌سر و بی‌پا شد
&&
زان مه که نمودستی زان راز که گفتستی
\\
برپر به پر روزه زین گنبد پیروزه
&&
ای آنک در این سودا بس شب که نخفتستی
\\
چون دید که می‌سوزم گفتا که قلاوزم
&&
راهیت بیاموزم کان راه نرفتستی
\\
من پیش توام حاضر گر چه پس دیواری
&&
من خویش توام گر چه با جور تو جفتستی
\\
ای طالب خوش جمله من راست کنم جمله
&&
هر خواب که دیدستی هر دیگ که پختستی
\\
آن یار که گم کردی عمری است کز او فردی
&&
بیرونش بجستستی در خانه نجستستی
\\
این طرفه که آن دلبر با توست در این جستن
&&
دست تو گرفته‌ست او هر جا که بگشتستی
\\
در جستن او با او همره شده و می‌جو
&&
ای دوست ز پیدایی گویی که نهفتستی
\\
\end{longtable}
\end{center}
