\begin{center}
\section*{غزل شماره ۱۶۲۹: دل چه خورده‌ست عجب دوش که من مخمورم}
\label{sec:1629}
\addcontentsline{toc}{section}{\nameref{sec:1629}}
\begin{longtable}{l p{0.5cm} r}
دل چه خورده‌ست عجب دوش که من مخمورم
&&
یا نمکدان کی دیده‌ست که من در شورم
\\
هر چه امروز بریزم شکنم تاوان نیست
&&
هر چه امروز بگویم بکنم معذورم
\\
بوی جان هر نفسی از لب من می آید
&&
تا شکایت نکند جان که ز جانان دورم
\\
گر نهی تو لب خود بر لب من مست شوی
&&
آزمون کن که نه کمتر ز می انگورم
\\
ساقیا آب درانداز مرا تا گردن
&&
زانک اندیشه چو زنبور بود من عورم
\\
شب گه خواب از این خرقه برون می آیم
&&
صبح بیدار شوم باز در او محشورم
\\
هین که دجال بیامد بگشا راه مسیح
&&
هین که شد روز قیامت بزن آن ناقورم
\\
گر به هوش است خرد رو جگرش را خون کن
&&
ور نه پاره‌ست دلم پاره کن از ساطورم
\\
باده آمد که مرا بیهده بر باد دهد
&&
ساقی آمد به خرابی تن معمورم
\\
روز و شب حامل می گشته که گویی قدحم
&&
بی‌کمر چست میان بسته که گویی مورم
\\
سوی خم آمده ساغر که بکن تیمارم
&&
خم سر خویش گرفته‌ست که من رنجورم
\\
ما همه پرده دریده طلب می رفته
&&
می نشسته به بن خم که چه من مستورم
\\
تو که مست عنبی دور شو از مجلس ما
&&
که دلت را ز جهان سرد کند کافورم
\\
چون تنم را بخورد خاک لحد چون جرعه
&&
بر سر چرخ جهد جان که نه جسمم نورم
\\
نیم آن شاه که از تخت به تابوت روم
&&
خالدین ابدا شد رقم منشورم
\\
اگر آمیخته‌ام هم ز فرح ممزوجم
&&
وگر آویخته‌ام هم رسن منصورم
\\
جام فرعون نگیرم که دهان گنده کند
&&
جان موسی است روان در تن همچون طورم
\\
هله خاموش که سرمست خموش اولیتر
&&
من فغان را چه کنم نی ز لبش مهجورم
\\
شمس تبریز که مشهورتر از خورشید است
&&
من که همسایه شمسم چو قمر مشهورم
\\
\end{longtable}
\end{center}
