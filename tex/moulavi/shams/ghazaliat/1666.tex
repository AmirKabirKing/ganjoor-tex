\begin{center}
\section*{غزل شماره ۱۶۶۶: گفته‌ای من یار دیگر می کنم}
\label{sec:1666}
\addcontentsline{toc}{section}{\nameref{sec:1666}}
\begin{longtable}{l p{0.5cm} r}
گفته‌ای من یار دیگر می کنم
&&
بر تو دل چون سنگ مرمر می کنم
\\
پس تو خود این گو که از تیغ جفا
&&
عاشقی را قصد و بی‌سر می کنم
\\
گوهری را زیر مرمر می کشم
&&
مرمری را لعل و گوهر می کنم
\\
صد هزاران مؤمن توحید را
&&
بسته آن زلف کافر می کنم
\\
عاشقان را در کشاکش همچو ماه
&&
گاه فربه گاه لاغر می کنم
\\
کله‌های عشق را از خنب جان
&&
کیل باده همچو ساغر می کنم
\\
باغ دل سرسبز و تر باشد ولیک
&&
از فراقش خشک و بی‌بر می کنم
\\
گلبنان را جمله گردن می زنم
&&
قصد شاخ تازه و تر می کنم
\\
چونک بی‌من باغ حال خود بدید
&&
جور هشتم داد و داور می کنم
\\
از بهار وصل بر بیمار دی
&&
مغفرت را روح پرور می کنم
\\
بار دیگر از بر سیمین خود
&&
دست بی‌سیمان پر از زر می کنم
\\
بندگان خویش را بر هر دو کون
&&
خسرو و خاقان و سنجر می کنم
\\
شمس تبریزی همی‌گوید به روح
&&
من ز عین روح سرور می کنم
\\
\end{longtable}
\end{center}
