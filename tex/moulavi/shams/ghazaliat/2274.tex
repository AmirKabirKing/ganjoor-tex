\begin{center}
\section*{غزل شماره ۲۲۷۴: مررت بدر فی هواه بحار}
\label{sec:2274}
\addcontentsline{toc}{section}{\nameref{sec:2274}}
\begin{longtable}{l p{0.5cm} r}
مررت بدر فی هواه بحار
&&
راوه بدر و فی الدلال و حاروا
\\
و شاهدت ماء شابه الروح فی الصفا
&&
و یعشق ذاک الماء ما هو نار
\\
و للعشق نور لیس للشمس مثله
&&
فظل دلیل العاشقین و ساروا
\\
عروس الهوی بدر تلالا فی الدجی
&&
علیها دماء العاشقین خمار
\\
ظللت من الدنیا علی طلب الهوی
&&
اضاء لنا غیر الدیار دیار
\\
فشاهدت رکبانا قریحا مطیهم
&&
و کان لهم عند المسیر بدار
\\
فقلت لهم فی ذاک قالوا لفی الهوی
&&
لمن فر من هذا الدیار دمار
\\
و ان شئت برهانا فسافر ببلده
&&
یقال لها تبریز و هی مزار
\\
فیشتم اهل العشق من ترباته
&&
و للروح منها زخرف و سوار
\\
تروح کلیل مظلم فی هوائه
&&
و ترجع مسرورا و انت نهار
\\
\end{longtable}
\end{center}
