\begin{center}
\section*{غزل شماره ۱۳۲۹: چو زد فراق تو بر سر مرا به نیرو سنگ}
\label{sec:1329}
\addcontentsline{toc}{section}{\nameref{sec:1329}}
\begin{longtable}{l p{0.5cm} r}
چو زد فراق تو بر سر مرا به نیرو سنگ
&&
رسید بر سر من بعد از آن ز هر سو سنگ
\\
هزار سنگ ز آفاق بر سرم آید
&&
چنان نباشد کز دست یار خوش خو سنگ
\\
مرا ز مطبخ عشق خوش تو بویی بود
&&
فراق می‌زند از بخت من بر آن بو سنگ
\\
ز دست تو شود آن سنگ لعل می‌دانم
&&
به امتحان به کف آور به دست خود تو سنگ
\\
اگر فتد نظر لطف تو به کوه و به سنگ
&&
شود همه زر و گویند در جهان کو سنگ
\\
سخای کف تو گر چربشی به کوه دهد
&&
دهد به خشک دماغان همیشه چربوسنگ
\\
ز لطف گر به جهان در نظر کنی یک دم
&&
روان کند ز عرق صد فرات و صد جو سنگ
\\
اگر ز آب حیات تو سنگ تر گردد
&&
حیات گیرد و مشک آکند چو آهو سنگ
\\
به آبگینه این دل نظر کن از سر لطف
&&
که می طلب کند از وصل تو به جان او سنگ
\\
عصای هجر تو گویی عصای موسی بود
&&
ز هر دو چشم روان کرد آب و هر دو سنگ
\\
ز بخت من ز دل تو سدیست از آهن
&&
که آهن آید فرزند از زن و شو سنگ
\\
کنون ز هجر زنم سنگ بر دلم لیکن
&&
بیاورید ز تبریز نزد من زو سنگ
\\
ز بس که روی نهادم به سنگ در تبریز
&&
به هر طرف دهدت خود نشانه رو سنگ
\\
نگردم از هوسش گر ببارد از سر خشم
&&
به سوی جان و دلم درشمار هر مو سنگ
\\
ولیک از کرم بی‌نظیر شمس الدین
&&
کجاست خاک رهش را امید و مرجو سنگ
\\
دعای جانم اینست که جان فدای تو باد
&&
وگر زنند همه بر سر دعاگو سنگ
\\
\end{longtable}
\end{center}
