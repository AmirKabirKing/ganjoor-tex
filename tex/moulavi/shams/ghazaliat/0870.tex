\begin{center}
\section*{غزل شماره ۸۷۰: چشمم همی‌پرد مگر آن یار می‌رسد}
\label{sec:0870}
\addcontentsline{toc}{section}{\nameref{sec:0870}}
\begin{longtable}{l p{0.5cm} r}
چشمم همی‌پرد مگر آن یار می‌رسد
&&
دل می‌جهد نشانه که دلدار می‌رسد
\\
این هدهد از سپاه سلیمان همی‌پرد
&&
وین بلبل از نواحی گلزار می‌رسد
\\
جامی بخر به جانی ور زانک مفلسی
&&
بفروش خویش را که خریدار می‌رسد
\\
آن گوش انتظار خبر نوش می‌کند
&&
وان چشم اشکبار به دیدار می‌رسد
\\
آن دل که پاره پاره شد و پاره‌هاش خون
&&
آن پاره پاره رفته به یک بار می‌رسد
\\
قد چو چنگ را که دلش تار تار شد
&&
نک زخمه نشاط به هر تار می‌رسد
\\
آن خارخار باغ و تقاضاش رد نشد
&&
گل‌های خوش عذار سوی خار می‌رسد
\\
آن زینهار گفتن عاشق تهی نبود
&&
اینک سپاه وصل به زنهار می‌رسد
\\
نک طوطیان عشق گشادند پر و بال
&&
کز سوی مصر قند به قنطار می‌رسد
\\
شهر ایمنست جمله دزدان گریختند
&&
از بیم آنک شحنه قهار می‌رسد
\\
چندین هزار جعفر طرار شب گریخت
&&
کآمد خبر که جعفر طیار می‌رسد
\\
فاش و صریح گو که صفات بشر گریخت
&&
زیرا صفات خالق جبار می‌رسد
\\
ای مفلسان باغ خزان راهتان بزد
&&
سلطان نوبهار به ایثار می‌رسد
\\
در خامشیست تابش خورشید بی‌حجاب
&&
خاموش کاین حجاب ز گفتار می‌رسد
\\
\end{longtable}
\end{center}
