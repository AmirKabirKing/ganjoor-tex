\begin{center}
\section*{غزل شماره ۲۴۶۱: چون دل من جست ز تن بازنگشتی چه شدی}
\label{sec:2461}
\addcontentsline{toc}{section}{\nameref{sec:2461}}
\begin{longtable}{l p{0.5cm} r}
چون دل من جست ز تن بازنگشتی چه شدی
&&
بی‌دل من بی‌دل من راست شدی هر چه بدی
\\
گر کژ و گر راست شدی ور کم ور کاست شدی
&&
فارغ و آزاد بدی خواجه ز هر نیک و بدی
\\
هیچ فضولی نبدی هیچ ملولی نبدی
&&
دانش و گولی نبدی طبل تحیات زدی
\\
خواجه چه گیری گروم تو نروی من بروم
&&
کهنه نه‌ام خواجه نوم در مدد اندر مددی
\\
آتش و نفتم نخورد ور بخورد بازدهد
&&
چون عددی را بخورد بازدهد بی‌عددی
\\
بر سر خرپشته من بانگ زن ای کشته من
&&
دانک من اندر چمنم صورت من در لحدی
\\
گر چه بود در لحدی خوش بودش با احدی
&&
آنک در آن دام بود کی خوردش دام و ددی
\\
و آنک از او دور بود گر چه که منصور بود
&&
زارتر از مور بود ز آنک ندارد سندی
\\
\end{longtable}
\end{center}
