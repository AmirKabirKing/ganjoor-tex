\begin{center}
\section*{غزل شماره ۱۹۵۶: سر فروکرد از فلک آن ماه روی سیمتن}
\label{sec:1956}
\addcontentsline{toc}{section}{\nameref{sec:1956}}
\begin{longtable}{l p{0.5cm} r}
سر فروکرد از فلک آن ماه روی سیمتن
&&
آستین را می فشاند در اشارت سوی من
\\
همچو چشم کشتگان چشمان من حیران او
&&
وز شراب عشق او این جان من بی‌خویشتن
\\
زیر جعد زلف مشکش صد قیامت را مقام
&&
در صفای صحن رویش آفت هر مرد و زن
\\
مرغ جان اندر قفس می کند پر و بال خویش
&&
تا قفس را بشکند اندر هوای آن شکن
\\
از فلک آمد همایی بر سر من سایه کرد
&&
من فغان کردم که دور از پیش آن خوب ختن
\\
در سخن آمد همای و گفت بی‌روزی کسی
&&
کز سعادت می گریزی ای شقی ممتحن
\\
گفتمش آخر حجابی در میان ما و دوست
&&
من جمال دوست خواهم کو است مر جان را سکن
\\
آن همای از بس تعجب سوی آن مه بنگرید
&&
از من او دیوانه تر شد در جمالش مفتتن
\\
میر مست و خواجه مست و روح مست و جسم مست
&&
از خداوند شمس دین آن شاه تبریز و زمن
\\
\end{longtable}
\end{center}
