\begin{center}
\section*{غزل شماره ۳۵۸: نگار خوب شکربار چونست}
\label{sec:0358}
\addcontentsline{toc}{section}{\nameref{sec:0358}}
\begin{longtable}{l p{0.5cm} r}
نگار خوب شکربار چونست
&&
چراغ دیده و دیدار چونست
\\
عجب آن غمزه غماز چونست
&&
عجب آن طره طرار چونست
\\
عجب آن شهره بازار خوبی
&&
عجب آن رونق گلزار چونست
\\
دلم از مهر در ماتم نشسته‌ست
&&
عجب در مهر دل دلدار چونست
\\
ز لطف خویش یارم خواند آن یار
&&
عجب آن یار بی این یار چونست
\\
به ظاهر بندگان را می‌نوازد
&&
عجب با بنده در اسرار چونست
\\
چو اول دیدمش جانیم بخشید
&&
بدانستم که در ایثار چونست
\\
اگر دوباره کردی آن کرم را
&&
یقین گشتی که در تکرار چونست
\\
عجب آن شعر اطلس پوش جعدش
&&
بگرد اطلس رخسار چونست
\\
طبیب عاشقان را بازپرسید
&&
که تا آن نرگس بیمار چونست
\\
عجب آن نافه تاتار چونست
&&
عجب آن طره بلغار چونست
\\
عجب بر دایره خط محقق
&&
که بشکسته‌ست صد پرگار چونست
\\
من زارم اسیر ناله زیر
&&
نپرسد روزکی کان زار چونست
\\
دلم دزد نظر او دزد این دزد
&&
عجب آن دزد دزدافشار چونست
\\
تو را ای دوست چون من یار غارم
&&
سری در غار کن کاین غار چونست
\\
که تا بینم تو را جان برفشانم
&&
نمایم خلق را نظار چونست
\\
نهایت نیست گفتم را ولیکن
&&
نمودم شکل آن گفتار چونست
\\
\end{longtable}
\end{center}
