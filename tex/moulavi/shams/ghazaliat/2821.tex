\begin{center}
\section*{غزل شماره ۲۸۲۱: تو ز هر ذره وجودت بشنو ناله و زاری}
\label{sec:2821}
\addcontentsline{toc}{section}{\nameref{sec:2821}}
\begin{longtable}{l p{0.5cm} r}
تو ز هر ذره وجودت بشنو ناله و زاری
&&
تو یکی شهر بزرگی نه یکی بلکه هزاری
\\
همه اجزات خموشند ز تو اسرار نیوشند
&&
همه روزی بخروشند که بیا تا تو چه داری
\\
تویی دریای مخلد که در او ماهی بی‌حد
&&
ز سر جهل مکن رد سر انکار چه خاری
\\
همه خاموش به ظاهر همه قلاش و مقامر
&&
همه غایب همه حاضر همه صیاد و شکاری
\\
همه ماهند نه ماهی همه کیخسرو و شاهی
&&
همه چون یوسف چاهی ز تو اندر چه تاری
\\
همه ذرات چو ذاالنون همه رقاص چو گردون
&&
همه خاموش چو مریم همه در بانگ چو قاری
\\
همه اجزای وجودت به تو گویند چه بودت
&&
که همه گفت و شنودت نه ز مهر است و ز یاری
\\
مثل نفس خزان است که در او باغ نهان است
&&
ز درون باغ بخندد چو رسد جان بهاری
\\
تو بر این شمع چه گردی چو از آن شهد بخوردی
&&
تو چو پروانه چه سوزی که ز نوری نه ز ناری
\\
\end{longtable}
\end{center}
