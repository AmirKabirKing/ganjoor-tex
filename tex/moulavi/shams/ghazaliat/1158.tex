\begin{center}
\section*{غزل شماره ۱۱۵۸: رحم بر یار کی کند هم یار}
\label{sec:1158}
\addcontentsline{toc}{section}{\nameref{sec:1158}}
\begin{longtable}{l p{0.5cm} r}
رحم بر یار کی کند هم یار
&&
آه بیمار کی شنود بیمار
\\
اشک‌های بهار مشفق کو
&&
تا ز گل پر کنند دامن خار
\\
اکثروا ذکر هادم اللذات
&&
بشنوید از خزان بی‌زنهار
\\
غار جنت شود چو هست در او
&&
ثانی اثنین اذ هما فی الغار
\\
ز آه عاشق فلک شکاف کند
&&
ناله عاشقان نباشد خوار
\\
فلک از بهر عاشقان گردد
&&
بهر عشقست گنبد دوار
\\
نی برای خباز و آهنگر
&&
نی برای دروگر و عطار
\\
آسمان گرد عشق می‌گردد
&&
خیز تا ما کنیم نیز دوار
\\
بین که لو لاک ما خلقت چه گفت
&&
کان عشق است احمد مختار
\\
مدتی گرد عاشقی گردیم
&&
چند گردیم گرد این مردار
\\
چشم کو تا که جان‌ها بیند
&&
سر برون کرده از در و دیوار
\\
در و دیوار نکته گویانند
&&
آتش و خاک و آب قصه گزار
\\
چون ترازو و چون گز و چو محک
&&
بی‌زبانند و قاضی بازار
\\
عاشقا رو تو همچو چرخ بگرد
&&
خامش از گفت و جملگی گفتار
\\
\end{longtable}
\end{center}
