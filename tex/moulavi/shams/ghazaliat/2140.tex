\begin{center}
\section*{غزل شماره ۲۱۴۰: دل دی خراب و مست و خوش هر سو همی‌افتاد از او}
\label{sec:2140}
\addcontentsline{toc}{section}{\nameref{sec:2140}}
\begin{longtable}{l p{0.5cm} r}
دل دی خراب و مست و خوش هر سو همی‌افتاد از او
&&
در گلبنش جان صدزبان چون سوسن آزاد از او
\\
دل‌ها چو خسرو از لبش شیرین چو شکر تا ابد
&&
گر یک زمان پنهان شود نالند چون فرهاد از او
\\
چون صد بهشت از لطف او این قالب خاکی نگر
&&
رشک دم عیسی شده در زنده کردن باد از او
\\
در طبع همچون گولخن ناگه خلیفه رو نمود
&&
از روی میر مؤمنان شد فخر صد بغداد از او
\\
ای ذوق تسبیح ملک بر آسمان از فر او
&&
چشم و چراغ رهبری جان همه عباد از او
\\
جان صد هزاران گرد او چون انجم او مه در میان
&&
مست و خرامان می‌رود چشم بدان کم باد از او
\\
شعشاع ماه چارده از پرتو رخسار او
&&
هم جعدهای عنبرین در طره شمشاد از او
\\
گر یک جهان ویرانه شد از لشکر سلطان عشق
&&
خود صد جهان جان جان شد در عوض بنیاد از او
\\
گر چه که بیدادی کند بر عاشقان آن غمزه‌ها
&&
داده جمال و حسن را در هر دو عالم داد از او
\\
پا برنهادی بر فلک از ناز و نخوت این زمین
&&
گر فهم کردی ذره‌ای کاین شاه خوبان زاد از او
\\
عقل از سر گستاخیی پیشش دوید و زخم خورد
&&
چون دید روح آن زخم را شد در ادب استاد از او
\\
صد غلغله اندر بتان افتاد و اندر بتگران
&&
تا دست‌ها برداشتند بر چرخ در فریاد از او
\\
کآخر چه خورشید است این کز چرخ خوبی تافته‌ست
&&
این آب حیوان چون چنین دریا شد و بگشاد از او
\\
تا بردرید این عشق او پرده عروس جان‌ها
&&
تا خان و مان بگذاشتند یک عالمی داماد از او
\\
بر سر نهاده غاشیه مخدوم شمس الدین کسی
&&
کز بس جمال عزتش جبریل پر بنهاد از او
\\
زو برگشاید سر خود تبریز و جان بینا شود
&&
تا کور گردد دیده نادیده حساد از او
\\
\end{longtable}
\end{center}
