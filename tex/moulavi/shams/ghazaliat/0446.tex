\begin{center}
\section*{غزل شماره ۴۴۶: گر چپ و راست طعنه و تشنیع بیهده‌ست}
\label{sec:0446}
\addcontentsline{toc}{section}{\nameref{sec:0446}}
\begin{longtable}{l p{0.5cm} r}
گر چپ و راست طعنه و تشنیع بیهده‌ست
&&
از عشق برنگردد آن کس که دلشده‌ست
\\
مه نور می‌فشاند و سگ بانگ می‌کند
&&
مه را چه جرم خاصیت سگ چنین بده‌ست
\\
کوهست نیست که که به بادی ز جا رود
&&
آن گله پشه‌ست که بادیش ره زده‌ست
\\
گر قاعده است این که ملامت بود ز عشق
&&
کری گوش عشق از آن نیز قاعده‌ست
\\
ویرانی دو کون در این ره عمارتست
&&
ترک همه فواید در عشق فایده‌ست
\\
عیسی ز چرخ چارم می‌گوید الصلا
&&
دست و دهان بشوی که هنگام مایده‌ست
\\
رو محو یار شو به خرابات نیستی
&&
هر جا دو مست باشد ناچار عربده‌ست
\\
در بارگاه دیو درآیی که داد داد
&&
داد از خدای خواه که این جا همه دده‌ست
\\
گفتست مصطفی که ز زن مشورت مگیر
&&
این نفس ما زن‌ست اگر چه که زاهده‌ست
\\
چندان بنوش می که بمانی ز گفت و گو
&&
آخر نه عاشقی و نه این عشق میکده‌ست
\\
گر نظم و نثر گویی چون زر جعفری
&&
آن سو که جعفرست خرافات فاسده‌ست
\\
\end{longtable}
\end{center}
