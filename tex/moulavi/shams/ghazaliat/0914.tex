\begin{center}
\section*{غزل شماره ۹۱۴: ز بانگ پست تو ای دل بلند گشت وجود}
\label{sec:0914}
\addcontentsline{toc}{section}{\nameref{sec:0914}}
\begin{longtable}{l p{0.5cm} r}
ز بانگ پست تو ای دل بلند گشت وجود
&&
تو نفخ صوری یا خود قیامت موعود
\\
شنوده‌ام که بسی خلق جان بداد و بمرد
&&
ز ذوق و لذت آواز و نغمه داوود
\\
شها نوای تو برعکس بانگ داوودست
&&
کز آن بمرد و از این زنده می‌شود موجود
\\
ز حلق نیست نوایت ولیک حلقه رباست
&&
هزار حلقه ربا را چو حلقه او بربود
\\
دلا تو راست بگو دوش می کجا خوردی
&&
که از پگاه تو امروز مولعی به سرود
\\
سرود و بانگ تو زان رو گشاد می‌آرد
&&
که آن ز روح معلاست نی ز جسم فرود
\\
چو بند جسم نگشتی گشاد جان دیدی
&&
که هر که تخم نکو کشت دخل بد ندرود
\\
یقین که بوی گل فقر از گلستانیست
&&
مرود هیچ کسی دید بی‌درخت مرود
\\
خنک کسی که چو بو برد بوی او را برد
&&
خنک کسی که گشادی بیافت چشم گشود
\\
خنک کسی که از این بوی کرته یوسف
&&
دلش چو دیده یعقوب خسته واشد زود
\\
ز ناسپاسی ما بسته است روزن دل
&&
خدای گفت که انسان لربه لکنود
\\
تو سود می طلبی سود می‌رسد از یار
&&
ولی چو پی نبری کز کجاست سود چه سود
\\
ستاره ایست خدا را که در زمین گردد
&&
که در هوای ویست آفتاب و چرخ کبود
\\
بسا سحر که درآید به صومعه مؤمن
&&
که من ستاره سعدم ز من بجو مقصود
\\
ستاره‌ام که من اندر زمینم و بر چرخ
&&
به صد مقامم یابند چون خیال خدود
\\
زمینیان را شمعم سماییان را نور
&&
فرشتگان را روحم ستارگان را بود
\\
اگر چه ذره نمایم ولیک خورشیدم
&&
اگر چه جزو نمایم مراست کل وجود
\\
اگر چه قبله حاجات آسمان بوده‌ست
&&
به آسمان منگر سوی من نگر بین جود
\\
ز روی نخوت و تقلید ننگ دارد از او
&&
بلیس وار که خود بس بود خدا مسجود
\\
جواب گویدش آدم که این سجود او راست
&&
تو احولی و دو می‌بینی از ضلال و جحود
\\
ز گرد چون و چرا پرده‌ای فرود آورد
&&
میان اختر دولت میان چشم حسود
\\
ستاره گوید رو پرده تو افزون باد
&&
ز من نماندی تنها ز حضرتی مردود
\\
بسا سال و جوابی که اندر این پرده‌ست
&&
بدین حجاب ندیدی خلیل را نمرود
\\
چه پرده است حسد ای خدا میان دو یار
&&
که دی چو جان بده‌اند این زمان چو گرگ عنود
\\
چه پرده بود که ابلیس پیش از این پرده
&&
به سجده بام سموات و ارض می‌پیمود
\\
به رغبت و به نشاط و به رقت و به نیاز
&&
به گونه گونه مناجات مهر می‌افزود
\\
ز پرده حسدی ماند همچو خر بر یخ
&&
که آن همه پر و بالش بدین حدث آلود
\\
ز مسجد فلکش راند رو حدث کردی
&&
حدیث می‌نشنود و حدث همی‌پالود
\\
چرا روم به چه حجت چه کرده‌ام چه سبب
&&
بیا که بحث کنیم ای خدای فرد ودود
\\
اگر به دست تو کردی که جمله کرده تست
&&
ضلالت و ثنی و مسیحیان و یهود
\\
مرا چه گمره کردی مراد تو این بود
&&
چنان کنم که نبینی ز خلق یک محمود
\\
بگفت اگر بگذارم برآ به کوه بلند
&&
وگر نه قعر فرورو چو لنگر مشدود
\\
تو را چه بحث رسد با من ای غراب غروب
&&
اگر نه مسخ شدستی ز لعنت مورود
\\
خری که مات تو گردد ببرد از در ما
&&
نخواهمش که بود عابد چو ما معبود
\\
ولی کسی که به دستش چراغ عقل بود
&&
کجا گذارد نور و کجا رود سوی دود
\\
بگفت من به دمی آن چراغ را بکشم
&&
بگفت باد نتاند چراغ صدق ربود
\\
هر آنک پف کند او بر چراغ موهبتم
&&
بسوزد آن سر و ریشش چو هیزم موقود
\\
هزار شکر خدا را که عقل کلی باز
&&
ز بعد فرقت آمد به طالع مسعود
\\
همه سپند بسوزیم بهر آمدنش
&&
سپند چه که بسوزیم خویش را چون عود
\\
چو خویش را بنمود او ز خویش خود ببریم
&&
به کوه طور چه آریم کاه دودآلود
\\
چو موش و مار شدستیم ساکن ظلمت
&&
درون خاک مقیمان عالم محدود
\\
چو موش جز پی دزدی برون نه‌ایم از خاک
&&
چه برخوریم از آن رفتن کژ مفسود
\\
چو موش ماش رها کرد اژدهاش کنی
&&
چو گربه طالع خوانش شود جمله اسود
\\
خدای گربه بدان آفرید تا موشان
&&
نهان شوند به خاک اندرون به حبس خلود
\\
دم مسیح غلام دمت که پیش از تو
&&
بد از زمانه دم گیر راه دم مسدود
\\
همه کسان کس آنند کش کسی کرد او
&&
همه جهانش ببخشید چون بر او بخشود
\\
خموش باش که گفتار بی‌زبان داری
&&
که تار او نبود نطق و بانگ و حرفش پود
\\
چو سر ز سجده برآورد شمس تبریزی
&&
هزار کافر و مؤمن نهاد سر به سجود
\\
\end{longtable}
\end{center}
