\begin{center}
\section*{غزل شماره ۲۳۵۱: دیدی که چه کرد آن یگانه}
\label{sec:2351}
\addcontentsline{toc}{section}{\nameref{sec:2351}}
\begin{longtable}{l p{0.5cm} r}
دیدی که چه کرد آن یگانه
&&
برساخت پریر یک بهانه
\\
ما را و تو را کجا فرستاد
&&
او ماند و دو سه پری خانه
\\
ما را بفریفت ما چه باشیم
&&
با آن حرکات ساحرانه
\\
آن سلسله کو به دست دارد
&&
بربندد گردن زمانه
\\
از سنگ برون کشید مکری
&&
شاباش زهی شکر فسانه
\\
بست او گرهی میان ابرو
&&
گم گشت خرد از این میانه
\\
بر درگه او است دل چو مسمار
&&
بردوخته خویش بر ستانه
\\
بر مرکب مملکت سوار او است
&&
در دست وی است تازیانه
\\
گر او کمر کهی بگیرد
&&
که را چو کهی کند کشانه
\\
خود آن که قاف همچو سیمرغ
&&
کرده‌ست به کویش آشیانه
\\
از شرم عقیق درفشانش
&&
درها بگداخت دانه دانه
\\
بادی که ز عشق او است در تن
&&
ساکن نشود به رازیانه
\\
عشاق مذکرند وین خلق
&&
درمانده‌اند در مثانه
\\
ساقی درده قدح که ماییم
&&
مخمور ز باده شبانه
\\
آبی برزن که آتش دل
&&
بر چرخ همی‌زند زبانه
\\
در دست همیشه مصحفم بود
&&
وز عشق گرفته‌ام چغانه
\\
اندر دهنی که بود تسبیح
&&
شعر است و دوبیتی و ترانه
\\
بس صومعه‌ها که سیل بربود
&&
چه سیل که بحر بی‌کرانه
\\
هشیار ز من فسانه ناید
&&
مانند رباب بی‌کمانه
\\
مستم کن و برپران چو تیرم
&&
بشنو قصص بنی کنانه
\\
چون مست بود ز باده حق
&&
شهباز شود کمین سمانه
\\
بی‌خویش گذر کند ز دیوار
&&
بر روی هوا شود روانه
\\
باخویش ز حق شوند و بی‌خویش
&&
می‌ها بکشند عاشقانه
\\
دیدم که لبش شراب نوشد
&&
کی دید ز لب می مغانه
\\
و آن گاه چی می می خدایی
&&
نه از خنب فلان و یا فلانه
\\
ماهی ز کنار چرخ درتافت
&&
گم گشت دلم از این میانه
\\
این طرفه که شخص بی‌دل و جان
&&
چون چنگ همی‌کند فغانه
\\
مشنو غم عشق را ز هشیار
&&
کو سردلب است و سردچانه
\\
هرگز دیدی تو یا کسی دید
&&
یخدان ز آتش دهد نشانه
\\
دم درکش و فضل و فن رها کن
&&
با باز چه فن زند سمانه
\\
\end{longtable}
\end{center}
