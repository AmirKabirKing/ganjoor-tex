\begin{center}
\section*{غزل شماره ۴۵۵: آن روح را که عشق حقیقی شعار نیست}
\label{sec:0455}
\addcontentsline{toc}{section}{\nameref{sec:0455}}
\begin{longtable}{l p{0.5cm} r}
آن روح را که عشق حقیقی شعار نیست
&&
نابوده به که بودن او غیر عار نیست
\\
در عشق باش که مست عشقست هر چه هست
&&
بی کار و بار عشق بر دوست بار نیست
\\
گویند عشق چیست بگو ترک اختیار
&&
هر کو ز اختیار نرست اختیار نیست
\\
عاشق شهنشهیست دو عالم بر او نثار
&&
هیچ التفات شاه به سوی نثار نیست
\\
عشقست و عاشقست که باقیست تا ابد
&&
دل بر جز این منه که به جز مستعار نیست
\\
تا کی کنار گیری معشوق مرده را
&&
جان را کنار گیر که او را کنار نیست
\\
آن کز بهار زاد بمیرد گه خزان
&&
گلزار عشق را مدد از نوبهار نیست
\\
آن گل که از بهار بود خار یار اوست
&&
وان می که از عصیر بود بی‌خمار نیست
\\
نظاره گو مباش در این راه و منتظر
&&
والله که هیچ مرگ بتر ز انتظار نیست
\\
بر نقد قلب زن تو اگر قلب نیستی
&&
این نکته گوش کن اگرت گوشوار نیست
\\
بر اسب تن ملرز سبکتر پیاده شو
&&
پرش دهد خدای که بر تن سوار نیست
\\
اندیشه را رها کن و دل ساده شو تمام
&&
چون روی آینه که به نقش و نگار نیست
\\
چون ساده شد ز نقش همه نقش‌ها در اوست
&&
آن ساده رو ز روی کسی شرمسار نیست
\\
از عیب ساده خواهی خود را در او نگر
&&
کو را ز راست گویی شرم و حذار نیست
\\
چون روی آهنین ز صفا این هنر بیافت
&&
تا روی دل چه یابد کو را غبار نیست
\\
گویم چه یابد او نه نگویم خمش به است
&&
تا دلستان نگوید کو رازدار نیست
\\
\end{longtable}
\end{center}
