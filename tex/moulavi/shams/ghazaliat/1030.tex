\begin{center}
\section*{غزل شماره ۱۰۳۰: نیمیت ز زهر آمد نیمی دگر از شکر}
\label{sec:1030}
\addcontentsline{toc}{section}{\nameref{sec:1030}}
\begin{longtable}{l p{0.5cm} r}
نیمیت ز زهر آمد نیمی دگر از شکر
&&
بالله که چنین منگر بالله که چنان منگر
\\
هر چند که زهر از تو کانیست شکرها را
&&
زان رو که چنین نوری زان رنگ چنان انور
\\
نوری که نیارم گفت در پای تو می‌افتد
&&
معنیش که درویشا در ما بنگر خوشتر
\\
در من که توم بنگر خودبین شو و همچین شو
&&
ای نور ز سر تا پا از پای مگو وز سر
\\
چون در بصر خلقی گویی تو پر از زرقی
&&
ای آنک تو هم غرقی در خون دل من تر
\\
ار زانک گهر داری دریای دو چشمم بین
&&
ور سنگ محک داری اندر رخ من بین زر
\\
آن شیر خدایی را شمس الحق تبریزی
&&
صیدی که نه روبه شد او را به سگی مشمر
\\
\end{longtable}
\end{center}
