\begin{center}
\section*{غزل شماره ۷۷۹: همه خفتند و من دلشده را خواب نبرد}
\label{sec:0779}
\addcontentsline{toc}{section}{\nameref{sec:0779}}
\begin{longtable}{l p{0.5cm} r}
همه خفتند و من دلشده را خواب نبرد
&&
همه شب دیده من بر فلک استاره شمرد
\\
خوابم از دیده چنان رفت که هرگز ناید
&&
خواب من زهر فراق تو بنوشید و بمرد
\\
چه شود گر ز ملاقات دوایی سازی
&&
خسته‌ای را که دل و دیده به دست تو سپرد
\\
نه به یک بار نشاید در احسان بستن
&&
صافی ار می‌ندهی کم ز یکی جرعه درد
\\
همه انواع خوشی حق به یکی حجره نهاد
&&
هیچ کس بی‌تو در آن حجره ره راست نبرد
\\
گر شدم خاک ره عشق مرا خرد مبین
&&
آنک کوبد در وصل تو کجا باشد خرد
\\
آستینم ز گهرهای نهانی پر دار
&&
آستینی که بسی اشک از این دیده سترد
\\
شحنه عشق چو افشرد کسی را شب تار
&&
ماهت اندر بر سیمینش به رحمت بفشرد
\\
دل آواره اگر از کرمت بازآید
&&
قصه شب بود و قرص مه و اشتر و کرد
\\
این جمادات ز آغاز نه آبی بودند
&&
سرد سیرست جهان آمد و یک یک بفسرد
\\
خون ما در تن ما آب حیاتست و خوش است
&&
چون برون آید از جای ببینش همه ارد
\\
مفسران آب سخن را و از آن چشمه میار
&&
تا وی اطلس بود آن سوی و در این جانب برد
\\
\end{longtable}
\end{center}
