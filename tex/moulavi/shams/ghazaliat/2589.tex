\begin{center}
\section*{غزل شماره ۲۵۸۹: نه چرخ زمرد را محبوس هوا کردی}
\label{sec:2589}
\addcontentsline{toc}{section}{\nameref{sec:2589}}
\begin{longtable}{l p{0.5cm} r}
نه چرخ زمرد را محبوس هوا کردی
&&
تا صورت خاکی را در چرخ درآوردی
\\
ای آب چه می‌شویی وی باد چه می‌جویی
&&
ای رعد چه می غری وی چرخ چه می‌گردی
\\
ای عشق چه می‌خندی وی عقل چه می‌بندی
&&
وی صبر چه خرسندی وی چهره چرا زردی
\\
سر را چه محل باشد در راه وفاداری
&&
جان خود چه قدر باشد در دین جوانمردی
\\
کامل صفت آن باشد کو صید فنا باشد
&&
یک موی نمی‌گنجد در دایره فردی
\\
گه غصه و گه شادی دور است ز آزادی
&&
ای سرد کسی کو ماند در گرمی و در سردی
\\
کو تابش پیشانی گر ماه مرا دیدی
&&
کو شعشعه مستی گر باده جان خوردی
\\
زین کیسه و زان کاسه نگرفت تو را تاسه
&&
آخر نه خر کوری بر گرد چه می‌گردی
\\
با سینه ناشسته چه سود ز رو شستن
&&
کز حرص چو جارویی پیوسته در این گردی
\\
هر روز من آدینه وین خطبه من دایم
&&
وین منبر من عالی مقصوره من مردی
\\
چون پایه این منبر خالی شود از مردم
&&
ارواح و ملک از حق آرند ره آوردی
\\
\end{longtable}
\end{center}
