\begin{center}
\section*{غزل شماره ۲۶۱۴: ما می‌نرویم ای جان زین خانه دگر جایی}
\label{sec:2614}
\addcontentsline{toc}{section}{\nameref{sec:2614}}
\begin{longtable}{l p{0.5cm} r}
ما می‌نرویم ای جان زین خانه دگر جایی
&&
یا رب چه خوش است این جا هر لحظه تماشایی
\\
هر گوشه یکی باغی هر کنج یکی لاغی
&&
بی‌ولوله زاغی بی‌گرگ جگرخایی
\\
افکند خبر دشمن در شهر اراجیفی
&&
کو عزم سفر دارد از بیم تقاضایی
\\
از رشک همی‌گوید والله که دروغ است آن
&&
بی‌جان کی رود جایی بی‌سر کی نهد پایی
\\
من زیر فلک چون او ماهی ز کجا یابم
&&
او هر طرفی یابد شوریده و شیدایی
\\
مه گرد درت گردد زیرا که کجا یابد
&&
چو چشم تو خماری چون روی تو صحرایی
\\
این عشق اگر چه او پاک است ز هر صورت
&&
در عشق پدید آید هر یوسف زیبایی
\\
بی‌عشق نه یوسف را اخوان چو سگی دیدند
&&
وز عشق پدر دیدش زیبا و مطرایی
\\
گر نام سفر گویم بشکن تو دهانم را
&&
دوزخ کی رود آخر از جنت مأوایی
\\
من بی‌سر و پا گشتم خوش غرقه این دریا
&&
بی‌پای همی‌گردم چون کشتی دریایی
\\
از در اگرم رانی آیم ز ره روزن
&&
چون ذره به زیر آیم در رقص ز بالایی
\\
چون ذره رسن سازم از نور و رسن بازم
&&
در روزن این خانه در گردش سودایی
\\
بنشین که در این مجلس لاغر نشود عیسی
&&
برگو که در این دولت تیره نشود رایی
\\
بربند دهان برگو در گنبد سر خود
&&
تا ناله در آن گنبد یابی تو مثنایی
\\
شمس الحق تبریزی از لطف صفات خود
&&
از حرف همی‌گردد این نکته مصفایی
\\
\end{longtable}
\end{center}
