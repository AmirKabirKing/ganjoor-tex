\begin{center}
\section*{غزل شماره ۲۰۴۶: مستی و عاشقی و جوانی و جنس این}
\label{sec:2046}
\addcontentsline{toc}{section}{\nameref{sec:2046}}
\begin{longtable}{l p{0.5cm} r}
مستی و عاشقی و جوانی و جنس این
&&
آمد بهار خرم و گشتند همنشین
\\
صورت نداشتند مصور شدند خوش
&&
یعنی مخیلات مصورشده ببین
\\
دهلیز دیده است دل آنچ به دل رسید
&&
در دیده اندرآید صورت شود یقین
\\
تبلی السرایر است و قیامت میان باغ
&&
دل‌ها همی‌نمایند آن دلبران چین
\\
یعنی تو نیز دل بنما گر دلیت هست
&&
تا کی نهان بود دل تو در میان طین
\\
ایاک نعبد است زمستان دعای باغ
&&
در نوبهار گوید ایاک نستعین
\\
ایاک نعبد آنک به دریوزه آمدم
&&
بگشا در طرب مگذارم دگر حزین
\\
ایاک نستعین که ز پری میوه‌ها
&&
اشکسته می‌شوم نگهم دار ای معین
\\
هر لحظه لاله گوید با گل که ای عجب
&&
نرگس چه خیره می‌نگرد سوی یاسمین
\\
سوسن زبان برون کند افسوس می‌کند
&&
گوید سمن فسوس مکن بر کس ای لسین
\\
یکتا مزوری است بنفشه شده دوتا
&&
نیلوفر است واقف تزویرش ای قرین
\\
سر چپ و راست می‌فکند سنبل از خمار
&&
اریاح بر یسارش و ریحانش در یمین
\\
سبزه پیاده می‌دود اندر رکاب سرو
&&
غنچه نهان همی‌کند از چشم بد جبین
\\
بید پیاده بر لب جو اندر آینه
&&
حیران که شاخ تر ز چه افشاند آستین
\\
اول فشاندنی است که تا جمع آورد
&&
وآنگه کند نثار درافشان واپسین
\\
در باغ مجلسی چو نهاد آفریدگار
&&
مرغان چو مطربان بسرایند آفرین
\\
آن میر مطربان که ورا نام بلبل است
&&
مست است و عاشق گل از آن است خوش حنین
\\
گوید به کبک فاخته کآخر کجا بدیت
&&
گوید بدان طرف که مکان نبود و مکین
\\
شاهین به باز گوید کاین صیدهای خوب
&&
کی صید کرد از عدم آورد بر زمین
\\
یک جوق گلرخان و دگر جوق نوخطان
&&
کاندر حجاب غیب کرامند و کاتبین
\\
ما چند صورتیم یزک وار آمده
&&
نک می‌رسند لشکر خوبان از آن کمین
\\
یوسف رخان رسند ز کنعان آن جهان
&&
شیرین لبان رسند ز دریای انگبین
\\
نک نامه شان رسید به خرما و نیشکر
&&
و آن نار دانه دانه و بی‌هیچ دانه بین
\\
ای وادیی که سیب در او رنگ و بوی یافت
&&
مغز ترنج نیز معطر شد و ثمین
\\
انگور دیر آمد زیرا پیاده بود
&&
دیر آ و پخته آ که تویی فتنه‌ای مهین
\\
ای آخرین سابق و ای ختم میوه‌ها
&&
وی چنگ درزده تو به حبل الله متین
\\
شیرینیت عجایب و تلخیت خود مپرس
&&
چون عقل کز وی است شر و خیر و کفر و دین
\\
اندر بلا چو شکر و اندر رخا نبات
&&
تلخی بلای توست چو خار ترنگبین
\\
ای عارف معارف و ای واصل اصول
&&
ای دست تو دراز و زمانه تو را رهین
\\
از دست توست خربزه در خانه‌ای نهان
&&
در نی دریچه نی که تو جانی و من جنین
\\
از تو کدو گریخت رسن بازیی گرفت
&&
آن نیم کوزه کی رهد از چشمه معین
\\
چون گوش تو نداشت ببستند گردنش
&&
گوشش اگر بدی بکشیدیش خوش طنین
\\
فی جیدها ببست خدا حبل من مسد
&&
زیرا نداشت گوش به پیغام مستبین
\\
گوشی که نشنود ز خدا گوش خر بود
&&
از حق شنو تو هر نفسی دعوت مبین
\\
ای حلق تو ببسته تقاضای حلق و فرج
&&
بی‌گوش چون کدو تو رسن بسته بر وتین
\\
حلقه به گوش شه شو و حلق از رسن بخر
&&
مردم ز راه گوش شود فربه و سمین
\\
باقیش برنویسد آن شهریار لوح
&&
نقاش چین بگوید تو نقش‌ها مچین
\\
نقاش چین بگفتم آن روح محض را
&&
آن خسرو یگانه تبریز شمس دین
\\
\end{longtable}
\end{center}
