\begin{center}
\section*{غزل شماره ۳۱۴۲: حکم نو کن که شاه دورانی}
\label{sec:3142}
\addcontentsline{toc}{section}{\nameref{sec:3142}}
\begin{longtable}{l p{0.5cm} r}
حکم نو کن که شاه دورانی
&&
سکه تازه زن که سلطانی
\\
حکم مطلق تو راست در عالم
&&
حاکمان قالب‌اند و تو جانی
\\
آن چه شاهان به خواب می‌جستند
&&
چون مسلم شدت به آسانی
\\
همه مرغان چو دانه چین تواند
&&
تو همایی میان مرغانی
\\
بر سر آمد رواق دولت تو
&&
ز آن که تو صاف صاف انسانی
\\
برتر آید ز جان ملک و ملک
&&
گر دهی دل به روح حیوانی
\\
شرط‌ها را ز عاشقان برگیر
&&
که تو احوال شان همی‌دانی
\\
دام‌ها را ز راه شان بردار
&&
خواه تقدیر و خواه شیطانی
\\
تا شوم سرخ رو در این دعوی
&&
که تو چون حق لطیف فرمانی
\\
شمس تبریز رحمت صرفی
&&
ز آن که سر صفات رحمانی
\\
\end{longtable}
\end{center}
