\begin{center}
\section*{غزل شماره ۲۸۱۲: ای دهان آلوده جانی از کجا می خورده‌ای}
\label{sec:2812}
\addcontentsline{toc}{section}{\nameref{sec:2812}}
\begin{longtable}{l p{0.5cm} r}
ای دهان آلوده جانی از کجا می خورده‌ای
&&
و آن طرف کاین باده بودت از کجا ره برده‌ای
\\
با کدامین چشم تو از ظلمتی بگذشته‌ای
&&
با کدامین پای راه بی‌رهی بسپرده‌ای
\\
با کدامین دست بردی حادثات دهر را
&&
از جمال دلربایی آینه بسترده‌ای
\\
نی هزاران بار خون خویشتن را ریختی
&&
نی هزاران بار تو در زندگی خود مرده‌ای
\\
نی هزاران بار اندر کوره‌های امتحان
&&
درگدازیدی چو مس و همچو مس بفسرده‌ای
\\
نی تو بر دریای آتش بال و پر را سوختی
&&
نی تو بر پشت فلک پاهای خود افشرده‌ای
\\
چون از این ره هیچ گردی نیست بر نعلین تو
&&
از ورای این همه تو چونک اهل پرده‌ای
\\
چشم بگشا سوی ما آخر جوابی بازگو
&&
کز درون بحر دانش صافیی نی درده‌ای
\\
گفت جانم کز عنایت‌های مخدوم زمان
&&
صدر شمس الدین تبریزی تو ره گم کرده‌ای
\\
گر یکی غمزه رساند مر تو را ای سنگ دل
&&
از ورای این نشان‌ها که به گفت آورده‌ای
\\
بی علاج و حیله‌ها گر سنگ باشی در زمان
&&
گوهری گردی از آن جنسی که تو نشمرده‌ای
\\
\end{longtable}
\end{center}
