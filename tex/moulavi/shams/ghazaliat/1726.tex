\begin{center}
\section*{غزل شماره ۱۷۲۶: بیار باده که دیر است در خمار توام}
\label{sec:1726}
\addcontentsline{toc}{section}{\nameref{sec:1726}}
\begin{longtable}{l p{0.5cm} r}
بیار باده که دیر است در خمار توام
&&
اگر چه دلق کشانم نه یار غار توام
\\
بیار رطل و سبو کارم از قدح بگذشت
&&
غلام همت و داد بزرگوار توام
\\
در این زمان که خمارم مطیع من می باش
&&
چو مست گشتم از آن پس به اختیار توام
\\
بیار جام اناالحق شراب منصوری
&&
در این زمان که چو منصور زیر دار توام
\\
به یاد آر سخن‌ها و شرط‌ها که ز الست
&&
قرار دادی با من بر آن قرار توام
\\
بگو به ساغرش ای کف تو گر سوار منی
&&
عجبتر اینک در این لحظه من سوار توام
\\
میان حلقه به ظاهر تو در دوار منی
&&
ولی چو درنگرم نیک در دوار توام
\\
به زیر چرخ ننوشم شراب ای زهره
&&
که من عدو قدح‌های زهربار توام
\\
چو شیشه زان شده‌ام تا که جام شه باشم
&&
شها بگیر به دستم که دست کار توام
\\
عجب که شیشه شکافید و می نمی‌ریزد
&&
چگونه ریزد داند که بر کنار توام
\\
اگر به قد چو کمانم ولی ز تیر توام
&&
چو زعفران شدم اما به لاله زار توام
\\
چگونه کافر باشم چو بت پرست توام
&&
چگونه فاسق باشم شرابخوار توام
\\
بیا بیا که تو راز زمانه می دانی
&&
بپوش راز دل من که رازدار توام
\\
چو آفتاب رخ تو بتافت بر رخ من
&&
گمان فتاد رخم را که هم عذار توام
\\
شمرد مرغ دلم حلقه‌های دام تو را
&&
از آن خویش شمارم که در شمار توام
\\
اگر چه در چه پستم نه سربلند توام
&&
وگر چه اشتر مستم نه در قطار توام
\\
میان خون دل پرخون بگفت خاک تو را
&&
اگر چه غرقه خونم نه در تغار توام
\\
اگر چه مال ندارم نه دستمال توام
&&
اگر چه کار ندارم نه مست کار توام
\\
برآی مفخر آفاق شمس تبریزی
&&
که عاشق رخ پرنور شمس وار توام
\\
\end{longtable}
\end{center}
