\begin{center}
\section*{غزل شماره ۷۵۰: شاد شد جانم که چشمت وعده احسان نهاد}
\label{sec:0750}
\addcontentsline{toc}{section}{\nameref{sec:0750}}
\begin{longtable}{l p{0.5cm} r}
شاد شد جانم که چشمت وعده احسان نهاد
&&
ساده دل مردی که دل بر وعده مستان نهاد
\\
چون حدیث بی‌دلان بشنید جان خوشدلم
&&
جان بداد و این سخن را در میان جان نهاد
\\
برج برج و خانه خانه جویم آن خورشید را
&&
کو کلید خانه از همسایگان پنهان نهاد
\\
مشک گفتم زلف او را زین سخن بشکست زلف
&&
هندوی زلفش شکسته رو به ترکستان نهاد
\\
من نیم سلطان ولیکن خاک پای او شدم
&&
خاک پای خویشتن را او لقب سلطان نهاد
\\
همچو گربه عطسه شیری بدم از ابتدا
&&
بس شدم زیر و زبر کو گربه در انبان نهاد
\\
گفت ار تو زاده شیری نه‌ای گربه برآ
&&
بردر انبان شیر در انبان درون نتوان نهاد
\\
من چو انبان بردریدم گفت آن انبان مرا
&&
چون تویی را هر که گربه دید او بهتان نهاد
\\
شمس تبریزیست تابان از ورای هفت چرخ
&&
لاجرم تاب نوآیین بر چهارارکان نهاد
\\
\end{longtable}
\end{center}
