\begin{center}
\section*{غزل شماره ۳۰۵۶: خورانمت می جان تا دگر تو غم نخوری}
\label{sec:3056}
\addcontentsline{toc}{section}{\nameref{sec:3056}}
\begin{longtable}{l p{0.5cm} r}
خورانمت می جان تا دگر تو غم نخوری
&&
چه جای غم که ز هر شادمان گرو ببری
\\
فرشته‌ای کنمت پاک با دو صد پر و بال
&&
که در تو هیچ نماند کدورت بشری
\\
نمایمت که چگونه‌ست جان رسته ز تن
&&
فشانده دامن خود از غبار جانوری
\\
در آن صبوح که ارواح راح خاص خورند
&&
تو را خلاص نمایم ز روز و شب شمری
\\
قضا که تیر حوادث به تو همی‌انداخت
&&
تو را کند به عنایت از آن سپس سپری
\\
روان شده‌ست نسیم از شکرستان وصال
&&
که از حلاوت آن گم کند شکر شکری
\\
ز بامداد بیاورد جام چون خورشید
&&
که جزو جزو من از وی گرفت رقص گری
\\
چو سخت مست شدم گفت هین دگر بدهم
&&
که تا میان من و تو نماند این دگری
\\
بده بده هله ای جان ساقیان جهان
&&
کرم کریم نماید قمر کند قمری
\\
به آفتاب جلال خدای بی‌همتا
&&
نیافت چون تو مهی چرخ ازرق سفری
\\
تمام این تو بگو ای تمام در خوبی
&&
که بسته کرد مرا سکر باده سحری
\\
\end{longtable}
\end{center}
