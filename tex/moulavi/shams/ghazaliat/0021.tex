\begin{center}
\section*{غزل شماره ۲۱: جرمی ندارم بیش از این کز دل هوا دارم تو را}
\label{sec:0021}
\addcontentsline{toc}{section}{\nameref{sec:0021}}
\begin{longtable}{l p{0.5cm} r}
جرمی ندارم بیش از این کز دل هوا دارم تو را
&&
از زعفران روی من رو می‌بگردانی چرا
\\
یا این دل خون خواره را لطف و مراعاتی بکن
&&
یا قوت صبرش بده در یفعل الله ما یشا
\\
این دو ره آمد در روش یا صبر یا شکر نعم
&&
بی شمع روی تو نتان دیدن مر این دو راه را
\\
هر گه بگردانی تو رو آبی ندارد هیچ جو
&&
کی ذره‌ها پیدا شود بی‌شعشعه شمس الضحی
\\
بی باده تو کی فتد در مغز نغزان مستی یی
&&
بی عصمت تو کی رود شیطان بلا حول و لا
\\
نی قرص سازد قرصی یی مطبوخ هم مطبوخیی
&&
تا درنیندازی کفی ز اهلیله خود در دوا
\\
امرت نغرد کی رود خورشید در برج اسد
&&
بی تو کجا جنبد رگی در دست و پای پارسا
\\
در مرگ هشیاری نهی در خواب بیداری نهی
&&
در سنگ سقایی نهی در برق میرنده وفا
\\
سیل سیاه شب برد هر جا که عقلست و خرد
&&
زان سیلشان کی واخرد جز مشتری هل اتی
\\
ای جان جان جزو و کل وی حله بخش باغ و گل
&&
وی کوفته هر سو دهل کای جان حیران الصلا
\\
هر کس فریباند مرا تا عشر بستاند مرا
&&
آن کم دهد فهم بیا گوید که پیش من بیا
\\
زان سو که فهمت می‌رسد باید که فهم آن سو رود
&&
آن کت دهد طال بقا او را سزد طال بقا
\\
هم او که دلتنگت کند سرسبز و گلرنگت کند
&&
هم اوت آرد در دعا هم او دهد مزد دعا
\\
هم ری و بی و نون را کردست مقرون با الف
&&
در باد دم اندر دهن تا خوش بگویی ربنا
\\
لبیک لبیک ای کرم سودای تست اندر سرم
&&
ز آب تو چرخی می‌زنم مانند چرخ آسیا
\\
هرگز نداند آسیا مقصود گردش‌های خود
&&
کاستون قوت ماست او یا کسب و کار نانبا
\\
آبیش گردان می‌کند او نیز چرخی می‌زند
&&
حق آب را بسته کند او هم نمی‌جنبد ز جا
\\
خامش که این گفتار ما می‌پرد از اسرار ما
&&
تا گوید او که گفت او هرگز بننماید قفا
\\
\end{longtable}
\end{center}
