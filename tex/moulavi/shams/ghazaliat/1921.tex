\begin{center}
\section*{غزل شماره ۱۹۲۱: ای ساقی و دستگیر مستان}
\label{sec:1921}
\addcontentsline{toc}{section}{\nameref{sec:1921}}
\begin{longtable}{l p{0.5cm} r}
ای ساقی و دستگیر مستان
&&
دل را ز وفای مست مستان
\\
ای ساقی تشنگان مخمور
&&
بس تشنه شدند می پرستان
\\
از دست به دست می روان کن
&&
بر دست مگیر مکر و دستان
\\
سررشته نیستی به ما ده
&&
در حسرت نیستند هستان
\\
چون قیصر ما به قیصریه‌ست
&&
ما را منشان به آبلستان
\\
هر جا که می است بزم آن جاست
&&
هر جا که وی است نک گلستان
\\
یک جام برآر همچو خورشید
&&
عالی کن از آن نهال پستان
\\
دیدار حق است مؤمنان را
&&
خوارزم نبیند و دهستان
\\
منکر ز برای چشم زخمت
&&
همچو سر خر میان بستان
\\
گر در دل او نمی‌نشیند
&&
خوش در دل ما نشسته است آن
\\
\end{longtable}
\end{center}
