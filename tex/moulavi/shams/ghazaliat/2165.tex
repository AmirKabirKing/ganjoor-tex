\begin{center}
\section*{غزل شماره ۲۱۶۵: فقیر است او فقیر است او فقیر ابن الفقیر است او}
\label{sec:2165}
\addcontentsline{toc}{section}{\nameref{sec:2165}}
\begin{longtable}{l p{0.5cm} r}
فقیر است او فقیر است او فقیر ابن الفقیر است او
&&
خبیر است او خبیر است او خبیر ابن الخبیر است او
\\
لطیف است او لطیف است او لطیف ابن اللطیف است او
&&
امیر است او امیر است او امیر ملک گیر است او
\\
پناه است او پناه است او پناه هر گناه است او
&&
چراغ است او چراغ است او چراغ بی‌نظیر است او
\\
سکون است او سکون است او سکون هر جنون است او
&&
جهان است او جهان است او جهان شهد و شیر است او
\\
چو گفتی سر خود با او بگفتی با همه عالم
&&
وگر پنهان کنی می‌دان که دانای ضمیر است او
\\
وگر ردت کنند این‌ها بنگذارد تو را تنها
&&
درآ در ظل این دولت که شاه ناگریز است او
\\
به سوی خرمن او رو که سرسبزت کند ای جان
&&
به زیر دامن او رو که دفع تیغ و تیر است او
\\
هر آنچ او بفرماید سمعنا و اطعنا گو
&&
ز هر چیزی که می‌ترسی مجیر است او مجیر است او
\\
اگر کفر و گنه باشد وگر دیو سیه باشد
&&
چو زد بر آفتاب او یکی بدر منیر است او
\\
سخن با عشق می‌گویم سبق از عشق می‌گیرم
&&
به پیش او کشم جان را که بس اندک پذیر است او
\\
بتی دارد در این پرده بتی زیبا ولی مرده
&&
مکش اندر برش چندین که سرد و زمهریر است او
\\
دو دست و پا حنی کرده دو صد مکر و مری کرده
&&
جوان پیداست در چادر ولیکن سخت پیر است او
\\
اگر او شیر نر بودی غذای او جگر بودی
&&
ولیکن یوز را ماند که جویای پنیر است او
\\
ندارد فر سلطانی نشاید هم به دربانی
&&
که اندر عشق تتماجی برهنه همچو سیر است او
\\
اگر در تیر او باشی دوتا همچون کمان گردی
&&
از او شیری کجا آید ز خرگوشی اسیر است او
\\
دلم جوشید و می‌خواهد که صد چشمه روان گردد
&&
ببست او راه آب من به ره بستن نکیر است او
\\
\end{longtable}
\end{center}
