\begin{center}
\section*{غزل شماره ۲۰۹۲: آن دلبر من آمد بر من}
\label{sec:2092}
\addcontentsline{toc}{section}{\nameref{sec:2092}}
\begin{longtable}{l p{0.5cm} r}
آن دلبر من آمد بر من
&&
زنده شد از او بام و در من
\\
گفتم قنقی امشب تو مرا
&&
ای فتنه من شور و شر من
\\
گفتا بروم کاری است مهم
&&
در شهر مرا جان و سر من
\\
گفتم به خدا گر تو بروی
&&
امشب نزید این پیکر من
\\
آخر تو شبی رحمی نکنی
&&
بر رنگ و رخ همچون زر من
\\
رحمی نکند چشم خوش تو
&&
بر نوحه و این چشم تر من
\\
بفشاند گل گلزار رخت
&&
بر اشک خوش چون کوثر من
\\
گفتا چه کنم چون ریخت قضا
&&
خون همه را در ساغر من
\\
مریخیم و جز خون نبود
&&
در طالع من در اختر من
\\
عودی نشود مقبول خدا
&&
تا درنرود در مجمر من
\\
گفتم چو تو را قصد است به جان
&&
جز خون نبود نقل و خور من
\\
تو سرو و گلی من سایه تو
&&
من کشته تو تو حیدر من
\\
گفتا نشود قربانی من
&&
جز نادره‌ای ای چاکر من
\\
جرجیس رسد کو هر نفسی
&&
نو کشته شود در کشور من
\\
اسحاق نبی باید که بود
&&
قربان شده بر خاک در من
\\
من عشقم و چون ریزم ز تو خون
&&
زنده کنمت در محشر من
\\
هان تا نطپی در پنجه من
&&
هان تا نرمی از خنجر من
\\
با مرگ مکن تو روی ترش
&&
تا شکر کند از تو بر من
\\
می‌خند چو گل چون برکندت
&&
تا به سر شدت در شکر من
\\
اسحاق تویی من والد تو
&&
کی بشکنمت ای گوهر من
\\
عشق است پدر عاشق رمه را
&&
زاینده از او کر و فر من
\\
این گفت و بشد چون باد صبا
&&
شد اشک روان از منظر من
\\
گفتم چه شود گر لطف کنی
&&
آهسته روی ای سرور من
\\
اشتاب مکن آهسته ترک
&&
ای جان و جهان ای صدپر من
\\
کس هیچ ندید اشتاب مرا
&&
این است تک کاهلتر من
\\
این چرخ فلک گر جهد کند
&&
هرگز نرسد در معبر من
\\
گفتا که خمش کاین خنگ فلک
&&
لنگانه رود در محضر من
\\
خامش که اگر خامش نکنی
&&
در بیشه فتد این آذر من
\\
باقیش مگو تا روز دگر
&&
تا دل نپرد از مصدر من
\\
\end{longtable}
\end{center}
