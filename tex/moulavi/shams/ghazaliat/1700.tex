\begin{center}
\section*{غزل شماره ۱۷۰۰: عالم گرفت نورم بنگر به چشم‌هایم}
\label{sec:1700}
\addcontentsline{toc}{section}{\nameref{sec:1700}}
\begin{longtable}{l p{0.5cm} r}
عالم گرفت نورم بنگر به چشم‌هایم
&&
نامم بها نهادند گر چه که بی‌بهایم
\\
زان لقمه کس نخورده‌ست یک ذره زان نبرده‌ست
&&
بنگر به عزت من کان را همی‌بخایم
\\
گر چرخ و عرش و کرسی از خلق سخت دور است
&&
بیدار و خفته هر دم مستانه می برآیم
\\
آن جا جهان نور است هم حور و هم قصور است
&&
شادی و بزم و سور است با خود از آن نیایم
\\
جبریل پرده دار است مردان درون پرده
&&
در حلقه شان نگینم در حلقه چون درآیم
\\
عیسی حریف موسی یونس حریف یوسف
&&
احمد نشسته تنها یعنی که من جدایم
\\
عشق است بحر معنی هر یک چو ماهی در بحر
&&
احمد گهر به دریا اینک همی‌نمایم
\\
\end{longtable}
\end{center}
