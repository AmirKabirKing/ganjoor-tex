\begin{center}
\section*{غزل شماره ۲۵۴۹: چو دید آن طره کافر مسلمان شد مسلمانی}
\label{sec:2549}
\addcontentsline{toc}{section}{\nameref{sec:2549}}
\begin{longtable}{l p{0.5cm} r}
چو دید آن طره کافر مسلمان شد مسلمانی
&&
صلا ای کهنه اسلامان به مهمانی به مهمانی
\\
دل ایمان ز تو شادان زهی استاد استادان
&&
تو خود اسلام اسلامی تو خود ایمان ایمانی
\\
بصیرت را بصیرت تو حقیقت را حقیقت تو
&&
تو نور نور اسراری تو روح روح را جانی
\\
اگر امداد لطف تو نباشد در جهان تابان
&&
درافتد سقف این گردون بیارد رو به ویرانی
\\
چو بردابرد جاه تو ورای هر دو کون آمد
&&
زهی سرگشتگی جان‌ها زهی تشکیک و حیرانی
\\
همی‌جویم به دو عالم مثالی تا تو را گویم
&&
نمی‌یابم خداوندا نمی‌گویی که را مانی
\\
ز درمان‌ها بری گشتم نخواهم درد را درمان
&&
بمیرم در وفای تو که تو درمان درمانی
\\
الا ای جان خون ریزم همی‌پر سوی تبریزم
&&
همی‌گو نام شمس الدین اگر جایی تو درمانی
\\
صفاتت ای مه روشن عجایب خاصیت دارد
&&
که او مر ابر گریان را دراندازد به خندانی
\\
ایا دولت چو بگریزی و زین بی‌دل بپرهیزی
&&
ز لطف شاه پابرجا به دست آیی به آسانی
\\
\end{longtable}
\end{center}
