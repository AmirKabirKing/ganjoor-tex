\begin{center}
\section*{غزل شماره ۱۲۹۸: امروز روز شادی و امسال سال لاغ}
\label{sec:1298}
\addcontentsline{toc}{section}{\nameref{sec:1298}}
\begin{longtable}{l p{0.5cm} r}
امروز روز شادی و امسال سال لاغ
&&
نیکوست حال ما که نکو باد حال باغ
\\
آمد بهار و گفت به نرگس به خنده گل
&&
چشم من و تو روشن بی‌روی زشت زاغ
\\
گل نقل بلبلان و شکر نقل طوطیان
&&
سبزه‌ست و لاله زار و چمن کوری کلاغ
\\
با سیب انار گفت که شفتالویی بده
&&
گفت این هوس پزند همه منبلان راغ
\\
شفتالوی مسیح به جان می‌توان خرید
&&
جانی نه کز دلست ترقیش نه از دماغ
\\
باغ و بهار هست رسول بهشت غیب
&&
بشنو که بر رسول نباشد به جز بلاغ
\\
در آفتاب فضل گشا پر و بال نو
&&
کز پیش آفتاب برفتست میغ و ماغ
\\
چندان شراب ریخت کنون ساقی ربیع
&&
مستسقیان خاک از این فیض کرده کاغ
\\
خورشید ما مقیم حمل در بهار جان
&&
فارغ ز بهمنست و ز کانون زهی مساغ
\\
سر همچنین بجنبان یعنی سر مرا
&&
خاریدن آرزوست ندارم بدو فراغ
\\
امروز پایدار که برپاست ساقیی
&&
کبست خاک را و فلک را دو صد چراغ
\\
گه آب می‌نماید و گه آتشی کز او
&&
دل داغ داغ بود و رهانیده شد ز داغ
\\
غم چیغ چیغ کرد چو در چنگ گربه موش
&&
گو چیغ چیغ می‌کن و گو چاغ چاغ چاغ
\\
آتش بزن به چرخه و پنبه دگر مریس
&&
گردن چو دوک گشت این حرف چون پناغ
\\
\end{longtable}
\end{center}
