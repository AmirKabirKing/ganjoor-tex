\begin{center}
\section*{غزل شماره ۴۴: در دو جهان لطیف و خوش همچو امیر ما کجا}
\label{sec:0044}
\addcontentsline{toc}{section}{\nameref{sec:0044}}
\begin{longtable}{l p{0.5cm} r}
در دو جهان لطیف و خوش همچو امیر ما کجا
&&
ابروی او گره نشد گر چه که دید صد خطا
\\
چشم گشا و رو نگر جرم بیار و خو نگر
&&
خوی چو آب جو نگر جمله طراوت و صفا
\\
من ز سلام گرم او آب شدم ز شرم او
&&
وز سخنان نرم او آب شوند سنگ‌ها
\\
زهر به پیش او ببر تا کندش به از شکر
&&
قهر به پیش او بنه تا کندش همه رضا
\\
آب حیات او ببین هیچ مترس از اجل
&&
در دو در رضای او هیچ ملرز از قضا
\\
سجده کنی به پیش او عزت مسجدت دهد
&&
ای که تو خوار گشته‌ای زیر قدم چو بوریا
\\
خواندم امیر عشق را فهم بدین شود تو را
&&
چونک تو رهن صورتی صورت توست ره نما
\\
از تو دل ار سفر کند با تپش جگر کند
&&
بر سر پاست منتظر تا تو بگوییش بیا
\\
دل چو کبوتری اگر می‌بپرد ز بام تو
&&
هست خیال بام تو قبله جانش در هوا
\\
بام و هوا تویی و بس نیست روی به جز هوس
&&
آب حیات جان تویی صورت‌ها همه سقا
\\
دور مرو سفر مجو پیش تو است ماه تو
&&
نعره مزن که زیر لب می‌شنود ز تو دعا
\\
می‌شنود دعای تو می‌دهدت جواب او
&&
کای کر من کری بهل گوش تمام برگشا
\\
گر نه حدیث او بدی جان تو آه کی زدی
&&
آه بزن که آه تو راه کند سوی خدا
\\
چرخ زنان بدان خوشم کآب به بوستان کشم
&&
میوه رسد ز آب جان شوره و سنگ و ریگ را
\\
باغ چو زرد و خشک شد تا بخورد ز آب جان
&&
شاخ شکسته را بگو آب خور و بیازما
\\
شب برود بیا به گه تا شنوی حدیث شه
&&
شب همه شب مثال مه تا به سحر مشین ز پا
\\
\end{longtable}
\end{center}
