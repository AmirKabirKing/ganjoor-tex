\begin{center}
\section*{غزل شماره ۱۵۴: دیده حاصل کن دلا آنگه ببین تبریز را}
\label{sec:0154}
\addcontentsline{toc}{section}{\nameref{sec:0154}}
\begin{longtable}{l p{0.5cm} r}
دیده حاصل کن دلا آنگه ببین تبریز را
&&
بی بصیرت کی توان دیدن چنین تبریز را
\\
هر چه بر افلاک روحانیست از بهر شرف
&&
می‌نهد بر خاک پنهانی جبین تبریز را
\\
پا نهادی بر فلک از کبر و نخوت بی‌درنگ
&&
گر به چشم سر بدیدستی زمین تبریز را
\\
روح حیوانی تو را و عقل شب کوری دگر
&&
با همین دیده دلا بینی همین تبریز را
\\
تو اگر اوصاف خواهی هست فردوس برین
&&
از صفا و نور سر بنده کمین تبریز را
\\
نفس تو عجل سمین و تو مثال سامری
&&
چون شناسد دیده عجل سمین تبریز را
\\
همچو دریاییست تبریز از جواهر و ز درر
&&
چشم درناید دو صد در ثمین تبریز را
\\
گر بدان افلاک کاین افلاک گردانست از آن
&&
وافروشی هست بر جانت غبین تبریز را
\\
گر نه جسمستی تو را من گفتمی بهر مثال
&&
جوهرین یا از زمرد یا زرین تبریز را
\\
چون همه روحانیون روح قدسی عاجزند
&&
چون بدانی تو بدین رای رزین تبریز را
\\
چون درختی را نبینی مرغ کی بینی برو
&&
پس چه گویم با تو جان جان این تبریز را
\\
\end{longtable}
\end{center}
