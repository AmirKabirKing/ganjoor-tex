\begin{center}
\section*{غزل شماره ۲۶۰۸: افند کلیمیرا از زحمت ما چونی}
\label{sec:2608}
\addcontentsline{toc}{section}{\nameref{sec:2608}}
\begin{longtable}{l p{0.5cm} r}
افند کلیمیرا از زحمت ما چونی
&&
ای جان صفا چونی وی کان وفا چونی
\\
ای فخر خردمندان وی بی‌تو جهان زندان
&&
وی عاشق بی‌دل را درمان و دوا چونی
\\
مه گوش همی‌خارد صد سجده همی‌آرد
&&
می‌گوید حسنت را کی خوب لقا چونی
\\
باری من بیچاره گشتم ز خود آواره
&&
زان روز که پرسیدی گفتی تو مرا چونی
\\
ماییم و هوای تو دو چشم سقای تو
&&
ای آب حیات ما زین آب و هوا چونی
\\
تلخ است فراق تو دوری ز وثاق تو
&&
ای آنک مبادا کس دور از تو جدا چونی
\\
زد طال بقای تو هر ذره که خورشیدی
&&
ای نیر اعظم تو زین طال بقا چونی
\\
ای آینه مانده در دست دو سه زنگی
&&
وی یوسف افتاده با اهل عما چونی
\\
ای دلدل آن میدان چونی تو در این زندان
&&
وی بلبل آن بستان با ناشنوا چونی
\\
ای آدم خوکرده با جنت و با حورا
&&
افتاده در این غربت با رنج و عنا چونی
\\
ای آنک نمی‌گنجی در شش جهت عالم
&&
با این همگی زفتی در زیر قبا چونی
\\
مصباح و زجاجی تو پیش دو سه نابینا
&&
از عربده کوران وز زخم عصا چونی
\\
پیغام و سلام ما ای باد بگو با دل
&&
با این همه بی‌برگی داوودنوا چونی
\\
بس کردم من اما برگو تو تمامش را
&&
کای تشنه پرخواره با جام خدا چونی
\\
\end{longtable}
\end{center}
