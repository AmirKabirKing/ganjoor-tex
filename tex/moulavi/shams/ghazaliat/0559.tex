\begin{center}
\section*{غزل شماره ۵۵۹: زهره عشق هر سحر بر در ما چه می‌کند}
\label{sec:0559}
\addcontentsline{toc}{section}{\nameref{sec:0559}}
\begin{longtable}{l p{0.5cm} r}
زهره عشق هر سحر بر در ما چه می‌کند
&&
دشمن جان صد قمر بر در ما چه می‌کند
\\
هر که بدید از او نظر باخبرست و بی‌خبر
&&
او ملکست یا بشر بر در ما چه می‌کند
\\
زیر جهان زبر شده آب مرا ز سر شده
&&
سنگ از او گهر شده بر در ما چه می‌کند
\\
ای بت شنگ پرده‌ای گر تو نه فتنه کرده‌ای
&&
هر نفسی چنین حشر بر در ما چه می‌کند
\\
گر نه که روز روشنی پیشه گرفته رهزنی
&&
روز به روز و ره گذر بر در ما چه می‌کند
\\
ور نه که دوش مست او آمد و درشکست او
&&
پس به نشانه این کمر بر در ما چه می‌کند
\\
گر نه جمال حسن او گرد برآرد از عدم
&&
این همه گرد شور و شر بر در ما چه می‌کند
\\
از تبریز شمس دین سوی که رای می‌کند
&&
بحر چه موج زد گهر بر در ما چه می‌کند
\\
\end{longtable}
\end{center}
