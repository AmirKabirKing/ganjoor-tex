\begin{center}
\section*{غزل شماره ۸۷۹: صبح آمد و صحیفه مصقول برکشید}
\label{sec:0879}
\addcontentsline{toc}{section}{\nameref{sec:0879}}
\begin{longtable}{l p{0.5cm} r}
صبح آمد و صحیفه مصقول برکشید
&&
وز آسمان سپیده کافور بردمید
\\
صوفی چرخ خرقه و شال کبود خویش
&&
تا جایگاه ناف به عمدا فرودرید
\\
رومی روز بعد هزیمت چو دست یافت
&&
از تخت ملک زنگی شب را فروکشید
\\
زان سو که ترک شادی و هندوی غم رسید
&&
آمد شدیست دایم و راهیست ناپدید
\\
یا رب سپاه شاه حبش تا کجا گریخت
&&
ناگه سپاه قیصر روم از کجا رسید
\\
زین راه نابدید معما کی بو برد
&&
آنک از شراب عشق ازل خورد یا چشید
\\
حیران شدست شب که کی رویش سیاه کرد
&&
حیران شدست روز که خوبش که آفرید
\\
حیران شده زمین که چو نیمیش شد گیاه
&&
نیمی دگر چرنده شد و زان همی‌چرید
\\
نیمیش شد خورنده و نیمیش خوردنی
&&
نیمی حریص پاکی و نیمی دگر پلید
\\
شب مرد و زنده گشت حیاتست بعد مرگ
&&
ای غم بکش مرا که حسینم توی یزید
\\
گوهر مزاد کرد که این را کی می‌خرد
&&
کس را بها نبود همو خود ز خود خرید
\\
امروز ساقیا همه مهمان تو شدیم
&&
هر شام قدر شد ز تو هر روز روز عید
\\
درده ز جام باده که یسقون من رحیق
&&
کاندیشه را نبرد جز عشرت جدید
\\
رندان تشنه دل چو به اسراف می‌خورند
&&
خود را چو گم کنند بیابند آن کلید
\\
پهلوی خم وحدت بگرفته‌ای مقام
&&
با نوح و لوط و کرخی و شبلی و بایزید
\\
خاموش کن که جان ز فرح بال می‌زند
&&
تا آن شراب در سر و رگ‌های جان دوید
\\
\end{longtable}
\end{center}
