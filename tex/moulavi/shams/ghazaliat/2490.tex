\begin{center}
\section*{غزل شماره ۲۴۹۰: ساقی جان فزای من بهر خدا ز کوثری}
\label{sec:2490}
\addcontentsline{toc}{section}{\nameref{sec:2490}}
\begin{longtable}{l p{0.5cm} r}
ساقی جان فزای من بهر خدا ز کوثری
&&
در سر مست من فکن جام شراب احمری
\\
بحر کرم تویی مرا از کف خود بده نوا
&&
باغ ارم تویی مها بر بر من بزن بری
\\
ای به زمین ز آسمان آمده چون فرشته‌ای
&&
وی ز خطاب اشربوا مغز مرا پیمبری
\\
بزم درآ و می بده رسم بهار نو بنه
&&
ای رخ تو چو گلشنی وی قد تو صنوبری
\\
گر چه به بتکده دلم هر نفسی است صورتی
&&
نیست و نباشد و نبد چون رخ تو مصوری
\\
می چو دود بر این سرم بسکلد از تو لنگرم
&&
چهره زرد چون زرم سرخ شود چو آذری
\\
بحر کرم چه کم شود گر بخورند جرعه‌ای
&&
فضل خدا چه کم شود گر برسد به کافری
\\
این دل بی‌قرار را از قدحی قرار ده
&&
وین صدف وجود را بخش صفای گوهری
\\
یا برهان ز فکرتم یا برسان به فطرتم
&&
یا به تراش نردبان باز کن از فلک دری
\\
\end{longtable}
\end{center}
