\begin{center}
\section*{غزل شماره ۲۸۵۳: تو ز عشق خود نپرسی که چه خوب و دلربایی}
\label{sec:2853}
\addcontentsline{toc}{section}{\nameref{sec:2853}}
\begin{longtable}{l p{0.5cm} r}
تو ز عشق خود نپرسی که چه خوب و دلربایی
&&
دو جهان به هم برآید چو جمال خود نمایی
\\
تو شراب و ما سبویی تو چو آب و ما چو جویی
&&
نه مکان تو را نه سویی و همه به سوی مایی
\\
به تو دل چگونه پوید نظرم چگونه جوید
&&
که سخن چگونه پرسد ز دهان که تو کجایی
\\
تو به گوش دل چه گفتی که به خنده‌اش شکفتی
&&
به دهان نی چه دادی که گرفت قندخایی
\\
تو به می چه جوش دادی به عسل چه نوش دادی
&&
به خرد چه هوش دادی که کند بلندرایی
\\
ز تو خاک‌ها منقش دل خاکیان مشوش
&&
ز تو ناخوشی شده خوش که خوشی و خوش فزایی
\\
طرب از تو باطرب شد عجب از تو بوالعجب شد
&&
کرم از تو نوش لب شد که کریم و پرعطایی
\\
دل خسته را تو جویی ز حوادثش تو شویی
&&
سخنی به درد گویی که همو کند دوایی
\\
ز تو است ابر گریان ز تو است برق خندان
&&
ز تو خود هزار چندان که تو معدن وفایی
\\
\end{longtable}
\end{center}
