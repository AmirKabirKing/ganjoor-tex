\begin{center}
\section*{غزل شماره ۱۸۸۳: بی او نتوان رفتن بی‌او نتوان گفتن}
\label{sec:1883}
\addcontentsline{toc}{section}{\nameref{sec:1883}}
\begin{longtable}{l p{0.5cm} r}
بی او نتوان رفتن بی‌او نتوان گفتن
&&
بی او نتوان شستن بی‌او نتوان خفتن
\\
ای حلقه زن این در در باز نتان کردن
&&
زیرا که تو هشیاری هر لحظه کشی گردن
\\
گردن ز طمع خیزد زر خواهد و خون ریزد
&&
او عاشق گل خوردن همچون زن آبستن
\\
کو عاشق شیرین خد زر بدهد و جان بدهد
&&
چون مرغ دل او پرد زین گنبد بی‌روزن
\\
این باید و آن باید از شرک خفی زاید
&&
آزاد بود بنده زین وسوسه چون سوسن
\\
آن باید کو آرد او جمله گهر بارد
&&
یا رب که چه‌ها دارد آن ساقی شیرین فن
\\
دو خواجه به یک خانه شد خانه چو ویرانه
&&
او خواجه و من بنده پستی بود و روغن
\\
\end{longtable}
\end{center}
