\begin{center}
\section*{غزل شماره ۲۱۱۹: اگر امروز دلدارم درآید همچو دی خندان}
\label{sec:2119}
\addcontentsline{toc}{section}{\nameref{sec:2119}}
\begin{longtable}{l p{0.5cm} r}
اگر امروز دلدارم درآید همچو دی خندان
&&
فلک اندر سجود آید نهد سر از بن دندان
\\
الا یا صاح لا تعجل بقتلی قد دنا المقتل
&&
ترفق ساعه و اسال وصل من باد بالهجران
\\
بگفتم ای دل خندان چرا دل کرده‌ای سندان
&&
ببین این اشک بی‌پایان طوافی کن بر این طوفان
\\
عذیری منک یا مولا فان الهم استولی
&&
و انت بالوفا اولی فلا تشمت بی الشیطان
\\
مرا گوید چه غم دارد دل آواره چه کم دارم
&&
نه بیمارم نه غمخوارم مرا نگرفت غم چندان
\\
الا یا متلفی زرنی لتحیینی و تنشرنی
&&
قد استولیت فانصرنی فان الفضل بالاحسان
\\
مکن جانا مکن جانا که هم خوبی و هم دانا
&&
کرم منسوخ شد مانا نشد منسوخ ای سلطان
\\
و ما ذنبی سوی انی عدیم الصبر فی فنی
&&
فلا تعرض بذا عنی وجد بالعفو و الغفران
\\
عجب گردد دل و رایش ز بی‌باکی ببخشایش
&&
خدایا مهر افزایش محالی را بساز امکان
\\
اتیناکم اتیناکم فاحیونا بلقیاکم
&&
و سقونا به سقیاکم خذوا بالجود یا اخوان
\\
شفیعی گر تو را گیرد که آن بیچاره می‌میرد
&&
دل تو پند نپذیرد پس این دردی است بی‌درمان
\\
دخلت النار سکرانا حسبت النار اوطانا
&&
الفت النار احیانا فمن ذایألف النیران
\\
چو بیند سوز من گوید که این زرق است یا برقی
&&
چو بیند گریه‌ام گوید که این اشک است یا باران
\\
خلیلی قد دنا نقلی بلا قلب و لا عقل
&&
و لا تعرض و لا تقل و لا تردینی بالنسیان
\\
مرا گوید که درد ما به از قند است و از حلوا
&&
تو را صرع است یا سودا کس از حلوا کند افغان
\\
یقول خادع المعشر بلاء العشق کالسکر
&&
و شوک الحب کالعبهر فما یبکیک یا فتان
\\
ز رنجم گنج‌ها داری ز خارم جفت گلزاری
&&
چه می‌نالی به طراری منم سلطان طراران
\\
جراحات الهوی تشفی کدورات الهوی تصفی
&&
برودات الهوی تدفی و نیران الهوی ریحان
\\
مگر خواهی که خامان را بیندازی ز راه ما
&&
که می‌مویی و می‌گویی چنین مقلوب با ایشان
\\
اذا استغنیت لا تبخل تصدق فی الهوی و انخل
&&
فبیس البخل فی المأکل و نعم الجود فی الانسان
\\
چو در بزم طرب باشی بخیلی کم کن ای ناشی
&&
مبادا یار ز اوباشی کند با تو همین دستان
\\
الا یا ساقیا اوفر و لا تمنن لتستکثر
&&
ادر کاستنا و اسکر فان العیش للسکران
\\
چو خوردی صرف خوش بو را بده یاران می‌جو را
&&
رها کن حرص بدخو را مخور می جز در این میدان
\\
فلا تسق بکاسات صغار بل بطاسات
&&
و امددنا بحرات عظام یا عظیم الشأن
\\
بهل جام عصیرانه که آوردی ز میخانه
&&
سبو را ساز پیمانه که بی‌گه آمدیم ای جان
\\
سقانا ربنا کاسا مراعاه و ایناسا
&&
فنعم الکاس مقیاسا و بیس الهم کالسرحان
\\
بیار آن جام خوش دم را که گردن می‌زند غم را
&&
بیار آن یار محرم را که خاک او است صد خاقان
\\
اذا ما شیت ابقائی فکن یا عشق سقائی
&&
و مل بالفقر تلقائی و انت الدین و الدیان
\\
میی کز روح می‌خیزد به جام فقر می‌ریزد
&&
حیات خلد انگیزد چو ذات عشق بی‌پایان
\\
الا یا ساقی السکری انل کاساتنا تتری
&&
تسلی القلب بالبشری تصفینا عن الشنن
\\
دغل بگذار ای ساقی بکن این جمله در باقی
&&
که صاف صاف راواقی مثال باده خم دان
\\
سنا برق لساقینا بکاسات تلاقینا
&&
تضیء فی تراقینا بنور لاح کالفرقان
\\
زهی آبی که صد آتش از او در دل زند شعله
&&
یکی لون است و صد الوان شود بر روی از او تابان
\\
فماء مشبه النار عزیز مثل دینار
&&
فدیناه به قنطار بلا عد و لا میزان
\\
شرابی چون زر سوری ولی نوری نه انگوری
&&
برد از دیده‌ها کوری بپراند سوی کیوان
\\
اذا افناک سقیاها و زاد الشرب طغواها
&&
فایاکم و ایاها و خلوا دهشته الحیران
\\
چو کرد آن می دگر سانش نمود آن جوش و برهانش
&&
اناالحق بجهد از جانش زهی فر و زهی برهان
\\
\end{longtable}
\end{center}
