\begin{center}
\section*{غزل شماره ۲۵۳۳: برآ بر بام ای عارف بکن هر نیم شب زاری}
\label{sec:2533}
\addcontentsline{toc}{section}{\nameref{sec:2533}}
\begin{longtable}{l p{0.5cm} r}
برآ بر بام ای عارف بکن هر نیم شب زاری
&&
کبوترهای دل‌ها را تویی شاهین اشکاری
\\
بود جان‌های پابسته شوند از بند تن رسته
&&
بود دل‌های افسرده ز حر تو شود جاری
\\
بسی اشکوفه و دل‌ها که بنهادند در گل‌ها
&&
همی‌پایند یاران را به دعوتشان بکن یاری
\\
به کوری دی و بهمن بهاری کن بر این گلشن
&&
درآور باغ مزمن را به پرواز و به طیاری
\\
ز بالا الصلایی زن که خندان است این گلشن
&&
بخندان خار محزون را که تو ساقی اقطاری
\\
دلی دارم پر از آتش بزن بر وی تو آبی خوش
&&
نه ز آب چشمه جیحون از آن آبی که تو داری
\\
به خاک پای تو امشب مبند از پرسش من لب
&&
بیا ای خوب خوش مذهب بکن با روح سیاری
\\
چو امشب خواب من بستی مبند آخر ره مستی
&&
که سلطان قوی دستی و هش بخشی و هشیاری
\\
چرا بستی تو خواب من برای نیکویی کردن
&&
ازیرا گنج پنهانی و اندر قصد اظهاری
\\
زهی بی‌خوابی شیرین بهیتر از گل و نسرین
&&
فزون از شهد و از شکر به شیرینی خوش خواری
\\
به جان پاکت ای ساقی که امشب ترک کن عاقی
&&
که جان از سوز مشتاقی ندارد هیچ صباری
\\
بیا تا روز بر روزن بگردیم ای حریف من
&&
ازیرا مرد خواب افکن درآمد شب به کراری
\\
بر این گردش حسد آرد دوار چرخ گردونی
&&
که این مغز است و آن قشر است و این نور است و آن ناری
\\
چه کوتاه است پیش من شب و روز اندر این مستی
&&
ز روز و شب رهیدم من بدین مستی و خماری
\\
حریف من شو ای سلطان به رغم دیده شیطان
&&
که تا بینی رخ خوبان سر آن شاهدان خاری
\\
مرا امشب شهنشاهی لطیف و خوب و دلخواهی
&&
برآورده‌ست از چاهی رهانیده ز بیماری
\\
به گرد بام می‌گردم که جام حارسان خوردم
&&
تو هم می‌گرد گرد من گرت عزم است میخواری
\\
چو با مستان او گردی اگر مسی تو زر گردی
&&
وگر پایی تو سر گردی وگر گنگی شوی قاری
\\
در این دل موج‌ها دارم سر غواص می‌خارم
&&
ولی کو دامن فهمی سزاوار گهرباری
\\
دهان بستم خمش کردم اگر چه پرغم و دردم
&&
خدایا صبرم افزون کن در این آتش به ستاری
\\
\end{longtable}
\end{center}
