\begin{center}
\section*{غزل شماره ۹۷۰: دیده‌ها شب فراز باید کرد}
\label{sec:0970}
\addcontentsline{toc}{section}{\nameref{sec:0970}}
\begin{longtable}{l p{0.5cm} r}
دیده‌ها شب فراز باید کرد
&&
روز شد دیده باز باید کرد
\\
ترک ما هر طرف که مرکب راند
&&
آن طرف ترک تاز باید کرد
\\
مطبخ جان به سوی بی‌سوییست
&&
پوز آن سو دراز باید کرد
\\
چون چنین کان زر پدید آمد
&&
خویش را جمله گاز باید کرد
\\
جامه عمر را ز آب حیات
&&
چون خضر خوش طراز باید کرد
\\
چون غیورست آن نبات حیات
&&
زین شکر احتراز باید کرد
\\
چون چنین نازنین به خانه ماست
&&
وقت نازست ناز باید کرد
\\
با گل و خار ساختن مردیست
&&
مرد را ساز ساز باید کرد
\\
قبله روی او چو پیدا شد
&&
کعبه‌ها را نماز باید کرد
\\
سجده‌هایی که آن سری باشد
&&
پیش آن سرفراز باید کرد
\\
پیش آن عشق عاقبت محمود
&&
خویشتن را ایاز باید کرد
\\
چون حقیقت نهفته در خمشیست
&&
ترک گفت مجاز باید کرد
\\
\end{longtable}
\end{center}
