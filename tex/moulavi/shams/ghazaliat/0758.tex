\begin{center}
\section*{غزل شماره ۷۵۸: دل من کار تو دارد گل و گلنار تو دارد}
\label{sec:0758}
\addcontentsline{toc}{section}{\nameref{sec:0758}}
\begin{longtable}{l p{0.5cm} r}
دل من کار تو دارد گل و گلنار تو دارد
&&
چه نکوبخت درختی که بر و بار تو دارد
\\
چه کند چرخ فلک را چه کند عالم شک را
&&
چو بر آن چرخ معانی مهش انوار تو دارد
\\
به خدا دیو ملامت برهد روز قیامت
&&
اگر او مهر تو دارد اگر اقرار تو دارد
\\
به خدا حور و فرشته به دو صد نور سرشته
&&
نبرد سر نبرد جان اگر انکار تو دارد
\\
تو کیی آنک ز خاکی تو و من سازی و گویی
&&
نه چنان ساختمت من که کس اسرار تو دارد
\\
ز بلاهای معظم نخورد غم نخورد غم
&&
دل منصور حلاجی که سر دار تو دارد
\\
چو ملک کوفت دمامه بنه ای عقل عمامه
&&
تو مپندار که آن مه غم دستار تو دارد
\\
بمر ای خواجه زمانی مگشا هیچ دکانی
&&
تو مپندار که روزی همه بازار تو دارد
\\
تو از آن روز که زادی هدف نعمت و دادی
&&
نه کلید در روزی دل طرار تو دارد
\\
بن هر بیخ و گیاهی خورد از رزق الهی
&&
همه وسواس و عقیله دل بیمار تو دارد
\\
طمع روزی جان کن سوی فردوس کشان کن
&&
که ز هر برگ و نباتش شکر انبار تو دارد
\\
نه کدوی سر هر کس می راوق تو دارد
&&
نه هر آن دست که خارد گل بی‌خار تو دارد
\\
چو کدو پاک بشوید ز کدو باده بروید
&&
که سر و سینه پاکان می از آثار تو دارد
\\
خمش ای بلبل جان‌ها که غبارست زبان‌ها
&&
که دل و جان سخن‌ها نظر یار تو دارد
\\
بنما شمس حقایق تو ز تبریز مشارق
&&
که مه و شمس و عطارد غم دیدار تو دارد
\\
\end{longtable}
\end{center}
