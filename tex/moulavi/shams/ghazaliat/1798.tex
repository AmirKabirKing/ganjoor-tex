\begin{center}
\section*{غزل شماره ۱۷۹۸: ای یار من ای یار من ای یار بی‌زنهار من}
\label{sec:1798}
\addcontentsline{toc}{section}{\nameref{sec:1798}}
\begin{longtable}{l p{0.5cm} r}
ای یار من ای یار من ای یار بی‌زنهار من
&&
ای دلبر و دلدار من ای محرم و غمخوار من
\\
ای در زمین ما را قمر ای نیم شب ما را سحر
&&
ای در خطر ما را سپر ای ابر شکربار من
\\
خوش می روی در جان من خوش می کنی درمان من
&&
ای دین و ای ایمان من ای بحر گوهردار من
\\
ای شب روان را مشعله ای بی‌دلان را سلسله
&&
ای قبله هر قافله ای قافله سالار من
\\
هم رهزنی هم ره بری هم ماهی و هم مشتری
&&
هم این سری هم آن سری هم گنج و استظهار من
\\
چون یوسف پیغامبری آیی که خواهم مشتری
&&
تا آتشی اندرزنی در مصر و در بازار من
\\
هم موسیی بر طور من عیسی هر رنجور من
&&
هم نور نور نور من هم احمد مختار من
\\
هم مونس زندان من هم دولت خندان من
&&
والله که صد چندان من بگذشته از بسیار من
\\
گویی مرا برجه بگو گویم چه گویم پیش تو
&&
گویی بیا حجت مجو ای بنده طرار من
\\
گویم که گنجی شایگان گوید بلی نی رایگان
&&
جان خواهم وانگه چه جان گویم سبک کن بار من
\\
گر گنج خواهی سر بنه ور عشق خواهی جان بده
&&
در صف درآ واپس مجه ای حیدر کرار من
\\
\end{longtable}
\end{center}
