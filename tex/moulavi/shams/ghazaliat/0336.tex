\begin{center}
\section*{غزل شماره ۳۳۶: بده یک جام ای پیر خرابات}
\label{sec:0336}
\addcontentsline{toc}{section}{\nameref{sec:0336}}
\begin{longtable}{l p{0.5cm} r}
بده یک جام ای پیر خرابات
&&
مگو فردا که فی التأخیر آفات
\\
به جای باده درده خون فرعون
&&
که آمد موسی جانم به میقات
\\
شراب ما ز خون خصم باشد
&&
که شیران را ز صیادیست لذات
\\
چه پرخونست پوز و پنجه شیر
&&
ز خون ما گرفتست این علامات
\\
نگیرم گور و نی هم خون انگور
&&
که من از نفی مستم نی ز اثبات
\\
چو بازم گرد صید زنده گردم
&&
نگردم همچو زاغان گرد اموات
\\
بیا ای زاغ و بازی شو به همت
&&
مصفا شو ز زاغی پیش مصفات
\\
بیفشان وصف‌های باز را هم
&&
مجردتر شو اندر خویش چون ذات
\\
نه خاکست این زمین طشتیست پرخون
&&
ز خون عاشقان و زخم شهمات
\\
خروسا چند گویی صبح آمد
&&
نماید صبح را خود نور مشکات
\\
\end{longtable}
\end{center}
