\begin{center}
\section*{غزل شماره ۱۹۳۶: عاشقان نالان چو نای و عشق همچون نای زن}
\label{sec:1936}
\addcontentsline{toc}{section}{\nameref{sec:1936}}
\begin{longtable}{l p{0.5cm} r}
عاشقان نالان چو نای و عشق همچون نای زن
&&
تا چه‌ها در می دمد این عشق در سرنای تن
\\
هست این سر ناپدید و هست سرنایی نهان
&&
از می لب‌هاش باری مست شد سرنای من
\\
گاه سرنا می نوازد گاه سرنا می گزد
&&
آه از این سرنایی شیرین نوای نی شکن
\\
شمع و شاهد روی او و نقل و باده لعل او
&&
ای ز لعلش مست گشته هم حسن هم بوالحسن
\\
بوحسن گو بوالحسن را کو ز بویش مست شد
&&
وان حسن از بو گذشت و قند دارد در دهن
\\
آسمان چون خرقه رقصان و صوفی ناپدید
&&
ای مسلمانان کی دیده‌ست خرقه رقصان بی‌بدن
\\
خرقه رقصان از تن است و جسم رقصان است ز جان
&&
گردن جان را ببسته عشق جانان در رسن
\\
ای دل مخمور گویی باده‌ات گیرا نبود
&&
باده گیرای او وانگه کسی با خویشتن
\\
\end{longtable}
\end{center}
