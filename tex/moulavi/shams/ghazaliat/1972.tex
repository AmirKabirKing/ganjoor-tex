\begin{center}
\section*{غزل شماره ۱۹۷۲: عاشقا دو چشم بگشا چارجو در خود ببین}
\label{sec:1972}
\addcontentsline{toc}{section}{\nameref{sec:1972}}
\begin{longtable}{l p{0.5cm} r}
عاشقا دو چشم بگشا چارجو در خود ببین
&&
جوی آب و جوی خمر و جوی شیر و انگبین
\\
عاشقا در خویش بنگر سخره مردم مشو
&&
تا فلان گوید چنان و آن فلان گوید چنین
\\
من غلام آن گل بینا که فارغ باشد او
&&
کان فلانم خار خواند وان فلانم یاسمین
\\
دیده بگشا زین سپس با دیده مردم مرو
&&
کان فلانت گبر گوید وان فلانت مرد دین
\\
ای خدا داده تو را چشم بصیرت از کرم
&&
کز خمارش سجده آرد شهپر روح الامین
\\
چشم نرگس را مبند و چشم کرکس را مگیر
&&
چشم اول را مبند و چشم احول را مبین
\\
عاشقان صورتی در صورتی افتاده‌اند
&&
چون مگس کز شهد افتد در طغار دوغگین
\\
شاد باش ای عشقباز ذوالجلال سرمدی
&&
با چنان پرها چه غم باشد تو را از آب و طین
\\
گر همی‌خواهی که جبریلت شود بنده برو
&&
سجده‌ای کن پیش آدم زود ای دیو لعین
\\
بادیه خون خوار اگر واقف شدی از کعبه‌ام
&&
هر طرف گلشن نمودی هر طرف ماء معین
\\
ای به نظاره بد و نیک کسان درمانده
&&
چون بدین راضی شدی یارب تو را بادا معین
\\
چون امانت‌های حق را آسمان طاقت نداشت
&&
شمس تبریزی چگونه گستریدش در زمین
\\
\end{longtable}
\end{center}
