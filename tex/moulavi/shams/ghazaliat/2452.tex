\begin{center}
\section*{غزل شماره ۲۴۵۲: ای دل نگویی چون شدی ور عشق روزافزون شدی}
\label{sec:2452}
\addcontentsline{toc}{section}{\nameref{sec:2452}}
\begin{longtable}{l p{0.5cm} r}
ای دل نگویی چون شدی ور عشق روزافزون شدی
&&
گاهی ز غم مجنون شدی گاهی ز محنت خون شدی
\\
در عشق تو چون دم زدم صد فتنه شد اندر عدم
&&
ای مطرب شیرین قدم می‌زن نوا تا صبحدم
\\
گفتم که شد هنگام می ما غرقه اندر وام می
&&
نی نی رها کن نام می مستان نگر بی‌جام می
\\
تو همچو آتش سرکشی من همچون خاکم مفرشی
&&
در من زدی تو آتشی خوشی خوشی خوشی خوشی
\\
ای نیست بر هستی بزن بر عیش سرمستی بزن
&&
دل بر دل مستی بزن دستی بزن دستی بزن
\\
گفتم مها در ما نگر در چشم چون دریا نگر
&&
آن جا مرو این جا نگر گفتا که خه سودا نگر
\\
ای بلبل از گلشن بگو زان سرو و زان سوسن بگو
&&
زان شاخ آبستن بگو پنهان مکن روشن بگو
\\
آخر همه صورت مبین بنگر به جان نازنین
&&
کز تابش روح الامین چون چرخ شد روی زمین
\\
هر نقش چون اسپر بود در دست صورتگر بود
&&
صورت یکی چادر بود در پرده آزر بود
\\
\end{longtable}
\end{center}
