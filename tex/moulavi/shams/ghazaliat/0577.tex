\begin{center}
\section*{غزل شماره ۵۷۷: چو برقی می‌جهد چیزی عجب آن دلستان باشد}
\label{sec:0577}
\addcontentsline{toc}{section}{\nameref{sec:0577}}
\begin{longtable}{l p{0.5cm} r}
چو برقی می‌جهد چیزی عجب آن دلستان باشد
&&
از آن گوشه چه می‌تابد عجب آن لعل کان باشد
\\
چیست از دور آن گوهر عجب ماهست یا اختر
&&
که چون قندیل نورانی معلق ز آسمان باشد
\\
عجب قندیل جان باشد درفش کاویان باشد
&&
عجب آن شمع جان باشد که نورش بی‌کران باشد
\\
گر از وی درفشان گردی ز نورش بی‌نشان گردی
&&
نگه دار این نشانی را میان ما نشان باشد
\\
ایا ای دل برآور سر که چشم توست روشنتر
&&
بمال آن چشم و خوش بنگر که بینی هر چه آن باشد
\\
چو دیدی تاب و فر او فنا شو زیر پر او
&&
ازیرا بیضه مقبل به زیر ماکیان باشد
\\
چو ما اندر میان آییم او از ما کران گیرد
&&
چو ما از خود کران گیریم او اندر میان باشد
\\
نماید ساکن و جنبان نه جنبانست و نه ساکن
&&
نماید در مکان لیکن حقیقت بی‌مکان باشد
\\
چو آبی را بجنبانی میان نور عکس او
&&
بجنبد از لگن بینی و آن از آسمان باشد
\\
نه آن باشد نه این باشد صلاح الحق و دین باشد
&&
اگر همدم امین باشد بگویم کان فلان باشد
\\
\end{longtable}
\end{center}
