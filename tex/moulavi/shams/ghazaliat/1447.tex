\begin{center}
\section*{غزل شماره ۱۴۴۷: رفتم به طبیب جان گفتم که ببین دستم}
\label{sec:1447}
\addcontentsline{toc}{section}{\nameref{sec:1447}}
\begin{longtable}{l p{0.5cm} r}
رفتم به طبیب جان گفتم که ببین دستم
&&
هم بی‌دل و بیمارم هم عاشق و سرمستم
\\
صد گونه خلل دارم ای کاش یکی بودی
&&
با این همه علت‌ها در شنقصه پیوستم
\\
گفتا که نه تو مردی گفتم که بلی اما
&&
چون بوی توام آمد از گور برون جستم
\\
آن صورت روحانی وان مشرق یزدانی
&&
وان یوسف کنعانی کز وی کف خود خستم
\\
خوش خوش سوی من آمد دستی به دلم برزد
&&
گفتا ز چه دستی تو گفتم که از این دستم
\\
چون عربده می کردم درداد می و خوردم
&&
افروخت رخ زردم وز عربده وارستم
\\
پس جامه برون کردم مستانه جنون کردم
&&
در حلقه آن مستان در میمنه بنشستم
\\
صد جام بنوشیدم صد گونه بجوشیدم
&&
صد کاسه بریزیدم صد کوزه دراشکستم
\\
گوساله زرین را آن قوم پرستیده
&&
گوساله گرگینم گر عشق بنپرستم
\\
بازم شه روحانی می خواند پنهانی
&&
بر می کشدم بالا شاهانه از این پستم
\\
پابست توام جانا سرمست توام جانا
&&
در دست توام جانا گر تیرم وگر شستم
\\
چست توام ار چستم مست توام ار مستم
&&
پست توام ار پستم هست توام ار هستم
\\
در چرخ درآوردی چون مست خودم کردی
&&
چون تو سر خم بستی من نیز دهان بستم
\\
\end{longtable}
\end{center}
