\begin{center}
\section*{غزل شماره ۱۱۸۶: چنان مستم چنان مستم من امروز}
\label{sec:1186}
\addcontentsline{toc}{section}{\nameref{sec:1186}}
\begin{longtable}{l p{0.5cm} r}
چنان مستم چنان مستم من امروز
&&
که پیروزه نمی‌دانم ز پیروز
\\
به هر ره راهبر هشیار باید
&&
در این ره نیست جز مجنون قلاوز
\\
اگر زنده‌ست آن مجنون بیا گو
&&
ز من مجنونی نادر بیاموز
\\
اگر خواهی که تو دیوانه گردی
&&
مثال نقش من بر جامه بردوز
\\
خلیل آن روز با آتش همی‌گفت
&&
اگر مویی ز من باقیست درسوز
\\
بدو می‌گفت آن آتش که ای شه
&&
به پیشت من بمیرم تو برافروز
\\
بهشت و دوزخ آمد دو غلامت
&&
تو از غیر خدا محفوظ و محروز
\\
پیاپی می‌ستان از حق شرابی
&&
ندارد غیر عاشق اندر آن پوز
\\
بده صحت به بیماران عالم
&&
که در صحت نه معلومی نه مهموز
\\
چو ناگفته به پیش روح پیداست
&&
چو پوشیده شود بر روح مرموز
\\
خمش کن از خصال شمس تبریز
&&
همان بهتر که باشد گنج مکنوز
\\
\end{longtable}
\end{center}
