\begin{center}
\section*{غزل شماره ۱۷۵۲: کون خر را نظام دین گفتم}
\label{sec:1752}
\addcontentsline{toc}{section}{\nameref{sec:1752}}
\begin{longtable}{l p{0.5cm} r}
کون خر را نظام دین گفتم
&&
پشک را عنبر ثمین گفتم
\\
اندر این آخرجهان ز گزاف
&&
بس چمن نام هر چمین گفتم
\\
طوق بر گردن کپی بستم
&&
نام اعلا بر اسفلین گفتم
\\
عجز خواهید روح را که ز عجز
&&
صفت روح بهر طین گفتم
\\
حلیه آدم و خلیفه حق
&&
بهر ابلیس و هر لعین گفتم
\\
زاغ را بلبل چمن خواندم
&&
خار را سرو و یاسمین گفتم
\\
دیو را جبرئیل کردم نام
&&
ژاژ را حجت مبین گفتم
\\
ای دریغا که کان نفرین را
&&
از طمع چند آفرین گفتم
\\
از خری بود آن نبد ز خرد
&&
که خر ماده را تکین گفتم
\\
توبه کردم از این خطا گفتن
&&
همه عمرم بس ار همین گفتم
\\
\end{longtable}
\end{center}
