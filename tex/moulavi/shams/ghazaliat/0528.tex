\begin{center}
\section*{غزل شماره ۵۲۸: آن کیست آن آن کیست آن کو سینه را غمگین کند}
\label{sec:0528}
\addcontentsline{toc}{section}{\nameref{sec:0528}}
\begin{longtable}{l p{0.5cm} r}
آن کیست آن آن کیست آن کو سینه را غمگین کند
&&
چون پیش او زاری کنی تلخ تو را شیرین کند
\\
اول نماید مار کر آخر بود گنج گهر
&&
شیرین شهی کاین تلخ را در دم نکوآیین کند
\\
دیوی بود حورش کند ماتم بود سورش کند
&&
وان کور مادرزاد را دانا و عالم بین کند
\\
تاریک را روشن کند وان خار را گلشن کند
&&
خار از کفت بیرون کشد وز گل تو را بالین کند
\\
بهر خلیل خویشتن آتش دهد افروختن
&&
وان آتش نمرود را اشکوفه و نسرین کند
\\
روشن کن استارگان چاره گر بیچارگان
&&
بر بنده او احسان کند هم بند را تحسین کند
\\
جمله گناه مجرمان چون برگ دی ریزان کند
&&
در گوش بدگویان خود عذر گنه تلقین کند
\\
گوید بگو یا ذا الوفا اغفر لذنب قد هفا
&&
چون بنده آید در دعا او در نهان آمین کند
\\
آمین او آنست کو اندر دعا ذوقش دهد
&&
او را برون و اندرون شیرین و خوش چون تین کند
\\
ذوقست کاندر نیک و بد در دست و پا قوت دهد
&&
کاین ذوق زور رستمان جفت تن مسکین کند
\\
با ذوق مسکین رستمی بی‌ذوق رستم پرغمی
&&
گر ذوق نبود یار جان جان را چه باتمکین کند
\\
دل را فرستادم به گه کو تیز داند رفت ره
&&
تا سوی تبریز وفا اوصاف شمس الدین کند
\\
\end{longtable}
\end{center}
