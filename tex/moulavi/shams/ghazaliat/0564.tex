\begin{center}
\section*{غزل شماره ۵۶۴: همی‌بینیم ساقی را که گرد جام می‌گردد}
\label{sec:0564}
\addcontentsline{toc}{section}{\nameref{sec:0564}}
\begin{longtable}{l p{0.5cm} r}
همی‌بینیم ساقی را که گرد جام می‌گردد
&&
ز زر پخته بویی بر که سیم اندام می‌گردد
\\
دگر دل دل نمی‌باشد دگر جان می‌نیارامد
&&
که آن ماه دل و جان‌ها به گرد بام می‌گردد
\\
چو خرمن کرد ماه ما بر آن شد تا بسوزاند
&&
چو پخته کرد جان‌ها را به گرد خام می‌گردد
\\
دل بیچاره مفتون شد خرد افتاد و مجنون شد
&&
به دست اوست آن دانه چه گرد دام می‌گردد
\\
ز گردش فارغست آن مه چه منزل پیش او چه ره
&&
برای حاجت ما دان که چون ایام می‌گردد
\\
شهی که کان و دریاها زکات از وی همی‌خواهند
&&
به گرد کوی هر مفلس برای وام می‌گردد
\\
از این جمله گذر کردم بده ساقی یکی جامی
&&
ز انعامت که این عالم بر آن انعام می‌گردد
\\
شبی گفتی به دلداری شبت را روز گردانم
&&
چو سنگ آسیا جانم بر آن پیغام می‌گردد
\\
به لطف خویش مستش کن خوش جام الستش کن
&&
خراب و می پرستش کن که بی‌آرام می‌گردد
\\
گشا خنب حقایق را بده بی‌صرفه عاشق را
&&
می آشامش کن ایرا دل خیال آشام می‌گردد
\\
بده زان باده خوش بو مپرسش مستحقی تو
&&
ازیرا آفتابی که همه بر عام می‌گردد
\\
نهان ار رهزنی باشد نهان بینا ببر حلقش
&&
چه نقصان قهرمانت را که چون صمصام می‌گردد
\\
اگر گبرم اگر شاکر تویی اول تویی آخر
&&
چو تو پنهان شوی شادی غم و سرسام می‌گردد
\\
دلم پرست و آن اولی که هم تو گویی ای مولی
&&
حدیث خفته‌ای چه بود که بر احلام می‌گردد
\\
\end{longtable}
\end{center}
