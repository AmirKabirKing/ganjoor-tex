\begin{center}
\section*{غزل شماره ۶۴۳: در کوی خرابات مرا عشق کشان کرد}
\label{sec:0643}
\addcontentsline{toc}{section}{\nameref{sec:0643}}
\begin{longtable}{l p{0.5cm} r}
در کوی خرابات مرا عشق کشان کرد
&&
آن دلبر عیار مرا دید نشان کرد
\\
من در پی آن دلبر عیار برفتم
&&
او روی خود آن لحظه ز من باز نهان کرد
\\
من در عجب افتادم از آن قطب یگانه
&&
کز یک نظرش جمله وجودم همه جان کرد
\\
ناگاه یک آهو به دو صد رنگ عیان شد
&&
کز تابش حسنش مه و خورشید فغان کرد
\\
آن آهوی خوش ناف به تبریز روان گشت
&&
بغداد جهان را به بصیرت همدان کرد
\\
آن کس که ورا کرد به تقلید سجودی
&&
فرخنده و بگزیده و محبوب زمان کرد
\\
آن‌ها که بگفتند که ما کامل و فردیم
&&
سرگشته و سودایی و رسوای جهان کرد
\\
سلطان عرفناک بدش محرم اسرار
&&
تا سر تجلی ازل جمله بیان کرد
\\
شمس الحق تبریز چو بگشاد پر عشق
&&
جبریل امین را ز پی خویش دوان کرد
\\
\end{longtable}
\end{center}
