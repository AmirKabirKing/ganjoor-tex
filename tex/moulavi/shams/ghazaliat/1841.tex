\begin{center}
\section*{غزل شماره ۱۸۴۱: تا چه خیال بسته‌ای ای بت بدگمان من}
\label{sec:1841}
\addcontentsline{toc}{section}{\nameref{sec:1841}}
\begin{longtable}{l p{0.5cm} r}
تا چه خیال بسته‌ای ای بت بدگمان من
&&
تا چو خیال گشته‌ام ای قمر چو جان من
\\
از پس مرگ من اگر دیده شود خیال تو
&&
زود روان روان شود در پی تو روان من
\\
بنده‌ام آن جمال را تا چه کنم کمال را
&&
بس بودم کمال تو آن تو است آن من
\\
جانب خویش نگذرم در رخ خویش ننگرم
&&
زانک به عیب ننگرد دیده غیب دان من
\\
چشم مرا نگارگر ساخت به سوی آن قمر
&&
تا جز ماه ننگرد زهره آسمان من
\\
چون نگرم به غیر تو ای به دو دیده سیر تو
&&
خاصه که در دو دیده شد نور تو پاسبان من
\\
من چو که بی‌نشان شدم چون قمر جهان شدم
&&
دیده بود مگر کسی در رخ تو نشان من
\\
شاد شده زمان‌ها از عجب زمانه‌ای
&&
صاف شده مکان‌ها زان مه بی‌مکان من
\\
از تبریز شمس دین تا که فشاند آستین
&&
خشک نشد ز اشک و خون یک نفس آستان من
\\
\end{longtable}
\end{center}
