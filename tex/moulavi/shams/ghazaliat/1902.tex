\begin{center}
\section*{غزل شماره ۱۹۰۲: چرا منکر شدی ای میر کوران}
\label{sec:1902}
\addcontentsline{toc}{section}{\nameref{sec:1902}}
\begin{longtable}{l p{0.5cm} r}
چرا منکر شدی ای میر کوران
&&
نمی‌گویم که مجنون را مشوران
\\
تو می گویی که بنما غیبیان را
&&
ستیران را چه نسبت با ستوران
\\
در این دریا چه کشتی و چه تخته
&&
در این بخشش چه نزدیکان چه دوران
\\
عدم دریاست وین عالم یکی کف
&&
سلیمانی است وین خلقان چو موران
\\
ز جوش بحر آید کف به هستی
&&
دو پاره کف بود ایران و توران
\\
در آن جوشش بگو کوشش چه باشد
&&
چه می لافند از صبر این صبوران
\\
از این بحرند زشتان گشته نغزان
&&
از این موجند شیرین گشته شوران
\\
نپردازی به من ای شمس تبریز
&&
که در عشقت همی‌سوزند حوران
\\
\end{longtable}
\end{center}
