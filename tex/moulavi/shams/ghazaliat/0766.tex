\begin{center}
\section*{غزل شماره ۷۶۶: خضری که عمر ز آبت بکشد دراز گردد}
\label{sec:0766}
\addcontentsline{toc}{section}{\nameref{sec:0766}}
\begin{longtable}{l p{0.5cm} r}
خضری که عمر ز آبت بکشد دراز گردد
&&
در مرگ برخورنده ابدا فرازگردد
\\
چو نظر کنی به بالا سوی آسمان اعلا
&&
دو هزار در ز رحمت ز بهشت باز گردد
\\
چو فتاد سایه تو سوی مفسدان مجرم
&&
همه جرم‌های ایشان چله و نماز گردد
\\
چو رکاب مصطفایی سوی عفو روی آرد
&&
دو هزار بولهب هم خوش و پرنیاز گردد
\\
چو دو دست همچو بحرت به کرم گهرفشان شد
&&
رخ چون زرم زر آرد که به گرد گاز گردد
\\
کف تست کیمیایی لب بحر کبریایی
&&
چه عجب که نیم حبه ز کفت رکاز گردد
\\
دو هزار جان و دیده ز فزع عنان کشیده
&&
چو صلای وصل آید گه ترک تاز گردد
\\
همه زهر دین و دنیا ز تو شهد و نوش آمد
&&
غم و درد سینه سوزان ز تو دلنواز گردد
\\
همه دامن تو گیرد دل و این قدر نداند
&&
که به گرد شیر آهو به صد احتراز گردد
\\
در وصل چون ببستی و به لامکان نشستی
&&
ز کجا رسد گشایش چو دری فراز گردد
\\
خمش و سخن رها کن جز اله را تو لا کن
&&
به فنا چو ساز گیری همه کارساز گردد
\\
\end{longtable}
\end{center}
