\begin{center}
\section*{غزل شماره ۲۲۹۳: یکی ماهی همی‌بینم برون از دیده در دیده}
\label{sec:2293}
\addcontentsline{toc}{section}{\nameref{sec:2293}}
\begin{longtable}{l p{0.5cm} r}
یکی ماهی همی‌بینم برون از دیده در دیده
&&
نه او را دیده‌ای دیده نه او را گوش بشنیده
\\
زبان و جان و دل را من نمی‌بینم مگر بیخود
&&
از آن دم که نظر کردم در آن رخسار دزدیده
\\
گر افلاطون بدیدستی جمال و حسن آن مه را
&&
ز من دیوانه‌تر گشتی ز من بتر بشوریده
\\
قدم آیینه حادث حدث آیینه قدمت
&&
در آن آیینه این هر دو چو زلفینش بپیچیده
\\
یکی ابری ورای حس که بارانش همه جان است
&&
نثار خاک جسم او چه باران‌ها بباریده
\\
قمررویان گردونی بدیده عکس رخسارش
&&
خجل گشته از آن خوبی پس گردن بخاریده
\\
ابد دست ازل بگرفت سوی قصر آن مه برد
&&
بدیده هر دو را غیرت بدین هر دو بخندیده
\\
که گرداگرد قصر او چه شیرانند کز غیرت
&&
به قصد خون جانبازان و صدیقان بغریده
\\
به ناگه جست از لفظم که آن شه کیست شمس الدین
&&
شه تبریز و خون من در این گفتن بجوشیده
\\
\end{longtable}
\end{center}
