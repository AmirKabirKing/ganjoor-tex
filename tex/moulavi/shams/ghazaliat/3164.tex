\begin{center}
\section*{غزل شماره ۳۱۶۴: مست و خوشی باده کجا خوردهٔ}
\label{sec:3164}
\addcontentsline{toc}{section}{\nameref{sec:3164}}
\begin{longtable}{l p{0.5cm} r}
مست و خوشی باده کجا خورده‌ای؟
&&
این مه نو چیست که آورده‌ای؟
\\
ساغر شاهانه گرفتی به کف
&&
گلشکر نادره پرورده‌ای
\\
پردهٔ ناموس کی خواهی درید؟
&&
کآفت عقل و ادب و پرده‌ای
\\
می‌شکفد از نظرت باغ دل
&&
ای که بهار دل افسرده‌ای
\\
آتش در ملک سلیمان زدی
&&
ای که تو موری بنیازرده‌ای
\\
در سفر ای شاه سبک روح من
&&
زیر قدم چشم و دل اسپرده‌ای
\\
دارد خوبی و کشی بی‌شمار
&&
روی کسی کش بک اشمرده‌ای
\\
بنده کن هر دل آزاده‌ای
&&
زنده کن هر بدن مرده‌ای
\\
می‌کندت لابه و دریوزه جان
&&
جان ببر آنجا که دلم برده‌ای
\\
جان دو صد قرن در انگشت تست
&&
چونت بگویم؟! که توده مرده‌ای
\\
بس کن تا مطرب و ساقی شود
&&
آنکه می از باغ وی افشرده‌ای
\\
\end{longtable}
\end{center}
