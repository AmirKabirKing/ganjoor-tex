\begin{center}
\section*{غزل شماره ۱۶۴۰: گر تو خواهی که تو را بی‌کس و تنها نکنم}
\label{sec:1640}
\addcontentsline{toc}{section}{\nameref{sec:1640}}
\begin{longtable}{l p{0.5cm} r}
گر تو خواهی که تو را بی‌کس و تنها نکنم
&&
وامقت باشم هر لحظه و عذرا نکنم
\\
این تعلق به تو دارد سر رشته مگذار
&&
کژ مباز ای کژ کژباز مکن تا نکنم
\\
گفته‌ای جان دهمت نان جوین می ندهی
&&
بی‌خبر دانیم ار هیچ مکافا نکنم
\\
گوش تو تا بنمالم نگشاید چشمت
&&
دهمت بیم مبارات تو اما نکنم
\\
متفرق شود اجزای تو هنگام اجل
&&
تو گمان برده که جمعیت اجزا نکنم
\\
منشی روز و شبم نیست شود هست کنم
&&
پس چرا روز تو را عاقبت انشا نکنم
\\
هر دمی حشر نوستت ز ترح تا به فرح
&&
پس چرا صبر تو را شکر شکرخا نکنم
\\
هر کسی عاشق کاری ز تقاضای من است
&&
پس چه شد کار جزا را که تقاضا نکنم
\\
تا ز زهدان جهان همچو جنینت نبرم
&&
در جهان خرد و عقل تو را جا نکنم
\\
گلشن عقل و خرد پرگل و ریحان طری است
&&
چشم بستی به ستیزه که تماشا نکنم
\\
طبل باز شهم ای باز بر این بانگ بیا
&&
پیش از آن که بروم نظم غزل‌ها نکنم
\\
\end{longtable}
\end{center}
