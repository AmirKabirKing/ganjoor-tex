\begin{center}
\section*{غزل شماره ۱۲۸۸: چو رو نمود به منصور وصل دلدارش}
\label{sec:1288}
\addcontentsline{toc}{section}{\nameref{sec:1288}}
\begin{longtable}{l p{0.5cm} r}
چو رو نمود به منصور وصل دلدارش
&&
روا بود که رساند به اصل دل دارش
\\
من از قباش ربودم یکی کلهواری
&&
بسوخت عقل و سر و پایم از کلهوارش
\\
شکستم از سر دیوار باغ او خاری
&&
چه خارخار و طلب در دلست از آن خارش
\\
چو شیرگیر شد این دل یکی سحر ز میش
&&
سزد که زخم کشد از فراق سگسارش
\\
اگر چه کره گردون حرون و تند نمود
&&
به دست عشق وی آمد شکال و افسارش
\\
اگر چه صاحب صدرست عقل و بس دانا
&&
به جام عشق گرو شد ردا و دستارش
\\
بسا دلا که به زنهار آمد از عشقش
&&
کشان کشان بکشیدش نداد زنهارش
\\
به روز سرد یکی پوستین بد اندر جو
&&
به عور گفتم درجه به جو برون آرش
\\
نه پوستین بود آن خرس بود اندر جو
&&
فتاده بود همی‌برد آب جوبارش
\\
درآمد او به طمع تا به پوست خرس رسید
&&
به دست خرس بکرد آن طمع گرفتارش
\\
بگفتمش که رها کن تو پوستین بازآ
&&
چه دور و دیر بماندی به رنج و پیکارش
\\
بگفت رو که مرا پوستین چنان بگرفت
&&
که نیست امید رهایی ز چنگ جبارش
\\
هزار غوطه مرا می‌دهد به هر ساعت
&&
خلاص نیست از آن چنگ عاشق افشارش
\\
خمش بس است حکایت اشارتی بس کن
&&
چه حاجتست بر عقل طول طومارش
\\
\end{longtable}
\end{center}
