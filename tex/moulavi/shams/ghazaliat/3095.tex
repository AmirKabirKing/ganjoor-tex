\begin{center}
\section*{غزل شماره ۳۰۹۵: از این درخت بدان شاخ و بر نمی‌بینی}
\label{sec:3095}
\addcontentsline{toc}{section}{\nameref{sec:3095}}
\begin{longtable}{l p{0.5cm} r}
از این درخت بدان شاخ و بر نمی‌بینی
&&
سه شاخ داری کور و کری و گرگینی
\\
میان آب دری و ز آب می‌پرسی
&&
میان گنج زری مس قلب می‌چینی
\\
خدات گوید تدبیر چشم روشن کن
&&
تو چشم را بگذاری و می‌کنی بینی
\\
اگر چه تیره شبی رو به صبح صادق آر
&&
مگو که صبحم صبحی ولی دروغینی
\\
رسید نعره عشرت ز ناصر منصور
&&
غدوت اشربها و الخمار یسقینی
\\
مجردان همه شب نقل و باده می‌نوشند
&&
در این خوشی که در افواه سابق الدینی
\\
مثال دنب ز پس مانده‌ای ز سرمستان
&&
تو مست بستر گرمی حریف بالینی
\\
چو غافلی ز ثواب و مقام مسکینان
&&
مراقب ذهبی دشمن مساکینی
\\
گلست قوت تو همچون زنان آبستن
&&
تو را از آن چه که در روضه و بساتینی
\\
دی و بهار همه سال مار خاک خورد
&&
اگر انار زند خنده تین کند تینی
\\
اگر چه نقش لطیفی نه سر به سر نقشی
&&
وگر چه زاده طینی نه سر به سر طینی
\\
هلا خموش که دیوان دف تو تر کردند
&&
کانیس دفتری و طالب دواوینی
\\
\end{longtable}
\end{center}
