\begin{center}
\section*{غزل شماره ۲۰۰۹: نک بهاران شد صلا ای لولیان}
\label{sec:2009}
\addcontentsline{toc}{section}{\nameref{sec:2009}}
\begin{longtable}{l p{0.5cm} r}
نک بهاران شد صلا ای لولیان
&&
بانگ نای و سبزه و آب روان
\\
لولیان از شهر تن بیرون شوید
&&
لولیان را کی پذیرد خان و مان
\\
دیگران بردند حسرت زین جهان
&&
حسرتی بنهیم در جان جهان
\\
با جهان بی‌وفا ما آن کنیم
&&
هرچ او کرده‌ست با آن دیگران
\\
تا حریف خود ببیند او یکی
&&
امتحان او بیابد امتحان
\\
نی غلط گفتم جهان چون عاشق است
&&
او به جان جوید جفای نیکوان
\\
جان عاشق زنده از جور و جفاست
&&
ای مسلمان جان که را دارد زیان
\\
راه صحرا را فروبست این سخن
&&
کس نجوید راه صحرا را دهان
\\
تو بگو دارد دهان تنگ یار
&&
با لب بسته گشاد بی‌کران
\\
هر که بر وی آن لبان صحرا نشد
&&
او نه صحرا داند و نی آشیان
\\
هر که بر وی زان قمر نوری نتافت
&&
او چه بیند از زمین و آسمان
\\
هر کسی را کاین غزل صحرا شود
&&
عیش بیند زان سوی کون و مکان
\\
\end{longtable}
\end{center}
