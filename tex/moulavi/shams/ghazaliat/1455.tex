\begin{center}
\section*{غزل شماره ۱۴۵۵: ای خواجه سلام علیک من عزم سفر دارم}
\label{sec:1455}
\addcontentsline{toc}{section}{\nameref{sec:1455}}
\begin{longtable}{l p{0.5cm} r}
ای خواجه سلام علیک من عزم سفر دارم
&&
وز بام فلک پنهان من راه گذر دارم
\\
جان عزم سفر دارد تا معدن و اصل خود
&&
زان سو که نظر بخشد آن سوی نظر دارم
\\
نک می کشدم سیلم آن سوی که بد میلم
&&
کز فرقت آن دریا بس گرم جگر دارم
\\
می تازم ترکانه تا حضرت خاقانی
&&
کز وی مثل خرگه صد بند کمر دارم
\\
چون سایه فنا گردم در تابش خورشیدی
&&
کاندر پی او دایم من سیر قمر دارم
\\
چون لعل ز خورشیدش جز گرمی و جز تابش
&&
من فر دگر گیرم من عشق دگر دارم
\\
گر بشکند این جوزم هم مغزم و هم نغزم
&&
ور بشکندم چون نی صد قند شکر دارم
\\
چون سروم و چون سوسن هم بسته هم آزادم
&&
چون سنگم و چون آهن در سینه شرر دارم
\\
یا من هو فی قلبی یسبی ادبی یسبی
&&
حسبی ابدا حسبی آنچ از تو به بر دارم
\\
مولای فنی صبری لا تخرج من صدری
&&
لا تبعد نستبری کز هجر ضرر دارم
\\
ای عشق صلا گفتی می آیم بسم الله
&&
آخر به چه آرامم گر از تو حذر دارم
\\
گر در دل تابوتم مهر تو بود قوتم
&&
قوت ملکی دارم گر شکل بشر دارم
\\
آفندی کلیتیشی کالیسو کیتیشی
&&
شیلیسو نسندیشی دل زیر و زبر دارم
\\
افندی مناخوسی بویسی کلیمو بویسی
&&
تینما خو نتیلوسی یاد تو سمر دارم
\\
باقیش بفرما تو ای خسرو دریاخو
&&
بستم چو صدف من لب یعنی که گهر دارم
\\
\end{longtable}
\end{center}
