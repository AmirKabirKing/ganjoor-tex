\begin{center}
\section*{غزل شماره ۱۸۵۲: چو آمد روی مه رویم کی باشم من که باشم من}
\label{sec:1852}
\addcontentsline{toc}{section}{\nameref{sec:1852}}
\begin{longtable}{l p{0.5cm} r}
چو آمد روی مه رویم کی باشم من که باشم من
&&
چو زاید آفتاب جان کجا ماند شب آبستن
\\
چه باشد خار گریان رو که چون سور بهار آید
&&
نگیرد رنگ و بوی خوش نگیرد خوی خندیدن
\\
چه باشد سنگ بی‌قیمت چو خورشید اندر او تابد
&&
که از سنگی برون ناید نگردد گوهر روشن
\\
چه باشد شیر نوزاده ز یک گربه زبون باشد
&&
چو شیر شیر آشامد شود او شیر شیرافکن
\\
یکی قطره منی بودی منی انداز کردت حق
&&
چو سیمابی بدی وز حق شدستی شاه سیمین تن
\\
منی دیگری داری که آن بحر است و این قطره
&&
قراضه است این منی تو و آن من هست چون معدن
\\
منی حق شود پیدا منی ما فنا گردد
&&
بسوزد خرمن هستی چو ماه حق کند خرمن
\\
گرفتم دامن جان را که پوشیده‌ست تشریفی
&&
که آن را نی گریبان است و نی تیریز و نی دامن
\\
قبای اطلس معنی که برقش کفرسوز آمد
&&
گر این اطلس همی‌خواهی پلاس حرص را برکن
\\
اگر پوشیدم این اطلس سخن پوشیده گویم بس
&&
اگر خود صد زبان دارم نگویم حرف چون سوسن
\\
چنین خلعت بدش در سر که نامش کرد مدثر
&&
شعارش صورت نیر دثارش سیرت احسن
\\
\end{longtable}
\end{center}
