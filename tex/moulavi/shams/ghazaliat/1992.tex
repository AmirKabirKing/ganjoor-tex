\begin{center}
\section*{غزل شماره ۱۹۹۲: خوی با ما کن و با بی‌خبران خوی مکن}
\label{sec:1992}
\addcontentsline{toc}{section}{\nameref{sec:1992}}
\begin{longtable}{l p{0.5cm} r}
خوی با ما کن و با بی‌خبران خوی مکن
&&
دم هر ماده خری را چو خران بوی مکن
\\
اول و آخر تو عشق ازل خواهد بود
&&
چون زن فاحشه هر شب تو دگر شوی مکن
\\
دل بنه بر هوسی که دل از آن برنکنی
&&
شیرمردا دل خود را سگ هر کوی مکن
\\
هم بدان سو که گه درد دوا می خواهی
&&
وقف کن دیده و دل روی به هر سوی مکن
\\
همچو اشتر بمدو جانب هر خاربنی
&&
ترک این باغ و بهار و چمن و جوی مکن
\\
هان که خاقان بنهاده است شهانه بزمی
&&
اندر این مزبله از بهر خدا طوی مکن
\\
میر چوگانی ما جانب میدان آمد
&&
پی اسپش دل و جان را هله جز گوی مکن
\\
روی را پاک بشو عیب بر آیینه منه
&&
نقد خود را سره کن عیب ترازوی مکن
\\
جز بر آن که لبت داد لب خود مگشا
&&
جز سوی آنک تکت داد تکاپوی مکن
\\
روی و مویی که بتان راست دروغین می دان
&&
نامشان را تو قمرروی زره موی مکن
\\
بر کلوخی است رخ و چشم و لب عاریتی
&&
پیش بی‌چشم به جد شیوه ابروی مکن
\\
قامت عشق صلا زد که سماع ابدی است
&&
جز پی قامت او رقص و هیاهوی مکن
\\
دم مزن ور بزنی زیر لب آهسته بزن
&&
دم حجاب است یکی تو کن و صدتوی مکن
\\
\end{longtable}
\end{center}
