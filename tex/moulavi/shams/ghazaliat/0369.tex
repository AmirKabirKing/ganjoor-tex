\begin{center}
\section*{غزل شماره ۳۶۹: در شهر شما یکی نگاریست}
\label{sec:0369}
\addcontentsline{toc}{section}{\nameref{sec:0369}}
\begin{longtable}{l p{0.5cm} r}
در شهر شما یکی نگاریست
&&
کز وی دل و عقل بی‌قراریست
\\
هر نفسی را از او نصیبیست
&&
هر باغی را از او بهاریست
\\
در هر کویی از او فغانیست
&&
در هر راهی از او غباریست
\\
در هر گوشی از او سماعیست
&&
هر چشم از او در اعتباریست
\\
در کار شوید ای حریفان
&&
کاین جا ما را عظیم کاریست
\\
پنهان یاری به گوش من گفت
&&
کاین جا پنهان لطیف یاریست
\\
او بد که به این طریق می‌گفت
&&
کز تعبیه‌هاش دل نزاریست
\\
او بود رسول خویش و مرسل
&&
کان لهجه از آن شهریاریست
\\
نوحست و امان غرقگانست
&&
روحست و نهان و آشکاریست
\\
گرد ترشان مگرد زین پس
&&
چون پهلوی تو شکرنثاریست
\\
گرد شکران طبع کم گرد
&&
کان شهوت نیز برگذاریست
\\
این جا شکریست بی‌نهایت
&&
این جا سر وقت پایداریست
\\
خاموش کن ای دل و مپندار
&&
کو را حدیست یا کناریست
\\
\end{longtable}
\end{center}
