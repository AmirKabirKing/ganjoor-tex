\begin{center}
\section*{غزل شماره ۱۷۹۳: این کیست این این کیست این هذا جنون العاشقین}
\label{sec:1793}
\addcontentsline{toc}{section}{\nameref{sec:1793}}
\begin{longtable}{l p{0.5cm} r}
این کیست این این کیست این هذا جنون العاشقین
&&
از آسمان خوشتر شده در نور او روی زمین
\\
بی‌هوشی جان‌هاست این یا گوهر کان‌هاست این
&&
یا سرو بستان‌هاست این یا صورت روح الامین
\\
سرمستی جان جهان معشوقه چشم و دهان
&&
ویرانی کسب و دکان یغماجی تقوا و دین
\\
خورشید و ماه از وی خجل گوهر نثار سنگ دل
&&
کز بیم او پشمین شود هر لحظه کوه آهنین
\\
خورشید اندر سایه‌اش افزون شده سرمایه‌اش
&&
صد ماه اندر خرمنش چون نسر طایر دانه چین
\\
بسم الله ای روح البقا بسم الله ای شیرین لقا
&&
بسم الله ای شمس الضحا بسم الله ای عین الیقین
\\
هین روی‌ها را تاب ده هین کشت دل را آب ده
&&
نعلین برون کن برگذر بر تارک جان‌ها نشین
\\
ای هوش ما از خود برو وی گوش ما مژده شنو
&&
وی عقل ما سرمست شو وی چشم ما دولت ببین
\\
ایوب را آمد نظر یعقوب را آمد پسر
&&
خورشید شد جفت قمر در مجلس آ عشرت گزین
\\
من کیسه‌ها می دوختم در حرص زر می سوختم
&&
ترک گدارویی کنم چون گنج دیدم در کمین
\\
ای شهسوار امر قل ای پیش عقلت نفس کل
&&
چون کودکی کز کودکی وز جهل خاید آستین
\\
چون بیندش صاحب نظر صدتو شود او را بصر
&&
دستک زنان بالای سر گوید که یا نعم المعین
\\
در سایه سدره نظر جبریل خو آمد بشر
&&
درخورد او نبود دگر مهمانی عجل سمین
\\
بر خوان حق ره یافت او با خاصگان دریافت او
&&
بنهاده بر کف‌ها طبق بهر نثارش حور عین
\\
این نامه اسرار جان تا چند خوانی بر چپان
&&
این نامه می پرد عیان تا کف اصحاب الیمین
\\
\end{longtable}
\end{center}
