\begin{center}
\section*{غزل شماره ۲۶۲۸: هر روز بگه ای شه دلدار درآیی}
\label{sec:2628}
\addcontentsline{toc}{section}{\nameref{sec:2628}}
\begin{longtable}{l p{0.5cm} r}
هر روز بگه ای شه دلدار درآیی
&&
جان را و جهان را شکفانی و فزایی
\\
یا رب چه خجسته‌ست ملاقات جمالت
&&
آن لحظه که چون بدر بر این صدر برآیی
\\
هر جا که ملاقات دو یار است اثر توست
&&
خود ذوق و نمک بخش وصالی و لقایی
\\
معنی ندهد وصلت این حرف بدان حرف
&&
تا تو ننهی در کلمه فایده زایی
\\
ای داده تو دندان و شکرها که بخایند
&&
دندان دگر داده پی فایده خایی
\\
بیزارم از آن گوش که آواز نیاشنود
&&
و آگاه نشد از خرد و دانش نایی
\\
این مشک به خود چون رود و آب کشاند
&&
تا خواجه سقا نکند جهد سقایی
\\
این چرخ که می‌گردد بی‌آب نگردد
&&
تا سر نبود پای کجا یابد پایی
\\
هان ای دل پرسنده که دلدار کجای است
&&
تو ای دل جوینده و پرسنده کجایی
\\
تیهی ز کجا یابد گلزار و شقایق
&&
پیهی ز کجا یابد تمییز ضیایی
\\
اصداف حواسی که به شب ماند ز در دور
&&
دانند که در هست ز دریای عطایی
\\
درهاست در آن بحر در اصداف نگنجد
&&
آن سوی برو ای صدف این سوی چه پایی
\\
آن نیستی ای خواجه که کعبه به تو آید
&&
گوید بر ما آی اگر حاجی مایی
\\
این کعبه نه جا دارد نی گنجد در جا
&&
می‌گوید العزه و الحسن ردایی
\\
هین غرقه عزت شو و فانی ردا شو
&&
تا جان دهدت چونک ببیند که فنایی
\\
خامش کن و از راه خموشی به عدم رو
&&
معدوم چو گشتی همگی حد و ثنایی
\\
\end{longtable}
\end{center}
