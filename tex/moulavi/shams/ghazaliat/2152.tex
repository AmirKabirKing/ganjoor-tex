\begin{center}
\section*{غزل شماره ۲۱۵۲: سخت خوش است چشم تو و آن رخ گلفشان تو}
\label{sec:2152}
\addcontentsline{toc}{section}{\nameref{sec:2152}}
\begin{longtable}{l p{0.5cm} r}
سخت خوش است چشم تو و آن رخ گلفشان تو
&&
دوش چه خورده‌ای دلا راست بگو به جان تو
\\
فتنه گر است نام تو پرشکر است دام تو
&&
باطرب است جام تو بانمک است نان تو
\\
مرده اگر ببیندت فهم کند که سرخوشی
&&
چند نهان کنی که می فاش کند نهان تو
\\
بوی کباب می‌زند از دل پرفغان من
&&
بوی شراب می‌زند از دم و از فغان تو
\\
بهر خدا بیا بگو ور نه بهل مرا که تا
&&
یک دو سخن به نایبی بردهم از زبان تو
\\
خوبی جمله شاهدان مات شد و کساد شد
&&
چون بنمود ذره‌ای خوبی بی‌کران تو
\\
بازبدید چشم ما آنچ ندید چشم کس
&&
بازرسید پیر ما بیخود و سرگران تو
\\
هر نفسی بگوییم عقل تو کو چه شد تو را
&&
عقل نماند بنده را در غم و امتحان تو
\\
هر سحری چو ابر دی بارم اشک بر درت
&&
پاک کنم به آستین اشک ز آستان تو
\\
مشرق و مغرب ار روم ور سوی آسمان شوم
&&
نیست نشان زندگی تا نرسد نشان تو
\\
زاهد کشوری بدم صاحب منبری بدم
&&
کرد قضا دل مرا عاشق و کف زنان تو
\\
از می این جهانیان حق خدا نخورده‌ام
&&
سخت خراب می‌شوم خائفم از گمان تو
\\
صبر پرید از دلم عقل گریخت از سرم
&&
تا به کجا کشد مرا مستی بی‌امان تو
\\
شیر سیاه عشق تو می‌کند استخوان من
&&
نی تو ضمان من بدی پس چه شد این ضمان تو
\\
ای تبریز بازگو بهر خدا به شمس دین
&&
کاین دو جهان حسد برد بر شرف جهان تو
\\
\end{longtable}
\end{center}
