\begin{center}
\section*{غزل شماره ۱۰۱۲: یا شبه الطیف لی انت قریب بعید}
\label{sec:1012}
\addcontentsline{toc}{section}{\nameref{sec:1012}}
\begin{longtable}{l p{0.5cm} r}
یا شبه الطیف لی انت قریب بعید
&&
جمله ارواحنا تغمس فیما ترید
\\
نوبت آدم گذشت نوبت مرغان رسید
&&
طبل قیامت زدند خیز که فرمان رسید
\\
انت لطیف الفعال انت لذیذ المقال
&&
انت جمال الکمال زدت فهل من مزید
\\
از پس دور قمر دولت بگشاد در
&&
دلق برون کن ز سر خلعت سلطان رسید
\\
جاء اوان السرور زال زمان الفتور
&&
لیس لدنیا غرور یا سندی لا تحید
\\
دیو و پری داشت تخت ظلم از آن بود سخت
&&
دیو رها کرد رخت چتر سلیمان رسید
\\
هل طرب یا غلام فاملا کاس المدام
&&
انت بدار السلام ساکن قصر مشید
\\
عشق چه خوش حاکمیست ظالم و بی‌قول نیست
&&
حاجت لاحول نیست دیو مسلمان رسید
\\
یا لمع المشرق مثلک لم یخلق
&&
خذ بیدی ارتقی نحوک انت المجید
\\
عاشق از دست شد نیست شد و هست شد
&&
بلبل جان مست شد سوی گلستان رسید
\\
پرده برانداخت حور جمله جهان همچو طور
&&
زیر و زبر بست نور موسی عمران رسید
\\
هر چه خیال نکوست عشق هیولای اوست
&&
صورت از رشک حق پرده گر جان رسید
\\
هست تنت چون غبار بر سر بادی سوار
&&
چونک جدا گشت باد خاک به ماچان رسید
\\
اعلم ان الغبار مرتفع بالریاح
&&
مثل هوی اختفی وسط صیاح شدید
\\
\end{longtable}
\end{center}
