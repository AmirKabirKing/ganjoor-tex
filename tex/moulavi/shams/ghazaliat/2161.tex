\begin{center}
\section*{غزل شماره ۲۱۶۱: اگر نه عاشق اویم چه می‌پویم به کوی او}
\label{sec:2161}
\addcontentsline{toc}{section}{\nameref{sec:2161}}
\begin{longtable}{l p{0.5cm} r}
اگر نه عاشق اویم چه می‌پویم به کوی او
&&
وگر نه تشنه اویم چه می‌جویم به جوی او
\\
بر این مجنون چه می‌بندم مگر بر خویش می‌خندم
&&
که او زنجیر نپذیرد مگر زنجیر موی او
\\
ببر عقلم ببر هوشم که چون پنبه‌ست در گوشم
&&
چو گوشم رست از این پنبه درآید های هوی او
\\
همی‌گوید دل زارم که با خود عهدها دارم
&&
نیاشامم شراب خوش مگر خون عدوی او
\\
دلم را می‌کند پرخون سرم را پرمی و افیون
&&
دل من شد تغار او سر من شد کدوی او
\\
چه باشد ماه یا زهره چو او بگشود آن چهره
&&
چه دارد قند یا حلوا ز شیرینی خوی او
\\
مرا گوید چرا زاری ز ذوق آن شکرباری
&&
مرا گوید چرا زردی ز لاله ستان روی او
\\
مرا هر دم برانگیزی به سوی شمس تبریزی
&&
بگو در گوش من ای دل چه می‌تازی به سوی او
\\
\end{longtable}
\end{center}
