\begin{center}
\section*{غزل شماره ۲۶۱۳: ای شادی آن روزی کز راه تو بازآیی}
\label{sec:2613}
\addcontentsline{toc}{section}{\nameref{sec:2613}}
\begin{longtable}{l p{0.5cm} r}
ای شادی آن روزی کز راه تو بازآیی
&&
در روزن جان تابی چون ماه ز بالایی
\\
زان ماه پرافزایش آن فارغ از آرایش
&&
این فرش زمینی را چون عرش بیارایی
\\
بس عاقل پابسته کز خویش شود رسته
&&
بس جان که ز سر گیرد قانون شکرخایی
\\
زین منزل شش گوشه بی‌مرکب و بی‌توشه
&&
بس قافله ره یابد در عالم بی‌جایی
\\
روشن کن جان من تا گوید جان با تن
&&
کامروز مرا بنگر ای خواجه فردایی
\\
تو آبی و من جویم جز وصل تو کی جویم
&&
رونق نبود جو را چون آب بنگشایی
\\
ای شاد تو از پیشی یعنی ز همه بیشی
&&
والله که چو با خویشی از خویش نیاسایی
\\
در جستن دل بودم بر راه خودش دیدم
&&
افتاده در این سودا چون مردم صفرایی
\\
شمس الحق تبریزی پالود مرا هجرت
&&
جز عشق نبینی گر صد بار بپالایی
\\
\end{longtable}
\end{center}
