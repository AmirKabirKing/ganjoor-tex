\begin{center}
\section*{غزل شماره ۵۴۰: مستی سلامت می‌کند پنهان پیامت می‌کند}
\label{sec:0540}
\addcontentsline{toc}{section}{\nameref{sec:0540}}
\begin{longtable}{l p{0.5cm} r}
مستی سلامت می‌کند پنهان پیامت می‌کند
&&
آن کو دلش را برده‌ای جان هم غلامت می‌کند
\\
ای نیست کرده هست را بشنو سلام مست را
&&
مستی که هر دو دست را پابند دامت می‌کند
\\
ای آسمان عاشقان ای جان جان عاشقان
&&
حسنت میان عاشقان نک دوستکامت می‌کند
\\
ای چاشنی هر لبی وی قبله هر مذهبی
&&
مه پاسبانی هر شبی بر گرد بامت می‌کند
\\
ای دل چه مستی و خوشی سلطانی و سلطان وشی
&&
با این دماغ و سرکشی چون عشق رامت می‌کند
\\
آن کو ز خاکی جان کند او دود را کیوان کند
&&
ای خاک تن وی دود دل بنگر کدامت می‌کند
\\
بستان ز شاه ساقیان سرمست شو چون باقیان
&&
گر نیم مست ناقصی مست تمامت می‌کند
\\
از لب سلامت ای احد چون برگ بیرون می‌جهد
&&
اندازه لب نیست این این لطف عامت می‌کند
\\
ماه از غمت دو نیم شد رخساره‌ها چون سیم شد
&&
قد الف چون جیم شد وین جیم جامت می‌کند
\\
در عشق زاری‌ها نگر وین اشک باری‌ها نگر
&&
وان پخته کاری‌ها نگر کان رطل خامت می‌کند
\\
ای باده خوش رنگ و بو بنگر که دست جود او
&&
بر جان حلالت می‌کند بر تن حرامت می‌کند
\\
پس تن نباشم جان شوم جوهر نباشم کان شوم
&&
ای دل مترس از نام بد کو نیک نامت می‌کند
\\
بس کن رها کن گفت و گو نی نظم گو نی نثر گو
&&
کان حیله ساز و حیله جو بدو کلامت می‌کند
\\
\end{longtable}
\end{center}
