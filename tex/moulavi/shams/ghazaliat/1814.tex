\begin{center}
\section*{غزل شماره ۱۸۱۴: عشق تو آورد قدح پر ز بلای دل من}
\label{sec:1814}
\addcontentsline{toc}{section}{\nameref{sec:1814}}
\begin{longtable}{l p{0.5cm} r}
عشق تو آورد قدح پر ز بلای دل من
&&
گفتم می می نخورم گفت برای دل من
\\
داد می معرفتش با تو بگویم صفتش
&&
تلخ و گوارنده و خوش همچو وفای دل من
\\
از طرفی روح امین آمد و ما مست چنین
&&
پیش دویدم که ببین کار و کیای دل من
\\
گفت که ای سر خدا روی به هر کس منما
&&
شکر خدا کرد و ثنا بهر لقای دل من
\\
گفتم خود آن نشود عشق تو پنهان نشود
&&
چیست که آن پرده شود پیش صفای دل من
\\
عشق چو خون خواره شود رستم بیچاره شود
&&
کوه احد پاره شود آه چه جای دل من
\\
شاد دمی کان شه من آید در خرگه من
&&
باز گشاید به کرم بند قبای دل من
\\
گوید که افسرده شدی بی‌من و پژمرده شدی
&&
پیشتر آ تا بزند بر تو هوای دل من
\\
گویم کان لطف تو کو بنده خود را تو بجو
&&
کیست که داند جز تو بند و گشای دل من
\\
گوید نی تازه شوی بی‌حد و اندازه شوی
&&
تازه‌تر از نرگس و گل پیش صبای دل من
\\
گویم ای داده دوا لایق هر رنج و عنا
&&
نیست مرا جز تو دوا ای تو دوای دل من
\\
میوه هر شاخ و شجر هست گوای دل او
&&
روی چو زر اشک چو در هست گوای دل من
\\
\end{longtable}
\end{center}
