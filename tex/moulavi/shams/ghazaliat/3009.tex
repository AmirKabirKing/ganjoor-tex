\begin{center}
\section*{غزل شماره ۳۰۰۹: آه که چه شیرین بتیست در تتق زرکشی}
\label{sec:3009}
\addcontentsline{toc}{section}{\nameref{sec:3009}}
\begin{longtable}{l p{0.5cm} r}
آه که چه شیرین بتیست در تتق زرکشی
&&
اه که چه می‌زیبدش بدخوی و سرکشی
\\
گاه چو مه می‌رود قاعده شب روی
&&
می‌کند از اختران شیوه لشکرکشی
\\
گاه ز غیرت رود از همه چشمی نهان
&&
تا دل خود را ز هجر تو سوی آذر کشی
\\
ای خنک آن دم که تو خسرو و خورشید را
&&
سخت بگیری کمر خانه خود درکشی
\\
از طرب آن زمان جامه جان برکنی
&&
وز سر این بیخودی گوش فلک برکشی
\\
هر شکری زین هوس عود کند خویش را
&&
تا که بسوزد بر او چونک به مجمر کشی
\\
آن نفس از ساقیان سستی و تقصیر نیست
&&
نیست گنه باده را چونک تو کمتر کشی
\\
بخت عظیمست آنک نقل ز جنت بری
&&
خیر کثیرست آنک باده ز کوثر کشی
\\
مست برآیی ز خود دست بخایی ز خود
&&
قاصد خون ریز خود نیزه و خنجر کشی
\\
گوید کز نور من ظلمت و کافر کجاست
&&
تا که به شمشیر دین بر سر کافر کشی
\\
وقت شد ای شمس دین مفخر تبریزیان
&&
تا تو مرا چون قدح در می احمر کشی
\\
\end{longtable}
\end{center}
