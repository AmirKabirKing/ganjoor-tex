\begin{center}
\section*{غزل شماره ۵۹۶: آن مه که ز پیدایی در چشم نمی‌آید}
\label{sec:0596}
\addcontentsline{toc}{section}{\nameref{sec:0596}}
\begin{longtable}{l p{0.5cm} r}
آن مه که ز پیدایی در چشم نمی‌آید
&&
جان از مزه عشقش بی‌گشن همی‌زاید
\\
عقل از مزه بویش وز تابش آن رویش
&&
هم خیره همی‌خندد هم دست همی‌خاید
\\
هر صبح ز سیرانش می‌باشم حیرانش
&&
تا جان نشود حیران او روی ننماید
\\
هر چیز که می‌بینی در بی‌خبری بینی
&&
تا باخبری والله او پرده بنگشاید
\\
دم همدم او نبود جان محرم او نبود
&&
و اندیشه که این داند او نیز نمی‌شاید
\\
تن پرده بدوزیده جان برده بسوزیده
&&
با این دو مخالف دل بر عشق بنبساید
\\
دو لشکر بیگانه تا هست در این خانه
&&
در چالش و در کوشش جز گرد بنفزاید
\\
در زیر درخت او می‌ناز به بخت او
&&
تا جان پر از رحمت تا حشر بیاساید
\\
از شاه صلاح الدین چون دیده شود حق بین
&&
دل رو به صلاح آرد جان مشعله برباید
\\
\end{longtable}
\end{center}
