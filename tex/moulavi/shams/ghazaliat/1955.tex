\begin{center}
\section*{غزل شماره ۱۹۵۵: ای زیان و ای زیان و ای زیان و ای زیان}
\label{sec:1955}
\addcontentsline{toc}{section}{\nameref{sec:1955}}
\begin{longtable}{l p{0.5cm} r}
ای زیان و ای زیان و ای زیان و ای زیان
&&
هوشیاری در میان بیخودان و مستیان
\\
بی محابا درده ای ساقی مدام اندر مدام
&&
تا نماند هوشیاری عاقلی اندر جهان
\\
یار دعوی می کند گر عاشقی دیوانه شو
&&
سرد باشد عاقلی در حلقه دیوانگان
\\
گر درآید عاقلی گو کار دارم راه نیست
&&
ور درآید عاشقی دستش بگیر و درکشان
\\
عیب بینی از چه خیزد خیزد از عقل ملول
&&
تشنه هرگز عیب داند دید در آب روان
\\
عقل منکر هیچ گونه از نشان‌ها نگذرد
&&
بی نشان رو بی‌نشان تا زخم ناید بر نشان
\\
یوسفی شو گر تو را خامی بنخاسی برد
&&
گلشنی شو گر تو را خاری نداند گو مدان
\\
عیسیی شو گر تو را خانه نباشد گو مباش
&&
دیده‌ای شو گرت روپوشی نماند گو ممان
\\
\end{longtable}
\end{center}
