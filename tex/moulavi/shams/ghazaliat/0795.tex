\begin{center}
\section*{غزل شماره ۷۹۵: وقت آن شد که ز خورشید ضیایی برسد}
\label{sec:0795}
\addcontentsline{toc}{section}{\nameref{sec:0795}}
\begin{longtable}{l p{0.5cm} r}
وقت آن شد که ز خورشید ضیایی برسد
&&
سوی زنگی شب از روم لوایی برسد
\\
به برهنه شده عشق قبایی بدهند
&&
وز شکرخانه آن دوست نوایی برسد
\\
این همه کاسه زرین ز بر خوان فلک
&&
بهر آنست که یک روز صلایی برسد
\\
بره و خوشه گردون ز برای خورش است
&&
تا ز خرمنگه آن ماه عطایی برسد
\\
عاشقان را که جز این عشق غذایی دگرست
&&
کاسه کدیه ایشان به ابایی برسد
\\
نوخرانی که رهیدند ز بازار کهن
&&
کهنه کاسد ایشان به بهایی برسد
\\
مه پرستان که ستاره همه شب می‌شمرند
&&
آخر این کوشش و اومید به جایی برسد
\\
رو ترش کرده چو ابری که ببارید جفا
&&
از وفا رست جفا هم به وفایی برسد
\\
آنک دانست یقین مادر گل‌ها خارست
&&
همچو گل خندد چون خار جفایی برسد
\\
خضری گرد جهان لاف زد از آب حیات
&&
تا به گوش دل ما طبل بقایی برسد
\\
گر ز یاران گل آلود بریدی مگری
&&
چون ز گل دور شود آب صفایی برسد
\\
دل خود زین دودلان سرد کن و پاک بشوی
&&
دل خم شسته شود چون به سقایی برسد
\\
ناسزا گفتن از آن دلبر شیرین عجبست
&&
ناسزا گفت که تا جان به سزایی برسد
\\
یار چون سنگ دلان خانه ما را بشکست
&&
تا که هر خانه شکسته به سرایی برسد
\\
دوش در خواب بدیدم صلاح الدین را
&&
گسترد سایه دولت چو همایی برسد
\\
\end{longtable}
\end{center}
