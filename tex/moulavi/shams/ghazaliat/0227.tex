\begin{center}
\section*{غزل شماره ۲۲۷: به جان پاک تو ای معدن سخا و وفا}
\label{sec:0227}
\addcontentsline{toc}{section}{\nameref{sec:0227}}
\begin{longtable}{l p{0.5cm} r}
به جان پاک تو ای معدن سخا و وفا
&&
که صبر نیست مرا بی‌تو ای عزیز بیا
\\
چه جای صبر که گر کوه قاف بود این صبر
&&
ز آفتاب جدایی چو برف گشت فنا
\\
ز دور آدم تا دور اعور دجال
&&
چو جان بنده نبودست جان سپرده تو را
\\
تو خواه باور کن یا بگو که نیست چنین
&&
وفای عشق تو دارم به جان پاک وفا
\\
ملامتم مکنید ار دراز می‌گویم
&&
بود که کشف شود حال بنده پیش شما
\\
که آتشیست که دیگ مرا همی‌جوشد
&&
کز او شکاف کند گر رسد به سقف سما
\\
اگر چه سقف سما ز آفتاب و آتش او
&&
خلل نکرد و نگشت از تفش سیه سیما
\\
روان شدست یکی جوی خون ز هستی من
&&
خبر ندارم من کز کجاست تا به کجا
\\
به جو چه گویم کای جو مرو چه جنگ کنم
&&
برو بگو تو به دریا مجوش ای دریا
\\
به حق آن لب شیرین که می‌دمی در من
&&
که اختیار ندارد به ناله این سرنا
\\
خموش باش و مزن آتش اندر این بیشه
&&
نمی‌شکیبی می‌نال پیش او تنها
\\
\end{longtable}
\end{center}
