\begin{center}
\section*{غزل شماره ۲۶۳۳: در خانه خود یافتم از شاه نشانی}
\label{sec:2633}
\addcontentsline{toc}{section}{\nameref{sec:2633}}
\begin{longtable}{l p{0.5cm} r}
در خانه خود یافتم از شاه نشانی
&&
انگشتری لعل و کمر خاصه کانی
\\
دوش آمده بوده‌ست و مرا خواب ببرده
&&
آن شاه دلارامم و آن محرم جانی
\\
بشکسته دو صد کاسه و کوزه شه من دوش
&&
از عربده مستانه بدان شیوه که دانی
\\
گویی که گزیده‌ست ز مستی رخ من بر
&&
کز شاه رخ من بر کاری است نهانی
\\
امروز در این خانه همی‌بوی نگار است
&&
زین بوی به هر گوشه نگاری است عیانی
\\
خون در تن من باده صرف است از این بوی
&&
هر موی ز من هندوی مست است شبانی
\\
گوشی بنه و نعره مستانه شنو تو
&&
از قامت چون چنگ من الحان اغانی
\\
هم آتش و هم باده و خرگاه چو نقد است
&&
پیران طریقت بپذیرند جوانی
\\
در آینه شمس حق و دین شه تبریز
&&
هم صورت کل شهره و هم بحر معانی
\\
\end{longtable}
\end{center}
