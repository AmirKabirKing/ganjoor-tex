\begin{center}
\section*{غزل شماره ۱۴۹۰: از شهر تو رفتیم تو را سیر ندیدیم}
\label{sec:1490}
\addcontentsline{toc}{section}{\nameref{sec:1490}}
\begin{longtable}{l p{0.5cm} r}
از شهر تو رفتیم تو را سیر ندیدیم
&&
از شاخ درخت تو چنین خام فتیدیم
\\
در سایه سرو تو مها سیر نخفتیم
&&
وز باغ تو از بیم نگهبان نچریدیم
\\
بر تابه سودای تو گشتیم چو ماهی
&&
تا سوخته گشتیم ولیکن نپزیدیم
\\
گشتیم به ویرانه به سودای چو تو گنج
&&
چون مار به آخر به تک خاک خزیدیم
\\
چون سایه گذشتیم به هر پاکی و ناپاک
&&
اکنون به تو محویم نه پاک و نه پلیدیم
\\
ما را چو بجویید بر دوست بجویید
&&
کز پوست فناییم و بر دوست پدیدیم
\\
تا بر نمک و نان تو انگشت زدستیم
&&
در فرقت و در شور بس انگشت گزیدیم
\\
چون طبل رحیل آمد و آواز جرس‌ها
&&
ما رخت و قماشات بر افلاک کشیدیم
\\
شکر است که تریاق تو با ماست اگر چه
&&
زهری که همه خلق چشیدند چشیدیم
\\
آن دم که بریده شد از این جوی جهان آب
&&
چون ماهی بی‌آب بر این خاک طپیدیم
\\
چون جوی شد این چشم ز بی‌آبی آن جوی
&&
تا عاقبت امر به سرچشمه رسیدیم
\\
چون صبر فرج آمد و بی‌صبر حرج بود
&&
خاموش مکن ناله که ما صبر گزیدیم
\\
\end{longtable}
\end{center}
