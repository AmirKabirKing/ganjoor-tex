\begin{center}
\section*{غزل شماره ۲۶۵۸: اگر درد مرا درمان فرستی}
\label{sec:2658}
\addcontentsline{toc}{section}{\nameref{sec:2658}}
\begin{longtable}{l p{0.5cm} r}
اگر درد مرا درمان فرستی
&&
وگر کشت مرا باران فرستی
\\
وگر آن میر خوبان را به حیلت
&&
ز خانه جانب میدان فرستی
\\
وگر ساقی جان عاشقان را
&&
میان حلقه مستان فرستی
\\
همه ذرات عالم زنده گردد
&&
چو جانم را بر جانان فرستی
\\
وگر لب را به رحمت برگشایی
&&
مفرح سوی بیماران فرستی
\\
به دربان گفته‌ای مگذار ما را
&&
مرا هر دم بر دربان فرستی
\\
منم کشتی در این بحر و نشاید
&&
که بر من باد سرگردان فرستی
\\
همی‌خواهم که کشتیبان تو باشی
&&
اگر بر عاشقان طوفان فرستی
\\
مرا تا کی مها چون ارمغانی
&&
به پیش این و پیش آن فرستی
\\
دل بریان عاشق باده خواهد
&&
تو او را غصه و گریان فرستی
\\
یکی رطلی گران برریز بر وی
&&
از آن رطلی که بر مردان فرستی
\\
دل و جان هر دو را در نامه پیچم
&&
اگر تو نامه پنهان فرستی
\\
تو چون خورشید از مشرق برآیی
&&
جهان بی‌خبر را جان فرستی
\\
چه باشد ای صبا گر این غزل را
&&
به خلوتخانه سلطان فرستی
\\
\end{longtable}
\end{center}
