\begin{center}
\section*{غزل شماره ۱۵۳۷: چرا شاید چو ما شه زادگانیم}
\label{sec:1537}
\addcontentsline{toc}{section}{\nameref{sec:1537}}
\begin{longtable}{l p{0.5cm} r}
چرا شاید چو ما شه زادگانیم
&&
که جز صورت ز یک دیگر ندانیم
\\
چو مرغ خانه تا کی دانه چینیم
&&
چه شد دریا چو ما مرغابیانیم
\\
برو ای مرغ خانه تو چه دانی
&&
که ما مرغان در آن دریا چه سانیم
\\
مزن بر عاشقان عشق تشنیع
&&
تو را چه کاین چنینیم و چنانیم
\\
چنینیم و چنان و هر چه هستیم
&&
اسیر دام عشق بی‌امانیم
\\
چرا از جهل بر ما می دوانی
&&
نه گردون را چنین ما می دوانیم
\\
عجب نبود اگر ما را بخایند
&&
که آتش دیده و پخته چو نانیم
\\
وگر چون گرگ ما را می درانند
&&
چه چاره چون به حکم آن شبانیم
\\
چو چرخ اندر زبان‌ها اوفتادیم
&&
چو چرخ بی‌گناه و بی‌زبانیم
\\
حریف کهرباییم ار چو کاهیم
&&
نه در زندان چو کاه کاهدانیم
\\
نتاند باد کاه ما ربودن
&&
که ما زان کهربا اندر امانیم
\\
تو را باد و دم شهوت رباید
&&
نه ما که کهربای عقل و جانیم
\\
خمش کن کاه و کوه و کهربا چیست
&&
که آنچ از فهم بیرون است آنیم
\\
\end{longtable}
\end{center}
