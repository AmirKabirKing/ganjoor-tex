\begin{center}
\section*{غزل شماره ۱۶۹: رو ترش کن که همه روترشانند این جا}
\label{sec:0169}
\addcontentsline{toc}{section}{\nameref{sec:0169}}
\begin{longtable}{l p{0.5cm} r}
رو ترش کن که همه روترشانند این جا
&&
کور شو تا نخوری از کف هر کور عصا
\\
لنگ رو چونک در این کوی همه لنگانند
&&
لته بر پای بپیچ و کژ و مژ کن سر و پا
\\
زعفران بر رخ خود مال اگر مه رویی
&&
روی خوب ار بنمایی بخوری زخم قفا
\\
آینه زیر بغل زن چو ببینی زشتی
&&
ور نه بدنام کنی آینه را ای مولا
\\
تا که هشیاری و با خویش مدارا می‌کن
&&
چونک سرمست شدی هر چه که بادا بادا
\\
ساغری چند بخور از کف ساقی وصال
&&
چونک بر کار شدی برجه و در رقص درآ
\\
گرد آن نقطه چو پرگار همی‌زن چرخی
&&
این چنین چرخ فریضه‌ست چنین دایره را
\\
بازگو آنچ بگفتی که فراموشم شد
&&
سلم الله علیک ای مه و مه پاره ما
\\
سلم الله علیک ای همه ایام تو خوش
&&
سلم الله علیک ای دم یحیی الموتی
\\
چشم بد دور از آن رو که چو بربود دلی
&&
هیچ سودش نکند چاره و لا حول و لا
\\
ما به دریوزه حسن تو ز دور آمده‌ایم
&&
ماه را از رخ پرنور بود جود و سخا
\\
ماه بشنود دعای من و کف‌ها برداشت
&&
پیش ماه تو و می‌گفت مرا نیز مها
\\
مه و خورشید و فلک‌ها و معانی و عقول
&&
سوی ما محتشمانند و به سوی تو گدا
\\
غیرتت لب بگزید و به دلم گفت خموش
&&
دل من تن زد و بنشست و بیفکند لوا
\\
\end{longtable}
\end{center}
