\begin{center}
\section*{غزل شماره ۱۶۰۹: چو یکی ساغر مردی ز خم یار برآرم}
\label{sec:1609}
\addcontentsline{toc}{section}{\nameref{sec:1609}}
\begin{longtable}{l p{0.5cm} r}
چو یکی ساغر مردی ز خم یار برآرم
&&
دو جهان را و نهان را همه از کار برآرم
\\
ز پس کوه برآیم علم عشق نمایم
&&
ز دل خاره و مرمر دم اقرار برآرم
\\
ز تک چاه کسی را تو به صد سال برآری
&&
من دیوانه بی‌دل به یکی بار برآرم
\\
چو از آن کوه بلندم کمر عشق ببندم
&&
ز کمرگاه منافق سر زنار برآرم
\\
بر من نیست من و ما عدمم بی‌سر و بی‌پا
&&
سر و دل زان بنهادم که سر از یار برآرم
\\
به تو دیوار نمایم سوی خود در بگشایم
&&
به میان دست نباشد در و دیوار برآرم
\\
تا چه از کار فزایی سر و دستار نمایی
&&
که من از هر سر مویی سر و دستار برآرم
\\
تو ز بی‌گاه چه لنگی ز شب تیره چه ترسی
&&
که من از جانب مغرب مه انوار برآرم
\\
تو ز تاتار هراسی که خدا را نشناسی
&&
که دو صد رایت ایمان سوی تاتار برآرم
\\
هله این لحظه خموشم چو می عشق بنوشم
&&
زره جنگ بپوشم صف پیکار برآرم
\\
هله شمس الحق تبریز ز فراق تو چنانم
&&
که هیاهوی و فغان از سر بازار برآرم
\\
\end{longtable}
\end{center}
