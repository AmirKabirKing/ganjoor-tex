\begin{center}
\section*{غزل شماره ۲۳۹۵: دیدم نگار خود را می‌گشت گرد خانه}
\label{sec:2395}
\addcontentsline{toc}{section}{\nameref{sec:2395}}
\begin{longtable}{l p{0.5cm} r}
دیدم نگار خود را می‌گشت گرد خانه
&&
برداشته ربابی می‌زد یکی ترانه
\\
با زخمه چو آتش می‌زد ترانه خوش
&&
مست و خراب و دلکش از باده مغانه
\\
در پرده عراقی می‌زد به نام ساقی
&&
مقصود باده بودش ساقی بدش بهانه
\\
ساقی ماه رویی در دست او سبویی
&&
از گوشه‌ای درآمد بنهاد در میانه
\\
پر کرد جام اول زان باده مشعل
&&
در آب هیچ دیدی کآتش زند زبانه
\\
بر کف نهاده آن را از بهر دلستان را
&&
آنگه بکرد سجده بوسید آستانه
\\
بستد نگار از وی اندرکشید آن می
&&
شد شعله‌ها از آن می بر روی او دوانه
\\
می‌دید حسن خود را می‌گفت چشم بد را
&&
نی بود و نی بیاید چون من در این زمانه
\\
\end{longtable}
\end{center}
