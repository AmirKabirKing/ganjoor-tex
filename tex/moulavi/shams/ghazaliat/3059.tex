\begin{center}
\section*{غزل شماره ۳۰۵۹: چه باک دارد عاشق ز ننگ و بدنامی}
\label{sec:3059}
\addcontentsline{toc}{section}{\nameref{sec:3059}}
\begin{longtable}{l p{0.5cm} r}
چه باک دارد عاشق ز ننگ و بدنامی
&&
که عشق سلطنت است و کمال و خودکامی
\\
پلنگ عشق چه ترسد ز رنگ و بوی جهان
&&
نهنگ فقر چه ترسد ز دوزخ آشامی
\\
چگونه باشد عاشق ز مستی آن می
&&
که جام نیز ز تیزیش گم کند جامی
\\
چه جای خاک که بر کوه جرعه‌ای برریخت
&&
هزار عربده آورد و شورش و خامی
\\
تو جام عشق چه دانی چه شیشه دل باشی
&&
تو دام عشق چه دانی چو مرغ این دامی
\\
ز صاف بحر نگویم اگر کفش بینی
&&
مثال زیبق بر هیچ کف نیارامی
\\
ملول و تیره شدی مر صفاش را چه گنه
&&
نبات را چه جنایت چو سرکه آشامی
\\
که خاک بر سر سرکا و مرد سرکه فروش
&&
که شهد صاف ننوشد ز تیره ایامی
\\
به من نگر که در این بزم کمترین عامم
&&
ز بیخودی نشناسم ز خاص تا عامی
\\
\end{longtable}
\end{center}
