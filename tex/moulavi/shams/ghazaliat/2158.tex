\begin{center}
\section*{غزل شماره ۲۱۵۸: من که ستیزه روترم در طلب لقای تو}
\label{sec:2158}
\addcontentsline{toc}{section}{\nameref{sec:2158}}
\begin{longtable}{l p{0.5cm} r}
من که ستیزه روترم در طلب لقای تو
&&
بدهم جان بی‌وفا از جهت وفای تو
\\
در دل من نهاده‌ای آنچ دلم گشاده‌ای
&&
از دو هزار یک بود آنچ کنم به جای تو
\\
گلشکر مقویم هست سپاس و شکر تو
&&
کحل عزیزیم بود سرمه خاک پای تو
\\
سبزه نرویدی اگر چاشنیش ندادیی
&&
چرخ نگرددی اگر نشنودی صلای تو
\\
هست جهاز گلبنان حله سرخ و سبز تو
&&
هست امید شب روان یقظت روزهای تو
\\
من ز لقای مردمان جانب که گریزمی
&&
گر نبدی لقایشان آینه لقای تو
\\
بخت نداشت دهریی منکر گشت بعث را
&&
ور نه بقاش بخشدی موهبت بقای تو
\\
پر ز جهاد و نامیه عالم همچو کاهدان
&&
کی برسیدی از عدم جز که به کهربای تو
\\
در دل خاک از کجا های بدی و هو بدی
&&
گر نه پیاپی آمدی دعوت های های تو
\\
هم به خود آید آن کرم کیست که جذب او کند
&&
هست خود آمدن دلا عاطفت خدای تو
\\
گوید ذره ذره را چند پریم بر هوا
&&
هست هوا و ذره هم دستخوش هوای تو
\\
گردد صدصفت هوا ز اول روز تا به شب
&&
چرخ زنان به هر صفت رقص کنان برای تو
\\
رقص هوا ندیده‌ای رقص درخت‌ها نگر
&&
یا سوی رقص جان نگر پیش و پس خدای تو
\\
بس کن تا که هر یکی سوی حدیث خود رود
&&
نبود طبع‌ها همه عاشق مقتضای تو
\\
\end{longtable}
\end{center}
