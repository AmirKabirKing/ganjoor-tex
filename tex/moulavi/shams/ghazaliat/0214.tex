\begin{center}
\section*{غزل شماره ۲۱۴: درخت اگر متحرک بدی ز جای به جا}
\label{sec:0214}
\addcontentsline{toc}{section}{\nameref{sec:0214}}
\begin{longtable}{l p{0.5cm} r}
درخت اگر متحرک بدی ز جای به جا
&&
نه رنج اره کشیدی نه زخم‌های جفا
\\
نه آفتاب و نه مهتاب نور بخشیدی
&&
اگر مقیم بدندی چو صخره صما
\\
فرات و دجله و جیحون چه تلخ بودندی
&&
اگر مقیم بدندی به جای چون دریا
\\
هوا چو حاقن گردد به چاه زهر شود
&&
ببین ببین چه زیان کرد از درنگ هوا
\\
چو آب بحر سفر کرد بر هوا در ابر
&&
خلاص یافت ز تلخی و گشت چون حلوا
\\
ز جنبش لهب و شعله چون بماند آتش
&&
نهاد روی به خاکستری و مرگ و فنا
\\
نگر به یوسف کنعان که از کنار پدر
&&
سفر فتادش تا مصر و گشت مستثنا
\\
نگر به موسی عمران که از بر مادر
&&
به مدین آمد و زان راه گشت او مولا
\\
نگر به عیسی مریم که از دوام سفر
&&
چو آب چشمه حیوان‌ست یحیی الموتی
\\
نگر به احمد مرسل که مکه را بگذاشت
&&
کشید لشکر و بر مکه گشت او والا
\\
چو بر براق سفر کرد در شب معراج
&&
بیافت مرتبه قاب قوس او ادنی
\\
اگر ملول نگردی یکان یکان شمرم
&&
مسافران جهان را دو تا دو تا و سه تا
\\
چو اندکی بنمودم بدان تو باقی را
&&
ز خوی خویش سفر کن به خوی و خلق خدا
\\
\end{longtable}
\end{center}
