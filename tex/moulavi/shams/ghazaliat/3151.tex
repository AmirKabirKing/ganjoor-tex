\begin{center}
\section*{غزل شماره ۳۱۵۱: چند اندر میان غوغایی}
\label{sec:3151}
\addcontentsline{toc}{section}{\nameref{sec:3151}}
\begin{longtable}{l p{0.5cm} r}
چند اندر میان غوغایی
&&
خوی کن پاره پاره تنهایی
\\
خلوتی را لطیف سوداییست
&&
رو بپرسش که در چه سودایی
\\
خلوت آنست که در پناه کسی
&&
خوش بخسپی و خوش بیاسایی
\\
زیر سایه درخت بخت آور
&&
زود منزل کنی فرود آیی
\\
ور تو خواهی که بخت بگشاید
&&
زیر هر سایه رخت نگشایی
\\
سوی انبان ما و من نروی
&&
گر چه او گویدت که از مایی
\\
رو به خود آر هر کجا باشی
&&
روسیاه‌ست مرد هرجایی
\\
خود تو چیست بیخودی زان کس
&&
که از او در چنین تماشایی
\\
چون رسیدی به شه صلاح الدین
&&
گر فسادی سوی صلاح آیی
\\
\end{longtable}
\end{center}
