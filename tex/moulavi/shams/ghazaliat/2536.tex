\begin{center}
\section*{غزل شماره ۲۵۳۶: مثال باز رنجورم زمین بر من ز بیماری}
\label{sec:2536}
\addcontentsline{toc}{section}{\nameref{sec:2536}}
\begin{longtable}{l p{0.5cm} r}
مثال باز رنجورم زمین بر من ز بیماری
&&
نه با اهل زمین جنسم نه امکان است طیاری
\\
چو دست شاه یاد آید فتد آتش به جان من
&&
نه پر دارم که بگریزم نه بالم می‌کند یاری
\\
الا ای باز مسکین تو میان جغدها چونی
&&
نفاقی کردیی گر عشق رو بستی به ستاری
\\
ولیکن عشق کی پنهان شود با شعله سینه
&&
خصوصا از دو دیده سیل همچون چشمه جاری
\\
بس استت عزت و دوران ز ذوق عشق پرلذت
&&
کجا پیدا شود با عشق یا تلخی و یا خواری
\\
اگر چه تو نداری هیچ مانند الف عشقت
&&
به صدر حرف‌ها دارد چرا زان رو که آن داری
\\
حلاوت‌های جاویدان درون جان عشاق است
&&
ز بهر چشم زخم است این نفیر و این همه زاری
\\
تن عاشق چو رنجوران فتاده زار بر خاکی
&&
نیابد گرد ایشان را به معنی مه به سیاری
\\
مغفل وار پنداری تو عاشق را ولیکن او
&&
به هر دم پرده می‌سوزد ز آتش‌های هشیاری
\\
لباس خویش می‌درد قبای جسم می‌سوزد
&&
که تا وقت کنار دوست باشد از همه عاری
\\
به غیر دوست هر چش هست طراران همی‌دزدند
&&
به معنی کرده او زین فعل بر طرار طراری
\\
که تا خلوت کند ز ایشان کند مشغول ایشان را
&&
بگیرد خانه تجرید و خلوت را به عیاری
\\
ندانی سر این را تو که علم و عقل تو پرده است
&&
برون غار و تو شادان که خود در عین آن غاری
\\
بدرد زهره جانت اگر ناگاه بینی تو
&&
که از اصحاب کهف دل چگونه دور و اغیاری
\\
ز یک حرفی ز رمز دل نبردی بوی اندر عمر
&&
اگر چه حافظ اهلی و استادی تو ای قاری
\\
چه دورت داشتند ایشان که قطب کارها گشتی
&&
و از این اشغال بی‌کاران نداری تاب بی‌کاری
\\
تو را دم دم همی‌آرند کاری نو به هر لحظه
&&
که تا نبود فراغت هیچ بر قانون مکاری
\\
گهی سودای استادی گهی شهوت درافتادی
&&
گهی پشت سپه باشی گهی دربند سالاری
\\
دمار و ویل بر جانت اگر مخدوم شمس الدین
&&
ز تبریزت نفرماید زکات جان خود یاری
\\
\end{longtable}
\end{center}
