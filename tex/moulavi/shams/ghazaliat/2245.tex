\begin{center}
\section*{غزل شماره ۲۲۴۵: مطرب مهتاب رو آنچ شنیدی بگو}
\label{sec:2245}
\addcontentsline{toc}{section}{\nameref{sec:2245}}
\begin{longtable}{l p{0.5cm} r}
مطرب مهتاب رو آنچ شنیدی بگو
&&
ما همگان محرمیم آنچ بدیدی بگو
\\
ای شه و سلطان ما ای طربستان ما
&&
در حرم جان ما بر چه رسیدی بگو
\\
نرگس خمار او ای که خدا یار او
&&
دوش ز گلزار او هر چه بچیدی بگو
\\
ای شده از دست من چون دل سرمست من
&&
ای همه را دیده تو آنچ گزیدی بگو
\\
عید بیاید رود عید تو ماند ابد
&&
کز فلک بی‌مدد چون برهیدی بگو
\\
در شکرستان جان غرقه شدم ای شکر
&&
زین شکرستان اگر هیچ چشیدی بگو
\\
می‌کشدم می به چپ می‌کشدم دل به راست
&&
رو که کشاکش خوش است تو چه کشیدی بگو
\\
می به قدح ریختی فتنه برانگیختی
&&
کوی خرابات را تو چه کلیدی بگو
\\
شور خرابات ما نور مناجات ما
&&
پرده حاجات ما هم تو دریدی بگو
\\
ماه به ابر اندرون تیره شده‌ست و زبون
&&
ای مه کز ابرها پاک و بعیدی بگو
\\
ظل تو پاینده باد ماه تو تابنده باد
&&
چرخ تو را بنده باد از چه رمیدی بگو
\\
عشق مرا گفت دی عاشق من چون شدی
&&
گفتم بر چون متن ز آنچ تنیدی بگو
\\
مرد مجاهد بدم عاقل و زاهد بدم
&&
عافیتا همچو مرغ از چه پریدی بگو
\\
\end{longtable}
\end{center}
