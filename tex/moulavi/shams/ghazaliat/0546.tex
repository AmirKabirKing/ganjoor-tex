\begin{center}
\section*{غزل شماره ۵۴۶: هین سخن تازه بگو تا دو جهان تازه شود}
\label{sec:0546}
\addcontentsline{toc}{section}{\nameref{sec:0546}}
\begin{longtable}{l p{0.5cm} r}
هین سخن تازه بگو تا دو جهان تازه شود
&&
وارهد از حد جهان بی‌حد و اندازه شود
\\
خاک سیه بر سر او کز دم تو تازه نشد
&&
یا همگی رنگ شود یا همه آوازه شود
\\
هر که شدت حلقهٔ در زود برد حقه زر
&&
خاصه که در باز کنی محرم دروازه شود
\\
آب چه دانست که او گوهر گوینده شود
&&
خاک چه دانست که او غمزه غمازه شود
\\
روی کسی سرخ نشد بی‌مدد لعل لبت
&&
بی تو اگر سرخ بود از اثر غازه شود
\\
ناقه صالح چو ز که زاد یقین گشت مرا
&&
کوه پی مژده تو اشتر جمازه شود
\\
راز نهان دار و خمش ور خمشی تلخ بود
&&
آنچ جگرسوزه بود باز جگرسازه شود
\\
\end{longtable}
\end{center}
