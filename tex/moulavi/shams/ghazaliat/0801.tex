\begin{center}
\section*{غزل شماره ۸۰۱: عشرتی هست در این گوشه غنیمت دارید}
\label{sec:0801}
\addcontentsline{toc}{section}{\nameref{sec:0801}}
\begin{longtable}{l p{0.5cm} r}
عشرتی هست در این گوشه غنیمت دارید
&&
دولتی هست حریفان سر دولت خارید
\\
چو شکر یک دل و آغشته این شیر شوید
&&
که ظریفید و لطیفید و نکومقدارید
\\
دانه چیدن چه مروت بود آخر مکنید
&&
که امیران دو صد خرمن و صد انبارید
\\
با چنین لاله رخان روح چرا نفزایید
&&
در چنین معصره‌ای غوره چرا افشارید
\\
دست در دامن همچون گل و ریحانش زنید
&&
نه که پرورده و بسرشته آن گلزارید
\\
رنگ دیدیت بسی جان و حیاتیش نبود
&&
مه خوبان مرا از چه چنین پندارید
\\
چون ره خانه ندانید که زاده وصلید
&&
چون سره و قلب ندانید کز این بازارید
\\
فخر مصرید چو یوسف هله تعبیر کنید
&&
چو لب نوش وفا جمله شکر می‌کارید
\\
ملکانید و ملک زاده ز آغاز و سرشت
&&
گر چه امروز گدایانه چنین می‌زارید
\\
ساقیان باده به کف گوش شما می‌پیچند
&&
گرد خمخانه برآیید اگر خمارید
\\
همه صیاد هنر گشته پی بی‌عیبی
&&
همه عیبید چو در مجلس جان هشیارید
\\
شمس تبریز درآمد به عیان عذر نماند
&&
دیده روح طلب را به رخش بسپارید
\\
\end{longtable}
\end{center}
