\begin{center}
\section*{غزل شماره ۱۲۷۰: مستی امروز من نیست چو مستی دوش}
\label{sec:1270}
\addcontentsline{toc}{section}{\nameref{sec:1270}}
\begin{longtable}{l p{0.5cm} r}
مستی امروز من نیست چو مستی دوش
&&
می‌نکنی باورم کاسه بگیر و بنوش
\\
غرق شدم در شراب عقل مرا برد آب
&&
گفت خرد الوداع بازنیایم به هوش
\\
عقل و خرد در جنون رفت ز دنیا برون
&&
چونک ز سر رفت دیگ چونک ز حد رفت جوش
\\
این دل مجنون مست بند بدرید و جست
&&
با سرمستان مپیچ هیچ مگو رو خموش
\\
صبحدم از نردبان گفت مرا پاسبان
&&
کز سوی هفتم فلک دوش شنیدم خروش
\\
گفت زحل زهره را زخمه آهسته زن
&&
وی اسد آن ثور را شاخ بگیر و بدوش
\\
خون شده بین از نهیب شیر به پستان ثور
&&
شیر فلک را نگر گشته ز هیبت چو موش
\\
گرم کن ای شیر تک چند گریزی چو سگ
&&
جلوه کن ای ماه رو چند کنی روی پوش
\\
چشم گشا شش جهت شعشعه نور بین
&&
گوش گشا سوی چرخ ای شده چشم تو گوش
\\
بشنو از جان سلام تا برهی از کلام
&&
بنگر در نقش گر تا برهی از نقوش
\\
گفتمش ای خواجه رو هر چه شود گو بشو
&&
صافم و آزاد نو بنده دردی فروش
\\
ترس و امید تو را هست حواله به عقل
&&
دانه و دام تو را هست شکاری وحوش
\\
دردی دردش مرا چون به حمایت گرفت
&&
با من از این‌ها مگو کار توست آن بکوش
\\
\end{longtable}
\end{center}
