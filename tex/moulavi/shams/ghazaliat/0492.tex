\begin{center}
\section*{غزل شماره ۴۹۲: ز دام چند بپرسی و دانه را چه شدست}
\label{sec:0492}
\addcontentsline{toc}{section}{\nameref{sec:0492}}
\begin{longtable}{l p{0.5cm} r}
ز دام چند بپرسی و دانه را چه شدست
&&
به بام چند برآیی و خانه را چه شدست
\\
فسرده چند نشینی میان هستی خویش
&&
تنور آتش عشق و زبانه را چه شدست
\\
بگرد آتش عشقش ز دور می‌گردی
&&
اگر تو نقره صافی میانه را چه شدست
\\
ز دردی غم و اندیشه سیر چون نشوی
&&
جمال یار و شراب مغانه را چه شدست
\\
اگر چه سرد وجودیت گرم درپیچید
&&
به ره کنش به بهانه بهانه را چه شدست
\\
شکایت ار ز زمانه کند بگو تو برو
&&
زمانه بی‌تو خوشست و زمانه را چه شدست
\\
درخت وار چرا شاخ شاخ وسوسه‌ای
&&
یگانه باش چو بیخ و یگانه را چه شدست
\\
در آن ختن که در او شخص هست و صورت نیست
&&
مگو فلان چه کس است و فلانه را چه شدست
\\
نشان عشق شد این دل ز شمس تبریزی
&&
ببین ز دولت عشقش نشانه را چه شدست
\\
\end{longtable}
\end{center}
