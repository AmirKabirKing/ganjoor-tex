\begin{center}
\section*{غزل شماره ۳۱۳۹: صنما خرگه توم که بسازی و برکنی}
\label{sec:3139}
\addcontentsline{toc}{section}{\nameref{sec:3139}}
\begin{longtable}{l p{0.5cm} r}
صنما خرگه توم که بسازی و برکنی
&&
قلمی‌ام به دست تو که تراشی و بشکنی
\\
منم آن شقه علم که گهم سرنگون کنی
&&
و گهی بر فراز کوه برآری و برزنی
\\
منم آن ذره هوا که در این نور روزنم
&&
سوی روزن از آن روم که تو بالای روزنی
\\
هله ذره مگو مرا چو جهان گیر خود مرا
&&
دو جهان بی‌تو آفتاب کجا یافت روشنی
\\
همگی پوستم هله تو مرا مغز نغز گیر
&&
همه خشک‌اند مغزها چو نبخشی تو روغنی
\\
اگرم شاه و بی‌توام چه دروغست ما و من
&&
و گرم خاک و با توام چه لطیفست آن منی
\\
به تو نالم تو گوییم که تو را دور کرده‌ام
&&
که ببینم در این هوا که تو ذره چه می‌کنی
\\
به یکی ذره آفتاب چرا مشورت کند
&&
تو بکش هم تو زنده کن مکن ای دوست کردنی
\\
تو چه می داده‌ای به دل که چپ و راست می‌فتد
&&
و گهی نی چپ و نه راست و نه ترس و نه ایمنی
\\
\end{longtable}
\end{center}
