\begin{center}
\section*{غزل شماره ۱۸۴۹: چرا کوشد مسلمان در مسلمان را فریبیدن}
\label{sec:1849}
\addcontentsline{toc}{section}{\nameref{sec:1849}}
\begin{longtable}{l p{0.5cm} r}
چرا کوشد مسلمان در مسلمان را فریبیدن
&&
بسی صنعت نمی‌باید پریشان را فریبیدن
\\
بدریدی همه هامون ز نقش لیلی و مجنون
&&
ولی چشمش نمی‌خواهد گران جان را فریبیدن
\\
نمی‌آید دریغ او را چو دریا گوهرافشانی
&&
ولیکن تو روا داری بدین آن را فریبیدن
\\
معلم خانه چشمش چه رسم آورد در عالم
&&
که طمع افتاد موران را سلیمان را فریبیدن
\\
دلم بدرید ز اندیشه شکسته گشته چون شیشه
&&
که عقل از چه طمع دارد نهان دان را فریبیدن
\\
برآمد عالم از صیقل چو جندرخانه شد گیتی
&&
که بشنیدند کو خواهد ملیحان را فریبیدن
\\
هر اندیشه که برجوشد روان گردد پی صیدی
&&
نمک‌ها را هوس چه بود نمکدان را فریبیدن
\\
پلیدی را بیاموزد بر آب پاک افزودن
&&
کلیدی را بیاموزد کلیدان را فریبیدن
\\
چو لونالون می داند شکنجه کردن آن قاهر
&&
چه رغبت دارد آن آتش سپندان را فریبیدن
\\
\end{longtable}
\end{center}
