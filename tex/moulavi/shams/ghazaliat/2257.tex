\begin{center}
\section*{غزل شماره ۲۲۵۷: قلم از عشق بشکند چو نویسد نشان تو}
\label{sec:2257}
\addcontentsline{toc}{section}{\nameref{sec:2257}}
\begin{longtable}{l p{0.5cm} r}
قلم از عشق بشکند چو نویسد نشان تو
&&
خردم راه گم کند ز فراق گران تو
\\
کی بود همنشین تو کی بیابد گزین تو
&&
کی رهد از کمین تو کی کشد خود کمان تو
\\
رخم از عشق همچو زر ز تو بر من هزار اثر
&&
صنما سوی من نگر که چنانم به جان تو
\\
چو خلیل اندر آتشم ز تف آتشت خوشم
&&
نه از آنم که سر کشم ز غم بی‌امان تو
\\
بگشا کار مشکلم تو دلم ده که بی‌دلم
&&
مکن ای دوست منزلم به جز از گلستان تو
\\
کی بیاید به کوی تو صنما جز به بوی تو
&&
سبب جست و جوی تو چه بود گلفشان تو
\\
ملک و مردم و پری ملک و شاه و لشکری
&&
فلک و مهر و مشتری خجل از آستان تو
\\
چو تو سیمرغ روح را بکشانی در ابتلا
&&
چو مگس دوغ درفتد به گه امتحان تو
\\
ز اشارات عالیت ز بشارات شافیت
&&
ملکی گشته هر گدا به دم ترجمان تو
\\
همه خلقان چو مورکان به سوی خرمنت دوان
&&
همه عالم نواله‌ای ز عطاهای خوان تو
\\
به نواله قناعتی نکند جان آن فتی
&&
که طمع دارد از قضا که شود میهمان تو
\\
چه دواها که می‌کند پی هر رنج گنج تو
&&
چه نواها که می‌دهد به مکان لامکان تو
\\
طمع تن نوال تو طمع دل جمال تو
&&
نظر تن بنام تو هوس دل بنان تو
\\
جهت مصلحت بود نه بخیلی و مدخلی
&&
به سوی بام آسمان پنهان نردبان تو
\\
به امینان و نیکوان بنمودی تو نردبان
&&
که روان است کاروان به سوی آسمان تو
\\
خمش ای دل دگر مگو دگر اسرار او مجو
&&
که ندانی نهان آن که بداند نهان تو
\\
تو از این شهره نیشکر مطلب مغز اندرون
&&
که خود از قشر نیشکر شکرین شد لبان تو
\\
شه تبریز شمس دین که به هر لحظه آفرین
&&
برساد از جناب حق به مه خوش قران تو
\\
\end{longtable}
\end{center}
