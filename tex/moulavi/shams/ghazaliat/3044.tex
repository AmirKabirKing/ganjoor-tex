\begin{center}
\section*{غزل شماره ۳۰۴۴: گهی به سینه درآیی گهی ز روح برآیی}
\label{sec:3044}
\addcontentsline{toc}{section}{\nameref{sec:3044}}
\begin{longtable}{l p{0.5cm} r}
گهی به سینه درآیی گهی ز روح برآیی
&&
گهی به هجر گرایی چه آفتی چه بلایی
\\
گهی جمال بتانی گهی ز بت شکنانی
&&
گهی نه این و نه آنی چه آفتی چه بلایی
\\
بشر به پای دویده ملک به پر بپریده
&&
به غیر عجز ندیده چه آفتی چه بلایی
\\
چو پر و پاش نماند چو او ز هر دو بماند
&&
تو را به فقر بداند چه آفتی چه بلایی
\\
مثال لذت مستی میان چشم نشستی
&&
طریق فهم ببستی چه آفتی چه بلایی
\\
در آن دلی که گزیدی خیال وار دویدی
&&
بگفتی و بشنیدی چه آفتی چه بلایی
\\
چه دولتی ز چه سودی چه آتشی و چه دودی
&&
چه مجمری و چه عودی چه آفتی چه بلایی
\\
غم تو دامن جانی کشید جانب کانی
&&
به سوی گنج نهانی چه آفتی چه بلایی
\\
چه سوی گنج کشیدش ز جمله خلق بریدش
&&
دگر کسی بندیدش چه آفتی چه بلایی
\\
چه راحتی و چه روحی چه کشتیی و چه نوحی
&&
چه نعمتی چه فتوحی چه آفتی چه بلایی
\\
بگفتمت چه کس است این بگفتیم هوس است این
&&
خمش خمش که بس است این چه آفتی چه بلایی
\\
هوس چه باشد ای جان مرا مخند و مرنجان
&&
رهم نما و بگنجان چه آفتی چه بلایی
\\
تو عشق جمله جهانی ولی ز جمله نهانی
&&
نهان و عین چو جانی چه آفتی چه بلایی
\\
مرا چو دیک بجوشی مگو خمش چه خروشی
&&
چه جای صبر و خموشی چه آفتی چه بلایی
\\
بجوش دیک دلم را بسوز آب و گلم را
&&
بدر خط و سجلم را چه آفتی چه بلایی
\\
بسوز تا که برویم حدیث سوز بگویم
&&
به عود ماند خویم چه آفتی چه بلایی
\\
دگر مگوی پیامش رسید نوبت جامش
&&
ز جام ساز ختامش چه آفتی چه بلایی
\\
\end{longtable}
\end{center}
