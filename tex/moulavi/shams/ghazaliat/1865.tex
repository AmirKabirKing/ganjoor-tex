\begin{center}
\section*{غزل شماره ۱۸۶۵: ای سرو و گل بستان بنگر به تهی دستان}
\label{sec:1865}
\addcontentsline{toc}{section}{\nameref{sec:1865}}
\begin{longtable}{l p{0.5cm} r}
ای سرو و گل بستان بنگر به تهی دستان
&&
نانی ده و صد بستان‌هاده چه به درویشان
\\
بشنو تو ز پیغامبر فرمود که سیم و زر
&&
از صدقه نشد کمتر هاده چه به درویشان
\\
یک دانه اگر کاری صد سنبله برداری
&&
پس گوش چه می خاری‌هاده چه به درویشان
\\
کم کن تو فزایش بین بنواز و ستایش بین
&&
بگشا و گشایش بین هاده چه به درویشان
\\
صدقه تو به حق رفته و اندر شب آشفته
&&
او حارس و تو خفته‌هاده چه به درویشان
\\
هر لطف که بنمایی در سایه آن آیی
&&
بسیار بیاسایی‌هاده چه به درویشان
\\
حرمت کن و حرمت بین نعمت ده و نعمت بین
&&
رحمت کن و رحمت بین‌هاده چه به درویشان
\\
ای مکرم هر مسکین و ای راحم هر غمگین
&&
ای مالک یوم الدین‌هاده چه به درویشان
\\
آمد به تو آوازم واقف شدی از رازم
&&
محروم میندازم هاده چه به درویشان
\\
سرگشته تحویلم در قالم و در قیلم
&&
بنگر تو به زنبیلم هاده چه به درویشان
\\
دانی که دعا گویم هر جا که ثنا گویم
&&
بین کز تو چه واگویم هاده چه به درویشان
\\
رنجیت مبا آمین دور از تو قضا آمین
&&
یار تو خدا آمین هاده چه به درویشان
\\
ای کوی شما جنت وی خوی شما رحمت
&&
خاصه که در این ساعت هاده چه به درویشان
\\
گفتیم دعا رفتیم وز کوی شما رفتیم
&&
خوش باش که ما رفتیم هاده چه به درویشان
\\
\end{longtable}
\end{center}
