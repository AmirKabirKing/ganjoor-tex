\begin{center}
\section*{غزل شماره ۱۴۸۹: ساقی ز پی عشق روان است روانم}
\label{sec:1489}
\addcontentsline{toc}{section}{\nameref{sec:1489}}
\begin{longtable}{l p{0.5cm} r}
ساقی ز پی عشق روان است روانم
&&
لیکن ز ملولی تو کند است زبانم
\\
می پرم چون تیر سوی عشرت و نوشت
&&
ای دوست بمشکن به جفاهات کمانم
\\
چون خیمه به یک پای به پیش تو بپایم
&&
در خرگهت ای دوست درآر و بنشانم
\\
هین آن لب ساغر بنه اندر لب خشکم
&&
وانگه بشنو سحر محقق ز دهانم
\\
بشنو خبر بابل و افسانه وایل
&&
زیرا ز ره فکرت سیاح جهانم
\\
معذور همی‌دار اگر شور ز حد شد
&&
چون می ندهد عشق یکی لحظه امانم
\\
آن دم که ملولی ز ملولیت ملولم
&&
چون دست بشویی ز من انگشت گزانم
\\
آن شب که دهی نور چو مه تا به سحرگاه
&&
من در پی ماه تو چو سیاره دوانم
\\
وان روز که سر برزنی از شرق چو خورشید
&&
ماننده خورشید سراسر همه جانم
\\
وان روز که چون جان شوی از چشم نهانی
&&
من همچو دل مرغ ز اندیشه طپانم
\\
در روزن من نور تو روزی که بتابد
&&
در خانه چو ذره به طرب رقص کنانم
\\
این ناطقه خاموش و چو اندیشه نهان رو
&&
تا بازنیابد سبب اندیش نشانم
\\
\end{longtable}
\end{center}
