\begin{center}
\section*{غزل شماره ۷۰۷: آن یوسف خوش عذار آمد}
\label{sec:0707}
\addcontentsline{toc}{section}{\nameref{sec:0707}}
\begin{longtable}{l p{0.5cm} r}
آن یوسف خوش عذار آمد
&&
وان عیسی روزگار آمد
\\
وان سنجق صد هزار نصرت
&&
بر موکب نوبهار آمد
\\
ای کار تو مرده زنده کردن
&&
برخیز که روز کار آمد
\\
شیری که به صید شیر گیرد
&&
سرمست به مرغزار آمد
\\
دی رفت و پریر نقد بستان
&&
کان نقد خوش عیار آمد
\\
این شهر امروز چون بهشتست
&&
می‌گوید شهریار آمد
\\
می‌زن دهلی که روز عیدست
&&
می‌کن طربی که یار آمد
\\
ماهی از غیب سر برون کرد
&&
کاین مه بر او غبار آمد
\\
از خوبی آن قرار جان‌ها
&&
عالم همه بی‌قرار آمد
\\
هین دامن عشق برگشایید
&&
کز چرخ نهم نثار آمد
\\
ای مرغ غریب پربریده
&&
بر جای دو پر چهار آمد
\\
هان ای دل بسته سینه بگشا
&&
کان گمشده در کنار آمد
\\
ای پای بیا و پای می‌کوب
&&
کان سرده نامدار آمد
\\
از پیر مگو که او جوان شد
&&
وز پار مگو که پار آمد
\\
گفتی با شه چه عذر گویم
&&
خود شاه به اعتذار آمد
\\
گفتی که کجا رهم ز دستش
&&
دستش همه دستیار آمد
\\
ناری دیدی و نور آمد
&&
خونی دیدی عقار آمد
\\
آن کس که ز بخت خود گریزد
&&
بگریخته شرمسار آمد
\\
خامش کن و لطف‌هاش مشمر
&&
لطفیست که بی‌شمار آمد
\\
\end{longtable}
\end{center}
