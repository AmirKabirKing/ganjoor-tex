\begin{center}
\section*{غزل شماره ۲۱۷۲: هم آگه و هم ناگه مهمان من آمد او}
\label{sec:2172}
\addcontentsline{toc}{section}{\nameref{sec:2172}}
\begin{longtable}{l p{0.5cm} r}
هم آگه و هم ناگه مهمان من آمد او
&&
دل گفت که کی آمد جان گفت مه مه رو
\\
او آمد در خانه ما جمله چو دیوانه
&&
اندر طلب آن مه رفته به میان کو
\\
او نعره زنان گشته از خانه که این جایم
&&
ما غافل از این نعره هم نعره زنان هر سو
\\
آن بلبل مست ما بر گلشن ما نالان
&&
چون فاخته ما پران فریادکنان کوکو
\\
در نیم شبی جسته جمعی که چه دزد آمد
&&
و آن دزد همی‌گوید دزد آمد و آن دزد او
\\
آمیخته شد بانگش با بانگ همه زان سان
&&
پیدا نشود بانگش در غلغله شان یک مو
\\
و هو معکم یعنی با توست در این جستن
&&
آنگه که تو می‌جویی هم در طلب او را جو
\\
نزدیکتر است از تو با تو چه روی بیرون
&&
چون برف گدازان شو خود را تو ز خود می‌شو
\\
از عشق زبان روید جان را مثل سوسن
&&
می‌دار زبان خامش از سوسن گیر این خو
\\
\end{longtable}
\end{center}
