\begin{center}
\section*{غزل شماره ۲۹۵۰: ای آنک جان ما را در گلشکر کشیدی}
\label{sec:2950}
\addcontentsline{toc}{section}{\nameref{sec:2950}}
\begin{longtable}{l p{0.5cm} r}
ای آنک جان ما را در گلشکر کشیدی
&&
چون جان و دل ببردی خود را تو درکشیدی
\\
ما را چو سایه دیدی از پای درفتاده
&&
جانا چو سرو سرکش از سایه سر کشیدی
\\
چون سیل در کهستان ما سو به سو دوانه
&&
اندر پیت تو خیمه سوی دگر کشیدی
\\
تو آن مهی که هر کو آمد به خرمن تو
&&
مانند آفتابش در کان زر کشیدی
\\
کشتی ز رشک ما را باری چو اشک ما را
&&
از چشم خود میفکن چون در نظر کشیدی
\\
بر عاشقت ز صد سو از خلق زخم آید
&&
از لطف و رحمت خود پیشش سپر کشیدی
\\
یک قوم را به حیلت بستی به بند زرین
&&
یک قوم را به حجت اندر سفر کشیدی
\\
آوه که شد فضولی در خون چند گولی
&&
رحمی بکن بر آن کش در شور و شر کشیدی
\\
از چشم عاشقانت شب خواب شد رمیده
&&
زیرا که بی‌دلان را وقت سحر کشیدی
\\
ای عشق دل نداری تا که دلت بسوزد
&&
خود جمله دل تو داری دل را تو برکشیدی
\\
بس کن که نقل عیسی از بیخودی و مستی
&&
در آخر ستوران در پیش خر کشیدی
\\
\end{longtable}
\end{center}
