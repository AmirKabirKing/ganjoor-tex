\begin{center}
\section*{غزل شماره ۱۲۰۳: برو برو که نفورم ز عشق عارآمیز}
\label{sec:1203}
\addcontentsline{toc}{section}{\nameref{sec:1203}}
\begin{longtable}{l p{0.5cm} r}
برو برو که نفورم ز عشق عارآمیز
&&
برو برو گل سرخی ولیک خارآمیز
\\
مقام داشت به جنت صفی حق آدم
&&
جدا فتاد ز جنت که بود مارآمیز
\\
میان چرخ و زمین بس هوای پرنورست
&&
ولیک تیره شود چون شود غبارآمیز
\\
چو دوست با عدو تو نشست از او بگریز
&&
که احتراق دهد آب گرم نارآمیز
\\
برون کشم ز خمیر تو خویش را چون موی
&&
که ذوق خمر تو را دیده‌ام خمارآمیز
\\
ولیک موی کشان آردم بر تو غمت
&&
که اژدهاست غمت با دم شرارآمیز
\\
هزار بار گریزم چو تیر و بازآیم
&&
بدان کمان و بدان غمزه شکارآمیز
\\
به گردنامه سحرم به خانه بازآرد
&&
خیال یار به اکراه اختیارآمیز
\\
غم تو بر سفرم زیر زیر می‌خندد
&&
که واقفست از این عشق زینهارآمیز
\\
به پیش سلطنت توبه‌ام چو مسخره ایست
&&
که عشق را نبود صبر اعتبارآمیز
\\
سخن مگوی چو گویی ز صبر و توبه مگوی
&&
حدیث توبه مجنون بود فشارآمیز
\\
\end{longtable}
\end{center}
