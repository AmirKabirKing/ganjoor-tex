\begin{center}
\section*{غزل شماره ۱۴۶۰: بشکسته سر خلقی سر بسته که رنجورم}
\label{sec:1460}
\addcontentsline{toc}{section}{\nameref{sec:1460}}
\begin{longtable}{l p{0.5cm} r}
بشکسته سر خلقی سر بسته که رنجورم
&&
برده ز فلک خرقه آورده که من عورم
\\
وای از دل سنگینش وز عشوه رنگینش
&&
او نیست منم سنگین کاین فتنه همی‌شورم
\\
من در تک خونستم وز خوردن خون مستم
&&
گویی که نیم در خون در شیره انگورم
\\
ای عشق که از زفتی در چرخ نمی‌گنجی
&&
چون است که می گنجی اندر دل مستورم
\\
در خانه دل جستی در را ز درون بستی
&&
مشکات و زجاجم من یا نور علی نورم
\\
تن حامله زنگی دل در شکمش رومی
&&
پس نیم ز مشکم من یک نیم ز کافورم
\\
بردی دل و من قاصد دل از دگران جویم
&&
نادیده همی‌آرم اما نه چنین کورم
\\
گر چهره زرد من در خاک رود روزی
&&
روید گل زرد ای جان از خاک سر گورم
\\
آخر نه سلیمان هم بشنید غم موری
&&
آخر تو سلیمانی انگار که من مورم
\\
گفتی که چه می نالی صد خانه عسل داری
&&
می مالم و می نالم هم خرقه زنبورم
\\
می نالم از این علت اما به دو صد دولت
&&
نفروشم یک ذره زین علت ناسورم
\\
چون چنگ همی‌زارم چون بلبل گلزارم
&&
چون مار همی‌پیچم چون بر سر گنجورم
\\
گویی که انا گفتی با کبر و منی جفتی
&&
آن عکس تو است ای جان اما من از آن دورم
\\
من خامم و بریانم خندنده و گریانم
&&
حیران کن و حیرانم در وصلم و مهجورم
\\
\end{longtable}
\end{center}
