\begin{center}
\section*{غزل شماره ۱۳۵۳: چگونه برنپرد جان چو از جناب جلال}
\label{sec:1353}
\addcontentsline{toc}{section}{\nameref{sec:1353}}
\begin{longtable}{l p{0.5cm} r}
چگونه برنپرد جان چو از جناب جلال
&&
خطاب لطف چو شکر به جان رسد که تعال
\\
در آب چون نجهد زود ماهی از خشکی
&&
چو بانگ موج به گوشش رسد ز بحر زلال
\\
چرا ز صید نپرد به سوی سلطان باز
&&
چو بشنود خبر ارجعی ز طبل و دوال
\\
چرا چو ذره نیاید به رقص هر صوفی
&&
در آفتاب بقا تا رهاندش ز زوال
\\
چنان لطافت و خوبی و حسن و جان بخشی
&&
کسی از او بشکیبد زهی شقا و ضلال
\\
بپر بپر هله ای مرغ سوی معدن خویش
&&
که از قفس برهید و باز شد پر و بال
\\
ز آب شور سفر کن به سوی آب حیات
&&
رجوع کن به سوی صدر جان ز صف نعال
\\
برو برو تو که ما نیز می‌رسیم ای جان
&&
از این جهان جدایی بدان جهان وصال
\\
چو کودکان هله تا چند ما به عالم خاک
&&
کنیم دامن خود پر ز خاک و سنگ و سفال
\\
ز خاک دست بداریم و بر سما پریم
&&
ز کودکی بگریزیم سوی بزم رجال
\\
مبین که قالب خاکی چه در جوالت کرد
&&
جوال را بشکاف و برآر سر ز جوال
\\
به دست راست بگیر از هوا تو این نامه
&&
نه کودکی که ندانی یمین خود ز شمال
\\
بگفت پیک خرد را خدا که پا بردار
&&
بگفت دست اجل را که گوش حرص بمال
\\
ندا رسید روان را روان شو اندر غیب
&&
منال و گنج بگیر و دگر ز رنج منال
\\
تو کن ندا و تو آواز ده که سلطانی
&&
تو راست لطف جواب و تو راست علم سؤال
\\
\end{longtable}
\end{center}
