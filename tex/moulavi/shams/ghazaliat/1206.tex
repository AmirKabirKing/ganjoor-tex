\begin{center}
\section*{غزل شماره ۱۲۰۶: سوی لبش هر آنک شد زخم خورد ز پیش و پس}
\label{sec:1206}
\addcontentsline{toc}{section}{\nameref{sec:1206}}
\begin{longtable}{l p{0.5cm} r}
سوی لبش هر آنک شد زخم خورد ز پیش و پس
&&
زانک حوالی عسل نیش زنان بود مگس
\\
روی ویست گلستان مار بود در او نهان
&&
جعد ویست همچو شب مجمع دزد و هر عسس
\\
کان زمردی مها دیده مار برکنی
&&
ماه دوهفته‌ای شها غم نخوریم از غلس
\\
بی‌تو جهان چه فن زند بی‌تو چگونه تن زند
&&
جان و جهان غلام تو جان و جهان تویی و بس
\\
نصرت رستمان تویی فتح و ظفررسان تویی
&&
هست اثر حمایتت گر زره‌ست وگر فرس
\\
شمس تو معنوی بود آن نه که منطوی بود
&&
صد مه و آفتاب را نور توست مقتبس
\\
چرخ میان آب تو بر دوران همی‌زند
&&
عقل بر طبیبیت عرضه همی‌کند مجس
\\
ذره به ذره طمع‌ها صف زده پیش خوان تو
&&
سجده کنان و دم زنان بهر امید هر نفس
\\
دست چنین چنین کند لطف که من چنان دهم
&&
آنچ بهار می‌دهد از دم خود به خار و خس
\\
خاک که نور می‌خورد نقره و زر نبات او
&&
خاک که آب می‌خورد ماش شدست یا عدس
\\
رنگ جهان چو سحرها عشق عصای موسوی
&&
باز کند دهان خود درکشدش به یک نفس
\\
چند بترسی ای دل از نقش خود و خیال خود
&&
چند گریز می‌کنی بازنگر که نیست کس
\\
بس کن و بس که کمتر از اسب سقای نیستی
&&
چونک بیافت مشتری باز کند از او جرس
\\
\end{longtable}
\end{center}
