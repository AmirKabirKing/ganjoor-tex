\begin{center}
\section*{غزل شماره ۲۲۲۸: جان ما را هر نفس بستان نو}
\label{sec:2228}
\addcontentsline{toc}{section}{\nameref{sec:2228}}
\begin{longtable}{l p{0.5cm} r}
جان ما را هر نفس بستان نو
&&
گوش ما را هر نفس دستان نو
\\
ماهیانیم اندر آن دریا که هست
&&
روز روزش گوهر و مرجان نو
\\
تا فسون هیچ کس را نشنوی
&&
این جهان کهنه را برهان نو
\\
عیش ما نقد است وآنگه نقد نو
&&
ذات ما کان است وآنگه کان نو
\\
این شکر خور این شکر کز ذوق او
&&
می‌دهد اندر دهان دندان نو
\\
جمله جان شو ار کسی پرسد تو را
&&
تو کیی گو هر زمانی جان نو
\\
من زمین را لقمه‌ام لیکن زمین
&&
رویدش زین لقمه صد لقمان نو
\\
زرد گشتی از خزان غمگین مشو
&&
در خزان بین تاب تابستان نو
\\
\end{longtable}
\end{center}
