\begin{center}
\section*{غزل شماره ۱۹۰۰: بیا ای مونس جان‌های مستان}
\label{sec:1900}
\addcontentsline{toc}{section}{\nameref{sec:1900}}
\begin{longtable}{l p{0.5cm} r}
بیا ای مونس جان‌های مستان
&&
ببین اندیشه و سودای مستان
\\
بیا ای میر خوبان و برافروز
&&
ز شمع روی خود سیمای مستان
\\
نمی‌آیی سر از طاقی برون کن
&&
ببین این غلغل و غوغای مستان
\\
بیا ای خواب مستان را ببسته
&&
گشا این بند را از پای مستان
\\
همه شب می رود تا روز ای مه
&&
به اهل آسمان هیهای مستان
\\
همی‌گویند ما هم زو خرابیم
&&
چنین است آسمان پس وای مستان
\\
فرشته و آدمی دیوان و پریان
&&
ز تو زیر و زبر چون رای مستان
\\
کلاه جمله هشیاران ربودند
&&
در این بازارگه چه جای مستان
\\
میفکن وعده مستان به فردا
&&
تویی فردا و پس فردای مستان
\\
چو مستان گرد چشمت حلقه کردند
&&
کی بنشیند دگر بالای مستان
\\
شنیدم چرخ گردون را که می گفت
&&
منم یک لقمه از حلوای مستان
\\
شنیدم از دهان عشق می گفت
&&
منم معشوقه زیبای مستان
\\
اگر گویند ماه روزه آمد
&&
نیابی جام جان افزای مستان
\\
بگو کان می ز دریاهای جان است
&&
که جان را می دهد سقای مستان
\\
همه مولای عقلند این غریب است
&&
که عقل آمد که من مولای مستان
\\
چو فرمان موقع داشت رویش
&&
کشید ابروی او طغرای مستان
\\
همه مستان نبشتند این غزل را
&&
به خون دل ز خون پالای مستان
\\
\end{longtable}
\end{center}
