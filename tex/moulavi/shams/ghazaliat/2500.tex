\begin{center}
\section*{غزل شماره ۲۵۰۰: چه افسردی در آن گوشه چرا تو هم نمی‌گردی}
\label{sec:2500}
\addcontentsline{toc}{section}{\nameref{sec:2500}}
\begin{longtable}{l p{0.5cm} r}
چه افسردی در آن گوشه چرا تو هم نمی‌گردی
&&
مگر تو فکر منحوسی که جز بر غم نمی‌گردی
\\
چو آمد موسی عمران چرا از آل فرعونی
&&
چو آمد عیسی خوش دم چرا همدم نمی‌گردی
\\
چو با حق عهدها بستی ز سستی عهد بشکستی
&&
چو قول عهد جانبازان چرا محکم نمی‌گردی
\\
میان خاک چون موشان به هر مطبخ رهی سازی
&&
چرا مانند سلطانان بر این طارم نمی‌گردی
\\
چرا چون حلقه بر درها برای بانگ و آوازی
&&
چرا در حلقه مردان دمی محرم نمی‌گردی
\\
چگونه بسته بگشاید چو دشمن دار مفتاحی
&&
چگونه خسته به گردد چو بر مرهم نمی‌گردی
\\
سر آنگه سر بود ای جان که خاک راه او باشد
&&
ز عشق رایتش ای سر چرا پرچم نمی‌گردی
\\
چرا چون ابر بی‌باران به پیش مه ترنجیدی
&&
چرا همچون مه تابان بر این عالم نمی‌گردی
\\
قلم آن جا نهد دستش که کم بیند در او حرفی
&&
چرا از عشق تصحیحش تو حرفی کم نمی‌گردی
\\
گلستان و گل و ریحان نروید جز ز دست تو
&&
دو چشمه داری ای چهره چرا پرنم نمی‌گردی
\\
چو طوافان گردونی همی‌گردند بر آدم
&&
مگر ابلیس ملعونی که بر آدم نمی‌گردی
\\
اگر خلوت نمی‌گیری چرا خامش نمی‌باشی
&&
اگر کعبه نه ای باری چرا زمزم نمی‌گردی
\\
\end{longtable}
\end{center}
