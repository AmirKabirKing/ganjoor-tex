\begin{center}
\section*{غزل شماره ۱۸۳۴: عید نمای عید را ای تو هلال عید من}
\label{sec:1834}
\addcontentsline{toc}{section}{\nameref{sec:1834}}
\begin{longtable}{l p{0.5cm} r}
عید نمای عید را ای تو هلال عید من
&&
گوش بمال ماه را ای مه ناپدید من
\\
بود من و فنای من خشم من و رضای من
&&
صدق من و ریای من قفل من و کلید من
\\
اصل من و سرشت من مسجد من کنشت من
&&
دوزخ من بهشت من تازه من قدید من
\\
جور کنی وفا بود درد دهی دوا بود
&&
لایق تو کجا بود دیده جان و دید من
\\
پیشتر از نهاد جان لطف تو داد داد جان
&&
ای همگی مراد جان پس تو بدی مرید من
\\
ای مه عید روی تو ای شب قدر موی تو
&&
چون برسم بجوی تو پاک شود پلید من
\\
جسم چو خانقاه جان فکرت‌ها چو صوفیان
&&
حلقه زدند و در میان دل چو ابایزید من
\\
دم نزم خمش کنم با همه رو ترش کنم
&&
تا که بگوییم تویی حاضر و مستفید من
\\
\end{longtable}
\end{center}
