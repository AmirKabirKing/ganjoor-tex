\begin{center}
\section*{غزل شماره ۳۰۸۴: به جان تو که بگویی وطن کجا داری}
\label{sec:3084}
\addcontentsline{toc}{section}{\nameref{sec:3084}}
\begin{longtable}{l p{0.5cm} r}
به جان تو که بگویی وطن کجا داری
&&
که سخت فتنه عقلی و خصم هشیاری
\\
چو خارپشت سر اندرکشید عقل امروز
&&
که ساقی می گلگون و رشک گلزاری
\\
سماع باره نبودم تو از رهم بردی
&&
به مکر راه زن صد هزار طراری
\\
به گوش چرخ چه گفتی که یاوه گرد شده‌ست
&&
به گوش ابر چه گفتی که کرد درباری
\\
به خاک هم چه نمودی که گشت آبستن
&&
ز باد هم چه ربودی که می‌کند زاری
\\
به کوه‌ها چه سپردی که گنج ساز شدند
&&
به بحرها تو بیاموختی گهرباری
\\
به گوش کفر چه گفتی که چشم و گوش ببست
&&
به گوش عقل چه گفتی که گشت انواری
\\
چگونه از کف غم می‌رهانیم در خواب
&&
چگونه در غم وا می‌کشی به بیداری
\\
به مثل خواب هزاران طریق و چاره‌استت
&&
که ره دهی دل و جان را به غصه نسپاری
\\
چنانک عارف بیدار و خفته از دنیا
&&
ز خار رست کسی که سرش تو می‌خاری
\\
به آفتاب و به ماه و به اختران و فلک
&&
چه داده‌ای تو که بی‌پر کنند طیاری
\\
به ذره‌های پرنده چه نغمه از تو رسید
&&
که گر به کوه رسانی همش به رقص آری
\\
دماغ آب و گلی را ز مکر پر کردی
&&
چنانک با تو همی‌پیچد او به مکاری
\\
دمی که درندمی تو تهی شوند چو خیک
&&
نه‌های و هوی بماند نه زور و رهواری
\\
خموش کردم و بگریختم ز خود صد بار
&&
کشان کشان تو مرا سوی گفت می‌آری
\\
\end{longtable}
\end{center}
