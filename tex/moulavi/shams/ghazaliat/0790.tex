\begin{center}
\section*{غزل شماره ۷۹۰: واقف سرمد تا مدرسه عشق گشود}
\label{sec:0790}
\addcontentsline{toc}{section}{\nameref{sec:0790}}
\begin{longtable}{l p{0.5cm} r}
واقف سرمد تا مدرسه عشق گشود
&&
فرقیی مشکل چون عاشق و معشوق نبود
\\
جز قیاس و دوران هست طرق لیک شدست
&&
بر اولوالفقه و طبیب و متنجم مسدود
\\
اندر این صورت و آن صورت بس فکرت تیز
&&
از پی بحث و تفکر ید بیضا بنمود
\\
فرق گفتند بسی جامعشان راه ببست
&&
رو به جامع چو نهادند دو صد فرق فزود
\\
فکر محدود بد و جامع و فارق بی‌حد
&&
آنچ محدود بد آن محو شد از نامحدود
\\
محو سکرست پس محو بود صحو یقین
&&
شمس عاقب بود ار چند بود ظل ممدود
\\
این از آنست که یطوی به زبان لایحکی
&&
زانک اثبات چنین نکته بود نفی وجود
\\
این سخن فرع وجودست و حجابست ز نفی
&&
کشف چیزی به حجابش نبود جز مردود
\\
نه ز مردود گریزی نه ز مقبول خلاص
&&
بهل این را که نگنجد نه به بحث و نه سرود
\\
تو پس این را بهلی لیک تو را آن نهلد
&&
جان از این قاعده نجهد به قیام و به قعود
\\
جان قعود آرد آنش بکشد سوی قیام
&&
جان قیام آرد آنش بکشد سوی سجود
\\
این یگانه نه دوگانه‌ست که از وی برهی
&&
به سلام و به تشهد نرهد جان ز شهود
\\
نه به تحریمه درآمد نه به تحلیله رود
&&
نه به تکبیره ببست و نه سلامش بگشود
\\
مگس روح درافتاد در این دوغ ابد
&&
نه مسلمان و نه ترسا و نه گبر و نه جهود
\\
هله می‌گو که سخن پر زدن آن مگس است
&&
پر زدن نیز نماند چو رود دوغ فرود
\\
پر زدن نوع دگر باشد اگر نیز بود
&&
رقص نادر بودت بر زبر چرخ کبود
\\
\end{longtable}
\end{center}
