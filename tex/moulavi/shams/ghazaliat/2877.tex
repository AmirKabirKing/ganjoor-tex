\begin{center}
\section*{غزل شماره ۲۸۷۷: بر یکی بوسه حقستت که چنان می‌لرزی}
\label{sec:2877}
\addcontentsline{toc}{section}{\nameref{sec:2877}}
\begin{longtable}{l p{0.5cm} r}
بر یکی بوسه حقستت که چنان می‌لرزی
&&
ز آنک جان است و پی دادن جان می‌لرزی
\\
از دم و دمدمه آیینه دل تیره شود
&&
جهت آینه بر آینه دان می‌لرزی
\\
این جهان روز و شب از خوف و رجا لرزان است
&&
چونک تو جان جهانی تو جهان می‌لرزی
\\
چون قماشات تو اندر همه بازار که راست
&&
سزدت گر جهت سود و زیان می‌لرزی
\\
تا که نخجیر تو از بیم تو خود چون لرزد
&&
که تو صیادی و با تیر و کمان می‌لرزی
\\
تو به صورت مهی اما به نظر مریخی
&&
قاصد کشتن خلقی چو سنان می‌لرزی
\\
گه پی فتنه گری چون می خم می‌جوشی
&&
گه چو اعضای غضوب از غلیان می‌لرزی
\\
دل چو ماه از پی خورشید رخت دق دارد
&&
تو چرا همچو دل اندر خفقان می‌لرزی
\\
به لطف جان بهاری تو و سرسبزی باغ
&&
باز چون برگ تو از باد خزان می‌لرزی
\\
خلق چون برگ و تو باد و همه لرزان تواند
&&
ظاهرا صف شکنی و به نهان می‌لرزی
\\
قصر شکری که به تو هر کی رسد شکر کند
&&
سقف صبری تو که از بار گران می‌لرزی
\\
چون که قاف یقین راسخ و بی‌لرزه بود
&&
در گمانی تو مگر که چو کمان می‌لرزی
\\
دم فروکش هله ای ناطق ظنی و خمش
&&
کز دم فال زنان همچو زنان می‌لرزی
\\
\end{longtable}
\end{center}
