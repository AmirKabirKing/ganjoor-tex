\begin{center}
\section*{غزل شماره ۱۹۶۵: در میان ظلمت جان تو نور چیست آن}
\label{sec:1965}
\addcontentsline{toc}{section}{\nameref{sec:1965}}
\begin{longtable}{l p{0.5cm} r}
در میان ظلمت جان تو نور چیست آن
&&
فر شاهی می نماید در دلم آن کیست آن
\\
می نماید کان خیال روی چون ماه شه است
&&
وان پناه دستگیر روز مسکینی است آن
\\
این چنین فر و جمال و لطف و خوبی و نمک
&&
فخر جان‌ها شمس حق و دین تبریزی است آن
\\
برنتابد جان آدم شرح اوصافش صریح
&&
آنچ می تابد ز اوصافش دلا مکنی است آن
\\
زانک اوصاف بقا اندر فنا کی رو دهد
&&
مر مزیجی را که آن از عالم فانی است آن
\\
آن جمالی کو که حقش نقش کرد از دست خویش
&&
یا یکی نقشی که آن آذر و مانی است آن
\\
هر بصر کو دید او را پس به غیرش بنگرید
&&
سنگسارش کرد می باید که ارزانی است آن
\\
ای دل اندر عاشقی تو نام نیکو ترک کن
&&
کابتدای عشق رسوایی و بدنامی است آن
\\
اندرون بحر عشقش جامه جان زحمت است
&&
نام و نان جستن به عشق اندر دلا خامی است آن
\\
عشق عامه خلق خود این خاصیت دارد دلا
&&
خاصه این عشقی که زان مجلس سامی است آن
\\
خاک تبریز ای صبا تحفه بیار از بهر من
&&
زانک در عزت به جای گوهر کانی است آن
\\
\end{longtable}
\end{center}
