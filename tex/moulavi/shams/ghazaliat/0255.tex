\begin{center}
\section*{غزل شماره ۲۵۵: خیز صبوحی کن و درده صلا}
\label{sec:0255}
\addcontentsline{toc}{section}{\nameref{sec:0255}}
\begin{longtable}{l p{0.5cm} r}
خیز صبوحی کن و درده صلا
&&
خیز که صبح آمد و وقت دعا
\\
کوزه پر از می کن و در کاسه ریز
&&
خیز مزن خنبک و خم برگشا
\\
دور بگردان و مرا ده نخست
&&
جان مرا تازه کن ای جان فزا
\\
خیز که از هر طرفی بانگ چنگ
&&
در فلک انداخت ندا و صدا
\\
تنتن تنتن شنو و تن مزن
&&
وقت تو خوش ای قمر خوش لقا
\\
در سرم افکن می و پابند کن
&&
تا نروم بیهده از جا به جا
\\
زان کف دریاصفت درنثار
&&
آب درانداز چو کشتی مرا
\\
پاره چوبی بدم و از کفت
&&
گشته‌ام ای موسی جان اژدها
\\
عازر وقتم به دمت ای مسیح
&&
حشر شدم از تک گور فنا
\\
یا چو درختم که به امر رسول
&&
بیخ کشان آمدم اندر فلا
\\
هم تو بده هم تو بگو زین سپس
&&
ای دهن و کف تو گنج بقا
\\
خسرو تبریز تویی شمس دین
&&
سرور شاهان جهان علا
\\
\end{longtable}
\end{center}
