\begin{center}
\section*{غزل شماره ۳۱۶۷: از مه من مست دو صد مشتری}
\label{sec:3167}
\addcontentsline{toc}{section}{\nameref{sec:3167}}
\begin{longtable}{l p{0.5cm} r}
از مه من مست دو صد مشتری
&&
غمزه او سحر دو صد سامری
\\
هر نفسی شعله زند دین از او
&&
سوز نهد در جگر کافری
\\
آتش دل بر شده تا آسمان
&&
وز تف او گشته افق احمری
\\
دوش جمال تو همی‌شد شتاب
&&
در کف او مشعله آذری
\\
گفتم هین قصد کی داری بگو
&&
شیر خدا حمله کجا می‌بری
\\
ای تو سلیمان به سپاه و لوا
&&
خاتم تو افسر دیو و پری
\\
جان و روان سخت روان می‌روی
&&
سوی من کشته دمی ننگری
\\
نعره مستان میت نشنوی
&&
هیچ کسی را به کسی نشمری
\\
تیز همی‌کرد خیالش نظر
&&
محو شدم در تف آن ناظری
\\
نیست شدم نیست از آن شور نیست
&&
رفت ز من مهتری و کهتری
\\
مفخر تبریز شهم شمس دین
&&
شرح دهد حال من ار منکری
\\
\end{longtable}
\end{center}
