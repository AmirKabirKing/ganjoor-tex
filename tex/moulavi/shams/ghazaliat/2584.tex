\begin{center}
\section*{غزل شماره ۲۵۸۴: ماییم در این گوشه پنهان شده از مستی}
\label{sec:2584}
\addcontentsline{toc}{section}{\nameref{sec:2584}}
\begin{longtable}{l p{0.5cm} r}
ماییم در این گوشه پنهان شده از مستی
&&
ای دوست حریفان بین یک جان شده از مستی
\\
از جان و جهان رسته چون پسته دهان بسته
&&
دم‌ها زده آهسته زان راز که گفتستی
\\
ماییم در این خلوت غرقه شده در رحمت
&&
دستی صنما دستی می‌زن که از این دستی
\\
عاشق شده بر پستی بر فقر و فرودستی
&&
ای جمله بلندی‌ها خاک در این پستی
\\
جز خویش نمی‌دیدی در خویش بپیچیدی
&&
شیخا چه ترنجیدی بی‌خویش شو و رستی
\\
بربند در خانه منمای به بیگانه
&&
آن چهره که بگشادی و آن زلف که بربستی
\\
امروز مکن جانا آن شیوه که دی کردی
&&
ما را غلطی دادی از خانه برون جستی
\\
صورت چه که بربودی در سر بر ما بودی
&&
برخاستی از دیده در دلکده بنشستی
\\
شد صافی بی‌دردی عقلی که توش بردی
&&
شد داروی هر خسته آن را که توش خستی
\\
ای دل بر آن ماهی زین گفت چه می‌خواهی
&&
در قعر رو ای ماهی گر دشمن این شستی
\\
\end{longtable}
\end{center}
