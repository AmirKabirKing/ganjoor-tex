\begin{center}
\section*{غزل شماره ۱۵: ای نوش کرده نیش را بی‌خویش کن باخویش را}
\label{sec:0015}
\addcontentsline{toc}{section}{\nameref{sec:0015}}
\begin{longtable}{l p{0.5cm} r}
ای نوش کرده نیش را بی‌خویش کن باخویش را
&&
باخویش کن بی‌خویش را چیزی بده درویش را
\\
تشریف ده عشاق را پرنور کن آفاق را
&&
بر زهر زن تریاق را چیزی بده درویش را
\\
با روی همچون ماه خود با لطف مسکین خواه خود
&&
ما را تو کن همراه خود چیزی بده درویش را
\\
چون جلوه مه می‌کنی وز عشق آگه می‌کنی
&&
با ما چه همره می‌کنی چیزی بده درویش را
\\
درویش را چه بود نشان جان و زبان درفشان
&&
نی دلق صدپاره کشان چیزی بده درویش را
\\
هم آدم و آن دم تویی هم عیسی و مریم تویی
&&
هم راز و هم محرم تویی چیزی بده درویش را
\\
تلخ از تو شیرین می‌شود کفر از تو چون دین می‌شود
&&
خار از تو نسرین می‌شود چیزی بده درویش را
\\
جان من و جانان من کفر من و ایمان من
&&
سلطان سلطانان من چیزی بده درویش را
\\
ای تن پرست بوالحزن در تن مپیچ و جان مکن
&&
منگر به تن بنگر به من چیزی بده درویش را
\\
امروز ای شمع آن کنم بر نور تو جولان کنم
&&
بر عشق جان افشان کنم چیزی بده درویش را
\\
امروز گویم چون کنم یک باره دل را خون کنم
&&
وین کار را یک سون کنم چیزی بده درویش را
\\
تو عیب ما را کیستی تو مار یا ماهیستی
&&
خود را بگو تو چیستی چیزی بده درویش را
\\
جان را درافکن در عدم زیرا نشاید ای صنم
&&
تو محتشم او محتشم چیزی بده درویش را
\\
\end{longtable}
\end{center}
