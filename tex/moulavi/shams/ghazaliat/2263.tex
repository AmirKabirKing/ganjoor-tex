\begin{center}
\section*{غزل شماره ۲۲۶۳: یا قمرا لوعه للقمرین سکن}
\label{sec:2263}
\addcontentsline{toc}{section}{\nameref{sec:2263}}
\begin{longtable}{l p{0.5cm} r}
یا قمرا لوعه للقمرین سکن
&&
حلت علی حریمهم فی خطر لیمنوا
\\
یا شجرا غصونه فوق سماء وهمنا
&&
هز هز فی قلوبنا مرحمه لنجتنوا
\\
هر کی تو گردنش زدی گشت درازگردن او
&&
خرمن هر کی سوختی گشت بزرگ خرمن او
\\
هر کی سرش شکافتی سر بفراخت بر فلک
&&
هر کی تو در چهش کنی یافت جهان روشن او
\\
یا بلدا مخلدا افلح من ثوی به
&&
للبرکات مطلع للثمرات معدن
\\
یا سحرا منورا لیس عقیبه دجی
&&
افلح کل منظر ذاک به مزین
\\
هر کی طرب رها کند پشت سوی وفا کند
&&
بازکشاندش به خود با کرم مفتن او
\\
می‌کشدش که ای رهی از کف من کجا رهی
&&
رو به من آورید هین ها الذین آمنوا
\\
جاء اوان وصلنا یلحقنا باصلنا
&&
شممنا عبیره فانتهضوا لتیقنوا
\\
ما بقی انسلاخنا ان هنا مناخنا
&&
فی عرفات معشر ابتکروا و احسنوا
\\
پند نگار خود شنو از بر او برون مرو
&&
ای دل و دیده دیده‌ای ای دل و دیده من او
\\
پیش خودم همی‌نشان بر سر من همی‌فشان
&&
تا ز تو لاف می‌زنم کم بگرفت دامن او
\\
قد نطق الهوی اسکتوا استمعوا و انصتوا
&&
ان لسان نطقنا عند لقاه الکن
\\
بستم من دهان خود دل بگشاد صد دهان
&&
بهر دل تو تن زدم بس بودم نوازن او
\\
در گل و در شکر نشین بهر خدای لطف بین
&&
سیب و انار تازه چین کمد در فشاندن او
\\
\end{longtable}
\end{center}
