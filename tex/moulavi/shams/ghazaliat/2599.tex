\begin{center}
\section*{غزل شماره ۲۵۹۹: ای بر سر بازارت صد خرقه به زناری}
\label{sec:2599}
\addcontentsline{toc}{section}{\nameref{sec:2599}}
\begin{longtable}{l p{0.5cm} r}
ای بر سر بازارت صد خرقه به زناری
&&
وز روی تو در عالم هر روی به دیواری
\\
هر ذره ز خورشیدت گویای اناالحقی
&&
هر گوشه چو منصوری آویخته بر داری
\\
این طرفه که از یک خم هر یک ز میی مستند
&&
این طرفه که از یک گل در هر قدمی خاری
\\
هر شاخ همی‌گوید من مست شدم دستی
&&
هر عقل همی‌گوید من خیره شدم باری
\\
گل از سر مشتاقی بدریده گریبانی
&&
عشق از سر بی‌خویشی انداخته دستاری
\\
از عقل گروهی مست بی‌عقل گروهی مست
&&
جز عاقل و لایعقل قومی دگرند آری
\\
ماییم چو کوه طور مست از قدح موسی
&&
بی‌زحمت فرعونی بی‌غصه اغیاری
\\
ماییم چو می جوشان در خم خراباتی
&&
گر چه سر خم بسته است از کهگل پنداری
\\
از جوشش می کهگل شد بر سر خم رقصان
&&
والله که از این خوشتر نبود به جهان کاری
\\
\end{longtable}
\end{center}
