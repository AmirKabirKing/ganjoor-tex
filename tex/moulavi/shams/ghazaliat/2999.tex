\begin{center}
\section*{غزل شماره ۲۹۹۹: تا چند از فراق مرا کار بشکنی}
\label{sec:2999}
\addcontentsline{toc}{section}{\nameref{sec:2999}}
\begin{longtable}{l p{0.5cm} r}
تا چند از فراق مرا کار بشکنی
&&
زاریم نشنوی و مرا زار بشکنی
\\
دستم شکست دست فراقت ز کار و بار
&&
دانستمی دگر به چه مقدار بشکنی
\\
هین شیشه باز هجر رسیدی به سنگلاخ
&&
کاین شیشه‌ام تنک شد هشدار بشکنی
\\
زین سنگلاخ هجر سوی سبزه زار وصل
&&
گر زوترک نرانی ناچار بشکنی
\\
خونم فسرده شد به دل اندر چو ناردانگ
&&
خونش چنین دود چو دل نار بشکنی
\\
باری چو بشکنی دل پرحسرت مرا
&&
در وصل روی دلبر عیار بشکنی
\\
مخدوم شمس دین که شهنشاه بینشی
&&
کز یک نظر دو صد دل و دلدار بشکنی
\\
تبریز از تو فخر به اینت مسلم است
&&
صد تاج را به ریشه دستار بشکنی
\\
\end{longtable}
\end{center}
