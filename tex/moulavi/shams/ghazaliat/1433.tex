\begin{center}
\section*{غزل شماره ۱۴۳۳: من آنم کز خیالاتش تراشنده وثن باشم}
\label{sec:1433}
\addcontentsline{toc}{section}{\nameref{sec:1433}}
\begin{longtable}{l p{0.5cm} r}
من آنم کز خیالاتش تراشنده وثن باشم
&&
چو هنگام وصال آمد بتان را بت شکن باشم
\\
مرا چون او ولی باشد چه سخره بوعلی باشم
&&
چو حسن خویش بنماید چه بند بوالحسن باشم
\\
دو صورت پیش می آرد گهی شمع است و گه شاهد
&&
دوم را من چو آیینه نخستین را لگن باشم
\\
مرا وامی است در گردن که بسپارم به عشقش جان
&&
ولی نگزارمش تا از تقاضا ممتحن باشم
\\
چو زندانم بود چاهی که در قعرش بود یوسف
&&
خنک جان من آن روزی که در زندان شدن باشم
\\
چو دست او رسن باشد که دست چاهیان گیرد
&&
چه دستک‌ها زنم آن دم که پابست رسن باشم
\\
مرا گوید چه می نالی ز عشقی تا که راهت زد
&&
خنک آن کاروان کش من در این ره راه زن باشم
\\
چو چنگم لیک اگر خواهی که دانی وقت ساز من
&&
غنیمت دار آن دم را که در تن تن تنن باشم
\\
چو یار ذوفنون من زند پرده جنون من
&&
خدا داند دگر کس نی که آن دم در چه فن باشم
\\
ز کوب غم چه غم دارم که با او پای می کوبم
&&
چه تلخی آیدم چون من بر شیرین ذقن باشم
\\
چو بیش از صد جهان دارم چرا در یک جهان باشم
&&
چو پخته شد کباب من چرا در بابزن باشم
\\
کبوترباز عشقش را کبوتر بود جان من
&&
چو برج خویش را دیدم چرا اندر بدن باشم
\\
گهی با خویش در جنگم گهی بی‌خویشم و دنگم
&&
چو آمد یار گلرنگم چرا با این سه فن باشم
\\
چو در گرمابه عشقش حجابی نیست جان‌ها را
&&
نیم من نقش گرمابه چرا در جامه کن باشم
\\
خمش کن ای دل گویا که من آواره خواهم شد
&&
وطن آتش گرفت از تو چگونه در وطن باشم
\\
اگر من در وطن باشم وگر بیرون ز تن باشم
&&
ز تاب شمس تبریزی سهیل اندر یمن باشم
\\
\end{longtable}
\end{center}
