\begin{center}
\section*{غزل شماره ۲۹۴: بریده شد از این جوی جهان آب}
\label{sec:0294}
\addcontentsline{toc}{section}{\nameref{sec:0294}}
\begin{longtable}{l p{0.5cm} r}
بریده شد از این جوی جهان آب
&&
بهارا بازگرد و وارسان آب
\\
از آن آبی که چشمه خضر و الیاس
&&
ندیدست و نبیند آن چنان آب
\\
زهی سرچشمه‌ای کز فر جوشش
&&
بجوشد هر دمی از عین جان آب
\\
چو باشد آب‌ها نان‌ها برویند
&&
ولی هرگز نرست ای جان ز نان آب
\\
برای لقمه‌ای نان چون گدایان
&&
مریز از روی فقر ای میهمان آب
\\
سراسر جمله عالم نیم لقمه‌ست
&&
ز حرص نیم لقمه شد نهان آب
\\
زمین و آسمان دلو و سبویند
&&
برون‌ست از زمین و آسمان آب
\\
تو هم بیرون رو از چرخ و زمین زود
&&
که تا بینی روان از لامکان آب
\\
رهد ماهی جان تو از این حوض
&&
بیاشامد ز بحر بی‌کران آب
\\
در آن بحری که خضرانند ماهی
&&
در او جاوید ماهی جاودان آب
\\
از آن دیدار آمد نور دیده
&&
از آن بام‌ست اندر ناودان آب
\\
از آن باغ‌ست این گل‌های رخسار
&&
از آن دولاب یابد گلستان آب
\\
از آن نخل‌ست خرماهای مریم
&&
نه ز اسباب‌ست و زین ابواب آن آب
\\
روان و جانت آنگه شاد گردد
&&
کز این جا سوی تو آید روان آب
\\
مزن چوبک دگر چون پاسبانان
&&
که هست این ماهیان را پاسبان آب
\\
\end{longtable}
\end{center}
