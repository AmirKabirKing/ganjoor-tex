\begin{center}
\section*{غزل شماره ۲۱۹۵: خوش خرامان می‌روی ای جان جان بی‌من مرو}
\label{sec:2195}
\addcontentsline{toc}{section}{\nameref{sec:2195}}
\begin{longtable}{l p{0.5cm} r}
خوش خرامان می‌روی ای جان جان بی‌من مرو
&&
ای حیات دوستان در بوستان بی‌من مرو
\\
ای فلک بی‌من مگرد و ای قمر بی‌من متاب
&&
ای زمین بی‌من مروی و ای زمان بی‌من مرو
\\
این جهان با تو خوش است و آن جهان با تو خوش است
&&
این جهان بی‌من مباش و آن جهان بی‌من مرو
\\
ای عیان بی‌من مدان و ای زبان بی‌من مخوان
&&
ای نظر بی‌من مبین و ای روان بی‌من مرو
\\
شب ز نور ماه روی خویش را بیند سپید
&&
من شبم تو ماه من بر آسمان بی‌من مرو
\\
خار ایمن گشت ز آتش در پناه لطف گل
&&
تو گلی من خار تو در گلستان بی‌من مرو
\\
در خم چوگانت می‌تازم چو چشمت با من است
&&
همچنین در من نگر بی‌من مران بی‌من مرو
\\
چون حریف شاه باشی ای طرب بی‌من منوش
&&
چون به بام شه روی ای پاسبان بی‌من مرو
\\
وای آن کس کو در این ره بی‌نشان تو رود
&&
چو نشان من تویی ای بی‌نشان بی‌من مرو
\\
وای آن کو اندر این ره می‌رود بی‌دانشی
&&
دانش راهم تویی ای راه دان بی‌من مرو
\\
دیگرانت عشق می‌خوانند و من سلطان عشق
&&
ای تو بالاتر ز وهم این و آن بی‌من مرو
\\
\end{longtable}
\end{center}
