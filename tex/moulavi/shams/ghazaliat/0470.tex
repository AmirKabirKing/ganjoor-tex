\begin{center}
\section*{غزل شماره ۴۷۰: ای غم اگر مو شوی پیش منت بار نیست}
\label{sec:0470}
\addcontentsline{toc}{section}{\nameref{sec:0470}}
\begin{longtable}{l p{0.5cm} r}
ای غم اگر مو شوی پیش منت بار نیست
&&
در شکرینه یقین سرکه انکار نیست
\\
گر چه تو خون خواره‌ای رهزن و عیاره‌ای
&&
قبله ما غیر آن دلبر عیار نیست
\\
کان شکرهاست او مستی سرهاست او
&&
ره نبرد با وی آنک مرغ شکرخوار نیست
\\
هر که دلی داشتست بنده دلبر شدست
&&
هر که ندارد دلی طالب دلدار نیست
\\
کل چه کند شانه را چونک ورا موی نیست
&&
پود چه کار آیدش آنک ورا تار نیست
\\
با سر میدان چه کار آن که بود خرسوار
&&
تا چه کند صیرفی هر کش دینار نیست
\\
جان کلیم و خلیل جانب آتش دوان
&&
نار نماید در او جز گل و گلزار نیست
\\
ای غم از این جا برو ور نه سرت شد گرو
&&
رنگ شب تیره را تاب مه یار نیست
\\
ای غم پرخار رو در دل غمخوار رو
&&
نقل بخیلانه‌ات طعمه خمار نیست
\\
حلقهٔ غین تو تنگ میمت از آن تنگتر
&&
تنگ متاع تو را عشق خریدار نیست
\\
ای غم شادی شکن پر شکرست این دهن
&&
کز شکرآکندگی ممکن گفتار نیست
\\
\end{longtable}
\end{center}
