\begin{center}
\section*{غزل شماره ۵۵۷: دل چو بدید روی تو چون نظرش به جان بود}
\label{sec:0557}
\addcontentsline{toc}{section}{\nameref{sec:0557}}
\begin{longtable}{l p{0.5cm} r}
دل چو بدید روی تو چون نظرش به جان بود
&&
جان ز لبت چو می‌کشد خیره و لب گزان بود
\\
تن برود به پیش دل کاین همه را چه می‌کنی
&&
گوید دل که از مهی کز نظرت نهان بود
\\
جز رخ دل نظر مکن جز سوی دل گذر مکن
&&
زانک به نور دل همه شعله آن جهان بود
\\
شیخ شیوخ عالمست آن که تو راست نومرید
&&
آن که گرفت دست تو خاصبک زمان بود
\\
دل به میان چو پیر دین حلقه تن به گرد او
&&
شاد تنی که پیر دل شسته در آن میان بود
\\
راز دل تو شمس دین در تبریز بشنود
&&
دور ز گوش و جان او کز سخنت گران بود
\\
\end{longtable}
\end{center}
