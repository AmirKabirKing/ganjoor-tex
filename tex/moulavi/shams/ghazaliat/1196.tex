\begin{center}
\section*{غزل شماره ۱۱۹۶: عاشقان را شد مسلم شب نشستن تا به روز}
\label{sec:1196}
\addcontentsline{toc}{section}{\nameref{sec:1196}}
\begin{longtable}{l p{0.5cm} r}
عاشقان را شد مسلم شب نشستن تا به روز
&&
خوردنی و خواب نی اندر هوای دلفروز
\\
گر تو یارا عاشقی ماننده این شمع باش
&&
جمله شب می‌گداز و جمله شب خوش می‌بسوز
\\
غیر عاشق دان که چون سرما بود اندر خزان
&&
در میان آن خزان باشد دل عاشق تموز
\\
گر تو عشقی داری ای جان از پی اعلام را
&&
عاشقانه نعره‌ای زن عاشقانه فوز فوز
\\
ور تو بند شهوتی دعوی عشاقی مکن
&&
در ببند اندر خلاء و شهوت خود را بسوز
\\
عاشق و شهوت کجا جمع آید ای تو ساده دل
&&
عیسی و خر در یکی آخر کجا دارند پوز
\\
گر همی‌خواهی که بویی بشنوی زین رمزها
&&
چشم را از غیر شمس الدین تبریزی بدوز
\\
ور نبینی کز دو عالم برتر آمد شمس دین
&&
بر تک دریای غفلت مرده ریگی تو هنوز
\\
رو به کتاب تعلم گرد علم فقه گرد
&&
تا سرافرازی شوی اندر یجوز و لایجوز
\\
جان من از عشق شمس الدین ز طفلی دور شد
&&
عشق او زین پس نماند با مویز و جوز و کوز
\\
عقل من از دست رفت و شعر من ناقص بماند
&&
زان کمانم هست عریان از لباس نقش و توز
\\
ای جلال الدین بخسپ و ترک کن املا بگو
&&
که تک آن شیر را اندرنیابد هیچ یوز
\\
\end{longtable}
\end{center}
