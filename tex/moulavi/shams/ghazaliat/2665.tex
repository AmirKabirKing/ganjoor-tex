\begin{center}
\section*{غزل شماره ۲۶۶۵: صلا ای صوفیان کامروز باری}
\label{sec:2665}
\addcontentsline{toc}{section}{\nameref{sec:2665}}
\begin{longtable}{l p{0.5cm} r}
صلا ای صوفیان کامروز باری
&&
سماع است و نشاط و عیش آری
\\
صلا کز شش جهت درها گشاده‌ست
&&
ز قعر بحر پیدا شد غباری
\\
صلا کاین مغزها امروز پر شد
&&
ز بوی وصل جانی جان سپاری
\\
صلا که یافت هر گوشی و هوشی
&&
ز بی‌هوشی مطلق گوشواری
\\
صلا که ساعتی دیگر نیابی
&&
ز مشرق تا به مغرب هوشیاری
\\
در آن میدان که دیاری نمی‌گشت
&&
به هر گوشه‌ست روحانی سواری
\\
چو هیزم اندر این آتش درآیید
&&
که تا هفتم فلک دارد شراری
\\
میان شوره خاک نفس جز وی
&&
به هر سویی درختی جویباری
\\
تو اندر باغ‌ها دیدی که گیرد
&&
درختی مر درختی را کناری
\\
\end{longtable}
\end{center}
