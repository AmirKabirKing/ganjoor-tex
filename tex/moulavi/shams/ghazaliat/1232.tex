\begin{center}
\section*{غزل شماره ۱۲۳۲: هنگام صبوح آمد ای مرغ سحرخوانش}
\label{sec:1232}
\addcontentsline{toc}{section}{\nameref{sec:1232}}
\begin{longtable}{l p{0.5cm} r}
هنگام صبوح آمد ای مرغ سحرخوانش
&&
با زهره درآ گویان در حلقه مستانش
\\
هر جان که بود محرم بیدار کنش آن دم
&&
وان کو نبود محرم تا حشر بخسبانش
\\
می‌گو سخنش بسته در گوش دل آهسته
&&
تا کفر به پیش آرد صد گوهر ایمانش
\\
یک برق ز عشق شه بر چرخ زند ناگه
&&
آتش فتد اندر مه برهم زند ارکانش
\\
آن جا که عنایت‌ها بخشید ولایت‌ها
&&
آن جا چه زند کوشش آن جا چه بود دانش
\\
آن جا که نظر باشد هر کار چو زر باشد
&&
بی‌دست برد چوگان هر گوی ز میدانش
\\
شمس الحق تبریزی کو هر دل بی‌دل را
&&
می‌آرد و می‌آرد تا حضرت سلطانش
\\
\end{longtable}
\end{center}
