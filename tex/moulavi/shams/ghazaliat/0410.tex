\begin{center}
\section*{غزل شماره ۴۱۰: دوش آمد بر من آنک شب افروز منست}
\label{sec:0410}
\addcontentsline{toc}{section}{\nameref{sec:0410}}
\begin{longtable}{l p{0.5cm} r}
دوش آمد بر من آنک شب افروز منست
&&
آمدن باری اگر در دو جهان آمدنست
\\
آنک سرسبزی خاک‌ست و گهربخش فلک
&&
چاشنی بخش وطن‌هاست اگر بی‌وطنست
\\
در کف عقل نهد شمع که بستان و بیا
&&
تا در من که شفاخانه هر ممتحن است
\\
شمع را تو گرو این لگن تن چه کنی
&&
این لگن گر نبود شمع تو را صد لگنست
\\
تا در این آب و گلی کار کلوخ اندازیست
&&
گفت و گو جمله کلوخ‌ست و یقین دل شکنست
\\
گوهر آینه جان همه در ساده دلی‌ست
&&
میل تو بهر تصدر همه در فضل و فن است
\\
زین گذر کن صفت یار شکربخش بگو
&&
که ز عشوه شکرش ذره به ذره دهن است
\\
خیره گشته است صفت‌ها همه کان چه صفت است
&&
کان صفت‌ها چو بتان و صفت او شمن است
\\
چشم نرگس نشناسد ز غمش کاندر باغ
&&
پیش او یاسمن است آن گل تر یا سمنست
\\
روش عشق روش بخش بود بی‌پا را
&&
خوش روانش کند ار خود زمن صد زمنست
\\
در جهان فتنه بسی بود و بسی خواهد بود
&&
فتنه‌ها جمله بر آن فتنه ما مفتتنست
\\
همه دل‌ها چو کبوتر گرو آن برجند
&&
زانک جانی است که او زنده کن هر بدنست
\\
بس کن آخر چه بر این گفت زبان چفسیدی
&&
عشق را چند بیان‌ها است که فوق سخنست
\\
\end{longtable}
\end{center}
