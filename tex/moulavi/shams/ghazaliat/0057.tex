\begin{center}
\section*{غزل شماره ۵۷: مسلمانان مسلمانان چه باید گفت یاری را}
\label{sec:0057}
\addcontentsline{toc}{section}{\nameref{sec:0057}}
\begin{longtable}{l p{0.5cm} r}
مسلمانان مسلمانان چه باید گفت یاری را
&&
که صد فردوس می‌سازد جمالش نیم خاری را
\\
مکان‌ها بی‌مکان گردد زمین‌ها جمله کان گردد
&&
چو عشق او دهد تشریف یک لحظه دیاری را
\\
خداوندا زهی نوری لطافت بخش هر حوری
&&
که آب زندگی سازد ز روی لطف ناری را
\\
چو لطفش را بیفشارد هزاران نوبهار آرد
&&
چه نقصان گر ز غیرت او زند برهم بهاری را
\\
جمالش آفتاب آمد جهان او را نقاب آمد
&&
ولیکن نقش کی بیند به جز نقش و نگاری را
\\
جمال گل گواه آمد که بخشش‌ها ز شاه آمد
&&
اگر چه گل بنشناسد هوای سازواری را
\\
اگر گل را خبر بودی همیشه سرخ و تر بودی
&&
ازیرا آفتی ناید حیات هوشیاری را
\\
به دست آور نگاری تو کز این دستست کار تو
&&
چرا باید سپردن جان نگاری جان سپاری را
\\
ز شمس الدین تبریزی منم قاصد به خون ریزی
&&
که عشقی هست در دستم که ماند ذوالفقاری را
\\
\end{longtable}
\end{center}
