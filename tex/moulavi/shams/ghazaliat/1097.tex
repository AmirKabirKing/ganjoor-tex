\begin{center}
\section*{غزل شماره ۱۰۹۷: عقل بند ره روانست ای پسر}
\label{sec:1097}
\addcontentsline{toc}{section}{\nameref{sec:1097}}
\begin{longtable}{l p{0.5cm} r}
عقل بند ره روانست ای پسر
&&
بند بشکن ره عیانست ای پسر
\\
عقل بند و دل فریب و جان حجاب
&&
راه از این هر سه نهانست ای پسر
\\
چون ز عقل و جان و دل برخاستی
&&
این یقین هم در گمانست ای پسر
\\
مرد کو از خود نرفت او مرد نیست
&&
عشق بی‌درد آفسانست ای پسر
\\
سینه خود را هدف کن پیش دوست
&&
هین که تیرش در کمانست ای پسر
\\
سینه‌ای کز زخم تیرش خسته شد
&&
در جبینش صد نشانست ای پسر
\\
عشق کار نازکان نرم نیست
&&
عشق کار پهلوانست ای پسر
\\
هر کی او مر عاشقان را بنده شد
&&
خسرو و صاحب قرانست ای پسر
\\
عشق را از کس مپرس از عشق پرس
&&
عشق ابر درفشانست ای پسر
\\
ترجمانی منش محتاج نیست
&&
عشق خود را ترجمانست ای پسر
\\
گر روی بر آسمان هفتمین
&&
عشق نیکونردبانست ای پسر
\\
هر کجا که کاروانی می‌رود
&&
عشق قبله کاروانست ای پسر
\\
این جهان از عشق تا نفریبدت
&&
کاین جهان از تو جهانست ای پسر
\\
هین دهان بربند و خامش چون صدف
&&
کاین زبانت خصم جانست ای پسر
\\
شمس تبریز آمد و جان شادمان
&&
چونک با شمسش قرانست ای پسر
\\
\end{longtable}
\end{center}
