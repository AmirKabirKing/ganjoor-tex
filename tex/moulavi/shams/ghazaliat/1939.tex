\begin{center}
\section*{غزل شماره ۱۹۳۹: می پرد این مرغ دیگر در جنان عاشقان}
\label{sec:1939}
\addcontentsline{toc}{section}{\nameref{sec:1939}}
\begin{longtable}{l p{0.5cm} r}
می پرد این مرغ دیگر در جنان عاشقان
&&
سوی عنقا می کشاند استخوان عاشقان
\\
ای دریغا چشم بودی تا بدیدی در هوا
&&
تا روان دیدی روان گشته روان عاشقان
\\
اشتران سربریده پای بالا می نهند
&&
اشتر باسر مجو در کاروان عاشقان
\\
آن جنازه برپریدی گر نگفتی غیرتش
&&
بی نشان رو بی‌نشان رو بی‌نشان عاشقان
\\
چون به گورستان درآید استخوان عاشقی
&&
صد نواله پیچد از وی میرخوان عاشقان
\\
ذره ذره دف زدی و کف زدی در عرس او
&&
گر روا بودی شدن پیدا نهان عاشقان
\\
چون تن عاشق درآید همچو گنجی در زمین
&&
صد دریچه برگشاید آسمان عاشقان
\\
در کفن پیچید بینید ای عزیزان کوه قاف
&&
چشم بند است این عجب یا امتحان عاشقان
\\
خرمن گل بود و شد از مرگ شاخ زعفران
&&
صد گلستان بیش ارزد زعفران عاشقان
\\
ای رسول غیرت مردان دهانم را مگیر
&&
تا دو سه نکته بگویم از زبان عاشقان
\\
\end{longtable}
\end{center}
