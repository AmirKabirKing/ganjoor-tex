\begin{center}
\section*{غزل شماره ۱۵۹۲: نی تو گفتی از جفای آن جفاگر نشکنم}
\label{sec:1592}
\addcontentsline{toc}{section}{\nameref{sec:1592}}
\begin{longtable}{l p{0.5cm} r}
نی تو گفتی از جفای آن جفاگر نشکنم
&&
نی تو گفتی عالمی در عشق او برهم زنم
\\
نی تو دست او گرفتی عهد کردی دو به دو
&&
کز پی آن جان و دل این جان و دل را برکنم
\\
نور چشمت چون منم دورم مبین ای نور چشم
&&
سوی بالا بنگر آخر زانک من بر روزنم
\\
ای سررشته طرب‌ها عیسی دوران تویی
&&
سر از این روزن فروکن گر چه من چون سوزنم
\\
عشق را روز قیامت آتش و دودی بود
&&
نور آن آتش تو باشی دود آن آتش منم
\\
تا نبینم روی چون گلزار آن صد نوبهار
&&
همچو لاله من سیه دل صدزبان چون سوسنم
\\
شاه شمس الدین تبریزی منت عاشق بسم
&&
روز بزمت همچو مومم روز رزمت آهنم
\\
\end{longtable}
\end{center}
