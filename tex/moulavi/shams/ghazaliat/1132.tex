\begin{center}
\section*{غزل شماره ۱۱۳۲: چون سر کس نیستت فتنه مکن دل مبر}
\label{sec:1132}
\addcontentsline{toc}{section}{\nameref{sec:1132}}
\begin{longtable}{l p{0.5cm} r}
چون سر کس نیستت فتنه مکن دل مبر
&&
چونک ببردی دلی پرده او را مدر
\\
چشم تو چون رهزند ره زده را ره نما
&&
زلف تو چون سر کشد عشوه هندو مخر
\\
عشق بود دلستان پرورش دوستان
&&
سبز و شکفته کند باغ تو را چون شجر
\\
وجهک وجه القمر قلبک مثل الحجر
&&
روحک روح البقا حسنک نور البصر
\\
عشق خران جو به جو تا لب دریای هو
&&
کهنه خران کو به کو اسکی ببج کآمدور
\\
دشمن ما در هنر شد به مثل دنب خر
&&
چند بپیماییش نیست فزون کم شمر
\\
اقسم بالعادیات احلف بالموریات
&&
غیرک یا ذالصلات فی نظری کالمدر
\\
هر که به جز عاشق است در ترشی لایقست
&&
لایق حلوا شکر لایق سرکا کبر
\\
هجرک روحی فداک زلزلنی فی هواک
&&
کل کریم سواک فهو خداع غرر
\\
عشق خوش و تازه رو عاشق او تازه تر
&&
شکل جهان کهنه‌ای عاشق او کهنه تر
\\
\end{longtable}
\end{center}
