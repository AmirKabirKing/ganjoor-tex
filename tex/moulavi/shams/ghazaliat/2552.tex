\begin{center}
\section*{غزل شماره ۲۵۵۲: کجا باشد دورویان را میان عاشقان جایی}
\label{sec:2552}
\addcontentsline{toc}{section}{\nameref{sec:2552}}
\begin{longtable}{l p{0.5cm} r}
کجا باشد دورویان را میان عاشقان جایی
&&
که با صد رو طمع دارد ز روز عشق فردایی
\\
طمع دارند و نبودشان که شاه جان کند ردشان
&&
ز آهن سازد او سدشان چو ذوالقرنین آسایی
\\
دورویی با چنان رویی پلیدی در چنان جویی
&&
چه گنجد پیش صدیقان نفاقی کارفرمایی
\\
که بیخ بیشه جان را همه رگ‌های شیران را
&&
بداند یک به یک آن را بدیده نورافزایی
\\
بداند عاقبت‌ها را فرستد راتبت‌ها را
&&
ببخشد عافیت‌ها را به هر صدیق و یکتایی
\\
براندازد نقابی را نماید آفتابی را
&&
دهد نوری خدایی را کند او تازه انشایی
\\
اگر این شه دورو باشد نه آتش خلق و خو باشد
&&
برای جست و جو باشد ز فکر نفس کژپایی
\\
دورویی او است بی‌کینه ازیرا او است آیینه
&&
ز عکس تو در آن سینه نماید کین و بدرایی
\\
مزن پهلو به آن نوری که مانی تا ابد کوری
&&
تو با شیران مکن زوری که روباهی به سودایی
\\
که با شیران مری کردن سگان را بشکند گردن
&&
نه مکری ماند و نی فن و نه دورویی نه صدتایی
\\
\end{longtable}
\end{center}
