\begin{center}
\section*{غزل شماره ۲۱۱: باز بنفشه رسید جانب سوسن دوتا}
\label{sec:0211}
\addcontentsline{toc}{section}{\nameref{sec:0211}}
\begin{longtable}{l p{0.5cm} r}
باز بنفشه رسید جانب سوسن دوتا
&&
باز گل لعل پوش می‌بدراند قبا
\\
بازرسیدند شاد زان سوی عالم چو باد
&&
مست و خرامان و خوش سبزقبایان ما
\\
سرو علمدار رفت سوخت خزان را به تفت
&&
وز سر که رخ نمود لاله شیرین لقا
\\
سنبله با یاسمین گفت سلام علیک
&&
گفت علیک السلام در چمن آی ای فتا
\\
یافته معروفیی هر طرفی صوفیی
&&
دست زنان چون چنار رقص کنان چون صبا
\\
غنچه چو مستوریان کرده رخ خود نهان
&&
باد کشد چادرش کای سره رو برگشا
\\
یار در این کوی ما آب در این جوی ما
&&
زینت نیلوفری تشنه و زردی چرا
\\
رفت دی روترش کشته شد آن عیش کش
&&
عمر تو بادا دراز ای سمن تیزپا
\\
نرگس در ماجرا چشمک زد سبزه را
&&
سبزه سخن فهم کرد گفت که فرمان تو را
\\
گفت قرنفل به بید من ز تو دارم امید
&&
گفت عزبخانه‌ام خلوت توست الصلا
\\
سیب بگفت ای ترنج از چه تو رنجیده‌ای
&&
گفت من از چشم بد می‌نشوم خودنما
\\
فاخته با کو و کو آمد کان یار کو
&&
کردش اشارت به گل بلبل شیرین نوا
\\
غیر بهار جهان هست بهاری نهان
&&
ماه رخ و خوش دهان باده بده ساقیا
\\
یا قمرا طالعا فی الظلمات الدجی
&&
نور مصابیحه یغلب شمس الضحی
\\
چند سخن ماند لیک بی‌گه و دیرست نیک
&&
هر چه به شب فوت شد آرم فردا قضا
\\
\end{longtable}
\end{center}
