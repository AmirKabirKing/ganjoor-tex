\begin{center}
\section*{غزل شماره ۲۵۳۱: چو سرمست منی ای جان ز درد سر چه غم داری}
\label{sec:2531}
\addcontentsline{toc}{section}{\nameref{sec:2531}}
\begin{longtable}{l p{0.5cm} r}
چو سرمست منی ای جان ز درد سر چه غم داری
&&
چو آهوی منی ای جان ز شیر نر چه غم داری
\\
چو مه روی تو من باشم ز سال و مه چه اندیشی
&&
چو شور و شوق من هستت ز شور و شر چه غم داری
\\
چو کان نیشکر گشتی ترش رو از چه می‌باشی
&&
براق عشق رامت شد ز مرگ خر چه غم داری
\\
چو من با تو چنین گرمم چه آه سرد می‌آری
&&
چو بر بام فلک رفتی ز خشک و تر چه غم داری
\\
خوش آوازی من دیدی دواسازی من دیدی
&&
رسن بازی من دیدی از این چنبر چه غم داری
\\
بر این صورت چه می‌چفسی ز بی‌معنی چه می‌ترسی
&&
چو گوهر در بغل داری ز بی‌گوهر چه غم داری
\\
ایا یوسف ز دست تو کی بگریزد ز شست تو
&&
همه مصرند مست تو ز کور و کر چه غم داری
\\
چو با دل یار غاری تو چراغ چار یاری تو
&&
فقیر ذوالفقاری تو از آن خنجر چه غم داری
\\
گرفتی باغ و برها را همی‌خور آن شکرها را
&&
اگر بستند درها را ز بند در چه غم داری
\\
چو مد و جر خود دیدی چو بال و پر خود دیدی
&&
چو کر و فر خود دیدی ز هر بی‌فر چه غم داری
\\
ایا ای جان جان جان پناه جان مهمانان
&&
ایا سلطان سلطانان تو از سنجر چه غم داری
\\
خمش کن همچو ماهی تو در آن دریای خوش دررو
&&
چو اندر قعر دریایی تو از آذر چه غم داری
\\
\end{longtable}
\end{center}
