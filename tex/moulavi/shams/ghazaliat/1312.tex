\begin{center}
\section*{غزل شماره ۱۳۱۲: فریفت یار شکربار من مرا به طریق}
\label{sec:1312}
\addcontentsline{toc}{section}{\nameref{sec:1312}}
\begin{longtable}{l p{0.5cm} r}
فریفت یار شکربار من مرا به طریق
&&
که شعر تازه بگو و بگیر جام عتیق
\\
چه چاره آنچ بگوید ببایدم کردن
&&
چگونه عاق شوم با حیات کان و عقیق
\\
غلام ساقی خویشم شکار عشوه او
&&
که سکر لذت عیش است و باده نعم رفیق
\\
به شب مثال چراغند و روز چون خورشید
&&
ز عاشقی و ز مستی زهی گزیده فریق
\\
شما و هر چه مراد شماست از بد و نیک
&&
من و منازل ساقی و جام‌های رحیق
\\
بیار باده لعلی که در معادن روح
&&
درافکند شررش صد هزار جوش و حریق
\\
روا بود چو تو خورشید و در زمین سایه
&&
روا بود چو تو ساقی و در زمانه مفیق
\\
گشای زانوی اشتر بدر عقال عقول
&&
بجه ز رق جهانی به جرعه‌های رقیق
\\
چو زانوی شتر تو گشاده شد ز عقال
&&
اگر چه خفته بود طایرست در تحقیق
\\
همی‌دود به که و دشت و بر و بحر روان
&&
به قدر عقل تو گفتم نمی‌کنم تعمیق
\\
کمال عشق در آمیزش‌ست پیش آیید
&&
به اختلاط مخلد چو روغن و چو سویق
\\
چو اختلاط کند خاک با حقایق پاک
&&
کند سجود مخلد به شکر آن توقیق
\\
\end{longtable}
\end{center}
