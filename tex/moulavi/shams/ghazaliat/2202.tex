\begin{center}
\section*{غزل شماره ۲۲۰۲: ای جهان برهم زده سودای تو سودای تو}
\label{sec:2202}
\addcontentsline{toc}{section}{\nameref{sec:2202}}
\begin{longtable}{l p{0.5cm} r}
ای جهان برهم زده سودای تو سودای تو
&&
چاشنی عمرم از حلوای تو حلوای تو
\\
دامن گردون پر از در است و مروارید و لعل
&&
می‌دوانند جانب دریای تو دریای تو
\\
جان‌های عاشقان چون سیل‌ها غلطان شده
&&
تا بریزد جمله را در پای تو در پای تو
\\
جان‌های عاشقان چون سیل‌ها غلطان شده
&&
می‌دوانند جانب دریای تو دریای تو
\\
ای خمار عاشقان از باده‌های دوش تو
&&
وی خراب امروزم از فردای تو فردای تو
\\
من نظر کردم به جان ساده بی‌رنگ خویش
&&
زرد دیدم نقشش از صفرای تو صفرای تو
\\
چون نظر کردم نکو من در صفای گوهرت
&&
ماه رخ بنمود از سیمای تو سیمای تو
\\
ماه خواندم من تو را بس جرم دارم زین سخن
&&
مه کی باشد کو بود همتای تو همتای تو
\\
این چنین گوید خداوند شمس تبریزی بنام
&&
ای همه شهر دلم غوغای تو غوغای تو
\\
\end{longtable}
\end{center}
