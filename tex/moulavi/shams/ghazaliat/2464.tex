\begin{center}
\section*{غزل شماره ۲۴۶۴: هر طربی که در جهان گشت ندیم کهتری}
\label{sec:2464}
\addcontentsline{toc}{section}{\nameref{sec:2464}}
\begin{longtable}{l p{0.5cm} r}
هر طربی که در جهان گشت ندیم کهتری
&&
می‌برمد از او دلم چون دل تو ز مقذری
\\
هر هنری و هر رهی کان برسد به ابلهی
&&
نیست به پیش همتم زو طربی و مفخری
\\
گر شکر است عسکری چون برسد به هر دهن
&&
زو نخورد شکرلبی فر ندهد به مخبری
\\
گر قمر است و گر فلک ور صنمی است بانمک
&&
کان همه است مشترک می‌نبود ورا فری
\\
آنچ بداد عامه را خلعت خاص نبود آن
&&
سور سگان کافران می‌نخورد غضنفری
\\
مجلس خاص بایدم گر چه بود سوی عدم
&&
شربت عام کم خورم گر چه بود ز کوثری
\\
لاف مسیح می‌زنی بول خران چه بو کنی
&&
با حدثی چه خو کنی همچو روان کافری
\\
گر نبدی متاع زر اصل وجود بول خر
&&
جان خران به بوی آن برنزدی چرا خوری
\\
مرد چو گوهری بود قیمت خویش خود کند
&&
شاد نشد به شحنگی هیچ قباد و سنجری
\\
زر تو بریز بر گهر چونک بماند زیر زر
&&
برنجهید بر زبر آن سبک است و ابتری
\\
ور بجهید بر زبر قیمت او است بیشتر
&&
بیش کنش نثار زر هست عزیز گوهری
\\
ما گهریم و این جهان همچو زری در امتحان
&&
بر سر زر برآ که لا گر تو نه‌ای محقری
\\
شهوت حلق بی‌نمک شهوت فرج پس دوک
&&
با سگ و خوک مشترک با خر و گاو همسری
\\
نیست سزای مهتری نیست هوای سروری
&&
همت شاه و سنجری قبله گه پیمبری
\\
عشق و نیاز و بندگی هست نشان زندگی
&&
در طلب تجلیی در نظری و منظری
\\
آب حیات جستنی جامه در آب شستنی
&&
بر در دل نشستنی تا بگشایدت دری
\\
در طرب و معاشقه در نظر و معانقه
&&
فرض بود مسابقه بر دل هر مظفری
\\
نیست روش طرنطران بنگر سوی آسمان
&&
در تک و پوی اختران هر یک چون مسخری
\\
روز خنوسشان ببین شام کنوسشان ببین
&&
سیر نفوسشان ببین گرد سرای مهتری
\\
غارب و شارقان حق طالب و عاشقان حق
&&
در تک و پوی و در سبق بی‌قدمی و بی‌پری
\\
گرم روی خور نگر شب روی قمر نگر
&&
ولوله سحر نگر راست چو روز محشری
\\
جان تقی فرشته‌ای جان شقی درشته‌ای
&&
نفس کریم کشتیی نفس لئیم لنگری
\\
رحم چو جوی شیر بین شهوت جوی انگبین
&&
عمر چو جوی آب دان شوق چو خمر احمری
\\
در تو نهان چهارجو هیچ نبینیش که کو
&&
همچو صفات و ذات هو هست نهان و ظاهری
\\
جوشش شوق از کجا جنبش ذوق از کجا
&&
لذت عمر در کمین رحم به زیر چادری
\\
خلق شده شکار او فرجه کنان کار او
&&
در پی اختیار او هر یک بسته زیوری
\\
شب به مثال هندوی روز مثال جادوی
&&
عدل مثال مشعله ظلم چو کور یا کری
\\
عقل حریف جنگیی نفس مثال زنگیی
&&
عشق چو مست و بنگیی صبر و حیا چو داوری
\\
شاه بگفته نکته ای خفیه به گوش هر کسی
&&
گفته به جان هر یکی غیر پیام دیگری
\\
جنگ میان بندگان کینه میان زندگان
&&
او فکند به هر زمان اینت ظریف یاوری
\\
گفت حدیث چرب و خوش با گل و داد خنده‌اش
&&
گفت به ابر نکته ای کرد دو چشم او تری
\\
گوید گل که بزم به گوید ابر گریه به
&&
هیچ یکی ز یک دگر پند نکرده باوری
\\
گفته به شاخ رقص کن گفته به برگ کف بزن
&&
گفته به چرخ چرخ زن گرد منازل ثری
\\
گفته به عقل طیره شو گفته به عشق خیره شو
&&
گفته به صبر خون گری در غم هجر دلبری
\\
گفته به رخ بخند خوش گفته به زلف پرده کش
&&
گفته به باد درربا پرده ز روی عبهری
\\
گفته به موج شور کن کف ز زلال دور کن
&&
گفته به دل عبور کن بر رخ هر مصوری
\\
هر طرفی علامتی هر نفسی قیامتی
&&
تا نکنی ملامتی گر شده‌ام سخنوری
\\
بر سر من نبشت حق در دل من چه کشت حق
&&
صبر مرا بکشت حق صبر نماند و صابری
\\
این همه آب و روغن است آنچ در این دل من است
&&
آه چه جای گفتن است آه ز عشق پروری
\\
لاح صبوح سره فاح نسیم بره
&&
جاء اوان دره برزه لمن یری
\\
انزله من العلی انشأه من الولا
&&
املاه من الملا فهمه لمن دری
\\
زینه لوصله الحقه باصله
&&
نوره بنوره ایقظه من الکری
\\
لیس لهم ندیده کلهم عبیده
&&
عز و جل و اغتنی لیس یرام بالشری
\\
اکرمنا ابرنا طیبنا و سرنا
&&
حدثنا به ما نجی اخبرنا بما جری
\\
طاب جوار ظله من علی مقله
&&
عز وجود مثله فی البلدان و القری
\\
از تبریز شمس دین یک سحری طلوع کرد
&&
ساخت شعاع نور او از دل بنده مظهری
\\
\end{longtable}
\end{center}
