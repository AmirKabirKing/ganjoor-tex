\begin{center}
\section*{غزل شماره ۲۴۸۲: ای که لب تو چون شکر هان که قرابه نشکنی}
\label{sec:2482}
\addcontentsline{toc}{section}{\nameref{sec:2482}}
\begin{longtable}{l p{0.5cm} r}
ای که لب تو چون شکر هان که قرابه نشکنی
&&
وی که دل تو چون حجر هان که قرابه نشکنی
\\
عشق درون سینه شد دل همه آبگینه شد
&&
نرم درآ تو ای پسر هان که قرابه نشکنی
\\
هر که اسیر سر بود دانک برون در بود
&&
خاصه که او بود دوسر هان که قرابه نشکنی
\\
آن صنم لطیف تو گر چه که شد حریف تو
&&
دست به زلف او مبر هان که قرابه نشکنی
\\
تا نکنی شناس او از دل خود قیاس او
&&
او دگر است و تو دگر هان که قرابه نشکنی
\\
چونک شوی تو مست او باده خوری ز دست او
&&
آن نفسی است باخطر هان که قرابه نشکنی
\\
مست درون سینه‌ها بر سر آبگینه‌ها
&&
نیک سبک تو برگذر هان که قرابه نشکنی
\\
حق چو نمود در بشر جمع شدند خیر و شر
&&
خیره مشو در این خبر هان که قرابه نشکنی
\\
یا تبریز شمس دین گر چه شدی تو همنشین
&&
تا تو نلافی از هنر هان که قرابه نشکنی
\\
\end{longtable}
\end{center}
