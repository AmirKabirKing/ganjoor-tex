\begin{center}
\section*{غزل شماره ۷۳۳: ذره ذره آفتاب عشق دردی خوار باد}
\label{sec:0733}
\addcontentsline{toc}{section}{\nameref{sec:0733}}
\begin{longtable}{l p{0.5cm} r}
ذره ذره آفتاب عشق دردی خوار باد
&&
مو به موی ما بدان سر جعفر طیار باد
\\
ذره‌ها بر آفتابت هر زمان بر می‌زنند
&&
هر که این بر خورد از تو از تو برخوردار باد
\\
هر کجا یک تار مویت بر هوس سر می‌نهد
&&
تار ما را پود باد و پود ما را تار باد
\\
در بیابان غم از دوری دارالملک وصل
&&
چند غم بردار بودستم که غم بر دار بود
\\
خار مسکینی که هر دم طعنه گل می‌کشد
&&
خواجه گلزار باد و از حسد گل زار باد
\\
گل پرستان چمن را دشمن مخفیست مار
&&
این چمن بی‌مار باد و دشمنش بیمار باد
\\
چونک غمخواری نباشد سخت دشوارست غم
&&
همنشین غمخوار باد و بعد از این غم خوار باد
\\
\end{longtable}
\end{center}
