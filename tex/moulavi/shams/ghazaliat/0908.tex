\begin{center}
\section*{غزل شماره ۹۰۸: چو درد گیرد دندان تو عدو گردد}
\label{sec:0908}
\addcontentsline{toc}{section}{\nameref{sec:0908}}
\begin{longtable}{l p{0.5cm} r}
چو درد گیرد دندان تو عدو گردد
&&
زبان تو به طبیبی بگرد او گردد
\\
یکی کدو ز کدوها اگر شکست آرد
&&
شکسته بند همه گرد آن کدو گردد
\\
ز صد سبو چو سبوی سبوگری برد آب
&&
همیشه خاطر او گرد آن سبو گردد
\\
شکستگان تویم ای حبیب و نیست عجب
&&
تو پادشاهی و لطف تو بنده جو گردد
\\
به قند لطف تو کاین لطف‌ها غلام ویند
&&
که زهر از او چو شکر خوب و خوب خو گردد
\\
اگر حلاوت لاحول تو به دیو رسد
&&
فرشته خو شود آن دیو و ماه رو گردد
\\
عنایتت گنهی را نظر کند به رضا
&&
چو طاعت آن گنه از دل گناه شو گردد
\\
پلید پاک شود مرده زنده مار عصا
&&
چو خون که در تن آهوست مشک بو گردد
\\
رونده‌ای که سوی بی‌سوییش ره دادی
&&
کجا چو خاطر گمراه سو به سو گردد
\\
تو جان جان جهانی و نام تو عشق است
&&
هر آنک از تو پری یافت بر علو گردد
\\
خمش که هر کی دهانش ز عشق شیرین شد
&&
روا نباشد کو گرد گفت و گو گردد
\\
خموش باش که آن کس که بحر جانان دید
&&
نشاید و نتواند که گرد جو گردد
\\
\end{longtable}
\end{center}
