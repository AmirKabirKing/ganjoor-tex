\begin{center}
\section*{غزل شماره ۱۱۶۷: مست توام نه از می و نه از کوکنار}
\label{sec:1167}
\addcontentsline{toc}{section}{\nameref{sec:1167}}
\begin{longtable}{l p{0.5cm} r}
مست توام نه از می و نه از کوکنار
&&
وقت کنارست بیا گو کنار
\\
برجه مستانه کناری بگیر
&&
چون شجر و باد به وقت بهار
\\
شاخ تر از باد کناری چو یافت
&&
رقص درآمد چو من بی‌قرار
\\
این خبر افتاد به خوبان غیب
&&
تا برسیدند هزاران نگار
\\
لاله رخ افروخته از که رسید
&&
سنبله پا به گل از مرغزار
\\
سوسن با تیغ و سمن با سپر
&&
سبزه پیادست و گل تر سوار
\\
فندق و خشخاش به دست آمده
&&
نعنع و حلبو به لب جویبار
\\
جدول هر گونه حویجی جدا
&&
تا مددی یابد از یار یار
\\
کرده دکان‌ها همه حلواییان
&&
پرشکر و فستق از بهر کار
\\
میوه فروشان همه با طبل‌ها
&&
بر سر هر پشته فشانده ثمار
\\
لیک ز گل گوی که همرنگ اوست
&&
جمله ز بو گو که پریست یار
\\
بلبل و قمری و دو صد نوع مرغ
&&
جانب باغ آمده قادم یزار
\\
می‌زندم نرگس چشمک خموش
&&
خطبه مرغان چمن گوش دار
\\
\end{longtable}
\end{center}
