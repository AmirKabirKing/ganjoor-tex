\begin{center}
\section*{غزل شماره ۲۹۶۴: دی دامنش گرفتم کای گوهر عطایی}
\label{sec:2964}
\addcontentsline{toc}{section}{\nameref{sec:2964}}
\begin{longtable}{l p{0.5cm} r}
دی دامنش گرفتم کای گوهر عطایی
&&
شب خوش مگو مرنجان کامشب از آن مایی
\\
افروخت روی دلکش شد سرخ همچو اخگر
&&
گفتا بس است درکش تا چند از این گدایی
\\
گفتم رسول حق گفت حاجت ز روی نیکو
&&
درخواه اگر بخواهی تا تو مظفر آیی
\\
گفتا که روی نیکو خودکامه است و بدخو
&&
زیرا که ناز و جورش دارد بسی روایی
\\
گفتم اگر چنان است جورش حیات جان است
&&
زیرا طلسم کان است هر گه بیازمایی
\\
گفت این حدیث خام است روی نکو کدام است
&&
این رنگ و نقش دام است مکر است و بی‌وفایی
\\
چون جان جان ندارد می‌دانک آن ندارد
&&
بس کس که جان سپارد در صورت فنایی
\\
گفتم که خوش عذارا تو هست کن فنا را
&&
زر ساز مس ما را تو جان کیمیایی
\\
تسلیم مس بباید تا کیمیا بیابد
&&
تو گندمی ولیکن بیرون آسیایی
\\
گفتا تو ناسپاسی تو مس ناشناسی
&&
در شک و در قیاسی زین‌ها که می‌نمایی
\\
گریان شدم به زاری گفتم که حکم داری
&&
فریاد رس به یاری ای اصل روشنایی
\\
چون دید اشک بنده آغاز کرد خنده
&&
شد شرق و غرب زنده زان لطف آشنایی
\\
ای همرهان و یاران گریید همچو باران
&&
تا در چمن نگاران آرند خوش لقایی
\\
\end{longtable}
\end{center}
