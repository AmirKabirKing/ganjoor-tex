\begin{center}
\section*{غزل شماره ۹۵۱: هر آن نوی که رسد سوی تو قدید شود}
\label{sec:0951}
\addcontentsline{toc}{section}{\nameref{sec:0951}}
\begin{longtable}{l p{0.5cm} r}
هر آن نوی که رسد سوی تو قدید شود
&&
چو آب پاک که در تن رود پلید شود
\\
ز شیر دیو مزیدی مزید تو هم از اوست
&&
که بایزید از این شیردان یزید شود
\\
مرید خواند خداوند دیو وسوسه را
&&
که هر که خورد دم او چو او مرید شود
\\
چو مشرقست و چو مغرب مثال این دو جهان
&&
بدین قریب شود مرد زان بعید شود
\\
هر آن دلی که بشورید و قی شدش آن شیر
&&
ز شورش و قی آن شیر بوسعید شود
\\
هر آنک صدر رها کرد و خاک این در شد
&&
هزار قفل گران را دلش کلید شود
\\
ترش ترش تو به خسرو مگو که شیرین کو
&&
پدید آید چون خواجه ناپدید شود
\\
چو غوره رست ز خامی خویش شد شیرین
&&
چو ماه روزه به پایان رسید عید شود
\\
خموش آینه منمای در ولایت زنگ
&&
نما به قیصر رومش که تا مرید شود
\\
\end{longtable}
\end{center}
