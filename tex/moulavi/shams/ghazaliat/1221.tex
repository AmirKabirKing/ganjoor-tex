\begin{center}
\section*{غزل شماره ۱۲۲۱: اگر گم گردد این بی‌دل از آن دلدار جوییدش}
\label{sec:1221}
\addcontentsline{toc}{section}{\nameref{sec:1221}}
\begin{longtable}{l p{0.5cm} r}
اگر گم گردد این بی‌دل از آن دلدار جوییدش
&&
وگر اندررمد عاشق به کوی یار جوییدش
\\
وگر این بلبل جانم بپرد ناگهان از تن
&&
زهر خاری مپرسیدش در آن گلزار جوییدش
\\
اگر بیمار عشق او شود یاوه از این مجلس
&&
به پیش نرگس بیمار آن عیار جوییدش
\\
وگر سرمست دل روزی زند بر سنگ آن شیشه
&&
به میخانه روید آن دم از آن خمار جوییدش
\\
هر آن عاشق که گم گردد هلا زنهار می‌گویم
&&
بر خورشید برق انداز بی‌زنهار جوییدش
\\
وگر دزدی زند نقبی بدزدد رخت عاشق را
&&
میان طره مشکین آن طرار جوییدش
\\
بت بیدار پرفن را که بیداری ز بخت اوست
&&
چنین خفته نیابیدش مگر بیدار جوییدش
\\
بپرسیدم به کوی دل ز پیری من از آن دلبر
&&
اشارت کرد آن پیرم که در اسرار جوییدش
\\
بگفتم پیر را بالله تویی اسرار گفت آری
&&
منم دریای پرگوهر به دریابار جوییدش
\\
زهی گوهر که دریا را به نور خویش پر دارد
&&
مسلمانان مسلمانان در آن انوار جوییدش
\\
چو یوسف شمس تبریزی به بازار صفا آمد
&&
مر اخوان صفا را گو در آن بازار جوییدش
\\
\end{longtable}
\end{center}
