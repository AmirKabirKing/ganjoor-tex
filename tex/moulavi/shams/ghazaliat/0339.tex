\begin{center}
\section*{غزل شماره ۳۳۹: سماع آرام جان زندگانیست}
\label{sec:0339}
\addcontentsline{toc}{section}{\nameref{sec:0339}}
\begin{longtable}{l p{0.5cm} r}
سماع آرام جان زندگانیست
&&
کسی داند که او را جان جانست
\\
کسی خواهد که او بیدار گردد
&&
که او خفته میان بوستان‌ست
\\
ولیک آن کو به زندان خفته باشد
&&
اگر بیدار گردد در زیان‌ست
\\
سماع آن جا بکن کان جا عروسیست
&&
نه در ماتم که آن جای فغانست
\\
کسی کو جوهر خود را ندیدهست
&&
کسی کان ماه از چشمش نهانست
\\
چنین کس را سماع و دف چه باید
&&
سماع از بهر وصل دلستان‌ست
\\
کسانی را که روشان سوی قبله‌ست
&&
سماع این جهان و آن جهانست
\\
خصوصا حلقه‌ای کاندر سماعند
&&
همی‌گردند و کعبه در میانست
\\
اگر کان شکر خواهی همان جاست
&&
ور انگشت شکر خود رایگانست
\\
\end{longtable}
\end{center}
