\begin{center}
\section*{غزل شماره ۳۱۴: تو را که عشق نداری تو را رواست بخسب}
\label{sec:0314}
\addcontentsline{toc}{section}{\nameref{sec:0314}}
\begin{longtable}{l p{0.5cm} r}
تو را که عشق نداری تو را رواست بخسب
&&
برو که عشق و غم او نصیب ماست بخسب
\\
ز آفتاب غم یار ذره ذره شدیم
&&
تو را که این هوس اندر جگر نخاست بخسب
\\
به جست و جوی وصالش چو آب می‌پویم
&&
تو را که غصه آن نیست کو کجاست بخسب
\\
طریق عشق ز هفتاد و دو برون باشد
&&
چو عشق و مذهب تو خدعه و ریاست بخسب
\\
صباح ماست صبوحش عشای ما عشوه ش
&&
تو را که رغبت لوت و غم عشاست بخسب
\\
ز کیمیاطلبی ما چو مس گدازانیم
&&
تو را که بستر و همخوابه کیمیاست بخسب
\\
چو مست هر طرفی می‌فتی و می‌خیزی
&&
که شب گذشت کنون نوبت دعاست بخسب
\\
قضا چو خواب مرا بست ای جوان تو برو
&&
که خواب فوت شدت خواب را قضاست بخسب
\\
به دست عشق درافتاده‌ایم تا چه کند
&&
چو تو به دست خودی رو به دست راست بخسب
\\
منم که خون خورم ای جان تویی که لوت خوری
&&
چو لوت را به یقین خواب اقتضاست بخسب
\\
من از دماغ بریدم امید و از سر نیز
&&
تو را دماغ تر و تازه مرتجاست بخسب
\\
لباس حرف دریدم سخن رها کردم
&&
تو که برهنه نه‌ای مر تو را قباست بخسب
\\
\end{longtable}
\end{center}
