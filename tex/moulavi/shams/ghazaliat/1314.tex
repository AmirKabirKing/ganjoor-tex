\begin{center}
\section*{غزل شماره ۱۳۱۴: به دلجویی و دلداری درآمد یار پنهانک}
\label{sec:1314}
\addcontentsline{toc}{section}{\nameref{sec:1314}}
\begin{longtable}{l p{0.5cm} r}
به دلجویی و دلداری درآمد یار پنهانک
&&
شب آمد چون مه تابان شه خون خوار پنهانک
\\
دهان بر می‌نهاد او دست یعنی دم مزن خامش
&&
و می‌فرمود چشم او درآ در کار پنهانک
\\
چو کرد آن لطف او مستم در گلزار بشکستم
&&
همی‌دزدیدم آن گل‌ها از آن گلزار پنهانک
\\
بدو گفتم که ای دلبر چه مکرانگیز و عیاری
&&
برانگیزان یکی مکری خوش ای عیار پنهانک
\\
بنه بر گوش من آن لب اگر چه خلوتست و شب
&&
مهل تا برزند بادی بر آن اسرار پنهانک
\\
از آن اسرار عاشق کش مشو امشب مها خامش
&&
نوای چنگ عشرت را بجنبان تار پنهانک
\\
بده ای دلبر خندان به رسم صدقه پنهان
&&
از آن دو لعل جان افزای شکربار پنهانک
\\
که غمازان همه مستند اندر خواب گفت آری
&&
ولیکن هست از این مستان یکی هشیار پنهانک
\\
مکن ای شمس تبریزی چنین تندی چنین تیزی
&&
کجا یابم تو را ای شاه دیگربار پنهانک
\\
\end{longtable}
\end{center}
