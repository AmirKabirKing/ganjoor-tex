\begin{center}
\section*{غزل شماره ۶۵۱: مهتاب برآمد کلک از گور برآمد}
\label{sec:0651}
\addcontentsline{toc}{section}{\nameref{sec:0651}}
\begin{longtable}{l p{0.5cm} r}
مهتاب برآمد کلک از گور برآمد
&&
وز ریگ سیه چرده سقنقور برآمد
\\
آنک از قلمش موسی و عیسیست مصور
&&
از نفخه او دمدمه صور برآمد
\\
در هاون اقبال عنایت گهری کوفت
&&
صد دیده حق بین ز دل کور برآمد
\\
از تف بهاری چه خبر یافت دل خاک
&&
کز خاک سیه قافله مور برآمد
\\
از بحر عسل‌هاش چه دید آن دل زنبور
&&
با مشک عسل گله زنبور برآمد
\\
در مخزن او کرم ضعیفی به چه ره یافت
&&
کز وی خز و ابریشم موفور برآمد
\\
بی دیده و بی‌گوش صدف رزق کجا یافت
&&
تا حاصل در گشت و چو گنجور برآمد
\\
نرم آهن و سنگی سوی انوار چه ره یافت
&&
کز آهن و سنگی علم نور برآمد
\\
بنگر که ز گلزار چه گلزار بخندید
&&
وز سرمه چون قیر چه کافور برآمد
\\
بی غازه و گلگونه گل آن رنگ کجا یافت
&&
کافروخته از پرده مستور برآمد
\\
در دولت و در عزت آن شاه نکوکار
&&
این لشگر بشکسته چه منصور برآمد
\\
یک سیب بنی دیدم در باغ جمالش
&&
هر سیب که بشکافت از او حور برآمد
\\
چون حور برآمد ز دل سیب بخندید
&&
از خنده او حاجت رنجور برآمد
\\
این هستی و این مستی و این جنبش مستان
&&
زان باده مدان کز دل انگور برآمد
\\
شمس الحق تبریز چو این شور برانگیخت
&&
از مشرق جان آن مه مشهور برآمد
\\
\end{longtable}
\end{center}
