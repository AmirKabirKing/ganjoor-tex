\begin{center}
\section*{غزل شماره ۸۶۵: جانا بیار باده که ایام می‌رود}
\label{sec:0865}
\addcontentsline{toc}{section}{\nameref{sec:0865}}
\begin{longtable}{l p{0.5cm} r}
جانا بیار باده که ایام می‌رود
&&
تلخی غم به لذت آن جام می‌رود
\\
جامی که عقل و روح حریف و جلیس اوست
&&
نی نفس کوردل که سوی دام می‌رود
\\
با جام آتشین چو تو از در درآمدی
&&
وسواس و غم چو دود سوی بام می‌رود
\\
گر بر سرت گلست مشویش شتاب کن
&&
بر آب و گل بساز که هنگام می‌رود
\\
آن چیز را بجوش که او هوش می‌برد
&&
وان خام را بپز که سخن خام می‌رود
\\
زان باده داده‌ای تو به خورشید و ماه و چرخ
&&
هر یک بدان نشاط چنین رام می‌رود
\\
والله که ذره نیز از آن جام بیخودست
&&
از کرم مست گشته به اکرام می‌رود
\\
آرام بخش جان را زان می که از تفش
&&
صبر و قرار و توبه و آرام می‌رود
\\
چون بوی وی رسد به خماران بود چنانک
&&
آن مادر رحیم بر ایتام می‌رود
\\
امروز خاک جرعه می سیر سیر خورد
&&
خورشیدوار جام کرم عام می‌رود
\\
سوی کشنده آید کشته چنانک زود
&&
خون از بدن به شیشه حجام می‌رود
\\
چون کعبه که رود به در خانه ولی
&&
این رحمت خدای به ارحام می‌رود
\\
تا مست نیست از همه لنگان سپس ترست
&&
در بیخودی به کعبه به یک گام می‌رود
\\
تا باخودست راز نهان دارد از ادب
&&
چون مست شد چه چاره که خودکام می‌رود
\\
خاموش و نام باده مگو پیش مرد خام
&&
چون خاطرش به باده بدنام می‌رود
\\
\end{longtable}
\end{center}
