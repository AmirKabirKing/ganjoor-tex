\begin{center}
\section*{غزل شماره ۱۶۱۶: ز یکی پسته دهانی صنمی بسته دهانم}
\label{sec:1616}
\addcontentsline{toc}{section}{\nameref{sec:1616}}
\begin{longtable}{l p{0.5cm} r}
ز یکی پسته دهانی صنمی بسته دهانم
&&
چو برویید نباتش چو شکر بست زبانم
\\
همه خوبی قمر او همه شادی است مگر او
&&
که از او من تن خود را ز شکر بازندانم
\\
تو چه پرسی که کدامی تو در این عشق چه نامی
&&
صنما شاه جهانی ز تو من شاد جهانم
\\
چو قدح ریخته گشتم به تو آمیخته گشتم
&&
چو بدیدم که تو جانی مثل جان پنهانم
\\
وگرم هست اگر من بنه انگشت تو بر من
&&
که من اندر طلب خود سر انگشت گزانم
\\
چو از او در تک و تابم ز پیش سخت شتابم
&&
چو مرا برد به نارم دو چو خود بازستانم
\\
چو شکرگیر تو گشتم چو من از تیر تو گشتم
&&
چه شد ار بهر شکارت شکند تیر و کمانم
\\
چو صلاح دل و دین را مه خورشید یقین را
&&
به تو افتاد محبت تو شدی جان و روانم
\\
\end{longtable}
\end{center}
