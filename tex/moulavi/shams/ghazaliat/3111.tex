\begin{center}
\section*{غزل شماره ۳۱۱۱: تو چنین نبودی تو چنین چرایی}
\label{sec:3111}
\addcontentsline{toc}{section}{\nameref{sec:3111}}
\begin{longtable}{l p{0.5cm} r}
تو چنین نبودی تو چنین چرایی
&&
چه کنی خصومت چو از آن مایی
\\
دل و جان غلامت چو رسد سلامت
&&
تو دو صد چنین را صنما سزایی
\\
تو قمرعذاری تو دل بهاری
&&
تو ملک نژادی تو ملک لقایی
\\
فلک از تو حارس زحل از تو فارس
&&
ز برای آن را که در این سرایی
\\
دل خسته گشته چو قدح شکسته
&&
تو چو گم شدستی تو چه ره نمایی
\\
بده آن قدح را بگشا فرح را
&&
که غم کهن را تو بهین دوایی
\\
دل و جان کی باشد دو جهان چه باشد
&&
همه سهل باشد تو عجب کجایی
\\
بگذار دستان برسان به مستان
&&
ز عطای سلطان قدح عطایی
\\
همگی امیدی شکری سپیدی
&&
چو مرا بدیدی بکن آشنایی
\\
شکری نباتی همگی حیاتی
&&
طبق زکاتی کرم خدایی
\\
طرب جهانی عجب قرانی
&&
تو سماع جان را تر لایلایی
\\
بزنی ز بالاتر لایلالا
&&
تو نه یک بلایی تو دو صد بلایی
\\
دل من ببردی به کجا سپردی
&&
نه جواب گویی نه دهی رهایی
\\
بفزا دغا را بفریب ما را
&&
بر توست عالم همه روستایی
\\
سر ما شکستی سر خود ببستی
&&
که خرف نگردد ز چنین دغایی
\\
به پلاس عوران به عصای کوران
&&
چه طمع ببستی ز چه می‌ربایی
\\
به طمع چنانی به عطا جهانی
&&
عجب از تو خیره به عجب نمایی
\\
خمش ای صفورا بگذار او را
&&
تو ز خویشتن گو که چه کیمیایی
\\
نه به اختیاری همه اضطراری
&&
تو به خود نگردی تو چو آسیایی
\\
تو یکی سبویی چو اسیر جویی
&&
جز جو چه جویی چو ز جو برآیی
\\
تو به خود چه سازی که اسیر گازی
&&
تو ز خود چه گویی چو ز که صدایی
\\
خمش ای ترانه بجه از کرانه
&&
که نوای جانی همگی نوایی
\\
\end{longtable}
\end{center}
