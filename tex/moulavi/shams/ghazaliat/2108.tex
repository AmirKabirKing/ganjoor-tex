\begin{center}
\section*{غزل شماره ۲۱۰۸: می‌نروم هیچ از این خانه من}
\label{sec:2108}
\addcontentsline{toc}{section}{\nameref{sec:2108}}
\begin{longtable}{l p{0.5cm} r}
می‌نروم هیچ از این خانه من
&&
در تک این خانه گرفتم وطن
\\
خانه یار من و دارالقرار
&&
کفر بود نیت بیرون شدن
\\
سر نهم آن جا که سرم مست شد
&&
گوش نهم سوی تنن تنتنن
\\
نکته مگو هیچ به راهم مکن
&&
راه من این است تو راهم مزن
\\
خانه لیلی است و مجنون منم
&&
جان من این جاست برو جان مکن
\\
هر کی در این خانه درآید ورا
&&
همچو منش باز بماند دهن
\\
خیز ببند آن در اما چه سود
&&
قارع در گشت دو صد درشکن
\\
ای خنک آن را که سرش گرم شد
&&
ز آتش روی چو تو شیرین ذقن
\\
آن رخ چون ماه به برقع مپوش
&&
ای رخ تو حسرت هر مرد و زن
\\
این در رحمت که گشادی مبند
&&
ای در تو قبله هر ممتحن
\\
شمع تویی شاهد تو باده تو
&&
هم تو سهیلی و عقیق یمن
\\
باقی عمر از تو نخواهم برید
&&
حلقه به گوش توام و مرتهن
\\
می‌نرمد شیر من از آتشت
&&
می‌نرمد پیل من از کرگدن
\\
تو گل و من خار که پیوسته‌ایم
&&
بی‌گل و بی‌خار نباشد چمن
\\
من شب و تو ماه به تو روشنم
&&
جان شبی دل ز شبم برمکن
\\
شمع تو پروانه جانم بسوخت
&&
سر پی شکرانه نهم بر لگن
\\
جان من و جان تو هر دو یکی است
&&
گشته یکی جان پنهان در دو تن
\\
جان من و تو چو یکی آفتاب
&&
روشن از او گشته هزار انجمن
\\
وقت حضور تو دو تا گشت جان
&&
رسته شد از تفرقه خویشتن
\\
تن زدم از غیرت و خامش شدم
&&
مطرب عشاق بگو تن مزن
\\
خطه تبریز و رخ شمس دین
&&
ماهی جان راست چو بحر عدن
\\
\end{longtable}
\end{center}
