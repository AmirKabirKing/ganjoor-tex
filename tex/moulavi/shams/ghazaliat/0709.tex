\begin{center}
\section*{غزل شماره ۷۰۹: جان از سفر دراز آمد}
\label{sec:0709}
\addcontentsline{toc}{section}{\nameref{sec:0709}}
\begin{longtable}{l p{0.5cm} r}
جان از سفر دراز آمد
&&
بر خاک در تو بازآمد
\\
در نقد وجود هر چه زر بود
&&
از گنج عدم به گاز آمد
\\
بی مهر تو هر که آسمان رفت
&&
درهای فلک فرازآمد
\\
بی آبی خویش جمله دیدند
&&
هرک از تو نه سرفراز آمد
\\
جان رفت که بی‌تو کار سازد
&&
سوزید و نه کارساز آمد
\\
اندر سفرش بشد حقیقت
&&
کو بی‌تو همه مجاز آمد
\\
از گرد ره آمدست امروز
&&
رحم آر که پرنیاز آمد
\\
سر را ز دریچه‌ای برون کن
&&
تا بیند کان طراز آمد
\\
تا نعره عاشقان برآید
&&
کان قبله هر نماز آمد
\\
از پیش تو رفت باز جانم
&&
طبل تو شنید و بازآمد
\\
ای اهل رباط وارهیدیت
&&
کز خط خوشش جواز آمد
\\
آن چنگ طرب که بی‌نوا بود
&&
رقصی که کنون به ساز آمد
\\
از سلسله نیاز رستید
&&
کان بند هزار ناز آمد
\\
ترک خر کالبد بگویید
&&
کان شاه براق تاز آمد
\\
نور رخ شمس حق تبریز
&&
عالم بگرفت و راز آمد
\\
\end{longtable}
\end{center}
