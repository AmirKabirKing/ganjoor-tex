\begin{center}
\section*{غزل شماره ۱۹۰۴: کجا خواهی ز چنگ ما پریدن}
\label{sec:1904}
\addcontentsline{toc}{section}{\nameref{sec:1904}}
\begin{longtable}{l p{0.5cm} r}
کجا خواهی ز چنگ ما پریدن
&&
کی داند دام قدرت را دریدن
\\
چو پایت نیست تا از ما گریزی
&&
بنه گردن رها کن سر کشیدن
\\
دوان شو سوی شیرینی چو غوره
&&
به باطن گر نمی‌دانی دویدن
\\
رسن را می گزی ای صید بسته
&&
نبرد این رسن هیچ از گزیدن
\\
نمی‌بینی سرت اندر زه ماست
&&
کمانی بایدت از زه خمیدن
\\
چه جفته می زنی کز بار رستم
&&
یکی دم هشتمت بهر چریدن
\\
دل دریا ز بیم و هیبت ما
&&
همی‌جوشد ز موج و از طپیدن
\\
که سنگین اگر آن زخم یابد
&&
ز بند ما نیارد برجهیدن
\\
فلک را تا نگوید امر ما بس
&&
به گرد خاک ما باید تنیدن
\\
هوا شیری است از پستان شیطان
&&
بود عقل تو شیر خر مکیدن
\\
دهان خاک خشک از حسرت ماست
&&
نیارد جرعه‌ای بی‌ما چشیدن
\\
کی یارد صید ما را قصد کردن
&&
کی یارد بنده ما را خریدن
\\
کسی را که ربودیم و گزیدیم
&&
که را خواهد به غیر ما گزیدن
\\
امانی نیست جان را در جز عشق
&&
میان عاشقان باید خزیدن
\\
امان هر دو عالم عاشقان راست
&&
چنین بودند وقت آفریدن
\\
نشاید بره را از جور چوپان
&&
ز چوپان جانب گرگان رمیدن
\\
که این چوپان نریزد خون بره
&&
که او جاوید داند پروریدن
\\
بدان کاصحاب تن اصحاب فیلند
&&
به کعبه کی تواند بررسیدن
\\
که کعبه ناف عالم پیل بینی است
&&
نتان بینی بر نافی کشیدن
\\
ابابیلی شو و از پیل مگریز
&&
ابابیل است دل در دانه چیدن
\\
بچینند دشمنان را همچو دانه
&&
پیام کعبه را داند شنیدن
\\
ز دل خواهی شدن بر آسمان‌ها
&&
ز دل خواهد گل دولت دمیدن
\\
ز دل خواهی به دلبر راه بردن
&&
ز دل خواهی ز ننگ تن رهیدن
\\
دل از بهر تو یک دیکی بپخته‌ست
&&
زمانی صبر می کن تا پزیدن
\\
دل دل‌هاست شمس الدین تبریز
&&
نتاند شمس را خفاش دیدن
\\
\end{longtable}
\end{center}
