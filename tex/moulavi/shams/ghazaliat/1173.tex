\begin{center}
\section*{غزل شماره ۱۱۷۳: بشنو خبر صادق از گفته پیغامبر}
\label{sec:1173}
\addcontentsline{toc}{section}{\nameref{sec:1173}}
\begin{longtable}{l p{0.5cm} r}
بشنو خبر صادق از گفته پیغامبر
&&
اندر صفت مؤمن المؤمن کالمزهر
\\
جاء الملک الاکبر ما احسن ذا المنظر
&&
حتی ملاء الدنیا بالعبهر و العنبر
\\
چون بربط شد مؤمن در ناله و در زاری
&&
بربط ز کجا نالد بی‌زخمه زخم آور
\\
جاء الفرج الاعظم جاء الفرج الاکبر
&&
جاء الکرم الادوم جاء القمر الاقمر
\\
خو کرد دل بربط نشکیبد از آن زخمه
&&
اندر قدم مطرب می‌مالد رو و سر
\\
الدوله عیشیه و القهوه عرشیه
&&
و المجلس منثور باللوز مع السکر
\\
اینک غزلی دیگر الخمس مع الخمسین
&&
زان پیش که برخوانم که شانیک الابتر
\\
الرب هو الساقی و العیش به باقی
&&
و السعد هو الراقی یا خایف لا تحذر
\\
الروح غداً سکری من قهوتنا الکبری
&&
و ازینت الدنیا بالاخضر و الاحمر
\\
خاموش شو و محرم می‌خور می جان هر دم
&&
در مجلس ربانی بی‌حلق و لب و ساغر
\\
\end{longtable}
\end{center}
