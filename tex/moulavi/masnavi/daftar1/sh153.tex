\begin{center}
\section*{بخش ۱۵۳ - باقی قصهٔ هاروت و ماروت و نکال و عقوبت ایشان هم در دنیا بچاه بابل}
\label{sec:sh153}
\addcontentsline{toc}{section}{\nameref{sec:sh153}}
\begin{longtable}{l p{0.5cm} r}
چون گناه و فسق خلقان جهان
&&
می‌شدی بر هر دو روشن آن زمان
\\
دست خاییدن گرفتندی ز خشم
&&
لیک عیب خود ندیدندی به چشم
\\
خویش در آیینه دید آن زشت مرد
&&
رو بگردانید از آن و خشم کرد
\\
خویش‌بین چون از کسی جرمی بدید
&&
آتشی در وی ز دوزخ شد پدید
\\
حمیت دین خواند او آن کبر را
&&
ننگرد در خویش نفس گبر را
\\
حمیت دین را نشانی دیگرست
&&
که از آن آتش جهانی اخضرست
\\
گفت حقشان گر شما روشن گرید
&&
در سیه‌کاران مغفل منگرید
\\
شکر گویید ای سپاه و چاکران
&&
رسته‌اید از شهوت و از چاک‌ران
\\
گر از آن معنی نهم من بر شما
&&
مر شما را بیش نپذیرد سما
\\
عصمتی که مر شما را در تنست
&&
آن ز عکس عصمت و حفظ منست
\\
آن ز من بینید نه از خود هین و هین
&&
تا نچربد بر شما دیو لعین
\\
آنچنان که کاتب وحی رسول
&&
دید حکمت در خود و نور اصول
\\
خویش را هم صوت مرغان خدا
&&
می‌شمرد آن بد صفیری چون صدا
\\
لحن مرغان را اگر واصف شوی
&&
بر مراد مرغ کی واقف شوی
\\
گر بیاموزی صفیر بلبلی
&&
تو چه دانی کو چه دارد با گلی
\\
ور بدانی باشد آن هم از گمان
&&
چون ز لب‌جنبان گمانهای کران
\\
\end{longtable}
\end{center}
