\begin{center}
\section*{بخش ۱۱۲ - مغرور شدن مریدان محتاج به مدعیان مزور و ایشان را شیخ و محتشم و واصل پنداشتن و نقل را از نقد فرق نادانستن و بر بسته را از بر رسته}
\label{sec:sh112}
\addcontentsline{toc}{section}{\nameref{sec:sh112}}
\begin{longtable}{l p{0.5cm} r}
بهر این گفتند دانایان بفن
&&
میهمان محسنان باید شدن
\\
تو مرید و میهمان آن کسی
&&
کو ستاند حاصلت را از خسی
\\
نیست چیره چون ترا چیره کند
&&
نور ندهد مر ترا تیره کند
\\
چون ورا نوری نبود اندر قران
&&
نور کی یابند از وی دیگران
\\
همچو اعمش کو کند داروی چشم
&&
چه کشد در چشمها الا که یشم
\\
حال ما اینست در فقر و عنا
&&
هیچ مهمانی مبا مغرور ما
\\
قحط ده سال ار ندیدی در صور
&&
چشمها بگشا و اندر ما نگر
\\
ظاهر ما چون درون مدعی
&&
در دلش ظلمت زبانش شعشعی
\\
از خدا بویی نه او را نه اثر
&&
دعویش افزون ز شیث و بوالبشر
\\
دیو ننموده ورا هم نقش خویش
&&
او همی‌گوید ز ابدالیم بیش
\\
حرف درویشان بدزدیده بسی
&&
تا گمان آید که هست او خود کسی
\\
خرده گیرد در سخن بر بایزید
&&
ننگ دارد از درون او یزید
\\
بی‌نوا از نان و خوان آسمان
&&
پیش او ننداخت حق یک استخوان
\\
او ندا کرده که خوان بنهاده‌ام
&&
نایب حقم خلیفه‌زاده‌ام
\\
الصلا ساده‌دلان پیچ پیچ
&&
تا خورید از خوان جودم سیر هیچ
\\
سالها بر وعدهٔ فردا کسان
&&
گرد آن در گشته فردا نارسان
\\
دیر باید تا که سر آدمی
&&
آشکارا گردد از بیش و کمی
\\
زیر دیوار بدن گنجست یا
&&
خانهٔ مارست و مور و اژدها
\\
چونک پیدا گشت کو چیزی نبود
&&
عمر طالب رفت آگاهی چه سود
\\
\end{longtable}
\end{center}
