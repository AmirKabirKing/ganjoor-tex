\begin{center}
\section*{بخش ۱۶۲ - رجوع به حکایت زید}
\label{sec:sh162}
\addcontentsline{toc}{section}{\nameref{sec:sh162}}
\begin{longtable}{l p{0.5cm} r}
زید را اکنون نیابی کو گریخت
&&
جست از صف نعال و نعل ریخت
\\
تو که باشی زید هم خود را نیافت
&&
همچو اختر که برو خورشید تافت
\\
نه ازو نقشی بیابی نه نشان
&&
نه کهی یابی به راه کهکشان
\\
شد حواس و نطق بابایان ما
&&
محو نور دانش سلطان ما
\\
حسها و عقلهاشان در درون
&&
موج در موج لدینا محضرون
\\
چون بیاید صبح وقت بار شد
&&
انجم پنهان شده بر کار شد
\\
بیهشان را وا دهد حق هوشها
&&
حلقه حلقه حلقه‌ها در گوشها
\\
پای‌کوبان دست‌افشان در ثنا
&&
ناز نازان ربنا احییتنا
\\
آن جلود و آن عظام ریخته
&&
فارسان گشته غبار انگیخته
\\
حمله آرند از عدم سوی وجود
&&
در قیامت هم شکور و هم کنود
\\
سر چه می‌پیچی کنی نادیده‌ای
&&
در عدم ز اول نه سر پیچیده‌ای
\\
در عدم افشرده بودی پای خویش
&&
که مرا کی بر کند از جای خویش
\\
می‌نبینی صنع ربانیت را
&&
که کشید او موی پیشانیت را
\\
تا کشیدت اندرین انواع حال
&&
که نبودت در گمان و در خیال
\\
آن عدم او را هماره بنده است
&&
کار کن دیوا سلیمان زنده است
\\
دیو می‌سازد جفان کالجواب
&&
زهره نه تا دفع گوید یا جواب
\\
خویش را بین چون همی‌لرزی ز بیم
&&
مر عدم را نیز لرزان دان مقیم
\\
ور تو دست اندر مناصب می‌زنی
&&
هم ز ترس است آن که جانی می‌کنی
\\
هرچه جز عشق خدای احسنست
&&
گر شکرخواریست آن جان کندنست
\\
چیست جان کندن سوی مرگ آمدن
&&
دست در آب حیاتی نازدن
\\
خلق را دو دیده در خاک و ممات
&&
صد گمان دارند در آب حیات
\\
جهد کن تا صد گمان گردد نود
&&
شب برو ور تو بخسپی شب رود
\\
در شب تاریک جوی آن روز را
&&
پیش کن آن عقل ظلمت‌سوز را
\\
در شب بدرنگ بس نیکی بود
&&
آب حیوان جفت تاریکی بود
\\
سر ز خفتن کی توان برداشتن
&&
با چنین صد تخم غفلت کاشتن
\\
خواب مرده لقمه مرده یار شد
&&
خواجه خفت و دزد شب بر کار شد
\\
تو نمی‌دانی که خصمانت کیند
&&
ناریان خصم وجود خاکیند
\\
نار خصم آب و فرزندان اوست
&&
همچنانک آب خصم جان اوست
\\
آب آتش را کشد زیرا که او
&&
خصم فرزندان آبست و عدو
\\
بعد از آن این نار نار شهوتست
&&
کاندرو اصل گناه و زلتست
\\
نار بیرونی به آبی بفسرد
&&
نار شهوت تا به دوزخ می‌برد
\\
نار شهوت می‌نیارامد بب
&&
زانک دارد طبع دوزخ در عذاب
\\
نار شهوت را چه چاره نور دین
&&
نورکم اطفاء نار الکافرین
\\
چه کشد این نار را نور خدا
&&
نور ابراهیم را ساز اوستا
\\
تا ز نار نفس چون نمرود تو
&&
وا رهد این جسم همچون عود تو
\\
شهوت ناری براندن کم نشد
&&
او بماندن کم شود بی هیچ بد
\\
تا که هیزم می‌نهی بر آتشی
&&
کی بمیرد آتش از هیزم‌کشی
\\
چونک هیزم باز گیری نار مرد
&&
زانک تقوی آب سوی نار برد
\\
کی سیه گردد ز آتش روی خوب
&&
کو نهد گلگونه از تقوی القلوب
\\
\end{longtable}
\end{center}
