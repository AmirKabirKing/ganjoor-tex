\begin{center}
\section*{بخش ۱۰۴ - در خواب گفتن هاتف مر عمر را رضی الله عنه کی چندین زر از بیت المال بن مرد ده کی در گورستان خفته است}
\label{sec:sh104}
\addcontentsline{toc}{section}{\nameref{sec:sh104}}
\begin{longtable}{l p{0.5cm} r}
آن زمان حق بر عمر خوابی گماشت
&&
تا که خویش از خواب نتوانست داشت
\\
در عجب افتاد کین معهود نیست
&&
این ز غیب افتاد بی مقصود نیست
\\
سر نهاد و خواب بردش خواب دید
&&
کامدش از حق ندا جانش شنید
\\
آن ندایی کاصل هر بانگ و نواست
&&
خود ندا آنست و این باقی صداست
\\
ترک و کرد و پارسی‌گو و عرب
&&
فهم کرده آن ندا بی‌گوش و لب
\\
خود چه جای ترک و تاجیکست و زنگ
&&
فهم کردست آن ندا را چوب و سنگ
\\
هر دمی از وی همی‌آید الست
&&
جوهر و اعراض می‌گردند هست
\\
گر نمی‌آید بلی زیشان ولی
&&
آمدنشان از عدم باشد بلی
\\
زانچ گفتم من ز فهم سنگ و چوب
&&
در بیانش قصه‌ای هش‌دار خوب
\\
\end{longtable}
\end{center}
