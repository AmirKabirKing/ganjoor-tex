\begin{center}
\section*{بخش ۱۴۶ - تهدید کردن نوح علیه‌السلام مر قوم را کی با من مپیچید کی من روپوشم با خدای می‌پیچید در میان این بحقیقت ای مخذولان}
\label{sec:sh146}
\addcontentsline{toc}{section}{\nameref{sec:sh146}}
\begin{longtable}{l p{0.5cm} r}
گفت نوح ای سرکشان من من نیم
&&
من ز جان مردم بجانان می‌زیم
\\
چون بمردم از حواس بوالبشر
&&
حق مرا شد سمع و ادراک و بصر
\\
چونک من من نیستم این دم ز هوست
&&
پیش این دم هرکه دم زد کافر اوست
\\
هست اندر نقش این روباه شیر
&&
سوی این روبه نشاید شد دلیر
\\
گر ز روی صورتش می‌نگروی
&&
غره شیران ازو می‌نشنوی
\\
گر نبودی نوح را از حق یدی
&&
پس جهانی را چرا بر هم زدی
\\
صد هزاران شیر بود او در تنی
&&
او چو آتش بود و عالم خرمنی
\\
چونک خرمن پاس عشر او نداشت
&&
او چنان شعله بر آن خرمن گماشت
\\
هر که او در پیش این شیر نهان
&&
بی‌ادب چون گرگ بگشاید دهان
\\
همچو گرگ آن شیر بر دراندش
&&
فانتقمنا منهم بر خواندش
\\
زخم یابد همچو گرگ از دست شیر
&&
پیش شیر ابله بود کو شد دلیر
\\
کاشکی آن زخم بر تن آمدی
&&
تا بدی کایمان و دل سالم بدی
\\
قوتم بگسست چون اینجا رسید
&&
چون توانم کرد این سر را پدید
\\
همچو آن روبه کم اشکم کنید
&&
پیش او روباه‌بازی کم کنید
\\
جمله ما و من به پیش او نهید
&&
ملک ملک اوست ملک او را دهید
\\
چون فقیر آیید اندر راه راست
&&
شیر و صید شیر خود آن شماست
\\
زانک او پاکست و سبحان وصف اوست
&&
بی نیازست او ز نغز و مغز و پوست
\\
هر شکار و هر کراماتی که هست
&&
از برای بندگان آن شهست
\\
نیست شه را طمع بهر خلق ساخت
&&
این همه دولت خنک آنکو شناخت
\\
آنک دولت آفرید و دو سرا
&&
ملک و دولتها چه کار آید ورا
\\
پیش سبحان پس نگه دارید دل
&&
تا نگردید از گمان بد خجل
\\
کو ببیند سر و فکر و جست و جو
&&
همچو اندر شیر خالص تار مو
\\
آنک او بی نقش ساده‌سینه شد
&&
نقشهای غیب را آیینه شد
\\
سر ما را بی‌گمان موقن شود
&&
زانکمؤمنآینهٔمؤمنبود
\\
چون زند او نقد ما را بر محک
&&
پس یقین را باز داند او ز شک
\\
چون شود جانش محک نقدها
&&
پس ببیند قلب را و قلب را
\\
\end{longtable}
\end{center}
