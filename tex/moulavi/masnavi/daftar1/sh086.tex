\begin{center}
\section*{بخش ۸۶ - دیدن خواجه طوطیان هندوستان را در دشت و پیغام رسانیدن از آن طوطی}
\label{sec:sh086}
\addcontentsline{toc}{section}{\nameref{sec:sh086}}
\begin{longtable}{l p{0.5cm} r}
چونک تا اقصای هندستان رسید
&&
در بیابان طوطیی چندی بدید
\\
مرکب استانید پس آواز داد
&&
آن سلام و آن امانت باز داد
\\
طوطیی زان طوطیان لرزید بس
&&
اوفتاد و مرد و بگسستش نفس
\\
شد پشیمان خواجه از گفت خبر
&&
گفت رفتم در هلاک جانور
\\
این مگر خویشست با آن طوطیک
&&
این مگر دو جسم بود و روح یک
\\
این چرا کردم چرا دادم پیام
&&
سوختم بیچاره را زین گفت خام
\\
این زبان چون سنگ و هم آهن وشست
&&
وانچ بجهد از زبان چون آتشست
\\
سنگ و آهن را مزن بر هم گزاف
&&
گه ز روی نقل و گه از روی لاف
\\
زانک تاریکست و هر سو پنبه‌زار
&&
درمیان پنبه چون باشد شرار
\\
ظالم آن قومی که چشمان دوختند
&&
زان سخنها عالمی را سوختند
\\
عالمی را یک سخن ویران کند
&&
روبهان مرده را شیران کند
\\
جانها در اصل خود عیسی‌دمند
&&
یک زمان زخمند و گاهی مرهمند
\\
گر حجاب از جانها بر خاستی
&&
گفت هر جانی مسیح‌آساستی
\\
گر سخن خواهی که گویی چون شکر
&&
صبر کن از حرص و این حلوا مخور
\\
صبر باشد مشتهای زیرکان
&&
هست حلوا آرزوی کودکان
\\
هرکه صبر آورد گردون بر رود
&&
هر که حلوا خورد واپس‌تر رود
\\
\end{longtable}
\end{center}
