\begin{center}
\section*{بخش ۱۲۸ - تعیین کردن زن طریق طلب روزی کدخدای خود را و قبول کردن او}
\label{sec:sh128}
\addcontentsline{toc}{section}{\nameref{sec:sh128}}
\begin{longtable}{l p{0.5cm} r}
گفت زن یک آفتابی تافتست
&&
عالمی زو روشنایی یافتست
\\
نایب رحمان خلیفهٔ کردگار
&&
شهر بغدادست از وی چون بهار
\\
گر بپیوندی بدان شه شه شوی
&&
سوی هر ادبیر تا کی می‌روی
\\
همنشینی با شهان چون کیمیاست
&&
چون نظرشان کیمیایی خود کجاست
\\
چشم احمد بر ابوبکری زده
&&
او ز یک تصدیق صدیق آمده
\\
گفت من شه را پذیرا چون شوم
&&
بی بهانه سوی او من چون روم
\\
نسبتی باید مرا یا حیلتی
&&
هیچ پیشه راست شد بی‌آلتی
\\
همچو مجنونی که بشنید از یکی
&&
که مرض آمد به لیلی اندکی
\\
گفت آوه بی بهانه چون روم
&&
ور بمانم از عیادت چون شوم
\\
لیتنی کنت طبیبا حاذقا
&&
کنت امشی نحو لیلی سابقا
\\
قل تعالوا گفت حق ما را بدان
&&
تا بود شرم‌اشکنی ما را نشان
\\
شب‌پران را گر نظر و آلت بدی
&&
روزشان جولان و خوش حالت بدی
\\
گفت چون شاه کرم میدان رود
&&
عین هر بی‌آلتی آلت شود
\\
زانک آلت دعوی است و هستی است
&&
کار در بی‌آلتی و پستی است
\\
گفت کی بی‌آلتی سودا کنم
&&
تا نه من بی‌آلتی پیدا کنم
\\
پس گواهی بایدم بر مفلسی
&&
تا مرا رحمی کند شاه غنی
\\
تو گواهی غیر گفت و گو و رنگ
&&
وا نما تا رحم آرد شاه شنگ
\\
کین گواهی که ز گفت و رنگ بد
&&
نزد آن قاضی القضاة آن جرح شد
\\
صدق می‌خواهد گواه حال او
&&
تا بتابد نور او بی قال او
\\
\end{longtable}
\end{center}
