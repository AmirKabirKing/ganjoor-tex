\begin{center}
\section*{بخش ۴۰ - عتاب کردن آتش را آن پادشاه جهود}
\label{sec:sh040}
\addcontentsline{toc}{section}{\nameref{sec:sh040}}
\begin{longtable}{l p{0.5cm} r}
رو به آتش کرد شه کای تندخو
&&
آن جهان سوز طبیعی خوت کو
\\
چون نمی‌سوزی چه شد خاصیتت
&&
یا ز بخت ما دگر شد نیتت
\\
می‌نبخشایی تو بر آتش‌پرست
&&
آنک نپرستد ترا او چون برست
\\
هرگز ای آتش تو صابر نیستی
&&
چون نسوزی چیست قادر نیستی
\\
چشم‌بندست این عجب یا هوش‌بند
&&
چون نسوزاند چنین شعلهٔ بلند
\\
جادوی کردت کسی یا سیمیاست
&&
یا خلاف طبع تو از بخت ماست
\\
گفت آتش من همانم ای شمن
&&
اندر آ تا تو ببینی تاب من
\\
طبع من دیگر نگشت و عنصرم
&&
تیغ حقم هم بدستوری برم
\\
بر در خرگهٔ سگان ترکمان
&&
چاپلوسی کرده پیش میهمان
\\
ور بخرگه بگذرد بیگانه‌رو
&&
حمله بیند از سگان شیرانه او
\\
من ز سگ کم نیستم در بندگی
&&
کم ز ترکی نیست حق در زندگی
\\
آتش طبعت اگر غمگین کند
&&
سوزش از امر ملیک دین کند
\\
آتش طبعت اگر شادی دهد
&&
اندرو شادی ملیک دین نهد
\\
چونک غم‌بینی تو استغفار کن
&&
غم بامر خالق آمد کار کن
\\
چون بخواهد عین غم شادی شود
&&
عین بند پای آزادی شود
\\
باد و خاک و آب و آتش بنده‌اند
&&
با من و تو مرده با حق زنده‌اند
\\
پیش حق آتش همیشه در قیام
&&
همچو عاشق روز و شب پیچان مدام
\\
سنگ بر آهن زنی بیرون جهد
&&
هم به امر حق قدم بیرون نهد
\\
آهن و سنگ هوا بر هم مزن
&&
کین دو می‌زایند همچون مرد و زن
\\
سنگ و آهن خود سبب آمد ولیک
&&
تو به بالاتر نگر ای مرد نیک
\\
کین سبب را آن سبب آورد پیش
&&
بی‌سبب کی شد سبب هرگز ز خویش
\\
و آن سببها کانبیا را رهبرند
&&
آن سببها زین سببها برترند
\\
این سبب را آن سبب عامل کند
&&
باز گاهی بی بر و عاطل کند
\\
این سبب را محرم آمد عقلها
&&
و آن سببهاراست محرم انبیا
\\
این سبب چه بود بتازی گو رسن
&&
اندرین چه این رسن آمد بفن
\\
گردش چرخه رسن را علتست
&&
چرخه گردان را ندیدن زلتست
\\
این رسنهای سببها در جهان
&&
هان و هان زین چرخ سرگردان مدان
\\
تا نمانی صفر و سرگردان چو چرخ
&&
تا نسوزی تو ز بی‌مغزی چو مرخ
\\
باد آتش می‌شود از امر حق
&&
هر دو سرمست آمدند از خمر حق
\\
آب حلم و آتش خشم ای پسر
&&
هم ز حق بینی چو بگشایی بصر
\\
گر نبودی واقف از حق جان باد
&&
فرق کی کردی میان قوم عاد
\\
هود گرد مؤمنان خطی کشید
&&
نرم می‌شد باد کانجا می‌رسید
\\
هر که بیرون بود زان خط جمله را
&&
پاره پاره می‌گسست اندر هوا
\\
همچنین شیبان راعی می‌کشید
&&
گرد بر گرد رمه خطی پدید
\\
چون بجمعه می‌شد او وقت نماز
&&
تا نیارد گرگ آنجا ترک‌تاز
\\
هیچ گرگی در نرفتی اندر آن
&&
گوسفندی هم نگشتی زان نشان
\\
باد حرص گرگ و حرص گوسفند
&&
دایرهٔ مرد خدا را بود بند
\\
همچنین باد اجل با عارفان
&&
نرم و خوش همچون نسیم یوسفان
\\
آتش ابراهیم را دندان نزد
&&
چون گزیدهٔ حق بود چونش گزد
\\
ز آتش شهوت نسوزد اهل دین
&&
باقیان را برده تا قعر زمین
\\
موج دریا چون بامر حق بتاخت
&&
اهل موسی را ز قبطی وا شناخت
\\
خاک قارون را چو فرمان در رسید
&&
با زر و تختش به قعر خود کشید
\\
آب و گل چون از دم عیسی چرید
&&
بال و پر بگشاد مرغی شد پرید
\\
هست تسبیحت بخار آب و گل
&&
مرغ جنت شد ز نفخ صدق دل
\\
کوه طور از نور موسی شد به رقص
&&
صوفی کامل شد و رست او ز نقص
\\
چه عجب گر کوه صوفی شد عزیز
&&
جسم موسی از کلوخی بود نیز
\\
\end{longtable}
\end{center}
