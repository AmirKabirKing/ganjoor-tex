\begin{center}
\section*{بخش ۱۵۱ - دعا کردن بلعم با عور کی موسی و قومش را از این شهر کی حصار داده‌اند بی مراد باز گردان و مستجاب شدن دعای او}
\label{sec:sh151}
\addcontentsline{toc}{section}{\nameref{sec:sh151}}
\begin{longtable}{l p{0.5cm} r}
بلعم با عور را خلق جهان
&&
سغبه شد مانند عیسی زمان
\\
سجدهٔ ناوردند کس را دون او
&&
صحت رنجور بود افسون او
\\
پنجه زد با موسی از کبر و کمال
&&
آنچنان شد که شنیدستی تو حال
\\
صد هزار ابلیس و بلعم در جهان
&&
همچنین بودست پیدا و نهان
\\
این دو را مشهور گردانید اله
&&
تا که باشد این دو بر باقی گواه
\\
این دو دزد آویخت از دار بلند
&&
ورنه اندر قهر بس دزدان بدند
\\
این دو را پرچم به سوی شهر برد
&&
کشتگان قهر را نتوان شمرد
\\
نازنینی تو ولی در حد خویش
&&
الله الله پا منه از حد بیش
\\
گر زنی بر نازنین‌تر از خودت
&&
در تگ هفتم زمین زیر آردت
\\
قصهٔ عاد و ثمود از بهر چیست
&&
تا بدانی کانبیا را نازکیست
\\
این نشان خسف و قذف و صاعقه
&&
شد بیان عز نفس ناطقه
\\
جمله حیوان را پی انسان بکش
&&
جمله انسان را بکش از بهر هش
\\
هش چه باشد عقل کل هوشمند
&&
هوش جزوی هش بود اما نژند
\\
جمله حیوانات وحشی ز آدمی
&&
باشد از حیوان انسی در کمی
\\
خون آنها خلق را باشد سبیل
&&
زانک وحشی‌اند از عقل جلیل
\\
عزت وحشی بدین افتاد پست
&&
که مر انسان را مخالف آمدست
\\
پس چه عزت باشدت ای نادره
&&
چون شدی تو حمر مستنفره
\\
خر نشاید کشت از بهر صلاح
&&
چون شود وحشی شود خونش مباح
\\
گرچه خر را دانش زاجر نبود
&&
هیچ معذورش نمی‌دارد ودود
\\
پس چو وحشی شد از آن دم آدمی
&&
کی بود معذور ای یار سمی
\\
لاجرم کفار را شد خون مباح
&&
همچو وحشی پیش نشاب و رماح
\\
جفت و فرزندانشان جمله سبیل
&&
زانک بی‌عقلند و مردود و ذلیل
\\
باز عقلی کو رمد از عقل عقل
&&
کرد از عقلی به حیوانات نقل
\\
\end{longtable}
\end{center}
