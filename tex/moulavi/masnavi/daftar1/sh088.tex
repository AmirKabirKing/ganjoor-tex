\begin{center}
\section*{بخش ۸۸ - تعظیم ساحران مر موسی را علیه‌السلام کی چه می‌فرمایی اول تو اندازی عصا}
\label{sec:sh088}
\addcontentsline{toc}{section}{\nameref{sec:sh088}}
\begin{longtable}{l p{0.5cm} r}
ساحران در عهد فرعون لعین
&&
چون مری کردند با موسی بکین
\\
لیک موسی را مقدم داشتند
&&
ساحران او را مکرم داشتند
\\
زانک گفتندش که فرمان آن تست
&&
گر همی خواهی عصا تو فکن نخست
\\
گفت نی اول شما ای ساحران
&&
افکنید ان مکرها را درمیان
\\
این قدر تعظیم دینشان را خرید
&&
کز مری آن دست و پاهاشان برید
\\
ساحران چون حق او بشناختند
&&
دست و پا در جرم آن در باختند
\\
لقمه و نکته‌ست کامل را حلال
&&
تو نه‌ای کامل مخور می‌باش لال
\\
چون تو گوشی او زبان نی جنس تو
&&
گوشها را حق بفرمود انصتوا
\\
کودک اول چون بزاید شیرنوش
&&
مدتی خامش بود او جمله گوش
\\
مدتی می‌بایدش لب دوختن
&&
از سخن تا او سخن آموختن
\\
ور نباشد گوش و تی‌تی می‌کند
&&
خویشتن را گنگ گیتی می‌کند
\\
کر اصلی کش نبد ز آغاز گوش
&&
لال باشد کی کند در نطق جوش
\\
زانک اول سمع باید نطق را
&&
سوی منطق از ره سمع اندر آ
\\
وادخلوا الابیات من ابوابها
&&
واطلبوا الاغراض فی اسبابها
\\
نطق کان موقوف راه سمع نیست
&&
جز که نطق خالق بی‌طمع نیست
\\
مبدعست او تابع استاد نی
&&
مسند جمله ورا اسناد نی
\\
باقیان هم در حرف هم در مقال
&&
تابع استاد و محتاج مثال
\\
زین سخن گر نیستی بیگانه‌ای
&&
دلق و اشکی گیر در ویرانه‌ای
\\
زانک آدم زان عتاب از اشک رست
&&
اشک تر باشد دم توبه‌پرست
\\
بهر گریه آمد آدم بر زمین
&&
تا بود گریان و نالان و حزین
\\
آدم از فردوس و از بالای هفت
&&
پای ماچان از برای عذر رفت
\\
گر ز پشت آدمی وز صلب او
&&
در طلب می‌باش هم در طلب او
\\
ز آتش دل و آب دیده نقل ساز
&&
بوستان از ابر و خورشیدست باز
\\
تو چه دانی ذوق آب دیدگان
&&
عاشق نانی تو چون نادیدگان
\\
گر تو این انبان ز نان خالی کنی
&&
پر ز گوهرهای اجلالی کنی
\\
طفل جان از شیر شیطان باز کن
&&
بعد از آنش با ملک انباز کن
\\
تا تو تاریک و ملول و تیره‌ای
&&
دان که با دیو لعین همشیره‌ای
\\
لقمه‌ای کو نور افزود و کمال
&&
آن بود آورده از کسب حلال
\\
روغنی کاید چراغ ما کشد
&&
آب خوانش چون چراغی را کشد
\\
علم و حکمت زاید از لقمهٔ حلال
&&
عشق و رقت آید از لقمهٔ حلال
\\
چون ز لقمه تو حسد بینی و دام
&&
جهل و غفلت زاید آن را دان حرام
\\
هیچ گندم کاری و جو بر دهد
&&
دیده‌ای اسپی که کرهٔ خر دهد
\\
لقمه تخمست و برش اندیشه‌ها
&&
لقمه بحر و گوهرش اندیشه‌ها
\\
زاید از لقمهٔ حلال اندر دهان
&&
میل خدمت عزم رفتن آن جهان
\\
\end{longtable}
\end{center}
