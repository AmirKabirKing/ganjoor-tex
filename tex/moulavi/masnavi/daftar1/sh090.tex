\begin{center}
\section*{بخش ۹۰ - شنیدن آن طوطی حرکت آن طوطیان و مردن آن طوطی در قفص و نوحهٔ خواجه بر وی}
\label{sec:sh090}
\addcontentsline{toc}{section}{\nameref{sec:sh090}}
\begin{longtable}{l p{0.5cm} r}
چون شنید آن مرغ کان طوطی چه کرد
&&
پس بلرزید اوفتاد و گشت سرد
\\
خواجه چون دیدش فتاده همچنین
&&
بر جهید و زد کله را بر زمین
\\
چون بدین رنگ و بدین حالش بدید
&&
خواجه بر جست و گریبان را درید
\\
گفت ای طوطی خوب خوش‌حنین
&&
این چه بودت این چرا گشتی چنین
\\
ای دریغا مرغ خوش‌آواز من
&&
ای دریغا همدم و همراز من
\\
ای دریغا مرغ خوش‌الحان من
&&
راح روح و روضه و ریحان من
\\
گر سلیمان را چنین مرغی بدی
&&
کی خود او مشغول آن مرغان شدی
\\
ای دریغا مرغ کارزان یافتم
&&
زود روی از روی او بر تافتم
\\
ای زبان تو بس زیانی بر وری
&&
چون توی گویا چه گویم من ترا
\\
ای زبان هم آتش و هم خرمنی
&&
چند این آتش درین خرمن زنی
\\
در نهان جان از تو افغان می‌کند
&&
گرچه هر چه گوییش آن می‌کند
\\
ای زبان هم گنج بی‌پایان توی
&&
ای زبان هم رنج بی‌درمان توی
\\
هم صفیر و خدعهٔ مرغان توی
&&
هم انیس وحشت هجران توی
\\
چند امانم می‌دهی ای بی امان
&&
ای تو زه کرده به کین من کمان
\\
نک بپرانیده‌ای مرغ مرا
&&
در چراگاه ستم کم کن چرا
\\
یا جواب من بگو یا داد ده
&&
یا مرا ز اسباب شادی یاد ده
\\
ای دریغا نور ظلمت‌سوز من
&&
ای دریغا صبح روز افروز من
\\
ای دریغا مرغ خوش‌پرواز من
&&
ز انتها پریده تا آغاز من
\\
عاشق رنجست نادان تا ابد
&&
خیز لا اقسم بخوان تا فی کبد
\\
از کبد فارغ بدم با روی تو
&&
وز زبد صافی بدم در جوی تو
\\
این دریغاها خیال دیدنست
&&
وز وجود نقد خود ببریدنست
\\
غیرت حق بود و با حق چاره نیست
&&
کو دلی کز عشق حق صد پاره نیست
\\
غیرت آن باشد که او غیر همه‌ست
&&
آنک افزون از بیان و دمدمه‌ست
\\
ای دریغا اشک من دریا بدی
&&
تا نثار دلبر زیبا بدی
\\
طوطی من مرغ زیرکسار من
&&
ترجمان فکرت و اسرار من
\\
هرچه روزی داد و ناداد آیدم
&&
او ز اول گفته تا یاد آیدم
\\
طوطیی کآید ز وحی آواز او
&&
پیش از آغاز وجود آغاز او
\\
اندرون تست آن طوطی نهان
&&
عکس او را دیده تو بر این و آن
\\
می برد شادیت را تو شاد ازو
&&
می‌پذیری ظلم را چون داد ازو
\\
ای که جان را بهر تن می‌سوختی
&&
سوختی جان را و تن افروختی
\\
سوختم من سوخته خواهد کسی
&&
تا زمن آتش زند اندر خسی
\\
سوخته چون قابل آتش بود
&&
سوخته بستان که آتش‌کش بود
\\
ای دریغا ای دریغا ای دریغ
&&
کانچنان ماهی نهان شد زیر میغ
\\
چون زنم دم کآتش دل تیز شد
&&
شیر هجر آشفته و خون‌ریز شد
\\
آنک او هشیار خود تندست و مست
&&
چون بود چون او قدح گیرد به دست
\\
شیر مستی کز صفت بیرون بود
&&
از بسیط مرغزار افزون بود
\\
قافیه‌اندیشم و دلدار من
&&
گویدم مندیش جز دیدار من
\\
خوش نشین ای قافیه‌اندیش من
&&
قافیهٔ دولت توی در پیش من
\\
حرف چه بود تا تو اندیشی از آن
&&
حرف چه بود خار دیوار رزان
\\
حرف و صوت و گفت را بر هم زنم
&&
تا که بی این هر سه با تو دم زنم
\\
آن دمی کز آدمش کردم نهان
&&
با تو گویم ای تو اسرار جهان
\\
آن دمی را که نگفتم با خلیل
&&
و آن غمی را که نداند جبرئیل
\\
آن دمی کز وی مسیحا دم نزد
&&
حق ز غیرت نیز بی ما هم نزد
\\
ما چه باشد در لغت اثبات و نفی
&&
من نه اثباتم منم بی‌ذات و نفی
\\
من کسی در ناکسی در یافتم
&&
پس کسی در ناکسی در بافتم
\\
جمله شاهان بندهٔ بندهٔ خودند
&&
جمله خلقان مردهٔ مردهٔ خودند
\\
جمله شاهان پست پست خویش را
&&
جمله خلقان مست مست خویش را
\\
می‌شود صیاد مرغان را شکار
&&
تا کند ناگاه ایشان را شکار
\\
بی‌دلان را دلبران جسته بجان
&&
جمله معشوقان شکار عاشقان
\\
هر که عاشق دیدیش معشوق دان
&&
کو به نسبت هست هم این و هم آن
\\
تشنگان گر آب جویند از جهان
&&
آب جوید هم به عالم تشنگان
\\
چونک عاشق اوست تو خاموش باش
&&
او چو گوشت می‌کشد تو گوش باش
\\
بند کن چون سیل سیلانی کند
&&
ور نه رسوایی و ویرانی کند
\\
من چه غم دارم که ویرانی بود
&&
زیر ویران گنج سلطانی بود
\\
غرق حق خواهد که باشد غرق‌تر
&&
همچو موج بحر جان زیر و زبر
\\
زیر دریا خوشتر آید یا زبر
&&
تیر او دلکش‌تر آید یا سپر
\\
پاره کردهٔ وسوسه باشی دلا
&&
گر طرب را باز دانی از بلا
\\
گر مرادت را مذاق شکرست
&&
بی‌مرادی نه مراد دلبرست
\\
هر ستاره‌ش خونبهای صد هلال
&&
خون عالم ریختن او را حلال
\\
ما بها و خونبها را یافتیم
&&
جانب جان باختن بشتافتیم
\\
ای حیات عاشقان در مردگی
&&
دل نیابی جز که در دل‌بردگی
\\
من دلش جسته به صد ناز و دلال
&&
او بهانه کرده با من از ملال
\\
گفتم آخر غرق تست این عقل و جان
&&
گفت رو رو بر من این افسون مخوان
\\
من ندانم آنچ اندیشیده‌ای
&&
ای دو دیده دوست را چون دیده‌ای
\\
ای گرانجان خوار دیدستی ورا
&&
زانک بس ارزان خریدستی ورا
\\
هرکه او ارزان خرد ارزان دهد
&&
گوهری طفلی به قرصی نان دهد
\\
غرق عشقی‌ام که غرقست اندرین
&&
عشقهای اولین و آخرین
\\
مجملش گفتم نکردم زان بیان
&&
ورنه هم افهام سوزد هم زبان
\\
من چو لب گویم لب دریا بود
&&
من چو لا گویم مراد الا بود
\\
من ز شیرینی نشستم رو ترش
&&
من ز بسیاری گفتارم خمش
\\
تا که شیرینی ما از دو جهان
&&
در حجاب رو ترش باشد نهان
\\
تا که در هر گوش ناید این سخن
&&
یک همی گویم ز صد سر لدن
\\
\end{longtable}
\end{center}
