\begin{center}
\section*{بخش ۱۲۵ - در معنی آنک آنچ ولی کند مرید را نشاید گستاخی کردن و همان فعل کردن کی حلوا طبیب را زیان ندارد اما بیماران را زیان دارد و سرما و برف انگور را زیان ندارد اما غوره را زیان دارد کی در راهست کی لیغفرلک الله ما تقدم من ذنبک و ما تاخر}
\label{sec:sh125}
\addcontentsline{toc}{section}{\nameref{sec:sh125}}
\begin{longtable}{l p{0.5cm} r}
گر ولی زهری خورد نوشی شود
&&
ور خورد طالب سیه‌هوشی شود
\\
رب هب لی از سلیمان آمدست
&&
که مده غیر مرا این ملک دست
\\
تو مکن با غیر من این لطف و جود
&&
این حسد را ماند اما آن نبود
\\
نکتهٔ لا ینبغی می‌خوان بجان
&&
سر من بعدی ز بخل او مدان
\\
بلک اندر ملک دید او صد خطر
&&
موبمو ملک جهان بد بیم سر
\\
بیم سر با بیم سر با بیم دین
&&
امتحانی نیست ما را مثل این
\\
پس سلیمان همتی باید که او
&&
بگذرد زین صد هزاران رنگ و بو
\\
با چنان قوت که او را بود هم
&&
موج آن ملکش فرو می‌بست دم
\\
چون برو بنشست زین اندوه گرد
&&
بر همه شاهان عالم رحم کرد
\\
شد شفیع و گفت این ملک و لوا
&&
با کمالی ده که دادی مر مرا
\\
هرکه را بدهی و بکنی آن کرم
&&
او سلیمانست وانکس هم منم
\\
او نباشد بعدی او باشد معی
&&
خود معی چه بود منم بی‌مدعی
\\
شرح این فرضست گفتن لیک من
&&
باز می‌گردم به قصهٔ مرد و زن
\\
\end{longtable}
\end{center}
