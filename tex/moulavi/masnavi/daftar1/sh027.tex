\begin{center}
\section*{بخش ۲۷ - مکر کردن مریدان کی خلوت را بشکن}
\label{sec:sh027}
\addcontentsline{toc}{section}{\nameref{sec:sh027}}
\begin{longtable}{l p{0.5cm} r}
جمله گفتند ای حکیم رخنه‌جو
&&
این فریب و این جفا با ما مگو
\\
چارپا را قدر طاقت با رنه
&&
بر ضعیفان قدر قوت کار نه
\\
دانهٔ هر مرغ اندازهٔ ویست
&&
طعمهٔ هر مرغ انجیری کیست
\\
طفل را گر نان دهی بر جای شیر
&&
طفل مسکین را از آن نان مرده گیر
\\
چونک دندانها بر آرد بعد از آن
&&
هم بخود گردد دلش جویای نان
\\
مرغ پر نارسته چون پران شود
&&
لقمهٔ هر گربهٔ دران شود
\\
چون بر آرد پر بپرد او بخود
&&
بی‌تکلف بی‌صفیر نیک و بد
\\
دیو را نطق تو خامش می‌کند
&&
گوش ما را گفت تو هش می‌کند
\\
گوش ما هوشست چون گویا توی
&&
خشک ما بحرست چون دریا توی
\\
با تو ما را خاک بهتر از فلک
&&
ای سماک از تو منور تا سمک
\\
بی‌تو ما را بر فلک تاریکیست
&&
با تو ای ماه این فلک باری کیست
\\
صورت رفعت بود افلاک را
&&
معنی رفعت روان پاک را
\\
صورت رفعت برای جسمهاست
&&
جسمها در پیش معنی اسمهاست
\\
\end{longtable}
\end{center}
