\begin{center}
\section*{بخش ۱۱۶ - نصیحت کردن مرد مر زن را کی در فقیران به خواری منگر و در کار حق به گمان کمال نگر و طعنه مزن در فقر و فقیران به خیال و گمان بی‌نوایی خویشتن}
\label{sec:sh116}
\addcontentsline{toc}{section}{\nameref{sec:sh116}}
\begin{longtable}{l p{0.5cm} r}
گفت ای زن تو زنی یا بوالحزن
&&
فقر فخر آمد مرا بر سر مزن
\\
مال و زر سر را بود همچون کلاه
&&
کل بود او کز کله سازد پناه
\\
آنک زلف جعد و رعنا باشدش
&&
چون کلاهش رفت خوشتر آیدش
\\
مرد حق باشد بمانند بصر
&&
پس برهنه به که پوشیده نظر
\\
وقت عرضه کردن آن برده‌فروش
&&
بر کند از بنده جامهٔ عیب‌پوش
\\
ور بود عیبی برهنه‌ش کی کند
&&
بل بجامه خدعه‌ای با وی کند
\\
گوید ای شرمنده است از نیک و بد
&&
از برهنه کردن او از تو رمد
\\
خواجه در عیبست غرقه تا به گوش
&&
خواجه را مالست و مالش عیب‌پوش
\\
کز طمع عیبش نبیند طامعی
&&
گشت دلها را طمعها جامعی
\\
ور گدا گوید سخن چون زر کان
&&
ره نیابد کالهٔ او در دکان
\\
کار درویشی ورای فهم تست
&&
سوی درویشی بمنگر سست سست
\\
زانک درویشان ورای ملک و مال
&&
روزیی دارند ژرف از ذوالجلال
\\
حق تعالی عادلست و عادلان
&&
کی کنند استم‌گری بر بی‌دلان
\\
آن یکی را نعمت و کالا دهند
&&
وین دگر را بر سر آتش نهند
\\
آتشش سوزا که دارد این گمان
&&
بر خدا و خالق هر دو جهان
\\
فقر فخری از گزافست و مجاز
&&
نه هزاران عز پنهانست و ناز
\\
از غضب بر من لقبها راندی
&&
یارگیر و مارگیرم خواندی
\\
گر بگیرم برکنم دندان مار
&&
تاش از سر کوفتن نبود ضرار
\\
زانک آن دندان عدو جان اوست
&&
من عدو را می‌کنم زین علم دوست
\\
از طمع هرگز نخوانم من فسون
&&
این طمع را کرده‌ام من سرنگون
\\
حاش لله طمع من از خلق نیست
&&
از قناعت در دل من عالمیست
\\
بر سر امرودبن بینی چنان
&&
زان فرود آ تا نماند آن گمان
\\
چون که بر گردی تو سرگشته شوی
&&
خانه را گردنده بینی و آن توی
\\
\end{longtable}
\end{center}
