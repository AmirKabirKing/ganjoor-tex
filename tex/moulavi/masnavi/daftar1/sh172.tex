\begin{center}
\section*{بخش ۱۷۲ - گفتن امیر الممنین علی کرم الله وجهه با قرین خود کی چون خدو انداختی در روی من نفس من جنبید و اخلاص عمل نماند مانع کشتن تو آن شد}
\label{sec:sh172}
\addcontentsline{toc}{section}{\nameref{sec:sh172}}
\begin{longtable}{l p{0.5cm} r}
گفت امیر المؤمنین با آن جوان
&&
که به هنگام نبرد ای پهلوان
\\
چون خدو انداختی در روی من
&&
نفس جنبید و تبه شد خوی من
\\
نیم بهر حق شد و نیمی هوا
&&
شرکت اندر کار حق نبود روا
\\
تو نگاریدهٔ کف مولیستی
&&
آن حقی کردهٔ من نیستی
\\
نقش حق را هم به امر حق شکن
&&
بر زجاجهٔ دوست سنگ دوست زن
\\
گبر این بشنید و نوری شد پدید
&&
در دل او تا که زناری برید
\\
گفت من تخم جفا می‌کاشتم
&&
من ترا نوعی دگر پنداشتم
\\
تو ترازوی احدخو بوده‌ای
&&
بل زبانهٔ هر ترازو بوده‌ای
\\
تو تبار و اصل و خویشم بوده‌ای
&&
تو فروغ شمع کیشم بوده‌ای
\\
من غلام آن چراغ چشم‌جو
&&
که چراغت روشنی پذرفت ازو
\\
من غلام موج آن دریای نور
&&
که چنین گوهر بر آرد در ظهور
\\
عرضه کن بر من شهادت را که من
&&
مر ترا دیدم سرافراز زمن
\\
قرب پنجه کس ز خویش و قوم او
&&
عاشقانه سوی دین کردند رو
\\
او به تیغ حلم چندین حلق را
&&
وا خرید از تیغ و چندین خلق را
\\
تیغ حلم از تیغ آهن تیزتر
&&
بل ز صد لشکر ظفر انگیزتر
\\
ای دریغا لقمه‌ای دو خورده شد
&&
جوشش فکرت از آن افسرده شد
\\
گندمی خورشید آدم را کسوف
&&
چون ذنب شعشاع بدری را خسوف
\\
اینت لطف دل که از یک مشت گل
&&
ماه او چون می‌شود پروین‌گسل
\\
نان چو معنی بود خوردش سود بود
&&
چونک صورت گشت انگیزد جحود
\\
همچو خار سبز کاشتر می‌خورد
&&
زان خورش صد نفع و لذت می‌برد
\\
چونک آن سبزیش رفت و خشک گشت
&&
چون همان را می‌خورد اشتر ز دشت
\\
می‌دراند کام و لنجش ای دریغ
&&
کانچنان ورد مربی گشت تیغ
\\
نان چو معنی بود بود آن خار سبز
&&
چونک صورت شد کنون خشکست و گبز
\\
تو بدان عادت که او را پیش ازین
&&
خورده بودی ای وجود نازنین
\\
بر همان بو می‌خوری این خشک را
&&
بعد از آن کامیخت معنی با ثری
\\
گشت خاک‌آمیز و خشک و گوشت‌بر
&&
زان گیاه اکنون بپرهیز ای شتر
\\
سخت خاک‌آلود می‌آید سخن
&&
آب تیره شد سر چه بند کن
\\
تا خدایش باز صاف و خوش کند
&&
او که تیره کرد هم صافش کند
\\
صبر آرد آرزو را نه شتاب
&&
صبر کن والله اعلم بالصواب
\\
\end{longtable}
\end{center}
