\begin{center}
\section*{بخش ۶۱ - تولیدن شیر از دیر آمدن خرگوش}
\label{sec:sh061}
\addcontentsline{toc}{section}{\nameref{sec:sh061}}
\begin{longtable}{l p{0.5cm} r}
همچو آن خرگوش کو بر شیر زد
&&
روح او کی بود اندر خورد قد
\\
شیر می‌گفت از سر تیزی و خشم
&&
کز ره گوشم عدو بر بست چشم
\\
مکرهای جبریانم بسته کرد
&&
تیغ چوبینشان تنم را خسته کرد
\\
زین سپس من نشنوم آن دمدمه
&&
بانگ دیوانست و غولان آن همه
\\
بر دران ای دل تو ایشان را مه‌ایست
&&
پوستشان برکن کشان جز پوست نیست
\\
پوست چه بود گفته‌های رنگ رنگ
&&
چون زره بر آب کش نبود درنگ
\\
این سخن چون پوست و معنی مغز دان
&&
این سخن چون نقش و معنی همچو جان
\\
پوست باشد مغز بد را عیب‌پوش
&&
مغز نیکو را ز غیرت غیب‌پوش
\\
چون قلم از باد بد دفتر ز آب
&&
هرچه بنویسی فنا گردد شتاب
\\
نقش آبست ار وفا جویی از آن
&&
باز گردی دستهای خود گزان
\\
باد در مردم هوا و آرزوست
&&
چون هوا بگذاشتی پیغام هوست
\\
خوش بود پیغامهای کردگار
&&
کو ز سر تا پای باشد پایدار
\\
خطبهٔ شاهان بگردد و آن کیا
&&
جز کیا و خطبه‌های انبیا
\\
زانک بوش پادشاهان از هواست
&&
بارنامهٔ انبیا از کبریاست
\\
از درمها نام شاهان برکنند
&&
نام احمد تا ابد بر می‌زنند
\\
نام احمد نام جملهٔ انبیاست
&&
چونک صد آمد نود هم پیش ماست
\\
\end{longtable}
\end{center}
