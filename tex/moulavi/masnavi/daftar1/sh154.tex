\begin{center}
\section*{بخش ۱۵۴ - به عیادت رفتن کر بر همسایهٔ رنجور خویش}
\label{sec:sh154}
\addcontentsline{toc}{section}{\nameref{sec:sh154}}
\begin{longtable}{l p{0.5cm} r}
آن کری را گفت افزون مایه‌ای
&&
که ترا رنجور شد همسایه‌ای
\\
گفت با خود کر که با گوش گران
&&
من چه دریابم ز گفت آن جوان
\\
خاصه رنجور و ضعیف آواز شد
&&
لیک باید رفت آنجا نیست بد
\\
چون ببینم کان لبش جنبان شود
&&
من قیاسی گیرم آن را هم ز خود
\\
چون بگویم چونی ای محنت‌کشم
&&
او بخواهد گفت نیکم یا خوشم
\\
من بگویم شکر چه خوردی ابا
&&
او بگوید شربتی یا ماش با
\\
من بگویم صحه نوشت کیست آن
&&
از طبیبان پیش تو گوید فلان
\\
من بگویم بس مبارک‌پاست او
&&
چونک او آمد شود کارت نکو
\\
پای او را آزمودستیم ما
&&
هر کجا شد می‌شود حاجت روا
\\
این جوابات قیاسی راست کرد
&&
پیش آن رنجور شد آن نیک‌مرد
\\
گفت چونی گفت مردم گفت شکر
&&
شد ازین رنجور پر آزار و نکر
\\
کین چه شکرست او مگر با ما بدست
&&
کر قیاسی کرد و آن کژ آمدست
\\
بعد از آن گفتش چه خوردی گفت زهر
&&
گفت نوشت باد افزون گشت قهر
\\
بعد از آن گفت از طبیبان کیست او
&&
که همی‌آید به چاره پیش تو
\\
گفت عزرائیل می‌آید برو
&&
گفت پایش بس مبارک شاد شو
\\
کر برون آمد بگفت او شادمان
&&
شکر کش کردم مراعات این زمان
\\
گفت رنجور این عدو جان ماست
&&
ما ندانستیم کو کان جفاست
\\
خاطر رنجور جویان شد سقط
&&
تا که پیغامش کند از هر نمط
\\
چون کسی که خورده باشد آش بد
&&
می‌بشوراند دلش تا قی کند
\\
کظم غیظ اینست آن را قی مکن
&&
تا بیابی در جزا شیرین سخن
\\
چون نبودش صبر می‌پیچید او
&&
کین سگ زن‌روسپی حیز کو
\\
تا بریزم بر وی آنچ گفته بود
&&
کان زمان شیر ضمیرم خفته بود
\\
چون عیادت بهر دل‌آرامیست
&&
این عیادت نیست دشمن کامیست
\\
تا ببیند دشمن خود را نزار
&&
تا بگیرد خاطر زشتش قرار
\\
بس کسان کایشان ز طاعت گمرهند
&&
دل به رضوان و ثواب آن دهند
\\
خود حقیقت معصیت باشد خفی
&&
بس کدر کان را تو پنداری صفی
\\
همچو آن کر کو همی پنداشتست
&&
کو نکویی کرد و آن بر عکس جست
\\
او نشسته خوش که خدمت کرده‌ام
&&
حق همسایه بجا آورده‌ام
\\
بهر خود او آتشی افروختست
&&
در دل رنجور و خود را سوختست
\\
فاتقوا النار التی اوقدتم
&&
انکم فی المعصیه ازددتم
\\
گفت پیغامبر به یک صاحب‌ریا
&&
صل انک لم تصل یا فتی
\\
از برای چارهٔ این خوفها
&&
آمد اندر هر نمازی اهدنا
\\
کین نمازم را میامیز ای خدا
&&
با نماز ضالین و اهل ریا
\\
از قیاسی که بکرد آن کر گزین
&&
صحبت ده‌ساله باطل شد بدین
\\
خاصه ای خواجه قیاس حس دون
&&
اندر آن وحیی که هست از حد فزون
\\
گوش حس تو به حرف ار در خورست
&&
دان که گوش غیب‌گیر تو کرست
\\
\end{longtable}
\end{center}
