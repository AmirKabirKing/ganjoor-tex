\begin{center}
\section*{بخش ۱۶۹ - بازگشتن به حکایت علی کرم الله وجهه و مسامحت کردن او با خونی خویش}
\label{sec:sh169}
\addcontentsline{toc}{section}{\nameref{sec:sh169}}
\begin{longtable}{l p{0.5cm} r}
باز رو سوی علی و خونیش
&&
وان کرم با خونی و افزونیش
\\
گفت دشمن را همی‌بینم به چشم
&&
روز و شب بر وی ندارم هیچ خشم
\\
زانک مرگم همچو من خوش آمدست
&&
مرگ من در بعث چنگ اندر زدست
\\
مرگ بی مرگی بود ما را حلال
&&
برگ بی برگی بود ما را نوال
\\
ظاهرش مرگ و به باطن زندگی
&&
ظاهرش ابتر نهان پایندگی
\\
در رحم زادن جنین را رفتنست
&&
در جهان او را ز نو بشکفتنست
\\
چون مرا سوی اجل عشق و هواست
&&
نهی لا تلقوا بایدیکم مراست
\\
زانک نهی از دانهٔ شیرین بود
&&
تلخ را خود نهی حاجت کی شود
\\
دانه‌ای کش تلخ باشد مغز و پوست
&&
تلخی و مکروهیش خود نهی اوست
\\
دانهٔ مردن مرا شیرین شدست
&&
بل هم احیاء پی من آمدست
\\
اقتلونی یا ثقاتی لائما
&&
ان فی قتلی حیاتی دائما
\\
ان فی موتی حیاتی یا فتی
&&
کم افارق موطنی حتی متی
\\
فرقتی لو لم تکن فی ذا السکون
&&
لم یقل انا الیه راجعون
\\
راجع آن باشد که باز آید به شهر
&&
سوی وحدت آید از تفریق دهر
\\
\end{longtable}
\end{center}
