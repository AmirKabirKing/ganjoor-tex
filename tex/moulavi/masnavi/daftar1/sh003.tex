\begin{center}
\section*{بخش ۳ - ظاهر شدن عجز حکیمان از معالجهٔ کنیزک و روی آوردن پادشاه به درگاه اله و در خواب دیدن او ولیی را}
\label{sec:sh003}
\addcontentsline{toc}{section}{\nameref{sec:sh003}}
\begin{longtable}{l p{0.5cm} r}
شه چو عجز آن حکیمان را بدید
&&
پا برهنه جانب مسجد دوید
\\
رفت در مسجد سوی محراب شد
&&
سجده‌گاه از اشک شه پر آب شد
\\
چون به خویش آمد ز غرقاب فنا
&&
خوش زبان بگشاد در مدح و دعا
\\
کای کمینه بخششت ملک جهان
&&
من چه گویم چون تو می‌دانی نهان
\\
ای همیشه حاجت ما را پناه
&&
بار دیگر ما غلط کردیم راه
\\
لیک گفتی گرچه می‌دانم سرت
&&
زود هم پیدا کنش بر ظاهرت
\\
چون برآورد از میان جان خروش
&&
اندر آمد بحر بخشایش به جوش
\\
درمیان گریه خوابش در ربود
&&
دید در خواب او که پیری رو نمود
\\
گفت ای شه مژده حاجاتت رواست
&&
گر غریبی آیدت فردا ز ماست
\\
چونک آید او حکیمی حاذقست
&&
صادقش دان کو امین و صادقست
\\
در علاجش سحر مطلق را ببین
&&
در مزاجش قدرت حق را ببین
\\
چون رسید آن وعده‌گاه و روز شد
&&
آفتاب از شرق اخترسوز شد
\\
بود اندر منظره شه منتظر
&&
تا ببیند آنچ بنمودند سر
\\
دید شخصی فاضلی پر مایه‌ای
&&
آفتابی درمیان سایه‌ای
\\
می‌رسید از دور مانند هلال
&&
نیست بود و هست بر شکل خیال
\\
نیست‌وش باشد خیال اندر روان
&&
تو جهانی بر خیالی بین روان
\\
بر خیالی صلحشان و جنگشان
&&
وز خیالی فخرشان و ننگشان
\\
آن خیالاتی که دام اولیاست
&&
عکس مه‌رویان بستان خداست
\\
آن خیالی که شه اندر خواب دید
&&
در رخ مهمان همی آمد پدید
\\
شه به جای حاجبان فا پیش رفت
&&
پیش آن مهمان غیب خویش رفت
\\
هر دو بحری آشنا آموخته
&&
هر دو جان بی دوختن بر دوخته
\\
گفت معشوقم تو بودستی نه آن
&&
لیک کار از کار خیزد در جهان
\\
ای مرا تو مصطفی من چو عمر
&&
از برای خدمتت بندم کمر
\\
\end{longtable}
\end{center}
