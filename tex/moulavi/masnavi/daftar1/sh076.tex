\begin{center}
\section*{بخش ۷۶ - تفسیر رجعنا من الجهاد الاصغر الی‌الجهاد الاکبر}
\label{sec:sh076}
\addcontentsline{toc}{section}{\nameref{sec:sh076}}
\begin{longtable}{l p{0.5cm} r}
ای شهان کشتیم ما خصم برون
&&
ماند خصمی زو بتر در اندرون
\\
کشتن این کار عقل و هوش نیست
&&
شیر باطن سخرهٔ خرگوش نیست
\\
دوزخست این نفس و دوزخ اژدهاست
&&
کو به دریاها نگردد کم و کاست
\\
هفت دریا را در آشامد هنوز
&&
کم نگردد سوزش آن خلق‌سوز
\\
سنگها و کافران سنگ‌دل
&&
اندر آیند اندرو زار و خجل
\\
هم نگردد ساکن از چندین غذا
&&
تا ز حق آید مرورا این ندا
\\
سیر گشتی سیر گوید نه هنوز
&&
اینت آتش اینت تابش اینت سوز
\\
عالمی را لقمه کرد و در کشید
&&
معده‌اش نعره زنان هل من مزید
\\
حق قدم بر وی نهد از لامکان
&&
آنگه او ساکن شود از کن فکان
\\
چونک جزو دوزخست این نفس ما
&&
طبع کل دارد همیشه جزوها
\\
این قدم حق را بود کو را کشد
&&
غیر حق خود کی کمان او کشد
\\
در کمان ننهند الا تیر راست
&&
این کمان را بازگون کژ تیرهاست
\\
راست شو چون تیر و واره از کمان
&&
کز کمان هر راست بجهد بی‌گمان
\\
چونک وا گشتم ز پیگار برون
&&
روی آوردم به پیگار درون
\\
قد رجعنا من جهاد الاصغریم
&&
با نبی اندر جهاد اکبریم
\\
قوت از حق خواهم و توفیق و لاف
&&
تا به سوزن بر کنم این کوه قاف
\\
سهل شیری دان که صفها بشکند
&&
شیر آنست آن که خود را بشکند
\\
\end{longtable}
\end{center}
