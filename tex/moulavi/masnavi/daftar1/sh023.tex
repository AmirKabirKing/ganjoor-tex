\begin{center}
\section*{بخش ۲۳ - در بیان آنک این اختلافات در صورت روش است نی در حقیقت راه}
\label{sec:sh023}
\addcontentsline{toc}{section}{\nameref{sec:sh023}}
\begin{longtable}{l p{0.5cm} r}
او ز یک رنگی عیسی بو نداشت
&&
وز مزاج خم عیسی خو نداشت
\\
جامهٔ صد رنگ از آن خم صفا
&&
ساده و یک‌رنگ گشتی چون صبا
\\
نیست یک‌رنگی کزو خیزد ملال
&&
بل مثال ماهی و آب زلال
\\
گرچه در خشکی هزاران رنگهاست
&&
ماهیان را با یبوست جنگهاست
\\
کیست ماهی چیست دریا در مثل
&&
تا بدان ماند ملک عز و جل
\\
صد هزاران بحر و ماهی در وجود
&&
سجده آرد پیش آن اکرام و جود
\\
چند باران عطا باران شده
&&
تا بدان آن بحر در افشان شده
\\
چند خورشید کرم افروخته
&&
تا که ابر و بحر جود آموخته
\\
پرتو دانش زده بر خاک و طین
&&
تا که شد دانه پذیرنده زمین
\\
خاک امین و هر چه در وی کاشتی
&&
بی‌خیانت جنس آن برداشتی
\\
این امانت زان امانت یافتست
&&
کآفتاب عدل بر وی تافتست
\\
تا نشان حق نیارد نوبهار
&&
خاک سرها را نکرده آشکار
\\
آن جوادی که جمادی را بداد
&&
این خبرها وین امانت وین سداد
\\
مر جمادی را کند فضلش خبیر
&&
عاقلان را کرده قهر او ضریر
\\
جان و دل را طاقت آن جوش نیست
&&
با که گویم در جهان یک گوش نیست
\\
هر کجا گوشی بد از وی چشم گشت
&&
هر کجا سنگی بد از وی یشم گشت
\\
کیمیاسازست چه بود کیمیا
&&
معجزه بخش است چه بود سیمیا
\\
این ثنا گفتن ز من ترک ثناست
&&
کین دلیل هستی و هستی خطاست
\\
پیش هست او بباید نیست بود
&&
چیست هستی پیش او کور و کبود
\\
گر نبودی کور زو بگداختی
&&
گرمی خورشید را بشناختی
\\
ور نبودی او کبود از تعزیت
&&
کی فسردی همچو یخ این ناحیت
\\
\end{longtable}
\end{center}
