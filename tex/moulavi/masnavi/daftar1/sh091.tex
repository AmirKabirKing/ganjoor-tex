\begin{center}
\section*{بخش ۹۱ - تفسیر قول حکیم بهرچ از راه و امانی چه کفر آن حرف و چه ایمان بهرچ از دوست دورافتی چه زشت آن نقش و چه زیبا در معنی قوله علیه‌السلام ان سعدا لغیور و انا اغیر من سعد و الله اغیر منی و من غیر ته حرم الفواحش ما ظهر منها و ما بطن}
\label{sec:sh091}
\addcontentsline{toc}{section}{\nameref{sec:sh091}}
\begin{longtable}{l p{0.5cm} r}
جمله عالم زان غیور آمد که حق
&&
برد در غیرت برین عالم سبق
\\
او چو جانست و جهان چون کالبد
&&
کالبد از جان پذیرد نیک و بد
\\
هر که محراب نمازش گشت عین
&&
سوی ایمان رفتنش می‌دان تو شین
\\
هر که شد مر شاه را او جامه‌دار
&&
هست خسران بهر شاهش اتجار
\\
هر که با سلطان شود او همنشین
&&
بر درش شستن بود حیف و غبین
\\
دستبوسش چون رسید از پادشاه
&&
گر گزیند بوس پا باشد گناه
\\
گرچه سر بر پا نهادن خدمتست
&&
پیش آن خدمت خطا و زلتست
\\
شاه را غیرت بود بر هر که او
&&
بو گزیند بعد از آن که دید رو
\\
غیرت حق بر مثل گندم بود
&&
کاه خرمن غیرت مردم بود
\\
اصل غیرتها بدانید از اله
&&
آن خلقان فرع حق بی‌اشتباه
\\
شرح این بگذارم و گیرم گله
&&
از جفای آن نگار ده دله
\\
نالم ایرا ناله‌ها خوش آیدش
&&
از دو عالم ناله و غم بایدش
\\
چون ننالم تلخ از دستان او
&&
چون نیم در حلقهٔ مستان او
\\
چون نباشم همچو شب بی روز او
&&
بی وصال روی روز افروز او
\\
ناخوش او خوش بود در جان من
&&
جان فدای یار دل‌رنجان من
\\
عاشقم بر رنج خویش و درد خویش
&&
بهر خشنودی شاه فرد خویش
\\
خاک غم را سرمه سازم بهر چشم
&&
تا ز گوهر پر شود دو بحر چشم
\\
اشک کان از بهر او بارند خلق
&&
گوهرست و اشک پندارند خلق
\\
من ز جان جان شکایت می‌کنم
&&
من نیم شاکی روایت می‌کنم
\\
دل همی‌گوید کزو رنجیده‌ام
&&
وز نفاق سست می‌خندیده‌ام
\\
راستی کن ای تو فخر راستان
&&
ای تو صدر و من درت را آستان
\\
آستانه و صدر در معنی کجاست
&&
ما و من کو آن طرف کان یار ماست
\\
ای رهیده جان تو از ما و من
&&
ای لطیفهٔ روح اندر مرد و زن
\\
مرد و زن چون یک شود آن یک توی
&&
چونک یکها محو شد انک توی
\\
این من و ما بهر آن بر ساختی
&&
تا تو با خود نرد خدمت باختی
\\
تا من و توها همه یک جان شوند
&&
عاقبت مستغرق جانان شوند
\\
این همه هست و بیا ای امر کن
&&
ای منزه از بیا و از سخن
\\
جسم جسمانه تواند دیدنت
&&
در خیال آرد غم و خندیدنت
\\
دل که او بستهٔ غم و خندیدنست
&&
تو مگو کو لایق آن دیدنست
\\
آنک او بستهٔ غم و خنده بود
&&
او بدین دو عاریت زنده بود
\\
باغ سبز عشق کو بی منتهاست
&&
جز غم و شادی درو بس میوه‌هاست
\\
عاشقی زین هر دو حالت برترست
&&
بی بهار و بی خزان سبز و ترست
\\
ده زکات روی خوب ای خوب‌رو
&&
شرح جان شرحه شرحه بازگو
\\
کز کرشم غمزه‌ای غمازه‌ای
&&
بر دلم بنهاد داغی تازه‌ای
\\
من حلالش کردم ار خونم بریخت
&&
من همی‌گفتم حلال او می‌گریخت
\\
چون گریزانی ز نالهٔ خاکیان
&&
غم چه ریزی بر دل غمناکیان
\\
ای که هر صبحی که از مشرق بتافت
&&
همچو چشمهٔ مشرقت در جوش یافت
\\
چون بهانه دادی این شیدات را
&&
ای بها نه شکر لبهات را
\\
ای جهان کهنه را تو جان نو
&&
از تن بی جان و دل افغان شنو
\\
شرح گل بگذار از بهر خدا
&&
شرح بلبل گو که شد از گل جدا
\\
از غم و شادی نباشد جوش ما
&&
با خیال و وهم نبود هوش ما
\\
حالتی دیگر بود کان نادرست
&&
تو مشو منکر که حق بس قادرست
\\
تو قیاس از حالت انسان مکن
&&
منزل اندر جور و در احسان مکن
\\
جور و احسان رنج و شادی حادثست
&&
حادثان میرند و حقشان وارثست
\\
صبح شد ای صبح را صبح و پناه
&&
عذر مخدومی حسام‌الدین بخواه
\\
عذرخواه عقل کل و جان توی
&&
جان جان و تابش مرجان توی
\\
تافت نور صبح و ما از نور تو
&&
در صبوحی با می منصور تو
\\
دادهٔ تو چون چنین دارد مرا
&&
باده کی بود کو طرب آرد مرا
\\
باده در جوشش گدای جوش ماست
&&
چرخ در گردش گدای هوش ماست
\\
باده از ما مست شد نه ما ازو
&&
قالب از ما هست شد نه ما ازو
\\
ما چو زنبوریم و قالبها چو موم
&&
خانه خانه کرده قالب را چو موم
\\
\end{longtable}
\end{center}
