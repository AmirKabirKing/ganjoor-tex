\begin{center}
\section*{بخش ۹۸ - در بیان این حدیث کی ان لربکم فی ایام دهرکم نفحات الا فتعر ضوا لها}
\label{sec:sh098}
\addcontentsline{toc}{section}{\nameref{sec:sh098}}
\begin{longtable}{l p{0.5cm} r}
گفت پیغامبر که نفحتهای حق
&&
اندرین ایام می‌آرد سبق
\\
گوش و هش دارید این اوقات را
&&
در ربایید این چنین نفحات را
\\
نفحه آمد مر شما را دید و رفت
&&
هر که را می‌خواست جان بخشید و رفت
\\
نفحهٔ دیگر رسید آگاه باش
&&
تا ازین هم وانمانی خواجه‌تاش
\\
جان آتش یافت زو آتش کشی
&&
جان مرده یافت از وی جنبشی
\\
جان ناری یافت از وی انطفا
&&
مرده پوشید از بقای او قبا
\\
تازگی و جنبش طوبیست این
&&
همچو جنبشهای حیوان نیست این
\\
گر در افتد در زمین و آسمان
&&
زهره‌هاشان آب گردد در زمان
\\
خود ز بیم این دم بی‌منتها
&&
باز خوان فابین ان یحملنها
\\
ورنه خود اشفقن منها چون بدی
&&
گرنه از بیمش دل که خون شدی
\\
دوش دیگر لون این می‌داد دست
&&
لقمهٔ چندی درآمد ره ببست
\\
بهر لقمه گشته لقمانی گرو
&&
وقت لقمانست ای لقمه برو
\\
از هوای لقمهٔ این خارخار
&&
از کف لقمان همی جویید خار
\\
در کف او خار و سایه‌ش نیز نیست
&&
لیکتان از حرص آن تمییز نیست
\\
خار دان آن را که خرما دیده‌ای
&&
زانک بس نان کور و بس نادیده‌ای
\\
جان لقمان که گلستان خداست
&&
پای جانش خستهٔ خاری چراست
\\
اشتر آمد این وجود خارخوار
&&
مصطفی‌زادی برین اشتر سوار
\\
اشترا تنگ گلی بر پشت تست
&&
کز نسیمش در تو صد گلزار رست
\\
میل تو سوی مغیلانست و ریگ
&&
تا چه گل چینی ز خار مردریگ
\\
ای بگشته زین طلب از کو بکو
&&
چند گویی کین گلستان کو و کو
\\
پیش از آن کین خار پا بیرون کنی
&&
چشم تاریکست جولان چون کنی
\\
آدمی کو می‌نگنجد در جهان
&&
در سر خاری همی گردد نهان
\\
مصطفی آمد که سازد همدمی
&&
کلمینی یا حمیرا کلمی
\\
ای حمیرا اندر آتش نه تو نعل
&&
تا ز نعل تو شود این کوه لعل
\\
این حمیرا لفظ تانیشست و جان
&&
نام تانیثش نهند این تازیان
\\
لیک از تانیث جان را باک نیست
&&
روح را با مرد و زن اشراک نیست
\\
از مؤنث وز مذکر برترست
&&
این نی آن جانست کز خشک و ترست
\\
این نه آن جانست کافزاید ز نان
&&
یا گهی باشد چنین گاهی چنان
\\
خوش کننده‌ست و خوش و عین خوشی
&&
بی خوشی نبود خوشی ای مرتشی
\\
چون تو شیرین از شکر باشی بود
&&
کان شکر گاهی ز تو غایب شود
\\
چون شکر گردی ز تاثیر وفا
&&
پس شکر کی از شکر باشد جدا
\\
عاشق از خود چون غذا یابد رحیق
&&
عقل آنجا گم شود گم ای رفیق
\\
عقل جزوی عشق را منکر بود
&&
گرچه بنماید که صاحب‌سر بود
\\
زیرک و داناست اما نیست نیست
&&
تا فرشته لا نشد آهرمنیست
\\
او بقول و فعل یار ما بود
&&
چون بحکم حال آیی لا بود
\\
لا بود چون او نشد از هست نیست
&&
چونک طوعا لا نشد کرها بسیست
\\
جان کمالست و ندای او کمال
&&
مصطفی گویان ارحنا یا بلال
\\
ای بلال افراز بانگ سلسلت
&&
زان دمی کاندر دمیدم در دلت
\\
زان دمی کادم از آن مدهوش گشت
&&
هوش اهل آسمان بیهوش گشت
\\
مصطفی بی‌خویش شد زان خوب صوت
&&
شد نمازش از شب تعریس فوت
\\
سر از آن خواب مبارک بر نداشت
&&
تا نماز صبحدم آمد بچاشت
\\
در شب تعریس پیش آن عروس
&&
یافت جان پاک ایشان دستبوس
\\
عشق و جان هر دو نهانند و ستیر
&&
گر عروسش خوانده‌ام عیبی مگیر
\\
از ملولی یار خامش کردمی
&&
گر همو مهلت بدادی یکدمی
\\
لیک می‌گوید بگو هین عیب نیست
&&
جز تقاضای قضای غیب نیست
\\
عیب باشد کو نبیند جز که عیب
&&
عیب کی بیند روان پاک غیب
\\
عیب شد نسبت به مخلوق جهول
&&
نی به نسبت با خداوند قبول
\\
کفر هم نسبت به خالق حکمتست
&&
چون به ما نسبت کنی کفر آفتست
\\
ور یکی عیبی بود با صد حیات
&&
بر مثال چوب باشد در نبات
\\
در ترازو هر دو را یکسان کشند
&&
زانک آن هر دو چو جسم و جان خوشند
\\
پس بزرگان این نگفتند از گزاف
&&
جسم پاکان عین جان افتاد صاف
\\
گفتشان و نفسشان و نقششان
&&
جمله جان مطلق آمد بی نشان
\\
جان دشمن‌دارشان جسمست صرف
&&
چون زیاد از نرد او اسمست صرف
\\
آن به خاک اندر شد و کل خاک شد
&&
وین نمک اندر شد و کل پاک شد
\\
آن نمک کز وی محمد املحست
&&
زان حدیث با نمک او افصحست
\\
این نمک باقیست از میراث او
&&
با توند آن وارثان او بجو
\\
پیش تو شسته ترا خود پیش کو
&&
پیش هستت جان پیش‌اندیش کو
\\
گر تو خود را پیش و پس داری گمان
&&
بستهٔ جسمی و محرومی ز جان
\\
زیر و بالا پیش و پس وصف تنست
&&
بی‌جهتها ذات جان روشنست
\\
برگشا از نور پاک شه نظر
&&
تا نپنداری تو چون کوته‌نظر
\\
که همینی در غم و شادی و بس
&&
ای عدم کو مر عدم را پیش و پس
\\
روز بارانست می‌رو تا به شب
&&
نه ازین باران از آن باران رب
\\
\end{longtable}
\end{center}
