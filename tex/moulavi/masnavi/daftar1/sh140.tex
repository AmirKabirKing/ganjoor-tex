\begin{center}
\section*{بخش ۱۴۰ - وصیت کردن رسول صلی الله علیه و سلم مر علی را کرم الله وجهه کی چون هر کسی به نوع طاعتی تقرب جوید به حق تو تقرب جوی به صحبت عاقل و بندهٔ خاص تا ازیشان همه پیش‌قدم‌تر باشی}
\label{sec:sh140}
\addcontentsline{toc}{section}{\nameref{sec:sh140}}
\begin{longtable}{l p{0.5cm} r}
گفت پیغامبر علی را کای علی
&&
شیر حقی پهلوان پردلی
\\
لیک بر شیری مکن هم اعتماد
&&
اندر آ در سایهٔ نخل امید
\\
اندر آ در سایهٔ آن عاقلی
&&
کش نداند برد از ره ناقلی
\\
ظل او اندر زمین چون کوه قاف
&&
روح او سیمرغ بس عالی‌طواف
\\
گر بگویم تا قیامت نعت او
&&
هیچ آن را مقطع و غایت مجو
\\
در بشر روپوش کردست آفتاب
&&
فهم کن والله اعلم بالصواب
\\
یا علی از جملهٔ طاعات راه
&&
بر گزین تو سایهٔ خاص اله
\\
هر کسی در طاعتی بگریختند
&&
خویشتن را مخلصی انگیختند
\\
تو برو در سایهٔ عاقل گریز
&&
تا رهی زان دشمن پنهان‌ستیز
\\
از همه طاعات اینت بهترست
&&
سبق یابی بر هر آن سابق که هست
\\
چون گرفتت پیر هین تسلیم شو
&&
همچو موسی زیر حکم خضر رو
\\
صبر کن بر کار خضری بی نفاق
&&
تا نگوید خضر رو هذا فراق
\\
گرچه کشتی بشکند تو دم مزن
&&
گرچه طفلی را کشد تو مو مکن
\\
دست او را حق چو دست خویش خواند
&&
تا ید الله فوق ایدیهم براند
\\
دست حق میراندش زنده‌ش کند
&&
زنده چه بود جان پاینده‌ش کند
\\
هرکه تنها نادرا این ره برید
&&
هم به عون همت پیران رسید
\\
دست پیر از غایبان کوتاه نیست
&&
دست او جز قبضه الله نیست
\\
غایبان را چون چنین خلعت دهند
&&
حاضران از غایبان لا شک به‌اند
\\
غایبان را چون نواله می‌دهند
&&
پیش مهمان تا چه نعمتها نهند
\\
کو کسی کو پیش شه بندد کمر
&&
تا کسی کو هست بیرون سوی در
\\
چون گزیدی پیر نازک‌دل مباش
&&
سست و ریزیده چو آب و گل مباش
\\
ور بهر زخمی تو پر کینه شوی
&&
پس کجا بی‌صیقل آیینه شوی
\\
\end{longtable}
\end{center}
