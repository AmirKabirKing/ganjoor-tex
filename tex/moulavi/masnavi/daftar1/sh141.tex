\begin{center}
\section*{بخش ۱۴۱ - کبودی زدن قزوینی بر شانه‌گاه صورت شیر و پشیمان شدن او به سبب زخم سوزن}
\label{sec:sh141}
\addcontentsline{toc}{section}{\nameref{sec:sh141}}
\begin{longtable}{l p{0.5cm} r}
این حکایت بشنو از صاحب بیان
&&
در طریق و عادت قزوینیان
\\
بر تن و دست و کتفها بی‌گزند
&&
از سر سوزن کبودیها زنند
\\
سوی دلاکی بشد قزوینیی
&&
که کبودم زن بکن شیرینیی
\\
گفت چه صورت زنم ای پهلوان
&&
گفت بر زن صورت شیر ژیان
\\
طالعم شیرست نقش شیر زن
&&
جهد کن رنگ کبودی سیر زن
\\
گفت بر چه موضعت صورت زنم
&&
گفت بر شانه گهم زن آن رقم
\\
چونک او سوزن فرو بردن گرفت
&&
درد آن در شانه‌گه مسکن گرفت
\\
پهلوان در ناله آمد کای سنی
&&
مر مرا کشتی چه صورت می‌زنی
\\
گفت آخر شیر فرمودی مرا
&&
گفت از چه عضو کردی ابتدا
\\
گفت از دمگاه آغازیده‌ام
&&
گفت دم بگذار ای دو دیده‌ام
\\
از دم و دمگاه شیرم دم گرفت
&&
دمگه او دمگهم محکم گرفت
\\
شیر بی‌دم باش گو ای شیرساز
&&
که دلم سستی گرفت از زخم گاز
\\
جانب دیگر گرفت آن شخص زخم
&&
بی‌محابا و مواسایی و رحم
\\
بانگ کرد او کین چه اندامست ازو
&&
گفت این گوشست ای مرد نکو
\\
گفت تا گوشش نباشد ای حکیم
&&
گوش را بگذار و کوته کن گلیم
\\
جانب دیگر خلش آغاز کرد
&&
باز قزوینی فغان را ساز کرد
\\
کین سوم جانب چه اندامست نیز
&&
گفت اینست اشکم شیر ای عزیز
\\
گفت تا اشکم نباشد شیر را
&&
گشت افزون درد کم زن زخمها
\\
خیره شد دلاک و پس حیران بماند
&&
تا بدیر انگشت در دندان بماند
\\
بر زمین زد سوزن از خشم اوستاد
&&
گفت در عالم کسی را این فتاد
\\
شیر بی‌دم و سر و اشکم کی دید
&&
این‌چنین شیری خدا خود نافرید
\\
ای برادر صبر کن بر درد نیش
&&
تا رهی از نیش نفس گبر خویش
\\
کان گروهی که رهیدند از وجود
&&
چرخ و مهر و ماهشان آرد سجود
\\
هر که مرد اندر تن او نفس گبر
&&
مر ورا فرمان برد خورشید و ابر
\\
چون دلش آموخت شمع افروختن
&&
آفتاب او را نیارد سوختن
\\
گفت حق در آفتاب منتجم
&&
ذکر تزاور کذی عن کهفهم
\\
خار جمله لطف چون گل می‌شود
&&
پیش جزوی کو سوی کل می‌رود
\\
چیست تعظیم خدا افراشتن
&&
خویشتن را خوار و خاکی داشتن
\\
چیست توحید خدا آموختن
&&
خویشتن را پیش واحد سوختن
\\
گر همی‌خواهی که بفروزی چو روز
&&
هستی همچون شب خود را بسوز
\\
هستیت در هست آن هستی‌نواز
&&
همچو مس در کیمیا اندر گداز
\\
در من و ما سخت کردستی دو دست
&&
هست این جمله خرابی از دو هست
\\
\end{longtable}
\end{center}
