\begin{center}
\section*{بخش ۱۶۵ - سال کردن آن کافر از علی کرم الله وجهه کی بر چون منی مظفر شدی شمشیر از دست چون انداختی}
\label{sec:sh165}
\addcontentsline{toc}{section}{\nameref{sec:sh165}}
\begin{longtable}{l p{0.5cm} r}
پس بگفت آن نو مسلمان ولی
&&
از سر مستی و لذت با علی
\\
که بفرما یا امیر المؤمنین
&&
تا بجنبد جان بتن در چون جنین
\\
هفت اختر هر جنین را مدتی
&&
می‌کنند ای جان به نوبت خدمتی
\\
چونک وقت آید که جان گیرد جنین
&&
آفتابش آن زمان گردد معین
\\
این جنین در جنبش آید ز آفتاب
&&
کآفتابش جان همی‌بخشد شتاب
\\
از دگر انجم به جز نقشی نیافت
&&
این جنین تا آفتابش بر نتافت
\\
از کدامین ره تعلق یافت او
&&
در رحم با آفتاب خوب‌رو
\\
از ره پنهان که دور از حس ماست
&&
آفتاب چرخ را بس راههاست
\\
آن رهی که زر بیابد قوت ازو
&&
و آن رهی که سنگ شد یاقوت ازو
\\
آن رهی که سرخ سازد لعل را
&&
وان رهی که برق بخشد نعل را
\\
آن رهی که پخته سازد میوه را
&&
و آن رهی که دل دهد کالیوه را
\\
بازگو ای باز پر افروخته
&&
با شه و با ساعدش آموخته
\\
باز گو ای بار عنقاگیر شاه
&&
ای سپاه‌اشکن بخود نه با سپاه
\\
امت وحدی یکی و صد هزار
&&
بازگو ای بنده بازت را شکار
\\
در محل قهر این رحمت ز چیست
&&
اژدها را دست دادن راه کیست
\\
\end{longtable}
\end{center}
