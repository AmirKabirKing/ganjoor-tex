\begin{center}
\section*{بخش ۱۲۲ - سبب حرمان اشقیا از دو جهان کی خسر الدنیا و اخرة}
\label{sec:sh122}
\addcontentsline{toc}{section}{\nameref{sec:sh122}}
\begin{longtable}{l p{0.5cm} r}
چون حکیمک اعتقادی کرده است
&&
کآسمان بیضه زمین چون زرده است
\\
گفت سایل چون بماند این خاکدان
&&
در میان این محیط آسمان
\\
همچو قندیلی معلق در هوا
&&
نه باسفل می‌رود نه بر علا
\\
آن حکیمش گفت کز جذب سما
&&
از جهات شش بماند اندر هوا
\\
چون ز مغناطیس قبهٔ ریخته
&&
درمیان ماند آهنی آویخته
\\
آن دگر گفت آسمان با صفا
&&
کی کشد در خود زمین تیره را
\\
بلک دفعش می‌کند از شش جهات
&&
زان بماند اندر میان عاصفات
\\
پس ز دفع خاطر اهل کمال
&&
جان فرعونان بماند اندر ضلال
\\
پس ز دفع این جهان و آن جهان
&&
مانده‌اند این بی‌رهان بی این و آن
\\
سر کشی از بندگان ذوالجلال
&&
دان که دارند از وجود تو ملال
\\
کهربا دارند چون پیدا کنند
&&
کاه هستی ترا شیدا کنند
\\
کهربای خویش چون پنهان کنند
&&
زود تسلیم ترا طغیان کنند
\\
آنچنان که مرتبهٔ حیوانیست
&&
کو اسیر و سغبهٔ انسانیست
\\
مرتبهٔ انسان به دست اولیا
&&
سغبه چون حیوان شناسش ای کیا
\\
بندهٔ خود خواند احمد در رشاد
&&
جمله عالم را بخوان قل یا عباد
\\
عقل تو همچون شتربان تو شتر
&&
می‌کشاند هر طرف در حکم مر
\\
عقل عقلند اولیا و عقلها
&&
بر مثال اشتران تا انتها
\\
اندریشان بنگر آخر ز اعتبار
&&
یک قلاووزست جان صد هزار
\\
چه قلاووز و چه اشتربان بیاب
&&
دیده‌ای کان دیده بیند آفتاب
\\
یک جهان در شب بمانده میخ‌دوز
&&
منتظر موقوف خورشیدست و روز
\\
اینت خورشیدی نهان در ذره‌ای
&&
شیر نر در پوستین بره‌ای
\\
اینت دریایی نهان در زیر کاه
&&
پا برین که هین منه با اشتباه
\\
اشتباهی و گمانی را درون
&&
رحمت حقست بهر رهنمون
\\
هر پیمبر فرد آمد در جهان
&&
فرد بود آن رهنمایش در نهان
\\
عالک کبری بقدرت سحر کرد
&&
کرد خود را در کهین نقشی نورد
\\
ابلهانش فرد دیدند و ضعیف
&&
کی ضعیفست آن که با شه شد حریف
\\
ابلهان گفتند مردی بیش نیست
&&
وای آنکو عاقبت‌اندیش نیست
\\
\end{longtable}
\end{center}
