\begin{center}
\section*{بخش ۳۷ - آتش کردن پادشاه جهود و بت نهادن پهلوی آتش کی هر که این بت را سجود کند از آتش برست}
\label{sec:sh037}
\addcontentsline{toc}{section}{\nameref{sec:sh037}}
\begin{longtable}{l p{0.5cm} r}
آن جهود سگ ببین چه رای کرد
&&
پهلوی آتش بتی بر پای کرد
\\
کانک این بت را سجود آرد برست
&&
ور نیارد در دل آتش نشست
\\
چون سزای این بت نفس او نداد
&&
از بت نفسش بتی دیگر بزاد
\\
مادر بتها بت نفس شماست
&&
زانک آن بت مار و این بت اژدهاست
\\
آهن و سنگست نفس و بت شرار
&&
آن شرار از آب می‌گیرد قرار
\\
سنگ و آهن زآب کی ساکن شود
&&
آدمی با این دو کی ایمن بود
\\
بت سیاهابه‌ست در کوزه نهان
&&
نفس مر آب سیه را چشمه دان
\\
آن بت منحوت چون سیل سیاه
&&
نفس بتگر چشمه‌ای بر آب راه
\\
صد سبو را بشکند یکپاره سنگ
&&
و آب چشمه می‌زهاند بی‌درنگ
\\
بت‌شکستن سهل باشد نیک سهل
&&
سهل دیدن نفس را جهلست جهل
\\
صورت نفس ار بجویی ای پسر
&&
قصهٔ دوزخ بخوان با هفت در
\\
هر نفس مکری و در هر مکر زان
&&
غرقه صد فرعون با فرعونیان
\\
در خدای موسی و موسی گریز
&&
آب ایمان را ز فرعونی مریز
\\
دست را اندر احد و احمد بزن
&&
ای برادر وا ره از بوجهل تن
\\
\end{longtable}
\end{center}
