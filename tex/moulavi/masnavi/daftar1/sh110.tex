\begin{center}
\section*{بخش ۱۱۰ - قصهٔ خلیفه کی در کرم در زمان خود از حاتم طائی گذشته بود و نظیر خود نداشت}
\label{sec:sh110}
\addcontentsline{toc}{section}{\nameref{sec:sh110}}
\begin{longtable}{l p{0.5cm} r}
یک خلیفه بود در ایام پیش
&&
کرده حاتم را غلام جود خویش
\\
رایت اکرام و داد افراشته
&&
فقر و حاجت از جهان بر داشته
\\
بحر و در از بخششش صاف آمده
&&
داد او از قاف تا قاف آمده
\\
در جهان خاک ابر و آب بود
&&
مظهر بخشایش وهاب بود
\\
از عطااش بحر و کان در زلزله
&&
سوی جودش قافله بر قافله
\\
قبلهٔ حاجت در و دروازه‌اش
&&
رفته در عالم بجود آوازه‌اش
\\
هم عجم هم روم هم ترک و عرب
&&
مانده از جود و سخااش در عجب
\\
آب حیوان بود و دریای کرم
&&
زنده گشته هم عرب زو هم عجم
\\
\end{longtable}
\end{center}
