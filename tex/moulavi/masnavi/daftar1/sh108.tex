\begin{center}
\section*{بخش ۱۰۸ - گردانیدن عمر رضی الله عنه نظر او را از مقام گریه کی هستیست بمقام استغراق}
\label{sec:sh108}
\addcontentsline{toc}{section}{\nameref{sec:sh108}}
\begin{longtable}{l p{0.5cm} r}
پس عمر گفتش که این زاری تو
&&
هست هم آثار هشیاری تو
\\
راه فانی گشته راهی دیگرست
&&
زانک هشیاری گناهی دیگرست
\\
هست هشیاری ز یاد ما مضی
&&
ماضی و مستقبلت پردهٔ خدا
\\
آتش اندر زن بهر دو تا بکی
&&
پر گره باشی ازین هر دو چو نی
\\
تا گره با نی بود همراز نیست
&&
همنشین آن لب و آواز نیست
\\
چون بطوفی خود بطوفی مرتدی
&&
چون به خانه آمدی هم با خودی
\\
ای خبرهات از خبرده بی‌خبر
&&
توبهٔ تو از گناه تو بتر
\\
ای تو از حال گذشته توبه‌جو
&&
کی کنی توبه ازین توبه بگو
\\
گاه بانگ زیر را قبله کنی
&&
گاه گریهٔ زار را قبله زنی
\\
چونک فاروق آینهٔ اسرار شد
&&
جان پیر از اندرون بیدار شد
\\
همچو جان بی‌گریه و بی‌خنده شد
&&
جانش رفت و جان دیگر زنده شد
\\
حیرتی آمد درونش آن زمان
&&
که برون شد از زمین و آسمان
\\
جست و جویی از ورای جست و جو
&&
من نمی‌دانم تو می‌دانی بگو
\\
حال و قالی از ورای حال و قال
&&
غرقه گشته در جمال ذوالجلال
\\
غرقه‌ای نه که خلاصی باشدش
&&
یا به جز دریا کسی بشناسدش
\\
عقل جزو از کل گویا نیستی
&&
گر تقاضا بر تقاضا نیستی
\\
چون تقاضا بر تقاضا می‌رسد
&&
موج آن دریا بدینجا می‌رسد
\\
چونک قصهٔ حال پیر اینجا رسید
&&
پیر و حالش روی در پرده کشید
\\
پیر دامن را ز گفت و گو فشاند
&&
نیم گفته در دهان ما بماند
\\
از پی این عیش و عشرت ساختن
&&
صد هزاران جان بشاید باختن
\\
در شکار بیشهٔ جان باز باش
&&
همچو خورشید جهان جان‌باز باش
\\
جان‌فشان افتاد خورشید بلند
&&
هر دمی تی می‌شود پر می‌کنند
\\
جان فشان ای آفتاب معنوی
&&
مر جهان کهنه را بنما نوی
\\
در وجود آدمی جان و روان
&&
می‌رسد از غیب چون آب روان
\\
\end{longtable}
\end{center}
