\begin{center}
\section*{بخش ۹۲ - رجوع به حکایت خواجهٔ تاجر}
\label{sec:sh092}
\addcontentsline{toc}{section}{\nameref{sec:sh092}}
\begin{longtable}{l p{0.5cm} r}
بس دراز است این حدیث خواجه گو
&&
تا چه شد احوال آن مرد نکو
\\
خواجه اندر آتش و درد و حنین
&&
صد پراکنده همی‌گفت این چنین
\\
گه تناقض گاه ناز و گه نیاز
&&
گاه سودای حقیقت گه مجاز
\\
مرد غرقه گشته جانی می‌کند
&&
دست را در هر گیاهی می‌زند
\\
تا کدامش دست گیرد در خطر
&&
دست و پایی می‌زند از بیم سر
\\
دوست دارد یار این آشفتگی
&&
کوشش بیهوده به از خفتگی
\\
آنک او شاهست او بی کار نیست
&&
ناله از وی طرفه کو بیمار نیست
\\
بهر این فرمود رحمان ای پسر
&&
کل یوم هو فی شان ای پسر
\\
اندرین ره می‌تراش و می‌خراش
&&
تا دم آخر دمی فارغ مباش
\\
تا دم آخر دمی آخر بود
&&
که عنایت با تو صاحب‌سر بود
\\
هر چه می‌کوشند اگر مرد و زنست
&&
گوش و چشم شاه جان بر روزنست
\\
\end{longtable}
\end{center}
