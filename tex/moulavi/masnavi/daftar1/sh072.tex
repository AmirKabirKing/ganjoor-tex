\begin{center}
\section*{بخش ۷۲ - نظر کردن شیر در چاه و دیدن عکس خود را و آن خرگوش را}
\label{sec:sh072}
\addcontentsline{toc}{section}{\nameref{sec:sh072}}
\begin{longtable}{l p{0.5cm} r}
چونک شیر اندر بر خویشش کشید
&&
در پناه شیر تا چه می‌دوید
\\
چونک در چه بنگریدند اندر آب
&&
اندر آب از شیر و او در تافت تاب
\\
شیر عکس خویش دید از آب تفت
&&
شکل شیری در برش خرگوش زفت
\\
چونک خصم خویش را در آب دید
&&
مر ورا بگذاشت و اندر چه جهید
\\
در فتاد اندر چهی کو کنده بود
&&
زانک ظلمش در سرش آینده بود
\\
چاه مظلم گشت ظلم ظالمان
&&
این چنین گفتند جملهٔ عالمان
\\
هر که ظالم‌تر چهش با هول‌تر
&&
عدل فرمودست بتر را بتر
\\
ای که تو از جاه ظلمی می‌کنی
&&
دانک بهر خویش چاهی می‌کنی
\\
گرد خود چون کرم پیله بر متن
&&
بهر خود چه می‌کنی اندازه کن
\\
مر ضعیفان را تو بی‌خصمی مدان
&&
از نبی ذا جاء نصر الله خوان
\\
گر تو پیلی خصم تو از تو رمید
&&
نک جزا طیرا ابابیلت رسید
\\
گر ضعیفی در زمین خواهد امان
&&
غلغل افتد در سپاه آسمان
\\
گر بدندانش گزی پر خون کنی
&&
درد دندانت بگیرد چون کنی
\\
شیر خود را دید در چه وز غلو
&&
خویش را نشناخت آن دم از عدو
\\
عکس خود را او عدو خویش دید
&&
لاجرم بر خویش شمشیری کشید
\\
ای بسا ظلمی که بینی در کسان
&&
خوی تو باشد دریشان ای فلان
\\
اندریشان تافته هستی تو
&&
از نفاق و ظلم و بد مستی تو
\\
آن توی و آن زخم بر خود می‌زنی
&&
بر خود آن دم تار لعنت می‌تنی
\\
در خود آن بد را نمی‌بینی عیان
&&
ورنه دشمن بودیی خود را بجان
\\
حمله بر خود می‌کنی ای ساده مرد
&&
همچو آن شیری که بر خود حمله کرد
\\
چون به قعر خوی خود اندر رسی
&&
پس بدانی کز تو بود آن ناکسی
\\
شیر را در قعر پیدا شد که بود
&&
نقش او آنکش دگر کس می‌نمود
\\
هر که دندان ضعیفی می‌کند
&&
کار آن شیر غلط‌بین می‌کند
\\
می‌ببیند خال بد بر روی عم
&&
عکس خال تست آن از عم مرم
\\
مؤمنان آیینهٔ همدیگرند
&&
این خبر می از پیمبر آورند
\\
پیش چشمت داشتی شیشهٔ کبود
&&
زان سبب عالم کبودت می‌نمود
\\
گر نه کوری این کبودی دان ز خویش
&&
خویش را بد گو مگو کس را تو بیش
\\
مؤمن ار ینظر بنور الله نبود
&&
غیب مؤمن را برهنه چون نمود
\\
چون که تو ینظر بنار الله بدی
&&
در بدی از نیکوی غافل شدی
\\
اندک اندک آب بر آتش بزن
&&
تا شود نار تو نور ای بوالحزن
\\
تو بزن یا ربنا آب طهور
&&
تا شود این نار عالم جمله نور
\\
آب دریا جمله در فرمان تست
&&
آب و آتش ای خداوند آن تست
\\
گر تو خواهی آتش آب خوش شود
&&
ور نخواهی آب هم آتش شود
\\
این طلب در ما هم از ایجاد تست
&&
رستن از بیداد یا رب داد تست
\\
بی‌طلب تو این طلب‌مان داده‌ای
&&
گنج احسان بر همه بگشاده‌ای
\\
\end{longtable}
\end{center}
