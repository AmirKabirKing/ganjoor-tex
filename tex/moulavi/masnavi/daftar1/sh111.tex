\begin{center}
\section*{بخش ۱۱۱ - قصهٔ اعرابی درویش و ماجرای زن با او به سبب قلت و درویشی}
\label{sec:sh111}
\addcontentsline{toc}{section}{\nameref{sec:sh111}}
\begin{longtable}{l p{0.5cm} r}
یک شب اعرابی زنی مر شوی را
&&
گفت و از حد برد گفت و گوی را
\\
کین همه فقر و جفا ما می‌کشیم
&&
جمله عالم در خوشی ما ناخوشیم
\\
نان‌مان نه نان خورش‌مان درد و رشک
&&
کوزه‌مان نه آب‌مان از دیده اشک
\\
جامهٔ ما روز تاب آفتاب
&&
شب نهالین و لحاف از ماهتاب
\\
قرص مه را قرص نان پنداشته
&&
دست سوی آسمان برداشته
\\
ننگ درویشان ز درویشی ما
&&
روز شب از روزی اندیشی ما
\\
خویش و بیگانه شده از ما رمان
&&
بر مثال سامری از مردمان
\\
گر بخواهم از کسی یک مشت نسک
&&
مر مرا گوید خمش کن مرگ و جسک
\\
مر عرب را فخر غزوست و عطا
&&
در عرب تو همچو اندر خط خطا
\\
چه غزا ما بی‌غزا خود کشته‌ایم
&&
ما به تیغ فقر بی سر گشته‌ایم
\\
چه عطا ما بر گدایی می‌تنیم
&&
مر مگس را در هوا رگ می‌زنیم
\\
گر کسی مهمان رسد گر من منم
&&
شب بخسپد دلقش از تن بر کنم
\\
\end{longtable}
\end{center}
