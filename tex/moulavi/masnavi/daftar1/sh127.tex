\begin{center}
\section*{بخش ۱۲۷ - دل نهادن عرب بر التماس دلبر خویش و سوگند خوردن کی درین تسلیم مرا حیلتی و امتحانی نیست}
\label{sec:sh127}
\addcontentsline{toc}{section}{\nameref{sec:sh127}}
\begin{longtable}{l p{0.5cm} r}
مرد گفت اکنون گذشتم از خلاف
&&
حکم داری تیغ برکش از غلاف
\\
هرچه گویی من ترا فرمان برم
&&
در بد و نیک آمد آن ننگرم
\\
در وجود تو شوم من منعدم
&&
چون محبم حب یعمی و یصم
\\
گفت زن آهنگ برم می‌کنی
&&
یا بحیلت کشف سرم می‌کنی
\\
گفت والله عالم السر الخفی
&&
کافرید از خاک آدم را صفی
\\
در سه گز قالب که دادش وا نمود
&&
هر چه در الواح و در ارواح بود
\\
تا ابد هرچه بود او پیش پیش
&&
درس کرد از علم الاسماء خویش
\\
تا ملک بی‌خود شد از تدریس او
&&
قدس دیگر یافت از تقدیس او
\\
آن گشادیشان کز آدم رو نمود
&&
در گشاد آسمانهاشان نبود
\\
در فراخی عرصهٔ آن پاک جان
&&
تنگ آمد عرصهٔ هفت آسمان
\\
گفت پیغامبر که حق فرموده است
&&
من نگنجم هیچ در بالا و پست
\\
در زمین و آسمان و عرش نیز
&&
من نگنجم این یقین دان ای عزیز
\\
در دل مؤمن بگنجم ای عجب
&&
گر مرا جویی در آن دلها طلب
\\
گفت ادخل فی عبادی تلتقی
&&
جنة من رویتی یا متقی
\\
عرش با آن نور با پهنای خویش
&&
چون بدید آن را برفت از جای خویش
\\
خود بزرگی عرش باشد بس مدید
&&
لیک صورت کیست چون معنی رسید
\\
هر ملک می‌گفت ما را پیش ازین
&&
الفتی می‌بود بر روی زمین
\\
تخم خدمت بر زمین می‌کاشتیم
&&
زان تعلق ما عجب می‌داشتیم
\\
کین تعلق چیست با این خاکمان
&&
چون سرشت ما بدست از آسمان
\\
الف ما انوار با ظلمات چیست
&&
چون تواند نور با ظلمات زیست
\\
آدما آن الف از بوی تو بود
&&
زانک جسمت را زمین بد تار و پود
\\
جسم خاکت را ازینجا بافتند
&&
نور پاکت را درینجا یافتند
\\
این که جان ما ز روحت یافتست
&&
پیش پیش از خاک آن می‌تافتست
\\
در زمین بودیم و غافل از زمین
&&
غافل از گنجی که در وی بد دفین
\\
چون سفر فرمود ما را زان مقام
&&
تلخ شد ما را از آن تحویل کام
\\
تا که حجتها همی گفتیم ما
&&
که به جای ما کی آید ای خدا
\\
نور این تسبیح و این تهلیل را
&&
می‌فروشی بهر قال و قیل را
\\
حکم حق گسترد بهر ما بساط
&&
که بگویید ازطریق انبساط
\\
هرچه آید بر زبانتان بی‌حذر
&&
همچو طفلان یگانه با پدر
\\
زانک این دمها چه گر نالایقست
&&
رحمت من بر غضب هم سابقست
\\
از پی اظهار این سبق ای ملک
&&
در تو بنهم داعیهٔ اشکال و شک
\\
تا بگویی و نگیرم بر تو من
&&
منکر حلمم نیارد دم زدن
\\
صد پدر صد مادر اندر حلم ما
&&
هر نفس زاید در افتد در فنا
\\
حلم ایشان کف بحر حلم ماست
&&
کف رود آید ولی دریا بجاست
\\
خود چه گویم پیش آن در این صدف
&&
نیست الا کف کف کف کف
\\
حق آن کف حق آن دریای صاف
&&
کامتحانی نیست این گفت و نه لاف
\\
از سر مهر و صفا است و خضوع
&&
حق آنکس که بدو دارم رجوع
\\
گر بپیشت امتحانست این هوس
&&
امتحان را امتحان کن یک نفس
\\
سر مپوشان تا پدید آید سرم
&&
امر کن تو هر چه بر وی قادرم
\\
دل مپوشان تا پدید آید دلم
&&
تا قبول آرم هر آنچ قابلم
\\
چون کنم در دست من چه چاره است
&&
درنگر تا جان من چه کاره است
\\
\end{longtable}
\end{center}
