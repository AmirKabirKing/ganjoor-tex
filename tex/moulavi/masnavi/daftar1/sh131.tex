\begin{center}
\section*{بخش ۱۳۱ - در بیان آنک چنانک گدا عاشق کرمست و عاشق کریم کرم کریم هم عاشق گداست اگر گدا را صبر بیش بود کریم بر در او آید و اگر کریم را صبر بیش بود گدا بر در او آید اما صبر گدا کمال گداست و صبر کریم نقصان اوست}
\label{sec:sh131}
\addcontentsline{toc}{section}{\nameref{sec:sh131}}
\begin{longtable}{l p{0.5cm} r}
بانگ می‌آمد که ای طالب بیا
&&
جود محتاج گدایان چون گدا
\\
جود می‌جوید گدایان و ضعاف
&&
همچو خوبان کآینه جویند صاف
\\
روی خوبان ز آینه زیبا شود
&&
روی احسان از گدا پیدا شود
\\
پس ازین فرمود حق در والضحی
&&
بانگ کم زن ای محمد بر گدا
\\
چون گدا آیینهٔ جودست هان
&&
دم بود بر روی آیینه زیان
\\
آن یکی جودش گدا آرد پدید
&&
و آن دگر بخشد گدایان را مزید
\\
پس گدایان آیت جود حقند
&&
وانک با حقند جود مطلقند
\\
وانک جز این دوست او خود مرده‌ایست
&&
او برین در نیست نقش پرده‌ایست
\\
\end{longtable}
\end{center}
