\begin{center}
\section*{بخش ۱۴۷ - نشاندن پادشاه صوفیان عارف را پیش روی خویش تا چشمشان بدیشان روشن شود}
\label{sec:sh147}
\addcontentsline{toc}{section}{\nameref{sec:sh147}}
\begin{longtable}{l p{0.5cm} r}
پادشاهان را چنان عادت بود
&&
این شنیده باشی ار یادت بود
\\
دست چپشان پهلوانان ایستند
&&
زانک دل پهلوی چپ باشد ببند
\\
مشرف و اهل قلم بر دست راست
&&
زانک علم خط و ثبت آن دست راست
\\
صوفیان را پیش رو موضع دهند
&&
کاینهٔ جانند و ز آیینه بهند
\\
سینه صیقلها زده در ذکر و فکر
&&
تا پذیرد آینهٔ دل نقش بکر
\\
هر که او از صلب فطرت خوب زاد
&&
آینه در پیش او باید نهاد
\\
عاشق آیینه باشد روی خوب
&&
صیقل جان آمد و تقوی القلوب
\\
\end{longtable}
\end{center}
