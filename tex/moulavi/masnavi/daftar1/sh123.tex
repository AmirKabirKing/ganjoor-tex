\begin{center}
\section*{بخش ۱۲۳ - حقیر و بی‌خصم دیدن دیده‌های حس صالح و ناقهٔ صالح علیه‌السلام را چون خواهد کی حق لشکری را هلاک کند در نظر ایشان حقیر نماید خصمان را و اندک اگرچه غالب باشد آن خصم و یقللکم فی اعینهم لیقضی الله امرا کان مفعولا}
\label{sec:sh123}
\addcontentsline{toc}{section}{\nameref{sec:sh123}}
\begin{longtable}{l p{0.5cm} r}
ناقهٔ صالح بصورت بد شتر
&&
پی بریدندش ز جهل آن قوم مر
\\
از برای آب چون خصمش شدند
&&
نان کور و آب کور ایشان بدند
\\
ناقة الله آب خورد از جوی و میغ
&&
آب حق را داشتند از حق دریغ
\\
ناقهٔ صالح چو جسم صالحان
&&
شد کمینی در هلاک طالحان
\\
تا بر آن امت ز حکم مرگ و درد
&&
ناقةالله و سقیاها چه کرد
\\
شحنهٔ قهر خدا زیشان بجست
&&
خونبهای اشتری شهری درست
\\
روح همچون صالح و تن ناقه است
&&
روح اندر وصل و تن در فاقه است
\\
روح صالح قابل آفات نیست
&&
زخم بر ناقه بود بر ذات نیست
\\
روح صالح قابل آزار نیست
&&
نور یزدان سغبهٔ کفار نیست
\\
حق از آن پیوست با جسمی نهان
&&
تاش آزارند و بینند امتحان
\\
بی‌خبر کزار این آزار اوست
&&
آب این خم متصل با آب جوست
\\
زان تعلق کرد با جسمی اله
&&
تا که گردد جمله عالم را پناه
\\
ناقهٔ جسم ولی را بنده باش
&&
تا شوی با روح صالح خواجه‌تاش
\\
گفت صالح چونک کردید این حسد
&&
بعد سه روز از خدا نقمت رسد
\\
بعد سه روز دگر از جانستان
&&
آفتی آید که دارد سه نشان
\\
رنگ روی جمله‌تان گردد دگر
&&
رنگ رنگ مختلف اندر نظر
\\
روز اول رویتان چون زعفران
&&
در دوم رو سرخ همچون ارغوان
\\
در سوم گردد همه روها سیاه
&&
بعد از آن اندر رسد قهر اله
\\
گر نشان خواهید از من زین وعید
&&
کرهٔ ناقه به سوی که دوید
\\
گر توانیدش گرفتن چاره هست
&&
ورنه خود مرغ امید از دام جست
\\
کس نتانست اندر آن کره رسید
&&
رفت در کهسارها شد ناپدید
\\
گفت دیدیت آن قضا معلن شدست
&&
صورت اومید را گردن زدست
\\
کرهٔ ناقه چه باشد خاطرش
&&
که بجا آرید ز احسان و برش
\\
گر بجا آید دلش رستید از آن
&&
ورنه نومیدیت و ساعد را گزان
\\
چون شنیدند این وعید منکدر
&&
چشم بنهادند و آن را منتظر
\\
روز اول روی خود دیدند زرد
&&
می‌زدند از ناامیدی آه سرد
\\
سرخ شد روی همه روز دوم
&&
نوبت اومید و توبه گشت گم
\\
شد سیه روز سیم روی همه
&&
حکم صالح راست شد بی ملحمه
\\
چون همه در ناامیدی سر زدند
&&
همچو مرغان در دو زانو آمدند
\\
در نبی آورد جبریل امین
&&
شرح این زانو زدن را جاثمین
\\
زانو آن دم زن که تعلیمت کنند
&&
وز چنین زانو زدن بیمت کنند
\\
منتظر گشتند زخم قهر را
&&
قهر آمد نیست کرد آن شهر را
\\
صالح از خلوت بسوی شهر رفت
&&
شهر دید اندر میان دود و نفت
\\
ناله از اجزای ایشان می‌شنید
&&
نوحه پیدا نوحه‌گویان ناپدید
\\
ز استخوانهاشان شنید او ناله‌ها
&&
اشک‌ریزان جانشان چون ژاله‌ها
\\
صالح آن بشنید و گریه ساز کرد
&&
نوحه بر نوحه‌گران آغاز کرد
\\
گفت ای قومی به باطل زیسته
&&
وز شما من پیش حق بگریسته
\\
حق بگفته صبر کن بر جورشان
&&
پندشان ده بس نماند از دورشان
\\
من بگفته پند شد بند از جفا
&&
شیر پند از مهر جوشد وز صفا
\\
بس که کردید از جفا بر جای من
&&
شیر پند افسرد در رگهای من
\\
حق مرا گفته ترا لطفی دهم
&&
بر سر آن زخمها مرهم نهم
\\
صاف کرده حق دلم را چون سما
&&
روفته از خاطرم جور شما
\\
در نصیحت من شده بار دگر
&&
گفته امثال و سخنها چون شکر
\\
شیر تازه از شکر انگیخته
&&
شیر و شهدی با سخن آمیخته
\\
در شما چون زهر گشته آن سخن
&&
زانک زهرستان بدیت از بیخ و بن
\\
چون شوم غمگین که غم شد سرنگون
&&
غم شما بودیت ای قوم حرون
\\
هیچ کس بر مرگ غم نوحه کند
&&
ریش سر چون شد کسی مو بر کند
\\
رو بخود کرد و بگفت ای نوحه‌گر
&&
نوحه‌ات را می‌نیرزند آن نفر
\\
کژ مخوان ای راست‌خوانندهٔ مبین
&&
کیف آسی خلف قوم ظالمین
\\
باز اندر چشم و دل او گریه یافت
&&
رحمتی بی‌علتی در وی بتافت
\\
قطره می‌بارید و حیران گشته بود
&&
قطره‌ای بی‌علت از دریای جود
\\
عقل او می‌گفت کین گریه ز چیست
&&
بر چنان افسوسیان شاید گریست
\\
بر چه می‌گریی بگو بر فعلشان
&&
بر سپاه کینه‌توز بد نشان
\\
بر دل تاریک پر زنگارشان
&&
بر زبان زهر همچون مارشان
\\
بر دم و دندان سگسارانه‌شان
&&
بر دهان و چشم کزدم خانه‌شان
\\
بر ستیز و تسخر و افسوسشان
&&
شکر کن چون کرد حق محبوسشان
\\
دستشان کژ پایشان کژ چشم کژ
&&
مهرشان کژ صلحشان کژ خشم کژ
\\
از پی تقلید و معقولات نقل
&&
پا نهاده بر سر این پیر عقل
\\
پیرخر نه جمله گشته پیر خر
&&
از ریای چشم و گوش همدگر
\\
از بهشت آورد یزدان بندگان
&&
تا نمایدشان سقر پروردگان
\\
\end{longtable}
\end{center}
