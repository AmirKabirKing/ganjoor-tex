\begin{center}
\section*{بخش ۵۵ - جواب خرگوش نخچیران را}
\label{sec:sh055}
\addcontentsline{toc}{section}{\nameref{sec:sh055}}
\begin{longtable}{l p{0.5cm} r}
گفت ای یاران حقم الهام داد
&&
مر ضعیفی را قوی رایی فتاد
\\
آنچ حق آموخت مر زنبور را
&&
آن نباشد شیر را و گور را
\\
خانه‌ها سازد پر از حلوای تر
&&
حق برو آن علم را بگشاد در
\\
آنچ حق آموخت کرم پیله را
&&
هیچ پیلی داند آن گون حیله را
\\
آدم خاکی ز حق آموخت علم
&&
تا به هفتم آسمان افروخت علم
\\
نام و ناموس ملک را در شکست
&&
کوری آنکس که در حق درشکست
\\
زاهد ششصد هزاران ساله را
&&
پوزبندی ساخت آن گوساله را
\\
تا نتاند شیر علم دین کشید
&&
تا نگردد گرد آن قصر مشید
\\
علمهای اهل حس شد پوزبند
&&
تا نگیرد شیر از آن علم بلند
\\
قطرهٔ دل را یکی گوهر فتاد
&&
کان به دریاها و گردونها نداد
\\
چند صورت آخر ای صورت‌پرست
&&
جان بی‌معنیت از صورت نرست
\\
گر بصورت آدمی انسان بدی
&&
احمد و بوجهل خود یکسان بدی
\\
نقش بر دیوار مثل آدمست
&&
بنگر از صورت چه چیز او کمست
\\
جان کمست آن صورت با تاب را
&&
رو بجو آن گوهر کم‌یاب را
\\
شد سر شیران عالم جمله پست
&&
چون سگ اصحاب را دادند دست
\\
چه زیانستش از آن نقش نفور
&&
چونک جانش غرق شد در بحر نور
\\
وصف و صورت نیست اندر خامه‌ها
&&
عالم و عادل بود در نامه‌ها
\\
عالم و عادل همه معنیست بس
&&
کش نیابی در مکان و پیش و پس
\\
می‌زند بر تن ز سوی لامکان
&&
می‌نگنجد در فلک خورشید جان
\\
\end{longtable}
\end{center}
