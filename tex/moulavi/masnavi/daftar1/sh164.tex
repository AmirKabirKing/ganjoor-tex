\begin{center}
\section*{بخش ۱۶۴ - خدو انداختن خصم در روی امیر الممنین علی کرم الله وجهه و انداختن امیرالممنین علی شمشیر از دست}
\label{sec:sh164}
\addcontentsline{toc}{section}{\nameref{sec:sh164}}
\begin{longtable}{l p{0.5cm} r}
از علی آموز اخلاص عمل
&&
شیر حق را دان مطهر از دغل
\\
در غزا بر پهلوانی دست یافت
&&
زود شمشیری بر آورد و شتافت
\\
او خدو انداخت در روی علی
&&
افتخار هر نبی و هر ولی
\\
آن خدو زد بر رخی که روی ماه
&&
سجده آرد پیش او در سجده‌گاه
\\
در زمان انداخت شمشیر آن علی
&&
کرد او اندر غزااش کاهلی
\\
گشت حیران آن مبارز زین عمل
&&
وز نمودن عفو و رحمت بی‌محل
\\
گفت بر من تیغ تیز افراشتی
&&
از چه افکندی مرا بگذاشتی
\\
آن چه دیدی بهتر از پیکار من
&&
تا شدی تو سست در اشکار من
\\
آن چه دیدی که چنین خشمت نشست
&&
تا چنان برقی نمود و باز جست
\\
آن چه دیدی که مرا زان عکس دید
&&
در دل و جان شعله‌ای آمد پدید
\\
آن چه دیدی برتر از کون و مکان
&&
که به از جان بود و بخشیدیم جان
\\
در شجاعت شیر ربانیستی
&&
در مروت خود کی داند کیستی
\\
در مروت ابر موسیی بتیه
&&
کآمد از وی خوان و نان بی‌شبیه
\\
ابرها گندم دهد کان را بجهد
&&
پخته و شیرین کند مردم چو شهد
\\
ابر موسی پر رحمت بر گشاد
&&
پخته و شیرین بی زحمت بداد
\\
از برای پخته‌خواران کرم
&&
رحمتش افراخت در عالم علم
\\
تا چهل سال آن وظیفه و آن عطا
&&
کم نشد یک روز زان اهل رجا
\\
تا هم ایشان از خسیسی خاستند
&&
گندنا و تره و خس خواستند
\\
امت احمد که هستید از کرام
&&
تا قیامت هست باقی آن طعام
\\
چون ابیت عند ربی فاش شد
&&
یطعم و یسقی کنایت ز آش شد
\\
هیچ بی‌تاویل این را در پذیر
&&
تا در آید در گلو چون شهد و شیر
\\
زانک تاویلست وا داد عطا
&&
چونک بیند آن حقیقت را خطا
\\
آن خطا دیدن ز ضعف عقل اوست
&&
عقل کل مغزست و عقل جزو پوست
\\
خویش را تاویل کن نه اخبار را
&&
مغز را بد گوی نه گلزار را
\\
ای علی که جمله عقل و دیده‌ای
&&
شمه‌ای واگو از آنچ دیده‌ای
\\
تیغ حلمت جان ما را چاک کرد
&&
آب علمت خاک ما را پاک کرد
\\
بازگو دانم که این اسرار هوست
&&
زانک بی شمشیر کشتن کار اوست
\\
صانع بی آلت و بی جارحه
&&
واهب این هدیه‌های رابحه
\\
صد هزاران می چشاند هوش را
&&
که خبر نبود دو چشم و گوش را
\\
باز گو ای باز عرش خوش‌شکار
&&
تا چه دیدی این زمان از کردگار
\\
چشم تو ادراک غیب آموخته
&&
چشمهای حاضران بر دوخته
\\
آن یکی ماهی همی‌بیند عیان
&&
وان یکی تاریک می‌بیند جهان
\\
وان یکی سه ماه می‌بیند بهم
&&
این سه کس بنشسته یک موضع نعم
\\
چشم هر سه باز و گوش هر سه تیز
&&
در تو آویزان و از من در گریز
\\
سحر عین است این عجب لطف خفیست
&&
بر تو نقش گرگ و بر من یوسفیست
\\
عالم ار هجده هزارست و فزون
&&
هر نظر را نیست این هجده زبون
\\
راز بگشا ای علی مرتضی
&&
ای پس سؤ القضا حسن القضا
\\
یا تو واگو آنچ عقلت یافتست
&&
یا بگویم آنچ برمن تافتست
\\
از تو بر من تافت چون داری نهان
&&
می‌فشانی نور چون مه بی زبان
\\
لیک اگر در گفت آید قرص ماه
&&
شب روان را زودتر آرد به راه
\\
از غلط ایمن شوند و از ذهول
&&
بانگ مه غالب شود بر بانگ غول
\\
ماه بی گفتن چو باشد رهنما
&&
چون بگوید شد ضیا اندر ضیا
\\
چون تو بابی آن مدینهٔ علم را
&&
چون شعاعی آفتاب حلم را
\\
باز باش ای باب بر جویای باب
&&
تا رسد از تو قشور اندر لباب
\\
باز باش ای باب رحمت تا ابد
&&
بارگاه ما له کفوا احد
\\
هر هوا و ذره‌ای خود منظریست
&&
نا گشاده کی گود کانجا دریست
\\
تا بنگشاید دری را دیدبان
&&
در درون هرگز نجنبد این گمان
\\
چون گشاده شد دری حیران شود
&&
مرغ اومید و طمع پران شود
\\
غافلی ناگه به ویران گنج یافت
&&
سوی هر ویران از آن پس می‌شتافت
\\
تا ز درویشی نیابی تو گهر
&&
کی گهر جویی ز درویشی دگر
\\
سالها گر ظن دود با پای خویش
&&
نگذرد ز اشکاف بینیهای خویش
\\
تا ببینی نایدت از غیب بو
&&
غیر بینی هیچ می‌بینی بگو
\\
\end{longtable}
\end{center}
