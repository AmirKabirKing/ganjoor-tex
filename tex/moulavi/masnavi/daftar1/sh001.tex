\begin{center}
\section*{بخش ۱ - سر آغاز}
\label{sec:sh001}
\addcontentsline{toc}{section}{\nameref{sec:sh001}}
\begin{longtable}{l p{0.5cm} r}
بشنو این نی چون شکایت می‌کند
&&
از جداییها حکایت می‌کند
\\
کز نیستان تا مرا ببریده‌اند
&&
در نفیرم مرد و زن نالیده‌اند
\\
سینه خواهم شرحه شرحه از فراق
&&
تا بگویم شرح درد اشتیاق
\\
هر کسی کو دور ماند از اصل خویش
&&
باز جوید روزگار وصل خویش
\\
من به هر جمعیتی نالان شدم
&&
جفت بدحالان و خوش‌حالان شدم
\\
هرکسی از ظن خود شد یار من
&&
از درون من نجست اسرار من
\\
سر من از نالهٔ من دور نیست
&&
لیک چشم و گوش را آن نور نیست
\\
تن ز جان و جان ز تن مستور نیست
&&
لیک کس را دید جان دستور نیست
\\
آتشست این بانگ نای و نیست باد
&&
هر که این آتش ندارد نیست باد
\\
آتش عشقست کاندر نی فتاد
&&
جوشش عشقست کاندر می فتاد
\\
نی حریف هرکه از یاری برید
&&
پرده‌هااش پرده‌های ما درید
\\
همچو نی زهری و تریاقی کی دید
&&
همچو نی دمساز و مشتاقی کی دید
\\
نی حدیث راه پر خون می‌کند
&&
قصه‌های عشق مجنون می‌کند
\\
محرم این هوش جز بیهوش نیست
&&
مر زبان را مشتری جز گوش نیست
\\
در غم ما روزها بیگاه شد
&&
روزها با سوزها همراه شد
\\
روزها گر رفت گو رو باک نیست
&&
تو بمان ای آنک چون تو پاک نیست
\\
هر که جز ماهی ز آبش سیر شد
&&
هرکه بی روزیست روزش دیر شد
\\
در نیابد حال پخته هیچ خام
&&
پس سخن کوتاه باید والسلام
\\
بند بگسل باش آزاد ای پسر
&&
چند باشی بند سیم و بند زر
\\
گر بریزی بحر را در کوزه‌ای
&&
چند گنجد قسمت یک روزه‌ای
\\
کوزهٔ چشم حریصان پر نشد
&&
تا صدف قانع نشد پر در نشد
\\
هر که را جامه ز عشقی چاک شد
&&
او ز حرص و عیب کلی پاک شد
\\
شاد باش ای عشق خوش سودای ما
&&
ای طبیب جمله علتهای ما
\\
ای دوای نخوت و ناموس ما
&&
ای تو افلاطون و جالینوس ما
\\
جسم خاک از عشق بر افلاک شد
&&
کوه در رقص آمد و چالاک شد
\\
عشق جان طور آمد عاشقا
&&
طور مست و خر موسی صاعقا
\\
با لب دمساز خود گر جفتمی
&&
همچو نی من گفتنیها گفتمی
\\
هر که او از هم‌زبانی شد جدا
&&
بی زبان شد گرچه دارد صد نوا
\\
چونک گل رفت و گلستان درگذشت
&&
نشنوی زان پس ز بلبل سر گذشت
\\
جمله معشوقست و عاشق پرده‌ای
&&
زنده معشوقست و عاشق مرده‌ای
\\
چون نباشد عشق را پروای او
&&
او چو مرغی ماند بی‌پر وای او
\\
من چگونه هوش دارم پیش و پس
&&
چون نباشد نور یارم پیش و پس
\\
عشق خواهد کین سخن بیرون بود
&&
آینه غماز نبود چون بود
\\
آینت دانی چرا غماز نیست
&&
زانک زنگار از رخش ممتاز نیست
\\
\end{longtable}
\end{center}
