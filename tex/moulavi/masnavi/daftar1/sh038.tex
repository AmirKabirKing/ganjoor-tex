\begin{center}
\section*{بخش ۳۸ - به سخن آمدن طفل درمیان آتش و تحریض کردن خلق را در افتادن بتش}
\label{sec:sh038}
\addcontentsline{toc}{section}{\nameref{sec:sh038}}
\begin{longtable}{l p{0.5cm} r}
یک زنی با طفل آورد آن جهود
&&
پیش آن بت و آتش اندر شعله بود
\\
طفل ازو بستد در آتش در فکند
&&
زن بترسید و دل از ایمان بکند
\\
خواست تا او سجده آرد پیش بت
&&
بانگ زد آن طفل انی لم امت
\\
اندر آ ای مادر اینجا من خوشم
&&
گر چه در صورت میان آتشم
\\
چشم‌بندست آتش از بهر حجیب
&&
رحمتست این سر برآورده ز جیب
\\
اندر آ مادر ببین برهان حق
&&
تا ببینی عشرت خاصان حق
\\
اندر آ و آب بین آتش‌مثال
&&
از جهانی کآتش است آبش مثال
\\
اندر آ اسرار ابراهیم بین
&&
کو در آتش یافت سرو و یاسمین
\\
مرگ می‌دیدم گه زادن ز تو
&&
سخت خوفم بود افتادن ز تو
\\
چون بزادم رستم از زندان تنگ
&&
در جهان خوش‌هوای خوب‌رنگ
\\
من جهان را چون رحم دیدم کنون
&&
چون درین آتش بدیدم این سکون
\\
اندرین آتش بدیدم عالمی
&&
ذره ذره اندرو عیسی‌دمی
\\
نک جهان نیست‌شکل هست‌ذات
&&
و آن جهان هست شکل بی‌ثبات
\\
اندر آ مادر بحق مادری
&&
بین که این آذر ندارد آذری
\\
اندر آ مادر که اقبال آمدست
&&
اندر آ مادر مده دولت ز دست
\\
قدرت آن سگ بدیدی اندر آ
&&
تا ببینی قدرت و لطف خدا
\\
من ز رحمت می‌کشانم پای تو
&&
کز طرب خود نیستم پروای تو
\\
اندر آ و دیگران را هم بخوان
&&
کاندر آتش شاه بنهادست خوان
\\
اندر آیید ای مسلمانان همه
&&
غیر عذب دین عذابست آن همه
\\
اندر آیید ای همه پروانه‌وار
&&
اندرین بهره که دارد صد بهار
\\
بانگ می‌زد درمیان آن گروه
&&
پر همی شد جان خلقان از شکوه
\\
خلق خود را بعد از آن بی‌خویشتن
&&
می‌فکندند اندر آتش مرد و زن
\\
بی‌موکل بی‌کشش از عشق دوست
&&
زانک شیرین کردن هر تلخ ازوست
\\
تا چنان شد کان عوانان خلق را
&&
منع می‌کردند کآتش در میا
\\
آن یهودی شد سیه‌رو و خجل
&&
شد پشیمان زین سبب بیماردل
\\
کاندر ایمان خلق عاشق‌تر شدند
&&
در فنای جسم صادق‌تر شدند
\\
مکر شیطان هم درو پیچید شکر
&&
دیو هم خود را سیه‌رو دید شکر
\\
آنچ می‌مالید در روی کسان
&&
جمع شد در چهرهٔ آن ناکس آن
\\
آنک می‌درید جامهٔ خلق چست
&&
شد دریده آن او ایشان درست
\\
\end{longtable}
\end{center}
