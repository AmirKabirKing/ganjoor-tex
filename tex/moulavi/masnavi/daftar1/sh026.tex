\begin{center}
\section*{بخش ۲۶ - دفع گفتن وزیر مریدان را}
\label{sec:sh026}
\addcontentsline{toc}{section}{\nameref{sec:sh026}}
\begin{longtable}{l p{0.5cm} r}
گفت هان ای سخرگان گفت و گو
&&
وعظ و گفتار زبان و گوش جو
\\
پنبه اندر گوش حس دون کنید
&&
بند حس از چشم خود بیرون کنید
\\
پنبهٔ آن گوش سر گوش سرست
&&
تا نگردد این کر آن باطن کرست
\\
بی‌حس و بی‌گوش و بی‌فکرت شوید
&&
تا خطاب ارجعی را بشنوید
\\
تا به گفت و گوی بیداری دری
&&
تو زگفت خواب بویی کی بری
\\
سیر بیرونیست قول و فعل ما
&&
سیر باطن هست بالای سما
\\
حس خشکی دید کز خشکی بزاد
&&
عیسی جان پای بر دریا نهاد
\\
سیر جسم خشک بر خشکی فتاد
&&
سیر جان پا در دل دریا نهاد
\\
چونک عمر اندر ره خشکی گذشت
&&
گاه کوه و گاه دریا گاه دشت
\\
آب حیوان از کجا خواهی تو یافت
&&
موج دریا را کجا خواهی شکافت
\\
موج خاکی وهم و فهم و فکر ماست
&&
موج آبی محو و سکرست و فناست
\\
تا درین سکری از آن سکری تو دور
&&
تا ازین مستی از آن جامی نفور
\\
گفت و گوی ظاهر آمد چون غبار
&&
مدتی خاموش خو کن هوش‌دار
\\
\end{longtable}
\end{center}
