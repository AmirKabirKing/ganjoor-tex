\begin{center}
\section*{بخش ۱۵۹ - متهم کردن غلامان و خواجه‌تاشان مر لقمان را کی آن میوه‌های ترونده را که می‌آوردیم او خورده است}
\label{sec:sh159}
\addcontentsline{toc}{section}{\nameref{sec:sh159}}
\begin{longtable}{l p{0.5cm} r}
بود لقمان پیش خواجهٔ خویشتن
&&
در میان بندگانش خوارتن
\\
می‌فرستاد او غلامان را به باغ
&&
تا که میوه آیدش بهر فراغ
\\
بود لقمان در غلامان چون طفیل
&&
پر معانی تیره‌صورت همچو لیل
\\
آن غلامان میوه‌های جمع را
&&
خوش بخوردند از نهیب طمع را
\\
خواجه را گفتند لقمان خورد آن
&&
خواجه بر لقمان ترش گشت و گران
\\
چون تفحص کرد لقمان از سبب
&&
در عتاب خواجه‌اش بگشاد لب
\\
گفت لقمان سیدا پیش خدا
&&
بندهٔ خاین نباشد مرتضی
\\
امتحان کن جمله‌مان را ای کریم
&&
سیرمان در ده تو از آب حمیم
\\
بعد از آن ما را به صحرایی کلان
&&
تو سواره ما پیاده می‌دوان
\\
آنگهان بنگر تو بدکردار را
&&
صنعهای کاشف الاسرار را
\\
گشت ساقی خواجه از آب حمیم
&&
مر غلامان را و خوردند آن ز بیم
\\
بعد از آن می‌راندشان در دشتها
&&
می‌دویدند آن نفر تحت و علا
\\
قی در افتادند ایشان از عنا
&&
آب می‌آورد زیشان میوه‌ها
\\
چون که لقمان را در آمد قی ز ناف
&&
می بر آمد از درونش آب صاف
\\
حکمت لقمان چو داند این نمود
&&
پس چه باشد حکمت رب الوجود
\\
یوم تبلی والسرائر کلها
&&
بان منکم کامن لا یشتهی
\\
چون سقوا ماء حمیما قطعت
&&
جملة الاستار مما افضعت
\\
نار زان آمد عذاب کافران
&&
که حجر را نار باشد امتحان
\\
آن دل چون سنگ را ما چند چند
&&
نرم گفتیم و نمی‌پذرفت پند
\\
ریش بد را داروی بد یافت رگ
&&
مر سر خر را سر دندان سگ
\\
الخبیثات الخبیثین حکمتست
&&
زشت را هم زشت جفت و بابتست
\\
پس تو هر جفتی که می‌خواهی برو
&&
محو و هم‌شکل و صفات او بشو
\\
نور خواهی مستعد نور شو
&&
دور خواهی خویش‌بین و دور شو
\\
ور رهی خواهی ازین سجن خرب
&&
سر مکش از دوست و اسجد واقترب
\\
\end{longtable}
\end{center}
