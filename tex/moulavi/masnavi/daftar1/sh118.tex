\begin{center}
\section*{بخش ۱۱۸ - مراعات کردن زن شوهر را و استغفار کردن از گفتهٔ خویش}
\label{sec:sh118}
\addcontentsline{toc}{section}{\nameref{sec:sh118}}
\begin{longtable}{l p{0.5cm} r}
زن چو دید او را که تند و توسنست
&&
گشت گریان گریه خود دام زنست
\\
گفت از تو کی چنین پنداشتم
&&
از تو من اومید دیگر داشتم
\\
زن در آمد ازطریق نیستی
&&
گفت من خاک شماام نی ستی
\\
جسم و جان و هرچه هستم آن تست
&&
حکم و فرمان جملگی فرمان تست
\\
گر ز درویشی دلم از صبر جست
&&
بهر خویشم نیست آن بهر تو است
\\
تو مرا در دردها بودی دوا
&&
من نمی‌خواهم که باشی بی‌نوا
\\
جان تو کز بهر خویشم نیست این
&&
از برای تستم این ناله و حنین
\\
خویش من والله که بهر خویش تو
&&
هر نفس خواهد که میرد پیش تو
\\
کاش جانت کش روان من فدا
&&
از ضمیر جان من واقف بدی
\\
چون تو با من این چنین بودی بظن
&&
هم ز جان بیزار گشتم هم ز تن
\\
خاک را بر سیم و زر کردیم چون
&&
تو چنینی با من ای جان را سکون
\\
تو که در جان و دلم جا می‌کنی
&&
زین قدر از من تبرا می‌کنی
\\
تو تبرا کن که هستت دستگاه
&&
ای تبرای ترا جان عذرخواه
\\
یاد می‌کن آن زمانی را که من
&&
چون صنم بودم تو بودی چون شمن
\\
بنده بر وفق تو دل افروختست
&&
هرچه گویی پخت گوید سوختست
\\
من سپاناخ تو با هرچم پزی
&&
یا ترش‌با یا که شیرین می‌سزی
\\
کفر گفتم نک بایمان آمدم
&&
پیش حکمت از سر جان آمدم
\\
خوی شاهانهٔ ترا نشناختم
&&
پیش تو گستاخ خر در تاختم
\\
چون ز عفو تو چراغی ساختم
&&
توبه کردم اعتراض انداختم
\\
می‌نهم پیش تو شمشیر و کفن
&&
می‌کشم پیش تو گردن را بزن
\\
از فراق تلخ می‌گویی سخن
&&
هر چه خواهی کن ولیکن این مکن
\\
در تو از من عذرخواهی هست سر
&&
با تو بی من او شفیعی مستمر
\\
عذر خواهم در درونت خلق تست
&&
ز اعتماد او دل من جرم جست
\\
رحم کن پنهان ز خود ای خشمگین
&&
ای که خلقت به ز صد من انگبین
\\
زین نسق می‌گفت با لطف و گشاد
&&
در میانه گریه‌ای بر وی فتاد
\\
گریه چون از حد گذشت و های های
&&
زو که بی گریه بد او خود دلربای
\\
شد از آن باران یکی برقی پدید
&&
زد شراری در دل مرد وحید
\\
آنک بندهٔ روی خوبش بود مرد
&&
چون بود چون بندگی آغاز کرد
\\
آنک از کبرش دلت لرزان بود
&&
چون شوی چون پیش تو گریان شود
\\
آنک از نازش دل و جان خون بود
&&
چونک آید در نیاز او چون بود
\\
آنک در جور و جفااش دام ماست
&&
عذر ما چه بود چو او در عذر خاست
\\
زین للناس حق آراستست
&&
زانچ حق آراست چون دانند جست
\\
چون پی یسکن الیهاش آفرید
&&
کی تواند آدم از حوا برید
\\
رستم زال ار بود وز حمزه بیش
&&
هست در فرمان اسیر زال خویش
\\
آنک عالم مست گفتش آمدی
&&
کلمینی یا حمیرا می‌زدی
\\
آب غالب شد بر آتش از لهیب
&&
زآتش او جوشد چو باشد در حجیب
\\
چونک دیگی حایل آمد هر دو را
&&
نیست کرد آن آب را کردش هوا
\\
ظاهرا بر زن چو آب ار غالبی
&&
باطنا مغلوب و زن را طالبی
\\
این چنین خاصیتی در آدمیست
&&
مهر حیوان را کمست آن از کمیست
\\
\end{longtable}
\end{center}
