\begin{center}
\section*{بخش ۳۷ - وحی آمدن موسی را علیه السلام در عذر آن شبان}
\label{sec:sh037}
\addcontentsline{toc}{section}{\nameref{sec:sh037}}
\begin{longtable}{l p{0.5cm} r}
بعد از آن در سر موسی حق نهفت
&&
رازهایی گفت کان ناید به گفت
\\
بر دل موسی سخنها ریختند
&&
دیدن و گفتن بهم آمیختند
\\
چند بی‌خود گشت و چند آمد بخود
&&
چند پرید از ازل سوی ابد
\\
بعد ازین گر شرح گویم ابلهیست
&&
زانک شرح این ورای آگهیست
\\
ور بگویم عقلها را بر کند
&&
ور نویسم بس قلمها بشکند
\\
چونک موسی این عتاب از حق شنید
&&
در بیابان در پی چوپان دوید
\\
بر نشان پای آن سرگشته راند
&&
گرد از پرهٔ بیابان بر فشاند
\\
گام پای مردم شوریده خود
&&
هم ز گام دیگران پیدا بود
\\
یک قدم چون رخ ز بالا تا نشیب
&&
یک قدم چون پیل رفته بر وریب
\\
گاه چون موجی بر افرازان علم
&&
گاه چون ماهی روانه بر شکم
\\
گاه بر خاکی نبشته حال خود
&&
همچو رمالی که رملی بر زند
\\
عاقبت دریافت او را و بدید
&&
گفت مژده ده که دستوری رسید
\\
هیچ آدابی و ترتیبی مجو
&&
هرچه می‌خواهد دل تنگت بگو
\\
کفر تو دینست و دینت نور جان
&&
آمنی وز تو جهانی در امان
\\
ای معاف یفعل الله ما یشا
&&
بی‌محابا رو زبان را بر گشا
\\
گفت ای موسی از آن بگذشته‌ام
&&
من کنون در خون دل آغشته‌ام
\\
من ز سدرهٔ منتهی بگذشته‌ام
&&
صد هزاران ساله زان سو رفته‌ام
\\
تازیانه بر زدی اسپم بگشت
&&
گنبدی کرد و ز گردون بر گذشت
\\
محرم ناسوت ما لاهوت باد
&&
آفرین بر دست و بر بازوت باد
\\
حال من اکنون برون از گفتنست
&&
اینچ می‌گویم نه احوال منست
\\
نقش می‌بینی که در آیینه‌ایست
&&
نقش تست آن نقش آن آیینه نیست
\\
دم که مرد نایی اندر نای کرد
&&
درخور نایست نه درخورد مرد
\\
هان و هان گر حمد گویی گر سپاس
&&
همچو نافرجام آن چوپان شناس
\\
حمد تو نسبت بدان گر بهترست
&&
لیک آن نسبت بحق هم ابترست
\\
چند گویی چون غطا برداشتند
&&
کین نبودست آنک می‌پنداشتند
\\
این قبول ذکر تو از رحمتست
&&
چون نماز مستحاضه رخصتست
\\
با نماز او بیالودست خون
&&
ذکر تو آلودهٔ تشبیه و چون
\\
خون پلیدست و به آبی می‌رود
&&
لیک باطن را نجاستها بود
\\
کان بغیر آب لطف کردگار
&&
کم نگردد از درون مرد کار
\\
در سجودت کاش رو گردانیی
&&
معنی سبحان ربی دانیی
\\
کای سجودم چون وجودم ناسزا
&&
مر بدی را تو نکویی ده جزا
\\
این زمین از حلم حق دارد اثر
&&
تا نجاست برد و گلها داد بر
\\
تا بپوشد او پلیدیهای ما
&&
در عوض بر روید از وی غنچه‌ها
\\
پس چو کافر دید کو در داد و جود
&&
کمتر و بی‌مایه‌تر از خاک بود
\\
از وجود او گل و میوه نرست
&&
جز فساد جمله پاکیها نجست
\\
گفت واپس رفته‌ام من در ذهاب
&&
حسر تا یا لیتنی کنت تراب
\\
کاش از خاکی سفر نگزیدمی
&&
همچو خاکی دانه‌ای می‌چیدمی
\\
چون سفر کردم مرا راه آزمود
&&
زین سفر کردن ره‌آوردم چه بود
\\
زان همه میلش سوی خاکست کو
&&
در سفر سودی نبیند پیش رو
\\
روی واپس کردنش آن حرص و آز
&&
روی در ره کردنش صدق و نیاز
\\
هر گیا را کش بود میل علا
&&
در مزیدست و حیات و در نما
\\
چونک گردانید سر سوی زمین
&&
در کمی و خشکی و نقص و غبین
\\
میل روحت چون سوی بالا بود
&&
در تزاید مرجعت آنجا بود
\\
ور نگوساری سرت سوی زمین
&&
آفلی حق لا یحب الافلین
\\
\end{longtable}
\end{center}
