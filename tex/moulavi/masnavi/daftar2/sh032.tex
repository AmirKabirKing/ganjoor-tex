\begin{center}
\section*{بخش ۳۲ - تتمهٔ حسد آن حشم بر آن غلام خاص}
\label{sec:sh032}
\addcontentsline{toc}{section}{\nameref{sec:sh032}}
\begin{longtable}{l p{0.5cm} r}
قصهٔ شاه و امیران و حسد
&&
بر غلام خاص و سلطان خرد
\\
دور ماند از جر جرار کلام
&&
باز باید گشت و کرد آن را تمام
\\
باغبان ملک با اقبال و بخت
&&
چون درختی را نداند از درخت
\\
آن درختی را که تلخ و رد بود
&&
و آن درختی که یکش هفصد بود
\\
کی برابر دارد اندر تربیت
&&
چون ببیندشان به چشم عاقبت
\\
کان درختان را نهایت چیست بر
&&
گرچه یکسانند این دم در نظر
\\
شیخ کو ینظر بنور الله شد
&&
از نهایت وز نخست آگاه شد
\\
چشم آخربین ببست از بهر حق
&&
چشم آخربین گشاد اندر سبق
\\
آن حسودان بد درختان بوده‌اند
&&
تلخ گوهر شوربختان بوده‌اند
\\
از حسد جوشان و کف می‌ریختند
&&
در نهانی مکر می‌انگیختند
\\
تا غلام خاص را گردن زنند
&&
بیخ او را از زمانه بر کنند
\\
چون شود فانی چو جانش شاه بود
&&
بیخ او در عصمت الله بود
\\
شاه از آن اسرار واقف آمده
&&
همچو بوبکر ربابی تن زده
\\
در تماشای دل بدگوهران
&&
می‌زدی خنبک بر آن کوزه‌گران
\\
مکر می‌سازند قومی حیله‌مند
&&
تا که شه را در فقاعی در کنند
\\
پادشاهی بس عظیمی بی کران
&&
در فقاعی کی بگنجد ای خران
\\
از برای شاه دامی دوختند
&&
آخر این تدبیر ازو آموختند
\\
نحس شاگردی که با استاد خویش
&&
همسری آغازد و آید به پیش
\\
با کدام استاد استاد جهان
&&
پیش او یکسان هویدا و نهان
\\
چشم او ینظر بنور الله شده
&&
پرده‌های جهل را خارق بده
\\
از دل سوراخ چون کهنه گلیم
&&
پرده‌ای بندد به پیش آن حکیم
\\
پرده می‌خندد برو با صد دهان
&&
هر دهانی گشته اشکافی بر آن
\\
گوید آن استاد مر شاگرد را
&&
ای کم از سگ نیستت با من وفا
\\
خود مرا استا مگیر آهن‌گسل
&&
همچو خود شاگرد گیر و کوردل
\\
نه از منت یاریست در جان و روان
&&
بی منت آبی نمی‌گردد روان
\\
پس دل من کارگاه بخت تست
&&
چه شکنی این کارگاه ای نادرست
\\
گوییش پنهان زنم آتش‌زنه
&&
نی به قلب از قلب باشد روزنه
\\
آخر از روزن ببیند فکر تو
&&
دل گواهیی دهد زین ذکر تو
\\
گیر در رویت نمالد از کرم
&&
هرچه گویی خندد و گوید نعم
\\
او نمی‌خندد ز ذوق مالشت
&&
او همی‌خندد بر آن اسگالشت
\\
پس خداعی را خداعی شد جزا
&&
کاسه زن کوزه بخور اینک سزا
\\
گر بدی با تو ورا خندهٔ رضا
&&
صد هزاران گل شکفتی مر ترا
\\
چون دل او در رضا آرد عمل
&&
آفتابی دان که آید در حمل
\\
زو بخندد هم نهار و هم بهار
&&
در هم آمیزد شکوفه و سبزه‌زار
\\
صد هزاران بلبل و قمری نوا
&&
افکنند اندر جهان بی‌نوا
\\
چونک برگ روح خود زرد و سیاه
&&
می‌ببینی چون ندانی خشم شاه
\\
آفتاب شاه در برج عتاب
&&
می‌کند روها سیه همچون کتاب
\\
آن عطارد را ورقها جان ماست
&&
آن سپیدی و آن سیه میزان ماست
\\
باز منشوری نویسد سرخ و سبز
&&
تا رهند ارواح از سودا و عجز
\\
سرخ و سبز افتاد نسخ نوبهار
&&
چون خط قوس و قزح در اعتبار
\\
\end{longtable}
\end{center}
