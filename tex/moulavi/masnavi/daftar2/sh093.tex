\begin{center}
\section*{بخش ۹۳ - کرامات ابراهیم ادهم قدس الله سره بر لب دریا}
\label{sec:sh093}
\addcontentsline{toc}{section}{\nameref{sec:sh093}}
\begin{longtable}{l p{0.5cm} r}
هم ز ابراهیم ادهم آمدست
&&
کو ز راهی بر لب دریا نشست
\\
دلق خود می‌دوخت آن سلطان جان
&&
یک امیری آمد آنجا ناگهان
\\
آن امیر از بندگان شیخ بود
&&
شیخ را بشناخت سجده کرد زود
\\
خیره شد در شیخ و اندر دلق او
&&
شکل دیگر گشته خلق و خلق او
\\
کو رها کرد آنچنان ملکی شگرف
&&
بر گزید آن فقر بس باریک‌حرف
\\
ترک کرد او ملک هفت اقلیم را
&&
می‌زند بر دلق سوزن چون گدا
\\
شیخ واقف گشت از اندیشه‌اش
&&
شیخ چون شیرست و دلها بیشه‌اش
\\
چون رجا و خوف در دلها روان
&&
نیست مخفی بر وی اسرار جهان
\\
دل نگه دارید ای بی حاصلان
&&
در حضور حضرت صاحب‌دلان
\\
پیش اهل تن ادب بر ظاهرست
&&
که خدا زیشان نهان را ساترست
\\
پیش اهل دل ادب بر باطنست
&&
زانک دلشان بر سرایر فاطنست
\\
تو بعکسی پیش کوران بهر جاه
&&
با حضور آیی نشینی پایگاه
\\
پیش بینایان کنی ترک ادب
&&
نار شهوت از آن گشتی حطب
\\
چون نداری فطنت و نور هدی
&&
بهر کوران روی را می‌زن جلا
\\
پیش بینایان حدث در روی مال
&&
ناز می‌کن با چنین گندیده حال
\\
شیخ سوزن زود در دریا فکند
&&
خواست سوزن را به آواز بلند
\\
صد هزاران ماهی اللهیی
&&
سوزن زر در لب هر ماهیی
\\
سر بر آوردند از دریای حق
&&
که بگیر ای شیخ سوزنهای حق
\\
رو بدو کرد و بگفتش ای امیر
&&
ملک دل به یا چنان ملک حقیر
\\
این نشان ظاهرست این هیچ نیست
&&
تا بباطن در روی بینی تو بیست
\\
سوی شهر از باغ شاخی آورند
&&
باغ و بستان را کجا آنجا برند
\\
خاصه باغی کین فلک یک برگ اوست
&&
بلک آن مغزست و این عالم چو پوست
\\
بر نمی‌داری سوی آن باغ گام
&&
بوی افزون جوی و کن دفع زکام
\\
تا که آن بو جاذب جانت شود
&&
تا که آن بو نور چشمانت شود
\\
گفت یوسف ابن یعقوب نبی
&&
بهر بو القوا علی وجه ابی
\\
بهر این بو گفت احمد در عظات
&&
دائما قرة عینی فی الصلوة
\\
پنج حس با همدگر پیوسته‌اند
&&
رسته این هر پنج از اصلی بلند
\\
قوت یک قوت باقی شود
&&
ما بقی را هر یکی ساقی شود
\\
دیدن دیده فزاید عشق را
&&
عشق در دیده فزاید صدق را
\\
صدق بیداری هر حس می‌شود
&&
حسها را ذوق مونس می‌شود
\\
\end{longtable}
\end{center}
