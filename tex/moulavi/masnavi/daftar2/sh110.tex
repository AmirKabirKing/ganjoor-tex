\begin{center}
\section*{بخش ۱۱۰ - جستن آن درخت کی هر که میوهٔ  آن درخت خورد نمیرد}
\label{sec:sh110}
\addcontentsline{toc}{section}{\nameref{sec:sh110}}
\begin{longtable}{l p{0.5cm} r}
گفت دانایی برای داستان
&&
که درختی هست در هندوستان
\\
هر کسی کز میوهٔ او خورد و برد
&&
نی شود او پیر نی هرگز بمرد
\\
پادشاهی این شنید از صادقی
&&
بر درخت و میوه‌اش شد عاشقی
\\
قاصدی دانا ز دیوان ادب
&&
سوی هندوستان روان کرد از طلب
\\
سالها می‌گشت آن قاصد ازو
&&
گرد هندوستان برای جست و جو
\\
شهر شهر از بهر این مطلوب گشت
&&
نی جزیره ماند و نی کوه و نی دشت
\\
هر که را پرسید کردش ریش‌خند
&&
کین کی جوید جز مگر مجنون بند
\\
بس کسان صفعش زدند اندر مزاح
&&
بس کسان گفتند ای صاحب‌فلاح
\\
جست و جوی چون تو زیرک سینه‌صاف
&&
کی تهی باشد کجا باشد گزاف
\\
وین مراعاتش یکی صفع دگر
&&
وین ز صفع آشکارا سخت‌تر
\\
می‌ستودندش بتسخر کای بزرگ
&&
در فلان اقلیم بس هول و سترگ
\\
در فلان بیشه درختی هست سبز
&&
بس بلند و پهن و هر شاخیش گبز
\\
قاصد شه بسته در جستن کمر
&&
می‌شنید از هر کسی نوعی خبر
\\
بس سیاحت کرد آنجا سالها
&&
می‌فرستادش شهنشه مالها
\\
چون بسی دید اندر آن غربت تعب
&&
عاجز آمد آخر الامر از طلب
\\
هیچ از مقصود اثر پیدا نشد
&&
زان غرض غیر خبر پیدا نشد
\\
رشتهٔ اومید او بگسسته شد
&&
جستهٔ او عاقبت ناجسته شد
\\
کرد عزم بازگشتن سوی شاه
&&
اشک می‌بارید و می‌برید راه
\\
\end{longtable}
\end{center}
