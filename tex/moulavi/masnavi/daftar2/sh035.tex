\begin{center}
\section*{بخش ۳۵ - انکار کردن موسی علیه السلام بر مناجات شبان}
\label{sec:sh035}
\addcontentsline{toc}{section}{\nameref{sec:sh035}}
\begin{longtable}{l p{0.5cm} r}
دید موسی یک شبانی را براه
&&
کو همی‌گفت ای گزیننده اله
\\
تو کجایی تا شوم من چاکرت
&&
چارقت دوزم کنم شانه سرت
\\
جامه‌ات شویم شپشهاات کشم
&&
شیر پیشت آورم ای محتشم
\\
دستکت بوسم بمالم پایکت
&&
وقت خواب آید بروبم جایکت
\\
ای فدای تو همه بزهای من
&&
ای بیادت هیهی و هیهای من
\\
این نمط بیهوده می‌گفت آن شبان
&&
گفت موسی با کی است این ای فلان
\\
گفت با آنکس که ما را آفرید
&&
این زمین و چرخ ازو آمد پدید
\\
گفت موسی های بس مدبر شدی
&&
خود مسلمان ناشده کافر شدی
\\
این چه ژاژست این چه کفرست و فشار
&&
پنبه‌ای اندر دهان خود فشار
\\
گند کفر تو جهان را گنده کرد
&&
کفر تو دیبای دین را ژنده کرد
\\
چارق و پاتابه لایق مر تراست
&&
آفتابی را چنینها کی رواست
\\
گر نبندی زین سخن تو حلق را
&&
آتشی آید بسوزد خلق را
\\
آتشی گر نامدست این دود چیست
&&
جان سیه گشته روان مردود چیست
\\
گر همی‌دانی که یزدان داورست
&&
ژاژ و گستاخی ترا چون باورست
\\
دوستی بی‌خرد خود دشمنیست
&&
حق تعالی زین چنین خدمت غنیست
\\
با کی می‌گویی تو این با عم و خال
&&
جسم و حاجت در صفات ذوالجلال
\\
شیر او نوشد که در نشو و نماست
&&
چارق او پوشد که او محتاج پاست
\\
ور برای بنده‌شست این گفت تو
&&
آنک حق گفت او منست و من خود او
\\
آنک گفت انی مرضت لم تعد
&&
من شدم رنجور او تنها نشد
\\
آنک بی یسمع و بی یبصر شده‌ست
&&
در حق آن بنده این هم بیهده‌ست
\\
بی ادب گفتن سخن با خاص حق
&&
دل بمیراند سیه دارد ورق
\\
گر تو مردی را بخوانی فاطمه
&&
گرچه یک جنس‌اند مرد و زن همه
\\
قصد خون تو کند تا ممکنست
&&
گرچه خوش‌خو و حلیم و ساکنست
\\
فاطمه مدحست در حق زنان
&&
مرد را گویی بود زخم سنان
\\
دست و پا در حق ما استایش است
&&
در حق پاکی حق آلایش است
\\
لم یلد لم یولد او را لایق است
&&
والد و مولود را او خالق است
\\
هرچه جسم آمد ولادت وصف اوست
&&
هرچه مولودست او زین سوی جوست
\\
زانک از کون و فساد است و مهین
&&
حادثست و محدثی خواهد یقین
\\
گفت ای موسی دهانم دوختی
&&
وز پشیمانی تو جانم سوختی
\\
جامه را بدرید و آهی کرد تفت
&&
سر نهاد اندر بیابانی و رفت
\\
\end{longtable}
\end{center}
