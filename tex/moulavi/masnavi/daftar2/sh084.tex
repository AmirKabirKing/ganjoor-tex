\begin{center}
\section*{بخش ۸۴ - بیان آنک در هر نفسی فتنهٔ مسجد ضرار هست}
\label{sec:sh084}
\addcontentsline{toc}{section}{\nameref{sec:sh084}}
\begin{longtable}{l p{0.5cm} r}
چون پدید آمد که آن مسجد نبود
&&
خانهٔ حیلت بد و دام جهود
\\
پس نبی فرمود کان را بر کنند
&&
مطرحهٔ خاشاک و خاکستر کنند
\\
صاحب مسجد چو مسجد قلب بود
&&
دانه‌ها بر دام ریزی نیست جود
\\
گوشت اندر شست تو ماهی‌رباست
&&
آنچنان لقمه نی بخشش نه سخاست
\\
مسجد اهل قبا کان بد جماد
&&
آنچ کفو او نبد راهش نداد
\\
در جمادات این چنین حیفی نرفت
&&
زد در آن ناکفو امیر داد نفت
\\
پس حقایق را که اصل اصلهاست
&&
دان که آنجا فرقها و فصلهاست
\\
نه حیاتش چون حیات او بود
&&
نه مماتش چون ممات او بود
\\
گور او هرگز چو گور او مدان
&&
خود چه گویم حال فرق آن جهان
\\
بر محک زن کار خود ای مرد کار
&&
تا نسازی مسجد اهل ضرار
\\
بس در آن مسجدکنان تسخر زدی
&&
چون نظر کردی تو خود زیشان بدی
\\
\end{longtable}
\end{center}
