\begin{center}
\section*{بخش ۸۹ - قصهٔ جوحی و آن کودک کی پیش جنازهٔ پدر خویش نوحه می‌کرد}
\label{sec:sh089}
\addcontentsline{toc}{section}{\nameref{sec:sh089}}
\begin{longtable}{l p{0.5cm} r}
کودکی در پیش تابوت پدر
&&
زار می‌نالید و بر می‌کوفت سر
\\
کای پدر آخر کجاات می‌برند
&&
تا ترا در زیر خاکی آورند
\\
می‌برندت خانه‌ای تنگ و زحیر
&&
نی درو قالی و نه در وی حصیر
\\
نی چراغی در شب و نه روز نان
&&
نه درو بوی طعام و نه نشان
\\
نی درش معمور نی بر بام راه
&&
نی یکی همسایه کو باشد پناه
\\
چشم تو که بوسه‌گاه خلق بود
&&
چون شود در خانهٔ کور و کبود
\\
خانهٔ بی‌زینهار و جای تنگ
&&
که درو نه روی می‌ماند نه رنگ
\\
زین نسق اوصاف خانه می‌شمرد
&&
وز دو دیده اشک خونین می‌فشرد
\\
گفت جوحی با پدر ای ارجمند
&&
والله این را خانهٔ ما می‌برند
\\
گفت جوحی را پدر ابله مشو
&&
گفت ای بابا نشانیها شنو
\\
این نشانیها که گفت او یک بیک
&&
خانهٔ ما راست بی تردید و شک
\\
نه حصیر و نه چراغ و نه طعام
&&
نه درش معمور و نه صحن و نه بام
\\
زین نمط دارند بر خود صد نشان
&&
لیک کی بینند آن را طاغیان
\\
خانهٔ آن دل که ماند بی ضیا
&&
از شعاع آفتاب کبریا
\\
تنگ و تاریکست چون جان جهود
&&
بی نوا از ذوق سلطان ودود
\\
نه در آن دل تافت نور آفتاب
&&
نه گشاد عرصه و نه فتح باب
\\
گور خوشتر از چنین دل مر ترا
&&
آخر از گور دل خود برتر آ
\\
زنده‌ای و زنده‌زاد ای شوخ و شنگ
&&
دم نمی‌گیرد ترا زین گور تنگ
\\
یوسف وقتی و خورشید سما
&&
زین چه و زندان بر آ و رو نما
\\
یونست در بطن ماهی پخته شد
&&
مخلصش را نیست از تسبیح بد
\\
گر نبودی او مسبح بطن نون
&&
حبس و زندانش بدی تا یبعثون
\\
او بتسبیح از تن ماهی بجست
&&
چیست تسبیح آیت روز الست
\\
گر فراموشت شد آن تسبیح جان
&&
بشنو این تسبیحهای ماهیان
\\
هر که دید الله را اللهیست
&&
هر که دید آن بحر را آن ماهیست
\\
این جهان دریاست و تن ماهی و روح
&&
یونس محجوب از نور صبوح
\\
گر مسبح باشد از ماهی رهید
&&
ورنه در وی هضم گشت و ناپدید
\\
ماهیان جان درین دریا پرند
&&
تو نمی‌بینی که کوری ای نژند
\\
بر تو خود را می‌زنند آن ماهیان
&&
چشم بگشا تا ببینیشان عیان
\\
ماهیان را گر نمی‌بینی پدید
&&
گوش تو تسبیحشان آخر شنید
\\
صبر کردن جان تسبیحات تست
&&
صبر کن کانست تسبیح درست
\\
هیچ تسبیحی ندارد آن درج
&&
صبر کن الصبر مفتاح الفرج
\\
صبر چون پول صراط آن سو بهشت
&&
هست با هر خوب یک لالای زشت
\\
تا ز لالا می‌گریزی وصل نیست
&&
زانک لالا را ز شاهد فصل نیست
\\
تو چه دانی ذوق صبر ای شیشه‌دل
&&
خاصه صبر از بهر آن نقش چگل
\\
مرد را ذوق از غزا و کر و فر
&&
مر مخنث را بود ذوق از ذکر
\\
جز ذکر نه دین او و ذکر او
&&
سوی اسفل برد او را فکر او
\\
گر برآید تا فلک از وی مترس
&&
کو به عشق سفل آموزید درس
\\
او به سوی سفل می‌راند فرس
&&
گرچه سوی علو جنباند جرس
\\
از علمهای گدایان ترس چیست
&&
کان علمها لقمهٔ نان را رهیست
\\
\end{longtable}
\end{center}
