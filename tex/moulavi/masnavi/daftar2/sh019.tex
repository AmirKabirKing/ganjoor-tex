\begin{center}
\section*{بخش ۱۹ - مثل}
\label{sec:sh019}
\addcontentsline{toc}{section}{\nameref{sec:sh019}}
\begin{longtable}{l p{0.5cm} r}
آن غریبی خانه می‌جست از شتاب
&&
دوستی بردش سوی خانهٔ خراب
\\
گفت او این را اگر سقفی بدی
&&
پهلوی من مر ترا مسکن شدی
\\
هم عیال تو بیاسودی اگر
&&
در میانه داشتی حجرهٔ دگر
\\
گفت آری پهلوی یاران بهست
&&
لیک ای جان در اگر نتوان نشست
\\
این همه عالم طلب‌کار خوشند
&&
وز خوش تزویر اندر آتشند
\\
طالب زر گشته جمله پیر و خام
&&
لیک قلب از زر نداند چشم عام
\\
پرتوی بر قلب زد خالص ببین
&&
بی محک زر را مکن از ظن گزین
\\
گر محک داری گزین کن ور نه رو
&&
نزد دانا خویشتن را کن گرو
\\
یا محک باید میان جان خویش
&&
ور ندانی ره مرو تنها تو پیش
\\
بانگ غولان هست بانگ آشنا
&&
آشنایی که کشد سوی فنا
\\
بانگ می‌دارد که هان ای کاروان
&&
سوی من آیید نک راه و نشان
\\
نام هر یک می‌برد غول ای فلان
&&
تا کند آن خواجه را از آفلان
\\
چون رسد آنجا ببیند گرگ و شیر
&&
عمر ضایع راه دور و روز دیر
\\
چون بود آن بانگ غول آخر بگو
&&
مال خواهم جاه خواهم و آب رو
\\
از درون خویش این آوازها
&&
منع کن تا کشف گردد رازها
\\
ذکر حق کن بانگ غولان را بسوز
&&
چشم نرگس را ازین کرکس بدوز
\\
صبح کاذب را ز صادق وا شناس
&&
رنگ می را باز دان از رنگ کاس
\\
تا بود کز دیدگان هفت رنگ
&&
دیده‌ای پیدا کند صبر و درنگ
\\
رنگها بینی به جز این رنگها
&&
گوهران بینی به جای سنگها
\\
گوهر چه بلک دریایی شوی
&&
آفتاب چرخ‌پیمایی شوی
\\
کارکن در کارگه باشد نهان
&&
تو برو در کارگه بینش عیان
\\
کار چون بر کارکن پرده تنید
&&
خارج آن کار نتوانیش دید
\\
کارگه چون جای باش عاملست
&&
آنک بیرونست از وی غافلست
\\
پس در آ در کارگه یعنی عدم
&&
تا ببینی صنع و صانع را بهم
\\
کارگه چون جای روشن‌دیدگیست
&&
پس برون کارگه پوشیدگیست
\\
رو بهستی داشت فرعون عنود
&&
لاجرم از کارگاهش کور بود
\\
لاجرم می‌خواست تبدیل قدر
&&
تا قضا را باز گرداند ز در
\\
خود قضا بر سبلت آن حیله‌مند
&&
زیر لب می‌کرد هر دم ریش‌خند
\\
صد هزاران طفل کشت او بی‌گناه
&&
تا بگردد حکم و تقدیر اله
\\
تا که موسی نبی ناید برون
&&
کرد در گردن هزاران ظلم و خون
\\
آن همه خون کرد و موسی زاده شد
&&
وز برای قهر او آماده شد
\\
گر بدیدی کارگاه لایزال
&&
دست و پایش خشک گشتی ز احتیال
\\
اندرون خانه‌اش موسی معاف
&&
وز برون می‌کشت طفلان را گزاف
\\
همچو صاحب‌نفس کو تن پرورد
&&
بر دگر کس ظن حقدی می‌برد
\\
کین عدو و آن حسود و دشمنست
&&
خود حسود و دشمن او آن تنست
\\
او چو فرعون و تنش موسی او
&&
او به بیرون می‌دود که کو عدو
\\
نفسش اندر خانهٔ تن نازنین
&&
بر دگر کس دست می‌خاید به کین
\\
\end{longtable}
\end{center}
