\begin{center}
\section*{بخش ۹۵ - طعن زدن بیگانه در شیخ و جواب گفتن مرید شیخ او را}
\label{sec:sh095}
\addcontentsline{toc}{section}{\nameref{sec:sh095}}
\begin{longtable}{l p{0.5cm} r}
آن یکی یک شیخ را تهمت نهاد
&&
کو بدست و نیست بر راه رشاد
\\
شارب خمرست و سالوس و خبیث
&&
مر مریدان را کجا باشد مغیث
\\
آن یکی گفتش ادب را هوش دار
&&
خرد نبود این چنین ظن بر کبار
\\
دور ازو و دور از آن اوصاف او
&&
که ز سیلی تیره گردد صاف او
\\
این چنین بهتان منه بر اهل حق
&&
کین خیال تست برگردان ورق
\\
این نباشد ور بود ای مرغ خاک
&&
بحر قلزم را ز مرداری چه باک
\\
نیست دون القلتین و حوض خرد
&&
که تواند قطره‌ایش از کار برد
\\
آتش ابراهیم را نبود زیان
&&
هر که نمرودیست گو می‌ترس از آن
\\
نفس نمرودست و عقل و جان خلیل
&&
روح در عینست و نفس اندر دلیل
\\
این دلیل راه ره‌رو را بود
&&
کو بهر دم در بیابان گم شود
\\
واصلان را نیست جز چشم و چراغ
&&
از دلیل و راهشان باشد فراغ
\\
گر دلیلی گفت آن مرد وصال
&&
گفت بهر فهم اصحاب جدال
\\
بهر طفل نو پدر تی‌تی کند
&&
گرچه عقلش هندسهٔ گیتی کند
\\
کم نگردد فضل استاد از علو
&&
گر الف چیزی ندارد گوید او
\\
از پی تعلیم آن بسته‌دهن
&&
از زبان خود برون باید شدن
\\
در زبان او بباید آمدن
&&
تا بیاموزد ز تو او علم و فن
\\
پس همه خلقان چو طفلان ویند
&&
لازمست این پیر را در وقت پند
\\
آن مرید شیخ بد گوینده را
&&
آن به کفر و گمرهی آکنده را
\\
گفت خود را تو مزن بر تیغ تیز
&&
هین مکن با شاه و با سلطان ستیز
\\
حوض با دریا اگر پهلو زند
&&
خویش را از بیخ هستی بر کند
\\
نیست بحری کو کران دارد که تا
&&
تیره گردد او ز مردار شما
\\
کفر را حدست و اندازه بدان
&&
شیخ و نور شیخ را نبود کران
\\
پیش بی حد هرچه محدودست لاست
&&
کل شیء غیر وجه الله فناست
\\
کفر و ایمان نیست آنجایی که اوست
&&
زانک او مغزست و این دو رنگ و پوست
\\
این فناها پردهٔ آن وجه گشت
&&
چون چراغ خفیه اندر زیر طشت
\\
پس سر این تن حجاب آن سرست
&&
پیش آن سر این سر تن کافرست
\\
کیست کافر غافل از ایمان شیخ
&&
کیست مرده بی خبر از جان شیخ
\\
جان نباشد جز خبر در آزمون
&&
هر که را افزون خبر جانش فزون
\\
جان ما از جان حیوان بیشتر
&&
از چه زان رو که فزون دارد خبر
\\
پس فزون از جان ما جان ملک
&&
کو منزه شد ز حس مشترک
\\
وز ملک جان خداوندان دل
&&
باشد افزون تو تحیر را بهل
\\
زان سبب آدم بود مسجودشان
&&
جان او افزونترست از بودشان
\\
ورنه بهتر را سجود دون‌تری
&&
امر کردن هیچ نبود در خوری
\\
کی پسندد عدل و لطف کردگار
&&
که گلی سجده کند در پیش خار
\\
جان چو افزون شد گذشت از انتها
&&
شد مطیعش جان جمله چیزها
\\
مرغ و ماهی و پری و آدمی
&&
زانک او بیشست و ایشان در کمی
\\
ماهیان سوزن‌گر دلقش شوند
&&
سوزنان را رشته‌ها تابع بوند
\\
\end{longtable}
\end{center}
