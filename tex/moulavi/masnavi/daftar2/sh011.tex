\begin{center}
\section*{بخش ۱۱ - حلوا خریدن شیخ احمد خضرویه جهت غریمان بالهام حق تعالی}
\label{sec:sh011}
\addcontentsline{toc}{section}{\nameref{sec:sh011}}
\begin{longtable}{l p{0.5cm} r}
بود شیخی دایما او وامدار
&&
از جوامردی که بود آن نامدار
\\
ده هزاران وام کردی از مهان
&&
خرج کردی بر فقیران جهان
\\
هم بوام او خانقاهی ساخته
&&
جان و مال و خانقه در باخته
\\
وام او را حق ز هر جا می‌گزارد
&&
کرد حق بهر خلیل از ریگ آرد
\\
گفت پیغامبر که در بازارها
&&
دو فرشته می‌کنند ایدر دعا
\\
کای خدا تو منفقان را ده خلف
&&
ای خدا تو ممسکان را ده تلف
\\
خاصه آن منفق که جان انفاق کرد
&&
حلق خود قربانی خلاق کرد
\\
حلق پیش آورد اسمعیل‌وار
&&
کارد بر حلقش نیارد کرد کار
\\
پس شهیدان زنده زین رویند و خوش
&&
تو بدان قالب بمنگر گبروش
\\
چون خلف دادستشان جان بقا
&&
جان ایمن از غم و رنج و شقا
\\
شیخ وامی سالها این کار کرد
&&
می‌ستد می‌داد همچون پای‌مرد
\\
تخمها می‌کاشت تا روز اجل
&&
تا بود روز اجل میر اجل
\\
چونک عمر شیخ در آخر رسید
&&
در وجود خود نشان مرگ دید
\\
وام‌داران گرد او بنشسته جمع
&&
شیخ بر خود خوش گدازان همچو شمع
\\
وام‌داران گشته نومید و ترش
&&
درد دلها یار شد با درد شش
\\
شیخ گفت این بدگمانان را نگر
&&
نیست حق را چار صد دینار زر
\\
کودکی حلوا ز بیرون بانگ زد
&&
لاف حلوا بر امید دانگ زد
\\
شیخ اشارت کرد خادم را بسر
&&
که برو آن جمله حلوا را بخر
\\
تا غریمان چونک آن حلوا خورند
&&
یک زمانی تلخ در من ننگرند
\\
در زمان خادم برون آمد بدر
&&
تا خرد او جمله حلوا را بزر
\\
گفت او را کوترو حلوا بچند
&&
گفت کودک نیم دینار و ادند
\\
گفت نه از صوفیان افزون مجو
&&
نیم دینارت دهم دیگر مگو
\\
او طبق بنهاد اندر پیش شیخ
&&
تو ببین اسرار سر اندیش شیخ
\\
کرد اشارت با غریمان کین نوال
&&
نک تبرک خوش خورید این را حلال
\\
چون طبق خالی شد آن کودک ستد
&&
گفت دینارم بده ای با خرد
\\
شیخ گفتا از کجا آرم درم
&&
وام دارم می‌روم سوی عدم
\\
کودک از غم زد طبق را بر زمین
&&
ناله و گریه بر آورد و حنین
\\
می‌گریست از غبن کودک های های
&&
کای مرا بشکسته بودی هر دو پای
\\
کاشکی من گرد گلخن گشتمی
&&
بر در این خانقه نگذشتمی
\\
صوفیان طبل‌خوار لقمه‌جو
&&
سگ‌دلان و همچو گربه روی‌شو
\\
از غریو کودک آنجا خیر و شر
&&
گرد آمد گشت بر کودک حشر
\\
پیش شیخ آمد که ای شیخ درشت
&&
تو یقین دان که مرا استاد کشت
\\
گر روم من پیش او دست تهی
&&
او مرا بکشد اجازت می‌دهی
\\
وان غریمان هم بانکار و جحود
&&
رو به شیخ آورده کین باری چه بود
\\
مال ما خوردی مظالم می‌بری
&&
از چه بود این ظلم دیگر بر سری
\\
تا نماز دیگر آن کودک گریست
&&
شیخ دیده بست و در وی ننگریست
\\
شیخ فارغ از جفا و از خلاف
&&
در کشیده روی چون مه در لحاف
\\
با ازل خوش با اجل خوش شادکام
&&
فارغ از تشنیع و گفت خاص و عام
\\
آنک جان در روی او خندد چو قند
&&
از ترش‌رویی خلقش چه گزند
\\
آنک جان بوسه دهد بر چشم او
&&
کی خورد غم از فلک وز خشم او
\\
در شب مهتاب مه را بر سماک
&&
از سگان و وعوع ایشان چه باک
\\
سگ وظیفهٔ خود بجا می‌آورد
&&
مه وظیفهٔ خود برخ می‌گسترد
\\
کارک خود می‌گزارد هر کسی
&&
آب نگذارد صفا بهر خسی
\\
خس خسانه می‌رود بر روی آب
&&
آب صافی می‌رود بی اضطراب
\\
مصطفی مه می‌شکافد نیم‌شب
&&
ژاژ می‌خاید ز کینه بولهب
\\
آن مسیحا مرده زنده می‌کند
&&
وان جهود از خشم سبلت می‌کند
\\
بانگ سگ هرگز رسد در گوش ماه
&&
خاصه ماهی کو بود خاص اله
\\
می خورد شه بر لب جو تا سحر
&&
در سماع از بانگ چغزان بی خبر
\\
هم شدی توزیع کودک دانگ چند
&&
همت شیخ آن سخا را کرد بند
\\
تا کسی ندهد به کودک هیچ چیز
&&
قوت پیران ازین بیش است نیز
\\
شد نماز دیگر آمد خادمی
&&
یک طبق بر کف ز پیش حاتمی
\\
صاحب مالی و حالی پیش پیر
&&
هدیه بفرستاد کز وی بد خبیر
\\
چارصد دینار بر گوشهٔ طبق
&&
نیم دینار دگر اندر ورق
\\
خادم آمد شیخ را اکرام کرد
&&
وان طبق بنهاد پیش شیخ فرد
\\
چون طبق را از غطا وا کرد رو
&&
خلق دیدند آن کرامت را ازو
\\
آه و افغان از همه برخاست زود
&&
کای سر شیخان و شاهان این چه بود
\\
این چه سرست این چه سلطانیست باز
&&
ای خداوند خداوندان راز
\\
ما ندانستیم ما را عفو کن
&&
بس پراکنده که رفت از ما سخن
\\
ما که کورانه عصاها می‌زنیم
&&
لاجرم قندیلها را بشکنیم
\\
ما چو کران ناشنیده یک خطاب
&&
هرزه گویان از قیاس خود جواب
\\
ما ز موسی پند نگرفتیم کو
&&
گشت از انکار خضری زردرو
\\
با چنان چشمی که بالا می‌شتافت
&&
نور چشمش آسمان را می‌شکافت
\\
کرده با چشمت تعصب موسیا
&&
از حماقت چشم موش آسیا
\\
شیخ فرمود آن همه گفتار و قال
&&
من بحل کردم شما را آن حلال
\\
سر این آن بود کز حق خواستم
&&
لاجرم بنمود راه راستم
\\
گفت آن دینار اگر چه اندکست
&&
لیک موقوف غریو کودکست
\\
تا نگرید کودک حلوا فروش
&&
بحر رحمت در نمی‌آید به جوش
\\
ای برادر طفل طفل چشم تست
&&
کام خود موقوف زاری دان درست
\\
گر همی‌خواهی که آن خلعت رسد
&&
پس بگریان طفل دیده بر جسد
\\
\end{longtable}
\end{center}
