\begin{center}
\section*{بخش ۶۹ - باز تقریر ابلیس تلبیس خود را}
\label{sec:sh069}
\addcontentsline{toc}{section}{\nameref{sec:sh069}}
\begin{longtable}{l p{0.5cm} r}
گفت هر مردی که باشد بد گمان
&&
نشنود او راست را با صد نشان
\\
هر درونی که خیال‌اندیش شد
&&
چون دلیل آری خیالش بیش شد
\\
چون سخن در وی رود علت شود
&&
تیغ غازی دزد را آلت شود
\\
پس جواب او سکوتست و سکون
&&
هست با ابله سخن گفتن جنون
\\
تو ز من با حق چه نالی ای سلیم
&&
تو بنال از شر آن نفس لئیم
\\
تو خوری حلوا ترا دنبل شود
&&
تب بگیرد طبع تو مختل شود
\\
بی گنه لعنت کنی ابلیس را
&&
چون نبینی از خود آن تلبیس را
\\
نیست از ابلیس از تست ای غوی
&&
که چو روبه سوی دنبه می‌روی
\\
چونک در سبزه ببینی دنبه‌ها
&&
دام باشد این ندانی تو چرا
\\
زان ندانی کت ز دانش دور کرد
&&
میل دنبه چشم و عقلت کور کرد
\\
حبک الاشیاء یعمیک یصم
&&
نفسک السودا جنت لا تختصم
\\
تو گنه بر من منه کژ کژ مبین
&&
من ز بد بیزارم و از حرص و کین
\\
من بدی کردم پشیمانم هنوز
&&
انتظارم تا دیم گردد تموز
\\
متهم گشتم میان خلق من
&&
فعل خود بر من نهد هر مرد و زن
\\
گرگ بیچاره اگرچه گرسنست
&&
متهم باشد که او در طنطنه‌ست
\\
از ضعیفی چون نتواند راه رفت
&&
خلق گوید تخمه است از لوت زفت
\\
\end{longtable}
\end{center}
