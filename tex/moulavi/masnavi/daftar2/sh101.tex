\begin{center}
\section*{بخش ۱۰۱ - کرامات آن درویش کی در کشتی متهمش کردند}
\label{sec:sh101}
\addcontentsline{toc}{section}{\nameref{sec:sh101}}
\begin{longtable}{l p{0.5cm} r}
بود درویشی درون کشتیی
&&
ساخته از رخت مردی پشتیی
\\
یاوه شد همیان زر او خفته بود
&&
جمله را جستند و او را هم نمود
\\
کین فقیر خفته را جوییم هم
&&
کرد بیدارش ز غم صاحب‌درم
\\
که درین کشتی حرمدان گم شدست
&&
جمله را جستیم نتوانی تو رست
\\
دلق بیرون کن برهنه شو ز دلق
&&
تا ز تو فارغ شود اوهام خلق
\\
گفت یا رب مر غلامت را خسان
&&
متهم کردند فرمان در رسان
\\
چون بدرد آمد دل درویش از آن
&&
سر برون کردند هر سو در زمان
\\
صد هزاران ماهی از دریای ژرف
&&
در دهان هر یکی دری شگرف
\\
صد هزاران ماهی از دریای پر
&&
در دهان هر یکی در و چه در
\\
هر یکی دری خراج ملکتی
&&
کز الهست این ندارد شرکتی
\\
در چند انداخت در کشتی و جست
&&
مر هوا را ساخت کرسی و نشست
\\
خوش مربع چون شهان بر تخت خویش
&&
او فراز اوج و کشتی‌اش بپیش
\\
گفت رو کشتی شما را حق مرا
&&
تا نباشد با شما دزد گدا
\\
تا که را باشد خسارت زین فراق
&&
من خوشم جفت حق و با خلق طاق
\\
نه مرا او تهمت دزدی نهد
&&
نه مهارم را به غمازی دهد
\\
بانگ کردند اهل کشتی کای همام
&&
از چه دادندت چنین عالی مقام
\\
گفت از تهمت نهادن بر فقیر
&&
وز حق‌آزاری پی چیزی حقیر
\\
حاش لله بل ز تعظیم شهان
&&
که نبودم در فقیران بدگمان
\\
آن فقیران لطیف خوش‌نفس
&&
کز پی تعظیمشان آمد عبس
\\
آن فقیری بهر پیچاپیچ نیست
&&
بل پی آن که به جز حق هیچ نیست
\\
متهم چون دارم آنها را که حق
&&
کرد امین مخزن هفتم طبق
\\
متهم نفس است نی عقل شریف
&&
متهم حس است نه نور لطیف
\\
نفس سوفسطایی آمد می‌زنش
&&
کش زدن سازد نه حجت گفتنش
\\
معجزه بیند فروزد آن زمان
&&
بعد از آن گوید خیالی بود آن
\\
ور حقیقت بود آن دید عجب
&&
چون مقیم چشم نامد روز و شب
\\
آن مقیم چشم پاکان می‌بود
&&
نی قرین چشم حیوان می‌شود
\\
کان عجب زین حس دارد عار و ننگ
&&
کی بود طاووس اندر چاه تنگ
\\
تا نگویی مر مرا بسیارگو
&&
من ز صد یک گویم و آن همچو مو
\\
\end{longtable}
\end{center}
