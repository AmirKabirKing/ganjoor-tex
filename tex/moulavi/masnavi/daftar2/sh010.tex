\begin{center}
\section*{بخش ۱۰ - یافتن شاه باز را به خانهٔ کمپیر زن}
\label{sec:sh010}
\addcontentsline{toc}{section}{\nameref{sec:sh010}}
\begin{longtable}{l p{0.5cm} r}
دین نه آن بازیست کو از شه گریخت
&&
سوی آن کمپیر کو می آرد بیخت
\\
تا که تتماجی پزد اولاد را
&&
دید آن باز خوش خوش‌زاد را
\\
پایکش بست و پرش کوتاه کرد
&&
ناخنش ببرید و قوتش کاه کرد
\\
گفت نااهلان نکردندت بساز
&&
پر فزود از حد و ناخن شد دراز
\\
دست هر نااهل بیمارت کند
&&
سوی مادر آ که تیمارت کند
\\
مهر جاهل را چنین دان ای رفیق
&&
کژ رود جاهل همیشه در طریق
\\
روز شه در جست و جو بیگاه شد
&&
سوی آن کمپیر و آن خرگاه شد
\\
دید ناگه باز را در دود و گرد
&&
شه برو بگریست زار و نوحه کرد
\\
گفت هرچند این جزای کار تست
&&
که نباشی در وفای ما درست
\\
چون کنی از خلد زی دوزخ فرار
&&
غافل از لا یستوی اصحاب نار
\\
این سزای آنک از شاه خبیر
&&
خیره بگریزد بخانهٔ گنده‌پیر
\\
باز می‌مالید پر بر دست شاه
&&
بی زبان می‌گفت من کردم گناه
\\
پس کجا زارد کجا نالد لئیم
&&
گر تو نپذیری به جز نیک ای کریم
\\
لطف شه جان را جنایت‌جو کند
&&
زانک شه هر زشت را نیکو کند
\\
رو مکن زشتی که نیکیهای ما
&&
زشت آمد پیش آن زیبای ما
\\
خدمت خود را سزا پنداشتی
&&
تو لوای جرم از آن افراشتی
\\
چون ترا ذکر و دعا دستور شد
&&
زان دعا کردن دلت مغرور شد
\\
هم‌سخن دیدی تو خود را با خدا
&&
ای بسا کو زین گمان افتد جدا
\\
گرچه با تو شه نشیند بر زمین
&&
خویشتن بشناس و نیکوتر نشین
\\
باز گفت ای شه پشیمان می‌شوم
&&
توبه کردم نو مسلمان می‌شوم
\\
آنک تو مستش کنی و شیرگیر
&&
گر ز مستی کژ رود عذرش پذیر
\\
گرچه ناخن رفت چون باشی مرا
&&
بر کنم من پرچم خورشید را
\\
ورچه پرم رفت چون بنوازیم
&&
چرخ بازی گم کند در بازیم
\\
گر کمر بخشیم که را بر کنم
&&
گر دهی کلکی علمها بشکنم
\\
آخر از پشه نه کم باشد تنم
&&
ملک نمرودی به پر برهم زنم
\\
در ضعیفی تو مرا بابیل گیر
&&
هر یکی خصم مرا چون پیل گیر
\\
قدر فندق افکنم بندق حریق
&&
بندقم در فعل صد چون منجنیق
\\
گرچه سنگم هست مقدار نخود
&&
لیک در هیجا نه سر ماند نه خود
\\
موسی آمد در وغا با یک عصاش
&&
زد بر آن فرعون و بر شمشیرهاش
\\
هر رسولی یک‌تنه کان در زدست
&&
بر همه آفاق تنها بر زدست
\\
نوح چون شمشیر در خواهید ازو
&&
موج طوفان گشت ازو شمشیرخو
\\
احمدا خود کیست اسپاه زمین
&&
ماه بین بر چرخ و بشکافش جبین
\\
تا بداند سعد و نحس بی‌خبر
&&
دور تست این دور نه دور قمر
\\
دور تست ایرا که موسی کلیم
&&
آرزو می‌برد زین دورت مقیم
\\
چونک موسی رونق دور تو دید
&&
کاندرو صبح تجلی می‌دمید
\\
گفت یا رب آن چه دور رحمتست
&&
آن گذشت از رحمت آنجا رؤیتست
\\
غوطه ده موسی خود را در بحار
&&
از میان دورهٔ احمد بر آر
\\
گفت یا موسی بدان بنمودمت
&&
راه آن خلوت بدان بگشودمت
\\
که تو زان دوری درین دور ای کلیم
&&
پا بکش زیرا درازست این گلیم
\\
من کریمم نان نمایم بنده را
&&
تا بگریاند طمع آن زنده را
\\
بینی طفلی بمالد مادری
&&
تا شود بیدار و وا جوید خوری
\\
کو گرسنه خفته باشد بی‌خبر
&&
وان دو پستان می‌خلد زو مهر در
\\
کنت کنزا رحمة مخفیة
&&
فابتعثت امة مهدیة
\\
هر کراماتی که می‌جویی بجان
&&
او نمودت تا طمع کردی در آن
\\
چند بت بشکست احمد در جهان
&&
تا که یا رب گوی گشتند امتان
\\
گر نبودی کوشش احمد تو هم
&&
می‌پرستیدی چو اجدادت صنم
\\
این سرت وا رست از سجدهٔ صنم
&&
تا بدانی حق او را بر امم
\\
گر بگویی شکر این رستن بگو
&&
کز بت باطن همت برهاند او
\\
مر سرت را چون رهانید از بتان
&&
هم بدان قوت تو دل را وا رهان
\\
سر ز شکر دین از آن برتافتی
&&
کز پدر میراث مفتش یافتی
\\
مرد میراثی چه داند قدر مال
&&
رستمی جان کند و مجان یافت زال
\\
چون بگریانم بجوشد رحمتم
&&
آن خروشنده بنوشد نعمتم
\\
گر نخواهم داد خود ننمایمش
&&
چونش کردم بسته دل بگشایمش
\\
رحمتم موقوف آن خوش گریه‌هاست
&&
چون گریست از بحر رحمت موج خاست
\\
\end{longtable}
\end{center}
