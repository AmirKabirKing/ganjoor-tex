\begin{center}
\section*{بخش ۳۸ - پرسیدن موسی از حق سر غلبهٔ ظالمان را}
\label{sec:sh038}
\addcontentsline{toc}{section}{\nameref{sec:sh038}}
\begin{longtable}{l p{0.5cm} r}
گفت موسی ای کریم کارساز
&&
ای که یکدم ذکر تو عمر دراز
\\
نقش کژمژ دیدم اندر آب و گل
&&
چون ملایک اعتراضی کرد دل
\\
که چه مقصودست نقشی ساختن
&&
واندرو تخم فساد انداختن
\\
آتش ظلم و فساد افروختن
&&
مسجد و سجده‌کنان را سوختن
\\
مایهٔ خونابه و زردآبه را
&&
جوش دادن از برای لابه را
\\
من یقین دانم که عین حکمتست
&&
لیک مقصودم عیان و رؤیتست
\\
آن یقین می‌گویدم خاموش کن
&&
حرص رؤیت گویدم نه جوش کن
\\
مر ملایک را نمودی سر خویش
&&
کین چنین نوشی همی ارزد به نیش
\\
عرضه کردی نور آدم را عیان
&&
بر ملایک گشت مشکلها بیان
\\
حشر تو گوید که سر مرگ چیست
&&
میوه‌ها گویند سر برگ چیست
\\
سر خون و نطفهٔ حسن آدمیست
&&
سابق هر بیشیی آخر کمیست
\\
لوح را اول بشوید بی وقوف
&&
آنگهی بر وی نویسد او حروف
\\
خون کند دل را و اشک مستهان
&&
بر نویسد بر وی اسرار آنگهان
\\
وقت شستن لوح را باید شناخت
&&
که مر آن را دفتری خواهند ساخت
\\
چون اساس خانه‌ای می‌افکنند
&&
اولین بنیاد را بر می‌کنند
\\
گل بر آرند اول از قعر زمین
&&
تا بخر بر کشی ماء معین
\\
از حجامت کودکان گریند زار
&&
که نمی‌دانند ایشان سر کار
\\
مرد خود زر می‌دهد حجام را
&&
می‌نوازد نیش خون آشام را
\\
مدود حمال زی بار گران
&&
می‌رباید بار را از دیگران
\\
جنگ حمالان برای بار بین
&&
این چنین است اجتهاد کاربین
\\
چون گرانیها اساس راحتست
&&
تلخها هم پیشوای نعمتست
\\
حفت الجنه بمکروهاتنا
&&
حفت النیران من شهواتنا
\\
تخم مایهٔ آتشت شاخ ترست
&&
سوختهٔ آتش قرین کوثرست
\\
هر که در زندان قرین محنتیست
&&
آن جزای لقمه‌ای و شهوتیست
\\
هر که در قصری قرین دولتیست
&&
آن جزای کارزار و محنتیست
\\
هر که را دیدی بزر و سیم فرد
&&
دانک اندر کسب کردن صبر کرد
\\
بی سبب بیند چو دیده شد گذار
&&
تو که در حسی سبب را گوش دار
\\
آنک بیرون از طبایع جان اوست
&&
منصب خرق سببها آن اوست
\\
بی سبب بیند نه از آب و گیا
&&
چشم چشمهٔ معجزات انبیا
\\
این سبب همچون طبیب است و علیل
&&
این سبب همچون چراغست و فتیل
\\
شب چراغت را فتیل نو بتاب
&&
پاک دان زینها چراغ آفتاب
\\
رو تو کهگل ساز بهر سقف خان
&&
سقف گردون را ز کهگل پاک دان
\\
اه که چون دلدار ما غمسوز شد
&&
خلوت شب در گذشت و روز شد
\\
جز بشب جلوه نباشد ماه را
&&
جز بدرد دل مجو دلخواه را
\\
ترک عیسی کرده خر پروده‌ای
&&
لاجرم چون خر برون پرده‌ای
\\
طالع عیسیست علم و معرفت
&&
طالع خر نیست ای تو خر صفت
\\
نالهٔ خر بشنوی رحم آیدت
&&
پس ندانی خر خری فرمایدت
\\
رحم بر عیسی کن و بر خر مکن
&&
طبع را بر عقل خود سرور مکن
\\
طبع را هل تا بگرید زار زار
&&
تو ازو بستان و وام جان گزار
\\
سالها خر بنده بودی بس بود
&&
زانک خربنده ز خر واپس بود
\\
ز اخروهن مرادش نفس تست
&&
کو بخر باید و عقلت نخست
\\
هم‌مزاج خر شدست این عقل پست
&&
فکرش این که چون علف آرم به دست
\\
آن خر عیسی مزاج دل گرفت
&&
در مقام عاقلان منزل گرفت
\\
زانک غالب عقل بود و خر ضعیف
&&
از سوار زفت گردد خر نحیف
\\
وز ضعیفی عقل تو ای خربها
&&
این خر پژمرده گشتست اژدها
\\
گر ز عیسی گشته‌ای رنجوردل
&&
هم ازو صحت رسد او را مهل
\\
چونی ای عیسی عیسی‌دم ز رنج
&&
که نبود اندر جهان بی مار گنج
\\
چونی ای عیسی ز دیدار جهود
&&
چونی ای یوسف ز مکار و حسود
\\
تو شب و روز از پی این قوم غمر
&&
چون شب و روزی مددبخشای عمر
\\
چونی از صفراییان بی‌هنر
&&
چه هنر زاید ز صفرا درد سر
\\
تو همان کن که کند خورشید شرق
&&
ما نفاق و حیله و دزدی و زرق
\\
تو عسل ما سرکه در دنیا و دین
&&
دفع این صفرا بود سرکنگبین
\\
سرکه افزودیم ما قوم زحیر
&&
تو عسل بفزا کرم را وا مگیر
\\
این سزید از ما چنان آمد ز ما
&&
ریگ اندر چشم چه فزاید عمی
\\
آن سزد از تو ایا کحل عزیز
&&
که بیابد از تو هر ناچیز چیز
\\
ز آتش این ظالمانت دل کباب
&&
از تو جمله اهد قومی بد خطاب
\\
کان عودی در تو گر آتش زنند
&&
این جهان از عطر و ریحان آگنند
\\
تو نه آن عودی کز آتش کم شود
&&
تو نه آن روحی که اسیر غم شود
\\
عود سوزد کان عود از سوز دور
&&
باد کی حمله برد بر اصل نور
\\
ای ز تو مر آسمانها را صفا
&&
ای جفای تو نکوتر از وفا
\\
زانک از عاقل جفایی گر رود
&&
از وفای جاهلان آن به بود
\\
گفت پیغامبر عداوت از خرد
&&
بهتر از مهری که از جاهل رسد
\\
\end{longtable}
\end{center}
