\begin{center}
\section*{بخش ۵۳ - حکایت}
\label{sec:sh053}
\addcontentsline{toc}{section}{\nameref{sec:sh053}}
\begin{longtable}{l p{0.5cm} r}
خانه‌ای نو ساخت روزی نو مرید
&&
پیر آمد خانهٔ او را بدید
\\
گفت شیخ آن نو مرید خویش را
&&
امتحان کرد آن نکو اندیش را
\\
روزن از بهر چه کردی ای رفیق
&&
گفت تا نور اندر آید زین طریق
\\
گفت آن فرعست این باید نیاز
&&
تا ازین ره بشنوی بانگ نماز
\\
بایزید اندر سفر جستی بسی
&&
تا بیابد خضر وقت خود کسی
\\
دید پیری با قدی همچون هلال
&&
دید در وی فر و گفتار رجال
\\
دیده نابینا و دل چون آفتاب
&&
همچو پیلی دیده هندستان به خواب
\\
چشم بسته خفته بیند صد طرب
&&
چون گشاید آن نبیند ای عجب
\\
بس عجب در خواب روشن می‌شود
&&
دل درون خواب روزن می‌شود
\\
آنک بیدارست و بیند خواب خوش
&&
عارفست او خاک او در دیده‌کش
\\
پیش او بنشست و می‌پرسید حال
&&
یافتش درویش و هم صاحب‌عیال
\\
گفت عزم تو کجا ای بایزید
&&
رخت غربت را کجا خواهی کشید
\\
گفت قصد کعبه دارم از پگه
&&
گفت هین با خود چه داری زاد ره
\\
گفت دارم از درم نقره دویست
&&
نک ببسته سخت بر گوشهٔ ردیست
\\
گفت طوفی کن بگردم هفت بار
&&
وین نکوتر از طواف حج شمار
\\
و آن درمها پیش من نه ای جواد
&&
دان که حج کردی و حاصل شد مراد
\\
عمره کردی عمر باقی یافتی
&&
صاف گشتی بر صفا بشتافتی
\\
حق آن حقی که جانت دیده است
&&
که مرا بر بیت خود بگزیده است
\\
کعبه هرچندی که خانهٔ بر اوست
&&
خلقت من نیز خانهٔ سر اوست
\\
تا بکرد آن خانه را در وی نرفت
&&
واندرین خانه به جز آن حی نرفت
\\
چون مرا دیدی خدا را دیده‌ای
&&
گرد کعبهٔ صدق بر گردیده‌ای
\\
خدمت من طاعت و حمد خداست
&&
تا نپنداری که حق از من جداست
\\
چشم نیکو باز کن در من نگر
&&
تا ببینی نور حق اندر بشر
\\
بایزید آن نکته‌ها را هوش داشت
&&
همچو زرین حلقه‌اش در گوش داشت
\\
آمد از وی بایزید اندر مزید
&&
منتهی در منتها آخر رسید
\\
\end{longtable}
\end{center}
