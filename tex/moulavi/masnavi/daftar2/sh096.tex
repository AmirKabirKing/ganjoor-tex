\begin{center}
\section*{بخش ۹۶ - بقیهٔ قصهٔ ابراهیم ادهم بر لب آن دریا}
\label{sec:sh096}
\addcontentsline{toc}{section}{\nameref{sec:sh096}}
\begin{longtable}{l p{0.5cm} r}
چون نفاذ امر شیخ آن میر دید
&&
ز آمد ماهی شدش وجدی پدید
\\
گفت اه ماهی ز پیران آگهست
&&
شه تنی را کو لعین درگهست
\\
ماهیان از پیر آگه ما بعید
&&
ما شقی زین دولت و ایشان سعید
\\
سجده کرد و رفت گریان و خراب
&&
گشت دیوانه ز عشق فتح باب
\\
پس تو ای ناشسته‌رو در چیستی
&&
در نزاع و در حسد با کیستی
\\
با دم شیری تو بازی می‌کنی
&&
بر ملایک ترک‌تازی می‌کنی
\\
بد چه می‌گویی تو خیر محض را
&&
هین ترفع کم شمر آن خفض را
\\
بد چه باشد مس محتاج مهان
&&
شیخ کی بود کیمیای بی‌کران
\\
مس اگر از کیمیا قابل نبد
&&
کیمیا از مس هرگز مس نشد
\\
بد چه باشد سرکشی آتش‌عمل
&&
شیخ کی بود عین دریای ازل
\\
دایم آتش را بترسانند از آب
&&
آب کی ترسید هرگز ز التهاب
\\
در رخ مه عیب‌بینی می‌کنی
&&
در بهشتی خارچینی می‌کنی
\\
گر بهشت اندر روی تو خارجو
&&
هیچ خار آنجا نیابی غیر تو
\\
می‌بپوشی آفتابی در گلی
&&
رخنه می‌جویی ز بدر کاملی
\\
آفتابی که بتابد در جهان
&&
بهر خفاشی کجا گردد نهان
\\
عیبها از رد پیران عیب شد
&&
غیبها از رشک ایشان غیب شد
\\
باری ار دوری ز خدمت یار باش
&&
در ندامت چابک و بر کار باش
\\
تا از آن راهت نسیمی می‌رسد
&&
آب رحمت را چه بندی از حسد
\\
گرچه دوری دور می‌جنبان تو دم
&&
حیث ما کنتم فولوا وجهکم
\\
چون خری در گل فتد از گام تیز
&&
دم بدم جنبد برای عزم خیز
\\
جای را هموار نکند بهر باش
&&
داند او که نیست آن جای معاش
\\
حس تو از حس خر کمتر بدست
&&
که دل تو زین وحلها بر نجست
\\
در وحل تاویل و رخصت می‌کنی
&&
چون نمی‌خواهی کز آن دل بر کنی
\\
کین روا باشد مرا من مضطرم
&&
حق نگیرد عاجزی را از کرم
\\
خود گرفتستت تو چون کفتار کور
&&
این گرفتن را نبینی از غرور
\\
می‌گوند اینجایگه کفتار نیست
&&
از برون جویید کاندر غار نیست
\\
این همی‌گویند و بندش می‌نهند
&&
او همی‌گوید ز من بی آگهند
\\
گر ز من آگاه بودی این عدو
&&
کی ندا کردی که آن کفتار کو
\\
\end{longtable}
\end{center}
