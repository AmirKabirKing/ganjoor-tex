\begin{center}
\section*{بخش ۶ - حکایت مشورت کردن خدای تعالی  در ایجاد خلق}
\label{sec:sh006}
\addcontentsline{toc}{section}{\nameref{sec:sh006}}
\begin{longtable}{l p{0.5cm} r}
مشورت می‌رفت در ایجاد خلق
&&
جانشان در بحر قدرت تا به حلق
\\
چون ملایک مانع آن می‌شدند
&&
بر ملایک خفیه خنبک می‌زدند
\\
مطلع بر نقش هر که هست شد
&&
پیش از آن کین نفس کل پابست شد
\\
پیشتر ز افلاک کیوان دیده‌اند
&&
پیشتر از دانه‌ها نان دیده‌اند
\\
بی دماغ و دل پر از فکرت بدند
&&
بی سپاه و جنگ بر نصرت زدند
\\
آن عیان نسبت بایشان فکرتست
&&
ورنه خود نسبت بدوران رؤیتست
\\
فکرت از ماضی و مستقبل بود
&&
چون ازین دو رست مشکل حل شود
\\
دیده چون بی‌کیف هر باکیف را
&&
دیده پیش از کان صحیح و زیف را
\\
پیشتر از خلقت انگورها
&&
خورده میها و نموده شورها
\\
در تموز گرم می‌بینند دی
&&
در شعاع شمس می‌بینند فی
\\
در دل انگور می را دیده‌اند
&&
در فنای محض شی را دیده‌اند
\\
آسمان در دور ایشان جرعه‌نوش
&&
آفتاب از جودشان زربفت‌پوش
\\
چون ازیشان مجتمع بینی دو یار
&&
هم یکی باشند و هم ششصد هزار
\\
بر مثال موجها اعدادشان
&&
در عدد آورده باشد بادشان
\\
مفترق شد آفتاب جانها
&&
در درون روزن ابدان ما
\\
چون نظر در قرص داری خود یکیست
&&
وانک شد محجوب ابدان در شکیست
\\
تفرقه در روح حیوانی بود
&&
نفس واحد روح انسانی بود
\\
چونک حق رش علیهم نوره
&&
مفترق هرگز نگردد نور او
\\
یک زمان بگذار ای همره ملال
&&
تا بگویم وصف خالی زان جمال
\\
در بیان ناید جمال حال او
&&
هر دو عالم چیست عکس خال او
\\
چونک من از خال خوبش دم زنم
&&
نطق می‌خواهد که بشکافد تنم
\\
همچو موری اندرین خرمن خوشم
&&
تا فزون از خویش باری می‌کشم
\\
\end{longtable}
\end{center}
