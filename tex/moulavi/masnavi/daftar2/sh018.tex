\begin{center}
\section*{بخش ۱۸ - تتمهٔ قصهٔ مفلس}
\label{sec:sh018}
\addcontentsline{toc}{section}{\nameref{sec:sh018}}
\begin{longtable}{l p{0.5cm} r}
گفت قاضی مفلسی را وا نما
&&
گفت اینک اهل زندانت گوا
\\
گفت ایشان متهم باشند چون
&&
می‌گریزند از تو می‌گریند خون
\\
وز تو می‌خواهند هم تا وارهند
&&
زین غرض باطل گواهی می‌دهند
\\
جمله اهل محکمه گفتند ما
&&
هم بر ادبار و بر افلاسش گوا
\\
هر که را پرسید قاضی حال او
&&
گفت مولا دست ازین مفلس بشو
\\
گفت قاضی کش بگردانید فاش
&&
گرد شهر این مفلس است و بس قلاش
\\
کو بکو او را منادیها زنید
&&
طبل افلاسش عیان هر جا زنید
\\
هیچ کس نسیه بنفروشد بدو
&&
قرض ندهد هیچ کس او را تسو
\\
هر که دعوی آردش اینجا بفن
&&
بیش زندانش نخواهم کرد من
\\
پیش من افلاس او ثابت شدست
&&
نقد و کالا نیستش چیزی بدست
\\
آدمی در حبس دنیا زان بود
&&
تا بود کافلاس او ثابت شود
\\
مفلسی دیو را یزدان ما
&&
هم منادی کرد در قرآن ما
\\
کو دغا و مفلس است و بد سخن
&&
هیچ با او شرکت و سودا مکن
\\
ور کنی او را بهانه آوری
&&
مفلس است او صرفه از وی کی بری
\\
حاضر آوردند چون فتنه فروخت
&&
اشتر کردی که هیزم می‌فروخت
\\
کرد بیچاره بسی فریاد کرد
&&
هم موکل را به دانگی شاد کرد
\\
اشترش بردند از هنگام چاشت
&&
تا شب و افغان او سودی نداشت
\\
بر شتر بنشست آن قحط گران
&&
صاحب اشتر پی اشتر دوان
\\
سو بسو و کو بکو می‌تاختند
&&
تا همه شهرش عیان بشناختند
\\
پیش هر حمام و هر بازارگه
&&
کرده مردم جمله در شکلش نگه
\\
ده منادی‌گر بلند آوازیان
&&
ترک و کرد و رومیان و تازیان
\\
مفلس است این و ندارد هیچ چیز
&&
قرض تا ندهد کس او را یک پشیز
\\
ظاهر و باطن ندارد حبه‌ای
&&
مفلسی قلبی دغایی دبه‌ای
\\
هان و هان با او حریفی کم کنید
&&
چونک گاو آرد گره محکم کنید
\\
ور بحکم آرید این پژمرده را
&&
من نخواهم کرد زندان مرده را
\\
خوش دمست او و گلویش بس فراخ
&&
با شعار نو دثار شاخ شاخ
\\
گر بپوشد بهر مکر آن جامه را
&&
عاریه‌ست آن تا فریبد عامه را
\\
حرف حکمت بر زبان ناحکیم
&&
حله‌های عاریت دان ای سلیم
\\
گرچه دزدی حله‌ای پوشیده است
&&
دست تو چون گیرد آن ببریده‌دست
\\
چون شبانه از شتر آمد به زیر
&&
کرد گفتش منزلم دورست و دیر
\\
بر نشستی اشترم را از پگاه
&&
جو رها کردم کم از اخراج کاه
\\
گفت تا اکنون چه می‌کردیم پس
&&
هوش تو کو نیست اندر خانه کس
\\
طبل افلاسم به چرخ سابعه
&&
رفت و تو نشنیده‌ای بد واقعه
\\
گوش تو پر بوده است از طمع خام
&&
پس طمع کر می‌کند کور ای غلام
\\
تا کلوخ و سنگ بشنید این بیان
&&
مفلسست و مفلسست این قلتبان
\\
تا بشب گفتند و در صاحب شتر
&&
بر نزد کو از طمع پر بود پر
\\
هست بر سمع و بصر مهر خدا
&&
در حجب بس صورتست و بس صدا
\\
آنچ او خواهد رساند آن به چشم
&&
از جمال و از کمال و از کرشم
\\
و آنچ او خواهد رساند آن به گوش
&&
از سماع و از بشارت وز خروش
\\
کون پر چاره‌ست هیچت چاره نی
&&
تا که نگشاید خدایت روزنی
\\
گرچه تو هستی کنون غافل از آن
&&
وقت حاجت حق کند آن را عیان
\\
گفت پیغامبر که یزدان مجید
&&
از پی هر درد درمان آفرید
\\
لیک زان درمان نبینی رنگ و بو
&&
بهر درد خویش بی فرمان او
\\
چشم را ای چاره‌جو در لامکان
&&
هین بنه چون چشم کشته سوی جان
\\
این جهان از بی جهت پیدا شدست
&&
که ز بی‌جایی جهان را جا شدست
\\
باز گرد از هست سوی نیستی
&&
طالب ربی و ربانیستی
\\
جای دخلست این عدم از وی مرم
&&
جای خرجست این وجود بیش و کم
\\
کارگاه صنع حق چون نیستیست
&&
جز معطل در جهان هست کیست
\\
یاد ده ما را سخنهای دقیق
&&
که ترا رحم آورد آن ای رفیق
\\
هم دعا از تو اجابت هم ز تو
&&
ایمنی از تو مهابت هم ز تو
\\
گر خطا گفتیم اصلاحش تو کن
&&
مصلحی تو ای تو سلطان سخن
\\
کیمیا داری که تبدیلش کنی
&&
گرچه جوی خون بود نیلش کنی
\\
این چنین میناگریها کار تست
&&
این چنین اکسیرها اسرار تست
\\
آب را و خاک را بر هم زدی
&&
ز آب و گل نقش تن آدم زدی
\\
نسبتش دادی و جفت و خال و عم
&&
با هزار اندیشه و شادی و غم
\\
باز بعضی را رهایی داده‌ای
&&
زین غم و شادی جدایی داده‌ای
\\
برده‌ای از خویش و پیوند و سرشت
&&
کرده‌ای در چشم او هر خوب زشت
\\
هر چه محسوس است او رد می‌کند
&&
وانچ ناپیداست مسند می‌کند
\\
عشق او پیدا و معشوقش نهان
&&
یار بیرون فتنهٔ او در جهان
\\
این رها کن عشقهای صورتی
&&
نیست بر صورت نه بر روی ستی
\\
آنچ معشوقست صورت نیست آن
&&
خواه عشق این جهان خواه آن جهان
\\
آنچ بر صورت تو عاشق گشته‌ای
&&
چون برون شد جان چرایش هشته‌ای
\\
صورتش بر جاست این سیری ز چیست
&&
عاشقا وا جو که معشوق تو کیست
\\
آنچ محسوسست اگر معشوقه است
&&
عاشقستی هر که او را حس هست
\\
چون وفا آن عشق افزون می‌کند
&&
کی وفا صورت دگرگون می‌کند
\\
پرتو خورشید بر دیوار تافت
&&
تابش عاریتی دیوار یافت
\\
بر کلوخی دل چه بندی ای سلیم
&&
وا طلب اصلی که تابد او مقیم
\\
ای که تو هم عاشقی بر عقل خویش
&&
خویش بر صورت‌پرستان دیده بیش
\\
پرتو عقلست آن بر حس تو
&&
عاریت می‌دان ذهب بر مس تو
\\
چون زراندودست خوبی در بشر
&&
ورنه چون شد شاهد تر پیره خر
\\
چون فرشته بود همچون دیو شد
&&
کان ملاحت اندرو عاریه بد
\\
اندک اندک می‌ستانند آن جمال
&&
اندک اندک خشک می‌گردد نهال
\\
رو نعمره ننکسه بخوان
&&
دل طلب کن دل منه بر استخوان
\\
کان جمال دل جمال باقیست
&&
دولتش از آب حیوان ساقیست
\\
خود هم او آبست و هم ساقی و مست
&&
هر سه یک شد چون طلسم تو شکست
\\
آن یکی را تو ندانی از قیاس
&&
بندگی کن ژاژ کم خا ناشناس
\\
معنی تو صورتست و عاریت
&&
بر مناسب شادی و بر قافیت
\\
معنی آن باشد که بستاند ترا
&&
بی نیاز از نقش گرداند ترا
\\
معنی آن نبود که کور و کر کند
&&
مرد را بر نقش عاشق‌تر کند
\\
کور را قسمت خیال غم‌فزاست
&&
بهرهٔ چشم این خیالات فناست
\\
حرف قرآن را ضریران معدنند
&&
خر نبینند و به پالان بر زنند
\\
چون تو بینایی پی خر رو که جست
&&
چند پالان دوزی ای پالان‌پرست
\\
خر چو هست آید یقین پالان ترا
&&
کم نگردد نان چو باشد جان ترا
\\
پشت خر دکان و مال و مکسبست
&&
در قلبت مایهٔ صد قالبست
\\
خر برهنه بر نشین ای بوالفضول
&&
خر برهنه نی که راکب شد رسول
\\
النبی قد رکب معروریا
&&
والنبی قیل سافر ماشیا
\\
شد خر نفس تو بر میخیش بند
&&
چند بگریزد ز کار و بار چند
\\
بار صبر و شکر او را بردنیست
&&
خواه در صد سال و خواهی سی و بیست
\\
هیچ وازر وزر غیری بر نداشت
&&
هیچ کس ندرود تا چیزی نکاشت
\\
طمع خامست آن مخور خام ای پسر
&&
خام خوردن علت آرد در بشر
\\
کان فلانی یافت گنجی ناگهان
&&
من همان خواهم مه کار و مه دکان
\\
کار بختست آن و آن هم نادرست
&&
کسب باید کرد تا تن قادرست
\\
کسب کردن گنج را مانع کیست
&&
پا مکش از کار آن خود در پیست
\\
تا نگردی تو گرفتار اگر
&&
که اگر این کردمی یا آن دگر
\\
کز اگر گفتن رسول با وفاق
&&
منع کرد و گفت آن هست از نفاق
\\
کان منافق در اگر گفتن بمرد
&&
وز اگر گفتن به جز حسرت نبرد
\\
\end{longtable}
\end{center}
