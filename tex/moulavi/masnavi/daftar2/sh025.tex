\begin{center}
\section*{بخش ۲۵ - کلوخ انداختن تشنه از سر دیوار در جوی آب}
\label{sec:sh025}
\addcontentsline{toc}{section}{\nameref{sec:sh025}}
\begin{longtable}{l p{0.5cm} r}
بر لب جو بوده دیواری بلند
&&
بر سر دیوار تشنهٔ دردمند
\\
مانعش از آب آن دیوار بود
&&
از پی آب او چو ماهی زار بود
\\
ناگهان انداخت او خشتی در آب
&&
بانگ آب آمد به گوشش چون خطاب
\\
چون خطاب یار شیرین لذیذ
&&
مست کرد آن بانگ آبش چون نبیذ
\\
از صفای بانگ آب آن ممتحن
&&
گشت خشت‌انداز از آنجا خشت‌کن
\\
آب می‌زد بانگ یعنی هی ترا
&&
فایده چه زین زدن خشتی مرا
\\
تشنه گفت آبا مرا دو فایده‌ست
&&
من ازین صنعت ندارم هیچ دست
\\
فایدهٔ اول سماع بانگ آب
&&
کو بود مر تشنگان را چون رباب
\\
بانگ او چون بانگ اسرافیل شد
&&
مرده را زین زندگی تحویل شد
\\
یا چو بانگ رعد ایام بهار
&&
باغ می‌یابد ازو چندین نگار
\\
یا چو بر درویش ایام زکات
&&
یا چو بر محبوس پیغام نجات
\\
چون دم رحمان بود کان از یمن
&&
می‌رسد سوی محمد بی دهن
\\
یا چو بوی احمد مرسل بود
&&
کان به عاصی در شفاعت می‌رسد
\\
یا چو بوی یوسف خوب لطیف
&&
می‌زند بر جان یعقوب نحیف
\\
فایدهٔ دیگر که هر خشتی کزین
&&
بر کنم آیم سوی ماء معین
\\
کز کمی خشت دیوار بلند
&&
پست‌تر گردد بهر دفعه که کند
\\
پستی دیوار قربی می‌شود
&&
فصل او درمان وصلی می‌بود
\\
سجده آمد کندن خشت لزب
&&
موجب قربی که واسجد واقترب
\\
تا که این دیوار عالی‌گردنست
&&
مانع این سر فرود آوردنست
\\
سجده نتوان کرد بر آب حیات
&&
تا نیابم زین تن خاکی نجات
\\
بر سر دیوار هر کو تشنه‌تر
&&
زودتر بر می‌کند خشت و مدر
\\
هر که عاشقتر بود بر بانگ آب
&&
او کلوخ زفت‌تر کند از حجاب
\\
او ز بانگ آب پر می تا عنق
&&
نشنود بیگانه جز بانگ بلق
\\
ای خنک آن را که او ایام پیش
&&
مغتنم دارد گزارد وام خویش
\\
اندر آن ایام کش قدرت بود
&&
صحت و زور دل و قوت بود
\\
وان جوانی همچو باغ سبز و تر
&&
می‌رساند بی دریغی بار و بر
\\
چشمه‌های قوت و شهوت روان
&&
سبز می‌گردد زمین تن بدان
\\
خانهٔ معمور و سقفش بس بلند
&&
معتدل ارکان و بی تخلیط و بند
\\
پیش از آن کایام پیری در رسد
&&
گردنت بندد به حبل من مسد
\\
خاک شوره گردد و ریزان و سست
&&
هرگز از شوره نبات خوش نرست
\\
آب زور و آب شهوت منقطع
&&
او ز خویش و دیگران نا منتفع
\\
ابروان چون پالدم زیر آمده
&&
چشم را نم آمده تاری شده
\\
از تشنج رو چو پشت سوسمار
&&
رفته نطق و طعم و دندانها ز کار
\\
روز بیگه لاشه لنگ و ره دراز
&&
کارگه ویران عمل رفته ز ساز
\\
بیخهای خوی بد محکم شده
&&
قوت بر کندن آن کم شده
\\
\end{longtable}
\end{center}
