\begin{center}
\section*{بخش ۳۶ - عتاب کردن حق تعالی موسی را علیه  السلام از بهر آن شبان}
\label{sec:sh036}
\addcontentsline{toc}{section}{\nameref{sec:sh036}}
\begin{longtable}{l p{0.5cm} r}
وحی آمد سوی موسی از خدا
&&
بندهٔ ما را ز ما کردی جدا
\\
تو برای وصل کردن آمدی
&&
یا برای فصل کردن آمدی
\\
تا توانی پا منه اندر فراق
&&
ابغض الاشیاء عندی الطلاق
\\
هر کسی را سیرتی بنهاده‌ام
&&
هر کسی را اصطلاحی داده‌ام
\\
در حق او مدح و در حق تو ذم
&&
در حق او شهد و در حق تو سم
\\
ما بری از پاک و ناپاکی همه
&&
از گرانجانی و چالاکی همه
\\
من نکردم امر تا سودی کنم
&&
بلک تا بر بندگان جودی کنم
\\
هندوان را اصطلاح هند مدح
&&
سندیان را اصطلاح سند مدح
\\
من نگردم پاک از تسبیحشان
&&
پاک هم ایشان شوند و درفشان
\\
ما زبان را ننگریم و قال را
&&
ما روان را بنگریم و حال را
\\
ناظر قلبیم اگر خاشع بود
&&
گرچه گفت لفظ ناخاضع رود
\\
زانک دل جوهر بود گفتن عرض
&&
پس طفیل آمد عرض جوهر غرض
\\
چند ازین الفاظ و اضمار و مجاز
&&
سوز خواهم سوز با آن سوز ساز
\\
آتشی از عشق در جان بر فروز
&&
سر بسر فکر و عبارت را بسوز
\\
موسیا آداب‌دانان دیگرند
&&
سوخته جان و روانان دیگرند
\\
عاشقان را هر نفس سوزیدنیست
&&
بر ده ویران خراج و عشر نیست
\\
گر خطا گوید ورا خاطی مگو
&&
گر بود پر خون شهید او را مشو
\\
خون شهیدان را ز آب اولیترست
&&
این خطا را صد صواب اولیترست
\\
در درون کعبه رسم قبله نیست
&&
چه غم از غواص را پاچیله نیست
\\
تو ز سرمستان قلاوزی مجو
&&
جامه‌چاکان را چه فرمایی رفو
\\
ملت عشق از همه دینها جداست
&&
عاشقان را ملت و مذهب خداست
\\
لعل را گر مهر نبود باک نیست
&&
عشق در دریای غم غمناک نیست
\\
\end{longtable}
\end{center}
