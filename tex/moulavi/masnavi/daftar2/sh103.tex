\begin{center}
\section*{بخش ۱۰۳ - عذر گفتن فقیر به شیخ}
\label{sec:sh103}
\addcontentsline{toc}{section}{\nameref{sec:sh103}}
\begin{longtable}{l p{0.5cm} r}
پس فقیر آن شیخ را احوال گفت
&&
عذر را با آن غرامت کرد جفت
\\
مر سؤال شیخ را داد او جواب
&&
چون جوابات خضر خوب و صواب
\\
آن جوابات سؤالات کلیم
&&
کش خضر بنمود از رب علیم
\\
گشت مشکلهاش حل وافزون ز یاد
&&
از پی هر مشکلش مفتاح داد
\\
از خضر درویش هم میراث داشت
&&
در جواب شیخ همت بر گماشت
\\
گفت راه اوسط ارچه حکمتست
&&
لیک اوسط نیز هم با نسبتست
\\
آب جو نسبت باشتر هست کم
&&
لیک باشد موش را آن همچو یم
\\
هر که را باشد وظیفه چار نان
&&
دو خورد یا سه خورد هست اوسط آن
\\
ور خورد هر چار دور از اوسط است
&&
او اسیر حرص مانند بط است
\\
هر که او را اشتها ده نان بود
&&
شش خورد می‌دان که اوسط آن بود
\\
چون مرا پنجاه نان هست اشتها
&&
مر ترا شش گرده هم‌دستیم نی
\\
تو بده رکعت نماز آیی ملول
&&
من به پانصد در نیایم در نحول
\\
آن یکی تا کعبه حافی می‌رود
&&
وین یکی تا مسجد از خود می‌شود
\\
آن یکی در پاک‌بازی جان بداد
&&
وین یکی جان کند تا یک نان بداد
\\
این وسط در با نهایت می‌رود
&&
که مر آن را اول و آخر بود
\\
اول و آخر بباید تا در آن
&&
در تصور گنجد اوسط یا میان
\\
بی‌نهایت چون ندارد دو طرف
&&
کی بود او را میانه منصرف
\\
اول و آخر نشانش کس نداد
&&
گفت لو کان له البحر مداد
\\
هفت دریا گر شود کلی مداد
&&
نیست مر پایان شدن را هیچ امید
\\
باغ و بیشه گر بود یکسر قلم
&&
زین سخن هرگز نگردد هیچ کم
\\
آن همه حبر و قلم فانی شود
&&
وین حدیث بی‌عدد باقی بود
\\
حالت من خواب را ماند گهی
&&
خواب پندارد مر آن را گم‌رهی
\\
چشم من خفته دلم بیدار دان
&&
شکل بی‌کار مرا بر کار دان
\\
گفت پیغامبر که عینای تنام
&&
لا ینام قلبی عن رب الانام
\\
چشم تو بیدار و دل خفته بخواب
&&
چشم من خفته دلم در فتح باب
\\
مر دلم را پنج حس دیگرست
&&
حس دل را هر دو عالم منظرست
\\
تو ز ضعف خود مکن در من نگاه
&&
بر تو شب بر من همان شب چاشتگاه
\\
بر تو زندان بر من آن زندان چو باغ
&&
عین مشغولی مرا گشته فراغ
\\
پای تو در گل مرا گل گشته گل
&&
مر ترا ماتم مرا سور و دهل
\\
در زمینم با تو ساکن در محل
&&
می‌دوم بر چرخ هفتم چون زحل
\\
همنشینت من نیم سایهٔ منست
&&
برتر از اندیشه‌ها پایهٔ منست
\\
زانک من ز اندیشه‌ها بگذشته‌ام
&&
خارج اندیشه پویان گشته‌ام
\\
حاکم اندیشه‌ام محکوم نی
&&
زانک بنا حاکم آمد بر بنا
\\
جمله خلقان سخرهٔ اندیشه‌اند
&&
زان سبب خسته دل و غم‌پیشه‌اند
\\
قاصدا خود را باندیشه دهم
&&
چون بخواهم از میانشان بر جهم
\\
من چو مرغ اوجم اندیشه مگس
&&
کی بود بر من مگس را دست‌رس
\\
قاصدا زیر آیم از اوج بلند
&&
تا شکسته‌پایگان بر من تنند
\\
چون ملالم گیرد از سفلی صفات
&&
بر پرم همچون طیور الصافات
\\
پر من رستست هم از ذات خویش
&&
بر نچفسانم دو پر من با سریش
\\
جعفر طیار را پر جاریه‌ست
&&
جعفر طرار را پر عاریه‌ست
\\
نزد آنک لم یذق دعویست این
&&
نزد سکان افق معنیست این
\\
لاف و دعوی باشد این پیش غراب
&&
دیگ تی و پر یکی پیش ذباب
\\
چونک در تو می‌شود لقمه گهر
&&
تن مزن چندانک بتوانی بخور
\\
شیخ روزی بهر دفع سؤ ظن
&&
در لگن قی کرد پر در شد لگن
\\
گوهر معقول را محسوس کرد
&&
پیر بینا بهر کم‌عقلی مرد
\\
چونک در معده شود پاکت پلید
&&
قفل نه بر خلق و پنهان کن کلید
\\
هر که در وی لقمه شد نور جلال
&&
هر چه خواهد تا خورد او را حلال
\\
\end{longtable}
\end{center}
