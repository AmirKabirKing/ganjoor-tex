\begin{center}
\section*{بخش ۱۰۴ - بیان دعویی که عین آن دعوی گواه  صدق خویش است}
\label{sec:sh104}
\addcontentsline{toc}{section}{\nameref{sec:sh104}}
\begin{longtable}{l p{0.5cm} r}
گر تو هستی آشنای جان من
&&
نیست دعوی گفت معنی‌لان من
\\
گر بگویم نیم‌شب پیش توم
&&
هین مترس از شب که من خویش توم
\\
این دو دعوی پیش تو معنی بود
&&
چون شناسی بانگ خویشاوند خود
\\
پیشی و خویشی دو دعوی بود لیک
&&
هر دو معنی بود پیش فهم نیک
\\
قرب آوازش گواهی می‌دهد
&&
کین دم از نزدیک یاری می‌جهد
\\
لذت آواز خویشاوند نیز
&&
شد گوا بر صدق آن خویش عزیز
\\
باز بی الهام احمق کو ز جهل
&&
می‌نداند بانگ بیگانه ز اهل
\\
پیش او دعوی بود گفتار او
&&
جهل او شد مایهٔ انکار او
\\
پیش زیرک کاندرونش نورهاست
&&
عین این آواز معنی بود راست
\\
یا به تازی گفت یک تازی‌زبان
&&
که همی‌دانم زبان تازیان
\\
عین تازی گفتنش معنی بود
&&
گرچه تازی گفتنش دعوی بود
\\
یا نویسد کاتبی بر کاغدی
&&
کاتب و خط‌خوانم و من امجدی
\\
این نوشته گرچه خود دعوی بود
&&
هم نوشته شاهد معنی بود
\\
یا بگوید صوفیی دیدی تو دوش
&&
در میان خواب سجاده‌بدوش
\\
من بدم آن وآنچ گفتم خواب در
&&
با تو اندر خواب در شرح نظر
\\
گوش کن چون حلقه اندر گوش کن
&&
آن سخن را پیشوای هوش کن
\\
چون ترا یاد آید آن خواب این سخن
&&
معجز نو باشد و زر کهن
\\
گرچه دعوی می‌نماید این ولی
&&
جان صاحب‌واقعه گوید بلی
\\
پس چو حکمت ضالهٔمؤمنبود
&&
آن ز هر که بشنود موقن بود
\\
چونک خود را پیش او یابد فقط
&&
چون بود شک چون کند او را غلط
\\
تشنه‌ای را چون بگویی تو شتاب
&&
در قدح آبست بستان زود آب
\\
هیچ گوید تشنه کین دعویست رو
&&
از برم ای مدعی مهجور شو
\\
یا گواه و حجتی بنما که این
&&
جنس آبست و از آن ماء معین
\\
یا به طفل شیر مادر بانگ زد
&&
که بیا من مادرم هان ای ولد
\\
طفل گوید مادرا حجت بیار
&&
تا که با شیرت بگیرم من قرار
\\
در دل هر امتی کز حق مزه‌ست
&&
روی و آواز پیمبر معجزه‌ست
\\
چون پیمبر از برون بانگی زند
&&
جان امت در درون سجده کند
\\
زانک جنس بانگ او اندر جهان
&&
از کسی نشنیده باشد گوش جان
\\
آن غریب از ذوق آواز غریب
&&
از زبان حق شنود انی قریب
\\
\end{longtable}
\end{center}
