\begin{center}
\section*{بخش ۲۹ - رجوع به حکایت ذاالنون رحمة الله علیه}
\label{sec:sh029}
\addcontentsline{toc}{section}{\nameref{sec:sh029}}
\begin{longtable}{l p{0.5cm} r}
چون رسیدند آن نفر نزدیک او
&&
بانگ بر زد هی کیانید اتقو
\\
با ادب گفتند ما از دوستان
&&
بهر پرسش آمدیم اینجا بجان
\\
چونی ای دریای عقل ذو فنون
&&
این چه بهتانست بر عقلت جنون
\\
دود گلخن کی رسد در آفتاب
&&
چون شود عنقا شکسته از غراب
\\
وا مگیر از ما بیان کن این سخن
&&
ما محبانیم با ما این مکن
\\
مر محبان را نشاید دور کرد
&&
یا بروپوش و دغل مغرور کرد
\\
راز را اندر میان آور شها
&&
رو مکن در ابر پنهانی مها
\\
ما محب و صادق و دل خسته‌ایم
&&
در دو عالم دل به تو در بسته‌ایم
\\
فحش آغازید و دشنام از گزاف
&&
گفت او دیوانگانه زی و قاف
\\
بر جهید و سنگ پران کرد و چوب
&&
جملگی بگریختند از بیم کوب
\\
قهقهه خندید و جنبانید سر
&&
گفت باد ریش این یاران نگر
\\
دوستان بین کو نشان دوستان
&&
دوستان را رنج باشد همچو جان
\\
کی کران گیرد ز رنج دوست دوست
&&
رنج مغز و دوستی آن را چو پوست
\\
نی نشان دوستی شد سرخوشی
&&
در بلا و آفت و محنت‌کشی
\\
دوست همچون زر بلا چون آتشست
&&
زر خالص در دل آتش خوشست
\\
\end{longtable}
\end{center}
