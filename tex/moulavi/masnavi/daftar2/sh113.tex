\begin{center}
\section*{بخش ۱۱۳ - برخاستن مخالفت و عداوت از میان انصار  به برکات رسول علیه السلام}
\label{sec:sh113}
\addcontentsline{toc}{section}{\nameref{sec:sh113}}
\begin{longtable}{l p{0.5cm} r}
دو قبیله کاوس و خزرج نام داشت
&&
یک ز دیگر جان خون‌آشام داشت
\\
کینه‌های کهنه‌شان از مصطفی
&&
محو شد در نور اسلام و صفا
\\
اولا اخوان شدند آن دشمنان
&&
همچو اعداد عنب در بوستان
\\
وز دم المؤمنون اخوه بپند
&&
در شکستند و تن واحد شدند
\\
صورت انگورها اخوان بود
&&
چون فشردی شیرهٔ واحد شود
\\
غوره و انگور ضدانند لیک
&&
چونک غوره پخته شد شد یار نیک
\\
غوره‌ای کو سنگ‌بست و خام ماند
&&
در ازل حق کافر اصلیش خواند
\\
نه اخی نه نفس واحد باشد او
&&
در شقاوت نحس ملحد باشد او
\\
گر بگویم آنچ او دارد نهان
&&
فتنهٔ افهام خیزد در جهان
\\
سر گبر کور نامذکور به
&&
دود دوزخ از ارم مهجور به
\\
غوره‌های نیک کایشان قابلند
&&
از دم اهل دل آخر یک دلند
\\
سوی انگوری همی‌رانند تیز
&&
تا دوی بر خیزد و کین و ستیز
\\
پس در انگوری همی‌درند پوست
&&
تا یکی گردند و وحدت وصف اوست
\\
دوست دشمن گردد ایرا هم دواست
&&
هیچ یک با خویش جنگی در نبست
\\
آفرین بر عشق کل اوستاد
&&
صد هزاران ذره را داد اتحاد
\\
همچو خاک مفترق در ره‌گذر
&&
یک سبوشان کرد دست کوزه‌گر
\\
که اتحاد جسمهای آب و طین
&&
هست ناقص جان نمی‌ماند بدین
\\
گر نظایر گویم اینجا در مثال
&&
فهم را ترسم که آرد اختلال
\\
هم سلیمان هست اکنون لیک ما
&&
از نشاط دوربینی در عمی
\\
دوربینی کور دارد مرد را
&&
همچو خفته در سرا کور از سرا
\\
مولعیم اندر سخنهای دقیق
&&
در گره ها باز کردن ما عشیق
\\
تا گره بندیم و بگشاییم ما
&&
در شکال و در جواب آیین‌فزا
\\
همچو مرغی کو گشاید بند دام
&&
گاه بندد تا شود در فن تمام
\\
او بود محروم از صحرا و مرج
&&
عمر او اندر گره کاریست خرج
\\
خود زبون او نگردد هیچ دام
&&
لیک پرش در شکست افتد مدام
\\
با گره کم کوش تا بال و پرت
&&
نسکلد یک یک ازین کر و فرت
\\
صد هزاران مرغ پرهاشان شکست
&&
و آن کمین‌گاه عمارض را نبست
\\
حال ایشان از نبی خوان ای حریص
&&
نقبوا فیها ببین هل من محیص
\\
از نزاع ترک و رومی و عرب
&&
حل نشد اشکال انگور و عنب
\\
تا سلیمان لسین معنوی
&&
در نیاید بر نخیزد این دوی
\\
جمله مرغان منازع بازوار
&&
بشنوید این طبل باز شهریار
\\
ز اختلاف خویش سوی اتحاد
&&
هین ز هر جانب روان گردید شاد
\\
حیث ما کنتم فولوا وجهکم
&&
نحوه هذا الذی لم ینهکم
\\
کور مرغانیم و بس ناساختیم
&&
کان سلیمان را دمی نشناختیم
\\
همچو جغدان دشمن بازان شدیم
&&
لاجرم وا ماندهٔ ویران شدیم
\\
می‌کنیم از غایت جهل و عما
&&
قصد آزار عزیزان خدا
\\
جمع مرغان کز سلیمان روشنند
&&
پر و بال بی گنه کی برکنند
\\
بلک سوی عاجزان چینه کشند
&&
بی خلاف و کینه آن مرغان خوشند
\\
هدهد ایشان پی تقدیس را
&&
می‌گشاید راه صد بلقیس را
\\
زاغ ایشان گر بصورت زاغ بود
&&
باز همت آمد و مازاغ بود
\\
لکلک ایشان که لک‌لک می‌زند
&&
آتش توحید در شک می‌زند
\\
و آن کبوترشان ز بازان نشکهد
&&
باز سر پیش کبوترشان نهد
\\
بلبل ایشان که حالت آرد او
&&
در درون خویش گلشن دارد او
\\
طوطی ایشان ز قند آزاد بود
&&
کز درون قند ابد رویش نمود
\\
پای طاووسان ایشان در نظر
&&
بهتر از طاووس‌پران دگر
\\
منطق الطیر آن خاقانی صداست
&&
منطق الطیر سلیمانی کجاست
\\
تو چه دانی بانگ مرغان را همی
&&
چون ندیدستی سلیمان را دمی
\\
پر آن مرغی که بانگش مطربست
&&
از برون مشرقست و مغربست
\\
هر یک آهنگش ز کرسی تا ثریست
&&
وز ثری تا عرش در کر و فریست
\\
مرغ کو بی این سلیمان می‌رود
&&
عاشق ظلمت چو خفاشی بود
\\
با سلیمان خو کن ای خفاش رد
&&
تا که در ظلمت نمانی تا ابد
\\
یک گزی ره که بدان سو می‌روی
&&
همچو گز قطب مساحت می‌شوی
\\
وانک لنگ و لوک آن سو می‌جهی
&&
از همه لنگی و لوکی می‌رهی
\\
\end{longtable}
\end{center}
