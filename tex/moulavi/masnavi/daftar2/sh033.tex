\begin{center}
\section*{بخش ۳۳ - عکس تعظیم پیغام سلیمان در دل بلقیس از صورت حقیر هدهد}
\label{sec:sh033}
\addcontentsline{toc}{section}{\nameref{sec:sh033}}
\begin{longtable}{l p{0.5cm} r}
رحمت صد تو بر آن بلقیس باد
&&
که خدایش عقل صد مرده بداد
\\
هدهدی نامه بیاورد و نشان
&&
از سلیمان چند حرفی با بیان
\\
خواند او آن نکته‌های با شمول
&&
با حقارت ننگرید اندر رسول
\\
جسم هدهد دید و جان عنقاش دید
&&
حس چو کفی دید و دل دریاش دید
\\
عقل با حس زین طلسمات دو رنگ
&&
چون محمد با ابوجهلان به جنگ
\\
کافران دیدند احمد را بشر
&&
چون ندیدند از وی انشق القمر
\\
خاک زن در دیدهٔ حس‌بین خویش
&&
دیدهٔ حس دشمن عقلست و کیش
\\
دیدهٔ حس را خدا اعماش خواند
&&
بت‌پرستش گفت و ضد ماش خواند
\\
زانک او کف دید و دریا را ندید
&&
زانک حالی دید و فردا را ندید
\\
خواجهٔ فردا و حالی پیش او
&&
او نمی‌بیند ز گنجی جز تسو
\\
ذره‌ای زان آفتاب آرد پیام
&&
آفتاب آن ذره را گردد غلام
\\
قطره‌ای کز بحر وحدت شد سفیر
&&
هفت بحر آن قطره را باشد اسیر
\\
گر کف خاکی شود چالاک او
&&
پیش خاکش سر نهد افلاک او
\\
خاک آدم چونک شد چالاک حق
&&
پیش خاکش سر نهند املاک حق
\\
السماء انشقت آخر از چه بود
&&
از یکی چشمی که خاکیی گشود
\\
خاک از دردی نشیند زیر آب
&&
خاک بین کز عرش بگذشت از شتاب
\\
آن لطافت پس بدان کز آب نیست
&&
جز عطای مبدع وهاب نیست
\\
گر کند سفلی هوا و نار را
&&
ور ز گل او بگذراند خار را
\\
حاکمست و یفعل الله ما یشا
&&
کو ز عین درد انگیزد دوا
\\
گر هوا و نار را سفلی کند
&&
تیرگی و دردی و ثفلی کند
\\
ور زمین و آب را علوی کند
&&
راه گردون را به پا مطوی کند
\\
پس یقین شد که تعز من تشا
&&
خاکیی را گفت پرها بر گشا
\\
آتشی را گفت رو ابلیس شو
&&
زیر هفتم خاک با تلبیس شو
\\
آدم خاکی برو تو بر سها
&&
ای بلیس آتشی رو تا ثری
\\
چار طبع و علت اولی نیم
&&
در تصرف دایما من باقیم
\\
کار من بی علتست و مستقیم
&&
هست تقدیرم نه علت ای سقیم
\\
عادت خود را بگردانم بوقت
&&
این غبار از پیش بنشانم بوقت
\\
بحر را گویم که هین پر نار شو
&&
گویم آتش را که رو گلزار شو
\\
کوه را گویم سبک شو همچو پشم
&&
چرخ را گویم فرو در پیش چشم
\\
گویم ای خورشید مقرون شو به ماه
&&
هر دو را سازم چو دو ابر سیاه
\\
چشمهٔ خورشید را سازیم خشک
&&
چشمهٔ خون را بفن سازیم مشک
\\
آفتاب و مه چو دو گاو سیاه
&&
یوغ بر گردن ببنددشان اله
\\
\end{longtable}
\end{center}
