\begin{center}
\section*{بخش ۵ - اندرز کردن صوفی خادم را در تیمار داشت بهیمه و لا حول خادم}
\label{sec:sh005}
\addcontentsline{toc}{section}{\nameref{sec:sh005}}
\begin{longtable}{l p{0.5cm} r}
صوفیی می‌گشت در دور افق
&&
تا شبی در خانقاهی شد قنق
\\
یک بهیمه داشت در آخر ببست
&&
او به صدر صفه با یاران نشست
\\
پس مراقب گشت با یاران خویش
&&
دفتری باشد حضور یار پیش
\\
دفتر صوفی سواد حرف نیست
&&
جز دل اسپید همچون برف نیست
\\
زاد دانشمند آثار قلم
&&
زاد صوفی چیست آثار قدم
\\
همچو صیادی سوی اشکار شد
&&
گام آهو دید و بر آثار شد
\\
چندگاهش گام آهو در خورست
&&
بعد از آن خود ناف آهو رهبرست
\\
چونک شکر گام کرد و ره برید
&&
لاجرم زان گام در کامی رسید
\\
رفتن یک منزلی بر بوی ناف
&&
بهتر از صد منزل گام و طواف
\\
آن دلی کو مطلع مهتابهاست
&&
بهر عارف فتحت ابوابهاست
\\
با تو دیوارست و با ایشان درست
&&
با تو سنگ و با عزیزان گوهرست
\\
آنچ تو در آینه بینی عیان
&&
پیر اندر خشت بیند بیش از آن
\\
پیر ایشانند کین عالم نبود
&&
جان ایشان بود در دریای جود
\\
پیش ازین تن عمرها بگذاشتند
&&
پیشتر از کشت بر برداشتند
\\
پیشتر از نقش جان پذرفته‌اند
&&
پیشتر از بحر درها سفته‌اند
\\
\end{longtable}
\end{center}
