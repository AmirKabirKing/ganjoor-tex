\begin{center}
\section*{بخش ۵۰ - تنها کردن باغبان صوفی و فقیه و علوی را از همدیگر}
\label{sec:sh050}
\addcontentsline{toc}{section}{\nameref{sec:sh050}}
\begin{longtable}{l p{0.5cm} r}
باغبانی چون نظر در باغ کرد
&&
دید چون دزدان بباغ خود سه مرد
\\
یک فقیه و یک شریف و صوفیی
&&
هر یکی شوخی بدی لا یوفیی
\\
گفت با اینها مرا صد حجتست
&&
لیک جمع‌اند و جماعت قوتست
\\
بر نیایم یک تنه با سه نفر
&&
پس ببرمشان نخست از همدگر
\\
هر یکی را من به سویی افکنم
&&
چونک تنها شد سبیلش بر کنم
\\
حیله کرد و کرد صوفی را به راه
&&
تا کند یارانش را با او تباه
\\
گفت صوفی را برو سوی وثاق
&&
یک گلیم آور برای این رفاق
\\
رفت صوفی گفت خلوت با دو یار
&&
تو فقیهی وین شریف نامدار
\\
ما به فتوی تو نانی می‌خوریم
&&
ما به پر دانش تو می‌پریم
\\
وین دگر شه‌زاده و سلطان ماست
&&
سیدست از خاندان مصطفاست
\\
کیست آن صوفی شکم‌خوار خسیس
&&
تا بود با چون شما شاهان جلیس
\\
چون بباید مر ورا پنبه کنید
&&
هفته‌ای بر باغ و راغ من زنید
\\
باغ چه بود جان من آن شماست
&&
ای شما بوده مرا چون چشم راست
\\
وسوسه کرد و مریشان را فریفت
&&
آه کز یاران نمی‌باید شکیفت
\\
چون بره کردند صوفی را و رفت
&&
خصم شد اندر پیش با چوب زفت
\\
گفت ای سگ صوفیی باشد که تیز
&&
اندر آیی باغ ما تو از ستیز
\\
این جنیدت ره نمود و بایزید
&&
از کدامین شیخ و پیرت این رسید
\\
کوفت صوفی را چو تنها یافتش
&&
نیم کشتش کرد و سر بشکافتش
\\
گفت صوفی آن من بگذشت لیک
&&
ای رفیقان پاس خود دارید نیک
\\
مر مرا اغیار دانستید هان
&&
نیستم اغیارتر زین قلتبان
\\
اینچ من خوردم شما را خوردنیست
&&
وین چنین شربت جزای هر دنیست
\\
این جهان کوهست و گفت و گوی تو
&&
از صدا هم باز آید سوی تو
\\
چون ز صوفی گشت فارغ باغبان
&&
یک بهانه کرد زان پس جنس آن
\\
کای شریف من برو سوی وثاق
&&
که ز بهر چاشت پختم من رقاق
\\
بر در خانه بگو قیماز را
&&
تا بیارد آن رقاق و قاز را
\\
چون بره کردش بگفت ای تیزبین
&&
تو فقیهی ظاهرست این و یقین
\\
او شریفی می‌کند دعوی سرد
&&
مادر او را که داند تا کی کرد
\\
بر زن و بر فعل زن دل می‌نهید
&&
عقل ناقص وانگهانی اعتماد
\\
خویشتن را بر علی و بر نبی
&&
بسته است اندر زمانه بس غبی
\\
هر که باشد از زنا و زانیان
&&
این برد ظن در حق ربانیان
\\
هر که بر گردد سرش از چرخها
&&
همچو خود گردنده بیند خانه را
\\
آنچ گفت آن باغبان بوالفضول
&&
حال او بد دور از اولاد رسول
\\
گر نبودی او نتیجهٔ مرتدان
&&
کی چنین گفتی برای خاندان
\\
خواند افسونها شنید آن را فقیه
&&
در پیش رفت آن ستمکار سفیه
\\
گفت ای خر اندرین باغت کی خواند
&&
دزدی از پیغامبرت میراث ماند
\\
شیر را بچه همی‌ماند بدو
&&
تو به پیغامبر بچه مانی بگو
\\
با شریف آن کرد مرد ملتجی
&&
که کند با آل یاسین خارجی
\\
تا چه کین دارند دایم دیو و غول
&&
چون یزید و شمر با آل رسول
\\
شد شریف از زخم آن ظالم خراب
&&
با فقیه او گفت ما جستیم از آب
\\
پای دار اکنون که ماندی فرد و کم
&&
چون دهل شو زخم می‌خور در شکم
\\
گر شریف و لایق و همدم نیم
&&
از چنین ظالم ترا من کم نیم
\\
مر مرا دادی بدین صاحب غرض
&&
احمقی کردی ترا بئس العوض
\\
شد ازو فارغ بیامد کای فقیه
&&
چه فقیهی ای تو ننگ هر سفیه
\\
فتوی‌ات اینست ای ببریده‌دست
&&
کاندر آیی و نگویی امر هست
\\
این چنین رخصت بخواندی در وسیط
&&
یا بدست این مساله اندر محیط
\\
گفت حقستت بزن دستت رسید
&&
این سزای آنک از یاران برید
\\
\end{longtable}
\end{center}
