\begin{center}
\section*{بخش ۲۷ - آمدن دوستان به بیمارستان جهت پرسش ذاالنون مصری رحمة الله علیه}
\label{sec:sh027}
\addcontentsline{toc}{section}{\nameref{sec:sh027}}
\begin{longtable}{l p{0.5cm} r}
این چنین ذاالنون مصری را فتاد
&&
کاندرو شور و جنونی نو بزاد
\\
شور چندان شد که تا فوق فلک
&&
می‌رسید از وی جگرها را نمک
\\
هین منه تو شور خود ای شوره‌خاک
&&
پهلوی شور خداوندان پاک
\\
خلق را تاب جنون او نبود
&&
آتش او ریشهاشان می‌ربود
\\
چونک در ریش عوام آتش فتاد
&&
بند کردندش به زندانی نهاد
\\
نیست امکان واکشیدن این لگام
&&
گرچه زین ره تنگ می‌آیند عام
\\
دیده این شاهان ز عامه خوف جان
&&
کین گره کورند و شاهان بی‌نشان
\\
چونک حکم اندر کف رندان بود
&&
لاجرم ذاالنون در زندان بود
\\
یکسواره می‌رود شاه عظیم
&&
در کف طفلان چنین در یتیم
\\
در چه دریا نهان در قطره‌ای
&&
آفتابی مخفی اندر ذره‌ای
\\
آفتابی خویش را ذره نمود
&&
واندک اندک روی خود را بر گشود
\\
جملهٔ ذرات در وی محو شد
&&
عالم از وی مست گشت و صحو شد
\\
چون قلم در دست غداری بود
&&
بی گمان منصور بر داری بود
\\
چون سفیهان‌راست این کار و کیا
&&
لازم آمد یقتلون الانبیا
\\
انبیا را گفته قومی راه گم
&&
از سفه انا تطیرنا بکم
\\
جهل ترسا بین امان انگیخته
&&
زان خداوندی که گشت آویخته
\\
چون بقول اوست مصلوب جهود
&&
پس مرورا امن کی تاند نمود
\\
چون دل آن شاه زیشان خون بود
&&
عصمت و انت فیهم چون بود
\\
زر خالص را و زرگر را خطر
&&
باشد از قلاب خاین بیشتر
\\
یوسفان از رشک زشتان مخفی‌اند
&&
کز عدو خوبان در آتش می‌زیند
\\
یوسفان از مکر اخوان در چهند
&&
کز حسد یوسف به گرگان می‌دهند
\\
از حسد بر یوسف مصری چه رفت
&&
این حسد اندر کمین گرگیست زفت
\\
لاجرم زین گرگ یعقوب حلیم
&&
داشت بر یوسف همیشه خوف و بیم
\\
گرگ ظاهر گرد یوسف خود نگشت
&&
این حسد در فعل از گرگان گذشت
\\
رحم کرد این گرگ وز عذر لبق
&&
آمده که انا ذهبنا نستبق
\\
صد هزاران گرگ را این مکر نیست
&&
عاقبت رسوا شود این گرگ بیست
\\
زانک حشر حاسدان روز گزند
&&
بی گمان بر صورت گرگان کنند
\\
حشر پر حرص خس مردارخوار
&&
صورت خوکی بود روز شمار
\\
زانیان را گند اندام نهان
&&
خمرخواران را بود گند دهان
\\
گند مخفی کان به دلها می‌رسید
&&
گشت اندر حشر محسوس و پدید
\\
بیشه‌ای آمد وجود آدمی
&&
بر حذر شو زین وجود ار زان دمی
\\
در وجود ما هزاران گرگ و خوک
&&
صالح و ناصالح و خوب و خشوک
\\
حکم آن خوراست کان غالبترست
&&
چونک زر بیش از مس آمد آن زرست
\\
سیرتی کان بر وجودت غالبست
&&
هم بر آن تصویر حشرت واجبست
\\
ساعتی گرگی در آید در بشر
&&
ساعتی یوسف‌رخی همچون قمر
\\
می‌رود از سینه‌ها در سینه‌ها
&&
از ره پنهان صلاح و کینه‌ها
\\
بلک خود از آدمی در گاو و خر
&&
می‌رود دانایی و علم و هنر
\\
اسپ سکسک می‌شود رهوار و رام
&&
خرس بازی می‌کند بز هم سلام
\\
رفت اندر سگ ز آدمیان هوس
&&
تا شبان شد یا شکاری یا حرس
\\
در سگ اصحاب خویی زان وفود
&&
رفت تا جویای الله گشته بود
\\
هر زمان در سینه نوعی سر کند
&&
گاه دیو و گه ملک گه دام و دد
\\
زان عجب بیشه که هر شیر آگهست
&&
تا به دام سینه‌ها پنهان رهست
\\
دزدیی کن از درون مرجان جان
&&
ای کم از سگ از درون عارفان
\\
چونک دزدی باری آن در لطیف
&&
چونک حامل می‌شوی باری شریف
\\
\end{longtable}
\end{center}
