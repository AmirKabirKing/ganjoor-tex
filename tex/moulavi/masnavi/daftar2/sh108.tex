\begin{center}
\section*{بخش ۱۰۸ - سخن گفتن به زبان حال و فهم کردن آن}
\label{sec:sh108}
\addcontentsline{toc}{section}{\nameref{sec:sh108}}
\begin{longtable}{l p{0.5cm} r}
ماجرای شمع با پروانه تو
&&
بشنو و معنی گزین ز افسانه تو
\\
گر چه گفتی نیست سر گفت هست
&&
هین به بالا پر مپر چون جغد پست
\\
گفت در شطرنج کین خانهٔ رخست
&&
گفت خانه از کجاش آمد بدست
\\
خانه را بخرید یا میراث یافت
&&
فرخ آنکس کو سوی معنی شتافت
\\
گفت نحوی زید عمروا قد ضرب
&&
گفت چونش کرد بی جرمی ادب
\\
عمرو را جرمش چه بد کان زید خام
&&
بی گنه او را بزد همچون غلام
\\
گفت این پیمانهٔ معنی بود
&&
گندمی بستان که پیمانه‌ست رد
\\
زید و عمرو از بهر اعرابست و ساز
&&
گر دروغست آن تو با اعراب ساز
\\
گفت نی من آن ندانم عمرو را
&&
زید چون زد بی‌گناه و بی‌خطا
\\
گفت از ناچار و لاغی بر گشود
&&
عمرو یک واو فزون دزدیده بود
\\
زید واقف گشت دزدش را بزد
&&
چونک از حد برد او را حد سزد
\\
\end{longtable}
\end{center}
