\begin{center}
\section*{بخش ۱۴ - خاریدن روستایی در تاریکی شیر را بظن آنک گاو اوست}
\label{sec:sh014}
\addcontentsline{toc}{section}{\nameref{sec:sh014}}
\begin{longtable}{l p{0.5cm} r}
روستایی گاو در آخر ببست
&&
شیر گاوش خورد و بر جایش نشست
\\
روستایی شد در آخر سوی گاو
&&
گاو را می‌جست شب آن کنج‌کاو
\\
دست می‌مالید بر اعضای شیر
&&
پشت و پهلو گاه بالا گاه زیر
\\
گفت شیر از روشنی افزون شدی
&&
زهره‌اش بدریدی و دل خون شدی
\\
این چنین گستاخ زان می‌خاردم
&&
کو درین شب گاو می‌پنداردم
\\
حق همی‌گوید که ای مغرور کور
&&
نه ز نامم پاره پاره گشت طور
\\
که لو انزلنا کتابا للجبل
&&
لانصدع ثم انقطع ثم ارتحل
\\
از من ار کوه احد واقف بدی
&&
پاره گشتی و دلش پر خون شدی
\\
از پدر وز مادر این بشنیده‌ای
&&
لاجرم غافل درین پیچیده‌ای
\\
گر تو بی‌تقلید ازین واقف شوی
&&
بی نشان از لطف چون هاتف شوی
\\
بشنو این قصه پی تهدید را
&&
تا بدانی آفت تقلید را
\\
\end{longtable}
\end{center}
