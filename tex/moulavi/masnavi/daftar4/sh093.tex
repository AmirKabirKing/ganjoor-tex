\begin{center}
\section*{بخش ۹۳ - بیان آنک تن خاکی آدمی هم‌چون آهن نیکو جوهر قابل آینه شدن است تا درو هم در دنیا بهشت و دوزخ و قیامت و غیر آن معاینه بنماید نه بر طریق خیال}
\label{sec:sh093}
\addcontentsline{toc}{section}{\nameref{sec:sh093}}
\begin{longtable}{l p{0.5cm} r}
پس چو آهن گرچه تیره‌هیکلی
&&
صیقلی کن صیقلی کن صیقلی
\\
تا دلت آیینه گردد پر صور
&&
اندرو هر سو ملیحی سیمبر
\\
آهن ار چه تیره و بی‌نور بود
&&
صیقلی آن تیرگی از وی زدود
\\
صیقلی دید آهن و خوش کرد رو
&&
تا که صورتها توان دید اندرو
\\
گر تن خاکی غلیظ و تیره است
&&
صیقلش کن زانک صیقل گیره است
\\
تا درو اشکال غیبی رو دهد
&&
عکس حوری و ملک در وی جهد
\\
صیقل عقلت بدان دادست حق
&&
که بدو روشن شود دل را ورق
\\
صیقلی را بسته‌ای ای بی‌نماز
&&
وآن هوا را کرده‌ای دو دست باز
\\
گر هوا را بند بنهاده شود
&&
صیقلی را دست بگشاده شود
\\
آهنی که آیینه غیبی بدی
&&
جمله صورتها درو مرسل شدی
\\
تیره کردی زنگ دادی در نهاد
&&
این بود یسعون فی الارض الفساد
\\
تاکنون کردی چنین اکنون مکن
&&
تیره کردی آب را افزون مکن
\\
بر مشوران تا شود این آب صاف
&&
واندرو بین ماه و اختر در طواف
\\
زانک مردم هست هم‌چون آب جو
&&
چون شود تیره نبینی قعر او
\\
قعر جو پر گوهرست و پر ز در
&&
هین مکن تیره که هست او صاف حر
\\
جان مردم هست مانند هوا
&&
چون بگرد آمیخت شد پردهٔ سما
\\
مانع آید او ز دید آفتاب
&&
چونک گردش رفت شد صافی و ناب
\\
با کمال تیرگی حق واقعات
&&
می‌نمودت تا روی راه نجات
\\
\end{longtable}
\end{center}
