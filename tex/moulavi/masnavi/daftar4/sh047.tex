\begin{center}
\section*{بخش ۴۷ - مانستن بدرایی این وزیر دون در افساد مروت شاه به وزیر فرعون یعنی هامان در افساد قابلیت فرعون}
\label{sec:sh047}
\addcontentsline{toc}{section}{\nameref{sec:sh047}}
\begin{longtable}{l p{0.5cm} r}
چند آن فرعون می‌شد نرم و رام
&&
چون شنیدی او ز موسی آن کلام
\\
آن کلامی که بدادی سنگ شیر
&&
از خوشی آن کلام بی‌نظیر
\\
چون بهامان که وزیرش بود او
&&
مشورت کردی که کینش بود خو
\\
پس بگفتی تا کنون بودی خدیو
&&
بنده گردی ژنده‌پوشی را بریو
\\
هم‌چو سنگ منجنیقی آمدی
&&
آن سخن بر شیشه خانهٔ او زدی
\\
هر چه صد روز آن کلیم خوش‌خطاب
&&
ساختی در یک‌دم او کردی خراب
\\
عقل تو دستور و مغلوب هواست
&&
در وجودت ره‌زن راه خداست
\\
ناصحی ربانیی پندت دهد
&&
آن سخن را او به فن طرحی نهد
\\
کین نه بر جایست هین از جا مشو
&&
نیست چندان با خود آ شیدا مشو
\\
وای آن شه که وزیرش این بود
&&
جای هر دو دوزخ پر کین بود
\\
شاد آن شاهی که او را دست‌گیر
&&
باشد اندر کار چون آصف وزیر
\\
شاه عادل چون قرین او شود
&&
نام آن نور علی نور این بود
\\
چون سلیمان شاه و چون آصف وزیر
&&
نور بر نورست و عنبر بر عبیر
\\
شاه فرعون و چو هامانش وزیر
&&
هر دو را نبود ز بدبختی گزیر
\\
پس بود ظلمات بعضی فوق بعض
&&
نه خرد یار و نه دولت روز عرض
\\
من ندیدم جز شقاوت در لئام
&&
گر تو دیدستی رسان از من سلام
\\
هم‌چو جان باشد شه و صاحب چو عقل
&&
عقل فاسد روح را آرد بنقل
\\
آن فرشتهٔ عقل چون هاروت شد
&&
سحرآموز دو صد طاغوت شد
\\
عقل جزوی را وزیر خود مگیر
&&
عقل کل را ساز ای سلطان وزیر
\\
مر هوا را تو وزیر خود مساز
&&
که برآید جان پاکت از نماز
\\
کین هوا پر حرص و حالی‌بین بود
&&
عقل را اندیشه یوم دین بود
\\
عقل را دو دیده در پایان کار
&&
بهر آن گل می‌کشد او رنج خار
\\
که نفرساید نریزد در خزان
&&
باد هر خرطوم اخشم دور از آن
\\
\end{longtable}
\end{center}
