\begin{center}
\section*{بخش ۱۵ - گفتن آن جهود علی را کرم الله وجهه کی اگر اعتماد داری بر حافظی حق از سر این کوشک خود را در انداز و جواب گفتن  امیرالممنین او را}
\label{sec:sh015}
\addcontentsline{toc}{section}{\nameref{sec:sh015}}
\begin{longtable}{l p{0.5cm} r}
مرتضی را گفت روزی یک عنود
&&
کو ز تعظیم خدا آگه نبود
\\
بر سر بامی و قصری بس بلند
&&
حفظ حق را واقفی ای هوشمند
\\
گفت آری او حفیظست و غنی
&&
هستی ما را ز طفلی و منی
\\
گفت خود را اندر افکن هین ز بام
&&
اعتمادی کن بحفظ حق تمام
\\
تا یقین گرددمرا ایقان تو
&&
و اعتقاد خوب با برهان تو
\\
پس امیرش گفت خامش کن برو
&&
تا نگردد جانت زین جرات گرو
\\
کی رسد مر بنده را که با خدا
&&
آزمایش پیش آرد ز ابتلا
\\
بنده را کی زهره باشد کز فضول
&&
امتحان حق کند ای گیج گول
\\
آن خدا را می‌رسد کو امتحان
&&
پیش آرد هر دمی با بندگان
\\
تا به ما ما را نماید آشکار
&&
که چه داریم از عقیده در سرار
\\
هیچ آدم گفت حق را که ترا
&&
امتحان کردم درین جرم و خطا
\\
تا ببینم غایت حلمت شها
&&
اه کرا باشد مجال این کرا
\\
عقل تو از بس که آمد خیره‌سر
&&
هست عذرت از گناه تو بتر
\\
آنک او افراشت سقف آسمان
&&
تو چه دانی کردن او را امتحان
\\
ای ندانسته تو شر و خیر را
&&
امتحان خود را کن آنگه غیر را
\\
امتحان خود چو کردی ای فلان
&&
فارغ آیی ز امتحان دیگران
\\
چون بدانستی که شکردانه‌ای
&&
پس بدانی کاهل شکرخانه‌ای
\\
پس بدان بی‌امتحانی که اله
&&
شکری نفرستدت ناجایگاه
\\
این بدان بی‌امتحان از علم شاه
&&
چون سری نفرستدت در پایگاه
\\
هیچ عاقل افکند در ثمین
&&
در میان مستراحی پر چمین
\\
زانک گندم را حکیم آگهی
&&
هیچ نفرستد به انبار کهی
\\
شیخ را که پیشوا و رهبرست
&&
گر مریدی امتحان کرد او خرست
\\
امتحانش گر کنی در راه دین
&&
هم تو گردی ممتحن ای بی‌یقین
\\
جرات و جهلت شود عریان و فاش
&&
او برهنه کی شود زان افتتاش
\\
گر بیاید ذره سنجد کوه را
&&
بر درد زان که ترازوش ای فتی
\\
کز قیاس خود ترازو می‌تند
&&
مرد حق را در ترازو می‌کند
\\
چون نگنجد او به میزان خرد
&&
پس ترازوی خرد را بر درد
\\
امتحان هم‌چون تصرف دان درو
&&
تو تصرف بر چنان شاهی مجو
\\
چه تصرف کرد خواهد نقشها
&&
بر چنان نقاش بهر ابتلا
\\
امتحانی گر بدانست و بدید
&&
نی که هم نقاش آن بر وی کشید
\\
چه قدر باشد خود این صورت که بست
&&
پیش صورتها که در علم ویست
\\
وسوسهٔ این امتحان چون آمدت
&&
بخت بد دان کآمد و گردن زدت
\\
چون چنین وسواس دیدی زود زود
&&
با خدا گرد و در آ اندر سجود
\\
سجده گه را تر کن از اشک روان
&&
کای خدا تو وا رهانم زین گمان
\\
آن زمان کت امتحان مطلوب شد
&&
مسجد دین تو پر خروب شد
\\
\end{longtable}
\end{center}
