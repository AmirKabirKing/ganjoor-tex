\begin{center}
\section*{بخش ۶۴ - زجر مدعی از دعوی و امر کردن او را به متابعت}
\label{sec:sh064}
\addcontentsline{toc}{section}{\nameref{sec:sh064}}
\begin{longtable}{l p{0.5cm} r}
بو مسیلم گفت خود من احمدم
&&
دین احمد را به فن برهم زدم
\\
بو مسیلم را بگو کم کن بطر
&&
غرهٔ اول مشو آخر نگر
\\
این قلاوزی مکن از حرص جمع
&&
پس‌روی کن تا رود در پیش شمع
\\
شمع مقصد را نماید هم‌چو ماه
&&
کین طرف دانه‌ست یا خود دامگاه
\\
گر بخواهی ور نخواهی با چراغ
&&
دیده گردد نقش باز و نقش زاغ
\\
ورنه این زاغان دغل افروختند
&&
بانگ بازان سپید آموختند
\\
بانگ هدهد گر بیاموزد فتی
&&
راز هدهد کو و پیغام سبا
\\
بانگ بر رسته ز بر بسته بدان
&&
تاج شاهان را ز تاج هدهدان
\\
حرف درویشان و نکتهٔ عارفان
&&
بسته‌اند این بی‌حیایان بر زبان
\\
هر هلاک امت پیشین که بود
&&
زانک چندل را گمان بردند عود
\\
بودشان تمییز کان مظهر کند
&&
لیک حرص و آز کور و کر کند
\\
کوری کوران ز رحمت دور نیست
&&
کوری حرص است که آن معذور نیست
\\
چارمیخ شه ز رحمت دور نی
&&
چار میخ حاسدی مغفور نی
\\
ماهیا آخر نگر بنگر بشست
&&
بدگلویی چشم آخربینت بست
\\
با دو دیده اول و آخر ببین
&&
هین مباش اعور چو ابلیس لعین
\\
اعور آن باشد که حالی دید و بس
&&
چون بهایم بی‌خبر از بازپس
\\
چون دو چشم گاو در جرم تلف
&&
هم‌چو یک چشمست کش نبود شرف
\\
نصف قیمت ارزد آن دو چشم او
&&
که دو چشمش راست مسند چشم تو
\\
ور کنی یک چشم آدم‌زاده‌ای
&&
نصف قیمت لایقست از جاده‌ای
\\
زانک چشم آدمی تنها به خود
&&
بی دو چشم یار کاری می‌کند
\\
چشم خر چون اولش بی آخرست
&&
گر دو چشمش هست حکمش اعورست
\\
این سخن پایان ندارد وان خفیف
&&
می‌نویسد رقعه در طمع رغیف
\\
\end{longtable}
\end{center}
