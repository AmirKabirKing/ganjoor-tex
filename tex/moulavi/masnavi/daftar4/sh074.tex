\begin{center}
\section*{بخش ۷۴ - رقعهٔ دیگر نوشتن آن غلام پیش شاه چون جواب آن رقعهٔ اول نیافت}
\label{sec:sh074}
\addcontentsline{toc}{section}{\nameref{sec:sh074}}
\begin{longtable}{l p{0.5cm} r}
نامهٔ دیگر نوشت آن بدگمان
&&
پر ز تشنیع و نفیر و پر فغان
\\
که یکی رقعه نبشتم پیش شه
&&
ای عجب آنجا رسید و یافت ره
\\
آن دگر را خواند هم آن خوب‌خد
&&
هم نداد او را جواب و تن بزد
\\
خشک می‌آورد او را شهریار
&&
او مکرر کرد رقعه پنج بار
\\
گفت حاجب آخر او بندهٔ شماست
&&
گر جوابش بر نویسی هم رواست
\\
از شهی تو چه کم گردد اگر
&&
برغلام و بنده اندازی نظر
\\
گفت این سهلست اما احمقست
&&
مرد احمق زشت و مردود حقست
\\
گرچه آمرزم گناه و زلتش
&&
هم کند بر من سرایت علتش
\\
صد کس از گرگین همه گرگین شوند
&&
خاصه این گر خبیث ناپسند
\\
گر کم عقلی مبادا گبر را
&&
شوم او بی‌آب دارد ابر را
\\
نم نبارد ابر از شومی او
&&
شهر شد ویرانه از بومی او
\\
از گر آن احمقان طوفان نوح
&&
کرد ویران عالمی را در فضوح
\\
گفت پیغامبر که احمق هر که هست
&&
او عدو ماست و غول ره‌زنست
\\
هر که او عاقل بود از جان ماست
&&
روح او و ریح او ریحان ماست
\\
عقل دشنامم دهد من راضیم
&&
زانک فیضی دارد از فیاضیم
\\
نبود آن دشنام او بی‌فایده
&&
نبود آن مهمانیش بی‌مایده
\\
احمق ار حلوا نهد اندر لبم
&&
من از آن حلوای او اندر تبم
\\
این یقین دان گر لطیف و روشنی
&&
نیست بوسهٔ کون خر را چاشنی
\\
سبلتت گنده کند بی‌فایده
&&
جامه از دیگش سیه بی‌مایده
\\
مایده عقلست نی نان و شوی
&&
نور عقلست ای پسر جان را غذی
\\
نیست غیر نور آدم را خورش
&&
از جز آن جان نیابد پرورش
\\
زین خورشها اندک اندک باز بر
&&
کین غذای خر بود نه آن حر
\\
تا غذای اصل را قابل شوی
&&
لقمه‌های نور را آکل شوی
\\
عکس آن نورست کین نان نان شدست
&&
فیض آن جانست کین جان جان شدست
\\
چون خوری یکبار از ماکول نور
&&
خاک ریزی بر سر نان و تنور
\\
عقل دو عقلست اول مکسبی
&&
که در آموزی چو در مکتب صبی
\\
از کتاب و اوستاد و فکر و ذکر
&&
از معانی وز علوم خوب و بکر
\\
عقل تو افزون شود بر دیگران
&&
لیک تو باشی ز حفظ آن گران
\\
لوح حافظ باشی اندر دور و گشت
&&
لوح محفوظ اوست کو زین در گذشت
\\
عقل دیگر بخشش یزدان بود
&&
چشمهٔ آن در میان جان بود
\\
چون ز سینه آب دانش جوش کرد
&&
نه شود گنده نه دیرینه نه زرد
\\
ور ره نبعش بود بسته چه غم
&&
کو همی‌جوشد ز خانه دم به دم
\\
عقل تحصیلی مثال جویها
&&
کان رود در خانه‌ای از کویها
\\
راه آبش بسته شد شد بی‌نوا
&&
از درون خویشتن جو چشمه را
\\
\end{longtable}
\end{center}
