\begin{center}
\section*{بخش ۱۰۲ - مشورت کردن فرعون با ایسیه در ایمان آوردن به موسی علیه‌السلام}
\label{sec:sh102}
\addcontentsline{toc}{section}{\nameref{sec:sh102}}
\begin{longtable}{l p{0.5cm} r}
باز گفت او این سخن با ایسیه
&&
گفت جان افشان برین ای دل‌سیه
\\
بس عنایتهاست متن این مقال
&&
زود در یاب ای شه نیکو خصال
\\
وقت کشت آمد زهی پر سود کشت
&&
این بگفت و گریه کرد و گرم گشت
\\
بر جهید از جا و گفتا بخ لک
&&
آفتابی تاجر گشتت ای کلک
\\
عیب کل را خود بپوشاند کلاه
&&
خاصه چون باشد کله خورشید و ماه
\\
هم در آن مجلس که بشنیدی تو این
&&
چون نگفتی آری و صد آفرین
\\
این سخن در گوش خورشید ار شدی
&&
سرنگون بر بوی این زیر آمدی
\\
هیچ می‌دانی چه وعده‌ست و چه داد
&&
می‌کند ابلیس را حق افتقاد
\\
چون بدین لطف آن کریمت باز خواند
&&
ای عجب چون زهره‌ات بر جای ماند
\\
زهره‌ات ندرید تا زان زهره‌ات
&&
بودی اندر هر دو عالم بهره‌ات
\\
زهره‌ای کز بهرهٔ حق بر درد
&&
چون شهیدان از دو عالم بر خورد
\\
غافلی هم حکمتست و این عمی
&&
تا بماند لیک تا این حد چرا
\\
غافلی هم حکمتست و نعمتست
&&
تا نپرد زود سرمایه ز دست
\\
لیک نی چندانک ناسوری شود
&&
زهر جان و عقل رنجوری شود
\\
خود کی یابد این چنین بازار را
&&
که به یک گل می‌خری گلزار را
\\
دانه‌ای را صد درختستان عوض
&&
حبه‌ای را آمدت صد کان عوض
\\
کان لله دادن آن حبه است
&&
تا که کان‌الله له آید به دست
\\
زآنک این هوی ضعیف بی‌قرار
&&
هست شد زان هوی رب پایدار
\\
هوی فانی چونک خود فا او سپرد
&&
گشت باقی دایم و هرگز نمرد
\\
هم‌چو قطرهٔ خایف از باد و ز خاک
&&
که فنا گردد بدین هر دو هلاک
\\
چون به اصل خود که دریا بود جست
&&
از تف خورشید و باد و خاک رست
\\
ظاهرش گم گشت در دریا و لیک
&&
ذات او معصوم و پا بر جا و نیک
\\
هین بده ای قطره خود را بی‌ندم
&&
تا بیابی در بهای قطره یم
\\
هین بده ای قطره خود را این شرف
&&
در کف دریا شو آمن از تلف
\\
خود کرا آید چنین دولت به دست
&&
قطره‌ای را بحری تقاضاگر شدست
\\
الله الله زود بفروش و بخر
&&
قطره‌ای ده بحر پر گوهر ببر
\\
الله الله هیچ تاخیری مکن
&&
که ز بحر لطف آمد این سخن
\\
لطف اندر لطف این گم می‌شود
&&
که اسفلی بر چرخ هفتم می‌شود
\\
هین که یک بازی فتادت بوالعجب
&&
هیچ طالب این نیابد در طلب
\\
گفت با هامان بگویم ای ستیر
&&
شاه را لازم بود رای وزیر
\\
گفت با هامان مگو این راز را
&&
کور کمپیری چه داند باز را
\\
\end{longtable}
\end{center}
