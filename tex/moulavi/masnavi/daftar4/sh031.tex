\begin{center}
\section*{بخش ۳۱ - حکایت آن مرد تشنه کی از سر جوز بن جوز می‌ریخت در جوی آب کی در گو بود و به آب نمی‌رسید تا به افتادن جوز بانگ آب بشنود و او را چو سماع خوش بانگ آب اندر طرب می‌آورد}
\label{sec:sh031}
\addcontentsline{toc}{section}{\nameref{sec:sh031}}
\begin{longtable}{l p{0.5cm} r}
در نغولی بود آب آن تشنه راند
&&
بر درخت جوز جوزی می‌فشاند
\\
می‌فتاد از جوزبن جوز اندر آب
&&
بانگ می‌آمد همی دید او حباب
\\
عاقلی گفتش که بگذار ای فتی
&&
جوزها خود تشنگی آرد ترا
\\
بیشتر در آب می‌افتد ثمر
&&
آب در پستیست از تو دور در
\\
تا تو از بالا فرو آیی به زور
&&
آب جویش برده باشد تا به دور
\\
گفت قصدم زین فشاندن جوز نیست
&&
تیزتر بنگر برین ظاهر مه‌ایست
\\
قصد من آنست که آید بانگ آب
&&
هم ببینم بر سر آب این حباب
\\
تشنه را خود شغل چه بود در جهان
&&
گرد پای حوض گشتن جاودان
\\
گرد جو و گرد آب و بانگ آب
&&
هم‌چو حاجی طایف کعبهٔ صواب
\\
هم‌چنان مقصود من زین مثنوی
&&
ای ضیاء الحق حسام‌الدین توی
\\
مثنوی اندر فروع و در اصول
&&
جمله آن تست کردستی قبول
\\
در قبول آرند شاهان نیک و بد
&&
چون قبول آرند نبود بیش رد
\\
چون نهالی کاشتی آبش بده
&&
چون گشادش داده‌ای بگشا گره
\\
قصدم از الفاظ او راز توست
&&
قصدم از انشایش آواز توست
\\
پیش من آوازت آواز خداست
&&
عاشق از معشوق حاشا که جداست
\\
اتصالی بی‌تکیف بی‌قیاس
&&
هست رب‌الناس را با جان ناس
\\
لیک گفتم ناس من نسناس نی
&&
ناس غیر جان جان‌اشناس نی
\\
ناس مردم باشد و کو مردمی
&&
تو سر مردم ندیدستی دمی
\\
ما رمیت اذ رمیت خوانده‌ای
&&
لیک جسمی در تجزی مانده‌ای
\\
ملک جسمت را چو بلقیس ای غبی
&&
ترک کن بهر سلیمان نبی
\\
می‌کنم لا حول نه از گفت خویش
&&
بلک از وسواس آن اندیشه کیش
\\
کو خیالی می‌کند در گفت من
&&
در دل از وسواس و انکارات ظن
\\
می‌کنم لا حول یعنی چاره نیست
&&
چون ترا در دل بضدم گفتنیست
\\
چونک گفت من گرفتت در گلو
&&
من خمش کردم تو آن خود بگو
\\
آن یکی نایی خوش نی می‌زدست
&&
ناگهان از مقعدش بادی بجست
\\
نای را بر کون نهاد او که ز من
&&
گر تو بهتر می‌زنی بستان بزن
\\
ای مسلمان خود ادب اندر طلب
&&
نیست الا حمل از هر بی‌ادب
\\
هر که را بینی شکایت می‌کند
&&
که فلان کس راست طبع و خوی بد
\\
این شکایت‌گر بدان که بدخو است
&&
که مر آن بدخوی را او بدگو است
\\
زانک خوش‌خو آن بود کو در خمول
&&
باشد از بدخو و بدطبعان حمول
\\
لیک در شیخ آن گله ز آمر خداست
&&
نه پی خشم و ممارات و هواست
\\
آن شکایت نیست هست اصلاح جان
&&
چون شکایت کردن پیغامبران
\\
ناحمولی انبیا از امر دان
&&
ورنه حمالست بد را حلمشان
\\
طبع را کشتند در حمل بدی
&&
ناحمولی گر بود هست ایزدی
\\
ای سلیمان در میان زاغ و باز
&&
حلم حق شو با همه مرغان بساز
\\
ای دو صد بلقیس حلمت را زبون
&&
که اهد قومی انهم لا یعلمون
\\
\end{longtable}
\end{center}
