\begin{center}
\section*{بخش ۵۲ - قصهٔ رستن خروب در گوشهٔ مسجد اقصی و غمگین شدن سلیمان علیه‌السلام از آن چون به سخن آمد با او و خاصیت و نام  خود بگفت}
\label{sec:sh052}
\addcontentsline{toc}{section}{\nameref{sec:sh052}}
\begin{longtable}{l p{0.5cm} r}
پس سلیمان دید اندر گوشه‌ای
&&
نوگیاهی رسته هم‌چون خوشه‌ای
\\
دید بس نادر گیاهی سبز و تر
&&
می‌ربود آن سبزیش نور از بصر
\\
پس سلامش کرد در حال آن حشیش
&&
او جوابش گفت و بشکفت از خوشیش
\\
گفت نامت چیست برگو بی‌دهان
&&
گفت خروبست ای شاه جهان
\\
گفت اندر تو چه خاصیت بود
&&
گفت من رستم مکان ویران شود
\\
من که خروبم خراب منزلم
&&
هادم بنیاد این آب و گلم
\\
پس سلیمان آن زمان دانست زود
&&
که اجل آمد سفر خواهد نمود
\\
گفت تا من هستم این مسجد یقین
&&
در خلل ناید ز آفات زمین
\\
تا که من باشم وجود من بود
&&
مسجداقصی مخلخل کی شود
\\
پس که هدم مسجد ما بی‌گمان
&&
نبود الا بعد مرگ ما بدان
\\
مسجدست آن دل که جسمش ساجدست
&&
یار بد خروب هر جا مسجدست
\\
یار بد چون رست در تو مهر او
&&
هین ازو بگریز و کم کن گفت وگو
\\
برکن از بیخش که گر سر بر زند
&&
مر ترا و مسجدت را بر کند
\\
عاشقا خروب تو آمد کژی
&&
هم‌چو طفلان سوی کژ چون می‌غژی
\\
خویش مجرم دان و مجرم گو مترس
&&
تا ندزدد از تو آن استاد درس
\\
چون بگویی جاهلم تعلیم ده
&&
این چنین انصاف از ناموس به
\\
از پدر آموز ای روشن‌جبین
&&
ربنا گفت و ظلمنا پیش ازین
\\
نه بهانه کرد و نه تزویر ساخت
&&
نه لوای مکر و حیلت بر فراخت
\\
باز آن ابلیس بحث آغاز کرد
&&
که بدم من سرخ رو کردیم زرد
\\
رنگ رنگ تست صباغم توی
&&
اصل جرم و آفت و داغم توی
\\
هین بخوان رب بما اغویتنی
&&
تا نگردی جبری و کژ کم تنی
\\
بر درخت جبر تا کی بر جهی
&&
اختیار خویش را یک‌سو نهی
\\
هم‌چو آن ابلیس و ذریات او
&&
با خدا در جنگ و اندر گفت و گو
\\
چون بود اکراه با چندان خوشی
&&
که تو در عصیان همی دامن کشی
\\
آن‌چنان خوش کس رود در مکرهی
&&
کس چنان رقصان دود در گم‌رهی
\\
بیست مرده جنگ می‌کردی در آن
&&
کت همی‌دادند پند آن دیگران
\\
که صواب اینست و راه اینست و بس
&&
کی زند طعنه مرا جز هیچ‌کس
\\
کی چنین گوید کسی کو مکر هست
&&
چون چنین جنگد کسی کو بی‌رهست
\\
هر چه نفست خواست داری اختیار
&&
هر چه عقلت خواست آری اضطرار
\\
داند او کو نیک‌بخت و محرمست
&&
زیرکی ز ابلیس و عشق از آدمست
\\
زیرکی سباحی آمد در بحار
&&
کم رهد غرقست او پایان کار
\\
هل سباحت را رها کن کبر و کین
&&
نیست جیحون نیست جو دریاست این
\\
وانگهان دریای ژرف بی‌پناه
&&
در رباید هفت دریا را چو کاه
\\
عشق چون کشتی بود بهر خواص
&&
کم بود آفت بود اغلب خلاص
\\
زیرکی بفروش و حیرانی بخر
&&
زیرکی ظنست و حیرانی نظر
\\
عقل قربان کن به پیش مصطفی
&&
حسبی الله گو که الله‌ام کفی
\\
هم‌چو کنعان سر ز کشتی وا مکش
&&
که غرورش داد نفس زیرکش
\\
که برآیم بر سر کوه مشید
&&
منت نوحم چرا باید کشید
\\
چون رمی از منتش بر جان ما
&&
چونک شکر و منتش گوید خدا
\\
تو چه دانی ای غرارهٔ پر حسد
&&
منت او را خدا هم می‌کشد
\\
کاشکی او آشنا ناموختی
&&
تا طمع در نوح و کشتی دوختی
\\
کاش چون طفل از حیل جاهل بدی
&&
تا چو طفلان چنگ در مادر زدی
\\
یا به علم نقل کم بودی ملی
&&
علم وحی دل ربودی از ولی
\\
با چنین نوری چو پیش آری کتاب
&&
جان وحی آسای تو آرد عتاب
\\
چون تیمم با وجود آب دان
&&
علم نقلی با دم قطب زمان
\\
خویش ابله کن تبع می‌رو سپس
&&
رستگی زین ابلهی یابی و بس
\\
اکثر اهل الجنه البله ای پسر
&&
بهر این گفتست سلطان البشر
\\
زیرکی چون کبر و باد انگیز تست
&&
ابلهی شو تا بماند دل درست
\\
ابلهی نه کو به مسخرگی دوتوست
&&
ابلهی کو واله و حیران هوست
\\
ابلهان‌اند آن زنان دست بر
&&
از کف ابله وز رخ یوسف نذر
\\
عقل را قربان کن اندر عشق دوست
&&
عقلها باری از آن سویست کوست
\\
عقلها آن سو فرستاده عقول
&&
مانده این سو که نه معشوقست گول
\\
زین سر از حیرت گر این عقلت رود
&&
هر سو مویت سر و عقلی شود
\\
نیست آن سو رنج فکرت بر دماغ
&&
که دماغ و عقل روید دشت و باغ
\\
سوی دشت از دشت نکته بشنوی
&&
سوی باغ آیی شود نخلت روی
\\
اندرین ره ترک کن طاق و طرنب
&&
تا قلاوزت نجنبد تو مجنب
\\
هر که او بی سر بجنبد دم بود
&&
جنبشش چون جنبش کزدم بود
\\
کژرو و شب کور و زشت و زهرناک
&&
پیشهٔ او خستن اجسام پاک
\\
سر بکوب آن را که سرش این بود
&&
خلق و خوی مستمرش این بود
\\
خود صلاح اوست آن سر کوفتن
&&
تا رهد جان‌ریزه‌اش زان شوم‌تن
\\
واستان آن دست دیوانه سلاح
&&
تا ز تو راضی شود عدل و صلاح
\\
چون سلاحش هست و عقلش نه ببند
&&
دست او را ورنه آرد صد گزند
\\
\end{longtable}
\end{center}
