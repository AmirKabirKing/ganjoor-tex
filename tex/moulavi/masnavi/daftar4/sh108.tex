\begin{center}
\section*{بخش ۱۰۸ - منازعت امیران عرب با مصطفی علیه‌السلام کی ملک را مقاسمت کن با ما تا نزاعی نباشد و جواب فرمودن مصطفی علیه‌السلام کی من مامورم درین امارت و بحث ایشان از طرفین}
\label{sec:sh108}
\addcontentsline{toc}{section}{\nameref{sec:sh108}}
\begin{longtable}{l p{0.5cm} r}
آن امیران عرب گرد آمدند
&&
نزد پیغامبر منازع می‌شدند
\\
که تو میری هر یک از ما هم امیر
&&
بخش کن این ملک و بخش خود بگیر
\\
هر یکی در بخش خود انصاف‌جو
&&
تو ز بخش ما دو دست خود بشو
\\
گفت میری مر مرا حق داده است
&&
سروری و امر مطلق داده است
\\
کین قران احمدست و دور او
&&
هین بگیرید امر او را اتقوا
\\
قوم گفتندش که ما هم زان قضا
&&
حاکمیم و داد امیریمان خدا
\\
گفت لیکن مر مرا حق ملک داد
&&
مر شما را عاریه از بهر زاد
\\
میری من تا قیامت باقیست
&&
میری عاریتی خواهد شکست
\\
قوم گفتند ای امیر افزون مگو
&&
چیست حجت بر فزون‌جویی تو
\\
در زمان ابری برآمد ز امر مر
&&
سیل آمد گشت آن اطراف پر
\\
رو به شهر آورد سیل بس مهیب
&&
اهل شهر افغان‌کنان جمله رعیب
\\
گفت پیغامبر که وقت امتحان
&&
آمد اکنون تا گمارد گردد عیان
\\
هر امیری نیزهٔ خود در فکند
&&
تا شود در امتحان آن سیل‌بند
\\
پس قضیب انداخت در وی مصطفی
&&
آن قضیب معجز فرمان روا
\\
نیزه‌ها را هم‌چو خاشاکی ربود
&&
آب تیز سیل پرجوش عنود
\\
نیزه‌ها گم گشت جمله و آن قضیب
&&
بر سر آب ایستاده چون رقیب
\\
ز اهتمام آن قضیب آن سیل زفت
&&
روبگردانید و آن سیلاب رفت
\\
چون بدیدند از وی آن امر عظیم
&&
پس مقر گشتند آن میران ز بیم
\\
جز سه کس که حقد ایشان چیره بود
&&
ساحرش گفتند و کاهی از جحود
\\
ملک بر بسته چنان باشد ضعیف
&&
ملک بر رسته چنین باشد شریف
\\
نیزه‌ها را گر ندیدی با قضیب
&&
نامشان بین نام او بین این نجیب
\\
نامشان را سیل تیز مرگ برد
&&
نام او و دولت تیزش نمرد
\\
پنج نوبت می‌زنندش بر دوام
&&
هم‌چنین هر روز تا روز قیام
\\
گر ترا عقلست کردم لطفها
&&
ور خری آورده‌ام خر را عصا
\\
آنچنان زین آخرت بیرون کنم
&&
کز عصا گوش و سرت پر خون کنم
\\
اندرین آخر خران و مردمان
&&
می‌نیابند از جفای تو امان
\\
نک عصا آورده‌ام بهر ادب
&&
هر خری را کو نباشد مستحب
\\
اژدهایی می‌شود در قهر تو
&&
که اژدهایی گشته‌ای در فعل و خو
\\
اژدهای کوهیی تو بی‌امان
&&
لیک بنگر اژدهای آسمان
\\
این عصا از دوزخ آمد چاشنی
&&
که هلا بگریز اندر روشنی
\\
ورنه در مانی تو در دندان من
&&
مخلصت نبود ز در بندان من
\\
این عصایی بود این دم اژدهاست
&&
تا نگویی دوزخ یزدان کجاست
\\
\end{longtable}
\end{center}
