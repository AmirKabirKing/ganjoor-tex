\begin{center}
\section*{بخش ۸۵ - شخصی به وقت استنجا می‌گفت اللهم ارحنی رائحة الجنه به جای آنک اللهم اجعلنی من التوابین واجعلنی من  المتطهرین کی ورد استنجاست و ورد استنجا را به وقت استنشاق می‌گفت عزیزی بشنید و این را طاقت نداشت}
\label{sec:sh085}
\addcontentsline{toc}{section}{\nameref{sec:sh085}}
\begin{longtable}{l p{0.5cm} r}
آن یکی در وقت استنجا بگفت
&&
که مرا با بوی جنت دار جفت
\\
گفت شخصی خوب ورد آورده‌ای
&&
لیک سوراخ دعا گم کرده‌ای
\\
این دعا چون ورد بینی بود چون
&&
ورد بینی را تو آوردی به کون
\\
رایحهٔ جنت ز بینی یافت حر
&&
رایحهٔ جنت کم آید از دبر
\\
ای تواضع برده پیش ابلهان
&&
وی تکبر برده تو پیش شهان
\\
آن تکبر بر خسان خوبست و چست
&&
هین مرو معکوس عکسش بند تست
\\
از پی سوراخ بینی رست گل
&&
بو وظیفهٔ بینی آمد ای عتل
\\
بوی گل بهر مشامست ای دلیر
&&
جای آن بو نیست این سوراخ زیر
\\
کی ازین جا بوی خلد آید ترا
&&
بو ز موضع جو اگر باید ترا
\\
هم‌چنین حب الوطن باشد درست
&&
تو وطن بشناس ای خواجه نخست
\\
گفت آن ماهی زیرک ره کنم
&&
دل ز رای و مشورتشان بر کنم
\\
نیست وقت مشورت هین راه کن
&&
چون علی تو آه اندر چاه کن
\\
محرم آن آه کم‌یابست بس
&&
شب رو و پنهان‌روی کن چون عسس
\\
سوی دریا عزم کن زین آب‌گیر
&&
بحر جو و ترک این گرداب گیر
\\
سینه را پا ساخت می‌رفت آن حذور
&&
از مقام با خطر تا بحر نور
\\
هم‌چو آهو کز پی او سگ بود
&&
می‌دود تا در تنش یک رگ بود
\\
خواب خرگوش و سگ اندر پی خطاست
&&
خواب خود در چشم ترسنده کجاست
\\
رفت آن ماهی ره دریا گرفت
&&
راه دور و پهنهٔ پهنا گرفت
\\
رنجها بسیار دید و عاقبت
&&
رفت آخر سوی امن و عافیت
\\
خویشتن افکند در دریای ژرف
&&
که نیابد حد آن را هیچ طرف
\\
پس چو صیادان بیاوردند دام
&&
نیم‌عاقل را از آن شد تلخ کام
\\
گفت اه من فوت کردم فرصه را
&&
چون نگشتم همره آن رهنما
\\
ناگهان رفت او ولیکن چونک رفت
&&
می‌ببایستم شدن در پی بتفت
\\
بر گذشته حسرت آوردن خطاست
&&
باز ناید رفته یاد آن هباست
\\
\end{longtable}
\end{center}
