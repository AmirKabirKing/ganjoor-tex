\begin{center}
\section*{بخش ۹۶ - گفتن موسی علیه‌السلام فرعون را کی از من یک پند قبول کن و چهار فضیلت عوض بستان}
\label{sec:sh096}
\addcontentsline{toc}{section}{\nameref{sec:sh096}}
\begin{longtable}{l p{0.5cm} r}
هین ز من بپذیر یک چیز و بیار
&&
پس ز من بستان عوض آن را چهار
\\
گفت ای موسی کدامست آن یکی
&&
شرح کن با من از آن یک اندکی
\\
گفت آن یک که بگویی آشکار
&&
که خدایی نیست غیر کردگار
\\
خالق افلاک و انجم بر علا
&&
مردم و دیو و پری و مرغ را
\\
خالق دریا و دشت و کوه و تیه
&&
ملکت او بی‌حد و او بی‌شبیه
\\
گفت ای موسی کدامست آن چهار
&&
که عوض بدهی مرا بر گو بیار
\\
تا بود کز لطف آن وعدهٔ حسن
&&
سست گردد چارمیخ کفر من
\\
بوک زان خوش وعده‌های مغتنم
&&
برگشاید قفل کفر صد منم
\\
بوک از تاثیر جوی انگبین
&&
شهد گردد در تنم این زهر کین
\\
یا ز عکس جوی آن پاکیزه شیر
&&
پرورش یابد دمی عقل اسیر
\\
یا بود کز عکس آن جوهای خمر
&&
مست گردم بو برم از ذوق امر
\\
یا بود کز لطف آن جوهای آب
&&
تازگی یابد تن شورهٔ خراب
\\
شوره‌ام را سبزه‌ای پیدا شود
&&
خارزارم جنت ماوی شود
\\
بوک از عکس بهشت و چار جو
&&
جان شود از یاری حق یارجو
\\
آنچنان که از عکس دوزخ گشته‌ام
&&
آتش و در قهر حق آغشته‌ام
\\
گه ز عکس مار دوزخ هم‌چو مار
&&
گشته‌ام بر اهل جنت زهربار
\\
گه ز عکس جوشش آب حمیم
&&
آب ظلمم کرده خلقان را رمیم
\\
من ز عکس زمهریرم زمهریر
&&
یا ز عکس آن سعیرم چون سعیر
\\
دوزخ درویش و مظلومم کنون
&&
وای آنک یابمش ناگه زبون
\\
\end{longtable}
\end{center}
