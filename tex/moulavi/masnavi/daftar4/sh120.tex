\begin{center}
\section*{بخش ۱۲۰ - اختیار کردن پادشاه دختر درویش زاهدی را از جهت پسر و اعتراض کردن اهل حرم و ننگ داشتن ایشان از پیوندی درویش}
\label{sec:sh120}
\addcontentsline{toc}{section}{\nameref{sec:sh120}}
\begin{longtable}{l p{0.5cm} r}
مادر شه‌زاده گفت از نقص عقل
&&
شرط کفویت بود در عقل نقل
\\
تو ز شح و بخل خواهی وز دها
&&
تا ببندی پور ما را بر گدا
\\
گفت صالح را گدا گفتن خطاست
&&
کو غنی القلب از داد خداست
\\
در قناعت می‌گریزد از تقی
&&
نه از لیمی و کسل هم‌چون گدا
\\
قلتی کان از قناعت وز تقاست
&&
آن ز فقر و قلت دونان جداست
\\
حبه‌ای آن گر بیابد سر نهد
&&
وین ز گنج زر به همت می‌جهد
\\
شه که او از حرص قصد هر حرام
&&
می‌کند او را گدا گوید همام
\\
گفت کو شهر و قلاع او را جهاز
&&
یا نثار گوهر و دینار ریز
\\
گفت رو هر که غم دین برگزید
&&
باقی غمها خدا از وی برید
\\
غالب آمد شاه و دادش دختری
&&
از نژاد صالحی خوش جوهری
\\
در ملاحت خود نظیر خود نداشت
&&
چهره‌اش تابان‌تر از خورشید چاشت
\\
حسن دختر این خصالش آنچنان
&&
کز نکویی می‌نگنجد در بیان
\\
صید دین کن تا رسد اندر تبع
&&
حسن و مال و جاه و بخت منتفع
\\
آخرت قطار اشتر دان به ملک
&&
در تبع دنیاش هم‌چون پشم و پشک
\\
پشم بگزینی شتر نبود ترا
&&
ور بود اشتر چه قیمت پشم را
\\
چون بر آمد این نکاح آن شاه را
&&
با نژاد صالحان بی مرا
\\
از قضا کمپیرکی جادو که بود
&&
عاشق شه‌زادهٔ با حسن و جود
\\
جادوی کردش عجوزهٔ کابلی
&&
کی برد زان رشک سحر بابلی
\\
شه بچه شد عاشق کمپیر زشت
&&
تا عروس و آن عروسی را بهشت
\\
یک سیه دیوی و کابولی زنی
&&
گشت به شه‌زاده ناگه ره‌زنی
\\
آن نودساله عجوزی گنده کس
&&
نه خرد هشت آن ملک را و نه نس
\\
تا به سالی بود شه‌زاده اسیر
&&
بوسه‌جایش نعل کفش گنده پیر
\\
صحبت کمپیر او را می‌درود
&&
تا ز کاهش نیم‌جانی مانده بود
\\
دیگران از ضعف وی با درد سر
&&
او ز سکر سحر از خود بی‌خبر
\\
این جهان بر شاه چون زندان شده
&&
وین پسر بر گریه‌شان خندان شده
\\
شاه بس بیچاره شد در برد و مات
&&
روز و شب می‌کرد قربان و زکات
\\
زانک هر چاره که می‌کرد آن پدر
&&
عشق کمپیرک همی‌شد بیشتر
\\
پس یقین گشتش که مطلق آن سریست
&&
چاره او را بعد از این لابه گریست
\\
سجده می‌کرد او که هم فرمان تراست
&&
غیر حق بر ملک حق فرمان کراست
\\
لیک این مسکین همی‌سوزد چو عود
&&
دست گیرش ای رحیم و ای ودود
\\
تا ز یا رب یا رب و افغان شاه
&&
ساحری استاد پیش آمد ز راه
\\
\end{longtable}
\end{center}
