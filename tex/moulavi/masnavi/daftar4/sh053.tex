\begin{center}
\section*{بخش ۵۳ - بیان آنک حصول علم و مال و جاه بدگوهران را فضیحت اوست و چون شمشیریست کی افتادست به دست راه‌زن}
\label{sec:sh053}
\addcontentsline{toc}{section}{\nameref{sec:sh053}}
\begin{longtable}{l p{0.5cm} r}
بدگهر را علم و فن آموختن
&&
دادن تیغی به دست راه‌زن
\\
تیغ دادن در کف زنگی مست
&&
به که آید علم ناکس را به دست
\\
علم و مال و منصب و جاه و قران
&&
فتنه آمد در کف بدگوهران
\\
پس غزا زین فرض شد بر مؤمنان
&&
تا ستانند از کف مجنون سنان
\\
جان او مجنون تنش شمشیر او
&&
واستان شمشیر را زان زشت‌خو
\\
آنچ منصب می‌کند با جاهلان
&&
از فضیحت کی کند صد ارسلان
\\
عیب او مخفیست چون آلت بیافت
&&
مارش از سوراخ بر صحرا شتافت
\\
جمله صحرا مار و کزدم پر شود
&&
چونک جاهل شاه حکم مر شود
\\
مال و منصب ناکسی که آرد به دست
&&
طالب رسوایی خویش او شدست
\\
یا کند بخل و عطاها کم دهد
&&
یا سخا آرد بنا موضع نهد
\\
شاه را در خانهٔ بیذق نهد
&&
این چنین باشد عطا که احمق دهد
\\
حکم چون در دست گمراهی فتاد
&&
جاه پندارید در چاهی فتاد
\\
راه نمی‌داند قلاووزی کند
&&
جان زشت او جهان‌سوزی کند
\\
طفل راه فقر چون پیری گرفت
&&
پی‌روان را غول ادباری گرفت
\\
که بیا تا ماه بنمایم ترا
&&
ماه را هرگز ندید آن بی‌صفا
\\
چون نمایی چون ندیدستی به عمر
&&
عکس مه در آب هم ای خام غمر
\\
احمقان سرور شدستند و ز بیم
&&
عاقلان سرها کشیده در گلیم
\\
\end{longtable}
\end{center}
