\begin{center}
\section*{بخش ۱۱۴ - گفتن خلیل مر جبرئیل را علیهماالسلام  چون پرسیدش کی الک حاجة خلیل  جوابش داد کی اما الیک فلا}
\label{sec:sh114}
\addcontentsline{toc}{section}{\nameref{sec:sh114}}
\begin{longtable}{l p{0.5cm} r}
من خلیل وقتم و او جبرئیل
&&
من نخواهم در بلا او را دلیل
\\
او ادب ناموخت از جبریل راد
&&
که بپرسید از خلیل حق مراد
\\
که مرادت هست تا یاری کنم
&&
ورنه بگریزم سبکباری کنم
\\
گفت ابراهیم نی رو از میان
&&
واسطه زحمت بود بعد العیان
\\
بهر این دنیاست مرسل رابطه
&&
مؤمنان را زانک هست او واسطه
\\
هر دل ار سامع بدی وحی نهان
&&
حرف و صوتی کی بدی اندر جهان
\\
گرچه او محو حقست و بی‌سرست
&&
لیک کار من از آن نازکترست
\\
کردهٔ او کردهٔ شاهست لیک
&&
پیش ضعفم بد نماینده‌ست نیک
\\
آنچ عین لطف باشد بر عوام
&&
قهر شد بر نازنینان کرام
\\
بس بلا و رنج می‌باید کشید
&&
عامه را تا فرق را توانند دید
\\
کین حروف واسطه ای یار غار
&&
پیش واصل خار باشد خار خار
\\
بس بلا و رنج بایست و وقوف
&&
تا رهد آن روح صافی از حروف
\\
لیک بعضی زین صدا کرتر شدند
&&
باز بعضی صافی و برتر شدند
\\
هم‌چو آب نیل آمد این بلا
&&
سعد را آبست و خون بر اشقیا
\\
هر که پایان‌بین‌تر او مسعودتر
&&
جدتر او کارد که افزون دید بر
\\
زانک داند کین جهان کاشتن
&&
هست بهر محشر و برداشتن
\\
هیچ عقدی بهر عین خود نبود
&&
بلک از بهر مقام ربح و سود
\\
هیچ نبود منکری گر بنگری
&&
منکری‌اش بهر عین منکری
\\
بل برای قهر خصم اندر حسد
&&
یا فزونی جستن و اظهار خود
\\
وآن فزونی هم پی طمع دگر
&&
بی‌معانی چاشنی ندهد صور
\\
زان همی‌پرسی چرا این می‌کنی
&&
که صور زیتست و معنی روشنی
\\
ورنه این گفتن چرا از بهر چیست
&&
چونک صورت بهر عین صورتیست
\\
این چرا گفتن سال از فایده‌ست
&&
جز برای این چرا گفتن بدست
\\
از چه رو فایدهٔ جویی ای امین
&&
چون بود فایده این خود همین
\\
پس نقوش آسمان و اهل زمین
&&
نیست حکمت کان بود بهر همین
\\
گر حکیمی نیست این ترتیب چیست
&&
ور حکیمی هست چون فعلش تهیست
\\
کس نسازد نقش گرمابه و خضاب
&&
جز پی قصد صواب و ناصواب
\\
\end{longtable}
\end{center}
