\begin{center}
\section*{بخش ۸۸ - بیان آنک عهد کردن احمق وقت گرفتاری و ندم هیچ وفایی ندارد کی لو ردوالعادوا لما نهوا عنه و انهم لکاذبون صبح کاذب وفا ندارد}
\label{sec:sh088}
\addcontentsline{toc}{section}{\nameref{sec:sh088}}
\begin{longtable}{l p{0.5cm} r}
عقل می‌گفتش حماقت با توست
&&
با حماقت عقل را آید شکست
\\
عقل را باشد وفای عهدها
&&
تو نداری عقل رو ای خربها
\\
عقل را یاد آید از پیمان خود
&&
پردهٔ نسیان بدراند خرد
\\
چونک عقلت نیست نسیان میر تست
&&
دشمن و باطل کن تدبیر تست
\\
از کمی عقل پروانهٔ خسیس
&&
یاد نارد ز آتش و سوز و حسیس
\\
چونک پرش سوخت توبه می‌کند
&&
آز و نسیانش بر آتش می‌زند
\\
ضبط و درک و حافظی و یادداشت
&&
عقل را باشد که عقل آن را فراشت
\\
چونک گوهر نیست تابش چون بود
&&
چون مذکر نیست ایابش چون بود
\\
این تمنی هم ز بی‌عقلی اوست
&&
که نبیند کان حماقت را چه خوست
\\
آن ندامت از نتیجهٔ رنج بود
&&
نه ز عقل روشن چون گنج بود
\\
چونک شد رنج آن ندامت شد عدم
&&
می‌نیرزد خاک آن توبه و ندم
\\
آن ندم از ظلمت غم بست بار
&&
پس کلام اللیل یمحوه النهار
\\
چون برفت آن ظلمت غم گشت خوش
&&
هم رود از دل نتیجه و زاده‌اش
\\
می‌کند او توبه و پیر خرد
&&
بانگ لو ردوا لعادوا می‌زند
\\
\end{longtable}
\end{center}
