\begin{center}
\section*{بخش ۸ - گفتن زن کی او در بند جهاز نیست مراد او ستر و صلاحست و جواب گفتن صوفی این را سرپوشیده}
\label{sec:sh008}
\addcontentsline{toc}{section}{\nameref{sec:sh008}}
\begin{longtable}{l p{0.5cm} r}
گفت گفتم من چنین عذری و او
&&
گفت نه من نیستم اسباب جو
\\
ما ز مال و زر ملول و تخمه‌ایم
&&
ما به حرص و جمع نه چون عامه‌ایم
\\
قصد ما سترست و پاکی و صلاح
&&
در دو عالم خود بدان باشد فلاح
\\
باز صوفی عذر درویشی بگفت
&&
و آن مکرر کرد تا نبود نهفت
\\
گفت زن من هم مکرر کرده‌ام
&&
بی‌جهازی را مقرر کرده‌ام
\\
اعتقاد اوست راسختر ز کوه
&&
که ز صد فقرش نمی‌آید شکوه
\\
او همی‌گوید مرادم عفتست
&&
از شما مقصود صدق و همتست
\\
گفت صوفی خود جهاز و مال ما
&&
دید و می‌بیند هویدا و خفا
\\
خانهٔ تنگی مقام یک تنی
&&
که درو پنهان نماند سوزنی
\\
باز ستر و پاکی و زهد و صلاح
&&
او ز ما به داند اندر انتصاح
\\
به ز ما می‌داند او احوال ستر
&&
وز پس و پیش و سر و دنبال ستر
\\
ظاهرا او بی‌جهاز و خادمست
&&
وز صلاح و ستر او خود عالمست
\\
شرح مستوری ز بابا شرط نیست
&&
چون برو پیدا چو روز روشنیست
\\
این حکایت را بدان گفتم که تا
&&
لاف کم بافی چو رسوا شد خطا
\\
مر ترا ای هم به دعوی مستزاد
&&
این بدستت اجتهاد و اعتقاد
\\
چون زن صوفی تو خاین بوده‌ای
&&
دام مکر اندر دغا بگشوده‌ای
\\
که ز هر ناشسته رویی کپ زنی
&&
شرم داری وز خدای خویش نی
\\
\end{longtable}
\end{center}
