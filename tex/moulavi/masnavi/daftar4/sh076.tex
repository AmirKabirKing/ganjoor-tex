\begin{center}
\section*{بخش ۷۶ - امیر کردن رسول علیه‌السلام جوان هذیلی را بر سریه‌ای کی در آن پیران  و جنگ آزمودگان بودند}
\label{sec:sh076}
\addcontentsline{toc}{section}{\nameref{sec:sh076}}
\begin{longtable}{l p{0.5cm} r}
یک سریه می‌فرستادش رسول
&&
به هر جنگ کافر و دفع فضول
\\
یک جوانی را گزید او از هذیل
&&
میر لشکر کردش و سالار خیل
\\
اصل لشکر بی‌گمان سرور بود
&&
قوم بی‌سرور تن بی‌سر بود
\\
این همه که مرده و پژمرده‌ای
&&
زان بود که ترک سرور کرده‌ای
\\
از کسل وز بخل وز ما و منی
&&
می‌کشی سر خویش را سر می‌کنی
\\
هم‌چو استوری که بگریزد ز بار
&&
او سر خود گیرد اندر کوهسار
\\
صاحبش در پی دوان کای خیره سر
&&
هر طرف گرگیست اندر قصد خر
\\
گر ز چشمم این زمان غایب شوی
&&
پیشت آید هر طرف گرگ قوی
\\
استخوانت را بخاید چون شکر
&&
که نبینی زندگانی را دگر
\\
آن مگیر آخر بمانی از علف
&&
آتش از بی‌هیزمی گردد تلف
\\
هین بمگریز از تصرف کردنم
&&
وز گرانی بار که جانت منم
\\
تو ستوری هم که نفست غالبست
&&
حکم غالب را بود ای خودپرست
\\
خر نخواندت اسپ خواندت ذوالجلال
&&
اسپ تازی را عرب گوید تعال
\\
میر آخر بود حق را مصطفی
&&
بهر استوران نفس پر جفا
\\
قل تعالوا گفت از جذب کرم
&&
تا ریاضتتان دهم من رایضم
\\
نفسها را تا مروض کرده‌ام
&&
زین ستوران بس لگدها خورده‌ام
\\
هر کجا باشد ریاضت‌باره‌ای
&&
از لگدهااش نباشد چاره‌ای
\\
لاجرم اغلب بلا بر انبیاست
&&
که ریاضت دادن خامان بلاست
\\
سکسکانید از دمم یرغا روید
&&
تا یواش و مرکب سلطان شوید
\\
قل تعالوا قل تعالو گفت رب
&&
ای ستوران رمیده از ادب
\\
گر نیایند ای نبی غمگین مشو
&&
زان دو بی‌تمکین تو پر از کین مشو
\\
گوش بعضی زین تعالواها کرست
&&
هر ستوری را صطبلی دیگرست
\\
منهزم گردند بعضی زین ندا
&&
هست هر اسپی طویلهٔ او جدا
\\
منقبض گردند بعضی زین قصص
&&
زانک هر مرغی جدا دارد قفس
\\
خود ملایک نیز ناهمتا بدند
&&
زین سبب بر آسمان صف صف شدند
\\
کودکان گرچه به یک مکتب درند
&&
در سبق هر یک ز یک بالاترند
\\
مشرقی و مغربی را حسهاست
&&
منصب دیدار حس چشم‌راست
\\
صد هزاران گوشها گر صف زنند
&&
جمله محتاجان چشم روشن‌اند
\\
باز صف گوشها را منصبی
&&
در سماع جان و اخبار و نبی
\\
صد هزاران چشم را آن راه نیست
&&
هیچ چشمی از سماع آگاه نیست
\\
هم‌چنین هر حس یک یک می‌شمر
&&
هر یکی معزول از آن کار دگر
\\
پنج حس ظاهر و پنج اندرون
&&
ده صف‌اند اندر قیام الصافون
\\
هر کسی کو از صف دین سرکشست
&&
می‌رود سوی صفی کان واپسست
\\
تو ز گفتار تعالوا کم مکن
&&
کیمیای بس شگرفست این سخن
\\
گر مسی گردد ز گفتارت نفیر
&&
کیمیا را هیچ از وی وام گیر
\\
این زمان گر بست نفس ساحرش
&&
گفت تو سودش کند در آخرش
\\
قل تعالوا قل تعالوا ای غلام
&&
هین که ان الله یدعوا للسلام
\\
خواجه باز آ از منی و از سری
&&
سروری جو کم طلب کن سروری
\\
\end{longtable}
\end{center}
