\begin{center}
\section*{بخش ۲۲ - قصهٔ هدیه فرستادن بلقیس از شهر سبا  سوی سلیمان علیه‌السلام}
\label{sec:sh022}
\addcontentsline{toc}{section}{\nameref{sec:sh022}}
\begin{longtable}{l p{0.5cm} r}
هدیهٔ بلقیس چل استر بدست
&&
بار آنها جمله خشت زر بدست
\\
چون به صحرای سلیمانی رسید
&&
فرش آن را جمله زر پخته دید
\\
بر سر زر تا چهل منزل براند
&&
تا که زر را در نظر آبی نماند
\\
بارها گفتند زر را وا بریم
&&
سوی مخزن ما چه بیگار اندریم
\\
عرصه‌ای کش خاک زر ده دهیست
&&
زر به هدیه بردن آنجا ابلهیست
\\
ای ببرده عقل هدیه تا اله
&&
عقل آنجا کمترست از خاک راه
\\
چون کساد هدیه آنجا شد پدید
&&
شرمساریشان همی واپس کشید
\\
باز گفتند ار کساد و ار روا
&&
چیست بر ما بنده فرمانیم ما
\\
گر زر و گر خاک ما را بردنیست
&&
امر فرمان‌ده به جا آوردنیست
\\
گر بفرمایند که واپس برید
&&
هم به فرمان تحفه را باز آورید
\\
خنده‌ش آمد چون سلیمان آن بدید
&&
کز شما من کی طلب کردم ثرید
\\
من نمی‌گویم مرا هدیه دهید
&&
بلک گفتم لایق هدیه شوید
\\
که مرا از غیب نادر هدیه‌هاست
&&
که بشر آن را نیارد نیز خواست
\\
می‌پرستید اختری کو زر کند
&&
رو باو آرید کو اختر کند
\\
می‌پرستید آفتاب چرخ را
&&
خوار کرده جان عالی‌نرخ را
\\
آفتاب از امر حق طباخ ماست
&&
ابلهی باشد که گوییم او خداست
\\
آفتابت گر بگیرد چون کنی
&&
آن سیاهی زو تو چون بیرون کنی
\\
نه به درگاه خدا آری صداع
&&
که سیاهی را ببر وا ده شعاع
\\
گر کشندت نیم‌شب خورشید کو
&&
تا بنالی یا امان خواهی ازو
\\
حادثات اغلب به شب واقع شود
&&
وان زمان معبود تو غایب بود
\\
سوی حق گر راستانه خم شوی
&&
وا رهی از اختران محرم شوی
\\
چون شوی محرم گشایم با تو لب
&&
تا ببینی آفتابی نیم‌شب
\\
جز روان پاک او را شرق نه
&&
در طلوعش روز و شب را فرق نه
\\
روز آن باشد که او شارق شود
&&
شب نماند شب چو او بارق شود
\\
چون نماید ذره پیش آفتاب
&&
هم‌چنانست آفتاب اندر لباب
\\
آفتابی را که رخشان می‌شود
&&
دیده پیشش کند و حیران می‌شود
\\
هم‌چو ذره بینیش در نور عرش
&&
پیش نور بی حد موفور عرش
\\
خوار و مسکین بینی او را بی‌قرار
&&
دیده را قوت شده از کردگار
\\
کیمیایی که ازو یک ماثری
&&
بر دخان افتاد گشت آن اختری
\\
نادر اکسیری که از وی نیم تاب
&&
بر ظلامی زد به گردش آفتاب
\\
بوالعجب میناگری کز یک عمل
&&
بست چندین خاصیت را بر زحل
\\
باقی اخترها و گوهرهای جان
&&
هم برین مقیاس ای طالب بدان
\\
دیدهٔ حسی زبون آفتاب
&&
دیدهٔ ربانیی جو و بیاب
\\
تا زبون گردد به پیش آن نظر
&&
شعشعات آفتاب با شرر
\\
که آن نظر نوری و این ناری بود
&&
نار پیش نور بس تاری بود
\\
\end{longtable}
\end{center}
