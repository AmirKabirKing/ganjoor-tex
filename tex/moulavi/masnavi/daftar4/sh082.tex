\begin{center}
\section*{بخش ۸۲ - علامت عاقل تمام و نیم‌عاقل و مرد تمام و نیم‌مرد و علامت شقی مغرور لاشی}
\label{sec:sh082}
\addcontentsline{toc}{section}{\nameref{sec:sh082}}
\begin{longtable}{l p{0.5cm} r}
عاقل آن باشد که او با مشعله‌ست
&&
او دلیل و پیشوای قافله‌ست
\\
پیرو نور خودست آن پیش‌رو
&&
تابع خویشست آن بی‌خویش‌رو
\\
مؤمن خویشست و ایمان آورید
&&
هم بدان نوری که جانش زو چرید
\\
دیگری که نیم‌عاقل آمد او
&&
عاقلی را دیدهٔ خود داند او
\\
دست در وی زد چو کور اندر دلیل
&&
تا بدو بینا شد و چست و جلیل
\\
وآن خری کز عقل جوسنگی نداشت
&&
خود نبودش عقل و عاقل را گذاشت
\\
ره نداند نه کثیر و نه قلیل
&&
ننگش آید آمدن خلف دلیل
\\
می‌رود اندر بیابان دراز
&&
گاه لنگان آیس و گاهی بتاز
\\
شمع نه تا پیشوای خود کند
&&
نیم شمعی نه که نوری کد کند
\\
نیست عقلش تا دم زنده زند
&&
نیم‌عقلی نه که خود مرده کند
\\
مردهٔ آن عاقل آید او تمام
&&
تا برآید از نشیب خود به بام
\\
عقل کامل نیست خود را مرده کن
&&
در پناه عاقلی زنده‌سخن
\\
زنده نی تا همدم عیسی بود
&&
مرده نی تا دمگه عیسی شود
\\
جان کورش گام هر سو می‌نهد
&&
عاقبت نجهد ولی بر می‌جهد
\\
\end{longtable}
\end{center}
