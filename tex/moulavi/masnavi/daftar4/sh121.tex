\begin{center}
\section*{بخش ۱۲۱ - مستجاب شدن دعای پادشاه در خلاص پسرش از جادوی کابلی}
\label{sec:sh121}
\addcontentsline{toc}{section}{\nameref{sec:sh121}}
\begin{longtable}{l p{0.5cm} r}
او شنیده بود از دور این خبر
&&
که اسیر پیرزن گشت آن پسر
\\
کان عجوزه بود اندر جادوی
&&
بی‌نظیر و آمن از مثل و دوی
\\
دست بر بالای دستست ای فتی
&&
در فن و در زور تا ذات خدا
\\
منتهای دستها دست خداست
&&
بحر بی‌شک منتهای سیلهاست
\\
هم ازو گیرند مایه ابرها
&&
هم بدو باشد نهایت سیل را
\\
گفت شاهش کین پسر از دست رفت
&&
گفت اینک آمدم درمان زفت
\\
نیست همتا زال را زین ساحران
&&
جز من داهی رسیده زان کران
\\
چون کف موسی به امر کردگار
&&
نک برآرم من ز سحر او دمار
\\
که مرا این علم آمد زان طرف
&&
نه ز شاگردی سحر مستخف
\\
آمدم تا بر گشایم سحر او
&&
تا نماند شاه‌زاده زردرو
\\
سوی گورستان برو وقت سحور
&&
پهلوی دیوار هست اسپید گور
\\
سوی قبله باز کاو آنجای را
&&
تا ببینی قدرت و صنع خدا
\\
بس درازست این حکایت تو ملول
&&
زبده را گویم رها کردم فضول
\\
آن گره‌های گران را بر گشاد
&&
پس ز محنت پور شه را راه داد
\\
آن پسر با خویش آمد شد دوان
&&
سوی تخت شاه با صد امتحان
\\
سجده کرد و بر زمین می‌زد ذقن
&&
در بغل کرده پسر تیغ و کفن
\\
شاه آیین بست و اهل شهر شاد
&&
وآن عروس ناامید بی‌مراد
\\
عالم از سر زنده گشت و پر فروز
&&
ای عجب آن روز روز امروز روز
\\
یک عروسی کرد شاه او را چنان
&&
که جلاب قند بد پیش سگان
\\
جادوی کمپیر از غصه بمرد
&&
روی و خوی زشت فا مالک سپرد
\\
شاه‌زاده در تعجب مانده بود
&&
کز من او عقل و نظر چون در ربود
\\
نو عروسی دید هم‌چون ماه حسن
&&
که همی زد بر ملیحان راه حسن
\\
گشت بیهوش و برو اندر فتاد
&&
تا سه روز از جسم وی گم شد فؤاد
\\
سه شبان روز او ز خود بیهوش گشت
&&
تا که خلق از غشی او پر جوش گشت
\\
از گلاب و از علاج آمد به خود
&&
اندک اندک فهم گشتش نیک و بد
\\
بعد سالی گفت شاهش در سخن
&&
کای پسر یاد آر از آن یار کهن
\\
یاد آور زان ضجیع و زان فراش
&&
تا بدین حد بی‌وفا و مر مباش
\\
گفت رو من یافتم دار السرور
&&
وا رهیدم از چه دار الغرور
\\
هم‌چنان باشد چومؤمنراه یافت
&&
سوی نور حق ز ظلمت روی تافت
\\
\end{longtable}
\end{center}
