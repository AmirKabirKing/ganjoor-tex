\begin{center}
\section*{بخش ۶۰ - حکایت آن فقیه با دستار بزرگ و  آنک بربود دستارش و بانگ می‌زد کی باز کن ببین کی چه می‌بری آنگه ببر}
\label{sec:sh060}
\addcontentsline{toc}{section}{\nameref{sec:sh060}}
\begin{longtable}{l p{0.5cm} r}
یک فقیهی ژنده‌ها در چیده بود
&&
در عمامهٔ خویش در پیچیده بود
\\
تا شود زفت و نماید آن عظیم
&&
چون در آید سوی محفل در حطیم
\\
ژنده‌ها از جامه‌ها پیراسته
&&
ظاهرا دستار از آن آراسته
\\
ظاهر دستار چون حلهٔ بهشت
&&
چون منافق اندرون رسوا و زشت
\\
پاره پاره دلق و پنبه و پوستین
&&
در درون آن عمامه بد دفین
\\
روی سوی مدرسه کرده صبوح
&&
تا بدین ناموس یابد او فتوح
\\
در ره تاریک مردی جامه کن
&&
منتظر استاده بود از بهر فن
\\
در ربود او از سرش دستار را
&&
پس دوان شد تا بسازد کار را
\\
پس فقیهش بانگ برزد کای پسر
&&
باز کن دستار را آنگه ببر
\\
این چنین که چار پره می‌پری
&&
باز کن آن هدیه را که می‌بری
\\
باز کن آن را به دست خود بمال
&&
آنگهان خواهی ببر کردم حلال
\\
چونک بازش کرد آنک می‌گریخت
&&
صد هزاران ژنده اندر ره بریخت
\\
زان عمامهٔ زفت نابایست او
&&
ماند یک گز کهنه‌ای در دست او
\\
بر زمین زد خرقه را کای بی‌عیار
&&
زین دغل ما را بر آوردی ز کار
\\
\end{longtable}
\end{center}
