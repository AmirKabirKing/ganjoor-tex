\begin{center}
\section*{بخش ۱۳ - عذر خواستن آن عاشق از گناه خویش به تلبیس و روی پوش و فهم کردن معشوق آن را نیز}
\label{sec:sh013}
\addcontentsline{toc}{section}{\nameref{sec:sh013}}
\begin{longtable}{l p{0.5cm} r}
گفت عاشق امتحان کردم مگیر
&&
تا ببینم تو حریفی یا ستیر
\\
من همی دانستمت بی‌امتحان
&&
لیک کی باشد خبر هم‌چون عیان
\\
آفتابی نام تو مشهور و فاش
&&
چه زیانست ار بکردم ابتلاش
\\
تو منی من خویشتن را امتحان
&&
می‌کنم هر روز در سود و زیان
\\
انبیا را امتحان کرده عدات
&&
تا شده ظاهر ازیشان معجزات
\\
امتحان چشم خود کردم به نور
&&
ای که چشم بد ز چشمان تو دور
\\
این جهان هم‌چون خرابست و تو گنج
&&
گر تفحص کردم از گنجت مرنج
\\
زان چنین بی‌خردگی کردم گزاف
&&
تا زنم با دشمنان هر بار لاف
\\
تا زبانم چون ترا نامی نهد
&&
چشم ازین دیده گواهیها دهد
\\
گر شدم در راه حرمت راه‌زن
&&
آمدم ای مه به شمشیر و کفن
\\
جز به دست خود مبرم پا و سر
&&
که ازین دستم نه از دست دگر
\\
از جدایی باز می‌رانی سخن
&&
هر چه خواهی کن ولیکن این مکن
\\
در سخن آباد این دم راه شد
&&
گفت امکان نیست چون بیگاه شد
\\
پوستها گفتیم و مغز آمد دفین
&&
گر بمانیم این نماند همچنین
\\
\end{longtable}
\end{center}
