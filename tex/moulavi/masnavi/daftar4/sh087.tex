\begin{center}
\section*{بخش ۸۷ - چاره اندیشیدن آن ماهی نیم‌عاقل و خود را مرده کردن}
\label{sec:sh087}
\addcontentsline{toc}{section}{\nameref{sec:sh087}}
\begin{longtable}{l p{0.5cm} r}
گفت ماهی دگر وقت بلا
&&
چونک ماند از سایهٔ عاقل جدا
\\
کو سوی دریا شد و از غم عتیق
&&
فوت شد از من چنان نیکو رفیق
\\
لیک زان نندیشم و بر خود زنم
&&
خویشتن را این زمان مرده کنم
\\
پس برآرم اشکم خود بر زبر
&&
پشت زیر و می‌روم بر آب بر
\\
می‌روم بر وی چنانک خس رود
&&
نی بسباحی چنانک کس رود
\\
مرده گردم خویش بسپارم به آب
&&
مرگ پیش از مرگ امنست از عذاب
\\
مرگ پیش از مرگ امنست ای فتی
&&
این چنین فرمود ما را مصطفی
\\
گفت موتواکلکم من قبل ان
&&
یاتی الموت تموتوا بالفتن
\\
هم‌چنان مرد و شکم بالا فکند
&&
آب می‌بردش نشیب و گه بلند
\\
هر یکی زان قاصدان بس غصه برد
&&
که دریغا ماهی بهتر بمرد
\\
شاد می‌شد او کز آن گفت دریغ
&&
پیش رفت این بازیم رستم ز تیغ
\\
پس گرفتش یک صیاد ارجمند
&&
پس برو تف کرد و بر خاکش فکند
\\
غلط غلطان رفت پنهان اندر آب
&&
ماند آن احمق همی‌کرد اضطراب
\\
از چپ و از راست می‌جست آن سلیم
&&
تا بجهد خویش برهاند گلیم
\\
دام افکندند و اندر دام ماند
&&
احمقی او را در آن آتش نشاند
\\
بر سر آتش به پشت تابه‌ای
&&
با حماقت گشت او همخوابه‌ایی
\\
او همی جوشید از تف سعیر
&&
عقل می‌گفتش الم یاتک نذیر
\\
او همی‌گفت از شکنجه وز بلا
&&
هم‌چو جان کافران قالوا بلی
\\
باز می‌گفت او که گر این بار من
&&
وا رهم زین محنت گردن‌شکن
\\
من نسازم جز به دریایی وطن
&&
آبگیری را نسازم من سکن
\\
آب بی‌حد جویم و آمن شوم
&&
تا ابد در امن و صحت می‌روم
\\
\end{longtable}
\end{center}
