\begin{center}
\section*{بخش ۱۴ - رد کردن معشوقه عذر عاشق را و تلبیس او را در روی او مالیدن}
\label{sec:sh014}
\addcontentsline{toc}{section}{\nameref{sec:sh014}}
\begin{longtable}{l p{0.5cm} r}
در جوابش بر گشاد آن یار لب
&&
کز سوی ما روز سوی تست شب
\\
حیله‌های تیره اندر داوری
&&
پیش بینایان چرا می‌آوری
\\
هر چه در دل داری از مکر و رموز
&&
پیش ما رسواست و پیدا هم‌چو روز
\\
گر بپوشیمش ز بنده‌پروری
&&
تو چرا بی‌رویی از حد می‌بری
\\
از پدر آموز که آدم در گناه
&&
خوش فرود آمد به سوی پایگاه
\\
چون بدید آن عالم الاسرار را
&&
بر دو پا استاد استغفار را
\\
بر سر خاکستر انده نشست
&&
از بهانه شاخ تا شاخی نجست
\\
ربنا انا ظلمنا گفت و بس
&&
چونک جانداران بدید از پیش و پس
\\
دید جانداران پنهان هم‌چو جان
&&
دورباش هر یکی تا آسمان
\\
که هلا پیش سلیمان مور باش
&&
تا بنشکافد ترا این دورباش
\\
جز مقام راستی یک دم مه‌ایست
&&
هیچ لالا مرد را چون چشم نیست
\\
کور اگر از پند پالوده شود
&&
هر دمی او باز آلوده شود
\\
آدما تو نیستی کور از نظر
&&
لیک اذا جاء القضا عمی البصر
\\
عمرها باید به نادر گاه‌گاه
&&
تا که بینا از قضا افتد به چاه
\\
کور را خود این قضا همراه اوست
&&
که مرورا اوفتادن طبع و خوست
\\
در حدث افتد نداند بوی چیست
&&
از منست این بوی یا ز آلودگیست
\\
ور کسی بر وی کند مشکی نثار
&&
هم ز خود داند نه از احسان یار
\\
پس دو چشم روشن ای صاحب‌نظر
&&
مر ترا صد مادرست و صد پدر
\\
خاصه چشم دل آن هفتاد توست
&&
وین دو چشم حس خوشه‌چین اوست
\\
ای دریغا ره‌زنان بنشسته‌اند
&&
صد گره زیر زبانم بسته‌اند
\\
پای‌بسته چون رود خوش راهوار
&&
بس گران بندیست این معذور دار
\\
این سخن اشکسته می‌آید دلا
&&
کین سخن درست غیرت آسیا
\\
در اگر چه خرد و اشکسته شود
&&
توتیای دیدهٔ خسته شود
\\
ای در از اشکست خود بر سر مزن
&&
کز شکستن روشنی خواهی شدن
\\
همچنین اشکسته بسته گفتنیست
&&
حق کند آخر درستش کو غنیست
\\
گندم ار بشکست و از هم در سکست
&&
بر دکان آمد که نک نان درست
\\
تو هم ای عاشق چو جرمت گشت فاش
&&
آب و روغن ترک کن اشکسته باش
\\
آنک فرزندان خاص آدم‌اند
&&
نفحهٔ انا ظلمنا می‌دمند
\\
حاجت خود عرضه کن حجت مگو
&&
هم‌چو ابلیس لعین سخت‌رو
\\
سخت‌رویی گر ورا شد عیب‌پوش
&&
در ستیز و سخت‌رویی رو بکوش
\\
آن ابوجهل از پیمبر معجزی
&&
خواست هم‌چون کینه‌ور ترکی غزی
\\
لیک آن صدیق حق معجز نخواست
&&
گفت این رو خود نگوید جز که راست
\\
کی رسد هم‌چون توی را کز منی
&&
امتحان هم‌چو من یاری کنی
\\
\end{longtable}
\end{center}
