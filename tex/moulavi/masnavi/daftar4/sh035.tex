\begin{center}
\section*{بخش ۳۵ - بقیهٔ قصهٔ اهل سبا و نصیحت و ارشاد سلیمان علیه‌السلام آل بلقیس را هر یکی را اندر خور خود و مشکلات دین و دل او و صید کردن هر جنس مرغ ضمیری به صفیر آن جنس مرغ و طعمهٔ او}
\label{sec:sh035}
\addcontentsline{toc}{section}{\nameref{sec:sh035}}
\begin{longtable}{l p{0.5cm} r}
قصه گویم از سبا مشتاق‌وار
&&
چون صبا آمد به سوی لاله‌زار
\\
لاقت الاشباح یوم وصلها
&&
عادت الاولاد صوب اصلها
\\
امة العشق الخفی فی الامم
&&
مثل جود حوله لوم السقم
\\
ذلة الارواح من اشباحها
&&
عزة الاشباح من ارواحها
\\
ایها العشاق السقیا لکم
&&
انتم الباقون و البقیالکم
\\
ایها السالون قوموا واعشقوا
&&
ذاک ریح یوسف فاستنشقوا
\\
منطق‌الطیر سلیمانی بیا
&&
بانگ هر مرغی که آید می‌سرا
\\
چون به مرغانت فرستادست حق
&&
لحن هر مرغی بدادستت سبق
\\
مرغ جبری را زبان جبر گو
&&
مرغ پر اشکسته را از صبر گو
\\
مرغ صابر را تو خوش دار و معاف
&&
مرغ عنقا را بخوان اوصاف قاف
\\
مر کبوتر را حذر فرما ز باز
&&
باز را از حلم گو و احتراز
\\
وان خفاشی را که ماند او بی‌نوا
&&
می‌کنش با نور جفت و آشنا
\\
کبک جنگی را بیاموزان تو صلح
&&
مر خروسان را نما اشراط صبح
\\
هم‌چنان می‌رو ز هدهد تا عقاب
&&
ره نما والله اعلم بالصواب
\\
\end{longtable}
\end{center}
