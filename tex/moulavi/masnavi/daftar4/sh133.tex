\begin{center}
\section*{بخش ۱۳۳ - حکایت آن زن پلیدکار کی شوهر را گفت کی آن خیالات از سر امرودبن می‌نماید ترا کی چنینها نماید چشم آدمی را سر آن امرودبن از سر امرودبن فرود آی تا آن خیالها برود و اگر کسی گوید کی آنچ آن مرد  می‌دید خیال نبود و جواب این مثالیست نه مثل در مثال همین  قدر بس بود کی اگر بر سر امرودبن نرفتی هرگز آنها ندیدی خواه خیال خواه حقیقت}
\label{sec:sh133}
\addcontentsline{toc}{section}{\nameref{sec:sh133}}
\begin{longtable}{l p{0.5cm} r}
آن زنی می‌خواست تا با مول خود
&&
بر زند در پیش شوی گول خود
\\
پس به شوهر گفت زن کای نیکبخت
&&
من برآیم میوه چیدن بر درخت
\\
چون برآمد بر درخت آن زن گریست
&&
چون ز بالا سوی شوهر بنگریست
\\
گفت شوهر را کای مابون رد
&&
کیست آن لوطی که بر تو می‌فتد
\\
تو به زیر او چو زن بغنوده‌ای
&&
ای فلان تو خود مخنث بوده‌ای
\\
گفت شوهر نه سرت گویی بگشت
&&
ورنه اینجا نیست غیر من به دشت
\\
زن مکرر کرد که آن با برطله
&&
کیست بر پشتت فرو خفته هله
\\
گفت ای زن هین فرود آ از درخت
&&
که سرت گشت و خرف گشتی تو سخت
\\
چون فرود آمد بر آمد شوهرش
&&
زن کشید آن مول را اندر برش
\\
گفت شوهر کیست آن ای روسپی
&&
که به بالای تو آمد چون کپی
\\
گفت زن نه نیست اینجا غیر من
&&
هین سرت برگشته شد هرزه متن
\\
او مکرر کرد بر زن آن سخن
&&
گفت زن این هست از امرودبن
\\
از سر امرودبن من هم‌چنان
&&
کژ همی دیدم که تو ای قلتبان
\\
هین فرود آ تا ببینی هیچ نیست
&&
این همه تخییل از امروبنیست
\\
هزل تعلیمست آن را جد شنو
&&
تو مشو بر ظاهر هزلش گرو
\\
هر جدی هزلست پیش هازلان
&&
هزلها جدست پیش عاقلان
\\
کاهلان امرودبن جویند لیک
&&
تا بدان امرودبن راهیست نیک
\\
نقل کن ز امرودبن که اکنون برو
&&
گشته‌ای تو خیره‌چشم و خیره‌رو
\\
این منی و هستی اول بود
&&
که برو دیده کژ و احول بود
\\
چون فرود آیی ازین امرودبن
&&
کژ نماند فکرت و چشم و سخن
\\
یک درخت بخت بینی گشته این
&&
شاخ او بر آسمان هفتمین
\\
چون فرود آیی ازو گردی جدا
&&
مبدلش گرداند از رحمت خدا
\\
زین تواضع که فرود آیی خدا
&&
راست بینی بخشد آن چشم ترا
\\
راست بینی گر بدی آسان و زب
&&
مصطفی کی خواستی آن را ز رب
\\
گفت بنما جزو جزو از فوق و پست
&&
آنچنان که پیش تو آن جزو هست
\\
بعد از آن بر رو بر آن امرودبن
&&
که مبدل گشت و سبز از امر کن
\\
چون درخت موسوی شد این درخت
&&
چون سوی موسی کشانیدی تو رخت
\\
آتش او را سبز و خرم می‌کند
&&
شاخ او انی انا الله می‌زند
\\
زیر ظلش جمله حاجاتت روا
&&
این چنین باشد الهی کیمیا
\\
آن منی و هستیت باشد حلال
&&
که درو بینی صفات ذوالجلال
\\
شد درخت کژ مقوم حق‌نما
&&
اصله ثابت و فرعه فی‌السما
\\
\end{longtable}
\end{center}
