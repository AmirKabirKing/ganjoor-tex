\begin{center}
\section*{بخش ۵ - قصد خیانت کردن عاشق و بانگ  بر زدن معشوق بر وی}
\label{sec:sh005}
\addcontentsline{toc}{section}{\nameref{sec:sh005}}
\begin{longtable}{l p{0.5cm} r}
چونک تنهااش بدید آن ساده مرد
&&
زود او قصد کنار و بوسه کرد
\\
بانگ بر وی زد به هیبت آن نگار
&&
که مرو گستاخ ادب را هوش دار
\\
گفت آخر خلوتست و خلق نی
&&
آب حاضر تشنهٔ هم‌چون منی
\\
کس نمی‌جنبد درینجا جز که باد
&&
کیست حاضر کیست مانع زین گشاد
\\
گفت ای شیدا تو ابله بوده‌ای
&&
ابلهی وز عاقلان نشنوده‌ای
\\
باد را دیدی که می‌جنبد بدان
&&
بادجنبانیست اینجا بادران
\\
جزو بادی که به حکم ما درست
&&
بادبیزن تا نجنبانی نجست
\\
جنبش این جزو باد ای ساده مرد
&&
بی‌تو و بی‌بادبیزن سر نکرد
\\
جنبش باد نفس کاندر لبست
&&
تابع تصریف جان و قالبست
\\
گاه دم را مدح و پیغامی کنی
&&
گاه دم را هجو و دشنامی کنی
\\
پس بدان احوال دیگر بادها
&&
که ز جز وی کل می‌بیند نهی
\\
باد را حق گه بهاری می‌کند
&&
در دیش زین لطف عاری می‌کند
\\
بر گروه عاد صرصر می‌کند
&&
باز بر هودش معطر می‌کند
\\
می‌کند یک باد را زهر سموم
&&
مر صبا را می‌کند خرم‌قدوم
\\
باد دم را بر تو بنهاد او اساس
&&
تا کنی هر باد را بر وی قیاس
\\
دم نمی‌گردد سخن بی‌لطف و قهر
&&
بر گروهی شهد و بر قومیست زهر
\\
مروحه جنبان پی انعام کس
&&
وز برای قهر هر پشه و مگس
\\
مروحهٔ تقدیر ربانی چرا
&&
پر نباشد ز امتحان و ابتلا
\\
چونک جزو باد دم یا مروحه
&&
نیست الا مفسده یا مصلحه
\\
این شمال و این صبا و این دبور
&&
کی بود از لطف و از انعام دور
\\
یک کف گندم ز انباری ببین
&&
فهم کن کان جمله باشد همچنین
\\
کل باد از برج باد آسمان
&&
کی جهد بی مروحهٔ آن بادران
\\
بر سر خرمن به وقت انتقاد
&&
نه که فلاحان ز حق جویند باد
\\
تا جدا گردد ز گندم کاهها
&&
تا به انباری رود یا چاهها
\\
چون بماند دیر آن باد وزان
&&
جمله را بینی به حق لابه‌کنان
\\
همچنین در طلق آن باد ولاد
&&
گر نیاید بانگ درد آید که داد
\\
گر نمی‌دانند کش راننده اوست
&&
باد را پس کردن زاری چه خوست
\\
اهل کشتی همچنین جویای باد
&&
جمله خواهانش از آن رب العباد
\\
همچنین در درد دندانها ز باد
&&
دفع می‌خواهی بسوز و اعتقاد
\\
از خدا لابه‌کنان آن جندیان
&&
که بده باد ظفر ای کامران
\\
رقعهٔ تعویذ می‌خواهند نیز
&&
در شکنجهٔ طلق زن از هر عزیز
\\
پس همه دانسته‌اند آن را یقین
&&
که فرستد باد رب‌العالمین
\\
پس یقین در عقل هر داننده هست
&&
اینک با جنبنده جنباننده هست
\\
گر تو او را می‌نبینی در نظر
&&
فهم کن آن را به اظهار اثر
\\
تن به جان جنبد نمی‌بینی تو جان
&&
لیک از جنبیدن تن جان بدان
\\
گفت او گر ابلهم من در ادب
&&
زیرکم اندر وفا و در طلب
\\
گفت ادب این بود خود که دیده شد
&&
آن دگر را خود همی‌دانی تو لد
\\
\end{longtable}
\end{center}
