\begin{center}
\section*{بخش ۴۷ - تشبیه کردن قرآن مجید را به عصای موسی و وفات مصطفی را علیه السلام نمودن بخواب موسی و قاصدان تغییر قرآن را با آن دو ساحر بچه کی قصد بردن عصا کردند چو موسی  را خفته یافتند}
\label{sec:sh047}
\addcontentsline{toc}{section}{\nameref{sec:sh047}}
\begin{longtable}{l p{0.5cm} r}
مصطفی را وعده کرد الطاف حق
&&
گر بمیری تو نمیرد این سبق
\\
من کتاب و معجزه‌ت را رافعم
&&
بیش و کم‌کن را ز قرآن مانعم
\\
من ترا اندر دو عالم حافظم
&&
طاعنان را از حدیثت رافضم
\\
کس نتاند بیش و کم کردن درو
&&
تو به از من حافظی دیگر مجو
\\
رونقت را روز روز افزون کنم
&&
نام تو بر زر و بر نقره زنم
\\
منبر و محراب سازم بهر تو
&&
در محبت قهر من شد قهر تو
\\
نام تو از ترس پنهان می‌گوند
&&
چون نماز آرند پنهان می‌شوند
\\
از هراس وترس کفار لعین
&&
دینت پنهان می‌شود زیر زمین
\\
من مناره پر کنم آفاق را
&&
کور گردانم دو چشم عاق را
\\
چاکرانت شهرها گیرند و جاه
&&
دین تو گیرد ز ماهی تا به ماه
\\
تا قیامت باقیش داریم ما
&&
تو مترس از نسخ دین ای مصطفی
\\
ای رسول ما تو جادو نیستی
&&
صادقی هم‌خرقهٔ موسیستی
\\
هست قرآن مر تو را همچون عصا
&&
کفرها را در کشد چون اژدها
\\
تو اگر در زیر خاکی خفته‌ای
&&
چون عصایش دان تو آنچ گفته‌ای
\\
قاصدان را بر عصایش دست نی
&&
تو بخسپ ای شه مبارک خفتنی
\\
تن بخفته نور تو بر آسمان
&&
بهر پیکار تو زه کرده کمان
\\
فلسفی و آنچ پوزش می‌کند
&&
قوس نورت تیردوزش می‌کند
\\
آنچنان کرد و از آن افزون که گفت
&&
او بخفت و بخت و اقبالش نخفت
\\
جان بابا چونک ساحر خواب شد
&&
کار او بی رونق و بی‌تاب شد
\\
هر دو بوسیدند گورش را و تفت
&&
تا بمصر از بهر آن پیگار زفت
\\
چون به مصر از بهر آن کار آمدند
&&
طالب موسی و خانهٔ او شدند
\\
اتفاق افتاد کان روز ورود
&&
موسی اندر زیر نخلی خفته بود
\\
پس نشان دادندشان مردم بدو
&&
که برو آن سوی نخلستان بجو
\\
چون بیامد دید در خرمابنان
&&
خفته‌ای که بود بیدار جهان
\\
بهر نازش بسته او دو چشم سر
&&
عرش و فرشش جمله در زیر نظر
\\
ای بسا بیدارچشم و خفته‌دل
&&
خود چه بیند دید اهل آب و گل
\\
آنک دل بیدار دارد چشم سر
&&
گر بخسپد بر گشاید صد بصر
\\
گر تو اهل دل نه‌ای بیدار باش
&&
طالب دل باش و در پیکار باش
\\
ور دلت بیدار شد می‌خسپ خوش
&&
نیست غایب ناظرت از هفت و شش
\\
گفت پیغامبر که خسپد چشم من
&&
لیک کی خسپد دلم اندر وسن
\\
شاه بیدارست حارس خفته گیر
&&
جان فدای خفتگان دل‌بصیر
\\
وصف بیداری دل ای معنوی
&&
در نگنجد در هزاران مثنوی
\\
چون بدیدندش که خفتست او دراز
&&
بهر دزدی عصا کردند ساز
\\
ساحران قصد عصا کردند زود
&&
کز پسش باید شدن وانگه ربود
\\
اندکی چون پیشتر کردند ساز
&&
اندر آمد آن عصا در اهتزاز
\\
آنچنان بر خود بلرزید آن عصا
&&
کان دو بر جا خشک گشتند از وجا
\\
بعد از آن شد اژدها و حمله کرد
&&
هر دوان بگریختند و روی‌زرد
\\
رو در افتادن گرفتند از نهیب
&&
غلط غلطان منهزم در هر نشیب
\\
پس یقینشان شد که هست از آسمان
&&
زانک می‌دیدند حد ساحران
\\
بعد از آن اطلاق و تبشان شد پدید
&&
کارشان تا نزع و جان کندن رسید
\\
پس فرستادند مردی در زمان
&&
سوی موسی از برای عذر آن
\\
کامتحان کردیم و ما را کی رسد
&&
امتحان تو اگر نبود حسد
\\
مجرم شاهیم ما را عفو خواه
&&
ای تو خاص الخاص درگاه اله
\\
عفو کرد و در زمان نیکو شدند
&&
پیش موسی بر زمین سر می‌زدند
\\
گفت موسی عفو کردم ای کرام
&&
گشت بر دوزخ تن و جانتان حرام
\\
من شما را خود ندیدم ای دو یار
&&
اعجمی سازید خود را ز اعتذار
\\
همچنان بیگانه‌شکل و آشنا
&&
در نبرد آیید بهر پادشا
\\
پس زمین را بوسه دادند و شدند
&&
انتظار وقت و فرصت می‌بدند
\\
\end{longtable}
\end{center}
