\begin{center}
\section*{بخش ۱۳ - دعوت باز بطان را از آب به صحرا}
\label{sec:sh013}
\addcontentsline{toc}{section}{\nameref{sec:sh013}}
\begin{longtable}{l p{0.5cm} r}
باز گوید بط را کز آب خیز
&&
تا ببینی دشتها را قندریز
\\
بط عاقل گویدش ای باز دور
&&
آب ما را حصن و امنست و سرور
\\
دیو چون باز آمد ای بطان شتاب
&&
هین به بیرون کم روید از حصن آب
\\
باز را گویند رو رو باز گرد
&&
از سر ما دست دار ای پای‌مرد
\\
ما بری از دعوتت دعوت ترا
&&
ما ننوشیم این دم تو کافرا
\\
حصن ما را قند و قندستان ترا
&&
من نخواهم هدیه‌ات بستان ترا
\\
چونک جان باشد نیاید لوت کم
&&
چونک لشکر هست کم ناید علم
\\
خواجهٔ حازم بسی عذر آورید
&&
بس بهانه کرد با دیو مرید
\\
گفت این دم کارها دارم مهم
&&
گر بیایم آن نگردد منتظم
\\
شاه کار نازکم فرموده است
&&
ز انتظارم شاه شب نغنوده است
\\
من نیارم ترک امر شاه کرد
&&
من نتانم شد بر شه روی‌زرد
\\
هر صباح و هر مسا سرهنگ خاص
&&
می‌رسد از من همی‌جوید مناص
\\
تو روا داری که آیم سوی ده
&&
تا در ابرو افکند سلطان گره
\\
بعد از آن درمان خشمش چون کنم
&&
زنده خود را زین مگر مدفون کنم
\\
زین نمط او صد بهانه باز گفت
&&
حیله‌ها با حکم حق نفتاد جفت
\\
گر شود ذرات عالم حیله‌پیچ
&&
با قضای آسمان هیچند هیچ
\\
چون گریزد این زمین از آسمان
&&
چون کند او خویش را از وی نهان
\\
هرچه آید ز آسمان سوی زمین
&&
نه مفر دارد نه چاره نه کمین
\\
آتش ار خورشید می‌بارد برو
&&
او بپیش آتشش بنهاده رو
\\
ور همی طوفان کند باران برو
&&
شهرها را می‌کند ویران برو
\\
او شده تسلیم او ایوب‌وار
&&
که اسیرم هرچه می‌خواهی ببار
\\
ای که جزو این زمینی سر مکش
&&
چونک بینی حکم یزدان در مکش
\\
چون خلقناکم شنودی من تراب
&&
خاک باشی جست از تو رو متاب
\\
بین که اندر خاک تخمی کاشتم
&&
کرد خاکی و منش افراشتم
\\
حملهٔ دیگر تو خاکی پیشه گیر
&&
تا کنم بر جمله میرانت امیر
\\
آب از بالا به پستی در رود
&&
آنگه از پستی به بالا بر رود
\\
گندم از بالا بزیر خاک شد
&&
بعد از آن او خوشه و چالاک شد
\\
دانهٔ هر میوه آمد در زمین
&&
بعد از آن سرها بر آورد از دفین
\\
اصل نعمتها ز گردون تا بخاک
&&
زیر آمد شد غذای جان پاک
\\
از تواضع چون ز گردون شد بزیر
&&
گشت جزو آدمی حی دلیر
\\
پس صفات آدمی شد آن جماد
&&
بر فراز عرش پران گشت شاد
\\
کز جهان زنده ز اول آمدیم
&&
باز از پستی سوی بالا شدیم
\\
جمله اجزا در تحرک در سکون
&&
ناطقان که انا الیه راجعون
\\
ذکر و تسبیحات اجزای نهان
&&
غلغلی افکند اندر آسمان
\\
چون قضا آهنگ نارنجات کرد
&&
روستایی شهریی را مات کرد
\\
با هزاران حزم خواجه مات شد
&&
زان سفر در معرض آفات شد
\\
اعتمادش بر ثبات خویش بود
&&
گرچه که بد نیم سیلش در ربود
\\
چون قضا بیرون کند از چرخ سر
&&
عاقلان گردند جمله کور و کر
\\
ماهیان افتند از دریا برون
&&
دام گیرد مرغ پران را زبون
\\
تا پری و دیو در شیشه شود
&&
بلک هاروتی به بابل در رود
\\
جز کسی کاندر قضا اندر گریخت
&&
خون او را هیچ تربیعی نریخت
\\
غیر آن که در گریزی در قضا
&&
هیچ حیله ندهدت از وی رها
\\
\end{longtable}
\end{center}
