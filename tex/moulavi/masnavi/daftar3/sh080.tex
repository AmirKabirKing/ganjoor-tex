\begin{center}
\section*{بخش ۸۰ - قصهٔ خواندن شیخ ضریر مصحف را در رو و بینا شدن وقت قرائت}
\label{sec:sh080}
\addcontentsline{toc}{section}{\nameref{sec:sh080}}
\begin{longtable}{l p{0.5cm} r}
دید در ایام آن شیخ فقیر
&&
مصحفی در خانهٔ پیری ضریر
\\
پیش او مهمان شد او وقت تموز
&&
هر دو زاهد جمع گشته چند روز
\\
گفت اینجا ای عجب مصحف چراست
&&
چونک نابیناست این درویش راست
\\
اندرین اندیشه تشویشش فزود
&&
که جز او را نیست اینجا باش و بود
\\
اوست تنها مصحفی آویخته
&&
من نیم گستاخ یا آمیخته
\\
تا بپرسم نه خمش صبری کنم
&&
تا به صبری بر مرادی بر زنم
\\
صبر کرد و بود چندی در حرج
&&
کشف شد کالصبر مفتاح الفرج
\\
\end{longtable}
\end{center}
