\begin{center}
\section*{بخش ۱۸۶ - در آمدن آن عاشق لاابالی در بخارا وتحذیر کردن دوستان او را از پیداشدن}
\label{sec:sh186}
\addcontentsline{toc}{section}{\nameref{sec:sh186}}
\begin{longtable}{l p{0.5cm} r}
اندر آمد در بخارا شادمان
&&
پیش معشوق خود و دارالامان
\\
همچو آن مستی که پرد بر اثیر
&&
مه کنارش گیرد و گوید که گیر
\\
هرکه دیدش در بخارا گفت خیز
&&
پیش از پیدا شدن منشین گریز
\\
که ترا می‌جوید آن شه خشمگین
&&
تا کشد از جان تو ده ساله کین
\\
الله الله درمیا در خون خویش
&&
تکیه کم کن بر دم و افسون خویش
\\
شحنهٔ صدر جهان بودی و راد
&&
معتمد بودی مهندس اوستاد
\\
غدو کردی وز جزا بگریختی
&&
رسته بودی باز چون آویختی
\\
از بلا بگریختی با صد حیل
&&
ابلهی آوردت اینجا یا اجل
\\
ای که عقلت بر عطارد دق کند
&&
عقل و عاقل را قضا احمق کند
\\
نحس خرگوشی که باشد شیرجو
&&
زیرکی و عقل و چالاکیت کو
\\
هست صد چندین فسونهای قضا
&&
گفت اذا جاء القضا ضاق الفضا
\\
صد ره و مخلص بود از چپ و راست
&&
از قضا بسته شود کو اژدهاست
\\
\end{longtable}
\end{center}
