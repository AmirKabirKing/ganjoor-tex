\begin{center}
\section*{بخش ۲۰۸ - مثل زدن در رمیدن کرهٔ اسپ از آب خوردن به سبب شخولیدن سایسان}
\label{sec:sh208}
\addcontentsline{toc}{section}{\nameref{sec:sh208}}
\begin{longtable}{l p{0.5cm} r}
آنک فرمودست او اندر خطاب
&&
کره و مادر همی‌خوردند آب
\\
می‌شخولیدند هر دم آن نفر
&&
بهر اسپان که هلا هین آب خور
\\
آن شخولیدن به کره می‌رسید
&&
سر همی بر داشت و از خور می‌رمید
\\
مادرش پرسید کای کره چرا
&&
می‌رمی هر ساعتی زین استقا
\\
گفت کره می‌شخولند این گروه
&&
ز اتفاق بانگشان دارم شکوه
\\
پس دلم می‌لرزد از جا می‌رود
&&
ز اتفاق نعره خوفم می‌رسد
\\
گفت مادر تا جهان بودست ازین
&&
کارافزایان بدند اندر زمین
\\
هین تو کار خویش کن ای ارجمند
&&
زود کایشان ریش خود بر می‌کنند
\\
وقت تنگ و می‌رود آب فراخ
&&
پیش از آن کز هجر گردی شاخ شاخ
\\
شهره کاریزیست پر آب حیات
&&
آب کش تا بر دمد از تو نبات
\\
آب خضر از جوی نطق اولیا
&&
می‌خوریم ای تشنهٔ غافل بیا
\\
گر نبینی آب کورانه بفن
&&
سوی جو آور سبو در جوی زن
\\
چون شنیدی کاندرین جو آب هست
&&
کور را تقلید باید کار بست
\\
جو فرو بر مشک آب‌اندیش را
&&
تا گران بینی تو مشک خویش را
\\
چون گران دیدی شوی تو مستدل
&&
رست از تقلید خشک آنگاه دل
\\
گر نبیند کور آب جو عیان
&&
لیک داند چون سبو بیند گران
\\
که ز جو اندر سبو آبی برفت
&&
کین سبک بود و گران شد ز آب و زفت
\\
زانک هر بادی مرا در می‌ربود
&&
باد می‌نربایدم ثقلم فزود
\\
مر سفیهان را رباید هر هوا
&&
زانک نبودشان گرانی قوی
\\
کشتی بی‌لنگر آمد مرد شر
&&
که ز باد کژ نیابد او حذر
\\
لنگر عقلست عاقل را امان
&&
لنگری در یوزه کن از عاقلان
\\
او مددهای خرد چون در ربود
&&
از خزینه در آن دریای جود
\\
زین چنین امداد دل پر فن شود
&&
بجهد از دل چشم هم روشن شود
\\
زانک نور از دل برین دیده نشست
&&
تا چو دل شد دیدهٔ تو عاطلست
\\
دل چو بر انوار عقلی نیز زد
&&
زان نصیبی هم بدو دیده دهد
\\
پس بدان کاب مبارک ز آسمان
&&
وحی دلها باشد و صدق بیان
\\
ما چو آن کره هم آب جو خوریم
&&
سوی آن وسواس طاعن ننگریم
\\
پی‌رو پیغمبرانی ره سپر
&&
طعنهٔ خلقان همه بادی شمر
\\
آن خداوندان که ره طی کرده‌اند
&&
گوش فا بانگ سگان کی کرده‌اند
\\
\end{longtable}
\end{center}
