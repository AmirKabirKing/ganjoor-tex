\begin{center}
\section*{بخش ۱۸۲ - پرسیدن معشوقی از عاشق غریب خود کی از شهرها کدام شهر را خوشتر یافتی و انبوه‌تر و محتشم‌تر و پر نعمت‌تر و دلگشاتر}
\label{sec:sh182}
\addcontentsline{toc}{section}{\nameref{sec:sh182}}
\begin{longtable}{l p{0.5cm} r}
گفت معشوقی به عاشق کای فتی
&&
تو به غربت دیده‌ای بس شهرها
\\
پس کدامین شهر ز آنها خوشترست
&&
گفت آن شهری که در وی دلبرست
\\
هرکجا باشد شه ما را بساط
&&
هست صحرا گر بود سم الخیاط
\\
هر کجا که یوسفی باشد چو ماه
&&
جنتست ارچه که باشد قعر چاه
\\
\end{longtable}
\end{center}
