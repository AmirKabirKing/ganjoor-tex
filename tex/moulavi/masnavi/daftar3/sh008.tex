\begin{center}
\section*{بخش ۸ - فریفتن روستایی شهری را و بدعوت  خواندن بلابه و الحاح بسیار}
\label{sec:sh008}
\addcontentsline{toc}{section}{\nameref{sec:sh008}}
\begin{longtable}{l p{0.5cm} r}
ای برادر بود اندر ما مضی
&&
شهریی با روستایی آشنا
\\
روستایی چون سوی شهر آمدی
&&
خرگه اندر کوی آن شهری زدی
\\
دو مه و سه ماه مهمانش بدی
&&
بر دکان او و بر خوانش بدی
\\
هر حوایج را که بودش آن زمان
&&
راست کردی مرد شهری رایگان
\\
رو به شهری کرد و گفت ای خواجه تو
&&
هیچ می‌نایی سوی ده فرجه‌جو
\\
الله الله جمله فرزندان بیار
&&
کین زمان گلشنست و نوبهار
\\
یا بتابستان بیا وقت ثمر
&&
تا ببندم خدمتت را من کمر
\\
خیل و فرزندان و قومت را بیار
&&
در ده ما باش سه ماه و چهار
\\
که بهاران خطهٔ ده خوش بود
&&
کشت‌زار و لالهٔ دلکش بود
\\
وعده دادی شهری او را دفع حال
&&
تا بر آمد بعد وعده هشت سال
\\
او بهر سالی همی‌گفتی که کی
&&
عزم خواهی کرد کامد ماه دی
\\
او بهانه ساختی کامسال‌مان
&&
از فلان خطه بیامد میهمان
\\
سال دیگر گر توانم وا رهید
&&
از مهمات آن طرف خواهم دوید
\\
گفت هستند آن عیالم منتظر
&&
بهر فرزندان تو ای اهل بر
\\
باز هر سالی چو لکلک آمدی
&&
تا مقیم قبهٔ شهری شدی
\\
خواجه هر سالی ز زر و مال خویش
&&
خرج او کردی گشادی بال خویش
\\
آخرین کرت سه ماه آن پهلوان
&&
خوان نهادش بامدادان و شبان
\\
از خجالت باز گفت او خواجه را
&&
چند وعده چند بفریبی مرا
\\
گفت خواجه جسم و جانم وصل‌جوست
&&
لیک هر تحویل اندر حکم هوست
\\
آدمی چون کشتی است و بادبان
&&
تا کی آرد باد را آن بادران
\\
باز سوگندان بدادش کای کریم
&&
گیر فرزندان بیا بنگر نعیم
\\
دست او بگرفت سه کرت بعهد
&&
کالله الله زو بیا بنمای جهد
\\
بعد ده سال و بهر سالی چنین
&&
لابه‌ها و وعده‌های شکرین
\\
کودکان خواجه گفتند ای پدر
&&
ماه و ابر و سایه هم دارد سفر
\\
حقها بر وی تو ثابت کرده‌ای
&&
رنجها در کار او بس برده‌ای
\\
او همی‌خواهد که بعضی حق آن
&&
وا گزارد چون شوی تو میهمان
\\
بس وصیت کرد ما را او نهان
&&
که کشیدش سوی ده لابه‌کنان
\\
گفت حقست این ولی ای سیبویه
&&
اتق من شر من احسنت الیه
\\
دوستی تخم دم آخر بود
&&
ترسم از وحشت که آن فاسد شود
\\
صحبتی باشد چو شمشیر قطوع
&&
همچو دی در بوستان و در زروع
\\
صحبتی باشد چو فصل نوبهار
&&
زو عمارتها و دخل بی‌شمار
\\
حزم آن باشد که ظن بد بری
&&
تا گریزی و شوی از بد بری
\\
حزم سؤ الظن گفتست آن رسول
&&
هر قدم را دام می‌دان ای فضول
\\
روی صحرا هست هموار و فراخ
&&
هر قدم دامیست کم ران اوستاخ
\\
آن بز کوهی دود که دام کو
&&
چون بتازد دامش افتد در گلو
\\
آنک می‌گفتی که کو اینک ببین
&&
دشت می‌دیدی نمی‌دیدی کمین
\\
بی کمین و دام و صیاد ای عیار
&&
دنبه کی باشد میان کشت‌زار
\\
آنک گستاخ آمدند اندر زمین
&&
استخوان و کله‌هاشان را ببین
\\
چون به گورستان روی ای مرتضا
&&
استخوانشان را بپرس از ما مضی
\\
تا بظاهر بینی آن مستان کور
&&
چون فرو رفتند در چاه غرور
\\
چشم اگر داری تو کورانه میا
&&
ور نداری چشم دست آور عصا
\\
آن عصای حزم و استدلال را
&&
چون نداری دید می‌کن پیشوا
\\
ور عصای حزم و استدلال نیست
&&
بی عصاکش بر سر هر ره مه‌ایست
\\
گام زان سان نه که نابینا نهد
&&
تا که پا از چاه و از سگ وا رهد
\\
لرز لرزان و بترس و احتیاط
&&
می‌نهد پا تا نیفتد در خباط
\\
ای ز دودی جسته در ناری شده
&&
لقمه جسته لقمهٔ ماری شده
\\
\end{longtable}
\end{center}
