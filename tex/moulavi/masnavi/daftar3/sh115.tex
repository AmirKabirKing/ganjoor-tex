\begin{center}
\section*{بخش ۱۱۵ - قصاص فرمودن داود علیه السلام خونی را بعد از الزام حجت برو}
\label{sec:sh115}
\addcontentsline{toc}{section}{\nameref{sec:sh115}}
\begin{longtable}{l p{0.5cm} r}
هم بدان تیغش بفرمود او قصاص
&&
کی کند مکرش ز علم حق خلاص
\\
حلم حق گرچه مواساها کند
&&
لیک چون از حد بشد پیدا کند
\\
خون نخسپد درفتد در هر دلی
&&
میل جست و جوی و کشف مشکلی
\\
اقتضای داوری رب دین
&&
سر بر آرد از ضمیر آن و این
\\
کان فلان چون شد چه شد حالش چه گشت
&&
همچنانک جوشد از گلزار کشت
\\
جوشش خون باشد آن وا جستها
&&
خارش دلها و بحث و ماجرا
\\
چونک پیداگشت سر کار او
&&
معجزه داود شد فاش و دوتو
\\
خلق جمله سر برهنه آمدند
&&
سر به سجده بر زمینها می‌زدند
\\
ما همه کوران اصلی بوده‌ایم
&&
از تو ما صد گون عجایب دیده‌ایم
\\
سنگ با تو در سخن آمد شهیر
&&
کز برای غزو طالوتم بگیر
\\
تو به سه سنگ و فلاخن آمدی
&&
صد هزاران مرد را بر هم زدی
\\
سنگهایت صدهزاران پاره شد
&&
هر یکی هر خصم را خون‌خواره شد
\\
آهن اندر دست تو چون موم شد
&&
چون زره‌سازی ترا معلوم شد
\\
کوهها با تو رسایل شد شکور
&&
با تو می‌خوانند چون مقری زبور
\\
صد هزاران چشم دل بگشاده شد
&&
از دم تو غیب را آماده شد
\\
و آن قوی‌تر زان همه کین دایمست
&&
زندگی بخشی که سرمد قایمست
\\
جان جملهٔ معجزات اینست خود
&&
کو ببخشد مرده را جان ابد
\\
کشته شد ظالم جهانی زنده شد
&&
هر یکی از نو خدا را بنده شد
\\
\end{longtable}
\end{center}
