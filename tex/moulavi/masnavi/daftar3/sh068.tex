\begin{center}
\section*{بخش ۶۸ - حکایت آن درویش کی در کوه خلوت کرده بود و بیان حلاوت انقطاع و خلوت و داخل شدن درین منقبت کی انا جلیس من ذکرنی و انیس من استانس بی گر با همه‌ای چو بی منی بی همه‌ای ور بی همه‌ای چو با منی با همه‌ای}
\label{sec:sh068}
\addcontentsline{toc}{section}{\nameref{sec:sh068}}
\begin{longtable}{l p{0.5cm} r}
بود درویشی بکهساری مقیم
&&
خلوت او را بود هم خواب و ندیم
\\
چون ز خالق می‌رسید او را شمول
&&
بود از انفاس مرد و زن ملول
\\
همچنانک سهل شد ما را حضر
&&
سهل شد هم قوم دیگر را سفر
\\
آنچنانک عاشقی بر سروری
&&
عاشقست آن خواجه بر آهنگری
\\
هر کسی را بهر کاری ساختند
&&
میل آن را در دلش انداختند
\\
دست و پا بی میل جنبان کی شود
&&
خار وخس بی آب و بادی کی رود
\\
گر ببینی میل خود سوی سما
&&
پر دولت بر گشا همچون هما
\\
ور ببینی میل خود سوی زمین
&&
نوحه می‌کن هیچ منشین از حنین
\\
عاقلان خود نوحه‌ها پیشین کنند
&&
جاهلان آخر بسر بر می‌زنند
\\
ز ابتدای کار آخر را ببین
&&
تا نباشی تو پشیمان یوم دین
\\
\end{longtable}
\end{center}
