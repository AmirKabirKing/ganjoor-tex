\begin{center}
\section*{بخش ۱۸۳ - منع کردن دوستان او را از رجوع کردن به بخارا وتهدید کردن و لاابالی گفتن او}
\label{sec:sh183}
\addcontentsline{toc}{section}{\nameref{sec:sh183}}
\begin{longtable}{l p{0.5cm} r}
گفت او را ناصحی ای بی‌خبر
&&
عاقبت اندیش اگر داری هنر
\\
درنگر پس را به عقل و پیش را
&&
همچو پروانه مسوزان خویش را
\\
چون بخارا می‌روی دیوانه‌ای
&&
لایق زنجیر و زندان‌خانه‌ای
\\
او ز تو آهن همی‌خاید ز خشم
&&
او همی‌جوید ترا با بیست چشم
\\
می‌کند او تیز از بهر تو کارد
&&
او سگ قحطست و تو انبان آرد
\\
چون رهیدی و خدایت راه داد
&&
سوی زندان می‌روی چونت فتاد
\\
بر تو گر ده‌گون موکل آمدی
&&
عقل بایستی کز ایشان کم زدی
\\
چون موکل نیست بر تو هیچ‌کس
&&
از چه بسته گشت بر تو پیش و پس
\\
عشق پنهان کرده بود او را اسیر
&&
آن موکل را نمی‌دید آن نذیر
\\
هر موکل را موکل مختفیست
&&
ورنه او در بند سگ طبعی ز چیست
\\
خشم شاه عشق بر جانش نشست
&&
بر عوانی و سیه‌روییش بست
\\
می‌زند او را که هین او رابزن
&&
زان عوانان نهان افغان من
\\
هرکه بینی در زیانی می‌رود
&&
گرچه تنها با عوانی می‌رود
\\
گر ازو واقف بدی افغان زدی
&&
پیش آن سلطان سلطانان شدی
\\
ریختی بر سر به پیش شاه خاک
&&
تا امان دیدی ز دیو سهمناک
\\
میر دیدی خویش را ای کم ز مور
&&
زان ندیدی آن موکل را تو کور
\\
غره گشتی زین دروغین پر و بال
&&
پر و بالی کو کشد سوی وبال
\\
پر سبک دارد ره بالا کند
&&
چون گل‌آلو شد گرانیها کند
\\
\end{longtable}
\end{center}
