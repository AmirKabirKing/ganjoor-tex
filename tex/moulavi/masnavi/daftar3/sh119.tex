\begin{center}
\section*{بخش ۱۱۹ - شرح آن کور دوربین و آن کر تیزشنو و  آن برهنه دراز دامن}
\label{sec:sh119}
\addcontentsline{toc}{section}{\nameref{sec:sh119}}
\begin{longtable}{l p{0.5cm} r}
کر امل را دان که مرگ ما شنید
&&
مرگ خود نشنید و نقل خود ندید
\\
حرص نابیناست بیند مو بمو
&&
عیب خلقان و بگوید کو بکو
\\
عیب خود یک ذره چشم کور او
&&
می‌نبیند گرچه هست او عیب‌جو
\\
عور می‌ترسد که دامانش برند
&&
دامن مرد برهنه چون درند
\\
مرد دنیا مفلس است و ترسناک
&&
هیچ او را نیست از دزدانش باک
\\
او برهنه آمد و عریان رود
&&
وز غم دزدش جگر خون می‌شود
\\
وقت مرگش که بود صد نوحه بیش
&&
خنده آید جانش را زین ترس خویش
\\
آن زمان داند غنی کش نیست زر
&&
هم ذکی داند که او بد بی‌هنر
\\
چون کنار کودکی پر از سفال
&&
کو بر آن لرزان بود چون رب مال
\\
گر ستانی پاره‌ای گریان شود
&&
پاره گر بازش دهی خندان شود
\\
چون نباشد طفل را دانش دثار
&&
گریه و خنده‌ش ندارد اعتبار
\\
محتشم چون عاریت را ملک دید
&&
پس بر آن مال دروغین می‌طپید
\\
خواب می‌بیند که او را هست مال
&&
ترسد از دزدی که برباید جوال
\\
چون ز خوابش بر جهاند گوش‌کش
&&
پس ز ترس خویش تسخر آیدش
\\
همچنان لرزانی این عالمان
&&
که بودشان عقل و علم این جهان
\\
از پی این عاقلان ذو فنون
&&
گفت ایزد در نبی لا یعلمون
\\
هر یکی ترسان ز دزدی کسی
&&
خویشتن را علم پندارد بسی
\\
گوید او که روزگارم می‌برند
&&
خود ندارد روزگار سودمند
\\
گوید از کارم بر آوردند خلق
&&
غرق بی‌کاریست جانش تابه حلق
\\
عور ترسان که منم دامن کشان
&&
چون رهانم دامن از چنگالشان
\\
صد هزاران فضل داند از علوم
&&
جان خود را می‌نداند آن ظلوم
\\
داند او خاصیت هر جوهری
&&
در بیان جوهر خود چون خری
\\
که همی‌دانم یجوز و لایجوز
&&
خود ندانی تو یجوزی یا عجوز
\\
این روا و آن ناروا دانی ولیک
&&
تو روا یا ناروایی بین تو نیک
\\
قیمت هر کاله می‌دانی که چیست
&&
قیمت خود را ندانی احمقیست
\\
سعدها و نحسها دانسته‌ای
&&
ننگری سعدی تو یا ناشسته‌ای
\\
جان جمله علمها اینست این
&&
که بدانی من کیم در یوم دین
\\
آن اصول دین بدانستی ولیک
&&
بنگر اندر اصل خود گر هست نیک
\\
از اصولینت اصول خویش به
&&
که بدانی اصل خود ای مرد مه
\\
\end{longtable}
\end{center}
