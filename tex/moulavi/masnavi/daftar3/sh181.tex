\begin{center}
\section*{بخش ۱۸۱ - عزم کردن آن وکیل ازعشق کی رجوع کند  به بخارا لاابالی‌وار}
\label{sec:sh181}
\addcontentsline{toc}{section}{\nameref{sec:sh181}}
\begin{longtable}{l p{0.5cm} r}
شمع مریم را بهل افروخته
&&
که بخارا می‌رود آن سوخته
\\
سخت بی‌صبر و در آتشدان تیز
&&
رو سوی صدر جهان می‌کن گریز
\\
این بخارا منبع دانش بود
&&
پس بخاراییست هر کنش بود
\\
پیش شیخی در بخارا اندری
&&
تا به خواری در بخارا ننگری
\\
جز به خواری در بخارای دلش
&&
راه ندهد جزر و مد مشکلش
\\
ای خنک آن را که ذلت نفسه
&&
وای آنکس را که یردی رفسه
\\
فرقت صدر جهان در جان او
&&
پاره پاره کرده بود ارکان او
\\
گفت بر خیزم هم‌آنجا واروم
&&
کافر ار گشتم دگر ره بگروم
\\
واروم آنجا بیفتم پیش او
&&
پیش آن صدر نکواندیش او
\\
گویم افکندم به پیشت جان خویش
&&
زنده کن یا سر ببر ما را چو میش
\\
کشته و مرده به پیشت ای قمر
&&
به که شاه زندگان جای دگر
\\
آزمودم من هزاران بار بیش
&&
بی تو شیرین می‌نبینم عیش خویش
\\
غن لی یا منیتی لحن النشور
&&
ابرکی یا ناقتی تم السرور
\\
ابلعی یا ارض دمعی قد کفی
&&
اشربی یا نفس وردا قد صفا
\\
عدت یا عیدی الینا مرحبا
&&
نعم ما روحت یا ریح الصبا
\\
گفت ای یاران روان گشتم وداع
&&
سوی آن صدری که میر است و مطاع
\\
دم‌بدم در سوز بریان می‌شوم
&&
هرچه بادا باد آنجا می‌روم
\\
گرچه دل چون سنگ خارا می‌کند
&&
جان من عزم بخارا می‌کند
\\
مسکن یارست و شهر شاه من
&&
پیش عاشق این بود حب الوطن
\\
\end{longtable}
\end{center}
