\begin{center}
\section*{بخش ۹۷ - پیش رفتن دقوقی به امامت آن قوم}
\label{sec:sh097}
\addcontentsline{toc}{section}{\nameref{sec:sh097}}
\begin{longtable}{l p{0.5cm} r}
در تحیات و سلام الصالحین
&&
مدح جملهٔ انبیا آمد عجین
\\
مدحها شد جملگی آمیخته
&&
کوزه‌ها در یک لگن در ریخته
\\
زانک خود ممدوح جز یک بیش نیست
&&
کیشها زین روی جز یک کیش نیست
\\
دان که هر مدحی بنور حق رود
&&
بر صور و اشخاص عاریت بود
\\
مدحها جز مستحق را کی کنند
&&
لیک بر پنداشت گم‌ره می‌شوند
\\
همچو نوری تافته بر حایطی
&&
حایط آن انوار را چون رابطی
\\
لاجرم چون سایه سوی اصل راند
&&
ضال مه گم کرد و ز استایش بماند
\\
یا ز چاهی عکس ماهی وا نمود
&&
سر بچه در کرد و آن را می‌ستود
\\
در حقیقت مادح ماهست او
&&
گرچه جهل او بعکسش کرد رو
\\
مدح او مه‌راست نه آن عکس را
&&
کفر شد آن چون غلط شد ماجرا
\\
کز شقاوت گشت گم‌ره آن دلیر
&&
مه به بالا بود و او پنداشت زیر
\\
زین بتان خلقان پریشان می‌شوند
&&
شهوت رانده پشیمان می‌شوند
\\
زآنک شهوت با خیالی رانده است
&&
وز حقیقت دورتر وا مانده است
\\
با خیالی میل تو چون پر بود
&&
تا بدان پر بر حقیقت بر شود
\\
چون براندی شهوتی پرت بریخت
&&
لنگ گشتی و آن خیال از تو گریخت
\\
پر نگه دار و چنین شهوت مران
&&
تا پر میلت برد سوی جنان
\\
خلق پندارند عشرت می‌کنند
&&
بر خیالی پر خود بر می‌کنند
\\
وام‌دار شرح این نکته شدم
&&
مهلتم ده معسرم زان تن زدم
\\
\end{longtable}
\end{center}
