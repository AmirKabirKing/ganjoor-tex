\begin{center}
\section*{بخش ۱۶۹ - حکمت ویران شدن تن به مرگ}
\label{sec:sh169}
\addcontentsline{toc}{section}{\nameref{sec:sh169}}
\begin{longtable}{l p{0.5cm} r}
من چو آدم بودم اول حبس کرب
&&
پر شد اکنون نسل جانم شرق و غرب
\\
من گدا بودم درین خانه چو چاه
&&
شاه گشتم قصر باید بهر شاه
\\
قصرها خود مر شهان را مانسست
&&
مرده را خانه و مکان گوری بسست
\\
انبیا را تنگ آمد این جهان
&&
چون شهان رفتند اندر لامکان
\\
مردگان را این جهان بنمود فر
&&
ظاهرش زفت و به معنی تنگ بر
\\
گر نبودی تنگ این افغان ز چیست
&&
چون دو تا شد هر که در وی بیش زیست
\\
در زمان خواب چون آزاد شد
&&
زان مکان بنگر که جان چون شاد شد
\\
ظالم از ظلم طبیعت باز رست
&&
مرد زندانی ز فکر حبس جست
\\
این زمین و آسمان بس فراخ
&&
سخت تنگ آمد به هنگام مناخ
\\
جسم بند آمد فراخ وسخت تنگ
&&
خندهٔ او گریه فخرش جمله ننگ
\\
\end{longtable}
\end{center}
