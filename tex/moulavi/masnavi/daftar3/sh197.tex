\begin{center}
\section*{بخش ۱۹۷ - جواب گفتن مهمان ایشان را و مثل آوردن بدفع کردن حارس کشت به بانگ دف از کشت شتری را کی کوس محمودی بر پشت او زدندی}
\label{sec:sh197}
\addcontentsline{toc}{section}{\nameref{sec:sh197}}
\begin{longtable}{l p{0.5cm} r}
گفت ای یاران از آن دیوان نیم
&&
که ز لا حولی ضعیف آید پیم
\\
کودکی کو حارس کشتی بدی
&&
طبلکی در دفع مرغان می‌زدی
\\
تا رمیدی مرغ زان طبلک ز کشت
&&
کشت از مرغان بد بی خوف گشت
\\
چونک سلطان شاه محمود کریم
&&
برگذر زد آن طرف خیمهٔ عظیم
\\
با سپاهی همچو استارهٔ اثیر
&&
انبه و پیروز و صفدر ملک‌گیر
\\
اشتری بد کو بدی حمال کوس
&&
بختیی بد پیش‌رو همچون خروس
\\
بانگ کوس و طبل بر وی روز و شب
&&
می‌زدی اندر رجوع و در طلب
\\
اندر آن مزرع در آمد آن شتر
&&
کودک آن طبلک بزد در حفظ بر
\\
عاقلی گفتش مزن طبلک که او
&&
پختهٔ طبلست با آنشست خو
\\
پیش او چه بود تبوراک تو طفل
&&
که کشد او طبل سلطان بیست کفل
\\
عاشقم من کشتهٔ قربان لا
&&
جان من نوبتگه طبل بلا
\\
خود تبوراکست این تهدیدها
&&
پیش آنچ دیده است این دیدها
\\
ای حریفان من از آنها نیستم
&&
کز خیالاتی درین ره بیستم
\\
من چو اسماعیلیانم بی‌حذر
&&
بل چو اسمعیل آزادم ز سر
\\
فارغم از طمطراق و از ریا
&&
قل تعالوا گفت جانم را بیا
\\
گفت پیغامبر که جاد فی السلف
&&
بالعطیه من تیقن بالخلف
\\
هر که بیند مر عطا را صد عوض
&&
زود دربازد عطا را زین غرض
\\
جمله در بازار از آن گشتند بند
&&
تا چو سود افتاد مال خود دهند
\\
زر در انبانها نشسته منتظر
&&
تا که سود آید ببذل آید مصر
\\
چون ببیند کاله‌ای در ربح بیش
&&
سرد گردد عشقش از کالای خویش
\\
گرم زان ماندست با آن کو ندید
&&
کاله‌های خویش را ربح و مزید
\\
همچنین علم و هنرها و حرف
&&
چون بدید افزون از آنها در شرف
\\
تا به از جان نیست جان باشد عزیز
&&
چون به آمد نام جان شد چیز لیز
\\
لعبت مرده بود جان طفل را
&&
تا نگشت او در بزرگی طفل‌زا
\\
این تصور وین تخیل لعبتست
&&
تا تو طفلی پس بدانت حاجتست
\\
چون ز طفلی رست جان شد در وصال
&&
فارغ از حس است و تصویر و خیال
\\
نیست محرم تا بگویم بی‌نفاق
&&
تن زدم والله اعلم بالوفاق
\\
مال و تن برف‌اند ریزان فنا
&&
حق خریدارش که الله اشتری
\\
برفها زان از ثمن اولیستت
&&
که هیی در شک یقینی نیستت
\\
وین عجب ظنست در تو ای مهین
&&
که نمی‌پرد به بستان یقین
\\
هر گمان تشنهٔ یقینست ای پسر
&&
می‌زند اندر تزاید بال و پر
\\
چون رسد در علم پس پر پا شود
&&
مر یقین را علم او بویا شود
\\
زانک هست اندر طریق مفتتن
&&
علم کمتر از یقین و فوق ظن
\\
علم جویای یقین باشد بدان
&&
و آن یقین جویای دیدست و عیان
\\
اندر الهیکم بجو این را کنون
&&
از پس کلا پس لو تعلمون
\\
می‌کشد دانش ببینش ای علیم
&&
گر یقین گشتی ببینندی جحیم
\\
دید زاید از یقین بی امتهال
&&
آنچنانک از ظن می‌زاید خیال
\\
اندر الهیکم بیان این ببین
&&
که شود علم الیقین عین الیقین
\\
از گمان و از یقین بالاترم
&&
وز ملامت بر نمی‌گردد سرم
\\
چون دهانم خورد از حلوای او
&&
چشم‌روشن گشتم و بینای او
\\
پا نهم گستاخ چون خانه روم
&&
پا نلرزانم نه کورانه روم
\\
آنچ گل را گفت حق خندانش کرد
&&
با دل من گفت و صد چندانش کرد
\\
آنچ زد بر سرو و قدش راست کرد
&&
و آنچ از وی نرگس و نسرین بخورد
\\
آنچ نی را کرد شیرین جان و دل
&&
و آنچ خاکی یافت ازو نقش چگل
\\
آنچ ابرو را چنان طرار ساخت
&&
چهره را گلگونه و گلنار ساخت
\\
مر زبان را داد صد افسون‌گری
&&
وانک کان را داد زر جعفری
\\
چون در زرادخانه باز شد
&&
غمزه‌های چشم تیرانداز شد
\\
بر دلم زد تیر و سوداییم کرد
&&
عاشق شکر و شکرخاییم کرد
\\
عاشق آنم که هر آن آن اوست
&&
عقل و جان جاندار یک مرجان اوست
\\
من نلافم ور بلافم همچو آب
&&
نیست در آتش‌کشی‌ام اضطراب
\\
چون بدزدم چون حفیظ مخزن اوست
&&
چون نباشم سخت‌رو پشت من اوست
\\
هر که از خورشید باشد پشت گرم
&&
سخت رو باشد نه بیم او را نه شرم
\\
همچو روی آفتاب بی‌حذر
&&
گشت رویش خصم‌سوز و پرده‌در
\\
هر پیمبر سخت‌رو بد در جهان
&&
یکسواره کوفت بر جیش شهان
\\
رو نگردانید از ترس و غمی
&&
یک‌تنه تنها بزد بر عالمی
\\
سنگ باشد سخت‌رو و چشم‌شوخ
&&
او نترسد از جهان پر کلوخ
\\
کان کلوخ از خشت‌زن یک‌لخت شد
&&
سنگ از صنع خدایی سخت شد
\\
گوسفندان گر برونند از حساب
&&
ز انبهیشان کی بترسد آن قصاب
\\
کلکم راع نبی چون راعیست
&&
خلق مانند رمه او ساعیست
\\
از رمه چوپان نترسد در نبرد
&&
لیکشان حافظ بود از گرم و سرد
\\
گر زند بانگی ز قهر او بر رمه
&&
دان ز مهرست آن که دارد بر همه
\\
هر زمان گوید به گوشم بخت نو
&&
که ترا غمگین کنم غمگین مشو
\\
من ترا غمگین و گریان زان کنم
&&
تا کت از چشم بدان پنهان کنم
\\
تلخ گردانم ز غمها خوی تو
&&
تا بگردد چشم بد از روی تو
\\
نه تو صیادی و جویای منی
&&
بنده و افکندهٔ رای منی
\\
حیله اندیشی که در من در رسی
&&
در فراق و جستن من بی‌کسی
\\
چاره می‌جوید پی من درد تو
&&
می‌شنودم دوش آه سرد تو
\\
من توانم هم که بی این انتظار
&&
ره دهم بنمایمت راه گذار
\\
تا ازین گرداب دوران وا رهی
&&
بر سر گنج وصالم پا نهی
\\
لیک شیرینی و لذات مقر
&&
هست بر اندازهٔ رنج سفر
\\
آنگه ا ز شهر و ز خویشان بر خوری
&&
کز غریبی رنج و محنتها بری
\\
\end{longtable}
\end{center}
