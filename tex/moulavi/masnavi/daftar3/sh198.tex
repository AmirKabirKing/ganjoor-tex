\begin{center}
\section*{بخش ۱۹۸ - تمثیل گریختن ممن و بی‌صبری او در بلا به اضطراب و بی‌قراری نخود و دیگر حوایج در جوش دیگ و بر دویدن تا بیرون جهند}
\label{sec:sh198}
\addcontentsline{toc}{section}{\nameref{sec:sh198}}
\begin{longtable}{l p{0.5cm} r}
بنگر اندر نخودی در دیگ چون
&&
می‌جهد بالا چو شد ز آتش زبون
\\
هر زمان نخود بر آید وقت جوش
&&
بر سر دیگ و برآرد صد خروش
\\
که چرا آتش به من در می‌زنی
&&
چون خریدی چون نگونم می‌کنی
\\
می‌زند کفلیز کدبانو که نی
&&
خوش بجوش و بر مجه ز آتش‌کنی
\\
زان نجوشانم که مکروه منی
&&
بلک تا گیری تو ذوق و چاشنی
\\
تا غذی گردی بیامیزی بجان
&&
بهرخواری نیستت این امتحان
\\
آب می‌خوردی به بستان سبز و تر
&&
بهراین آتش بدست آن آب خور
\\
رحمتش سابق بدست از قهر زان
&&
تا ز رحمت گردد اهل امتحان
\\
رحمتش بر قهر از آن سابق شدست
&&
تا که سرمایهٔ وجود آید بدست
\\
زانک بی‌لذت نروید لحم و پوست
&&
چون نروید چه گدازد عشق دوست
\\
زان تقاضا گر بیاید قهرها
&&
تا کنی ایثار آن سرمایه را
\\
باز لطف آید برای عذر او
&&
که بکردی غسل و بر جستی ز جو
\\
گوید ای نخود چریدی در بهار
&&
رنج مهمان تو شد نیکوش دار
\\
تا که مهمان باز گردد شکر ساز
&&
پیش شه گوید ز ایثار تو باز
\\
تا به جای نعمتت منعم رسد
&&
جمله نعمتها برد بر تو حسد
\\
من خلیلم تو پسر پیش بچک
&&
سر بنه انی ارانی اذبحک
\\
سر به پیش قهر نه دل بر قرار
&&
تا ببرم حلقت اسمعیل‌وار
\\
سر ببرم لیک این سر آن سریست
&&
کز بریده گشتن و مردن بریست
\\
لیک مقصود ازل تسلیم تست
&&
ای مسلمان بایدت تسلیم جست
\\
ای نخود می‌جوش اندر ابتلا
&&
تا نه هستی و نه خود ماند ترا
\\
اندر آن بستان اگر خندیده‌ای
&&
تو گل بستان جان و دیده‌ای
\\
گر جدا از باغ آب و گل شدی
&&
لقمه گشتی اندر احیا آمدی
\\
شو غذی و قوت و اندیشه‌ها
&&
شیر بودی شیر شو در بیشه‌ها
\\
از صفاتش رسته‌ای والله نخست
&&
در صفاتش باز رو چالاک و چست
\\
ز ابر و خورشید و ز گردون آمدی
&&
پس شدی اوصاف و گردون بر شدی
\\
آمدی در صورت باران و تاب
&&
می‌روی اندر صفات مستطاب
\\
جزو شید و ابر و انجمها بدی
&&
نفس و فعل و قول و فکرتها شدی
\\
هستی حیوان شد از مرگ نبات
&&
راست آمد اقتلونی یا ثقات
\\
چون چنین بردیست ما را بعد مات
&&
راست آمد ان فی قتلی حیات
\\
فعل و قول و صدق شد قوت ملک
&&
تا بدین معراج شد سوی فلک
\\
آنچنان کان طعمه شد قوت بشر
&&
از جمادی بر شد و شد جانور
\\
این سخن را ترجمهٔ پهناوری
&&
گفته آید در مقام دیگری
\\
کاروان دایم ز گردون می‌رسد
&&
تا تجارت می‌کند وا می‌رود
\\
پس برو شیرین و خوش با اختیار
&&
نه بتلخی و کراهت دزدوار
\\
زان حدیث تلخ می‌گویم ترا
&&
تا ز تلخیها فرو شویم ترا
\\
ز آب سرد انگور افسرده رهد
&&
سردی و افسردگی بیرون نهد
\\
تو ز تلخی چونک دل پر خون شوی
&&
پس ز تلخیها همه بیرون روی
\\
\end{longtable}
\end{center}
