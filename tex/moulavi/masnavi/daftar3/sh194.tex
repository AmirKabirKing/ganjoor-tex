\begin{center}
\section*{بخش ۱۹۴ - دیگر باره ملامت کردن اهل مسجد مهمان را از شب خفتن در آن مسجد}
\label{sec:sh194}
\addcontentsline{toc}{section}{\nameref{sec:sh194}}
\begin{longtable}{l p{0.5cm} r}
قوم گفتندش مکن جلدی برو
&&
تا نگردد جامه و جانت گرو
\\
آن ز دور آسان نماید به نگر
&&
که به آخر سخت باشد ره‌گذر
\\
خویشتن آویخت بس مرد و سکست
&&
وقت پیچاپیچ دست‌آویز جست
\\
پیشتر از واقعه آسان بود
&&
در دل مردم خیال نیک و بد
\\
چون در آید اندرون کارزار
&&
آن زمان گردد بر آنکس کار زار
\\
چون نه شیری هین منه تو پای پیش
&&
کان اجل گرگست و جان تست میش
\\
ور ز ابدالی و میشت شیر شد
&&
آمن آ که مرگ تو سرزیر شد
\\
کیست ابدال آنک او مبدل شود
&&
خمرش از تبدیل یزدان خل شود
\\
لیک مستی شیرگیری وز گمان
&&
شیر پنداری تو خود را هین مران
\\
گفت حق ز اهل نفاق ناسدید
&&
باسهم ما بینهم باس شدید
\\
در میان همدگر مردانه‌اند
&&
در غزا چون عورتان خانه‌اند
\\
گفت پیغامبر سپهدار غیوب
&&
لا شجاعة یا فتی قبل الحروب
\\
وقت لاف غزو مستان کف کنند
&&
وقت جوش جنگ چون کف بی‌فنند
\\
وقت ذکر غزو شمشیرش دراز
&&
وقت کر و فر تیغش چون پیاز
\\
وقت اندیشه دل او زخم‌جو
&&
پس به یک سوزن تهی شد خیک او
\\
من عجب دارم ز جویای صفا
&&
کو رمد در وقت صیقل از جفا
\\
عشق چون دعوی جفا دیدن گواه
&&
چون گواهت نیست شد دعوی تباه
\\
چون گواهت خواهد این قاضی مرنج
&&
بوسه ده بر مار تا یابی تو گنج
\\
آن جفا با تو نباشد ای پسر
&&
بلک با وصف بدی اندر تو در
\\
بر نمد چوبی که آن را مرد زد
&&
بر نمد آن را نزد بر گرد زد
\\
گر بزد مر اسپ را آن کینه کش
&&
آن نزد بر اسپ زد بر سکسکش
\\
تا ز سکسک وا رهد خوش‌پی شود
&&
شیره را زندان کنی تا می‌شود
\\
گفت چندان آن یتیمک را زدی
&&
چون نترسیدی ز قهر ایزدی
\\
گفت او را کی زدم ای جان و دوست
&&
من بر آن دیوی زدم کو اندروست
\\
مادر ار گوید ترا مرگ تو باد
&&
مرگ آن خو خواهد و مرگ فساد
\\
آن گروهی کز ادب بگریختند
&&
آب مردی و آب مردان ریختند
\\
عاذلانشان از وغا وا راندند
&&
تا چنین حیز و مخنث ماندند
\\
لاف و غرهٔ ژاژخا را کم شنو
&&
با چنینها در صف هیجا مرو
\\
زانک زاد و کم خبالا گفت حق
&&
کز رفاق سست برگردان ورق
\\
که گر ایشان با شما همره شوند
&&
غازیان بی‌مغز همچون که شوند
\\
خویشتن را با شما هم‌صف کنند
&&
پس گریزند و دل صف بشکنند
\\
پس سپاهی اندکی بی این نفر
&&
به که با اهل نفاق آید حشر
\\
هست بادام کم خوش بیخته
&&
به ز بسیاری به تلخ آمیخته
\\
تلخ و شیرین در ژغاژغ یک شی‌اند
&&
نقص از آن افتاد که همدل نیند
\\
گبر ترسان دل بود کو از گمان
&&
می‌زید در شک ز حال آن جهان
\\
می‌رود در ره نداند منزلی
&&
گام ترسان می‌نهد اعمی دلی
\\
چون نداند ره مسافر چون رود
&&
با ترددها و دل پرخون رود
\\
هرکه گویدهای این‌سو راه نیست
&&
او کند از بیم آنجا وقف و ایست
\\
ور بداند ره دل با هوش او
&&
کی رود هر های و هو در گوش او
\\
پس مشو همراه این اشتردلان
&&
زانک وقت ضیق و بیمند آفلان
\\
پس گریزند و ترا تنها هلند
&&
گرچه اندر لاف سحر بابلند
\\
تو ز رعنایان مجو هین کارزار
&&
تو ز طاوسان مجو صید و شکار
\\
طبع طاوسست و وسواست کند
&&
دم زند تا از مقامت بر کند
\\
\end{longtable}
\end{center}
