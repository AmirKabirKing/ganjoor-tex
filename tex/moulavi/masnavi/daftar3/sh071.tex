\begin{center}
\section*{بخش ۷۱ - تشبیه بند و دام قضا به صورت  پنهان به اثر پیدا}
\label{sec:sh071}
\addcontentsline{toc}{section}{\nameref{sec:sh071}}
\begin{longtable}{l p{0.5cm} r}
بینی اندر دلق مهتر زاده‌ای
&&
سر برهنه در بلا افتاده‌ای
\\
در هوای نابکاری سوخته
&&
اقمشه و املاک خود بفروخته
\\
خان و مان رفته شده بدنام و خوار
&&
کام دشمن می‌رود ادبیروار
\\
زاهدی بیند بگوید ای کیا
&&
همتی می‌دار از بهر خدا
\\
کاندرین ادبار زشت افتاده‌ام
&&
مال و زر و نعمت از کف داده‌ام
\\
همتی تا بوک من زین وا رهم
&&
زین گل تیره بود که بر جهم
\\
این دعا می‌خواهد او از عام و خاص
&&
کالخلاص و الخلاص و الخلاص
\\
دست باز و پای باز و بند نی
&&
نه موکل بر سرش نه آهنی
\\
از کدامین بند می‌جویی خلاص
&&
وز کدامین حبس می‌جویی مناص
\\
بند تقدیر و قضای مختفی
&&
کی نبیند آن به جز جان صفی
\\
گرچه پیدا نیست آن در مکمنست
&&
بتر از زندان و بند آهنست
\\
زانک آهنگر مر آن را بشکند
&&
حفره گر هم خشت زندان بر کند
\\
ای عجب این بند پنهان گران
&&
عاجز از تکسیر آن آهنگران
\\
دیدن آن بند احمد را رسد
&&
بر گلوی بسته حبل من مسد
\\
دید بر پشت عیال بولهب
&&
تنگ هیزم گفت حمالهٔ حطب
\\
حبل و هیزم را جز او چشمی ندید
&&
که پدید آید برو هر ناپدید
\\
باقیانش جمله تاویلی کنند
&&
کین ز بیهوشیست و ایشان هوشمند
\\
لیک از تاثیر آن پشتش دوتو
&&
گشته و نالان شده او پیش تو
\\
که دعایی همتی تا وا رهم
&&
تا ازین بند نهان بیرون جهم
\\
آنک بیند این علامتها پدید
&&
چون نداند او شقی را از سعید
\\
داند و پوشد بامر ذوالجلال
&&
که نباشد کشف راز حق حلال
\\
این سخن پایان ندارد آن فقیر
&&
از مجاعت شد زبون و تن اسیر
\\
\end{longtable}
\end{center}
