\begin{center}
\section*{بخش ۱۶۰ - خبر کردن خروس از مرگ خواجه}
\label{sec:sh160}
\addcontentsline{toc}{section}{\nameref{sec:sh160}}
\begin{longtable}{l p{0.5cm} r}
لیک فردا خواهد او مردن یقین
&&
گاو خواهد کشت وارث در حنین
\\
صاحب خانه بخواهد مرد رفت
&&
روز فردا نک رسیدت لوت زفت
\\
پاره‌های نان و لالنگ و طعام
&&
در میان کوی یابد خاص و عام
\\
گاو قربانی و نانهای تنک
&&
بر سگان و سایلان ریزد سبک
\\
مرگ اسپ و استر و مرگ غلام
&&
بد قضا گردان این مغرور خام
\\
از زیان مال و درد آن گریخت
&&
مال افزون کرد و خون خویش ریخت
\\
این ریاضتهای درویشان چراست
&&
کان بلا بر تن بقای جانهاست
\\
تا بقای خود نیابد سالکی
&&
چون کند تن را سقیم و هالکی
\\
دست کی جنبد به ایثار و عمل
&&
تا نبیند داده را جانش بدل
\\
آنک بدهد بی امید سودها
&&
آن خدایست آن خدایست آن خدا
\\
یا ولی حق که خوی حق گرفت
&&
نور گشت و تابش مطلق گرفت
\\
کو غنی است و جز او جمله فقیر
&&
کی فقیری بی عوض گوید که گیر
\\
تا نبیند کودکی که سیب هست
&&
او پیاز گنده را ندهد ز دست
\\
این همه بازار بهر این غرض
&&
بر دکانها شسته بر بوی عوض
\\
صد متاع خوب عرضه می‌کنند
&&
واندرون دل عوضها می‌تنند
\\
یک سلامی نشنوی ای مرد دین
&&
که نگیرد آخرت آن آستین
\\
بی طمع نشنیده‌ام از خاص و عام
&&
من سلامی ای برادر والسلام
\\
جز سلام حق هین آن را بجو
&&
خانه خانه جا بجا و کو بکو
\\
از دهان آدمی خوش‌مشام
&&
هم پیام حق شنودم هم سلام
\\
وین سلام باقیان بر بوی آن
&&
من همی‌نوشم به دل خوشتر ز جان
\\
زان سلام او سلام حق شدست
&&
کآتش اندر دودمان خود زدست
\\
مرده است از خود شده زنده برب
&&
زان بود اسرار حقش در دو لب
\\
مردن تن در ریاضت زندگیست
&&
رنج این تن روح را پایندگیست
\\
گوش بنهاده بد آن مرد خبیث
&&
می‌شنود او از خروسش آن حدیث
\\
\end{longtable}
\end{center}
