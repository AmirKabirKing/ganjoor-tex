\begin{center}
\section*{بخش ۵۱ - مثل در بیان آنک حیرت مانع بحث و فکرتست}
\label{sec:sh051}
\addcontentsline{toc}{section}{\nameref{sec:sh051}}
\begin{longtable}{l p{0.5cm} r}
آن یکی مرد دومو آمد شتاب
&&
پیش یک آیینه دار مستطاب
\\
گفت از ریشم سپیدی کن جدا
&&
که عروس نو گزیدم ای فتی
\\
ریش او ببرید و کل پیشش نهاد
&&
گفت تو بگزین مرا کاری فتاد
\\
این سؤال وآن جوابست آن گزین
&&
که سر اینها ندارد درد دین
\\
آن یکی زد سیلیی مر زید را
&&
حمله کرد او هم برای کید را
\\
گفت سیلی‌زن سؤالت می‌کنم
&&
پس جوابم گوی وانگه می‌زنم
\\
بر قفای تو زدم آمد طراق
&&
یک سؤالی دارم اینجا در وفاق
\\
این طراق از دست من بودست یا
&&
از قفاگاه تو ای فخر کیا
\\
گفت از درد این فراغت نیستم
&&
که درین فکر و تفکر بیستم
\\
تو که بی‌دردی همی اندیش این
&&
نیست صاحب‌درد را این فکر هین
\\
\end{longtable}
\end{center}
