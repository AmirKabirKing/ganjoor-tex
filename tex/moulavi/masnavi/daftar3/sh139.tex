\begin{center}
\section*{بخش ۱۳۹ - حکمت آفریدن دوزخ آن جهان و زندان این جهان تا معبد متکبران باشد کی ائتیا طوعا او کرها}
\label{sec:sh139}
\addcontentsline{toc}{section}{\nameref{sec:sh139}}
\begin{longtable}{l p{0.5cm} r}
که لئیمان در جفا صافی شوند
&&
چون وفا بینند خود جافی شوند
\\
مسجد طاعاتشان پس دوزخست
&&
پای‌بند مرغ بیگانه فخست
\\
هست زندان صومعهٔ دزد و لیم
&&
کاندرو ذاکر شود حق را مقیم
\\
چون عبادت بود مقصود از بشر
&&
شد عبادتگاه گردن‌کش سقر
\\
آدمی را هست در هر کار دست
&&
لیک ازو مقصود این خدمت بدست
\\
ما خلقت الجن و الانس این بخوان
&&
جز عبادت نیست مقصود از جهان
\\
گرچه مقصود از کتاب آن فن بود
&&
گر توش بالش کنی هم می‌شود
\\
لیک ازو مقصود این بالش نبود
&&
علم بود و دانش و ارشاد سود
\\
گر تو میخی ساختی شمشیر را
&&
برگزیدی بر ظفر ادبار را
\\
گرچه مقصود از بشر علم و هدیست
&&
لیک هر یک آدمی را معبدیست
\\
معبد مرد کریم اکرمته
&&
معبد مرد لئیم اسقمته
\\
مر لئیمان را بزن تا سر نهند
&&
مر کریمان را بده تا بر دهند
\\
لاجرم حق هر دو مسجد آفرید
&&
دوزخ آنها را و اینها را مزید
\\
ساخت موسی قدس در باب صغیر
&&
تا فرود آرند سر قوم زحیر
\\
زآنک جباران بدند و سرفراز
&&
دوزخ آن باب صغیرست و نیاز
\\
\end{longtable}
\end{center}
