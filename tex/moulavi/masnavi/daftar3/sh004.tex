\begin{center}
\section*{بخش ۴ - بازگشتن به حکایت پیل}
\label{sec:sh004}
\addcontentsline{toc}{section}{\nameref{sec:sh004}}
\begin{longtable}{l p{0.5cm} r}
گفت ناصح بشنوید این پند من
&&
تا دل و جانتان نگردد ممتحن
\\
با گیاه و برگها قانع شوید
&&
در شکار پیل‌بچگان کم روید
\\
من برون کردم ز گردن وام نصح
&&
جز سعادت کی بود انجام نصح
\\
من به تبلیغ رسالت آمدم
&&
تا رهانم مر شما را از ندم
\\
هین مبادا که طمع رهتان زند
&&
طمع برگ از بیخهاتان بر کند
\\
این بگفت و خیربادی کرد و رفت
&&
گشت قحط و جوعشان در راه زفت
\\
ناگهان دیدند سوی جاده‌ای
&&
پور پیلی فربهی نو زاده‌ای
\\
اندر افتادند چون گرگان مست
&&
پاک خوردندش فرو شستند دست
\\
آن یکی همره نخورد و پند داد
&&
که حدیث آن فقیرش بود یاد
\\
از کبابش مانع آمد آن سخن
&&
بخت نو بخشد ترا عقل کهن
\\
پس بیفتادند و خفتند آن همه
&&
وان گرسنه چون شبان اندر رمه
\\
دید پیلی سهمناکی می‌رسید
&&
اولا آمد سوی حارس دوید
\\
بوی می‌کرد آن دهانش را سه بار
&&
هیچ بویی زو نیامد ناگوار
\\
چند باری گرد او گشت و برفت
&&
مر ورا نازرد آن شه‌پیل زفت
\\
مر لب هر خفته‌ای را بوی کرد
&&
بوی می‌آمد ورا زان خفته مرد
\\
از کباب پیل‌زاده خورده بود
&&
بر درانید و بکشتش پیل زود
\\
در زمان او یک بیک را زان گروه
&&
می‌درانید و نبودش زان شکوه
\\
بر هوا انداخت هر یک را گزاف
&&
تا همی‌زد بر زمین می‌شد شکاف
\\
ای خورندهٔ خون خلق از راه برد
&&
تا نه آرد خون ایشانت نبرد
\\
مال ایشان خون ایشان دان یقین
&&
زانک مال از زور آید در یمین
\\
مادر آن پیل‌بچگان کین کشد
&&
پیل بچه‌خواره را کیفر کشد
\\
پیل‌بچه می‌خوری ای پاره‌خوار
&&
هم بر آرد خصم پیل از تو دمار
\\
بوی رسوا کرد مکر اندیش را
&&
پیل داند بوی طفل خویش را
\\
آنک یابد بوی حق را از یمن
&&
چون نیابد بوی باطل را ز من
\\
مصطفی چون برد بوی از راه دور
&&
چون نیابد از دهان ما بخور
\\
هم بیابد لیک پوشاند ز ما
&&
بوی نیک و بد بر آید بر سما
\\
تو همی‌خسپی و بوی آن حرام
&&
می‌زند بر آسمان سبزفام
\\
همره انفاس زشتت می‌شود
&&
تا به بوگیران گردون می‌رود
\\
بوی کبر و بوی حرص و بوی آز
&&
در سخن گفتن بیاید چون پیاز
\\
گر خوری سوگند من کی خورده‌ام
&&
از پیاز و سیر تقوی کرده‌ام
\\
آن دم سوگند غمازی کند
&&
بر دماغ همنشینان بر زند
\\
پس دعاها رد شود از بوی آن
&&
آن دل کژ می‌نماید در زبان
\\
اخسؤا آید جواب آن دعا
&&
چوب رد باشد جزای هر دغا
\\
گر حدیثت کژ بود معنیت راست
&&
آن کژی لفظ مقبول خداست
\\
\end{longtable}
\end{center}
