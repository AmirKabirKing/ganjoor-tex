\begin{center}
\section*{بخش ۱۵۳ - ربودن عقاب موزهٔ مصطفی علیه السلام و بردن بر هوا و نگون کردن و از موزه مار سیاه فرو افتادن}
\label{sec:sh153}
\addcontentsline{toc}{section}{\nameref{sec:sh153}}
\begin{longtable}{l p{0.5cm} r}
اندرین بودند کآواز صلا
&&
مصطفی بشنید از سوی علا
\\
خواست آبی و وضو را تازه کرد
&&
دست و رو را شست او زان آب سرد
\\
هر دو پا شست و به موزه کرد رای
&&
موزه را بربود یک موزه‌ربای
\\
دست سوی موزه برد آن خوش‌خطاب
&&
موزه را بربود از دستش عقاب
\\
موزه را اندر هوا برد او چو باد
&&
پس نگون کرد و از آن ماری فتاد
\\
در فتاد از موزه یک مار سیاه
&&
زان عنایت شد عقابش نیکخواه
\\
پس عقاب آن موزه را آورد باز
&&
گفت هین بستان و رو سوی نماز
\\
از ضرورت کردم این گستاخیی
&&
من ز ادب دارم شکسته‌شاخیی
\\
وای کو گستاخ پایی می‌نهد
&&
بی ضرورت کش هوا فتوی دهد
\\
پس رسولش شکر کرد و گفت ما
&&
این جفا دیدیم و بود این خود وفا
\\
موزه بربودی و من درهم شدم
&&
تو غمم بردی و من در غم شدم
\\
گرچه هر غیبی خدا ما را نمود
&&
دل در آن لحظه به خود مشغول بود
\\
گفت دور از تو که غفلت در تو رست
&&
دیدنم آن غیب را هم عکس تست
\\
مار در موزه ببینم بر هوا
&&
نیست از من عکس تست ای مصطفی
\\
عکس نورانی همه روشن بود
&&
عکس ظلمانی همه گلخن بود
\\
عکس عبدالله همه نوری بود
&&
عکس بیگانه همه کوری بود
\\
عکس هر کس را بدان ای جان ببین
&&
پهلوی جنسی که خواهی می‌نشین
\\
\end{longtable}
\end{center}
