\begin{center}
\section*{بخش ۵۷ - بیان آنک علم را دو پرست و گمان را یک پرست ناقص آمد ظن به پرواز  ابترست مثال ظن و یقین در علم}
\label{sec:sh057}
\addcontentsline{toc}{section}{\nameref{sec:sh057}}
\begin{longtable}{l p{0.5cm} r}
علم را دو پر گمان را یک پرست
&&
ناقص آمد ظن به پرواز ابترست
\\
مرغ یک‌پر زود افتد سرنگون
&&
باز بر پرد دو گامی یا فزون
\\
افت خیزان می‌رود مرغ گمان
&&
با یکی پر بر امید آشیان
\\
چون ز ظن وا رست علمش رو نمود
&&
شد دو پر آن مرغ یک‌پر پر گشود
\\
بعد از آن یمشی سویا مستقیم
&&
نه علی وجهه مکبا او سقیم
\\
با دو پر بر می‌پرد چون جبرئیل
&&
بی گمان و بی مگر بی قال و قیل
\\
گر همه عالم بگویندش توی
&&
بر ره یزدان و دین مستوی
\\
او نگردد گرم‌تر از گفتشان
&&
جان طاق او نگردد جفتشان
\\
ور همه گویند او را گم‌رهی
&&
کوه پنداری و تو برگ کهی
\\
او نیفتد در گمان از طعنشان
&&
او نگردد دردمند از ظعنشان
\\
بلک گر دریا و کوه آید بگفت
&&
گویدش با گم‌رهی گشتی تو جفت
\\
هیچ یک ذره نیفتد در خیال
&&
یا به طعن طاعنان رنجورحال
\\
\end{longtable}
\end{center}
