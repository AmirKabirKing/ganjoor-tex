\begin{center}
\section*{بخش ۷۳ - متهم کردن آن شیخ را با دزدان وبریدن دستش را}
\label{sec:sh073}
\addcontentsline{toc}{section}{\nameref{sec:sh073}}
\begin{longtable}{l p{0.5cm} r}
بیست از دزدان بدند آنجا و بیش
&&
بخش می‌کردند مسروقات خویش
\\
شحنه را غماز آگه کرده بود
&&
مردم شحنه بر افتادند زود
\\
هم بدان‌جا پای چپ و دست راست
&&
جمله را ببرید و غوغایی بخاست
\\
دست زاهد هم بریده شد غلط
&&
پاش را می‌خواست هم کردن سقط
\\
در زمان آمد سواری بس گزین
&&
بانگ بر زد بر عوان کای سگ ببین
\\
این فلان شیخست از ابدال خدا
&&
دست او را تو چرا کردی جدا
\\
آن عوان بدرید جامه تیز رفت
&&
پیش شحنه داد آگاهیش تفت
\\
شحنه آمد پا برهنه عذرخواه
&&
که ندانستم خدا بر من گواه
\\
هین بحل کن مر مرا زین کار زشت
&&
ای کریم و سرور اهل بهشت
\\
گفت می‌دانم سبب این نیش را
&&
می‌شناسم من گناه خویش را
\\
من شکستم حرمت ایمان او
&&
پس یمینم برد دادستان او
\\
من شکستم عهد و دانستم بدست
&&
تا رسید آن شومی جرات بدست
\\
دست ما و پای ما و مغز و پوست
&&
باد ای والی فدای حکم دوست
\\
قسم من بود این ترا کردم حلال
&&
تو ندانستی ترا نبود وبال
\\
و آنک او دانست او فرمان‌رواست
&&
با خدا سامان پیچیدن کجاست
\\
ای بسا مرغی پریده دانه‌جو
&&
که بریده حلق او هم حلق او
\\
ای بسا مرغی ز معده وز مغص
&&
بر کنار بام محبوس قفس
\\
ای بسا ماهی در آب دوردست
&&
گشته از حرص گلو ماخوذ شست
\\
ای بسا مستور در پرده بده
&&
شومی فرج و گلو رسوا شده
\\
ای بسا قاضی حبر نیک‌خو
&&
از گلو و رشوتی او زردرو
\\
بلک در هاروت و ماروت آن شراب
&&
از عروج چرخشان شد سد باب
\\
با یزید از بهر این کرد احتراز
&&
دید در خود کاهلی اندر نماز
\\
از سبب اندیشه کرد آن ذو لباب
&&
دید علت خوردن بسیار از آب
\\
گفت تا سالی نخواهم خورد آب
&&
آنچنان کرد و خدایش داد تاب
\\
این کمینه جهد او بد بهر دین
&&
گشت او سلطان و قطب العارفین
\\
چون بریده شد برای حلق دست
&&
مرد زاهد را در شکوی ببست
\\
شیخ اقطع گشت نامش پیش خلق
&&
کرد معروفش بدین آفات حلق
\\
\end{longtable}
\end{center}
