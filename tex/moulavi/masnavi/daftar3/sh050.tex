\begin{center}
\section*{بخش ۵۰ - توفیق میان این دو حدیث کی الرضا بالکفر کفر و حدیث دیگر من لم یرض بقضایی  فلیطلب ربا سوای}
\label{sec:sh050}
\addcontentsline{toc}{section}{\nameref{sec:sh050}}
\begin{longtable}{l p{0.5cm} r}
دی سؤالی کرد سایل مر مرا
&&
زانک عاشق بود او بر ماجرا
\\
گفت نکتهٔ الرضا بالکفر کفر
&&
این پیمبر گفت و گفت اوست مهر
\\
باز فرمود او که اندر هر قضا
&&
مر مسلمان را رضا باید رضا
\\
نه قضای حق بود کفر و نفاق
&&
گر بدین راضی شوم باشد شقاق
\\
ور نیم راضی بود آن هم زیان
&&
پس چه چاره باشدم اندر میان
\\
گفتمش این کفر مقضی نه قضاست
&&
هست آثار قضا این کفر راست
\\
پس قضا را خواجه از مقضی بدان
&&
تا شکالت دفع گردد در زمان
\\
راضیم در کفر زان رو که قضاست
&&
نه ازین رو که نزاع و خبث ماست
\\
کفر از روی قضا خود کفر نیست
&&
حق را کافر مخوان اینجا مه‌ایست
\\
کفر جهلست و قضای کفر علم
&&
هر دو کی یک باشد آخر حلم و خلم
\\
زشتی خط زشتی نقاش نیست
&&
بلک از وی زشت را بنمودنیست
\\
قوت نقاش باشد آنک او
&&
هم تواند زشت کردن هم نکو
\\
گر کشانم بحث این را من بساز
&&
تا سؤال و تا جواب آید دراز
\\
ذوق نکتهٔ عشق از من می‌رود
&&
نقش خدمت نقش دیگر می‌شود
\\
\end{longtable}
\end{center}
