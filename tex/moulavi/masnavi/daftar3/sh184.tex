\begin{center}
\section*{بخش ۱۸۴ - لاابالی گفتن عاشق ناصح و عاذل را از سر عشق}
\label{sec:sh184}
\addcontentsline{toc}{section}{\nameref{sec:sh184}}
\begin{longtable}{l p{0.5cm} r}
گفت ای ناصح خمش کن چند چند
&&
پند کم ده زانک بس سختست بند
\\
سخت‌تر شد بند من از پند تو
&&
عشق را نشناخت دانشمند تو
\\
آن طرف که عشق می‌افزود درد
&&
بوحنیفه و شافعی درسی نکرد
\\
تو مکن تهدید از کشتن که من
&&
تشنهٔ زارم به خون خویشتن
\\
عاشقان را هر زمانی مردنیست
&&
مردن عشاق خود یک نوع نیست
\\
او دو صد جان دارد از جان هدی
&&
وآن دوصد را می‌کند هر دم فدی
\\
هر یکی جان را ستاند ده بها
&&
از نبی خوان عشرة امثالها
\\
گر بریزد خون من آن دوست‌رو
&&
پای‌کوبان جان برافشانم برو
\\
آزمودم مرگ من در زندگیست
&&
چون رهم زین زندگی پایندگیست
\\
اقتلونی اقتلونی یا ثقات
&&
ان فی قتلی حیاتا فی حیات
\\
یا منیر الخد یا روح البقا
&&
اجتذب روحی وجد لی باللقا
\\
لی حبیب حبه یشوی الحشا
&&
لو یشا یمشی علی عینی مشی
\\
پارسی گو گرچه تازی خوشترست
&&
عشق را خود صد زبان دیگرست
\\
بوی آن دلبر چو پران می‌شود
&&
آن زبانها جمله حیران می‌شود
\\
بس کنم دلبر در آمد در خطاب
&&
گوش شو والله اعلم بالصواب
\\
چونک عاشق توبه کرد اکنون بترس
&&
کو چو عیاران کند بر دار درس
\\
گرچه این عاشق بخارا می‌رود
&&
نه به درس و نه به استا می‌رود
\\
عاشقان را شد مدرس حسن دوست
&&
دفتر و درس و سبقشان روی اوست
\\
خامشند و نعرهٔ تکرارشان
&&
می‌رود تا عرش و تخت یارشان
\\
درسشان آشوب و چرخ و زلزله
&&
نه زیاداتست و باب سلسله
\\
سلسلهٔ این قوم جعد مشکبار
&&
مسلهٔ دورست لیکن دور یار
\\
مسلهٔ کیس ار بپرسد کس ترا
&&
گو نگنجد گنج حق در کیسه‌ها
\\
گر دم خلع و مبارا می‌رود
&&
بد مبین ذکر بخارا می‌رود
\\
ذکر هر چیزی دهد خاصیتی
&&
زانک دارد هرصفت ماهیتی
\\
آن بخاری غصهٔ دانش نداشت
&&
چشم بر خورشید بینش می‌گماشت
\\
هرکه درخلوت ببینش یافت راه
&&
او ز دانشها نجوید دستگاه
\\
با جمال جان چوشد هم‌کاسه‌ای
&&
باشدش ز اخبار و دانش تاسه‌ای
\\
دید بردانش بود غالب فرا
&&
زان همی دنیا بچربد عامه را
\\
زانک دنیا را همی‌بینند عین
&&
وآن جهانی را همی‌دانند دین
\\
\end{longtable}
\end{center}
