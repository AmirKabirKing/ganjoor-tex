\begin{center}
\section*{بخش ۴۶ - جواب گفتن ساحر مرده با فرزندان خود}
\label{sec:sh046}
\addcontentsline{toc}{section}{\nameref{sec:sh046}}
\begin{longtable}{l p{0.5cm} r}
گفتشان در خواب کای اولاد من
&&
نیست ممکن ظاهر این را دم زدن
\\
فاش و مطلق گفتنم دستور نیست
&&
لیک راز از پیش چشمم دور نیست
\\
لیک بنمایم نشانی با شما
&&
تا شود پیدا شما را این خفا
\\
نور چشمانم چو آنجا گه روید
&&
از مقام خفتنش آگه شوید
\\
آن زمان که خفته باشد آن حکیم
&&
آن عصا را قصد کن بگذار بیم
\\
گر بدزدی و توانی ساحرست
&&
چارهٔ ساحر بر تو حاضرست
\\
ور نتانی هان و هان آن ایزدیست
&&
او رسول ذوالجلال و مهتدیست
\\
گر جهان فرعون گیرد شرق و غرب
&&
سرنگون آید خدا آنگاه حرب
\\
این نشان راست دادم جان باب
&&
بر نویس الله اعلم بالصواب
\\
جان بابا چون بخسپد ساحری
&&
سحر و مکرش را نباشد رهبری
\\
چونک چوپان خفت گرگ آمن شود
&&
چونک خفت آن جهد او ساکن شود
\\
لیک حیوانی که چوپانش خداست
&&
گرگ را آنجا امید و ره کجاست
\\
جادوی که حق کند حقست و راست
&&
جادوی خواندن مر آن حق را خطاست
\\
جان بابا این نشان قاطعست
&&
گر بمیرد نیز حقش رافعست
\\
\end{longtable}
\end{center}
