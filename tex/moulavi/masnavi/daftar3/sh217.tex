\begin{center}
\section*{بخش ۲۱۷ - تفسیر این آیت کی ان تستفتحوا فقد جائکم الفتح ایه‌ای طاعنان می‌گفتید کی از ما و محمد علیه السلام آنک حق است فتح و نصرتش ده و این بدان می‌گفتید تا گمان آید کی شما طالب حق‌اید بی غرض اکنون محمد را نصرت دادیم تا صاحب حق را ببینید}
\label{sec:sh217}
\addcontentsline{toc}{section}{\nameref{sec:sh217}}
\begin{longtable}{l p{0.5cm} r}
از بتان و از خدا در خواستیم
&&
که بکن ما را اگر ناراستیم
\\
آنک حق و راستست از ما و او
&&
نصرتش ده نصرت او را بجو
\\
این دعا بسیار کردیم و صلات
&&
پیش لات و پیش عزی و منات
\\
که اگر حقست او پیداش کن
&&
ور نباشد حق زبون ماش کن
\\
چونک وا دیدیم او منصور بود
&&
ما همه ظلمت بدیم او نور بود
\\
این جواب ماست کانچ خواستید
&&
گشت پیدا که شما ناراستید
\\
باز این اندیشه را از فکر خویش
&&
کور می‌کردند و دفع از ذکر خویش
\\
کین تفکرمان هم از ادبار رست
&&
که صواب او شود در دل درست
\\
خود چه شد گر غالب آمد چند بار
&&
هر کسی را غالب آرد روزگار
\\
ما هم از ایام بخت‌آور شدیم
&&
بارها بر وی مظفر آمدیم
\\
باز گفتندی که گرچه او شکست
&&
چون شکست ما نبود آن زشت و پست
\\
زانک بخت نیک او را در شکست
&&
داد صد شادی پنهان زیردست
\\
کو باشکسته نمی‌مانست هیچ
&&
که نه غم بودش در آن نه پیچ پیچ
\\
چون نشان مؤمنان مغلوبیست
&&
لیک در اشکست مؤمن خوبیست
\\
گر تو مشک و عنبری را بشکنی
&&
عالمی از فوح ریحان پر کنی
\\
ور شکستی ناگهان سرگین خر
&&
خانه‌ها پر گند گردد تا به سر
\\
وقت واگشت حدیبیه بذل
&&
دولت انا فتحنا زد دهل
\\
\end{longtable}
\end{center}
