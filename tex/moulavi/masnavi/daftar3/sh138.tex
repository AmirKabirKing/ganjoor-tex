\begin{center}
\section*{بخش ۱۳۸ - باز جواب انبیا علیهم السلام}
\label{sec:sh138}
\addcontentsline{toc}{section}{\nameref{sec:sh138}}
\begin{longtable}{l p{0.5cm} r}
انبیا گفتند فال زشت و بد
&&
از میان جانتان دارد مدد
\\
گر تو جایی خفته باشی با خطر
&&
اژدها در قصد تو از سوی سر
\\
مهربانی مر ترا آگاه کرد
&&
که بجه زود ار نه اژدرهات خورد
\\
تو بگویی فال بد چون می‌زنی
&&
فال چه بر جه ببین در روشنی
\\
از میان فال بد من خود ترا
&&
می‌رهانم می‌برم سوی سرا
\\
چون نبی آگه کننده‌ست از نهان
&&
کو بدید آنچ ندید اهل جهان
\\
گر طبیبی گویدت غوره مخور
&&
که چنین رنجی بر آرد شور و شر
\\
تو بگویی فال بد چون می‌زنی
&&
پس تو ناصح را مثم می‌کنی
\\
ور منجم گویدت کامروز هیچ
&&
آنچنان کاری مکن اندر پسیچ
\\
صد ره ار بینی دروغ اختری
&&
یک دوباره راست آید می‌خری
\\
این نجوم ما نشد هرگز خلاف
&&
صحتش چون ماند از تو در غلاف
\\
آن طبیب و آن منجم از گمان
&&
می‌کنند آگاه و ما خود از عیان
\\
دود می‌بینیم و آتش از کران
&&
حمله می‌آرد به سوی منکران
\\
تو همی‌گویی خمش کن زین مقال
&&
که زیان ماست قال شوم‌فال
\\
ای که نصح ناصحان را نشنوی
&&
فال بد با تست هر جا می‌روی
\\
افعیی بر پشت تو بر می‌رود
&&
او ز بامی بیندش آگه کند
\\
گوییش خاموش غمگینم مکن
&&
گوید او خوش باش خود رفت آن سخن
\\
چون زند افعی دهان بر گردنت
&&
تلخ گردد جمله شادی جستنت
\\
پس بدو گویی همین بود ای فلان
&&
چون بندریدی گریبان در فغان
\\
یا ز بالایم تو سنگی می‌زدی
&&
تا مرا آن جد نمودی و بدی
\\
او بگوید زآنک می‌آزرده‌ای
&&
تو بگویی نیک شادم کرده‌ای
\\
گفت من کردم جوامردی بپند
&&
تا رهانم من ترا زین خشک بند
\\
از لئیمی حق آن نشناختی
&&
مایهٔ ایذا و طغیان ساختی
\\
این بود خوی لئیمان دنی
&&
بد کند با تو چو نیکویی کنی
\\
نفس را زین صبر می‌کن منحنیش
&&
که لئیمست و نسازد نیکویش
\\
با کریمی گر کنی احسان سزد
&&
مر یکی را او عوض هفصد دهد
\\
با لئیمی چون کنی قهر و جفا
&&
بنده‌ای گردد ترا بس با وفا
\\
کافران کارند در نعمت جفا
&&
باز در دوزخ نداشان ربنا
\\
\end{longtable}
\end{center}
