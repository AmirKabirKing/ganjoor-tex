\begin{center}
\section*{بخش ۱۱ - باقی قصهٔ اهل سبا}
\label{sec:sh011}
\addcontentsline{toc}{section}{\nameref{sec:sh011}}
\begin{longtable}{l p{0.5cm} r}
آن سبا ز اهل صبا بودند و خام
&&
کارشان کفران نعمت با کرام
\\
باشد آن کفران نعمت در مثال
&&
که کنی با محسن خود تو جدال
\\
که نمی‌باید مرا این نیکوی
&&
من برنجم زین چه رنجم می‌شوی
\\
لطف کن این نیکوی را دور کن
&&
من نخواهم چشم زودم کور کن
\\
پس سبا گفتند باعد بیننا
&&
شیننا خیر لنا خذ زیننا
\\
ما نمی‌خواهیم این ایوان و باغ
&&
نه زنان خوب و نه امن و فراغ
\\
شهرها نزدیک همدیگر بدست
&&
آن بیابانست خوش کانجا ددست
\\
یطلب الانسان فی الصیف الشتا
&&
فاذا جاء الشتا انکر ذا
\\
فهو لا یرضی بحال ابدا
&&
لا بضیق لا بعیش رغدا
\\
قتل الانسان ما اکفره
&&
کلما نال هدی انکره
\\
نفس زین سانست زان شد کشتنی
&&
اقتلوا انفسکم گفت آن سنی
\\
خار سه سویست هر چون کش نهی
&&
در خلد وز زخم او تو کی جهی
\\
آتش ترک هوا در خار زن
&&
دست اندر یار نیکوکار زن
\\
چون ز حد بردند اصحاب سبا
&&
که بپیش ما وبا به از صبا
\\
ناصحانشان در نصیحت آمدند
&&
از فسوق و کفر مانع می‌شدند
\\
قصد خون ناصحان می‌داشتند
&&
تخم فسق و کافری می‌کاشتند
\\
چون قضا آید شود تنگ این جهان
&&
از قضا حلوا شود رنج دهان
\\
گفت اذا جاء القضا ضاق الفضا
&&
تحجب الابصار اذ جاء القضا
\\
چشم بسته می‌شود وقت قضا
&&
تا نبیند چشم کحل چشم را
\\
مکر آن فارس چو انگیزید گرد
&&
آن غبارت ز استغاثت دور کرد
\\
سوی فارس رو مرو سوی غبار
&&
ورنه بر تو کوبد آن مکر سوار
\\
گفت حق آن را که این گرگش بخورد
&&
دید گرد گرگ چون زاری نکرد
\\
او نمی‌دانست گرد گرگ را
&&
با چنین دانش چرا کرد او چرا
\\
گوسفندان بوی گرگ با گزند
&&
می‌بدانند و بهر سو می‌خزند
\\
مغز حیوانات بوی شیر را
&&
می‌بداند ترک می‌گوید چرا
\\
بوی شیر خشم دیدی باز گرد
&&
با مناجات و حذر انباز گرد
\\
وا نگشتند آن گروه از گرد گرگ
&&
گرگ محنت بعد گرد آمد سترگ
\\
بر درید آن گوسفندان را بخشم
&&
که ز چوپان خرد بستند چشم
\\
چند چوپانشان بخواند و نامدند
&&
خاک غم در چشم چوپان می‌زدند
\\
که برو ما از تو خود چوپان‌تریم
&&
چون تبع گردیم هر یک سروریم
\\
طعمهٔ گرگیم و آن یار نه
&&
هیزم ناریم و آن عار نه
\\
حمیتی بد جاهلیت در دماغ
&&
بانگ شومی بر دمنشان کرد زاغ
\\
بهر مظلومان همی‌کندند چاه
&&
در چه افتادند و می‌گفتند آه
\\
پوستین یوسفان بکشافتند
&&
آنچ می‌کردند یک یک یافتند
\\
کیست آن یوسف دل حق‌جوی تو
&&
چون اسیری بسته اندر کوی تو
\\
جبرئیلی را بر استن بسته‌ای
&&
پر و بالش را به صد جا خسته‌ای
\\
پیش او گوساله بریان آوری
&&
گه کشی او را به کهدان آوری
\\
که بخور اینست ما را لوت و پوت
&&
نیست او را جز لقاء الله قوت
\\
زین شکنجه و امتحان آن مبتلا
&&
می‌کند از تو شکایت با خدا
\\
کای خدا افغان ازین گرگ کهن
&&
گویدش نک وقت آمد صبر کن
\\
داد تو وا خواهم از هر بی‌خبر
&&
داد کی دهد جز خدای دادگر
\\
او همی‌گوید که صبرم شد فنا
&&
در فراق روی تو یا ربنا
\\
احمدم در مانده در دست یهود
&&
صالحم افتاده در حبس ثمود
\\
ای سعادت‌بخش جان انبیا
&&
یا بکش یا باز خوانم یا بیا
\\
با فراقت کافران را نیست تاب
&&
می‌گود یا لیتنی کنت تراب
\\
حال او اینست کو خود زان سوست
&&
چون بود بی تو کسی کان توست
\\
حق همی‌گوید که آری ای نزه
&&
لیک بشنو صبر آر و صبر به
\\
صبح نزدیکست خامش کم خروش
&&
من همی‌کوشم پی تو تو مکوش
\\
\end{longtable}
\end{center}
