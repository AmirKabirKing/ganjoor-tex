\begin{center}
\section*{بخش ۱۳۲ - حکایت نذر کردن سگان هر زمستان کی این تابستان چون بیاید خانه سازیم از بهر زمستان را}
\label{sec:sh132}
\addcontentsline{toc}{section}{\nameref{sec:sh132}}
\begin{longtable}{l p{0.5cm} r}
سگ زمستان جمع گردد استخوانش
&&
زخم سرما خرد گرداند چنانش
\\
کو بگوید کین قدر تن که منم
&&
خانه‌ای از سنگ باید کردنم
\\
چونک تابستان بیاید من بچنگ
&&
بهر سرما خانه‌ای سازم ز سنگ
\\
چونک تابستان بیاید از گشاد
&&
استخوانها پهن گردد پوست شاد
\\
گوید او چون زفت بیند خویش را
&&
در کدامین خانه گنجم ای کیا
\\
زفت گردد پا کشد در سایه‌ای
&&
کاهلی سیری غری خودرایه‌ای
\\
گویدش دل خانه‌ای ساز ای عمو
&&
گوید او در خانه کی گنجم بگو
\\
استخوان حرص تو در وقت درد
&&
درهم آید خرد گردد در نورد
\\
گویی از توبه بسازم خانه‌ای
&&
در زمستان باشدم استانه‌ای
\\
چون بشد درد و شدت آن حرص زفت
&&
همچو سگ سودای خانه از تو رفت
\\
شکر نعمت خوشتر از نعمت بود
&&
شکرباره کی سوی نعمت رود
\\
شکر جان نعمت و نعمت چو پوست
&&
ز آنک شکر آرد ترا تا کوی دوست
\\
نعمت آرد غفلت و شکر انتباه
&&
صید نعمت کن بدام شکر شاه
\\
نعمت شکرت کند پرچشم و میر
&&
تا کنی صد نعمت ایثار فقیر
\\
سیر نوشی از طعام و نقل حق
&&
تا رود از تو شکم‌خواری و دق
\\
\end{longtable}
\end{center}
