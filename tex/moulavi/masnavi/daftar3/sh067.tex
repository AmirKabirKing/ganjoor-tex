\begin{center}
\section*{بخش ۶۷ - در بیان آنک تن روح را چون لباسی است و این دست آستین دست روحست واین پای  موزهٔ پای روحست}
\label{sec:sh067}
\addcontentsline{toc}{section}{\nameref{sec:sh067}}
\begin{longtable}{l p{0.5cm} r}
تا بدانی که تن آمد چون لباس
&&
رو بجو لابس لباسی را ملیس
\\
روح را توحید الله خوشترست
&&
غیر ظاهر دست و پای دیگرست
\\
دست و پا در خواب بینی و ایتلاف
&&
آن حقیقت دان مدانش از گزاف
\\
آن توی که بی بدن داری بدن
&&
پس مترس از جسم و جان بیرون شدن
\\
\end{longtable}
\end{center}
