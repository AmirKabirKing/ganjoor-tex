\begin{center}
\section*{بخش ۱۴۳ - حکایت امیر و غلامش کی نماز باره بود وانس عظیم داشت در نماز و مناجات با حق}
\label{sec:sh143}
\addcontentsline{toc}{section}{\nameref{sec:sh143}}
\begin{longtable}{l p{0.5cm} r}
میرشد محتاج گرمابه سحر
&&
بانگ زد سنقر هلا بردار سر
\\
طاس و مندیل و گل از التون بگیر
&&
تابه گرمابه رویم ای ناگزیر
\\
سنقر آن دم طاس و مندیلی نکو
&&
برگرفت و رفت با او دو بدو
\\
مسجدی بر ره بد و بانگ صلا
&&
آمد اندر گوش سنقر در ملا
\\
بود سنقر سخت مولع در نماز
&&
گفت ای میر من ای بنده‌نواز
\\
تو برین دکان زمانی صبرکن
&&
تا گزارم فرض و خوانم لم یکن
\\
چون امام و قوم بیرون آمدند
&&
ازنماز و وردها فارغ شدند
\\
سنقر آنجا ماند تا نزدیک چاشت
&&
میر سنقر را زمانی چشم داشت
\\
گفت ای سنقر چرا نایی برون
&&
گفت می‌نگذاردم این ذو فنون
\\
صبر کن نک آمدم ای روشنی
&&
نیستم غافل که در گوش منی
\\
هفت نوبت صبر کرد و بانگ کرد
&&
تاکه عاجز گشت از تیباش مرد
\\
پاسخش این بود می‌نگذاردم
&&
تا برون آیم هنوز ای محترم
\\
گفت آخر مسجد اندر کس نماند
&&
کیت وا می‌دارد آنجا کت نشاند
\\
گفت آنک بسته‌استت از برون
&&
بسته است او هم مرا در اندرون
\\
آنک نگذارد ترا کایی درون
&&
می‌بنگذارد مرا کایم برون
\\
آنک نگذارد کزین سو پا نهی
&&
او بدین سو بست پای این رهی
\\
ماهیان را بحر نگذارد برون
&&
خاکیان را بحر نگذارد درون
\\
اصل ماهی آب و حیوان از گلست
&&
حیله و تدبیر اینجا باطلست
\\
قفل زفتست و گشاینده خدا
&&
دست در تسلیم زن واندر رضا
\\
ذره ذره گر شود مفتاحها
&&
این گشایش نیست جز از کبریا
\\
چون فراموشت شود تدبیر خویش
&&
یابی آن بخت جوان از پیر خویش
\\
چون فراموش خودی یادت کنند
&&
بنده گشتی آنگه آزادت کنند
\\
\end{longtable}
\end{center}
