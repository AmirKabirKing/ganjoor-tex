\begin{center}
\section*{بخش ۱۰۰ - شنیدن دقوقی در میان نماز افغان  آن کشتی کی غرق خواست شدن}
\label{sec:sh100}
\addcontentsline{toc}{section}{\nameref{sec:sh100}}
\begin{longtable}{l p{0.5cm} r}
آن دقوقی در امامت کرد ساز
&&
اندر آن ساحل در آمد در نماز
\\
و آن جماعت در پی او در قیام
&&
اینت زیبا قوم و بگزیده امام
\\
ناگهان چشمش سوی دریا فتاد
&&
چون شنید از سوی دریا داد داد
\\
در میان موج دید او کشتیی
&&
در قضا و در بلا و زشتیی
\\
هم شب و هم ابر و هم موج عظیم
&&
این سه تاریکی و از غرقاب بیم
\\
تند بادی همچو عزرائیل خاست
&&
موجها آشوفت اندر چپ و راست
\\
اهل کشتی از مهابت کاسته
&&
نعره وا ویلها برخاسته
\\
دستها در نوحه بر سر می‌زدند
&&
کافر و ملحد همه مخلص شدند
\\
با خدا با صد تضرع آن زمان
&&
عهدها و نذرها کرده بجان
\\
سر برهنه در سجود آنها که هیچ
&&
رویشان قبله ندید از پیچ پیچ
\\
گفته که بی‌فایده‌ست این بندگی
&&
آن زمان دیده در آن صد زندگی
\\
از همه اومید ببریده تمام
&&
دوستان و خال و عم بابا و مام
\\
زاهد و فاسق شد آن دم متقی
&&
همچو در هنگام جان کندن شقی
\\
نه ز چپشان چاره بود و نه ز راست
&&
حیله‌ها چون مرد هنگام دعاست
\\
در دعا ایشان و در زاری و آه
&&
بر فلک زیشان شده دود سیاه
\\
دیو آن دم از عداوت بین بین
&&
بانگ زد کای سگ‌پرستان علتین
\\
مرگ و جسک ای اهل انکار و نفاق
&&
عاقبت خواهد بدن این اتفاق
\\
چشمتان تر باشد از بعد خلاص
&&
که شوید از بهر شهوت دیو خاص
\\
یادتان ناید که روزی در خطر
&&
دستتان بگرفت یزدان از قدر
\\
این همی‌آمد ندا از دیو لیک
&&
این سخن را نشنود جز گوش نیک
\\
راست فرمودست با ما مصطفی
&&
قطب و شاهنشاه و دریای صفا
\\
کانچ جاهل دید خواهد عاقبت
&&
عاقلان بینند ز اول مرتبت
\\
کارها ز آغاز اگر غیبست و سر
&&
عاقل اول دید و آخر آن مصر
\\
اولش پوشیده باشد و آخر آن
&&
عاقل و جاهل ببیند در عیان
\\
گر نبینی واقعهٔ غیب ای عنود
&&
حزم را سیلاب کی اندر ربود
\\
حزم چه بود بدگمانی بر جهان
&&
دم بدم بیند بلای ناگهان
\\
\end{longtable}
\end{center}
