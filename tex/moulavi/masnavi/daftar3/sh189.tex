\begin{center}
\section*{بخش ۱۸۹ - صفت آن مسجد کی عاشق‌کش بود و آن عاشق مرگ‌جوی لا ابالی کی درو مهمان شد}
\label{sec:sh189}
\addcontentsline{toc}{section}{\nameref{sec:sh189}}
\begin{longtable}{l p{0.5cm} r}
یک حکایت گوش کن ای نیک‌پی
&&
مسجدی بد بر کنار شهر ری
\\
هیچ کس در وی نخفتی شب ز بیم
&&
که نه فرزندش شدی آن شب یتیم
\\
بس که اندر وی غریب عور رفت
&&
صبحدم چون اختران در گور رفت
\\
خویشتن را نیک ازین آگاه کن
&&
صبح آمد خواب را کوتاه کن
\\
هر کسی گفتی که پریانند تند
&&
اندرو مهمان کشان با تیغ کند
\\
آن دگر گفتی که سحرست و طلسم
&&
کین رصد باشد عدو جان و خصم
\\
آن دگر گفتی که بر نه نقش فاش
&&
بر درش کای میهمان اینجا مباش
\\
شب مخسپ اینجا اگر جان بایدت
&&
ورنه مرگ اینجا کمین بگشایدت
\\
وان یکی گفتی که شب قفلی نهید
&&
غافلی کاید شما کم ره دهید
\\
\end{longtable}
\end{center}
