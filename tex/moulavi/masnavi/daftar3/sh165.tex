\begin{center}
\section*{بخش ۱۶۵ - در آمدن حمزه رضی الله عنه در جنگ بی زره}
\label{sec:sh165}
\addcontentsline{toc}{section}{\nameref{sec:sh165}}
\begin{longtable}{l p{0.5cm} r}
اندر آخر حمزه چون در صف شدی
&&
بی زره سرمست در غزو آمدی
\\
سینه باز و تن برهنه پیش پیش
&&
در فکندی در صف شمشیر خویش
\\
خلق پرسیدند کای عم رسول
&&
ای هزبر صف‌شکن شاه فحول
\\
نه تو لا تلقوا بایدیکم الی
&&
تهلکه خواندی ز پیغام خدا
\\
پس چرا تو خویش را در تهلکه
&&
می در اندازی چنین در معرکه
\\
چون جوان بودی و زفت و سخت‌زه
&&
تو نمی‌رفتی سوی صف بی زره
\\
چون شدی پیر و ضعیف و منحنی
&&
پرده‌های لا ابالی می‌زنی
\\
لا ابالی‌وار با تیغ و سنان
&&
می‌نمایی دار و گیر و امتحان
\\
تیغ حرمت می‌ندارد پیر را
&&
کی بود تمییز تیغ و تیر را
\\
زین نسق غمخوارگان بی‌خبر
&&
پند می‌دادند او را از غیر
\\
\end{longtable}
\end{center}
