\begin{center}
\section*{بخش ۲۱۸ - سر آنک بی‌مراد بازگشتن رسول علیه السلام از حدیبیه حق تعالی لقب آن فتح کرد کی انا فتحنا کی به صورت غلق بود و به معنی فتح چنانک شکستن مشک به ظاهر شکستن است و به معنی درست کردنست مشکی او را و تکمیل فواید اوست}
\label{sec:sh218}
\addcontentsline{toc}{section}{\nameref{sec:sh218}}
\begin{longtable}{l p{0.5cm} r}
آمدش پیغام از دولت که رو
&&
تو ز منع این ظفر غمگین مشو
\\
کاندرین خواری نقدت فتحهاست
&&
نک فلان قلعه فلان بقعه تراست
\\
بنگر آخر چونک واگردید تفت
&&
بر قریظه و بر نضیر از وی چه رفت
\\
قلعه‌ها هم گرد آن دو بقعه‌ها
&&
شد مسلم وز غنایم نفعها
\\
ور نباشد آن تو بنگر کین فریق
&&
پر غم و رنجند و مفتون و عشیق
\\
زهر خواری را چو شکر می‌خورند
&&
خار غمها را چو اشتر می‌چرند
\\
بهر عین غم نه از بهر فرج
&&
این تسافل پیش ایشان چون درج
\\
آنچنان شادند اندر قعر چاه
&&
که همی‌ترسند از تخت و کلاه
\\
هر کجا دلبر بود خود همنشین
&&
فوق گردونست نه زیر زمین
\\
\end{longtable}
\end{center}
