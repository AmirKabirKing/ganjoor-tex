\begin{center}
\section*{بخش ۲۲۶ - با خویش آمدن عاشق بیهوش و روی آوردن به ثنا و شکر معشوق}
\label{sec:sh226}
\addcontentsline{toc}{section}{\nameref{sec:sh226}}
\begin{longtable}{l p{0.5cm} r}
گفت ای عنقای حق جان را مطاف
&&
شکر که باز آمدی زان کوه قاف
\\
ای سرافیل قیامتگاه عشق
&&
ای تو عشق عشق و ای دلخواه عشق
\\
اولین خلعت که خواهی دادنم
&&
گوش خواهم که نهی بر روزنم
\\
گرچه می‌دانی بصفوت حال من
&&
بنده‌پرور گوش کن اقوال من
\\
صد هزاران بار ای صدر فرید
&&
ز آرزوی گوش تو هوشم پرید
\\
آن سمیعی تو وان اصغای تو
&&
و آن تبسمهای جان‌افزای تو
\\
آن بنوشیدن کم و بیش مرا
&&
عشوهٔ جان بداندیش مرا
\\
قلبهای من که آن معلوم تست
&&
بس پذیرفتی تو چون نقد درست
\\
بهر گستاخی شوخ غره‌ای
&&
حلمها در پیش حلمت ذره‌ای
\\
اولا بشنو که چون ماندم ز شست
&&
اول و آخر ز پیش من بجست
\\
ثانیا بشنو تو ای صدر ودود
&&
که بسی جستم ترا ثانی نبود
\\
ثالثا تا از تو بیرون رفته‌ام
&&
گوییا ثالث ثلاثه گفته‌ام
\\
رابعا چون سوخت ما را مزرعه
&&
می ندانم خامسه از رابعه
\\
هر کجا یابی تو خون بر خاکها
&&
پی بری باشد یقین از چشم ما
\\
گفت من رعدست و این بانگ و حنین
&&
ز ابر خواهد تا ببارد بر زمین
\\
من میان گفت و گریه می‌تنم
&&
یا بگریم یا بگویم چون کنم
\\
گر بگویم فوت می‌گردد بکا
&&
ور نگویم چون کنم شکر و ثنا
\\
می‌فتد از دیده خون دل شها
&&
بین چه افتادست از دیده مرا
\\
این بگفت و گریه در شد آن نحیف
&&
که برو بگریست هم دون هم شریف
\\
از دلش چندان بر آمد های هوی
&&
حلقه کرد اهل بخارا گرد اوی
\\
خیره گویان خیره گریان خیره‌خند
&&
مرد و زن خرد و کلان حیران شدند
\\
شهر هم هم‌رنگ او شد اشک ریز
&&
مرد و زن درهم شده چون رستخیز
\\
آسمان می‌گفت آن دم با زمین
&&
گر قیامت را ندیدستی ببین
\\
عقل حیران که چه عشق است و چه حال
&&
تا فراق او عجب‌تر یا وصال
\\
چرخ بر خوانده قیامت‌نامه را
&&
تا مجره بر دریده جامه را
\\
با دو عالم عشق را بیگانگی
&&
اندرو هفتاد و دو دیوانگی
\\
سخت پنهانست و پیدا حیرتش
&&
جان سلطانان جان در حسرتش
\\
غیر هفتاد و دو ملت کیش او
&&
تخت شاهان تخته‌بندی پیش او
\\
مطرب عشق این زند وقت سماع
&&
بندگی بند و خداوندی صداع
\\
پس چه باشد عشق دریای عدم
&&
در شکسته عقل را آنجا قدم
\\
بندگی و سلطنت معلوم شد
&&
زین دو پرده عاشقی مکتوم شد
\\
کاشکی هستی زبانی داشتی
&&
تا ز هستان پرده‌ها برداشتی
\\
هر چه گویی ای دم هستی از آن
&&
پردهٔ دیگر برو بستی بدان
\\
آفت ادراک آن قالست و حال
&&
خون بخون شستن محالست و محال
\\
من چو با سوداییانش محرمم
&&
روز و شب اندر قفس در می‌دمم
\\
سخت مست و بی‌خود و آشفته‌ای
&&
دوش ای جان بر چه پهلو خفته‌ای
\\
هان و هان هش دار بر ناری دمی
&&
اولا بر جه طلب کن محرمی
\\
عاشق و مستی و بگشاده زبان
&&
الله الله اشتری بر ناودان
\\
چون ز راز و ناز او گوید زبان
&&
یا جمیل الستر خواند آسمان
\\
ستر چه در پشم و پنبه آذرست
&&
تا همی‌پوشیش او پیداترست
\\
چون بکوشم تا سرش پنهان کنم
&&
سر بر آرد چون علم کاینک منم
\\
رغم انفم گیردم او هر دو گوش
&&
کای مدمغ چونش می‌پوشی بپوش
\\
گویمش رو گرچه بر جوشیده‌ای
&&
همچو جان پیدایی و پوشیده‌ای
\\
گوید او محبوس خنبست این تنم
&&
چون می اندر بزم خنبک می‌زنم
\\
گویمش زان پیش که گردی گرو
&&
تا نیاید آفت مستی برو
\\
گوید از جام لطیف‌آشام من
&&
یار روزم تا نماز شام من
\\
چون بیاید شام و دزدد جام من
&&
گویمش وا ده که نامد شام من
\\
زان عرب بنهاد نام می مدام
&&
زانک سیری نیست می‌خور را مدام
\\
عشق جوشد بادهٔ تحقیق را
&&
او بود ساقی نهان صدیق را
\\
چون بجویی تو بتوفیق حسن
&&
باده آب جان بود ابریق تن
\\
چون بیفزاید می توفیق را
&&
قوت می بشکند ابریق را
\\
آب گردد ساقی و هم مست آب
&&
چون مگو والله اعلم بالصواب
\\
پرتو ساقیست کاندر شیره رفت
&&
شیره بر جوشید و رقصان گشت و زفت
\\
اندرین معنی بپرس آن خیره را
&&
که چنین کی دیده بودی شیره را
\\
بی تفکر پیش هر داننده هست
&&
آنک با شوریده شوراننده هست
\\
\end{longtable}
\end{center}
