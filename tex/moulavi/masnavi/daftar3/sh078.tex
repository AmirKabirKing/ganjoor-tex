\begin{center}
\section*{بخش ۷۸ - جزع ناکردن شیخی بر مرگ فرزندان خود}
\label{sec:sh078}
\addcontentsline{toc}{section}{\nameref{sec:sh078}}
\begin{longtable}{l p{0.5cm} r}
بود شیخی رهنمایی پیش ازین
&&
آسمانی شمع بر روی زمین
\\
چون پیمبر درمیان امتان
&&
در گشای روضهٔ دار الجنان
\\
گفت پیغامبر که شیخ رفته پیش
&&
چون نبی باشد میان قوم خویش
\\
یک صباحی گفتش اهل بیت او
&&
سخت‌دل چونی بگو ای نیک‌خو
\\
ماز مرگ و هجر فرزندان تو
&&
نوحه می‌داریم با پشت دوتو
\\
تو نمی‌گریی نمی‌زاری چرا
&&
یا که رحمت نیست در دل ای کیا
\\
چون ترا رحمی نباشد در درون
&&
پس چه اومیدست‌مان از تو کنون
\\
ما باومید تویم این پیش‌وا
&&
که بنگذاری توما را در فنا
\\
چون بیارایند روز حشر تخت
&&
خود شفیع ما توی آن روز سخت
\\
درچنان روز و شب بی‌زینهار
&&
ما به اکرام تویم اومیدوار
\\
دست ما و دامن تست آن زمان
&&
که نماند هیچ مجرم را امان
\\
گفت پیغامبر که روز رستخیز
&&
کی گذارم مجرمان را اشک‌ریز
\\
من شفیع عاصیان باشم بجان
&&
تا رهانمشان ز اشکنجهٔ گران
\\
عاصیان واهل کبایر را بجهد
&&
وا رهانم از عتاب نقض عهد
\\
صالحان امتم خود فارغ‌اند
&&
از شفاعتهای من روز گزند
\\
بلک ایشان را شفاعتها بود
&&
گفتشان چون حکم نافذ می‌رود
\\
هیچ وازر وزر غیری بر نداشت
&&
من نیم وازر خدایم بر فراشت
\\
آنک بی وزرست شیخست ای جوان
&&
در قبول حق چواندر کف کمان
\\
شیخ کی بود پیر یعنی مو سپید
&&
معنی این مو بدان ای کژ امید
\\
هست آن موی سیه هستی او
&&
تا ز هستی‌اش نماند تای مو
\\
چونک هستی‌اش نماند پیر اوست
&&
گر سیه‌مو باشد او یا خود دوموست
\\
هست آن موی سیه وصف بشر
&&
نیست آن مو موی ریش و موی سر
\\
عیسی اندر مهد بر دارد نفیر
&&
که جوان ناگشته ما شیخیم و پیر
\\
گر رهید از بعض اوصاف بشر
&&
شیخ نبود کهل باشد ای پسر
\\
چون یکی موی سیه کان وصف ماست
&&
نیست بر وی شیخ و مقبول خداست
\\
چون بود مویش سپید ار با خودست
&&
او نه پیرست و نه خاص ایزدست
\\
ور سر مویی ز وصفش باقیست
&&
او نه از عرش است او آفاقیست
\\
\end{longtable}
\end{center}
