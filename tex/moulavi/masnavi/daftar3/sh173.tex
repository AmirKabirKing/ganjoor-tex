\begin{center}
\section*{بخش ۱۷۳ - آداب المستمعین والمریدین عند فیض  الحکمة من لسان الشیخ}
\label{sec:sh173}
\addcontentsline{toc}{section}{\nameref{sec:sh173}}
\begin{longtable}{l p{0.5cm} r}
بر ملولان این مکرر کردنست
&&
نزد من عمر مکرر بردنست
\\
شمع از برق مکرر بر شود
&&
خاک از تاب مکرر زر شود
\\
گر هزاران طالب‌اند و یک ملول
&&
از رسالت باز می‌ماند رسول
\\
این رسولان ضمیر رازگو
&&
مستمع خواهند اسرافیل‌خو
\\
نخوتی دارند و کبری چون شهان
&&
چاکری خواهند از اهل جهان
\\
تا ادبهاشان بجاگه ناوری
&&
از رسالتشان چگونه بر خوری
\\
کی رسانند آن امانت را بتو
&&
تا نباشی پیششان راکع دوتو
\\
هر ادبشان کی همی‌آید پسند
&&
کامدند ایشان ز ایوان بلند
\\
نه گدایانند کز هر خدمتی
&&
از تو دارند ای مزور منتی
\\
لیک با بی‌رغبتیها ای ضمیر
&&
صدقهٔ سلطان بیفشان وا مگیر
\\
اسپ خود را ای رسول آسمان
&&
در ملولان منگر و اندر جهان
\\
فرخ آن ترکی که استیزه نهد
&&
اسپش اندر خندق آتش جهد
\\
گرم گرداند فرس را آنچنان
&&
که کند آهنگ اوج آسمان
\\
چشم را از غیر و غیرت دوخته
&&
همچو آتش خشک و تر را سوخته
\\
گر پشیمانی برو عیبی کند
&&
آتش اول در پشیمانی زند
\\
خود پشیمانی نروید از عدم
&&
چون ببیند گرمی صاحب‌قدم
\\
\end{longtable}
\end{center}
