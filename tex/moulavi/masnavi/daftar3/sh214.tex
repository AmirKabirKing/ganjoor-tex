\begin{center}
\section*{بخش ۲۱۴ - منجذب شدن جان نیز به عالم ارواح و تقاضای او و میل او به مقر خود و منقطع شدن از اجزای اجسام کی هم کندهٔ پای باز روح‌اند}
\label{sec:sh214}
\addcontentsline{toc}{section}{\nameref{sec:sh214}}
\begin{longtable}{l p{0.5cm} r}
گوید ای اجزای پست فرشیم
&&
غربت من تلختر من عرشیم
\\
میل تن در سبزه و آب روان
&&
زان بود که اصل او آمد از آن
\\
میل جان اندر حیات و در حی است
&&
زانک جان لامکان اصل وی است
\\
میل جان در حکمتست و در علوم
&&
میل تن در باغ و راغست و کروم
\\
میل جان اندر ترقی و شرف
&&
میل تن در کسب و اسباب علف
\\
میل و عشق آن شرف هم سوی جان
&&
زین یحب را و یحبون را بدان
\\
حاصل آنک هر که او طالب بود
&&
جان مطلوبش درو راغب بود
\\
گر بگویم شرح این بی حد شود
&&
مثنوی هشتاد تا کاغذ شود
\\
آدمی حیوان نباتی و جماد
&&
هر مرادی عاشق هر بی‌مراد
\\
بی‌مرادان بر مرادی می‌تنند
&&
و آن مرادان جذب ایشان می‌کنند
\\
لیک میل عاشقان لاغر کند
&&
میل معشوقان خوش و خوش‌فر کند
\\
عشق معشوقان دو رخ افروخته
&&
عشق عاشق جان او را سوخته
\\
کهربا عاشق به شکل بی‌نیاز
&&
کاه می‌کوشد در آن راه دراز
\\
این رها کن عشق آن تشنه‌دهان
&&
تافت اندر سینهٔ صدر جهان
\\
دود آن عشق و غم آتش‌کده
&&
رفته در مخدوم او مشفق شده
\\
لیکش از ناموس و بوش و آب رو
&&
شرم می‌آمد که وا جوید ازو
\\
رحمتش مشتاق آن مسکین شده
&&
سلطنت زین لطف مانع آمده
\\
عقل حیران کین عجب او را کشید
&&
یا کشش زان سو بدینجانب رسید
\\
ترک جلدی کن کزین ناواقفی
&&
لب ببند الله اعلم بالخفی
\\
این سخن را بعد ازین مدفون کنم
&&
آن کشنده می‌کشد من چون کنم
\\
کیست آن کت می‌کشد ای معتنی
&&
آنک می‌نگذاردت کین دم زنی
\\
صد عزیمت می‌کنی بهر سفر
&&
می‌کشاند مر ترا جای دگر
\\
زان بگرداند به هر سو آن لگام
&&
تا خبر یابد ز فارس اسپ خام
\\
اسپ زیرکسار زان نیکو پیست
&&
کو همی‌داند که فارس بر ویست
\\
او دلت را بر دو صد سودا ببست
&&
بی‌مرادت کرد پس دل را شکست
\\
چون شکست او بال آن رای نخست
&&
چون نشد هستی بال‌اشکن درست
\\
چون قضایش حبل تدبیرت سکست
&&
چون نشد بر تو قضای آن درست
\\
\end{longtable}
\end{center}
