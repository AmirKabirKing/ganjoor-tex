\begin{center}
\section*{بخش ۲۳ - تشبیه فرعون و دعوی الوهیت او بدان  شغال کی دعوی طاوسی می‌کرد}
\label{sec:sh023}
\addcontentsline{toc}{section}{\nameref{sec:sh023}}
\begin{longtable}{l p{0.5cm} r}
همچو فرعونی مرصع کرده ریش
&&
برتر از عیسی پریده از خریش
\\
او هم از نسل شغال ماده زاد
&&
در خم مالی و جاهی در فتاد
\\
هر که دید آن جاه و مالش سجده کرد
&&
سجدهٔ افسوسیان را او بخورد
\\
گشت مستک آن گدای ژنده‌دلق
&&
از سجود و از تحیرهای خلق
\\
مال مار آمد که در وی زهرهاست
&&
و آن قبول و سجدهٔ خلق اژدهاست
\\
های ای فرعون ناموسی مکن
&&
تو شغالی هیچ طاووسی مکن
\\
سوی طاووسان اگر پیدا شوی
&&
عاجزی از جلوه و رسوا شوی
\\
موسی و هارون چو طاووسان بدند
&&
پر جلوه بر سر و رویت زدند
\\
زشتیت پیدا شد و رسواییت
&&
سرنگون افتادی از بالاییت
\\
چون محک دیدی سیه گشتی چو قلب
&&
نقش شیری رفت و پیدا گشت کلب
\\
ای سگ‌گرگین زشت از حرص و جوش
&&
پوستین شیر را بر خود مپوش
\\
غرهٔ شیرت بخواهد امتحان
&&
نقش شیر و آنگه اخلاق سگان
\\
\end{longtable}
\end{center}
