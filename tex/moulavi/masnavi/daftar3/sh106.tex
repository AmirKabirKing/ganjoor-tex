\begin{center}
\section*{بخش ۱۰۶ - شنیدن داود علیه السلام سخن هر دو خصم وسال کردن از مدعی علیه}
\label{sec:sh106}
\addcontentsline{toc}{section}{\nameref{sec:sh106}}
\begin{longtable}{l p{0.5cm} r}
چونک داود نبی آمد برون
&&
گفت هین چونست این احوال چون
\\
مدعی گفت ای نبی الله داد
&&
گاو من در خانه او در فتاد
\\
کشت گاوم را بپرسش که چرا
&&
گاو من کشت او بیان کن ماجرا
\\
گفت داودش بگو ای بوالکرم
&&
چون تلف کردی تو ملک محترم
\\
هین پراکنده مگو حجت بیار
&&
تا به یک سو گردد این دعوی و کار
\\
گفت ای داود بودم هفت سال
&&
روز و شب اندر دعا و در سؤال
\\
این همی‌جستم ز یزدان کای خدا
&&
روزیی خواهم حلال و بی عنا
\\
مرد و زن بر ناله من واقف‌اند
&&
کودکان این ماجرا را واصف‌اند
\\
تو بپرس از هر که خواهی این خبر
&&
تا بگوید بی شکنجه بی ضرر
\\
هم هویدا پرس و هم پنهان ز خلق
&&
که چه می‌گفت این گدای ژنده‌دلق
\\
بعد این جمله دعا و این فغان
&&
گاوی اندر خانه دیدم ناگهان
\\
چشم من تاریک شد نه بهر لوت
&&
شادی آن که قبول آمد قنوت
\\
کشتم آن را تا دهم در شکر آن
&&
که دعای من شنود آن غیب‌دان
\\
\end{longtable}
\end{center}
