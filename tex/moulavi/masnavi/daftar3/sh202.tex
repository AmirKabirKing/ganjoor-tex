\begin{center}
\section*{بخش ۲۰۲ - ذکرخیال بد اندیشیدن قاصر فهمان}
\label{sec:sh202}
\addcontentsline{toc}{section}{\nameref{sec:sh202}}
\begin{longtable}{l p{0.5cm} r}
پیش از آنک این قصه تا مخلص رسد
&&
دود و گندی آمد از اهل حسد
\\
من نمی‌رنجم ازین لیک این لگد
&&
خاطر ساده‌دلی را پی کند
\\
خوش بیان کرد آن حکیم غزنوی
&&
بهر محجوبان مثال معنوی
\\
که ز قرآن گر نبیند غیر قال
&&
این عجب نبود ز اصحاب ضلال
\\
کز شعاع آفتاب پر ز نور
&&
غیر گرمی می‌نیابد چشم کور
\\
خربطی ناگاه از خرخانه‌ای
&&
سر برون آورد چون طعانه‌ای
\\
کین سخن پستست یعنی مثنوی
&&
قصه پیغامبرست و پی‌روی
\\
نیست ذکر بحث و اسرار بلند
&&
که دوانند اولیا آن سو سمند
\\
از مقامات تبتل تا فنا
&&
پله پله تا ملاقات خدا
\\
شرح و حد هر مقام و منزلی
&&
که بپر زو بر پرد صاحب‌دلی
\\
چون کتاب الله بیامد هم بر آن
&&
این چنین طعنه زدند آن کافران
\\
که اساطیرست و افسانهٔ نژند
&&
نیست تعمیقی و تحقیقی بلند
\\
کودکان خرد فهمش می‌کنند
&&
نیست جز امر پسند و ناپسند
\\
ذکر یوسف ذکر زلف پر خمش
&&
ذکر یعقوب و زلیخا و غمش
\\
ظاهرست و هرکسی پی می‌برد
&&
کو بیان که گم شود در وی خرد
\\
گفت اگر آسان نماید این به تو
&&
این چنین آسان یکی سوره بگو
\\
جنتان و انستان و اهل کار
&&
گو یکی آیت ازین آسان بیار
\\
\end{longtable}
\end{center}
