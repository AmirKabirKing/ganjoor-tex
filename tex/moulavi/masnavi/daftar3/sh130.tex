\begin{center}
\section*{بخش ۱۳۰ - معنی حزم و مثال مرد حازم}
\label{sec:sh130}
\addcontentsline{toc}{section}{\nameref{sec:sh130}}
\begin{longtable}{l p{0.5cm} r}
یا به حال اولینان بنگرید
&&
یا سوی آخر بحزمی در پرید
\\
حزم چه بود در دو تدبیر احتیاط
&&
از دو آن گیری که دورست از خباط
\\
آن یکی گوید درین ره هفت روز
&&
نیست آب و هست ریگ پای‌سوز
\\
آن دگر گوید دروغست این بران
&&
که بهر شب چشمه‌ای بینی روان
\\
حزم آن باشد که بر گیری تو آب
&&
تا رهی از ترس و باشی بر صواب
\\
گر بود در راه آب این را بریز
&&
ور نباشد وای بر مرد ستیز
\\
ای خلیفه‌زادگان دادی کنید
&&
حزم بهر روز میعادی کنید
\\
آن عدوی کز پدرتان کین کشید
&&
سوی زندانش ز علیین کشید
\\
آن شه شطرنج دل را مات کرد
&&
از بهشتش سخرهٔ آفات کرد
\\
چند جا بندش گرفت اندر نبرد
&&
تا بکشتی در فکندش روی‌زرد
\\
اینچنین کردست با آن پهلوان
&&
سست سستش منگرید ای دیگران
\\
مادر و بابای ما را آن حسود
&&
تاج و پیرایه بچالاکی ربود
\\
کردشان آنجا برهنه و زار و خوار
&&
سالها بگریست آدم زار زار
\\
که ز اشک چشم او رویید نبت
&&
که چرا اندر جریدهٔ لاست ثبت
\\
تو قیاسی گیر طراریش را
&&
که چنان سرور کند زو ریش را
\\
الحذر ای گل‌پرستان از شرش
&&
تیغ لا حولی زنید اندر سرش
\\
کو همی‌بیند شما را از کمین
&&
که شما او را نمی‌بینید هین
\\
دایما صیاد ریزد دانه‌ها
&&
دانه پیدا باشد و پنهان دغا
\\
هر کجا دانه بدیدی الحذر
&&
تا نبندد دام بر تو بال و پر
\\
زانک مرغی کو بترک دانه کرد
&&
دانه از صحرای بی تزویر خورد
\\
هم بدان قانع شد و از دام جست
&&
هیچ دامی پر و بالش را نبست
\\
\end{longtable}
\end{center}
