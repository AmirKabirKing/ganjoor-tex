\begin{center}
\section*{بخش ۲۲۷ - حکایت عاشقی دراز هجرانی بسیار امتحانی}
\label{sec:sh227}
\addcontentsline{toc}{section}{\nameref{sec:sh227}}
\begin{longtable}{l p{0.5cm} r}
یک جوانی بر زنی مجنون بدست
&&
می‌ندادش روزگار وصل دست
\\
بس شکنجه کرد عشقش بر زمین
&&
خود چرا دارد ز اول عشق کین
\\
عشق از اول چرا خونی بود
&&
تا گریزد آنک بیرونی بود
\\
چون فرستادی رسولی پیش زن
&&
آن رسول از رشک گشتی راه‌زن
\\
ور بسوی زن نبشتی کاتبش
&&
نامه را تصحیف خواندی نایبش
\\
ور صبا را پیک کردی در وفا
&&
از غباری تیره گشتی آن صبا
\\
رقعه گر بر پر مرغی دوختی
&&
پر مرغ از تف رقعه سوختی
\\
راههای چاره را غیرت ببست
&&
لشکر اندیشه را رایت شکست
\\
بود اول مونس غم انتظار
&&
آخرش بشکست کی هم انتظار
\\
گاه گفتی کین بلای بی‌دواست
&&
گاه گفتی نه حیات جان ماست
\\
گاه هستی زو بر آوردی سری
&&
گاه او از نیستی خوردی بری
\\
چونک بر وی سرد گشتی این نهاد
&&
جوش کردی گرم چشمهٔ اتحاد
\\
چونک با بی‌برگی غربت بساخت
&&
برگ بی‌برگی به سوی او بتاخت
\\
خوشه‌های فکرتش بی‌کاه شد
&&
شب‌روان را رهنما چون ماه شد
\\
ای بسا طوطی گویای خمش
&&
ای بسا شیرین‌روان رو ترش
\\
رو به گورستان دمی خامش نشین
&&
آن خموشان سخن‌گو را ببین
\\
لیک اگر یکرنگ بینی خاکشان
&&
نیست یکسان حالت چالاکشان
\\
شحم و لحم زندگان یکسان بود
&&
آن یکی غمگین دگر شادان بود
\\
تو چه دانی تا ننوشی قالشان
&&
زانک پنهانست بر تو حالشان
\\
بشنوی از قال های و هوی را
&&
کی ببینی حالت صدتوی را
\\
نقش ما یکسان بضدها متصف
&&
خاک هم یکسان روانشان مختلف
\\
همچنین یکسان بود آوازها
&&
آن یکی پر درد و آن پر نازها
\\
بانگ اسپان بشنوی اندر مصاف
&&
بانگ مرغان بشنوی اندر طواف
\\
آن یکی از حقد و دیگر ز ارتباط
&&
آن یکی از رنج و دیگر از نشاط
\\
هر که دور از حالت ایشان بود
&&
پیشش آن آوازها یکسان بود
\\
آن درختی جنبد از زخم تبر
&&
و آن درخت دیگر از باد سحر
\\
بس غلط گشتم ز دیگ مردریگ
&&
زانک سرپوشیده می‌جوشید دیگ
\\
جوش و نوش هرکست گوید بیا
&&
جوش صدق و جوش تزویر و ریا
\\
گر نداری بو ز جان روشناس
&&
رو دماغی دست آور بوشناس
\\
آن دماغی که بر آن گلشن تند
&&
چشم یعقوبان هم او روشن کند
\\
هین بگو احوال آن خسته‌جگر
&&
کز بخاری دور ماندیم ای پسر
\\
\end{longtable}
\end{center}
