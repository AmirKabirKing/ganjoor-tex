\begin{center}
\section*{بخش ۱۳۶ - باز جواب انبیا علیهم السلام ایشان را}
\label{sec:sh136}
\addcontentsline{toc}{section}{\nameref{sec:sh136}}
\begin{longtable}{l p{0.5cm} r}
انبیا گفتند نومیدی بدست
&&
فضل و رحمتهای باری بی‌حدست
\\
از چنین محسن نشاید ناامید
&&
دست در فتراک این رحمت زنید
\\
ای بسا کارا که اول صعب گشت
&&
بعد از آن بگشاده شد سختی گذشت
\\
بعد نومیدی بسی اومیدهاست
&&
از پس ظلمت بسی خورشیدهاست
\\
خود گرفتم که شما سنگین شدیت
&&
قفلها بر گوش و بر دل بر زدیت
\\
هیچ ما را با قبولی کار نیست
&&
کار ما تسلیم و فرمان کردنیست
\\
او بفرمودستمان این بندگی
&&
نیست ما را از خود این گویندگی
\\
جان برای امر او داریم ما
&&
گر به ریگی گوید او کاریم ما
\\
غیر حق جان نبی را یار نیست
&&
با قبول و رد خلقش کار نیست
\\
مزد تبلیغ رسالاتش ازوست
&&
زشت و دشمن‌رو شدیم از بهر دوست
\\
ما برین درگه ملولان نیستیم
&&
تا ز بعد راه هر جا بیستیم
\\
دل فرو بسته و ملول آنکس بود
&&
کز فراق یار در محبس بود
\\
دلبر و مطلوب با ما حاضرست
&&
در نثار رحمتش جان شاکرست
\\
در دل ما لاله‌زار و گلشنیست
&&
پیری و پژمردگی را راه نیست
\\
دایما تر و جوانیم و لطیف
&&
تازه و شیرین و خندان و ظریف
\\
پیش ما صد سال و یکساعت یکیست
&&
که دراز و کوته از ما منفکیست
\\
آن دراز و کوتهی در جسمهاست
&&
آن دراز و کوته اندر جان کجاست
\\
سیصد و نه سال آن اصحاب کهف
&&
پیششان یک روز بی اندوه و لهف
\\
وانگهی بنمودشان یک روز هم
&&
که به تن باز آمد ارواح از عدم
\\
چون نباشد روز و شب یا ماه و سال
&&
کی بود سیری و پیری و ملال
\\
در گلستان عدم چون بی‌خودیست
&&
مستی از سغراق لطف ایزدیست
\\
لم یذق لم یدر هر کس کو نخورد
&&
کی بوهم آرد جعل انفاس ورد
\\
نیست موهوم ار بدی موهوم آن
&&
همچو موهومان شدی معدوم آن
\\
دوزخ اندر وهم چون آرد بهشت
&&
هیچ تابد روی خوب از خوک زشت
\\
هین گلوی خود مبر هان ای مهان
&&
این‌چنین لقمه رسیده تا دهان
\\
راههای صعب پایان برده‌ایم
&&
ره بر اهل خویش آسان کرده‌ایم
\\
\end{longtable}
\end{center}
