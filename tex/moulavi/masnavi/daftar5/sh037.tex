\begin{center}
\section*{بخش ۳۷ - مناجات}
\label{sec:sh037}
\addcontentsline{toc}{section}{\nameref{sec:sh037}}
\begin{longtable}{l p{0.5cm} r}
ای مبدل کرده خاکی را به زر
&&
خاک دیگر را بکرده بوالبشر
\\
کار تو تبدیل اعیان و عطا
&&
کار من سهوست و نسیان و خطا
\\
سهو و نسیان را مبدل کن به علم
&&
من همه خلمم مرا کن صبر و حلم
\\
ای که خاک شوره را تو نان کنی
&&
وی که نان مرده را تو جان کنی
\\
ای که جان خیره را رهبر کنی
&&
وی که بی‌ره را تو پیغمبر کنی
\\
می‌کنی جزو زمین را آسمان
&&
می‌فزایی در زمین از اختران
\\
هر که سازد زین جهان آب حیات
&&
زوترش از دیگران آید ممات
\\
دیدهٔ دل کو به گردون بنگریست
&&
دید که اینجا هر دمی میناگریست
\\
قلب اعیانست و اکسیری محیط
&&
ایتلاف خرقهٔ تن بی‌مخیط
\\
تو از آن روزی که در هست آمدی
&&
آتشی یا بادی یا خاکی بدی
\\
گر بر آن حالت ترا بودی بقا
&&
کی رسیدی مر ترا این ارتقا
\\
از مبدل هستی اول نماند
&&
هستی بهتر به جای آن نشاند
\\
هم‌چنین تا صد هزاران هستها
&&
بعد یکدیگر دوم به ز ابتدا
\\
از مبدل بین وسایط را بمان
&&
کز وسایط دور گردی ز اصل آن
\\
واسطه هر جا فزون شد وصل جست
&&
واسطه کم ذوق وصل افزونترست
\\
از سبب‌دانی شود کم حیرتت
&&
حیرت تو ره دهد در حضرتت
\\
این بقاها از فناها یافتی
&&
از فنااش رو چرا برتافتی
\\
زان فناها چه زیان بودت که تا
&&
بر بقا چفسیده‌ای ای نافقا
\\
چون دوم از اولینت بهترست
&&
پس فنا جو و مبدل را پرست
\\
صد هزاران حشر دیدی ای عنود
&&
تاکنون هر لحظه از بدو وجود
\\
از جماد بی‌خبر سوی نما
&&
وز نما سوی حیات و ابتلا
\\
باز سوی عقل و تمییزات خوش
&&
باز سوی خارج این پنج و شش
\\
تا لب بحر این نشان پایهاست
&&
پس نشان پا درون بحر لاست
\\
زانک منزلهای خشکی ز احتیاط
&&
هست دهها و وطنها و رباط
\\
باز منزلهای دریا در وقوف
&&
وقت موج و حبس بی‌عرصه و سقوف
\\
نیست پیدا آن مراحل را سنام
&&
نه نشانست آن منازل را نه نام
\\
هست صد چندان میان منزلین
&&
آن طرف که از نما تا روح عین
\\
در فناها این بقاها دیده‌ای
&&
بر بقای جسم چون چفسیده‌ای
\\
هین بده ای زاغ این جان باز باش
&&
پیش تبدیل خدا جانباز باش
\\
تازه می‌گیر و کهن را می‌سپار
&&
که هر امسالت فزونست از سه پار
\\
گر نباشی نخل‌وار ایثار کن
&&
کهنه بر کهنه نه و انبار کن
\\
کهنه و گندیده و پوسیده را
&&
تحفه می‌بر بهر هر نادیده را
\\
آنک نو دید او خریدار تو نیست
&&
صید حقست او گرفتار تو نیست
\\
هر کجا باشند جوق مرغ کور
&&
بر تو جمع آیند ای سیلاب شور
\\
تا فزاید کوری از شورابها
&&
زانک آب شور افزاید عمی
\\
اهل دنیا زان سبب اعمی‌دل‌اند
&&
شارب شورابهٔ آب و گل‌اند
\\
شور می‌ده کور می‌خر در جهان
&&
چون نداری آب حیوان در نهان
\\
با چنین حالت بقا خواهی و یاد
&&
هم‌چو زنگی در سیه‌رویی تو شاد
\\
در سیاهی زنگی زان آسوده است
&&
کو ز زاد و اصل زنگی بوده است
\\
آنک روزی شاهد و خوش‌رو بود
&&
گر سیه‌گردد تدارک‌جو بود
\\
مرغ پرنده چو ماند در زمین
&&
باشد اندر غصه و درد و حنین
\\
مرغ خانه بر زمین خوش می‌رود
&&
دانه‌چین و شاد و شاطر می‌دود
\\
زآنک او از اصل بی‌پرواز بود
&&
وآن دگر پرنده و پرواز بود
\\
\end{longtable}
\end{center}
