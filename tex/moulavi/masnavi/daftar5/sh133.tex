\begin{center}
\section*{بخش ۱۳۳ - حکایت هم در جواب جبری و اثبات اختیار و صحت امر و نهی و بیان آنک عذر جبری در هیچ ملتی و در هیچ دینی مقبول نیست و موجب خلاص نیست از سزای آن کار  کی کرده است چنانک خلاص نیافت ابلیس جبری بدان کی گفت بما اغویتنی والقلیل یدل علی الکثیر}
\label{sec:sh133}
\addcontentsline{toc}{section}{\nameref{sec:sh133}}
\begin{longtable}{l p{0.5cm} r}
آن یکی می‌رفت بالای درخت
&&
می‌فشاند آن میوه را دزدانه سخت
\\
صاحب باغ آمد و گفت ای دنی
&&
از خدا شرمیت کو چه می‌کنی
\\
گفت از باغ خدا بندهٔ خدا
&&
گر خورد خرما که حق کردش عطا
\\
عامیانه چه ملامت می‌کنی
&&
بخل بر خوان خداوند غنی
\\
گفت ای ایبک بیاور آن رسن
&&
تا بگویم من جواب بوالحسن
\\
پس ببستش سخت آن دم بر درخت
&&
می‌زد او بر پشت و ساقش چوب سخت
\\
گفت آخر از خدا شرمی بدار
&&
می‌کشی این بی‌گنه را زار زار
\\
گفت از چوب خدا این بنده‌اش
&&
می‌زند بر پشت دیگر بنده خوش
\\
چوب حق و پشت و پهلو آن او
&&
من غلام و آلت فرمان او
\\
گفت توبه کردم از جبر ای عیار
&&
اختیارست اختیارست اختیار
\\
اختیارات اختیارش هست کرد
&&
اختیارش چون سواری زیر گرد
\\
اختیارش اختیار ما کند
&&
امر شد بر اختیاری مستند
\\
حاکمی بر صورت بی‌اختیار
&&
هست هر مخلوق را در اقتدار
\\
تا کشد بی‌اختیاری صید را
&&
تا برد بگرفته گوش او زید را
\\
لیک بی هیچ آلتی صنع صمد
&&
اختیارش را کمند او کند
\\
اختیارش زید را قدیش کند
&&
بی‌سگ و بی‌دام حق صیدش کند
\\
آن دروگر حاکم چوبی بود
&&
وآن مصور حاکم خوبی بود
\\
هست آهنگر بر آهن قیمی
&&
هست بنا هم بر آلت حاکمی
\\
نادر این باشد که چندین اختیار
&&
ساجد اندر اختیارش بنده‌وار
\\
قدرت تو بر جمادات از نبرد
&&
کی جمادی را از آنها نفی کرد
\\
قدرتش بر اختیارات آنچنان
&&
نفی نکند اختیاری را از آن
\\
خواستش می‌گوی بر وجه کمال
&&
که نباشد نسبت جبر و ضلال
\\
چونک گفتی کفر من خواست ویست
&&
خواست خود را نیز هم می‌دان که هست
\\
زانک بی‌خواه تو خود کفر تو نیست
&&
کفر بی‌خواهش تناقض گفتنیست
\\
امر عاجز را قبیحست و ذمیم
&&
خشم بتر خاصه از رب رحیم
\\
گاو گر یوغی نگیرد می‌زنند
&&
هیچ گاوی که نپرد شد نژند
\\
گاو چون معذور نبود در فضول
&&
صاحب گاو از چه معذورست و دول
\\
چون نه‌ای رنجور سر را بر مبند
&&
اختیارت هست بر سبلت مخند
\\
جهد کن کز جام حق یابی نوی
&&
بی‌خود و بی‌اختیار آنگه شوی
\\
آنگه آن می را بود کل اختیار
&&
تو شوی معذور مطلق مست‌وار
\\
هرچه گویی گفتهٔ می باشد آن
&&
هر چه روبی رفتهٔ می باشد آن
\\
کی کند آن مست جز عدل و صواب
&&
که ز جام حق کشیدست او شراب
\\
جادوان فرعون را گفتند بیست
&&
مست را پروای دست و پای نیست
\\
دست و پای ما می آن واحدست
&&
دست ظاهر سایه است و کاسدست
\\
\end{longtable}
\end{center}
