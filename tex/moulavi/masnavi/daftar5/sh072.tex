\begin{center}
\section*{بخش ۷۲ - جواب آن مغفل کی گفته است کی خوش بودی این جهان اگر مرگ نبودی وخوش بودی ملک دنیا اگر زوالش نبودی و علی هذه  الوتیرة من الفشارات}
\label{sec:sh072}
\addcontentsline{toc}{section}{\nameref{sec:sh072}}
\begin{longtable}{l p{0.5cm} r}
آن یکی می‌گفت خوش بودی جهان
&&
گر نبودی پای مرگ اندر میان
\\
آن دگر گفت ار نبودی مرگ هیچ
&&
که نیرزیدی جهان پیچ‌پیچ
\\
خرمنی بودی به دشت افراشته
&&
مهمل و ناکوفته بگذاشته
\\
مرگ را تو زندگی پنداشتی
&&
تخم را در شوره خاکی کاشتی
\\
عقل کاذب هست خود معکوس‌بین
&&
زندگی را مرگ بیند ای غبین
\\
ای خدا بنمای تو هر چیز را
&&
آنچنان که هست در خدعه‌سرا
\\
هیچ مرده نیست پر حسرت ز مرگ
&&
حسرتش آنست کش کم بود برگ
\\
ورنه از چاهی به صحرا اوفتاد
&&
در میان دولت و عیش و گشاد
\\
زین مقام ماتم و ننگین مناخ
&&
نقل افتادش به صحرای فراخ
\\
مقعد صدقی نه ایوان دروغ
&&
بادهٔ خاصی نه مستیی ز دوغ
\\
مقعد صدق و جلیسش حق شده
&&
رسته زین آب و گل آتشکده
\\
ور نکردی زندگانی منیر
&&
یک دو دم ماندست مردانه بمیر
\\
\end{longtable}
\end{center}
