\begin{center}
\section*{بخش ۹۳ - حکایت در بیان آنک کسی توبه کند و پشیمان شود و باز آن پشیمانیها را فراموش کند و آزموده را باز آزماید در خسارت ابد افتد چون توبهٔ او را ثباتی و قوتی و حلاوتی و قبولی مدد نرسد چون درخت بی‌بیخ هر روز زردتر و خشک‌تر نعوذ بالله}
\label{sec:sh093}
\addcontentsline{toc}{section}{\nameref{sec:sh093}}
\begin{longtable}{l p{0.5cm} r}
گازری بود و مر او را یک خری
&&
پشت ریش اشکم تهی و لاغری
\\
در میان سنگ لاخ بی‌گیاه
&&
روز تا شب بی‌نوا و بی‌پناه
\\
بهر خوردن جز که آب آنجا نبود
&&
روز و شب بد خر در آن کور و کبود
\\
آن حوالی نیستان و بیشه بود
&&
شیر بود آنجا که صیدش پیشه بود
\\
شیر را با پیل نر جنگ اوفتاد
&&
خسته شد آن شیر و ماند از اصطیاد
\\
مدتی وا ماند زان ضعف از شکار
&&
بی‌نوا ماندند دد از چاشت‌خوار
\\
زانک باقی‌خوار شیر ایشان بدند
&&
شیر چون رنجور شد تنگ آمدند
\\
شیر یک روباه را فرمود رو
&&
مر خری را بهر من صیاد شو
\\
گر خری یابی به گرد مرغزار
&&
رو فسونش خوان فریبانش بیار
\\
چون بیابم قوتی از گوشت خر
&&
پس بگیرم بعد از آن صیدی دگر
\\
اندکی من می‌خورم باقی شما
&&
من سبب باشم شما را در نوا
\\
یا خری یا گاو بهر من بجوی
&&
زان فسونهایی که می‌دانی بگوی
\\
از فسون و از سخنهای خوشش
&&
از سرش بیرون کن و اینجا کشش
\\
\end{longtable}
\end{center}
