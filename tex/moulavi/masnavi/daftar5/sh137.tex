\begin{center}
\section*{بخش ۱۳۷ - باز جواب گفتن آن کافر جبری آن سنی را کی باسلامش دعوت می‌کرد و به ترک اعتقاد جبرش دعوت می‌کرد و دراز شدن  مناظره از طرفین کی مادهٔ اشکال و جواب را نبرد الا عشق حقیقی کی او را پروای آن نماند و ذلک فضل الله یتیه من یشاء}
\label{sec:sh137}
\addcontentsline{toc}{section}{\nameref{sec:sh137}}
\begin{longtable}{l p{0.5cm} r}
کافر جبری جواب آغاز کرد
&&
که از آن حیران شد آن منطیق مرد
\\
لیک گر من آن جوابات و سؤال
&&
جمله را گویم بمانم زین مقال
\\
زان مهم‌تر گفتنیها هستمان
&&
که بدان فهم تو به یابد نشان
\\
اندکی گفتیم زان بحث ای عتل
&&
ز اندکی پیدا بود قانون کل
\\
هم‌چنین بحثست تا حشر بشر
&&
در میان جبری و اهل قدر
\\
گر فرو ماندی ز دفع خصم خویش
&&
مذهب ایشان بر افتادی ز پیش
\\
چون برون‌شوشان نبودی در جواب
&&
پس رمیدندی از آن راه تباب
\\
چونک مقضی بد دوام آن روش
&&
می‌دهدشان از دلایل پرورش
\\
تا نگردد ملزم از اشکال خصم
&&
تا بود محجوب از اقبال خصم
\\
تا که این هفتاد و دو ملت مدام
&&
در جهان ماند الی یوم القیام
\\
چون جهان ظلمتست و غیب این
&&
از برای سایه می‌باید زمین
\\
تا قیامت ماند این هفتاد و دو
&&
کم نیاید مبتدع را گفت و گو
\\
عزت مخزن بود اندر بها
&&
که برو بسیار باشد قفلها
\\
عزت مقصد بود ای ممتحن
&&
پیچ پیچ راه و عقبه و راه‌زن
\\
عزت کعبه بود و آن نادیه
&&
ره‌زنی اعراب و طول بادیه
\\
هر روش هر ره که آن محمود نیست
&&
عقبه‌ای و مانعی و ره‌زنیست
\\
این روش خصم و حقود آن شده
&&
تا مقلد در دو ره حیران شده
\\
صدق هر دو ضد بیند در روش
&&
هر فریقی در ره خود خوش منش
\\
گر جوابش نیست می‌بندد ستیز
&&
بر همان دم تا به روز رستخیز
\\
که مهان ما بدانند این جواب
&&
گرچه از ما شد نهان وجه صواب
\\
پوزبند وسوسه عشقست و بس
&&
ورنه کی وسواس را بستست کس
\\
عاشقی شو شاهدی خوبی بجو
&&
صید مرغابی همی‌کن جو بجو
\\
کی بری زان آب کان آبت برد
&&
کی کنی زان فهم فهمت را خورد
\\
غیر این معقولها معقولها
&&
یابی اندر عشق با فر و بها
\\
غیر این عقل تو حق را عقلهاست
&&
که بدان تدبیر اسباب سماست
\\
که بدین عقل آوری ارزاق را
&&
زان دگر مفرش کنی اطباق را
\\
چون ببازی عقل در عشق صمد
&&
عشر امثالت دهد یا هفت‌صد
\\
آن زنان چون عقلها درباختند
&&
بر رواق عشق یوسف تاختند
\\
عقلشان یک‌دم ستد ساقی عمر
&&
سیر گشتند از خرد باقی مرد
\\
اصل صد یوسف جمال ذوالجلال
&&
ای کم از زن شو فدای آن جمال
\\
عشق برد بحث را ای جان و بس
&&
کو ز گفت و گو شود فریاد رس
\\
حیرتی آید ز عشق آن نطق را
&&
زهره نبود که کند او ماجرا
\\
که بترسد گر جوابی وا دهد
&&
گوهری از لنج او بیرون فتد
\\
لب ببندد سخت او از خیر و شر
&&
تا نباید کز دهان افتد گهر
\\
هم‌چنانک گفت آن یار رسول
&&
چون نبی بر خواندی بر ما فصول
\\
آن رسول مجتبی وقت نثار
&&
خواستی از ما حضور و صد وقار
\\
آنچنان که بر سرت مرغی بود
&&
کز فواتش جان تو لرزان شود
\\
پس نیاری هیچ جنبیدن ز جا
&&
تا نگیرد مرغ خوب تو هوا
\\
دم نیاری زد ببندی سرفه را
&&
تا نباید که بپرد آن هما
\\
ور کست شیرین بگوید یا ترش
&&
بر لب انگشتی نهی یعنی خمش
\\
حیرت آن مرغست خاموشت کند
&&
بر نهد سردیگ و پر جوشت کند
\\
\end{longtable}
\end{center}
