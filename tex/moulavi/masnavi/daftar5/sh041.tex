\begin{center}
\section*{بخش ۴۱ - بقیهٔ قصهٔ آهو و آخر خران}
\label{sec:sh041}
\addcontentsline{toc}{section}{\nameref{sec:sh041}}
\begin{longtable}{l p{0.5cm} r}
روزها آن آهوی خوش‌ناف نر
&&
در شکنجه بود در اصطبل خر
\\
مضطرب در نزع چون ماهی ز خشک
&&
در یکی حقه معذب پشک و مشک
\\
یک خرش گفتی که ها این بوالوحوش
&&
طبع شاهان دارد و میران خموش
\\
وآن دگر تسخر زدی کز جر و مد
&&
گوهر آوردست کی ارزان دهد
\\
وآن خری گفتی که با این نازکی
&&
بر سریر شاه شو گو متکی
\\
آن خری شد تخمه وز خوردن بماند
&&
پس برسم دعوت آهو را بخواند
\\
سر چنین کرد او که نه رو ای فلان
&&
اشتهاام نیست هستم ناتوان
\\
گفت می‌دانم که نازی می‌کنی
&&
یا ز ناموس احترازی می‌کنی
\\
گفت او با خود که آن طعمهٔ توست
&&
که از آن اجزای تو زنده و نوست
\\
من الیف مرغزاری بوده‌ام
&&
در زلال و روضه‌ها آسوده‌ام
\\
گر قضا انداخت ما را در عذاب
&&
کی رود آن خو و طبع مستطاب
\\
گر گدا گشتم گدارو کی شوم
&&
ور لباسم کهنه گردد من نوم
\\
سنبل و لاله و سپرغم نیز هم
&&
با هزاران ناز و نفرت خورده‌ام
\\
گفت آری لاف می‌زن لاف‌لاف
&&
در غریبی بس توان گفتن گزاف
\\
گفت نافم خود گواهی می‌دهد
&&
منتی بر عود و عنبر می‌نهد
\\
لیک آن را کی شنود صاحب‌مشام
&&
بر خر سرگین‌پرست آن شد حرام
\\
خر کمیز خر ببوید بر طریق
&&
مشک چون عرضه کنم با این فریق
\\
بهر این گفت آن نبی مستجیب
&&
رمز الاسلام فی‌الدنیا غریب
\\
زانک خویشانش هم از وی می‌رمند
&&
گرچه با ذاتش ملایک هم‌دمند
\\
صورتش را جنس می‌بینند انام
&&
لیک از وی می‌نیابند آن مشام
\\
هم‌چو شیری در میان نقش گاو
&&
دور می‌بینش ولی او را مکاو
\\
ور بکاوی ترک گاو تن بگو
&&
که بدرد گاو را آن شیرخو
\\
طبع گاوی از سرت بیرون کند
&&
خوی حیوانی ز حیوان بر کند
\\
گاو باشی شیر گردی نزد او
&&
گر تو با گاوی خوشی شیری مجو
\\
\end{longtable}
\end{center}
