\begin{center}
\section*{بخش ۱۲۸ - دعوت کردن مسلمان مغ را}
\label{sec:sh128}
\addcontentsline{toc}{section}{\nameref{sec:sh128}}
\begin{longtable}{l p{0.5cm} r}
مر مغی را گفت مردی کای فلان
&&
هین مسلمان شو بباش از مؤمنان
\\
گفت اگر خواهد خدا مؤمن شوم
&&
ور فزاید فضل هم موقن شوم
\\
گفت می‌خواهد خدا ایمان تو
&&
تا رهد از دست دوزخ جان تو
\\
لیک نفس نحس و آن شیطان زشت
&&
می‌کشندت سوی کفران و کنشت
\\
گفت ای منصف چو ایشان غالب‌اند
&&
یار او باشم که باشد زورمند
\\
یار آن تانم بدن کو غالبست
&&
آن طرف افتم که غالب جاذبست
\\
چون خدا می‌خواست از من صدق زفت
&&
خواست او چه سود چون پیشش نرفت
\\
نفس و شیطان خواست خود را پیش برد
&&
وآن عنایت قهر گشت و خرد و مرد
\\
تو یکی قصر و سرایی ساختی
&&
اندرو صد نقش خوش افراختی
\\
خواستی مسجد بود آن جای خیر
&&
دیگری آمد مر آن را ساخت دیر
\\
یا تو بافیدی یکی کرباس تا
&&
خوش بسازی بهر پوشیدن قبا
\\
تو قبا می‌خواستی خصم از نبرد
&&
رغم تو کرباس را شلوار کرد
\\
او زبون شد جرم این کرباس چیست
&&
آنک او مغلوب غالب نیست کیست
\\
چون کسی بی‌خواست او بر وی براند
&&
خاربن در ملک و خانهٔ او نشاند
\\
صاحب خانه بدین خواری بود
&&
که چنین بر وی خلاقت می‌رود
\\
هم خلق گردم من ار تازه و نوم
&&
چونک یار این چنین خواری شوم
\\
چونک خواه نفس آمد مستعان
&&
تسخر آمد ایش شاء الله کان
\\
من اگر ننگ مغان یا کافرم
&&
آن نیم که بر خدا این ظن برم
\\
که کسی ناخواه او و رغم او
&&
گردد اندر ملکت او حکم جو
\\
ملکت او را فرو گیرد چنین
&&
که نیارد دم زدن دم آفرین
\\
دفع او می‌خواهد و می‌بایدش
&&
دیو هر دم غصه می‌افزایدش
\\
بندهٔ این دیو می‌باید شدن
&&
چونک غالب اوست در هر انجمن
\\
تا مبادا کین کشد شیطان ز من
&&
پس چه دستم گیرد آنجا ذوالمنن
\\
آنک او خواهد مراد او شود
&&
از کی کار من دگر نیکو شود
\\
\end{longtable}
\end{center}
