\begin{center}
\section*{بخش ۱۶۹ - خنده گرفتن آن کنیزک را از ضعف شهوت خلیفه و قوت شهوت آن امیر و فهم کردن خلیفه از خندهٔ کنیزک}
\label{sec:sh169}
\addcontentsline{toc}{section}{\nameref{sec:sh169}}
\begin{longtable}{l p{0.5cm} r}
زن بدید آن سستی او از شگفت
&&
آمد اندر قهقهه خنده‌ش گرفت
\\
یادش آمد مردی آن پهلوان
&&
که بکشت او شیر و اندامش چنان
\\
غالب آمد خندهٔ زن شد دراز
&&
جهد می‌کرد و نمی‌شد لب فراز
\\
سخت می‌خندید هم‌چون بنگیان
&&
غالب آمد خنده بر سود و زیان
\\
هرچه اندیشید خنده می‌فزود
&&
هم‌چو بند سیل ناگاهان گشود
\\
گریه و خنده غم و شادی دل
&&
هر یکی را معدنی دان مستقل
\\
هر یکی را مخزنی مفتاح آن
&&
ای برادر در کف فتاح دان
\\
هیچ ساکن می‌نشد آن خنده زو
&&
پس خلیفه طیره گشت و تندخو
\\
زود شمشیر از غلافش بر کشید
&&
گفت سر خنده واگو ای پلید
\\
در دلم زین خنده ظنی اوفتاد
&&
راستی گو عشوه نتوانیم داد
\\
ور خلاف راستی بفریبیم
&&
یا بهانهٔ چرب آری تو به دم
\\
من بدانم در دل من روشنیست
&&
بایدت گفتن هر آنچ گفتنیست
\\
در دل شاهان تو ماهی دان سطبر
&&
گرچه گه گه شد ز غفلت زیر ابر
\\
یک چراغی هست در دل وقت گشت
&&
وقت خشم و حرص آید زیر طشت
\\
آن فراست این زمان یار منست
&&
گر نگویی آنچ حق گفتنست
\\
من بدین شمشیر برم گردنت
&&
سود نبود خود بهانه کردنت
\\
ور بگویی راست آزادت کنم
&&
حق یزدان نشکنم شادت کنم
\\
هفت مصحف آن زمان برهم نهاد
&&
خورد سوگند و چنین تقریر داد
\\
\end{longtable}
\end{center}
