\begin{center}
\section*{بخش ۱۰۴ - فرق میان دعوت شیخ کامل واصل و میان سخن ناقصان  فاضل فضل تحصیلی بر بسته}
\label{sec:sh104}
\addcontentsline{toc}{section}{\nameref{sec:sh104}}
\begin{longtable}{l p{0.5cm} r}
شیخ نورانی ز ره آگه کند
&&
با سخن هم نور را همره کند
\\
جهد کن تا مست و نورانی شوی
&&
تا حدیثت را شود نورش روی
\\
هر چه در دوشاب جوشیده شود
&&
در عقیده طعم دوشابش بود
\\
از جزر وز سیب و به وز گردگان
&&
لذت دوشاب یابی تو از آن
\\
علم اندر نور چون فرغرده شد
&&
پس ز علمت نور یابد قوم لد
\\
هر چه گویی باشد آن هم نورناک
&&
که آسمان هرگز نبارد غیر پاک
\\
آسمان شو ابر شو باران ببار
&&
ناودان بارش کند نبود به کار
\\
آب اندر ناودان عاریتیست
&&
آب اندر ابر و دریا فطرتیست
\\
فکر و اندیشه‌ست مثل ناودان
&&
وحی و مکشوفست ابر و آسمان
\\
آب باران باغ صد رنگ آورد
&&
ناودان همسایه در جنگ آورد
\\
خر دو سه حمله به روبه بحث کرد
&&
چون مقلد بد فریب او بخورد
\\
طنطنهٔ ادراک بینایی نداشت
&&
دمدمهٔ روبه برو سکته گماشت
\\
حرص خوردن آنچنان کردش ذلیل
&&
که زبونش گشت با پانصد دلیل
\\
\end{longtable}
\end{center}
