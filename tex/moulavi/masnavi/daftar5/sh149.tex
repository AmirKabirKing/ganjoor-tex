\begin{center}
\section*{بخش ۱۴۹ - قصد انداختن مصطفی علیه‌السلام خود را از کوه حری از وحشت دیر نمودن جبرئیل علیه‌السلام خود را به وی و پیدا شدن  جبرئیل به وی کی مینداز کی ترا دولتها در پیش است}
\label{sec:sh149}
\addcontentsline{toc}{section}{\nameref{sec:sh149}}
\begin{longtable}{l p{0.5cm} r}
مصطفی را هجر چون بفراختی
&&
خویش را از کوه می‌انداختی
\\
تا بگفتی جبرئیلش هین مکن
&&
که ترا بس دولتست از امر کن
\\
مصطفی ساکن شدی ز انداختن
&&
باز هجران آوریدی تاختن
\\
باز خود را سرنگون از کوه او
&&
می‌فکندی از غم و اندوه او
\\
باز خود پیدا شدی آن جبرئیل
&&
که مکن این ای تو شاه بی‌بدیل
\\
هم‌چنین می‌بود تا کشف حجاب
&&
تا بیابید آن گهر را او ز جیب
\\
بهر هر محنت چو خود را می‌کشند
&&
اصل محنتهاست این چونش کشند
\\
از فدایی مردمان را حیرتیست
&&
هر یکی از ما فدای سیرتیست
\\
ای خنک آنک فدا کردست تن
&&
بهر آن کارزد فدای آن شدن
\\
هر یکی چونک فدایی فنیست
&&
کاندر آن ره صرف عمر و کشتنیست
\\
کشتنی اندر غروبی یا شروق
&&
که نه شایق ماند آنگه نه مشوق
\\
باری این مقبل فدای این فنست
&&
کاندرو صد زندگی در کشتنست
\\
عاشق و معشوق و عشقش بر دوام
&&
در دو عالم بهرمند و نیک‌نام
\\
یا کرامی ارحموا اهل الهوی
&&
شانهم ورد التوی بعد التوی
\\
عفو کن ای میر بر سختی او
&&
در نگر در درد و بدبختی او
\\
تا ز جرمت هم خدا عفوی کند
&&
زلتت را مغفرت در آکند
\\
تو ز غفلت بس سبو بشکسته‌ای
&&
در امید عفو دل در بسته‌ای
\\
عفو کن تا عفو یابی در جزا
&&
می‌شکافد مو قدر اندر سزا
\\
\end{longtable}
\end{center}
