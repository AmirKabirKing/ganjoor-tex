\begin{center}
\section*{بخش ۱۵۳ - تفسیر این آیت که و ان الدار الاخرة لهی الحیوان لوکانوا یعلمون کی در و دیوار و عرصهٔ آن عالم و آب  و کوزه و میوه و درخت همه زنده‌اند و سخن‌گوی و سخن‌شنو و جهت آن فرمود مصطفی علیه السلام کی الدنیا جیفه و طلابها کلاب و اگر آخرت را حیات نبودی آخرت هم جیفه بودی جیفه را برای مردگیش جیفه گویند نه برای بوی زشت و فرخجی}
\label{sec:sh153}
\addcontentsline{toc}{section}{\nameref{sec:sh153}}
\begin{longtable}{l p{0.5cm} r}
آن جهان چون ذره ذره زنده‌اند
&&
نکته‌دانند و سخن گوینده‌اند
\\
در جهان مرده‌شان آرام نیست
&&
کین علف جز لایق انعام نیست
\\
هر که را گلشن بود بزم و وطن
&&
کی خورد او باده اندر گولخن
\\
جای روح پاک علیین بود
&&
کرم باشد کش وطن سرگین بود
\\
بهر مخمور خدا جام طهور
&&
بهر این مرغان کور این آب شور
\\
هر که عدل عمرش ننمود دست
&&
پیش او حجاج خونی عادلست
\\
دختران را لعبت مرده دهند
&&
که ز لعب زندگان بی‌آگهند
\\
چون ندارند از فتوت زور و دست
&&
کودکان را تیغ چوبین بهترست
\\
کافران قانع بنقش انبیا
&&
که نگاریده‌ست اندر دیرها
\\
زان مهان ما را چو دور روشنیست
&&
هیچ‌مان پروای نقش سایه نیست
\\
این یکی نقشش نشسته در جهان
&&
وآن دگر نقشش چو مه در آسمان
\\
این دهانش نکته‌گویان با جلیس
&&
و آن دگر با حق به گفتار و انیس
\\
گوش ظاهر این سخن را ضبط کن
&&
گوش جانش جاذب اسرار کن
\\
چشم ظاهر ضابط حلیهٔ بشر
&&
چشم سر حیران مازاغ البصر
\\
پای ظاهر در صف مسجد صواف
&&
پای معنی فوق گردون در طواف
\\
جزو جزوش را تو بشمر هم‌چنین
&&
این درون وقت و آن بیرون حین
\\
این که در وقتست باشد تا اجل
&&
وان دگر یار ابد قرن ازل
\\
هست یک نامش ولی الدولتین
&&
هست یک نعتش امام القبلتین
\\
خلوت و چله برو لازم نماند
&&
هیچ غیمی مر ورا غایم نماند
\\
قرص خورشیدست خلوت‌خانه‌اش
&&
کی حجاب آرد شب بیگانه‌اش
\\
علت و پرهیز شد بحران نماند
&&
کفر او ایمان شد و کفران نماند
\\
چون الف از استقامت شد به پیش
&&
او ندارد هیچ از اوصاف خویش
\\
گشت فرد از کسوهٔ خوهای خویش
&&
شد برهنه جان به جان‌افزای خویش
\\
چون برهنه رفت پیش شاه فرد
&&
شاهش از اوصاف قدسی جامه کرد
\\
خلعتی پوشید از اوصاف شاه
&&
بر پرید از چاه بر ایوان جاه
\\
این چنین باشد چو دردی صاف گشت
&&
از بن طشت آمد او بالای طشت
\\
در بن طشت از چه بود او دردناک
&&
شومی آمیزش اجزای خاک
\\
یار ناخوش پر و بالش بسته بود
&&
ورنه او در اصل بس برجسته بود
\\
چون عتاب اهبطوا انگیختند
&&
هم‌چو هاروتش نگون آویختند
\\
بود هاروت از ملاک آسمان
&&
از عتابی شد معلق هم‌چنان
\\
سرنگون زان شد که از سر دور ماند
&&
خویش را سر ساخت و تنها پیش راند
\\
آن سپد خود را چو پر از آب دید
&&
کر استغنا و از دریا برید
\\
بر جگر آبش یکی قطره نماند
&&
بحر رحمت کرد و او را باز خواند
\\
رحمتی بی‌علتی بی‌خدمتی
&&
آید از دریا مبارک ساعتی
\\
الله الله گرد دریابار گرد
&&
گرچه باشند اهل دریابار زرد
\\
تا که آید لطف بخشایش‌گری
&&
سرخ گردد روی زرد از گوهری
\\
زردی رو بهترین رنگهاست
&&
زانک اندر انتظار آن لقاست
\\
لیک سرخی بر رخی که آن لامعست
&&
بهر آن آمد که جانش قانعست
\\
که طمع لاغر کند زرد و ذلیل
&&
نیست او از علت ابدان علیل
\\
چون ببیند روی زرد بی‌سقم
&&
خیره گردد عقل جالینوس هم
\\
چون طمع بستی تو در انوار هو
&&
مصطفی گوید که ذلت نفسه
\\
نور بی‌سایه لطیف و عالی است
&&
آن مشبک سایهٔ غربالی است
\\
عاشقان عریان همی‌خواهند تن
&&
پیش عنینان چه جامه چه بدن
\\
روزه‌داران را بود آن نان و خوان
&&
خرمگس را چه ابا چه دیگدان
\\
\end{longtable}
\end{center}
