\begin{center}
\section*{بخش ۵۶ - داستان آن عاشق کی با معشوق خود برمی‌شمرد خدمتها و وفاهای خود را و شبهای دراز تتجافی جنوبهم عن المضاجع را و بی‌نوایی و جگر تشنگی روزهای دراز را و می‌گفت کی من جزین خدمت نمی‌دانم اگر خدمت دیگر هست مرا ارشاد کن کی هر چه فرمایی منقادم اگر در آتش رفتن است چون خلیل علیه‌السلام  و اگر در دهان نهنگ دریا فتادنست چون یونس علیه‌السلام و اگر هفتاد بار کشته شدن است چون جرجیس علیه‌السلام  و اگر از گریه نابینا شدن است چون شعیب علیه‌السلام و وفا و جانبازی انبیا را علیهم‌السلام شمار نیست و جواب گفتن معشوق او را}
\label{sec:sh056}
\addcontentsline{toc}{section}{\nameref{sec:sh056}}
\begin{longtable}{l p{0.5cm} r}
آن یکی عاشق به پیش یار خود
&&
می‌شمرد از خدمت و از کار خود
\\
کز برای تو چنین کردم چنان
&&
تیرها خوردم درین رزم و سنان
\\
مال رفت و زور رفت و نام رفت
&&
بر من از عشقت بسی ناکام رفت
\\
هیچ صبحم خفته یا خندان نیافت
&&
هیچ شامم با سر و سامان نیافت
\\
آنچ او نوشیده بود از تلخ و درد
&&
او به تفصیلش یکایک می‌شمرد
\\
نه از برای منتی بل می‌نمود
&&
بر درستی محبت صد شهود
\\
عاقلان را یک اشارت بس بود
&&
عاشقان را تشنگی زان کی رود
\\
می‌کند تکرار گفتن بی‌ملال
&&
کی ز اشارت بس کند حوت از زلال
\\
صد سخن می‌گفت زان درد کهن
&&
در شکایت که نگفتم یک سخن
\\
آتشی بودش نمی‌دانست چیست
&&
لیک چون شمع از تف آن می‌گریست
\\
گفت معشوق این همه کردی ولیک
&&
گوش بگشا پهن و اندر یاب نیک
\\
کانچ اصل اصل عشقست و ولاست
&&
آن نکردی اینچ کردی فرعهاست
\\
گفتش آن عاشق بگو که آن اصل چیست
&&
گفت اصلش مردنست ونیستیست
\\
تو همه کردی نمردی زنده‌ای
&&
هین بمیر ار یار جان‌بازنده‌ای
\\
هم در آن دم شد دراز و جان بداد
&&
هم‌چو گل درباخت سر خندان و شاد
\\
ماند آن خنده برو وقف ابد
&&
هم‌چو جان و عقل عارف بی‌کبد
\\
نور مه‌آلوده کی گردد ابد
&&
گر زند آن نور بر هر نیک و بد
\\
او ز جمله پاک وا گردد به ماه
&&
هم‌چو نور عقل و جان سوی اله
\\
وصف پاکی وقف بر نور مه‌است
&&
تا بشش گر بر نجاسات ره‌است
\\
زان نجاسات ره و آلودگی
&&
نور را حاصل نگردد بدرگی
\\
ارجعی بشنود نور آفتاب
&&
سوی اصل خویش باز آمد شتاب
\\
نه ز گلحنها برو ننگی بماند
&&
نه ز گلشنها برو رنگی بماند
\\
نور دیده و نوردیده بازگشت
&&
ماند در سودای او صحرا و دشت
\\
\end{longtable}
\end{center}
