\begin{center}
\section*{بخش ۱۲۵ - حکایت آن گاو کی تنها در جزیره ایست بزرگ حق تعالی آن جزیرهٔ بزرگ را پر کند از نبات و ریاحین کی علف گاو باشد تا به شب آن گاو همه را بخورد و فربه شود چون کوه پاره‌ای چون شب شود خوابش نبرد از غصه و خوف کی همه صحرا را چریدم فردا چه خورم تا ازین غصه لاغر شود هم‌چون خلال روز برخیزد  همه صحرا را سبزتر و انبوه‌تر بیند از دی باز بخورد و فربه شود باز شبش همان غم بگیرد سالهاست کی او هم‌چنین می‌بیند و اعتماد نمی‌کند}
\label{sec:sh125}
\addcontentsline{toc}{section}{\nameref{sec:sh125}}
\begin{longtable}{l p{0.5cm} r}
یک جزیرهٔ سبز هست اندر جهان
&&
اندرو گاویست تنها خوش‌دهان
\\
جمله صحرا را چرد او تا به شب
&&
تا شود زفت و عظیم و منتجب
\\
شب ز اندیشه که فردا چه خورم
&&
گردد او چون تار مو لاغر ز غم
\\
چون برآید صبح گردد سبز دشت
&&
تا میان رسته قصیل سبز و کشت
\\
اندر افتد گاو با جوع البقر
&&
تا به شب آن را چرد او سر به سر
\\
باز زفت و فربه و لمتر شود
&&
آن تنش از پیه و قوت پر شود
\\
باز شب اندر تب افتد از فزع
&&
تا شود لاغر ز خوف منتجع
\\
که چه خواهم خورد فردا وقت خور
&&
سالها اینست کار آن بقر
\\
هیچ نندیشد که چندین سال من
&&
می‌خورم زین سبزه‌زار و زین چمن
\\
هیچ روزی کم نیامد روزیم
&&
چیست این ترس و غم و دلسوزیم
\\
باز چون شب می‌شود آن گاو زفت
&&
می‌شود لاغر که آوه رزق رفت
\\
نفس آن گاوست و آن دشت این جهان
&&
کو همی لاغر شود از خوف نان
\\
که چه خواهم خورد مستقبل عجب
&&
لوت فردا از کجا سازم طلب
\\
سالها خوردی و کم نامد ز خور
&&
ترک مستقبل کن و ماضی نگر
\\
لوت و پوت خورده را هم یاد آر
&&
منگر اندر غابر و کم باش زار
\\
\end{longtable}
\end{center}
