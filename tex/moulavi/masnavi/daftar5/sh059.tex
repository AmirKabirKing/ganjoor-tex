\begin{center}
\section*{بخش ۵۹ - داستان آن کنیزک کی با خر خاتون شهوت می‌راند و او را چون بز و خرس آموخته بود شهوت راندن آدمیانه و کدویی در قضیب خر می‌کرد تا از اندازه نگذرد خاتون بر آن وقوف یافت لکن دقیقهٔ کدو را ندید کنیزک را ببهانه براه کرد جای دور و با خر جمع شد بی‌کدو و  هلاک شد بفضیحت کنیزک بیگاه باز آمد و نوحه کرد که ای جانم و ای چشم روشنم کیر دیدی کدو ندیدی ذکر دیدی آن دگر ندیدی کل ناقص ملعون یعنی کل نظر و فهم ناقص ملعون و اگر نه ناقصان ظاهر  جسم مرحوم‌اند ملعون نه‌اند بر خوان لیس علی الاعمی حرج نفی حرج  کرد و نفی لعنت و نفی عتاب و غضب}
\label{sec:sh059}
\addcontentsline{toc}{section}{\nameref{sec:sh059}}
\begin{longtable}{l p{0.5cm} r}
یک کنیزک یک خری بر خود فکند
&&
از وفور شهوت و فرط گزند
\\
آن خر نر را بگان خو کرده بود
&&
خر جماع آدمی پی برده بود
\\
یک کدویی بود حیلت‌سازه را
&&
در نرش کردی پی اندازه را
\\
در ذکر کردی کدو را آن عجوز
&&
تا رود نیم ذکر وقت سپوز
\\
گر همه کیر خر اندر وی رود
&&
آن رحم و آن روده‌ها ویران شود
\\
خر همی شد لاغر و خاتون او
&&
مانده عاجز کز چه شد این خر چو مو
\\
نعل‌بندان را نمود آن خر که چیست
&&
علت او که نتیجه‌ش لاغریست
\\
هیچ علت اندرو ظاهر نشد
&&
هیچ کس از سر او مخبر نشد
\\
در تفحص اندر افتاد او به جد
&&
شد تفحص را دمادم مستعد
\\
جد را باید که جان بنده بود
&&
زانک جد جوینده یابنده بود
\\
چون تفحص کرد از حال اشک
&&
دید خفته زیر خر آن نرگسک
\\
از شکاف در بدید آن حال را
&&
بس عجب آمد از آن آن زال را
\\
خر همی‌گاید کنیزک را چنان
&&
که به عقل و رسم مردان با زنان
\\
در حسد شد گفت چون این ممکنست
&&
پس من اولیتر که خر ملک منست
\\
خر مهذب گشته و آموخته
&&
خوان نهادست و چراغ افروخته
\\
کرد نادیده و در خانه بکوفت
&&
کای کنیزک چند خواهی خانه روفت
\\
از پی روپوش می‌گفت این سخن
&&
کای کنیزک آمدم در باز کن
\\
کرد خاموش و کنیزک را نگفت
&&
راز را از بهر طمع خود نهفت
\\
پس کنیزک جمله آلات فساد
&&
کرد پنهان پیش شد در را گشاد
\\
رو ترش کرد و دو دیده پر ز نم
&&
لب فرو مالید یعنی صایمم
\\
در کف او نرمه جاروبی که من
&&
خانه را می‌روفتم بهر عطن
\\
چونک باع جاروب در را وا گشاد
&&
گفت خاتون زیر لب کای اوستاد
\\
رو ترش کردی و جاروبی به کف
&&
چیست آن خر برگسسته از علف
\\
نیم کاره و خشمگین جنبان ذکر
&&
ز انتظار تو دو چشمش سوی در
\\
زیر لب گفت این نهان کرد از کنیز
&&
داشتش آن دم چو بی‌جرمان عزیز
\\
بعد از آن گفتش که چادر نه به سر
&&
رو فلان خانه ز من پیغام بر
\\
این چنین گو وین چنین کن وآنچنان
&&
مختصر کردم من افسانهٔ زنان
\\
آنچ مقصودست مغز آن بگیر
&&
چون براهش کرد آن زال ستیر
\\
بود از مستی شهوت شادمان
&&
در فرو بست و همی‌گفت آن زمان
\\
یافتم خلوت زنم از شکر بانگ
&&
رسته‌ام از چار دانگ و از دو دانگ
\\
از طرب گشته بزان زن هزار
&&
در شرار شهوت خر بی‌قرار
\\
چه بزان که آن شهوت او را بز گرفت
&&
بز گرفتن گیج را نبود شگفت
\\
میل شهوت کر کند دل را و کور
&&
تا نماید خر چو یوسف نار نور
\\
ای بسا سرمست نار و نارجو
&&
خویشتن را نور مطلق داند او
\\
جز مگر بندهٔ خدا یا جذب حق
&&
با رهش آرد بگرداند ورق
\\
تا بداند که آن خیال ناریه
&&
در طریقت نیست الا عاریه
\\
زشتها را خوب بنماید شره
&&
نیست چون شهوت بتر ز آفتاب ره
\\
صد هزاران نام خوش را کرد ننگ
&&
صد هزاران زیرکان را کرد دنگ
\\
چون خری را یوسف مصری نمود
&&
یوسفی را چون نماید آن جهود
\\
بر تو سرگین را فسونش شهد کرد
&&
شهد را خود چون کند وقت نبرد
\\
شهوت از خوردن بود کم کن ز خور
&&
یا نکاحی کن گریزان شو ز شر
\\
چون بخوردی می‌کشد سوی حرم
&&
دخل را خرجی بباید لاجرم
\\
پس نکاح آمد چو لاحول و لا
&&
تا که دیوت نفکند اندر بلا
\\
چون حریص خوردنی زن خواه زود
&&
ورنه آمد گربه و دنبه ربود
\\
بار سنگی بر خری که می‌جهد
&&
زود بر نه پیش از آن کو بر نهد
\\
فعل آتش را نمی‌دانی تو برد
&&
گرد آتش با چنین دانش مگرد
\\
علم دیگ و آتش ار نبود ترا
&&
از شرر نه دیگ ماند نه ابا
\\
آب حاضر باید و فرهنگ نیز
&&
تا پزد آب دیگ سالم در ازیز
\\
چون ندانی دانش آهنگری
&&
ریش و مو سوزد چو آنجا بگذری
\\
در فرو بست آن زن و خر را کشید
&&
شادمانه لاجرم کیفر چشید
\\
در میان خانه آوردش کشان
&&
خفت اندر زیر آن نر خر ستان
\\
هم بر آن کرسی که دید او از کنیز
&&
تا رسد در کام خود آن قحبه نیز
\\
پا بر آورد و خر اندر ویی سپوخت
&&
آتشی از کیر خر در وی فروخت
\\
خر مؤدب گشته در خاتون فشرد
&&
تا بخایه در زمان خاتون بمرد
\\
بر درید از زخم کیر خر جگر
&&
روده‌ها بسکسته شد از همدگر
\\
دم نزد در حال آن زن جان بداد
&&
کرسی از یک‌سو زن از یک‌سو فتاد
\\
صحن خانه پر ز خون شد زن نگون
&&
مرد او و برد جان ریب المنون
\\
مرگ بد با صد فضیحت ای پدر
&&
تو شهیدی دیده‌ای از کیر خر
\\
تو عذاب الخزی بشنو از نبی
&&
در چنین ننگی مکن جان را فدی
\\
دانک این نفس بهیمی نر خرست
&&
زیر او بودن از آن ننگین‌ترست
\\
در ره نفس ار بمیری در منی
&&
تو حقیقت دان که مثل آن زنی
\\
نفس ما را صورت خر بدهد او
&&
زانک صورتها کند بر وفق خو
\\
این بود اظهار سر در رستخیز
&&
الله الله از تن چون خر گریز
\\
کافران را بیم کرد ایزد ز نار
&&
کافران گفتند نار اولی ز عار
\\
گفت نی آن نار اصل عارهاست
&&
هم‌چو این ناری که این زن را بکاست
\\
لقمه اندازه نخورد از حرص خود
&&
در گلو بگرفت لقمه مرگ بد
\\
لقمه اندازه خور ای مرد حریص
&&
گرچه باشد لقمه حلوا و خبیص
\\
حق تعالی داد میزان را زبان
&&
هین ز قرآن سورهٔ رحمن بخوان
\\
هین ز حرص خویش میزان را مهل
&&
آز و حرص آمد ترا خصم مضل
\\
حرص جوید کل بر آید او ز کل
&&
حرص مپرست ای فجل ابن الفجل
\\
آن کنیزک می‌شد و می‌گفت آه
&&
کردی ای خاتون تو استا را به راه
\\
کار بی‌استاد خواهی ساختن
&&
جاهلانه جان بخواهی باختن
\\
ای ز من دزدیده علمی ناتمام
&&
ننگ آمد که بپرسی حال دام
\\
هم بچیدی دانه مرغ از خرمنش
&&
هم نیفتادی رسن در گردنش
\\
دانه کمتر خور مکن چندین رفو
&&
چون کلوا خواندی بخوان لا تسرفوا
\\
تا خوری دانه نیفتی تو به دام
&&
این کند علم و قناعت والسلام
\\
نعمت از دنیا خورد عاقل نه غم
&&
جاهلان محروم مانده در ندم
\\
چون در افتد در گلوشان حبل دام
&&
دانه خوردن گشت بر جمله حرام
\\
مرغ اندر دام دانه کی خورد
&&
دانه چون زهرست در دام ار چرد
\\
مرغ غافل می‌خورد دانه ز دام
&&
هم‌چو اندر دام دنیا این عوام
\\
باز مرغان خبیر هوشمند
&&
کرده‌اند از دانه خود را خشک‌بند
\\
که اندرون دام دانه زهرباست
&&
کور آن مرغی که در فخ دانه خواست
\\
صاحب دام ابلهان را سر برید
&&
وآن ظریفان را به مجلسها کشید
\\
که از آنها گوشت می‌آید به کار
&&
وز ظریفان بانگ و نالهٔ زیر و زار
\\
پس کنیزک آمد از اشکاف در
&&
دید خاتون را به مرده زیر خر
\\
گفت ای خاتون احمق این چه بود
&&
گر ترا استاد خود نقشی نمود
\\
ظاهرش دیدی سرش از تو نهان
&&
اوستا ناگشته بگشادی دکان
\\
کیر دیدی هم‌چو شهد و چون خبیص
&&
آن کدو را چون ندیدی ای حریص
\\
یا چون مستغرق شدی در عشق خر
&&
آن کدو پنهان بماندت از نظر
\\
ظاهر صنعت بدیدی زوستاد
&&
اوستادی برگرفتی شاد شاد
\\
ای بسا زراق گول بی‌وقوف
&&
از ره مردان ندیده غیر صوف
\\
ای بسا شوخان ز اندک احتراف
&&
از شهان ناموخته جز گفت و لاف
\\
هر یکی در کف عصا که موسی‌ام
&&
می‌دمد بر ابلهان که عیسی‌ام
\\
آه از آن روزی که صدق صادقان
&&
باز خواهد از تو سنگ امتحان
\\
آخر از استاد باقی را بپرس
&&
یا حریصان جمله کورانند و خرس
\\
جمله جستی باز ماندی از همه
&&
صید گرگانند این ابله رمه
\\
صورتی بنشینده گشتی ترجمان
&&
بی‌خبر از گفت خود چون طوطیان
\\
\end{longtable}
\end{center}
