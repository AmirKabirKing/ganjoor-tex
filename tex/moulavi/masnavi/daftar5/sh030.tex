\begin{center}
\section*{بخش ۳۰ - در تفسیر قول رسول علیه‌السلام ما مات من مات الا و تمنی ان یموت قبل ما مات ان کان برا لیکون الی وصول البر اعجل و ان کان فاجرا لیقل فجوره}
\label{sec:sh030}
\addcontentsline{toc}{section}{\nameref{sec:sh030}}
\begin{longtable}{l p{0.5cm} r}
زین بفرمودست آن آگه رسول
&&
که هر آنک مرد و کرد از تن نزول
\\
نبود او را حسرت نقلان و موت
&&
لیک باشد حسرت تقصیر و فوت
\\
هر که میرد خود تمنی باشدش
&&
که بدی زین پیش نقل مقصدش
\\
گر بود بد تا بدی کمتر بدی
&&
ور تقی تا خانه زوتر آمدی
\\
گوید آن بد بی‌خبر می‌بوده‌ام
&&
دم به دم من پرده می‌افزوده‌ام
\\
گر ازین زودتر مرا معبر بدی
&&
این حجاب و پرده‌ام کمتر بدی
\\
از حریصی کم دران روی قنوع
&&
وز تکبر کم دران چهرهٔ خشوع
\\
هم‌چنین از بخل کم در روی جود
&&
وز بلیسی چهرهٔ خوب سجود
\\
بر مکن آن پر خلد آرای را
&&
بر مکن آن پر ره‌پیمای را
\\
چون شنید این پند در وی بنگریست
&&
بعد از آن در نوحه آمد می‌گریست
\\
نوحه و گریهٔ دراز دردمند
&&
هر که آنجا بود بر گریه‌ش فکند
\\
وآنک می‌پرسید پر کندن ز چیست
&&
بی‌جوابی شد پشیمان می‌گریست
\\
کز فضولی من چرا پرسیدمش
&&
او ز غم پر بود شورانیدمش
\\
می‌چکید از چشم تر بر خاک آب
&&
اندر آن هر قطره مدرج صد جواب
\\
گریهٔ با صدق بر جانها زند
&&
تا که چرخ و عرش را گریان کند
\\
عقل و دلها بی‌گمان عرشی‌اند
&&
در حجاب از نور عرشی می‌زیند
\\
\end{longtable}
\end{center}
