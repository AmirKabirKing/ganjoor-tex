\begin{center}
\section*{بخش ۱۶۳ - حکایت آن مجاهد کی از همیان سیم هر روز یک درم در خندق انداختی به تفاریق از بهر ستیزهٔ حرص و آرزوی نفس و وسوسهٔ نفس کی چون می‌اندازی به خندق باری به یک‌بار بینداز تا خلاص یابم کی الیاس احدی الراحتین او گفته کی این راحت نیز ندهم}
\label{sec:sh163}
\addcontentsline{toc}{section}{\nameref{sec:sh163}}
\begin{longtable}{l p{0.5cm} r}
آن یکی بودش به کف در چل درم
&&
هر شب افکندی یکی در آب یم
\\
تا که گردد سخت بر نفس مجاز
&&
در تانی درد جان کندن دراز
\\
با مسلمانان بکر او پیش رفت
&&
وقت فر او وا نگشت از خصم تفت
\\
زخم دیگر خورد آن را هم ببست
&&
بیست کرت رمح و تیر از وی شکست
\\
بعد از آن قوت نماند افتاد پیش
&&
مقعد صدق او ز صدق عشق خویش
\\
صدق جان دادن بود هین سابقوا
&&
از نبی برخوان رجال صدقوا
\\
این همه مردن نه مرگ صورتست
&&
این بدن مر روح را چون آلتست
\\
ای بسا خامی که ظاهر خونش ریخت
&&
لیک نفس زنده آن جانب گریخت
\\
آلتش بشکست و ره‌زن زنده ماند
&&
نفس زنده‌ست ارچه مرکب خون فشاند
\\
اسپ کشت و راه او رفته نشد
&&
جز که خام و زشت و آشفته نشد
\\
گر بهر خون ریزیی گشتی شهید
&&
کافری کشته بدی هم بوسعید
\\
ای بسا نفس شهید معتمد
&&
مرده در دنیا چو زنده می‌رود
\\
روح ره‌زن مرد و تن که تیغ اوست
&&
هست باقی در کف آن غزوجوست
\\
تیغ آن تیغست مرد آن مرد نیست
&&
لیک این صورت ترا حیران کنیست
\\
نفس چون مبدل شود این تیغ تن
&&
باشد اندر دست صنع ذوالمنن
\\
آن یکی مردیست قوتش جمله درد
&&
این دگر مردی میان‌تی هم‌چو گرد
\\
\end{longtable}
\end{center}
