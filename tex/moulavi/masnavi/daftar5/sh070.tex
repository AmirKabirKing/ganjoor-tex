\begin{center}
\section*{بخش ۷۰ - جواب آمدن کی آنک نظر او بر اسباب و مرض و زخم تیغ نیاید بر کار تو عزرائیل هم نیاید کی تو هم سببی اگر چه مخفی‌تری از آن سببها و بود کی بر آن رنجور مخفی نباشد کی و هو اقرب الیه منکم و لکن لا تبصرون}
\label{sec:sh070}
\addcontentsline{toc}{section}{\nameref{sec:sh070}}
\begin{longtable}{l p{0.5cm} r}
گفت یزدان آنک باشد اصل دان
&&
پس ترا کی بیند او اندر میان
\\
گرچه خویش را عامه پنهان کرده‌ای
&&
پیش روشن‌دیدگان هم پرده‌ای
\\
وانک ایشان را شکر باشد اجل
&&
چون نظرشان مست باشد در دول
\\
تلخ نبود پیش ایشان مرگ تن
&&
چون روند از چاه و زندان در چمن
\\
وا رهیدند از جهان پیچ‌پیچ
&&
کس نگرید بر فوات هیچ هیچ
\\
برج زندان را شکست ارکانیی
&&
هیچ ازو رنجد دل زندانیی
\\
کای دریغ این سنگ مرمر را شکست
&&
تا روان و جان ما از حبس رست
\\
آن رخام خوب و آن سنگ شریف
&&
برج زندان را بهی بود و الیف
\\
چون شکستش تا که زندانی برست
&&
دست او در جرم این باید شکست
\\
هیچ زندانی نگوید این فشار
&&
جز کسی کز حبس آرندش به دار
\\
تلخ کی باشد کسی را کش برند
&&
از میان زهر ماران سوی قند
\\
جان مجرد گشته از غوغای تن
&&
می‌پرد با پر دل بی‌پای تن
\\
هم‌چو زندانی چه که اندر شبان
&&
خسپد و بیند به خواب او گلستان
\\
گوید ای یزدان مرا در تن مبر
&&
تا درین گلشن کنم من کر و فر
\\
گویدش یزدان دعا شد مستجاب
&&
وا مرو والله اعلم بالصواب
\\
این چنین خوابی ببین چون خوش بود
&&
مرگ نادیده به جنت در رود
\\
هیچ او حسرت خورد بر انتباه
&&
بر تن با سلسله در قعر چاه
\\
مؤمنی آخر در آ در صف رزم
&&
که ترا بر آسمان بودست بزم
\\
بر امید راه بالا کن قیام
&&
هم‌چو شمعی پیش محراب ای غلام
\\
اشک می‌بار و همی‌سوز از طلب
&&
هم‌چو شمع سر بریده جمله شب
\\
لب فرو بند از طعام و از شراب
&&
سوی خوان آسمانی کن شتاب
\\
دم به دم بر آسمان می‌دار امید
&&
در هوای آسمان رقصان چو بید
\\
دم به دم از آسمان می‌آیدت
&&
آب و آتش رزق می‌افزایدت
\\
گر ترا آنجا برد نبود عجب
&&
منگر اندر عجز و بنگر در طلب
\\
کین طلب در تو گروگان خداست
&&
زانک هر طالب به مطلوبی سزاست
\\
جهد کن تا این طلب افزون شود
&&
تا دلت زین چاه تن بیرون شود
\\
خلق گوید مرد مسکین آن فلان
&&
تو بگویی زنده‌ام ای غافلان
\\
گر تن من هم‌چو تن‌ها خفته است
&&
هشت جنت در دلم بشکفته است
\\
جان چو خفته در گل و نسرین بود
&&
چه غمست ار تن در آن سرگین بود
\\
جان خفته چه خبر دارد ز تن
&&
کو به گلشن خفت یا در گولخن
\\
می‌زند جان در جهان آبگون
&&
نعره یا لیت قومی یعلمون
\\
گر نخواهد زیست جان بی این بدن
&&
پس فلک ایوان کی خواهد بدن
\\
گر نخواهد بی بدن جان تو زیست
&&
فی السماء رزقکم روزی کیست
\\
\end{longtable}
\end{center}
