\begin{center}
\section*{بخش ۷ - بیان آنک نماز و روزه و همه چیزهای برونی گواهیهاست بر نور اندرونی}
\label{sec:sh007}
\addcontentsline{toc}{section}{\nameref{sec:sh007}}
\begin{longtable}{l p{0.5cm} r}
این نماز و روزه و حج و جهاد
&&
هم گواهی دادنست از اعتقاد
\\
این زکات و هدیه و ترک حسد
&&
هم گواهی دادنست از سر خود
\\
خوان و مهمانی پی اظهار راست
&&
کای مهان ما با شما گشتیم راست
\\
هدیه‌ها و ارمغان و پیش‌کش
&&
شد گواه آنک هستم با تو خوش
\\
هر کسی کوشد به مالی یا فسون
&&
چیست دارم گوهری در اندرون
\\
گوهری دارم ز تقوی یا سخا
&&
این زکات و روزه در هر دو گوا
\\
روزه گوید کرد تقوی از حلال
&&
در حرامش دان که نبود اتصال
\\
وان زکاتش گفت کو از مال خویش
&&
می‌دهد پس چون بدزدد ز اهل کیش
\\
گر بطراری کند پس دو گواه
&&
جرح شد در محکمهٔ عدل اله
\\
هست صیاد ار کند دانه نثار
&&
نه ز رحم و جود بل بهر شکار
\\
هست گربهٔ روزه‌دار اندر صیام
&&
خفته کرده خویش بهر صید خام
\\
کرده بدظن زین کژی صد قوم را
&&
کرده بدنام اهل جود و صوم را
\\
فضل حق با این که او کژ می‌تند
&&
عاقبت زین جمله پاکش می‌کند
\\
سبق برده رحمتش وان غدر را
&&
داده نوری که نباشد بدر را
\\
کوششش را شسته حق زین اختلاط
&&
غسل داده رحمت او را زین خباط
\\
تا که غفاری او ظاهر شود
&&
مغفری کلیش را غافر شود
\\
آب بهر این ببارید از سماک
&&
تا پلیدان را کند از خبث پاک
\\
\end{longtable}
\end{center}
