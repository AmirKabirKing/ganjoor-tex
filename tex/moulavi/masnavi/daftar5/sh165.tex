\begin{center}
\section*{بخش ۱۶۵ - ایثار کردن صاحب موصل آن کنیزک را بدین خلیفه تا خون‌ریز مسلمانان بیشتر نشود}
\label{sec:sh165}
\addcontentsline{toc}{section}{\nameref{sec:sh165}}
\begin{longtable}{l p{0.5cm} r}
چون رسول آمد به پیش پهلوان
&&
داد کاغذ اندرو نقش و نشان
\\
بنگر اندر کاغذ این را طالبم
&&
هین بده ورنه کنون من غالبم
\\
چون رسول آمد بگفت آن شاه نر
&&
صورتی کم گیر زود این را ببر
\\
من نیم در عهد ایمان بت‌پرست
&&
بت بر آن بت‌پرست اولیترست
\\
چونک آوردش رسول آن پهلوان
&&
گشت عاشق بر جمالش آن زمان
\\
عشق بحری آسمان بر وی کفی
&&
چون زلیخا در هوای یوسفی
\\
دور گردونها ز موج عشق دان
&&
گر نبودی عشق بفسردی جهان
\\
کی جمادی محو گشتی در نبات
&&
کی فدای روح گشتی نامیات
\\
روح کی گشتی فدای آن دمی
&&
کز نسیمش حامله شد مریمی
\\
هر یکی بر جا ترنجیدی چو یخ
&&
کی بدی پران و جویان چون ملخ
\\
ذره ذره عاشقان آن کمال
&&
می‌شتابد در علو هم‌چون نهال
\\
سبح لله هست اشتابشان
&&
تنقیهٔ تن می‌کنند از بهر جان
\\
پهلوان چه را چو ره پنداشته
&&
شوره‌اش خوش آمده حب کاشته
\\
چون خیالی دید آن خفته به خواب
&&
جفت شد با آن و از وی رفت آب
\\
چون برفت آن خواب و شد بیدار زود
&&
دید که آن لعبت به بیداری نبود
\\
گفت بر هیچ آب خود بردم دریغ
&&
عشوهٔ آن عشوه‌ده خوردم دریغ
\\
پهلوان تن بد آن مردی نداشت
&&
تخم مردی در چنان ریگی بکاشت
\\
مرکب عشقش دریده صد لگام
&&
نعره می‌زد لا ابالی بالحمام
\\
ایش ابالی بالخلیفه فی‌الهوی
&&
استوی عندی وجودی والتوی
\\
این چنین سوزان و گرم آخر مکار
&&
مشورت کن با یکی خاوندگار
\\
مشورت کو عقل کو سیلاب آز
&&
در خرابی کرد ناخنها دراز
\\
بین ایدی سد و سوی خلف سد
&&
پیش و پس کم بیند آن مفتون خد
\\
آمده در قصدجان سیل سیاه
&&
تا که روبه افکند شیری به چاه
\\
از چهی بنموده معدومی خیال
&&
تا در اندازد اسودا کالجبال
\\
هیچ‌کس را با زنان محرم مدار
&&
که مثال این دو پنبه‌ست و شرار
\\
آتشی باید بشسته ز آب حق
&&
هم‌چو یوسف معتصم اندر زهق
\\
کز زلیخای لطیف سروقد
&&
هم‌چو شیران خویشتن را واکشد
\\
بازگشت از موصل و می‌شد به راه
&&
تا فرود آمد به بیشه و مرج‌گاه
\\
آتش عشقش فروزان آن چنان
&&
که نداند او زمین از آسمان
\\
قصد آن مه کرد اندر خیمه او
&&
عقل کو و از خلیفه خوف کو
\\
چون زند شهوت درین وادی دهل
&&
چیست عقل تو فجل ابن الفجل
\\
صد خلیفه گشته کمتر از مگس
&&
پیش چشم آتشینش آن نفس
\\
چون برون انداخت شلوار و نشست
&&
در میان پای زن آن زن‌پرست
\\
چون ذکر سوی مقر می‌رفت راست
&&
رستخیز و غلغل از لشکر بخاست
\\
برجهید و کون‌برهنه سوی صف
&&
ذوالفقاری هم‌چو آتش او به کف
\\
دید شیر نر سیه از نیستان
&&
بر زده بر قلب لشکر ناگهان
\\
تازیان چون دیو در جوش آمده
&&
هر طویله و خیمه اندر هم زده
\\
شیر نر گنبذ همی‌کرد از لغز
&&
در هوا چون موج دریا بیست گز
\\
پهلوان مردانه بود و بی‌حذر
&&
پیش شیر آمد چو شیر مست نر
\\
زد به شمشیر و سرش را بر شکافت
&&
زود سوی خیمهٔ مه‌رو شتافت
\\
چونک خود را او بدان حوری نمود
&&
مردی او هم‌چنین بر پای بود
\\
با چنان شیری به چالش گشت جفت
&&
مردی او مانده بر پای و نخفت
\\
آن بت شیرین‌لقای ماه‌رو
&&
در عجب در ماند از مردی او
\\
جفت شد با او به شهوت آن زمان
&&
متحد گشتند حالی آن دو جان
\\
ز اتصال این دو جان با همدگر
&&
می‌رسد از غیبشان جانی دگر
\\
رو نماید از طریق زادنی
&&
گر نباشد از علوقش ره‌زنی
\\
هر کجا دو کس به مهری یا به کین
&&
جمع آید ثالثی زاید یقین
\\
لیک اندر غیب زاید آن صور
&&
چون روی آن سو ببینی در نظر
\\
آن نتایج از قرانات تو زاد
&&
هین مگرد از هر قرینی زود شاد
\\
منتظر می‌باش آن میقات را
&&
صدق دان الحاق ذریات را
\\
کز عمل زاییده‌اند و از علل
&&
هر یکی را صورت و نطق و طلل
\\
بانگشان درمی‌رسد زان خوش حجال
&&
کای ز ما غافل هلا زوتر تعال
\\
منتظر در غیب جان مرد و زن
&&
مول مولت چیست زوتر گام زن
\\
راه گم کرد او از آن صبح دروغ
&&
چون مگس افتاد اندر دیگ دوغ
\\
\end{longtable}
\end{center}
