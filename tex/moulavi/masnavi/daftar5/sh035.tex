\begin{center}
\section*{بخش ۳۵ - در بیان آنک ما سوی الله هر چیزی آکل و ماکولست هم‌چون آن مرغی کی قصد صید ملخ می‌کرد و به صید ملخ مشغول می‌بود و غافل بود از باز گرسنه کی از پس قفای او قصد صید او داشت اکنون ای آدمی صیاد آکل از صیاد و آکل خود آمن مباش اگر چه نمی‌بینیش به نظر چشم به نظر دلیل و عبرتش می‌بین تا چشم نیز باز شدن}
\label{sec:sh035}
\addcontentsline{toc}{section}{\nameref{sec:sh035}}
\begin{longtable}{l p{0.5cm} r}
مرغکی اندر شکار کرم بود
&&
گربه فرصت یافت او را در ربود
\\
آکل و ماکول بود و بی‌خبر
&&
در شکار خود ز صیادی دگر
\\
دزد گرچه در شکار کاله‌ایست
&&
شحنه با خصمانش در دنباله‌ایست
\\
عقل او مشغول رخت و قفل و در
&&
غافل از شحنه‌ست و از آه سحر
\\
او چنان غرقست در سودای خود
&&
غافلست از طالب و جویای خود
\\
گر حشیش آب و هوایی می‌خورد
&&
معدهٔ حیوانش در پی می‌چرد
\\
آکل و ماکول آمد آن گیاه
&&
هم‌چنین هر هستیی غیر اله
\\
و هو یطعمکم و لا یطعم چو اوست
&&
نیست حق ماکول و آکل لحم و پوست
\\
آکل و ماکول کی ایمن بود
&&
ز آکلی که اندر کمین ساکن بود
\\
امن ماکولان جذوب ماتمست
&&
رو بدان درگاه کو لا یطعم است
\\
هر خیالی را خیالی می‌خورد
&&
فکر آن فکر دگر را می‌چرد
\\
تو نتانی کز خیالی وا رهی
&&
یا بخسپی که از آن بیرون جهی
\\
فکر زنبورست و آن خواب تو آب
&&
چون شوی بیدار باز آید ذباب
\\
چند زنبور خیالی در پرد
&&
می‌کشد این سو و آن سو می‌برد
\\
کمترین آکلانست این خیال
&&
وآن دگرها را شناسد ذوالجلال
\\
هین گریز از جوق اکال غلیظ
&&
سوی او که گفت ما ایمت حفیظ
\\
یا به سوی آن که او آن حفظ یافت
&&
گر نتانی سوی آن حافظ شتافت
\\
دست را مسپار جز در دست پیر
&&
حق شدست آن دست او را دستگیر
\\
پیر عقلت کودکی خو کرده است
&&
از جوار نفس که اندر پرده است
\\
عقل کامل را قرین کن با خرد
&&
تا که باز آید خرد زان خوی بد
\\
چونک دست خود به دست او نهی
&&
پس ز دست آکلان بیرون جهی
\\
دست تو از اهل آن بیعت شود
&&
که یدالله فوق ایدیهم بود
\\
چون بدادی دست خود در دست پیر
&&
پیر حکمت که علیمست و خطیر
\\
کو نبی وقت خویشست ای مرید
&&
تا ازو نور نبی آید پدید
\\
در حدیبیه شدی حاضر بدین
&&
وآن صحابهٔ بیعتی را هم‌قرین
\\
پس ز ده یار مبشر آمدی
&&
هم‌چو زر ده‌دهی خالص شدی
\\
تا معیت راست آید زانک مرد
&&
با کسی جفتست کو را دوست کرد
\\
این جهان و آن جهان با او بود
&&
وین حدیث احمد خوش‌خو بود
\\
گفت المرء مع محبوبه
&&
لا یفک القلب من مطلوبه
\\
هر کجا دامست و دانه کم نشین
&&
رو زبون‌گیرا زبون‌گیران ببین
\\
ای زبون‌گیر زبونان این بدان
&&
دست هم بالای دستست ای جوان
\\
تو زبونی و زبون‌گیر ای عجب
&&
هم تو صید و صیدگیر اندر طلب
\\
بین ایدی خلفهم سدا مباش
&&
که نبینی خصم را وآن خصم فاش
\\
حرص صیادی ز صیدی مغفلست
&&
دلبریی می‌کند او بی‌دلست
\\
تو کم از مرغی مباش اندر نشید
&&
بین ایدی خلف عصفوری بدید
\\
چون به نزد دانه آید پیش و پس
&&
چند گرداند سر و رو آن نفس
\\
کای عجب پیش و پسم صیاد هست
&&
تا کشم از بیم او زین لقمه دست
\\
تو ببین پس قصهٔ فجار را
&&
پیش بنگر مرگ یار و جار را
\\
که هلاکت دادشان بی‌آلتی
&&
او قرین تست در هر حالتی
\\
حق شکنجه کرد و گرز و دست نیست
&&
پس بدان بی‌دست حق داورکنیست
\\
آنک می‌گفتی اگر حق هست کو
&&
در شکنجه او مقر می‌شد که هو
\\
آنک می‌گفت این بعیدست و عجیب
&&
اشک می‌راند و همی گفت ای قریب
\\
چون فرار از دام واجب دیده است
&&
دام تو خود بر پرت چفسیده است
\\
بر کنم من میخ این منحوس دام
&&
از پی کامی نباشم طلخ‌کام
\\
درخور عقل تو گفتم این جواب
&&
فهم کن وز جست و جو رو بر متاب
\\
بسکل این حبلی که حرص است و حسد
&&
یاد کن فی جیدها حبل مسد
\\
\end{longtable}
\end{center}
