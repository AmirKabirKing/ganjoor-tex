\begin{center}
\section*{بخش ۱۱۲ - جواب گفتن روبه خر را}
\label{sec:sh112}
\addcontentsline{toc}{section}{\nameref{sec:sh112}}
\begin{longtable}{l p{0.5cm} r}
گفت روبه صاف ما را درد نیست
&&
لیک تخییلات وهمی خورد نیست
\\
این همه وهم توست ای ساده‌دل
&&
ورنه بر تو نه غشی دارم نه غل
\\
از خیال زشت خود منگر به من
&&
بر محبان از چه داری سؤ ظن
\\
ظن نیکو بر بر اخوان صفا
&&
گرچه آید ظاهرا زیشان جفا
\\
این خیال و وهم بد چون شد پدید
&&
صد هزاران یار را از هم برید
\\
مشفقی گر کرد جور و امتحان
&&
عقل باید که نباشد بدگمان
\\
خصاه من بدرگ نبودم زشت‌اسم
&&
آنک دیدی بد نبد بود آن طلسم
\\
ور بدی بد آن سگالش قدرا
&&
عفو فرمایند یاران زان خطا
\\
عالم وهم و خیال طمع و بیم
&&
هست ره‌رو را یکی سدی عظیم
\\
نقشهای این خیال نقش‌بند
&&
چون خلیلی را که که بد شد گزند
\\
گفت هذا ربی ابراهیم راد
&&
چونک اندر عالم وهم اوفتاد
\\
ذکر کوکب را چنین تاویل گفت
&&
آن کسی که گوهر تاویل سفت
\\
عالم وهم و خیال چشم‌بند
&&
آنچنان که را ز جای خویش کند
\\
تا که هذا ربی آمد قال او
&&
خربط و خر را چه باشد حال او
\\
غرق گشته عقلهای چون جبال
&&
در بحار وهم و گرداب خیال
\\
کوهها را هست زین طوفان فضوح
&&
کو امانی جز که در کشتی نوح
\\
زین خیال ره‌زن راه یقین
&&
گشت هفتاد و دو ملت اهل دین
\\
مرد ایقان رست از وهم و خیال
&&
موی ابرو را نمی‌گوید هلال
\\
وآنک نور عمرش نبود سند
&&
موی ابروی کژی راهش زند
\\
صد هزاران کشتی با هول و سهم
&&
تخته تخته گشته در دریای وهم
\\
کمترین فرعون چست فیلسوف
&&
ماه او در برج وهمی در خسوف
\\
کس نداند روسپی‌زن کیست آن
&&
وانک داند نیستش بر خود گمان
\\
چون ترا وهم تو دارد خیره‌سر
&&
از چه گردی گرد وهم آن دگر
\\
عاجزم من از منی خویشتن
&&
چه نشستی پر منی تو پیش من
\\
بی‌من و مایی همی‌جویم به جان
&&
تا شوم من گوی آن خوش صولجان
\\
هر که بی‌من شد همه من‌ها خود اوست
&&
دوست جمله شد چو خود را نیست دوست
\\
آینه بی‌نقش شد یابد بها
&&
زانک شد حاکی جمله نقشها
\\
\end{longtable}
\end{center}
