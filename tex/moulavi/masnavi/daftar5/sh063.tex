\begin{center}
\section*{بخش ۶۳ - بیان آنک عطای حق و قدرت موقوف قابلیت نیست هم‌چون داد خلقان کی آن را قابلیت باید  زیرا عطا قدیم است و قابلیت حادث عطا صفت حق است و قابلیت صفت مخلوق و قدیم موقوف حادث نباشد و اگر نه حدوث محال باشد}
\label{sec:sh063}
\addcontentsline{toc}{section}{\nameref{sec:sh063}}
\begin{longtable}{l p{0.5cm} r}
چارهٔ آن دل عطای مبدلیست
&&
داد او را قابلیت شرط نیست
\\
بلک شرط قابلیت داد اوست
&&
داد لب و قابلیت هست پوست
\\
اینک موسی را عصا ثعبان شود
&&
هم‌چو خورشیدی کفش رخشان شود
\\
صد هزاران معجزات انبیا
&&
که آن نگنجد در ضمیر و عقل ما
\\
نیست از اسباب تصریف خداست
&&
نیستها را قابلیت از کجاست
\\
قابلی گر شرط فعل حق بدی
&&
هیچ معدومی به هستی نامدی
\\
سنتی بنهاد و اسباب و طرق
&&
طالبان را زیر این ازرق تتق
\\
بیشتر احوال بر سنت رود
&&
گاه قدرت خارق سنت شود
\\
سنت و عادت نهاده با مزه
&&
باز کرده خرق عادت معجزه
\\
بی‌سبب گر عز به ما موصول نیست
&&
قدرت از عزل سبب معزول نیست
\\
ای گرفتار سبب بیرون مپر
&&
لیک عزل آن مسبب ظن مبر
\\
هر چه خواهد آن مسبب آورد
&&
قدرت مطلق سببها بر درد
\\
لیک اغلب بر سبب راند نفاذ
&&
تا بداند طالبی جستن مراد
\\
چون سبب نبود چه ره جوید مرید
&&
پس سبب در راه می‌باید بدید
\\
این سببها بر نظرها پرده‌هاست
&&
که نه هر دیدار صنعش را سزاست
\\
دیده‌ای باید سبب سوراخ کن
&&
تا حجب را بر کند از بیخ و بن
\\
تا مسبب بیند اندر لامکان
&&
هرزه داند جهد و اکساب و دکان
\\
از مسبب می‌رسد هر خیر و شر
&&
نیست اسباب و وسایط ای پدر
\\
جز خیالی منعقد بر شاه‌راه
&&
تا بماند دور غفلت چند گاه
\\
\end{longtable}
\end{center}
