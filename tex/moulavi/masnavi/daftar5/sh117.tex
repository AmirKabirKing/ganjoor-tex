\begin{center}
\section*{بخش ۱۱۷ - گریان شدن امیر از نصیحت شیخ و عکس صدق او و ایثار کردن مخزن بعد از آن گستاخی و استعصام شیخ و قبول ناکردن و گفتن کی من بی‌اشارت نیارم تصرفی کردن}
\label{sec:sh117}
\addcontentsline{toc}{section}{\nameref{sec:sh117}}
\begin{longtable}{l p{0.5cm} r}
این بگفت و گریه در شد های های
&&
اشک غلطان بر رخ او جای جای
\\
صدق او هم بر ضمیر میر زد
&&
عشق هر دم طرفه دیگی می‌پزد
\\
صدق عاشق بر جمادی می‌تند
&&
چه عجب گر بر دل دانا زند
\\
صدق موسی بر عصا و کوه زد
&&
بلک بر دریای پر اشکوه زد
\\
صدق احمد بر جمال ماه زد
&&
بلک بر خورشید رخشان راه زد
\\
رو برو آورده هر دو در نفیر
&&
گشته گریان هم امیر و هم فقیر
\\
ساعتی بسیار چون بگریستند
&&
گفت میر او را که خیز ای ارجمند
\\
هر چه خواهی از خزانه برگزین
&&
گرچه استحقاق داری صد چنین
\\
خانه آن تست هر چت میل هست
&&
بر گزین خود هر دو عالم اندکست
\\
گفت دستوری ندادندم چنین
&&
که کنم من این دخیلانه دخول
\\
این بهانه کرد و مهره در ربود
&&
مانع آن بدکان عطا صادق نبود
\\
نه که صادق بود و پاک از غل و خشم
&&
شیخ را هر صدق می‌نامد به چشم
\\
گفت فرمانم چنین دادست اله
&&
که گدایانه برو نانی بخواه
\\
\end{longtable}
\end{center}
