\begin{center}
\section*{بخش ۸ - پاک کردن آب همه پلیدیها را و باز پاک کردن خدای تعالی آب را از پلیدی لاجرم قدوس آمد حق تعالی}
\label{sec:sh008}
\addcontentsline{toc}{section}{\nameref{sec:sh008}}
\begin{longtable}{l p{0.5cm} r}
آب چون پیگار کرد و شد نجس
&&
تا چنان شد که آب را رد کرد حس
\\
حق ببردش باز در بحر صواب
&&
تا به شستش از کرم آن آب آب
\\
سال دیگر آمد او دامن‌کشان
&&
هی کجا بودی به دریای خوشان
\\
من نجس زینجا شدم پاک آمدم
&&
بستدم خلعت سوی خاک آمدم
\\
هین بیایید ای پلیدان سوی من
&&
که گرفت از خوی یزدان خوی من
\\
در پذیرم جملهٔ زشتیت را
&&
چون ملک پاکی دهم عفریت را
\\
چون شوم آلوده باز آنجا روم
&&
سوی اصل اصل پاکیها رو
\\
دلق چرکین بر کنم آنجا ز سر
&&
خلعت پاکم دهد بار دگر
\\
کار او اینست و کار من همین
&&
عالم‌آرایست رب العالمین
\\
گر نبودی این پلیدیهای ما
&&
کی بدی این بارنامه آب را
\\
کیسه‌های زر بدزدید از کسی
&&
می‌رود هر سو که هین کو مفلسی
\\
یا بریزد بر گیاه رسته‌ای
&&
یا بشوید روی رو ناشسته‌ای
\\
یا بگیرد بر سر او حمال‌وار
&&
کشتی بی‌دست و پا را در بحار
\\
صد هزاران دارو اندر وی نهان
&&
زانک هر دارو بروید زو چنان
\\
جان هر دری دل هر دانه‌ای
&&
می‌رود در جو چو داروخانه‌ای
\\
زو یتیمان زمین را پرورش
&&
بستگان خشک را از وی روش
\\
چون نماند مایه‌اش تیره شود
&&
هم‌چو ما اندر زمین خیره شود
\\
\end{longtable}
\end{center}
