\begin{center}
\section*{بخش ۲۳ - حکایت آن اعرابی کی سگ او از گرسنگی می‌مرد و انبان او پر نان و بر سگ نوحه می‌کرد و شعر می‌گفت و می‌گریست و سر و رو می‌زد و دریغش می‌آمد لقمه‌ای از انبان به سگ دادن}
\label{sec:sh023}
\addcontentsline{toc}{section}{\nameref{sec:sh023}}
\begin{longtable}{l p{0.5cm} r}
آن سگی می‌مرد و گریان آن عرب
&&
اشک می‌بارید و می‌گفت ای کرب
\\
سایلی بگذشت و گفت این گریه چیست
&&
نوحه و زاری تو از بهر کیست
\\
گفت در ملکم سگی بد نیک‌خو
&&
نک همی‌میرد میان راه او
\\
روز صیادم بد و شب پاسبان
&&
تیزچشم و صیدگیر و دزدران
\\
گفت رنجش چیست زخمی خورده است
&&
گفت جوع الکلب زارش کرده است
\\
گفت صبری کن برین رنج و حرض
&&
صابران را فضل حق بخشد عوض
\\
بعد از آن گفتش کای سالار حر
&&
چیست اندر دستت این انبان پر
\\
گفت نان و زاد و لوت دوش من
&&
می‌کشانم بهر تقویت بدن
\\
گفت چون ندهی بدان سگ نان و زاد
&&
گفت تا این حد ندارم مهر و داد
\\
دست ناید بی‌درم در راه نان
&&
لیک هست آب دو دیده رایگان
\\
گفت خاکت بر سر ای پر باد مشک
&&
که لب نان پیش تو بهتر ز اشک
\\
اشک خونست و به غم آبی شده
&&
می‌نیرزد خاک خون بیهده
\\
کل خود را خوار کرد او چون بلیس
&&
پارهٔ این کل نباشد جز خسیس
\\
من غلام آنک نفروشد وجود
&&
جز بدان سلطان با افضال و جود
\\
چون بگرید آسمان گریان شود
&&
چون بنالد چرخ یا رب خوان شود
\\
من غلام آن مس همت‌پرست
&&
کو به غیر کیمیا نارد شکست
\\
دست اشکسته برآور در دعا
&&
سوی اشکسته پرد فضل خدا
\\
گر رهایی بایدت زین چاه تنگ
&&
ای برادر رو بر آذر بی‌درنگ
\\
مکر حق را بین و مکر خود بهل
&&
ای ز مکرش مکر مکاران خجل
\\
چونک مکرت شد فنای مکر رب
&&
برگشایی یک کمینی بوالعجب
\\
که کمینهٔ آن کمین باشد بقا
&&
تا ابد اندر عروج و ارتقا
\\
\end{longtable}
\end{center}
