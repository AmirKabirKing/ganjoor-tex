\begin{center}
\section*{بخش ۲۸ - در بیان قول رسول علیه‌السلام لا رهبانیة فی‌الاسلام}
\label{sec:sh028}
\addcontentsline{toc}{section}{\nameref{sec:sh028}}
\begin{longtable}{l p{0.5cm} r}
بر مکن پر را و دل بر کن ازو
&&
زانک شرط این جهاد آمد عدو
\\
چون عدو نبود جهاد آمد محال
&&
شهوتت نبود نباشد امتثال
\\
صبر نبود چون نباشد میل تو
&&
خصم چون نبود چه حاجت حیل تو
\\
هین مکن خود را خصی رهبان مشو
&&
زانک عفت هست شهوت را گرو
\\
بی‌هوا نهی از هوا ممکن نبود
&&
غازیی بر مردگان نتوان نمود
\\
انفقوا گفتست پس کسپی بکن
&&
زانک نبود خرج بی‌دخل کهن
\\
گر چه آورد انفقوا را مطلق او
&&
تو بخوان که اکسبوا ثم انفقوا
\\
هم‌چنان چون شاه فرمود اصبروا
&&
رغبتی باید کزان تابی تو رو
\\
پس کلوا از بهر دام شهوتست
&&
بعد از آن لاتسرفوا آن عفتست
\\
چونک محمول به نبود لدیه
&&
نیست ممکن بود محمول علیه
\\
چونک رنج صبر نبود مر ترا
&&
شرط نبود پس فرو ناید جزا
\\
حبذا آن شرط و شادا آن جزا
&&
آن جزای دل‌نواز جان‌فزا
\\
\end{longtable}
\end{center}
