\begin{center}
\section*{بخش ۴۷ - در تفسیر قول مصطفی علیه‌السلام لا بد من قرین یدفن معک و هو حی و  تدفن معه و انت میت ان کان کریما اکرمک و ان کان لیما اسلمک و ذلک القرین عملک فاصلحه ما استطعت صدق رسول‌الله}
\label{sec:sh047}
\addcontentsline{toc}{section}{\nameref{sec:sh047}}
\begin{longtable}{l p{0.5cm} r}
پس پیمبر گفت بهر این طریق
&&
باوفاتر از عمل نبود رفیق
\\
گر بود نیکو ابد یارت شود
&&
ور بود بد در لحد مارت شود
\\
این عمل وین کسب در راه سداد
&&
کی توان کرد ای پدر بی‌اوستاد
\\
دون‌ترین کسبی که در عالم رود
&&
هیچ بی‌ارشاد استادی بود
\\
اولش علمست آنگاهی عمل
&&
تا دهد بر بعد مهلت یا اجل
\\
استعینوا فی‌الحرف یا ذا النهی
&&
من کریم صالح من اهلها
\\
اطلب الدر اخی وسط الصدف
&&
واطلب الفن من ارباب الحرف
\\
ان رایتم ناصحین انصفوا
&&
بادروا التعلیم لا تستنکفوا
\\
در دباغی گر خلق پوشید مرد
&&
خواجگی خواجه را آن کم نکرد
\\
وقت دم آهنگر ار پوشید دلق
&&
احتشام او نشد کم پیش خلق
\\
پس لباس کبر بیرون کن ز تن
&&
ملبس ذل پوش در آموختن
\\
علم آموزی طریقش قولی است
&&
حرفت آموزی طریقش فعلی است
\\
فقر خواهی آن به صحبت قایمست
&&
نه زبانت کار می‌آید نه دست
\\
دانش آن را ستاند جان ز جان
&&
نه ز راه دفتر و نه از زبان
\\
در دل سالک اگر هست آن رموز
&&
رمزدانی نیست سالک را هنوز
\\
تا دلش را شرح آن سازد ضیا
&&
پس الم نشرح بفرماید خدا
\\
که درون سینه شرحت داده‌ایم
&&
شرح اندر سینه‌ات بنهاده‌ایم
\\
تو هنوز از خارج آن را طالبی
&&
محلبی از دیگران چون حالبی
\\
چشمهٔ شیرست در تو بی‌کنار
&&
تو چرا می‌شیر جویی از تغار
\\
منفذی داری به بحر ای آبگیر
&&
ننگ دار از آب جستن از غدیر
\\
که الم نشرح نه شرحت هست باز
&&
چون شدی تو شرح‌جو و کدیه‌ساز
\\
در نگر در شرح دل در اندرون
&&
تا نیاید طعنهٔ لا تبصرون
\\
\end{longtable}
\end{center}
