\begin{center}
\section*{بخش ۸۷ - در بیان کسی کی سخنی گوید کی حال او مناسب آن سخن و آن دعوی نباشد چنان که کفره و لن سالتهم من خلق السموات والارض لیقولن الله خدمت بت سنگین کردن و جان و زر فدای او کردن چه مناسب باشد با جانی کی داند کی خالق سموات و ارض و خلایق الهیست سمیعی بصیری حاضری مراقبی مستولی غیوری الی آخره}
\label{sec:sh087}
\addcontentsline{toc}{section}{\nameref{sec:sh087}}
\begin{longtable}{l p{0.5cm} r}
زاهدی را یک زنی بد بس غیور
&&
هم بد او را یک کنیزک هم‌چو حور
\\
زان ز غیرت پاس شوهر داشتی
&&
با کنیزک خلوتش نگذاشتی
\\
مدتی زن شد مراقب هر دو را
&&
تاکشان فرصت نیفتد در خلا
\\
تا در آمد حکم و تقدیر اله
&&
عقل حارس خیره‌سر گشت و تباه
\\
حکم و تقدیرش چو آید بی‌وقوف
&&
عقل کی بود در قمر افتد خسوف
\\
بود در حمام آن زن ناگهان
&&
یادش آمد طشت و در خانه بد آن
\\
با کنیزک گفت رو هین مرغ‌وار
&&
طشت سیمین را ز خانهٔ ما بیار
\\
آن کنیزک زنده شد چون این شنید
&&
که به خواجه این زمان خواهد رسید
\\
خواجه در خانه‌ست و خلوت این زمان
&&
پس دوان شد سوی خانه شادمان
\\
عشق شش ساله کنیزک را بد این
&&
که بیابد خواجه را خلوت چنین
\\
گشت پران جانب خانه شتافت
&&
خواجه را در خانه در خلوت بیافت
\\
هر دو عاشق را چنان شهوت ربود
&&
که احتیاط و یاد در بستن نبود
\\
هر دو با هم در خزیدند از نشاط
&&
جان به جان پیوست آن دم ز اختلاط
\\
یاد آمد در زمان زن را که من
&&
چون فرستادم ورا سوی وطن
\\
پنبه در آتش نهادم من به خویش
&&
اندر افکندم قج نر را به میش
\\
گل فرو شست از سر و بی‌جان دوید
&&
در پی او رفت و چادر می‌کشید
\\
آن ز عشق جان دوید و این ز بیم
&&
عشق کو و بیم کو فرقی عظیم
\\
سیر عارف هر دمی تا تخت شاه
&&
سیر زاهد هر مهی یک روزه راه
\\
گرچه زاهد را بود روزی شگرف
&&
کی بود یک روز او خمسین الف
\\
قدر هر روزی ز عمر مرد کار
&&
باشد از سال جهان پنجه هزار
\\
عقلها زین سر بود بیرون در
&&
زهرهٔ وهم ار بدرد گو بدر
\\
ترس مویی نیست اندر پیش عشق
&&
جمله قربانند اندر کیش عشق
\\
عشق وصف ایزدست اما که خوف
&&
وصف بندهٔ مبتلای فرج و جوف
\\
چون یحبون بخواندی در نبی
&&
با یحبوهم قرین در مطلبی
\\
پس محبت وصف حق دان عشق نیز
&&
خوف نبود وصف یزدان ای عزیز
\\
وصف حق کو وصف مشتی خاک کو
&&
وصف حادث کو وصف پاک کو
\\
شرح عشق ار من بگویم بر دوام
&&
صد قیامت بگذرد و آن ناتمام
\\
زانک تاریخ قیامت را حدست
&&
حد کجا آنجا که وصف ایزدست
\\
عشق را پانصد پرست و هر پری
&&
از فراز عرش تا تحت‌الثری
\\
زاهد با ترس می‌تازد به پا
&&
عاشقان پران‌تر از برق و هوا
\\
کی رسند این خایفان در گرد عشق
&&
که آسمان را فرش سازد درد عشق
\\
جز مگر آید عنایتهای ضو
&&
کز جهان و زین روش آزاد شو
\\
از قش خود وز دش خود باز ره
&&
که سوی شه یافت آن شهباز ره
\\
این قش و دش هست جبر و اختیار
&&
از ورای این دو آمد جذب یار
\\
چون رسید آن زن به خانه در گشاد
&&
بانگ در در گوش ایشان در فتاد
\\
آن کنیزک جست آشفته ز ساز
&&
مرد بر جست و در آمد در نماز
\\
زن کنیزک را پژولیده بدید
&&
درهم و آشفته و دنگ و مرید
\\
شوی خود را دید قایم در نماز
&&
در گمان افتاد زن زان اهتزاز
\\
شوی را برداشت دامن بی‌خطر
&&
دید آلودهٔ منی خصیه و ذکر
\\
از ذکر باقی نطفه می‌چکید
&&
ران و زانو گشت آلوده و پلید
\\
بر سرش زد سیلی و گفت ای مهین
&&
خصیهٔ مرد نمازی باشد این
\\
لایق ذکر و نمازست این ذکر
&&
وین چنین ران و زهار پر قذر
\\
نامهٔ پر ظلم و فسق و کفر و کین
&&
لایقست انصاف ده اندر یمین
\\
گر بپرسی گبر را کین آسمان
&&
آفریدهٔ کیست وین خلق و جهان
\\
گوید او کین آفریدهٔ آن خداست
&&
که آفرینش بر خدایی‌اش گواست
\\
کفر و فسق و استم بسیار او
&&
هست لایق با چنین اقرار او
\\
هست لایق با چنین اقرار راست
&&
آن فضیحتها و آن کردار کاست
\\
فعل او کرده دروغ آن قول را
&&
تا شد او لایق عذاب هول را
\\
روز محشر هر نهان پیدا شود
&&
هم ز خود هر مجرمی رسوا شود
\\
دست و پا بدهد گواهی با بیان
&&
بر فساد او به پیش مستعان
\\
دست گوید من چنین دزدیده‌ام
&&
لب بگوید من چنین پرسیده‌ام
\\
پای گوید من شدستم تا منی
&&
فرج گوید من بکردستم زنی
\\
چشم گوید کرده‌ام غمزهٔ حرام
&&
گوش گوید چیده‌ام س الکلام
\\
پس دروغ آمد ز سر تا پای خویش
&&
که دروغش کرد هم اعضای خویش
\\
آنچنان که در نماز با فروغ
&&
از گواهی خصیه شد زرقش دروغ
\\
پس چنان کن فعل که آن خود بی‌زبان
&&
باشد اشهد گفتن و عین بیان
\\
تا همه تن عضو عضوت ای پسر
&&
گفته باشد اشهد اندر نفع و ضر
\\
رفتن بنده پی خواجه گواست
&&
که منم محکوم و این مولای ماست
\\
گر سیه کردی تو نامهٔ عمر خویش
&&
توبه کن زانها که کردستی تو پیش
\\
عمر اگر بگذشت بیخش این دمست
&&
آب توبه‌ش ده اگر او بی‌نمست
\\
بیخ عمرت را بده آب حیات
&&
تا درخت عمر گردد با نبات
\\
جمله ماضیها ازین نیکو شوند
&&
زهر پارینه ازین گردد چو قند
\\
سیئاتت را مبدل کرد حق
&&
تا همه طاعت شود آن ما سبق
\\
خواجه بر توبهٔ نصوحی خوش به تن
&&
کوششی کن هم به جان و هم به تن
\\
شرح این توبهٔ نصوح از من شنو
&&
بگرویدستی و لیک از نو گرو
\\
\end{longtable}
\end{center}
