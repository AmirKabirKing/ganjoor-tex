\begin{center}
\section*{بخش ۸۸ - حکایت در بیان توبهٔ نصوح کی چنانک شیر از پستان بیرون آید باز در پستان نرود آنک توبه نصوحی کرد هرگز از آن گناه یاد نکند به طریق رغبت بلک هر دم نفرتش افزون باشد و آن نفرت دلیل آن بود  کی لذت قبول یافت آن شهوت اول بی‌لذت شد این به جای آن نشست نبرد عشق را جز عشق دیگر  چرا یاری نجویی زو نکوتر وانک دلش باز بدان گناه رغبت می‌کند علامت آنست کی لذت قبول نیافته است و لذت قبول به جای آن لذت گناه ننشسته است سنیسره للیسری نشده است لذت و نیسره للعسری باقیست بر وی}
\label{sec:sh088}
\addcontentsline{toc}{section}{\nameref{sec:sh088}}
\begin{longtable}{l p{0.5cm} r}
بود مردی پیش ازین نامش نصوح
&&
بد ز دلاکی زن او را فتوح
\\
بود روی او چو رخسار زنان
&&
مردی خود را همی‌کرد او نهان
\\
او به حمام زنان دلاک بود
&&
در دغا و حیله بس چالاک بود
\\
سالها می‌کرد دلاکی و کس
&&
بو نبرد از حال و سر آن هوس
\\
زانک آواز و رخش زن‌وار بود
&&
لیک شهوت کامل و بیدار بود
\\
چادر و سربند پوشیده و نقاب
&&
مرد شهوانی و در غرهٔ شباب
\\
دختران خسروان را زین طریق
&&
خوش همی‌مالید و می‌شست آن عشیق
\\
توبه‌ها می‌کرد و پا در می‌کشید
&&
نفس کافر توبه‌اش را می‌درید
\\
رفت پیش عارفی آن زشت‌کار
&&
گفت ما را در دعایی یاد دار
\\
سر او دانست آن آزادمرد
&&
لیک چون حلم خدا پیدا نکرد
\\
بر لبش قفلست و در دل رازها
&&
لب خموش و دل پر از آوازها
\\
عارفان که جام حق نوشیده‌اند
&&
رازها دانسته و پوشیده‌اند
\\
هر کرا اسرار کار آموختند
&&
مهر کردند و دهانش دوختند
\\
سست خندید و بگفت ای بدنهاد
&&
زانک دانی ایزدت توبه دهاد
\\
\end{longtable}
\end{center}
