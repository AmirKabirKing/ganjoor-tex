\begin{center}
\section*{بخش ۸۱ - آمدن آن امیر نمام با سرهنگان نیم‌شب بگشادن آن حجرهٔ ایاز و پوستین و چارق دیدن آویخته و گمان بردن کی آن مکرست و روپوش و خانه را حفره کردن بهر گوشه‌ای کی گمان آمد چاه کنان  آوردن و دیوارها را سوراخ کردن و  چیزی نایافتن و خجل و نومید  شدن چنانک بدگمانان و خیال‌اندیشان در کار انبیا و اولیا کی  می‌گفتند کی ساحرند و خویشتن ساخته‌اند و تصدر می‌جویند بعد  از تفحص خجل شوند و سود ندارد}
\label{sec:sh081}
\addcontentsline{toc}{section}{\nameref{sec:sh081}}
\begin{longtable}{l p{0.5cm} r}
آن امینان بر در حجره شدند
&&
طالب گنج و زر و خمره بدند
\\
قفل را برمی‌گشادند از هوس
&&
با دو صد فرهنگ و دانش چند کس
\\
زانک قفل صعب و پر پیچیده بود
&&
از میان قفلها بگزیده بود
\\
نه ز بخل سیم و مال و زر خام
&&
از برای کتم آن سر از عوام
\\
که گروهی بر خیال بد تنند
&&
قوم دیگر نام سالوسم کنند
\\
پیش با همت بود اسرار جان
&&
از خسان محفوظ‌تر از لعل کان
\\
زر به از جانست پیش ابلهان
&&
زر نثار جان بود نزد شهان
\\
حرص تازد بیهده سوی سراب
&&
عقل گوید نیک بین که آن نیست آب
\\
حرص غالب بود و زر چون جان شده
&&
نعرهٔ عقل آن زمان پنهان شده
\\
گشته صدتو حرص و غوغاهای او
&&
گشته پنهان حکمت و ایمای او
\\
تا که در چاه غرور اندر فتد
&&
آنگه از حکمت ملامت بشنود
\\
چون ز بند دام باد او شکست
&&
نفس لوامه برو یابید دست
\\
تا به دیوار بلا ناید سرش
&&
نشنود پند دل آن گوش کرش
\\
کودکان را حرص گوزینه و شکر
&&
از نصیحتها کند دو گوش کر
\\
چونک دردت دنبلش آغاز شد
&&
در نصیحت هر دو گوشش باز شد
\\
حجره را با حرص و صدگونه هوس
&&
باز کردند آن زمان آن چند کس
\\
اندر افتادند از در ز ازدحام
&&
هم‌چو اندر دوغ گندیده هوام
\\
عاشقانه در فتد با کر و فر
&&
خورد امکان نی و بسته هر دو پر
\\
بنگریدند از یسار و از یمین
&&
چارقی بدریده بود و پوستین
\\
باز گفتند این مکان بی‌نوش نیست
&&
چارق اینجا جز پی روپوش نیست
\\
هین بیاور سیخهای تیز را
&&
امتحان کن حفره و کاریز را
\\
هر طرف کندند و جستند آن فریق
&&
حفره‌ها کردند و گوهای عمیق
\\
حفره‌هاشان بانگ می‌داد آن زمان
&&
کنده‌های خالییم ای کندگان
\\
زان سگالش شرم هم می‌داشتند
&&
کنده‌ها را باز می‌انباشتند
\\
بی‌عدد لا حول در هر سینه‌ای
&&
مانده مرغ حرصشان بی‌چینه‌ای
\\
زان ضلالتهای یاوه‌تازشان
&&
حفرهٔ دیوار و در غمازشان
\\
ممکن اندای آن دیوار نی
&&
با ایاز امکان هیچ انکار نی
\\
گر خداع بی‌گناهی می‌دهند
&&
حایط و عرصه گواهی می‌دهند
\\
باز می‌گشتند سوی شهریار
&&
پر ز گرد و روی زرد و شرمسار
\\
\end{longtable}
\end{center}
