\begin{center}
\section*{بخش ۶۴ - در ابتدای خلقت جسم آدم علیه‌السلام  کی جبرئیل علیه‌السلام را اشارت کرد کی برو از زمین مشتی خاک برگیر و به روایتی از هر نواحی مشت مشت بر گیر}
\label{sec:sh064}
\addcontentsline{toc}{section}{\nameref{sec:sh064}}
\begin{longtable}{l p{0.5cm} r}
چونک صانع خواست ایجاد بشر
&&
از برای ابتلای خیر و شر
\\
جبرئیل صدق را فرمود رو
&&
مشت خاکی از زمین بستان گرو
\\
او میان بست و بیامد تا زمین
&&
تا گزارد امر رب‌العالمین
\\
دست سوی خاک برد آن مؤتمر
&&
خاک خود را در کشید و شد حذر
\\
پس زبان بگشاد خاک و لابه کرد
&&
کز برای حرمت خلاق فرد
\\
ترک من گو و برو جانم ببخش
&&
رو بتاب از من عنان خنگ رخش
\\
در کشاکشهای تکلیف و خطر
&&
بهر لله هل مرا اندر مبر
\\
بهر آن لطفی که حقت بر گزید
&&
کرد بر تو علم لوح کل پدید
\\
تا ملایک را معلم آمدی
&&
دایما با حق مکلم آمدی
\\
که سفیر انبیا خواهی بدن
&&
تو حیات جان وحیی نی بدن
\\
بر سرافیلت فضیلت بود از آن
&&
کو حیات تن بود تو آن جان
\\
بانگ صورش نشات تن‌ها بود
&&
نفخ تو نشو دل یکتا بود
\\
جان جان تن حیات دل بود
&&
پس ز دادش داد تو فاضل بود
\\
باز میکائیل رزق تن دهد
&&
سعی تو رزق دل روشن دهد
\\
او بداد کیل پر کردست ذیل
&&
داد رزق تو نمی‌گنجد به کیل
\\
هم ز عزرائیل با قهر و عطب
&&
تو بهی چون سبق رحمت بر غضب
\\
حامل عرش این چهارند و تو شاه
&&
بهترین هر چهاری ز انتباه
\\
روز محشر هشت بینی حاملانش
&&
هم تو باشی افضل هشت آن زمانش
\\
هم‌چنین برمی‌شمرد و می‌گریست
&&
بوی می‌برد او کزین مقصود چیست
\\
معدن شرم و حیا بد جبرئیل
&&
بست آن سوگندها بر وی سبیل
\\
بس که لابه کردش و سوگند داد
&&
بازگشت و گفت یا رب العباد
\\
که نبودم من به کارت سرسری
&&
لیک زانچ رفت تو داناتری
\\
گفت نامی که ز هولش ای بصیر
&&
هفت گردون باز ماند از مسیر
\\
شرمم آمد گشتم از نامت خجل
&&
ورنه آسانست نقل مشت گل
\\
که تو زوری داده‌ای املاک را
&&
که بدرانند این افلاک را
\\
\end{longtable}
\end{center}
