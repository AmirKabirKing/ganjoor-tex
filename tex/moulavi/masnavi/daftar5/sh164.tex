\begin{center}
\section*{بخش ۱۶۴ - صفت کردن مرد غماز و نمودن صورت کنیزک مصور در کاغذ و عاشق شدن خلیفهٔ مصر بر آن صورت و فرستادن خلیفه امیری را با سپاه گران بدر موصل و قتل و ویرانی بسیار کردن بهر این غرض}
\label{sec:sh164}
\addcontentsline{toc}{section}{\nameref{sec:sh164}}
\begin{longtable}{l p{0.5cm} r}
مر خلیفهٔ مصر را غماز گفت
&&
که شه موصل به حوری گشت جفت
\\
یک کنیزک دارد او اندر کنار
&&
که به عالم نیست مانندش نگار
\\
در بیان ناید که حسنش بی‌حدست
&&
نقش او اینست که اندر کاغذست
\\
نقش در کاغذ چو دید آن کیقباد
&&
خیره گشت و جام از دستش فتاد
\\
پهلوانی را فرستاد آن زمان
&&
سوی موصل با سپاه بس گران
\\
که اگر ندهد به تو آن ماه را
&&
برکن از بن آن در و درگاه را
\\
ور دهد ترکش کن و مه را بیار
&&
تا کشم من بر زمین مه در کنار
\\
پهلوان شد سوی موصل با حشم
&&
با هزاران رستم و طبل و علم
\\
چون ملخها بی‌عدد بر گرد کشت
&&
قاصد اهلاک اهل شهر گشت
\\
هر نواحی منجنیقی از نبرد
&&
هم‌چو کوه قاف او بر کار کرد
\\
زخم تیر و سنگهای منجنیق
&&
تیغها در گرد چون برق از بریق
\\
هفته‌ای کرد این چنین خون‌ریز گرم
&&
برج سنگین سست شد چون موم نرم
\\
شاه موصل دید پیگار مهول
&&
پس فرستاد از درون پیشش رسول
\\
که چه می‌خواهی ز خون مؤمنان
&&
کشته می‌گردند زین حرب گران
\\
گر مرادت ملک شهر موصلست
&&
بی‌چنین خون‌ریز اینت حاصلست
\\
من روم بیرون شهر اینک در آ
&&
تا نگیرد خون مظلومان ترا
\\
ور مرادت مال و زر و گوهرست
&&
این ز ملک شهر خود آسان‌ترست
\\
\end{longtable}
\end{center}
