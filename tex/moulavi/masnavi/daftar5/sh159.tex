\begin{center}
\section*{بخش ۱۵۹ - وصیت کردن پدر دختر را کی خود را نگهدار تا حامله نشوی از شوهرت}
\label{sec:sh159}
\addcontentsline{toc}{section}{\nameref{sec:sh159}}
\begin{longtable}{l p{0.5cm} r}
خواجه‌ای بودست او را دختری
&&
زهره‌خدی مه‌رخی سیمین‌بری
\\
گشت بالغ داد دختر را به شو
&&
شو نبود اندر کفائت کفو او
\\
خربزه چون در رسد شد آبناک
&&
گر بنشکافی تلف گردد هلاک
\\
چون ضرورت بود دختر را بداد
&&
او بناکفوی ز تخویف فساد
\\
گفت دختر را کزین داماد نو
&&
خویشتن پرهیز کن حامل مشو
\\
کز ضرورت بود عقد این گدا
&&
این غریب‌اشمار را نبود وفا
\\
ناگهان به جهد کند ترک همه
&&
بر تو طفل او بماند مظلمه
\\
گفت دختر کای پدر خدمت کنم
&&
هست پندت دل‌پذیر و مغتنم
\\
هر دو روزی هر سه روزی آن پدر
&&
دختر خود را بفرمودی حذر
\\
حامله شد ناگهان دختر ازو
&&
چون بود هر دو جوان خاتون و شو
\\
از پدر او را خفی می‌داشتش
&&
پنج ماهه گشت کودک یا که شش
\\
گشت پیدا گفت بابا چیست این
&&
من نگفتم که ازو دوری گزین
\\
این وصیتهای من خود باد بود
&&
که نکردت پند و وعظم هیچ سود
\\
گفت بابا چون کنم پرهیز من
&&
آتش و پنبه‌ست بی‌شک مرد و زن
\\
پنبه را پرهیز از آتش کجاست
&&
یا در آتش کی حفاظست و تقاست
\\
گفت من گفتم که سوی او مرو
&&
تو پذیرای منی او مشو
\\
در زمان حال و انزال و خوشی
&&
خویشتن باید که از وی در کشی
\\
گفت کی دانم که انزالش کیست
&&
این نهانست و بغایت دوردست
\\
گفت چشمش چون کلاپیسه شود
&&
فهم کن که آن وقت انزالش بود
\\
گفت تا چشمش کلاپیسه شدن
&&
کور گشتست این دو چشم کور من
\\
نیست هر عقلی حقیری پایدار
&&
وقت حرص و وقت خشم و کارزار
\\
\end{longtable}
\end{center}
