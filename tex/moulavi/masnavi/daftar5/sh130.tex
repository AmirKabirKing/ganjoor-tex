\begin{center}
\section*{بخش ۱۳۰ - جواب گفتن ممن سنی کافر جبری  را و در اثبات اختیار بنده دلیل گفتن سنت راهی باشد کوفتهٔ اقدام انبیا علیهم‌السلام بر یمین آن راه بیابان جبر کی خود را اختیار نبیند و امر و نهی را منکر شود و تاویل کند و از منکر شدن امر و نهی لازم آید انکار بهشت کی جزای مطیعان امرست و دوزخ جزای مخالفان امر و دیگر نگویم بچه انجامد کی العاقل تکفیه الاشاره و بر یسار آن راه بیابان قدرست کی قدرت خالق را مغلوب قدرت خلق داند و از آن آن فسادها زاید کی آن مغ جبری بر می‌شمرد}
\label{sec:sh130}
\addcontentsline{toc}{section}{\nameref{sec:sh130}}
\begin{longtable}{l p{0.5cm} r}
گفت مؤمن بشنو ای جبری خطاب
&&
آن خود گفتی نک آوردم جواب
\\
بازی خود دیدی ای شطرنج‌باز
&&
بازی خصمت ببین پهن و دراز
\\
نامهٔ عذر خودت بر خواندی
&&
نامهٔ سنی بخوان چه ماندی
\\
نکته گفتی جبریانه در قضا
&&
سر آن بشنو ز من در ماجرا
\\
اختیاری هست ما را بی‌گمان
&&
حس را منکر نتانی شد عیان
\\
سنگ را هرگز بگوید کس بیا
&&
از کلوخی کس کجا جوید وفا
\\
آدمی را کس نگوید هین بپر
&&
یا بیا ای کور تو در من نگر
\\
گفت یزدان ما علی الاعمی حرج
&&
کی نهد بر کس حرج رب الفرج
\\
کس نگوید سنگ را دیر آمدی
&&
یا که چوبا تو چرا بر من زدی
\\
این چنین واجستها مجبور را
&&
کس بگوید یا زند معذور را
\\
امر و نهی و خشم و تشریف و عتاب
&&
نیست جز مختار را ای پاک‌جیب
\\
اختیاری هست در ظلم و ستم
&&
من ازین شیطان و نفس این خواستم
\\
اختیار اندر درونت ساکنست
&&
تا ندید او یوسفی کف را نخست
\\
اختیار و داعیه در نفس بود
&&
روش دید آنگه پر و بالی گشود
\\
سگ بخفته اختیارش گشته گم
&&
چون شکنبه دید جنبانید دم
\\
اسپ هم حو حو کند چون دید جو
&&
چون بجنبد گوشت گربه کرد مو
\\
دیدن آمد جنبش آن اختیار
&&
هم‌چو نفخی ز آتش انگیزد شرار
\\
پس بجنبد اختیارت چون بلیس
&&
شد دلاله آردت پیغام ویس
\\
چونک مطلوبی برین کس عرضه کرد
&&
اختیار خفته بگشاید نورد
\\
وآن فرشته خیرها بر رغم دیو
&&
عرضه دارد می‌کند در دل غریو
\\
تا بجنبد اختیار خیر تو
&&
زانک پیش از عرضه خفتست این دو خو
\\
پس فرشته و دیو گشته عرضه‌دار
&&
بهر تحریک عروق اختیار
\\
می‌شود ز الهامها و وسوسه
&&
اختیار خیر و شرت ده کسه
\\
وقت تحلیل نماز ای با نمک
&&
زان سلام آورد باید بر ملک
\\
که ز الهام و دعای خوبتان
&&
اختیار این نمازم شد روان
\\
باز از بعد گنه لعنت کنی
&&
بر بلیس ایرا کزویی منحنی
\\
این دو ضد عرضه کننده‌ت در سرار
&&
در حجاب غیب آمد عرضه‌دار
\\
چونک پردهٔ غیب برخیزد ز پیش
&&
تو ببینی روی دلالان خویش
\\
وآن سخنشان وا شناسی بی‌گزند
&&
که آن سخن‌گویان نهان اینها بدند
\\
دیو گوید ای اسیر طبع و تن
&&
عرضه می‌کردم نکردم زور من
\\
وآن فرشته گویدت من گفتمت
&&
که ازین شادی فزون گردد غمت
\\
آن فلان روزت نگفتم من چنان
&&
که از آن سویست ره سوی جنان
\\
آن فلان روزت نگفتم من چنان
&&
که از آن سویست ره سوی جنان
\\
ما محب جان و روح افزای تو
&&
ساجدان مخلص بابای تو
\\
این زمانت خدمتی هم می‌کنیم
&&
سوی مخدومی صلایت می‌زنیم
\\
آن گره بابات را بوده عدی
&&
در خطاب اسجدوا کرده ابا
\\
آن گرفتی آن ما انداختی
&&
حق خدمتهای ما نشناختی
\\
این زمان ما را و ایشان را عیان
&&
در نگر بشناس از لحن و بیان
\\
نیم شب چون بشنوی رازی ز دوست
&&
چون سخن گوید سحر دانی که اوست
\\
ور دو کس در شب خبر آرد ترا
&&
روز از گفتن شناسی هر دو را
\\
بانگ شیر و بانگ سگ در شب رسید
&&
صورت هر دو ز تاریکی ندید
\\
روز شد چون باز در بانگ آمدند
&&
پس شناسدشان ز بانگ آن هوشمند
\\
مخلص این که دیو و روح عرضه‌دار
&&
هر دو هستند از تتمهٔ اختیار
\\
اختیاری هست در ما ناپدید
&&
چون دو مطلب دید آید در مزید
\\
اوستادان کودکان را می‌زنند
&&
آن ادب سنگ سیه را کی کنند
\\
هیچ گویی سنگ را فردا بیا
&&
ور نیایی من دهم بد را سزا
\\
هیچ عاقل مر کلوخی را زند
&&
هیچ با سنگی عتابی کس کند
\\
در خرد جبر از قدر رسواترست
&&
زانک جبری حس خود را منکرست
\\
منکر حس نیست آن مرد قدر
&&
فعل حق حسی نباشد ای پسر
\\
منکر فعل خداوند جلیل
&&
هست در انکار مدلول دلیل
\\
آن بگوید دود هست و نار نی
&&
نور شمعی بی ز شمعی روشنی
\\
وین همی‌بیند معین نار را
&&
نیست می‌گوید پی انکار را
\\
جامه‌اش سوزد بگوید نار نیست
&&
جامه‌اش دوزد بگوید تار نیست
\\
پس تسفسط آمد این دعوی جبر
&&
لاجرم بدتر بود زین رو ز گبر
\\
گبر گوید هست عالم نیست رب
&&
یا ربی گوید که نبود مستحب
\\
این همی گوید جهان خود نیست هیچ
&&
هسته سوفسطایی اندر پیچ پیچ
\\
جملهٔ عالم مقر در اختیار
&&
امر و نهی این میار و آن بیار
\\
او همی گوید که امر و نهی لاست
&&
اختیاری نیست این جمله خطاست
\\
حس را حیوان مقرست ای رفیق
&&
لیک ادراک دلیل آمد دقیق
\\
زانک محسوسست ما را اختیار
&&
خوب می‌آید برو تکلیف کار
\\
\end{longtable}
\end{center}
