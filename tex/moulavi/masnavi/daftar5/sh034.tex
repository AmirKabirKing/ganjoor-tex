\begin{center}
\section*{بخش ۳۴ - در صفت آن بی‌خودان کی از شر خود و هنر خود آمن شده‌اند کی فانی‌اند  در بقای حق هم‌چون ستارگان کی  فانی‌اند روز در آفتاب و  فانی را خوف آفت و خطر نباشد}
\label{sec:sh034}
\addcontentsline{toc}{section}{\nameref{sec:sh034}}
\begin{longtable}{l p{0.5cm} r}
چون فناش از فقر پیرایه شود
&&
او محمدوار بی‌سایه شود
\\
فقر فخری را فنا پیرایه شد
&&
چون زبانهٔ شمع او بی‌سایه شد
\\
شمع جمله شد زبانه پا و سر
&&
سایه را نبود بگرد او گذر
\\
موم از خویش و ز سایه در گریخت
&&
در شعاع از بهر او کی شمع ریخت
\\
گفت او بهر فنایت ریختم
&&
گفت من هم در فنا بگریختم
\\
این شعاع باقی آمد مفترض
&&
نه شعاع شمع فانی عرض
\\
شمع چون در نار شد کلی فنا
&&
نه اثر بینی ز شمع و نه ضیا
\\
هست اندر دفع ظلمت آشکار
&&
آتش صورت به مومی پایدار
\\
برخلاف موم شمع جسم کان
&&
تا شود کم گردد افزون نور جان
\\
این شعاع باقی و آن فانیست
&&
شمع جان را شعلهٔ ربانیست
\\
این زبانهٔ آتشی چون نور بود
&&
سایهٔ فانی شدن زو دور بود
\\
ابر را سایه بیفتد در زمین
&&
ماه را سایه نباشد همنشین
\\
بی‌خودی بی‌ابریست ای نیک‌خواه
&&
باشی اندر بی‌خودی چون قرص ماه
\\
باز چون ابری بیاید رانده
&&
رفت نور از مه خیالی مانده
\\
از حجاب ابر نورش شد ضعیف
&&
کم ز ماه نو شد آن بدر شریف
\\
مه خیالی می‌نماید ز ابر و گرد
&&
ابر تن ما را خیال‌اندیش کرد
\\
لطف مه بنگر که این هم لطف اوست
&&
که بگفت او ابرها ما را عدوست
\\
مه فراغت دارد از ابر و غبار
&&
بر فراز چرخ دارد مه مدار
\\
ابر ما را شد عدو و خصم جان
&&
که کند مه را ز چشم ما نهان
\\
حور را این پرده زالی می‌کند
&&
بدر را کم از هلالی می‌کند
\\
ماه ما را در کنار عز نشاند
&&
دشمن ما را عدوی خویش خواند
\\
تاب ابر و آب او خود زین مهست
&&
هر که مه خواند ابر را بس گمرهست
\\
نور مه بر ابر چون منزل شدست
&&
روی تاریکش ز مه مبدل شدست
\\
گرچه همرنگ مهست و دولتیست
&&
اندر ابر آن نور مه عاریتیست
\\
در قیامت شمس و مه معزول شد
&&
چشم در اصل ضیا مشغول شد
\\
تا بداند ملک را از مستعار
&&
وین رباط فانی از دارالقرار
\\
دایه عاریه بود روزی سه چار
&&
مادرا ما را تو گیر اندر کنار
\\
پر من ابرست و پرده‌ست و کثیف
&&
ز انعکاس لطف حق شد او لطیف
\\
بر کنم پر را و حسنش را ز راه
&&
تا ببینم حسن مه را هم ز ماه
\\
من نخواهم دایه مادر خوشترست
&&
موسی‌ام من دایهٔ من مادرست
\\
من نخواهم لطف مه از واسطه
&&
که هلاک قوم شد این رابطه
\\
یا مگر ابری شود فانی راه
&&
تا نگردد او حجاب روی ماه
\\
صورتش بنماید او در وصف لا
&&
هم‌چو جسم انبیا و اولیا
\\
آنچنان ابری نباشد پرده‌بند
&&
پرده‌در باشد به معنی سودمند
\\
آن‌چنان که اندر صباح روشنی
&&
قطره می‌بارید و بالا ابر نی
\\
معجزهٔ پیغامبری بود آن سقا
&&
گشته ابر از محو هم‌رنگ سما
\\
بود ابر و رفته از وی خوی ابر
&&
این چنین گردد تن عاشق به صبر
\\
تن بود اما تنی گم گشته زو
&&
گشته مبدل رفته از وی رنگ و بو
\\
پر پی غیرست و سر از بهر من
&&
خانهٔ سمع و بصر استون تن
\\
جان فدا کردن برای صید غیر
&&
کفر مطلق دان و نومیدی ز خیر
\\
هین مشو چون قند پیش طوطیان
&&
بلک زهری شو شو آمن از زیان
\\
یا برای شادباشی در خطاب
&&
خویش چون مردار کن پی کلاب
\\
پس خضر کشتی برای این شکست
&&
تا که آن کشتی ز غاصب باز رست
\\
فقر فخری بهر آن آمد سنی
&&
تا ز طماعان گریزم در غنی
\\
گنجها را در خرابی زان نهند
&&
تا ز حرص اهل عمران وا رهند
\\
پر نتانی کند رو خلوت گزین
&&
تا نگردی جمله خرج آن و این
\\
زآنک تو هم لقمه‌ای هم لقمه‌خوار
&&
آکل و ماکولی ای جان هوش‌دار
\\
\end{longtable}
\end{center}
