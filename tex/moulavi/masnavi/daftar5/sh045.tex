\begin{center}
\section*{بخش ۴۵ - تفسیر اسفل سافلین الا الذین آمنوا و عملوا الصالحات فلهم اجر غیر ممنون}
\label{sec:sh045}
\addcontentsline{toc}{section}{\nameref{sec:sh045}}
\begin{longtable}{l p{0.5cm} r}
لیک گر باشد طبیبش نور حق
&&
نیست از پیری و تب نقصان و دق
\\
سستی او هست چون سستی مست
&&
که اندر آن سستیش رشک رستمست
\\
گر بمیرد استخوانش غرق ذوق
&&
ذره ذره‌ش در شعاع نور شوق
\\
وآنک آنش نیست باغ بی‌ثمر
&&
که خزانش می‌کند زیر و زبر
\\
گل نماند خارها ماند سیاه
&&
زرد و بی‌مغز آمده چون تل کاه
\\
تا چه زلت کرد آن باغ ای خدا
&&
که ازو این حله‌ها گردد جدا
\\
خویشتن را دید و دید خویشتن
&&
زهر قتالست هین ای ممتحن
\\
شاهدی کز عشق او عالم گریست
&&
عالمش می‌راند از خود جرم چیست
\\
جرم آنک زیور عاریه بست
&&
کرد دعوی کین حلل ملک منست
\\
واستانیم آن که تا داند یقین
&&
خرمن آن ماست خوبان دانه‌چین
\\
تا بداند کان حلل عاریه بود
&&
پرتوی بود آن ز خورشید وجود
\\
آن جمال و قدرت و فضل و هنر
&&
ز آفتاب حسن کرد این سو سفر
\\
باز می‌گردند چون استارها
&&
نور آن خورشید ازین دیوارها
\\
پرتو خورشید شد وا جایگاه
&&
ماند هر دیوار تاریک و سیاه
\\
آنک کرد او در رخ خوبانت دنگ
&&
نور خورشیدست از شیشهٔ سه رنگ
\\
شیشه‌های رنگ رنگ آن نور را
&&
می‌نمایند این چنین رنگین بما
\\
چون نماند شیشه‌های رنگ‌رنگ
&&
نور بی‌رنگت کند آنگاه دنگ
\\
خوی کن بی‌شیشه دیدن نور را
&&
تا چو شیشه بشکند نبود عمی
\\
قانعی با دانش آموخته
&&
در چراغ غیر چشم افروخته
\\
او چراغ خویش برباید که تا
&&
تو بدانی مستعیری نی‌فتا
\\
گر تو کردی شکر و سعی مجتهد
&&
غم مخور که صد چنان بازت دهد
\\
ور نکردی شکر اکنون خون گری
&&
که شدست آن حسن از کافر بری
\\
امة الکفران اضل اعمالهم
&&
امة الایمان اصلح بالهم
\\
گم شد از بی‌شکر خوبی و هنر
&&
که دگر هرگز نبیند زان اثر
\\
خویشی و بی‌خویشی و سکر وداد
&&
رفت زان سان که نیاردشان به یاد
\\
که اضل اعمالهم ای کافران
&&
جستن کامست از هر کام‌ران
\\
جز ز اهل شکر و اصحاب وفا
&&
که مریشان راست دولت در قفا
\\
دولت رفته کجا قوت دهد
&&
دولت آینده خاصیت دهد
\\
قرض ده زین دولت اندر اقرضوا
&&
تا که صد دولت ببینی پیش رو
\\
اندکی زین شرب کم کن بهر خویش
&&
تا که حوض کوثری یابی به پیش
\\
جرعه بر خاک وفا آنکس که ریخت
&&
کی تواند صید دولت زو گریخت
\\
خوش کند دلشان که اصلح بالهم
&&
رد من بعد التوی انزالهم
\\
ای اجل وی ترک غارت‌ساز ده
&&
هر چه بردی زین شکوران باز ده
\\
وا دهد ایشان بنپذیرند آن
&&
زانک منعم گشته‌اند از رخت جان
\\
صوفییم و خرقه‌ها انداختیم
&&
باز نستانیم چون در باختیم
\\
ما عوض دیدیم آنگه چون عوض
&&
رفت از ما حاجت و حرص و غرض
\\
ز آب شور و مهلکی بیرون شدیم
&&
بر رحیق و چشمهٔ کوثر زدیم
\\
آنچ کردی ای جهان با دیگران
&&
بی‌وفایی و فن و ناز گران
\\
بر سرت ریزیم ما بهر جزا
&&
که شهیدیم آمده اندر غزا
\\
تا بدانی که خدای پاک را
&&
بندگان هستند پر حمله و مری
\\
سبلت تزویر دنیا بر کنند
&&
خیمه را بر باروی نصرت زنند
\\
این شهیدان باز نو غازی شدند
&&
وین اسیران باز بر نصرت زدند
\\
سر برآوردند باز از نیستی
&&
که ببین ما را گر اکمه نیستی
\\
تا بدانی در عدم خورشیدهاست
&&
وآنچ اینجا آفتاب آنجا سهاست
\\
در عدم هستی برادر چون بود
&&
ضد اندر ضد چون مکنون بود
\\
یخرج الحی من المیت بدان
&&
که عدم آمد امید عابدان
\\
مرد کارنده که انبارش تهیست
&&
شاد و خوش نه بر امید نیستیست
\\
که بروید آن ز سوی نیستی
&&
فهم کن گر واقف معنیستی
\\
دم به دم از نیستی تو منتظر
&&
که بیابی فهم و ذوق آرام و بر
\\
نیست دستوری گشاد این راز را
&&
ورنه بغدادی کنم ابخاز را
\\
پس خزانهٔ صنع حق باشد عدم
&&
که بر آرد زو عطاها دم به دم
\\
مبدع آمد حق و مبدع آن بود
&&
که برآرد فرع بی‌اصل و سند
\\
\end{longtable}
\end{center}
