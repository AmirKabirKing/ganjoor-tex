\begin{center}
\section*{بخش ۱۳۶ - حکایت آن درویش کی در هری غلامان آراستهٔ عمید خراسان را دید و بر اسبان تازی و قباهای زربفت و کلاهای مغرق و غیر آن پرسید کی اینها کدام امیرانند و چه شاهانند گفت او را کی اینها امیران نیستند اینها غلامان عمید خراسانند روی به آسمان  کرد کی ای خدا غلام پروردن از عمید  بیاموز آنجا مستوفی را عمید گویند}
\label{sec:sh136}
\addcontentsline{toc}{section}{\nameref{sec:sh136}}
\begin{longtable}{l p{0.5cm} r}
آن یکی گستاخ رو اندر هری
&&
چون بدیدی او غلام مهتری
\\
جامهٔ اطلس کمر زرین روان
&&
روی کردی سوی قبلهٔ آسمان
\\
کای خدا زین خواجهٔ صاحب منن
&&
چون نیاموزی تو بنده داشتن
\\
بنده پروردن بیاموز ای خدا
&&
زین رئیس و اختیار شاه ما
\\
بود محتاج و برهنه و بی‌نوا
&&
در زمستان لرز لرزان از هوا
\\
انبساطی کرد آن از خود بری
&&
جراتی بنمود او از لمتری
\\
اعتمادش بر هزاران موهبت
&&
که ندیم حق شد اهل معرفت
\\
گر ندیم شاه گستاخی کند
&&
تو مکن آنک نداری آن سند
\\
حق میان داد و میان به از کمر
&&
گر کسی تاجی دهد او داد سر
\\
تا یکی روزی که شاه آن خواجه را
&&
متهم کرد و ببستش دست و پا
\\
آن غلامان را شکنجه می‌نمود
&&
که دفینهٔ خواجه بنمایید زود
\\
سر او با من بگویید ای خسان
&&
ورنه برم از شما حلق و لسان
\\
مدت یک ماهشان تعذیب کرد
&&
روز و شب اشکنجه و افشار و درد
\\
پاره پاره کردشان و یک غلام
&&
راز خواجه وا نگفت از اهتمام
\\
گفتش اندر خواب هاتف کای کیا
&&
بنده بودن هم بیاموز و بیا
\\
ای دریده پوستین یوسفان
&&
گر بدرد گرگت آن از خویش دان
\\
زانک می‌بافی همه‌ساله بپوش
&&
زانک می‌کاری همه ساله بنوش
\\
فعل تست این غصه‌های دم به دم
&&
این بود معنی قد جف القلم
\\
که نگردد سنت ما از رشد
&&
نیک را نیکی بود بد راست بد
\\
کار کن هین که سلیمان زنده است
&&
تا تو دیوی تیغ او برنده است
\\
چون فرشته گشته از تیغ آمنیست
&&
از سلیمان هیچ او را خوف نیست
\\
حکم او بر دیو باشد نه ملک
&&
رنج در خاکست نه فوق فلک
\\
ترک کن این جبر را که بس تهیست
&&
تا بدانی سر سر جبر چیست
\\
ترک کن این جبر جمع منبلان
&&
تا خبر یابی از آن جبر چو جان
\\
ترک معشوقی کن و کن عاشقی
&&
ای گمان برده که خوب و فایقی
\\
ای که در معنی ز شب خامش‌تری
&&
گفت خود را چند جویی مشتری
\\
سر بجنبانند پیشت بهر تو
&&
رفت در سودای ایشان دهر تو
\\
تو مرا گویی حسد اندر مپیچ
&&
چه حسد آرد کسی از فوت هیچ
\\
هست تعلیم خسان ای چشم‌شوخ
&&
هم‌چو نقش خرد کردن بر کلوخ
\\
خویش را تعلیم کن عشق و نظر
&&
که آن بود چون نقش فی جرم الحجر
\\
نفس تو با تست شاگرد وفا
&&
غیر فانی شد کجا جویی کجا
\\
تا کنی مر غیر را حبر و سنی
&&
خویش را بدخو و خالی می‌کنی
\\
متصل چون شد دلت با آن عدن
&&
هین بگو مهراس از خالی شدن
\\
امر قل زین آمدش کای راستین
&&
کم نخواهد شد بگو دریاست این
\\
انصتوا یعنی که آبت را بلاغ
&&
هین تلف کم کن که لب‌خشکست باغ
\\
این سخن پایان ندارد ای پدر
&&
این سخن را ترک کن پایان نگر
\\
غیرتم آید که پیشت بیستند
&&
بر تو می‌خندند عاشق نیستند
\\
عاشقانت در پس پردهٔ کرم
&&
بهر تو نعره‌زنان بین دم بدم
\\
عاشق آن عاشقان غیب باش
&&
عاشقان پنج روزه کم تراش
\\
که بخوردندت ز خدعه و جذبه‌ای
&&
سالها زیشان ندیدی حبه‌ای
\\
چند هنگامه نهی بر راه عام
&&
گام خستی بر نیامد هیچ کام
\\
وقت صحت جمله یارند و حریف
&&
وقت درد و غم به جز حق کو الیف
\\
وقت درد چشم و دندان هیچ کس
&&
دست تو گیرد به جز فریاد رس
\\
پس همان درد و مرض را یاد دار
&&
چون ایاز از پوستین کن اعتبار
\\
پوستین آن حالت درد توست
&&
که گرفتست آن ایاز آن را به دست
\\
\end{longtable}
\end{center}
