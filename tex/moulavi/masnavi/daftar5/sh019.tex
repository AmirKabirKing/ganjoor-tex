\begin{center}
\section*{بخش ۱۹ - سبب آنک فرجی را نام فرجی نهادند از اول}
\label{sec:sh019}
\addcontentsline{toc}{section}{\nameref{sec:sh019}}
\begin{longtable}{l p{0.5cm} r}
صوفیی بدرید جبه در حرج
&&
پیشش آمد بعد به دریدن فرج
\\
کرد نام آن دریده فرجی
&&
این لقب شد فاش زان مرد نجی
\\
این لقب شد فاش و صافش شیخ برد
&&
ماند اندر طبع خلقان حرف درد
\\
هم‌چنین هر نام صافی داشتست
&&
اسم را چون دردیی بگذاشتست
\\
هر که گل خوارست دردی را گرفت
&&
رفت صوفی سوی صافی ناشکفت
\\
گفت لابد درد را صافی بود
&&
زین دلالت دل به صفوت می‌رود
\\
درد عسر افتاد و صافش یسر او
&&
صاف چون خرما و دردی بسر او
\\
یسر با عسرست هین آیس مباش
&&
راه داری زین ممات اندر معاش
\\
روح خواهی جبه بشکاف ای پسر
&&
تا از آن صفوت برآری زود سر
\\
هست صوفی آنک شد صفوت‌طلب
&&
نه از لباس صوف و خیاطی و دب
\\
صوفیی گشته به پیش این لئام
&&
الخیاطه واللواطه والسلام
\\
بر خیال آن صفا و نام نیک
&&
رنگ پوشیدن نکو باشد ولیک
\\
بر خیالش گر روی تا اصل او
&&
نی چو عباد خیال تو به تو
\\
دور باش غیرتت آمد خیال
&&
گرد بر گرد سراپردهٔ جمال
\\
بسته هر جوینده را که راه نیست
&&
هر خیالش پیش می‌آید بیست
\\
جز مگر آن تیزکوش تیزهوش
&&
کش بود از جیش نصرتهاش جوش
\\
نجهد از تخییلها نی شه شود
&&
تیر شه بنماید آنگه ره شود
\\
این دل سرگشته را تدبیر بخش
&&
وین کمانهای دوتو را تیر بخش
\\
جرعه‌ای بر ریختی زان خفیه جام
&&
بر زمین خاک من کاس الکرام
\\
هست بر زلف و رخ از جرعه‌ش نشان
&&
خاک را شاهان همی‌لیسند از آن
\\
جرعه حسنست اندر خاک گش
&&
که به صد دل روز و شب می‌بوسیش
\\
جرعه خاک آمیز چون مجنون کند
&&
مر ترا تا صاف او خود چون کند
\\
هر کسی پیش کلوخی جامه‌چاک
&&
که آن کلوخ از حسن آمد جرعه‌ناک
\\
جرعه‌ای بر ماه و خورشید و حمل
&&
جرعه‌ای بر عرش و کرسی و زحل
\\
جرعه گوییش ای عجب یا کیمیا
&&
که ز اسیبش بود چندین بها
\\
جد طلب آسیب او ای ذوفنون
&&
لا یمس ذاک الا المطهرون
\\
جرعه‌ای بر زر و بر لعل و درر
&&
جرعه‌ای بر خمر و بر نقل و ثمر
\\
جرعه‌ای بر روی خوبان لطاف
&&
تا چگونه باشد آن راواق صاف
\\
چون همی مالی زبان را اندرین
&&
چون شوی چون بینی آن را بی ز طین
\\
چونک وقت مرگ آن جرعهٔ صفا
&&
زین کلوخ تن به مردن شد جدا
\\
آنچ می‌ماند کنی دفنش تو زود
&&
این چنین زشتی بدان چون گشته بود
\\
جان چو بی این جیفه بنماید جمال
&&
من نتانم گفت لطف آن وصال
\\
مه چو بی‌این ابر بنماید ضیا
&&
شرح نتوان کرد زان کار و کیا
\\
حبذا آن مطبخ پر نوش و قند
&&
کین سلاطین کاسه‌لیسان ویند
\\
حبذا آن خرمن صحرای دین
&&
که بود هر خرمن آن را دانه‌چین
\\
حبذا دریای عمر بی‌غمی
&&
که بود زو هفت دریا شب‌نمی
\\
جرعه‌ای چون ریخت ساقی الست
&&
بر سر این شوره خاک زیردست
\\
جوش کرد آن خاک و ما زان جوششیم
&&
جرعهٔ دیگر که بس بی‌کوششیم
\\
گر روا بد ناله کردم از عدم
&&
ور نبود این گفتنی نک تن زدم
\\
این بیان بط حرص منثنیست
&&
از خلیل آموز که آن بط کشتنیست
\\
هست در بط غیر این بس خیر و شر
&&
ترسم از فوت سخنهای دگر
\\
\end{longtable}
\end{center}
