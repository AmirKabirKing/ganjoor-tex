\begin{center}
\section*{بخش ۶۰ - تمثیل تلقین شیخ مریدان را و پیغامبر امت را کی ایشان طاقت تلقین حق ندارند و با حق‌الف ندارند چنانک طوطی با صورت آدمی الف ندارد کی ازو تلقین تواند گرفت حق تعالی شیخ را چون آیینه‌ای پیش مرید هم‌چو طوطی دارد و از پس آینه تلقین می‌کند لا تحرک به لسانک ان هو الا وحی یوحی اینست ابتدای مسلهٔ بی‌منتهی چنانک منقار جنبانیدن طوطی اندرون آینه کی خیالش می‌خوانی بی‌اختیار و تصرف اوست عکس خواندن طوطی برونی کی متعلمست نه عکس آن معلم کی پس آینه است و لیکن خواندن طوطی برونی تصرف آن معلم است پس این مثال آمد نه مثل}
\label{sec:sh060}
\addcontentsline{toc}{section}{\nameref{sec:sh060}}
\begin{longtable}{l p{0.5cm} r}
طوطیی در آینه می‌بیند او
&&
عکس خود را پیش او آورده رو
\\
در پس آیینه آن استا نهان
&&
حرف می‌گوید ادیب خوش‌زبان
\\
طوطیک پنداشته کین گفت پست
&&
گفتن طوطیست که اندر آینه‌ست
\\
پس ز جنس خویش آموز سخن
&&
بی‌خبر از مکر آن گرگ کهن
\\
از پس آیینه می‌آموزدش
&&
ورنه ناموزد جز از جنس خودش
\\
گفت را آموخت زان مرد هنر
&&
لیک از معنی و سرش بی‌خبر
\\
از بشر بگرفت منطق یک به یک
&&
از بشر جز این چه داند طوطیک
\\
هم‌چنان در آینهٔ جسم ولی
&&
خویش را بیند مردی ممتلی
\\
از پس آیینه عقل کل را
&&
کی ببیند وقت گفت و ماجرا
\\
او گمان دارد که می‌گوید بشر
&&
وان گر سرست و او زان بی‌خبر
\\
حرف آموزد ولی سر قدیم
&&
او نداند طوطی است او نی ندیم
\\
هم صفیر مرغ آموزند خلق
&&
کین سخن کار دهان افتاد و حلق
\\
لیک از معنی مرغان بی‌خبر
&&
جز سلیمان قرانی خوش‌نظر
\\
حرف درویشان بسی آموختند
&&
منبر و محفل بدان افروختند
\\
یا به جز آن حرفشان روزی نبود
&&
یا در آخر رحمت آمد ره نمود
\\
\end{longtable}
\end{center}
