\begin{center}
\section*{بخش ۱۰۸ - بردن روبه خر را پیش شیر و جستن خر از شیر و عتاب کردن روباه با شیر کی هنوز خر دور بود تعجیل کردی و عذر گفتن  شیر و لابه کردن روبه را شیر کی برو بار دگرش به فریب}
\label{sec:sh108}
\addcontentsline{toc}{section}{\nameref{sec:sh108}}
\begin{longtable}{l p{0.5cm} r}
چونک بر کوهش بسوی مرج برد
&&
تا کند شیرش به حمله خرد و مرد
\\
دور بود از شیر و آن شیر از نبرد
&&
تا به نزدیک آمدن صبری نکرد
\\
گنبدی کرد از بلندی شیر هول
&&
خود نبودش قوت و امکان حول
\\
خر ز دورش دید و برگشت و گریز
&&
تا به زیر کوه تازان نعل ریز
\\
گفت روبه شیر را ای شاه ما
&&
چون نکردی صبر در وقت وغا
\\
تا به نزدیک تو آید آن غوی
&&
تا باندک حمله‌ای غالب شوی
\\
مکر شیطانست تعجیل و شتاب
&&
لطف رحمانست صبر و احتساب
\\
دور بود و حمله را دید و گریخت
&&
ضعف تو ظاهر شد و آب تو ریخت
\\
گفت من پنداشتم بر جاست زور
&&
تا بدین حد می‌ندانستم فتور
\\
نیز جوع و حاجتم از حد گذشت
&&
صبر و عقلم از تجوع یاوه گشت
\\
گر توانی بار دیگر از خرد
&&
باز آوردن مر او را مسترد
\\
منت بسیار دارم از تو من
&&
جهد کن باشد بیاری‌اش به فن
\\
گفت آری گر خدا یاری دهد
&&
بر دل او از عمی مهری نهد
\\
پس فراموشش شود هولی که دید
&&
از خری او نباشد این بعید
\\
لیک چون آرم من او را بر متاز
&&
تا ببادش ندهی از تعجیل باز
\\
گفت آری تجربه کردم که من
&&
سخت رنجورم مخلخل گشته تن
\\
تا به نزدیکم نیاید خر تمام
&&
من نجنبم خفته باشم در قوام
\\
رفت روبه گفت ای شه همتی
&&
تا بپوشد عقل او را غفلتی
\\
توبه‌ها کردست خر با کردگار
&&
که نگردد غرهٔ هر نابکار
\\
توبه‌هااش را به فن بر هم زنیم
&&
ما عدوی عقل و عهد روشنیم
\\
کلهٔ خر گوی فرزندان ماست
&&
فکرتش بازیچهٔ دستان ماست
\\
عقل که آن باشد ز دوران زحل
&&
پیش عقل کل ندارد آن محل
\\
از عطارد وز زحل دانا شد او
&&
ما ز داد کردگار لطف‌خو
\\
علم الانسان خم طغرای ماست
&&
علم عند الله مقصدهای ماست
\\
تربیهٔ آن آفتاب روشنیم
&&
ربی الاعلی از آن رو می‌زنیم
\\
تجربه گر دارد او با این همه
&&
بشکند صد تجربه زین دمدمه
\\
بوک توبه بشکند آن سست‌خو
&&
در رسد شومی اشکستن درو
\\
\end{longtable}
\end{center}
