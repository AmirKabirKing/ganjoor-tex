\begin{center}
\section*{بخش ۴۹ - در تفسیر قول مصطفی علیه‌السلام من جعل الهموم هما واحدا کفاه الله سائر همومه و من تفرقت به الهموم لا یبالی الله فی ای واد اهلکه}
\label{sec:sh049}
\addcontentsline{toc}{section}{\nameref{sec:sh049}}
\begin{longtable}{l p{0.5cm} r}
هوش را توزیع کردی بر جهات
&&
می‌نیرزد تره‌ای آن ترهات
\\
آب هش را می‌کشد هر بیخ خار
&&
آب هوشت چون رسد سوی ثمار
\\
هین بزن آن شاخ بد را خو کنش
&&
آب ده این شاخ خوش را نو کنش
\\
هر دو سبزند این زمان آخر نگر
&&
کین شود باطل از آن روید ثمر
\\
آب باغ این را حلال آن را حرام
&&
فرق را آخر ببینی والسلام
\\
عدل چه بود آب ده اشجار را
&&
ظلم چه بود آب دادن خار را
\\
عدل وضع نعمتی در موضعش
&&
نه بهر بیخی که باشد آبکش
\\
ظلم چه بود وضع در ناموضعی
&&
که نباشد جز بلا را منبعی
\\
نعمت حق را به جان و عقل ده
&&
نه به طبع پر زحیر پر گره
\\
بار کن بیگار غم را بر تنت
&&
بر دل و جان کم نه آن جان کندنت
\\
بر سر عیسی نهاده تنگ بار
&&
خر سکیزه می‌زند در مرغزار
\\
سرمه را در گوش کردن شرط نیست
&&
کار دل را جستن از تن شرط نیست
\\
گر دلی رو ناز کن خواری مکش
&&
ور تنی شکر منوش و زهر چش
\\
زهر تن را نافعست و قند بد
&&
تن همان بهتر که باشد بی‌مدد
\\
هیزم دوزخ تنست و کم کنش
&&
ور بروید هیزمی رو بر کنش
\\
ورنه حمال حطب باشی حطب
&&
در دو عالم هم‌چو جفت بولهب
\\
از حطب بشناس شاخ سدره را
&&
گرچه هر دو سبز باشند ای فتی
\\
اصل آن شاخست هفتم آسمان
&&
اصل این شاخست از نار و دخان
\\
هست مانندا به صورت پیش حس
&&
که غلط‌بینست چشم و کیش حس
\\
هست آن پیدا به پیش چشم دل
&&
جهد کن سوی دل آ جهد المقل
\\
ور نداری پا بجنبان خویش را
&&
تا ببینی هر کم و هر بیش را
\\
\end{longtable}
\end{center}
