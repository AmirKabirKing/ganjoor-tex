\begin{center}
\section*{بخش ۱۷۴ - رسیدن گوهر از دست به دست آخر دور به ایاز و کیاست ایاز و مقلد ناشدن او ایشان را و مغرور ناشدن او به گال و مال دادن شاه و خلعتها و جامگیها افزون کردن و مدح عقل مخطان کردن به مکر و امتحان که کی روا باشد مقلد را مسلمان داشتن مسلمان باشد اما نادر باشد کی مقلد ازین امتحانها به سلامت بیرون آید کی ثبات بینایان ندارد الا من عصم الله زیرا حق یکیست و آن را ضد بسیار غلط‌افکن و مشابه حق مقلد چون آن ضد را نشناسد از آن رو حق را نشناخته باشد اما حق با آن ناشناخت او چو او را به عنایت نگاه دارد آن ناشناخت او را زیان ندارد}
\label{sec:sh174}
\addcontentsline{toc}{section}{\nameref{sec:sh174}}
\begin{longtable}{l p{0.5cm} r}
ای ایاز اکنون نگویی کین گهر
&&
چند می‌ارزد بدین تاب و هنر
\\
گفت افزون زانچ تانم گفت من
&&
گفت اکنون زود خردش در شکن
\\
سنگها در آستین بودش شتاب
&&
خرد کردش پیش او بود آن صواب
\\
ز اتفاق طالع با دولتش
&&
دست داد آن لحظه نادر حکمتش
\\
یا به خواب این دیده بود آن پر صفا
&&
کرده بود اندر بغل دو سنگ را
\\
هم‌چو یوسف که درون قعر چاه
&&
کشف شد پایان کارش از اله
\\
هر که را فتح و ظفر پیغام داد
&&
پیش او یک شد مراد و بی‌مراد
\\
هر که پایندان وی شد وصل یار
&&
او چه ترسد از شکست و کارزار
\\
چون یقین گشتش که خواهد کرد مات
&&
فوت اسپ و پیل هستش ترهات
\\
گر برد اسپش هر آنک اسپ‌جوست
&&
اسپ رو گو نه که پیش آهنگ اوست
\\
مرد را با اسپ کی خویشی بود
&&
عشق اسپش از پی پیشی بود
\\
بهر صورتها مکش چندین زحیر
&&
بی‌صداع صورتی معنی بگیر
\\
هست زاهد را غم پایان کار
&&
تا چه باشد حال او روز شمار
\\
عارفان ز آغاز گشته هوشمند
&&
از غم و احوال آخر فارغ‌اند
\\
بود عارف را همین خوف و رجا
&&
سابقه‌دانیش خورد آن هر دو را
\\
دید کو سابق زراعت کرد ماش
&&
او همی‌داند چه خواهد بود چاش
\\
عارفست و باز رست از خوف و بیم
&&
های هو را کرد تیغ حق دو نیم
\\
بود او را بیم و اومید از خدا
&&
خوف فانی شد عیان گشت آن رجا
\\
چون شکست او گوهر خاص آن زمان
&&
زان امیران خاست صد بانگ و فغان
\\
کین چه بی‌باکیست والله کافرست
&&
هر که این پر نور گوهر را شکست
\\
وآن جماعت جمله از جهل و عما
&&
در شکسته در امر شاه را
\\
قیمتی گوهر نتیجهٔ مهر و ود
&&
بر چنان خاطر چرا پوشیده شد
\\
\end{longtable}
\end{center}
