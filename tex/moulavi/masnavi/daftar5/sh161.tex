\begin{center}
\section*{بخش ۱۶۱ - نصیحت مبارزان او را کی با این دل و زهره کی تو داری کی از کلابیسه شدن چشم کافر اسیری دست بسته بیهوش شوی و دشنه از دست بیفتد زنهار زنهار ملازم مطبخ خانقاه باش و سوی پیکار مرو تا رسوا نشوی}
\label{sec:sh161}
\addcontentsline{toc}{section}{\nameref{sec:sh161}}
\begin{longtable}{l p{0.5cm} r}
قوم گفتندش به پیکار و نبرد
&&
با چنین زهره که تو داری مگرد
\\
چون ز چشم آن اسیر بسته‌دست
&&
غرقه گشتی کشتی تو در شکست
\\
پس میان حملهٔ شیران نر
&&
که بود با تیغشان چون گوی سر
\\
کی توانی کرد در خون آشنا
&&
چون نه‌ای با جنگ مردان آشنا
\\
که ز طاقاطاق گردنها زدن
&&
طاق‌طاق جامه کوبان ممتهن
\\
بس تن بی‌سر که دارد اضطراب
&&
بس سر بی‌تن به خون بر چون حباب
\\
زیر دست و پای اسپان در غزا
&&
صد فنا کن غرقه گشته در فنا
\\
این چنین هوشی که از موشی پرید
&&
اندر آن صف تیغ چون خواهد کشید
\\
چالش است آن حمزه خوردن نیست این
&&
تا تو برمالی بخوردن آستین
\\
نیست حمزه خوردن اینجا تیغ بین
&&
حمزه‌ای باید درین صف آهنین
\\
کار هر نازک‌دلی نبود قتال
&&
که گریزد از خیالی چون خیال
\\
کار ترکانست نه ترکان برو
&&
جای ترکان هست خانه خانه شو
\\
\end{longtable}
\end{center}
