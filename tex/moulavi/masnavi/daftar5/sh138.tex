\begin{center}
\section*{بخش ۱۳۸ - پرسیدن پادشاه قاصدا ایاز را کی چندین غم و شادی با چارق و پوستین کی جمادست می‌گویی تا ایاز را در سخن آورد}
\label{sec:sh138}
\addcontentsline{toc}{section}{\nameref{sec:sh138}}
\begin{longtable}{l p{0.5cm} r}
ای ایاز این مهرها بر چارقی
&&
چیست آخر هم‌چو بر بت عاشقی
\\
هم‌چو مجنون از رخ لیلی خویش
&&
کرده‌ای تو چارقی را دین و کیش
\\
با دو کهنه مهر جان آمیخته
&&
هر دو را در حجره‌ای آویخته
\\
چند گویی با دو کهنه نو سخن
&&
در جمادی می‌دمی سر کهن
\\
چون عرب با ربع و اطلال ای ایاز
&&
می‌کشی از عشق گفت خود دراز
\\
چارقت ربع کدامین آصفست
&&
پوستین گویی که کرتهٔ یوسفست
\\
هم‌چو ترسا که شمارد با کشش
&&
جرم یکساله زنا و غل و غش
\\
تا بیامرزد کشش زو آن گناه
&&
عفو او را عفو داند از اله
\\
نیست آگه آن کشش از جرم و داد
&&
لیک بس جادوست عشق و اعتقاد
\\
دوستی و وهم صد یوسف تند
&&
اسحر از هاروت و ماروتست خود
\\
صورتی پیدا کند بر یاد او
&&
جذب صورت آردت در گفت و گو
\\
رازگویی پیش صورت صد هزار
&&
آن چنان که یار گوید پیش یار
\\
نه بدانجا صورتی نه هیکلی
&&
زاده از وی صد الست و صد بلی
\\
آن چنان که مادری دل‌برده‌ای
&&
پیش گور بچهٔ نومرده‌ای
\\
رازها گوید به جد و اجتهاد
&&
می‌نماید زنده او را آن جماد
\\
حی و قایم داند او آن خاک را
&&
چشم و گوشی داند او خاشاک را
\\
پیش او هر ذرهٔ آن خاک گور
&&
گوش دارد هوش دارد وقت شور
\\
مستمع داند به جد آن خاک را
&&
خوش نگر این عشق ساحرناک را
\\
آنچنان بر خاک گور تازه او
&&
دم‌بدم خوش می‌نهد با اشک رو
\\
که بوقت زندگی هرگز چنان
&&
روی ننهادست بر پور چو جان
\\
از عزا چون چند روزی بگذرد
&&
آتش آن عشق او ساکن شود
\\
عشق بر مرده نباشد پایدار
&&
عشق را بر حی جان‌افزای دار
\\
بعد از آن زان گور خود خواب آیدش
&&
از جمادی هم جمادی زایدش
\\
زانک عشق افسون خود بربود و رفت
&&
ماند خاکستر چو آتش رفت تفت
\\
آنچ بیند آن جوان در آینه
&&
پیر اندر خشت می‌بیند همه
\\
پیر عشق تست نه ریش سپید
&&
دستگیر صد هزاران ناامید
\\
عشق صورتها بسازد در فراق
&&
نامصور سر کند وقت تلاق
\\
که منم آن اصل اصل هوش و مست
&&
بر صور آن حسن عکس ما بدست
\\
پرده‌ها را این زمان برداشتم
&&
حسن را بی‌واسطه بفراشتم
\\
زانک بس با عکس من در بافتی
&&
قوت تجرید ذاتم یافتی
\\
چون ازین سو جذبهٔ من شد روان
&&
او کشش را می‌نبیند در میان
\\
مغفرت می‌خواهد از جرم و خطا
&&
از پس آن پرده از لطف خدا
\\
چون ز سنگی چشمه‌ای جاری شود
&&
سنگ اندر چشمه متواری شود
\\
کس نخواهد بعد از آن او را حجر
&&
زانک جاری شد از آن سنگ آن گهر
\\
کاسه‌ها دان این صور را واندرو
&&
آنچ حق ریزد بدان گیرد علو
\\
\end{longtable}
\end{center}
