\begin{center}
\section*{بخش ۱۴۲ - حکایت کافری کی گفتندش در عهد ابا یزید کی مسلمان شو و جواب گفتن او ایشان را}
\label{sec:sh142}
\addcontentsline{toc}{section}{\nameref{sec:sh142}}
\begin{longtable}{l p{0.5cm} r}
بود گبری در زمان بایزید
&&
گفت او را یک مسلمان سعید
\\
که چه باشد گر تو اسلام آوری
&&
تا بیابی صد نجات و سروری
\\
گفت این ایمان اگر هست ای مرید
&&
آنک دارد شیخ عالم بایزید
\\
من ندارم طاقت آن تاب آن
&&
که آن فزون آمد ز کوششهای جان
\\
گرچه در ایمان و دین ناموقنم
&&
لیک در ایمان او بس مؤمنم
\\
دارم ایمان که آن ز جمله برترست
&&
بس لطیف و با فروغ و با فرست
\\
مؤمن ایمان اویم در نهان
&&
گرچه مهرم هست محکم بر دهان
\\
باز ایمان خود گر ایمان شماست
&&
نه بدان میلستم و نه مشتهاست
\\
آنک صد میلش سوی ایمان بود
&&
چون شما را دید آن فاتر شود
\\
زانک نامی بیند و معنیش نی
&&
چون بیابان را مفازه گفتنی
\\
عشق او ز آورد ایمان بفسرد
&&
چون به ایمان شما او بنگرد
\\
\end{longtable}
\end{center}
