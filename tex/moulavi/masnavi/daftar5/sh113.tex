\begin{center}
\section*{بخش ۱۱۳ - حکایت شیخ محمد سررزی غزنوی قدس الله سره}
\label{sec:sh113}
\addcontentsline{toc}{section}{\nameref{sec:sh113}}
\begin{longtable}{l p{0.5cm} r}
زاهدی در غزنی از دانش مزی
&&
بد محمد نام و کفیت سررزی
\\
بود افطارش سر رز هر شبی
&&
هفت سال او دایم اندر مطلبی
\\
بس عجایب دید از شاه وجود
&&
لیک مقصودش جمال شاه بود
\\
بر سر که رفت آن از خویش سیر
&&
گفت بنما یا فتادم من به زیر
\\
گفت نامد مهلت آن مکرمت
&&
ور فرو افتی نمیری نکشمت
\\
او فرو افکند خود را از وداد
&&
در میان عمق آبی اوفتاد
\\
چون نمرد از نکس آن جان‌سیر مرد
&&
از فراق مرگ بر خود نوحه کرد
\\
کین حیات او را چو مرگی می‌نمود
&&
کار پیشش بازگونه گشته بود
\\
موت را از غیب می‌کرد او کدی
&&
ان فی موتی حیاتی می‌زدی
\\
موت را چون زندگی قابل شده
&&
با هلاک جان خود یک دل شده
\\
سیف و خنجر چون علی ریحان او
&&
نرگس و نسرین عدوی جان او
\\
بانگ آمد رو ز صحرا سوی شهر
&&
بانگ طرفه از ورای سر و جهر
\\
گفت ای دانای رازم مو به مو
&&
چه کنم در شهر از خدمت بگو
\\
گفت خدمت آنک بهر ذل نفس
&&
خویش را سازی تو چون عباس دبس
\\
مدتی از اغنیا زر می‌ستان
&&
پس به درویشان مسکین می‌رسان
\\
خدمتت اینست تا یک چند گاه
&&
گفت سمعا طاعة ای جان‌پناه
\\
بس سؤال و بس جواب و ماجرا
&&
بد میان زاهد و رب الوری
\\
که زمین و آسمان پر نور شد
&&
در مقالات آن همه مذکور شد
\\
لیک کوته کردم آن گفتار را
&&
تا ننوشد هر خسی اسرار را
\\
\end{longtable}
\end{center}
