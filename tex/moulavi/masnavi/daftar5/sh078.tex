\begin{center}
\section*{بخش ۷۸ - در معنی این کی ارنا الاشیاء کما هی و معنی این کی لو کشف الغطاء ما از ددت یقینا و قوله  در هر که تو از دیدهٔ بد می‌نگری  از چنبرهٔ وجود خود می‌نگری  پایهٔ کژ کژ افکند سایه}
\label{sec:sh078}
\addcontentsline{toc}{section}{\nameref{sec:sh078}}
\begin{longtable}{l p{0.5cm} r}
ای خروسان از وی آموزید بانگ
&&
بانگ بهر حق کند نه بهر دانگ
\\
صبح کاذب آید و نفریبدش
&&
صبح کاذب عالم و نیک و بدش
\\
اهل دنیا عقل ناقص داشتند
&&
تا که صبح صادقش پنداشتند
\\
صبح کاذب کاروانها را زدست
&&
که به بوی روز بیرون آمدست
\\
صبح کاذب خلق را رهبر مباد
&&
کو دهد بس کاروانها را به باد
\\
ای شده تو صبح کاذب را رهین
&&
صبح صادق را تو کاذب هم مبین
\\
گر نداری از نفاق و بد امان
&&
از چه داری بر برادر ظن همان
\\
بدگمان باشد همیشه زشت‌کار
&&
نامهٔ خود خواند اندر حق یار
\\
آن خسان که در کژیها مانده‌اند
&&
انبیا را ساحر و کژ خوانده‌اند
\\
وآن امیران خسیس قلب‌ساز
&&
این گمان بردند بر حجرهٔ ایاز
\\
کو دفینه دارد و گنج اندر آن
&&
ز آینهٔ خود منگر اندر دیگران
\\
شاه می‌دانست خود پاکی او
&&
بهر ایشان کرد او آن جست و جو
\\
کای امیر آن حجره را بگشای در
&&
نیم شب که باشد او زان بی‌خبر
\\
تا پدید آید سگالشهای او
&&
بعد از آن بر ماست مالشهای او
\\
مر شما را دادم آن زر و گهر
&&
من از آن زرها نخواهم جز خبر
\\
این همی‌گفت و دل او می‌طپید
&&
از برای آن ایاز بی ندید
\\
که منم کین بر زبانم می‌رود
&&
این جفاگر بشنود او چون شود
\\
باز می‌گوید به حق دین او
&&
که ازین افزون بود تمکین او
\\
کی به قذف زشت من طیره شود
&&
وز غرض وز سر من غافل بود
\\
مبتلی چون دید تاویلات رنج
&&
برد بیند کی شود او مات رنج
\\
صاحب تاویل ایاز صابرست
&&
کو به بحر عاقبتها ناظرست
\\
هم‌چو یوسف خواب این زندانیان
&&
هست تعبیرش به پیش او عیان
\\
خواب خود را چون نداند مرد خیر
&&
کو بود واقف ز سر خواب غیر
\\
گر زنم صد تیغ او را ز امتحان
&&
کم نگردد وصلت آن مهربان
\\
داند او که آن تیغ بر خود می‌زنم
&&
من ویم اندر حقیقت او منم
\\
\end{longtable}
\end{center}
