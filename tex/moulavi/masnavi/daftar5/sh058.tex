\begin{center}
\section*{بخش ۵۸ - مریدی در آمد به خدمت شیخ و ازین شیخ پیر سن نمی‌خواهم بلک پیرعقل و معرفت و اگر چه عیسیست علیه‌السلام در گهواره و یحیی است علیه‌السلام در مکتب کودکان مریدی شیخ را گریان دید او نیز موافقت کرد و گریست چون فارغ شد و به در آمد مریدی دیگر کی از حال شیخ واقف‌تر بود از سر غیرت در عقب او تیز بیرون آمد گفتش ای برادر من ترا گفته باشم الله الله تا نیندیشی و نگویی کی شیخ می‌گریست و من نیز می‌گریستم کی سی سال ریاضت بی‌ریا باید کرد و از عقبات و دریاهای پر نهنگ و کوههای بلند پر شیر و پلنگ می‌باید گذشت تا بدان گریهٔ شیخ رسی یا نرسی اگر رسی شکر زویت لی الارض گویی بسیار}
\label{sec:sh058}
\addcontentsline{toc}{section}{\nameref{sec:sh058}}
\begin{longtable}{l p{0.5cm} r}
یک مریدی اندر آمد پیش پیر
&&
پیر اندر گریه بود و در نفیر
\\
شیخ را چون دید گریان آن مرید
&&
گشت گریان آب از چشمش دوید
\\
گوشور یک‌بار خندد کر دو بار
&&
چونک لاغ املی کند یاری بیار
\\
بار اول از ره تقلید و سوم
&&
که همی‌بیند که می‌خندند قوم
\\
کر بخندد هم‌چو ایشان آن زمان
&&
بیخبر از حالت خندندگان
\\
باز وا پرسد که خنده بر چه بود
&&
پس دوم کرت بخندد چون شنود
\\
پس مقلد نیز مانند کرست
&&
اندر آن شادی که او را در سرست
\\
پرتو شیخ آمد و منهل ز شیخ
&&
فیض شادی نه از مریدان بل ز شیخ
\\
چون سبد در آب و نوری بر زجاج
&&
گر ز خود دانند آن باشد خداج
\\
چون جدا گردد ز جو داند عنود
&&
که اندرو آن آب خوش از جوی بود
\\
آبگینه هم بداند از غروب
&&
که آن لمع بود از مه تابان خوب
\\
چونک چشمش را گشاید امر قم
&&
پس بخندد چون سحر بار دوم
\\
خنده‌ش آید هم بر آن خندهٔ خودش
&&
که در آن تقلید بر می‌آمدش
\\
گوید از چندین ره دور و دراز
&&
کین حقیقت بود و این اسرار و راز
\\
من در آن وادی چگونه خود ز دور
&&
شادیی می‌کردم از عمیا و شور
\\
من چه می‌بستم خیال و آن چه بود
&&
درک سستم سست نقشی می‌نمود
\\
طفل راه را فکرت مردان کجاست
&&
کو خیال او و کو تحقیق راست
\\
فکر طفلان دایه باشد یا که شیر
&&
یا مویز و جوز یا گریه و نفیر
\\
آن مقلد هست چون طفل علیل
&&
گر چه دارد بحث باریک و دلیل
\\
آن تعمق در دلیل و در شکال
&&
از بصیرت می‌کند او را گسیل
\\
مایه‌ای کو سرمهٔ سر ویست
&&
برد و در اشکال گفتن کار بست
\\
ای مقلد از بخارا باز گرد
&&
رو به خواری تا شوی تو شیرمرد
\\
تا بخارای دگر بینی درون
&&
صفدران در محفلش لا یفقهون
\\
پیک اگر چه در زمین چابک‌تگیست
&&
چون به دریا رفت بسکسته رگیست
\\
او حملناهم بود فی‌البر و بس
&&
آنک محمولست در بحر اوست کس
\\
بخشش بسیار دارد شه بدو
&&
ای شده در وهم و تصویری گرو
\\
آن مرید ساده از تقلید نیز
&&
گریه‌ای می‌کرد وفق آن عزیز
\\
او مقلدوار هم‌چون مرد کر
&&
گریه می‌دید و ز موجب بی‌خبر
\\
چون بسی بگریست خدمت کرد و رفت
&&
از پیش آمد مرید خاص تفت
\\
گفت ای گریان چو ابر بی‌خبر
&&
بر وفاق گریهٔ شیخ نظر
\\
الله الله الله ای وافی مرید
&&
گر چه درتقلید هستی مستفید
\\
تا نگویی دیدم آن شه می‌گریست
&&
من چو او بگریستم که آن منکریست
\\
گریهٔ پر جهل و پر تقلید و ظن
&&
نیست هم‌چون گریهٔ آن متمن
\\
تو قیاس گریه بر گریه مساز
&&
هست زین گریه بدان راه دراز
\\
هست آن از بعد سی‌ساله جهاد
&&
عقل آنجا هیچ نتواند فتاد
\\
هست زان سوی خرد صد مرحله
&&
عقل را واقف مدان زان قافله
\\
گریهٔ او نه از غمست و نه از فرح
&&
روح داند گریهٔ عین الملح
\\
گریهٔ او خندهٔ او آن سریست
&&
زانچ وهم عقل باشد آن بریست
\\
آب دیدهٔ او چو دیدهٔ او بود
&&
دیدهٔ نادیده دیده کی شود
\\
آنچ او بیند نتان کردن مساس
&&
نه از قیاس عقل و نه از راه حواس
\\
شب گریزد چونک نور آید ز دور
&&
پس چه داند ظلمت شب حال نور
\\
پشه بگریزد ز باد با دها
&&
پس چه داند پشه ذوق بادها
\\
چون قدیم آید حدث گردد عبث
&&
پس کجا داند قدیمی را حدث
\\
بر حدث چون زد قدم دنگش کند
&&
چونک کردش نیست هم‌رنگش کند
\\
گر بخواهی تو بیایی صد نظیر
&&
لیک من پروا ندارم ای فقیر
\\
این الم و حم این حروف
&&
چون عصای موسی آمد در وقوف
\\
حرفها ماند بدین حرف از برون
&&
لیک باشد در صفات این زبون
\\
هر که گیرد او عصایی ز امتحان
&&
کی بود چون آن عصا وقت بیان
\\
عیسویست این دم نه هر باد و دمی
&&
که برآید از فرح یا از غمی
\\
این الم است و حم ای پدر
&&
آمدست از حضرت مولی البشر
\\
هر الف لامی چه می‌ماند بدین
&&
گر تو جان داری بدین چشمش مبین
\\
گر چه ترکیبش حروفست ای همام
&&
می‌بماند هم به ترکیب عوام
\\
هست ترکیب محمد لحم و پوست
&&
گرچه در ترکیب هر تن جنس اوست
\\
گوشت دارد پوست دارد استخوان
&&
هیچ این ترکیب را باشد همان
\\
که اندر آن ترکیب آمد معجزات
&&
که همه ترکیبها گشتند مات
\\
هم‌چنان ترکیب حم کتاب
&&
هست بس بالا و دیگرها نشیب
\\
زانک زین ترکیب آید زندگی
&&
هم‌چو نفخ صور در درماندگی
\\
اژدها گردد شکافد بحر را
&&
چون عصا حم از داد خدا
\\
ظاهرش ماند به ظاهرها ولیک
&&
قرص نان از قرص مه دورست نیک
\\
گریهٔ او خندهٔ او نطق او
&&
نیست از وی هست محض خلق هو
\\
چونک ظاهرها گرفتند احمقان
&&
وآن دقایق شد ازیشان بس نهان
\\
لاجرم محجوب گشتند از غرض
&&
که دقیقه فوت شد در معترض
\\
\end{longtable}
\end{center}
