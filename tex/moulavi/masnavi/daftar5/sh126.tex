\begin{center}
\section*{بخش ۱۲۶ - صید کردن شیر آن خر را و تشنه  شدن شیر از کوشش رفت به چشمه تا آب خورد تا باز آمدن شیر جگربند و دل و گرده را روباه خورده بود کی لطیفترست شیر طلب کرد دل و جگر نیافت از روبه پرسید کی کو دل و جگر روبه گفت اگر او  را دل و جگر بودی آنچنان  سیاستی دیده بود آن روز و به هزار حیله جان برده کی بر تو باز آمدی لوکنا نسمع او نعقل ماکنا فی اصحاب السعیر}
\label{sec:sh126}
\addcontentsline{toc}{section}{\nameref{sec:sh126}}
\begin{longtable}{l p{0.5cm} r}
برد خر را روبهک تا پیش شیر
&&
پاره‌پاره کردش آن شیر دلیر
\\
تشنه شد از کوشش آن سلطان دد
&&
رفت سوی چشمه تا آبی خورد
\\
روبهک خورد آن جگربند و دلش
&&
آن زمان چون فرصتی شد حاصلش
\\
شیر چون وا گشت از چشمه به خور
&&
جست در خر دل نه دل بد نه جگر
\\
گفت روبه را جگر کو دل چه شد
&&
که نباشد جانور را زین دو بد
\\
گفت گر بودی ورا دل یا جگر
&&
کی بدینجا آمدی بار دگر
\\
آن قیامت دیده بود و رستخیز
&&
وآن ز کوه افتادن و هول و گریز
\\
گر جگر بودی ورا یا دل بدی
&&
بار دیگر کی بر تو آمدی
\\
چون نباشد نور دل دل نیست آن
&&
چون نباشد روح جز گل نیست آن
\\
آن زجاجی کو ندارد نور جان
&&
بول و قاروره‌ست قندیلش مخوان
\\
نور مصباحست داد ذوالجلال
&&
صنعت خلقست آن شیشه و سفال
\\
لاجرم در ظرف باشد اعتداد
&&
در لهبها نبود الا اتحاد
\\
نور شش قندیل چون آمیختند
&&
نیست اندر نورشان اعداد و چند
\\
آن جهود از ظرفها مشرک شده‌ست
&&
نور دید آنمؤمنو مدرک شده‌ست
\\
چون نظر بر ظرف افتد روح را
&&
پس دو بیند شیث را و نوح را
\\
جو که آبش هست جو خود آن بود
&&
آدمی آنست کو را جان بود
\\
این نه مردانند اینها صورتند
&&
مردهٔ نانند و کشتهٔ شهوتند
\\
\end{longtable}
\end{center}
