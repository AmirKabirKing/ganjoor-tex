\begin{center}
\section*{بخش ۱۰۰ - در تقریر معنی توکل حکایت آن زاهد کی توکل را امتحان می‌کرد از میان اسباب و شهر برون آمد و از قوارع و ره‌گذر خلق دور  شد و ببن کوهی مهجوری مفقودی در غایت گرسنگی سر بر سر سنگی  نهاد و خفت و با خود گفت توکل کردم بر سبب‌سازی و رزاقی تو و از اسباب منقطع شدم تا ببینم سببیت توکل را}
\label{sec:sh100}
\addcontentsline{toc}{section}{\nameref{sec:sh100}}
\begin{longtable}{l p{0.5cm} r}
آن یکی زاهد شنود از مصطفی
&&
که یقین آید به جان رزق از خدا
\\
گر بخواهی ور نخواهی رزق تو
&&
پیش تو آید دوان از عشق تو
\\
از برای امتحان آن مرد رفت
&&
در بیابان نزد کوهی خفت تفت
\\
که ببینم رزق می‌آید به من
&&
تا قوی گردد مرا در رزق ظن
\\
کاروانی راه گم کرد و کشید
&&
سوی کوه آن ممتحن را خفته دید
\\
گفت این مرد این طرف چونست عور
&&
در بیابان از ره و از شهر دور
\\
ای عجب مرده‌ست یا زنده که او
&&
می‌نترسد هیچ از گرگ و عدو
\\
آمدند و دست بر وی می‌زدند
&&
قاصدا چیزی نگفت آن ارجمند
\\
هم نجنبید و نجنبانید سر
&&
وا نکرد از امتحان هم او بصر
\\
پس بگفتند این ضعیف بی‌مراد
&&
از مجاعت سکته اندر اوفتاد
\\
نان بیاوردند و در دیگی طعام
&&
تا بریزندش به حلقوم و به کام
\\
پس بقاصد مرد دندان سخت کرد
&&
تا ببیند صدق آن میعاد مرد
\\
رحمشان آمد که این بس بی‌نواست
&&
وز مجاعت هالک مرگ و فناست
\\
کارد آوردند قوم اشتافتند
&&
بسته دندانهاش را بشکافتند
\\
ریختند اندر دهانش شوربا
&&
می‌فشردند اندرو نان‌پاره‌ها
\\
گفت ای دل گرچه خود تن می‌زنی
&&
راز می‌دانی و نازی می‌کنی
\\
گفت دل دانم و قاصد می‌کنم
&&
رازق الله است بر جان و تنم
\\
امتحان زین بیشتر خود چون بود
&&
رزق سوی صابران خوش می‌رود
\\
\end{longtable}
\end{center}
