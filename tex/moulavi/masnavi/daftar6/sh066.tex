\begin{center}
\section*{بخش ۶۶ - تمامی قصهٔ آن فقیر و نشان جای آن گنج}
\label{sec:sh066}
\addcontentsline{toc}{section}{\nameref{sec:sh066}}
\begin{longtable}{l p{0.5cm} r}
اندر آن رقعه نبشته بود این
&&
که برون شهر گنجی دان دفین
\\
آن فلان قبه که در وی مشهدست
&&
پشت او در شهر و در در فدفدست
\\
پشت با وی کن تو رو در قبله آر
&&
وانگهان از قوس تیری بر گذار
\\
چون فکندی تیر از قوس ای سعاد
&&
بر کن آن موضع که تیرت اوفتاد
\\
پس کمان سخت آورد آن فتی
&&
تیر پرانید در صحن فضا
\\
زو تبر آورد و بیل او شاد شاد
&&
کند آن موضع که تیرش اوفتاد
\\
کند شد هم او و هم بیل و تبر
&&
خود ندید از گنج پنهانی اثر
\\
هم‌چنین هر روز تیر انداختی
&&
لیک جای گنج را نشناختی
\\
چونک این را پیشه کرد او بر دوام
&&
فجفجی در شهر افتاد و عوام
\\
\end{longtable}
\end{center}
