\begin{center}
\section*{بخش ۴۹ - جواب دادن قاضی صوفی را}
\label{sec:sh049}
\addcontentsline{toc}{section}{\nameref{sec:sh049}}
\begin{longtable}{l p{0.5cm} r}
گفت قاضی واجب آیدمان رضا
&&
هر قفا و هر جفا کارد قضا
\\
خوش‌دلم در باطن از حکم زبر
&&
گرچه شد رویم ترش کالحق مر
\\
این دلم باغست و چشمم ابروش
&&
ابر گرید باغ خندد شاد و خوش
\\
سال قحط از آفتاب خیره‌خند
&&
باغها در مرگ و جان کندن رسند
\\
ز امر حق وابکوا کثیرا خوانده‌ای
&&
چون سر بریان چه خندان مانده‌ای
\\
روشنی خانه باشی هم‌چو شمع
&&
گر فرو پاشی تو هم‌چون شمع دمع
\\
آن ترش‌رویی مادر یا پدر
&&
حافظ فرزند شد از هر ضرر
\\
ذوق خنده دیده‌ای ای خیره‌خند
&&
ذوق گریه بین که هست آن کان قند
\\
چون جهنم گریه آرد یاد آن
&&
پس جهنم خوشتر آید از جنان
\\
خنده‌ها در گریه‌ها آمد کتیم
&&
گنج در ویرانه‌ها جو ای سلیم
\\
ذوق در غمهاست پی گم کرده‌اند
&&
آب حیوان را به ظلمت برده‌اند
\\
بازگونه نعل در ره تا رباط
&&
چشمها را چار کن در احتیاط
\\
چشمها را چار کن در اعتبار
&&
یار کن با چشم خود دو چشم یار
\\
امرهم شوری بخوان اندر صحف
&&
یار را باش و مگوش از ناز اف
\\
یار باشد راه را پشت و پناه
&&
چونک نیکو بنگری یارست راه
\\
چونک در یاران رسی خامش نشین
&&
اندر آن حلقه مکن خود را نگین
\\
در نماز جمعه بنگر خوش به هوش
&&
جمله جمعند و یک‌اندیشه و خموش
\\
رختها را سوی خاموشی کشان
&&
چون نشان جویی مکن خود را نشان
\\
گفت پیغامبر که در بحر هموم
&&
در دلالت دان تو یاران را نجوم
\\
چشم در استارگان نه ره بجو
&&
نطق تشویش نظر باشد مگو
\\
گر دو حرف صدق گویی ای فلان
&&
گفت تیره در تبع گردد روان
\\
این نخواندی کالکلام ای مستهام
&&
فی شجون حره جر الکلام
\\
هین مشو شارع در آن حرف رشد
&&
که سخن زو مر سخن را می‌کشد
\\
نیست در ضبطت چو بگشادی دهان
&&
از پی صافی شود تیره روان
\\
آنک معصوم ره وحی خداست
&&
چون همه صافست بگشاید رواست
\\
زانک ما ینطق رسول بالهوی
&&
کی هوا زاید ز معصوم خدا
\\
خویشتن را ساز منطیقی ز حال
&&
تا نگردی هم‌چو من سخرهٔ مقال
\\
\end{longtable}
\end{center}
