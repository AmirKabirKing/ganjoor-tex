\begin{center}
\section*{بخش ۱۱۹ - سبب تاخیر اجابت دعای ممن}
\label{sec:sh119}
\addcontentsline{toc}{section}{\nameref{sec:sh119}}
\begin{longtable}{l p{0.5cm} r}
ای بسا مخلص که نالد در دعا
&&
تا رود دود خلوصش بر سما
\\
تا رود بالای این سقف برین
&&
بوی مجمر از انین المذنبین
\\
پس ملایک با خدا نالند زار
&&
کای مجیب هر دعا وی مستجار
\\
بندهٔمؤمنتضرع می‌کند
&&
او نمی‌داند به جز تو مستند
\\
تو عطا بیگانگان را می‌دهی
&&
از تو دارد آرزو هر مشتهی
\\
حق بفرماید که نه از خواری اوست
&&
عین تاخیر عطا یاری اوست
\\
حاجت آوردش ز غفلت سوی من
&&
آن کشیدش مو کشان در کوی من
\\
گر بر آرم حاجتش او وا رود
&&
هم در آن بازیچه مستغرق شود
\\
گرچه می‌نالد به جان یا مستجار
&&
دل شکسته سینه‌خسته گو بزار
\\
خوش همی‌آید مرا آواز او
&&
وآن خدایا گفتن و آن راز او
\\
وانک اندر لابه و در ماجرا
&&
می‌فریباند بهر نوعی مرا
\\
طوطیان و بلبلان را از پسند
&&
از خوش آوازی قفس در می‌کنند
\\
زاغ را و چغد را اندر قفس
&&
کی کنند این خود نیامد در قصص
\\
پیش شاهد باز چون آید دو تن
&&
آن یکی کمپیر و دیگر خوش‌ذقن
\\
هر دو نان خواهند او زوتر فطیر
&&
آرد و کمپیر را گوید که گیر
\\
وآن دگر را که خوشستش قد و خد
&&
کی دهد نان بل به تاخیر افکند
\\
گویدش بنشین زمانی بی‌گزند
&&
که به خانه نان تازه می‌پزند
\\
چون رسد آن نان گرمش بعد کد
&&
گویدش بنشین که حلوا می‌رسد
\\
هم برین فن داردارش می‌کند
&&
وز ره پنهان شکارش می‌کند
\\
که مرا کاریست با تو یک زمان
&&
منتظر می‌باش ای خوب جهان
\\
بی‌مرادی مومنان از نیک و بد
&&
تو یقین می‌دان که بهر این بود
\\
\end{longtable}
\end{center}
