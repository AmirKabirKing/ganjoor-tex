\begin{center}
\section*{بخش ۱۲۷ - رفتن قاضی به خانهٔ زن جوحی و حلقه زدن جوحی به خشم بر در و گریختن قاضی در صندوقی الی آخره}
\label{sec:sh127}
\addcontentsline{toc}{section}{\nameref{sec:sh127}}
\begin{longtable}{l p{0.5cm} r}
مکر زن پایان ندارد رفت شب
&&
قاضی زیرک سوی زن بهر دب
\\
زن دو شمع و نقل مجلس راست کرد
&&
گفت ما مستیم بی این آب‌خورد
\\
اندر آن دم جوحی آمد در بزد
&&
جست قاضی مهربی تا در خزد
\\
غیر صندوقی ندید او خلوتی
&&
رفت در صندوق از خوف آن فتی
\\
اندر آمد جوحی و گفت ای حریف
&&
اتی وبالم در ربیع و در خریف
\\
من چه دارم که فداات نیست آن
&&
که ز من فریاد داری هر زمان
\\
بر لب خشکم گشادستی زبان
&&
گاه مفلس خوانیم گه قلتبان
\\
این دو علت گر بود ای جان مرا
&&
آن یکی از تست و دیگر از خدا
\\
من چه دارم غیر آن صندوق کان
&&
هست مایهٔ تهمت و پایهٔ گمان
\\
خلق پندارند زر دارم درون
&&
داد واگیرند از من زین ظنون
\\
صورت صندوق بس زیباست لیک
&&
از عروض و سیم و زر خالیست نیک
\\
چون تن زراق خوب و با وقار
&&
اندر آن سله نیابی غیر مار
\\
من برم صندوق را فردا به کو
&&
پس بسوزم در میان چارسو
\\
تا ببیند مؤمن و گبر و جهود
&&
که درین صندوق جز لعنت نبود
\\
گفت زن هی در گذر ای مرد ازین
&&
خورد سوگند او که نکنم جز چنین
\\
از پگه حمال آورد او چو باد
&&
زود آن صندوق بر پشتش نهاد
\\
اندر آن صندوق قاضی از نکال
&&
بانگ می‌زد کای حمال و ای حمال
\\
کرد آن حمال راست و چپ نظر
&&
کز چه سو در می‌رسد بانک و خبر
\\
هاتفست این داعی من ای عجب
&&
یا پری‌ام می‌کند پنهان طلب
\\
چون پیاپی گشت آن آواز و بیش
&&
گفت هاتف نیست باز آمد به خویش
\\
عاقبت دانست کان بانگ و فغان
&&
بد ز صندوق و کسی در وی نهان
\\
عاشقی کو در غم معشوق رفت
&&
گر چه بیرونست در صندوق رفت
\\
عمر در صندوق برد از اندهان
&&
جز که صندوقی نبیند از جهان
\\
آن سری که نیست فوق آسمان
&&
از هوس او را در آن صندوق دان
\\
چون ز صندوق بدن بیرون رود
&&
او ز گوری سوی گوری می‌شود
\\
این سخن پایان ندارد قاضیش
&&
گفت ای حمال و ای صندوق‌کش
\\
از من آگه کن درون محکمه
&&
نایبم را زودتر با این همه
\\
تا خرد این را به زر زین بی‌خرد
&&
هم‌چنین بسته به خانهٔ ما برد
\\
ای خدا بگمار قومی روحمند
&&
تا ز صندوق بدنمان وا خرند
\\
خلق را از بند صندوق فسون
&&
کی خرد جز انبیا و مرسلون
\\
از هزاران یک کسی خوش‌منظرست
&&
که بداند کو به صندوق اندرست
\\
او جهان را دیده باشد پیش از آن
&&
تا بدان ضد این ضدش گردد عیان
\\
زین سبب که علم ضالهٔمؤمنست
&&
عارف ضالهٔ خودست و موقنست
\\
آنک هرگز روز نیکو خود ندید
&&
او درین ادبار کی خواهد طپید
\\
یا به طفلی در اسیری اوفتاد
&&
یا خود از اول ز مادر بنده زاد
\\
ذوق آزادی ندیده جان او
&&
هست صندوق صور میدان او
\\
دایما محبوس عقلش در صور
&&
از قفس اندر قفس دارد گذر
\\
منفذش نه از قفس سوی علا
&&
در قفس‌ها می‌رود از جا به جا
\\
در نبی ان استطعتم فانفذوا
&&
این سخن با جن و انس آمد ز هو
\\
گفت منفذ نیست از گردونتان
&&
جز به سلطان و به وحی آسمان
\\
گر ز صندوقی به صندوقی رود
&&
او سمایی نیست صندوقی بود
\\
فرجه صندوق نو نو منکرست
&&
در نیابد کو به صندوق اندرست
\\
گر نشد غره بدین صندوق‌ها
&&
هم‌چو قاضی جوید اطلاق و رها
\\
آنک داند این نشانش آن شناس
&&
کو نباشد بی‌فغان و بی‌هراس
\\
هم‌چو قاضی باشد او در ارتعاد
&&
کی برآید یک دمی از جانش شاد
\\
\end{longtable}
\end{center}
