\begin{center}
\section*{بخش ۲۴ - نکته گفتن آن شاعر جهت طعن شیعه حلب}
\label{sec:sh024}
\addcontentsline{toc}{section}{\nameref{sec:sh024}}
\begin{longtable}{l p{0.5cm} r}
گفت آری لیک کو دور یزید
&&
کی بدست این غم چه دیر اینجا رسید
\\
چشم کوران آن خسارت را بدید
&&
گوش کران آن حکایت را شنید
\\
خفته بودستید تا اکنون شما
&&
که کنون جامه دریدیت از عزا
\\
پس عزا بر خود کنید ای خفتگان
&&
زانک بد مرگیست این خواب گران
\\
روح سلطانی ز زندانی بجست
&&
جامه چه درانیم و چون خاییم دست
\\
چونک ایشان خسرو دین بوده‌اند
&&
وقت شادی شد چو بشکستند بند
\\
سوی شادروان دولت تاختند
&&
کنده و زنجیر را انداختند
\\
روز ملکست و گش و شاهنشهی
&&
گر تو یک ذره ازیشان آگهی
\\
ور نه‌ای آگه برو بر خود گری
&&
زانک در انکار نقل و محشری
\\
بر دل و دین خرابت نوحه کن
&&
که نمی‌بیند جز این خاک کهن
\\
ور همی‌بیند چرا نبود دلیر
&&
پشتدار و جانسپار و چشم‌سیر
\\
در رخت کو از می دین فرخی
&&
گر بدیدی بحر کو کف سخی
\\
آنک جو دید آب را نکند دریغ
&&
خاصه آن کو دید آن دریا و میغ
\\
\end{longtable}
\end{center}
