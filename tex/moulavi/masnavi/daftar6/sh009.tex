\begin{center}
\section*{بخش ۹ - قصه‌ای هم در تقریر این}
\label{sec:sh009}
\addcontentsline{toc}{section}{\nameref{sec:sh009}}
\begin{longtable}{l p{0.5cm} r}
شرفه‌ای بشنید در شب معتمد
&&
برگرفت آتش‌زنه که آتش زند
\\
دزد آمد آن زمان پیشش نشست
&&
چون گرفت آن سوخته می‌کرد پست
\\
می‌نهاد آنجا سر انگشت را
&&
تا شود استارهٔ آتش فنا
\\
خواجه می‌پنداشت کز خود می‌مرد
&&
این نمی‌دید او که دزدش می‌کشد
\\
خواجه گفت این سوخته نمناک بود
&&
می‌مرد استاره از تریش زود
\\
بس که ظلمت بود و تاریکی ز پیش
&&
می‌ندید آتش‌کشی را پیش خویش
\\
این چنین آتش‌کشی اندر دلش
&&
دیدهٔ کافر نبیند از عمش
\\
چون نمی‌داند دل داننده‌ای
&&
هست با گردنده گرداننده‌ای
\\
چون نمی‌گویی که روز و شب به خود
&&
بی‌خداوندی کی آید کی رود
\\
گرد معقولات می‌گردی ببین
&&
این چنین بی‌عقلی خود ای مهین
\\
خانه با بنا بود معقول‌تر
&&
یا که بی‌بنا بگو ای کم‌هنر
\\
خط با کاتب بود معقول‌تر
&&
یا که بی‌کاتب بیندیش ای پسر
\\
جیم گوش و عین چشم و میم فم
&&
چون بود بی‌کاتبی ای متهم
\\
شمع روشن بی‌ز گیراننده‌ای
&&
یا بگیرانندهٔ داننده‌ای
\\
صنعت خوب از کف شل ضریر
&&
باشد اولی یا بگیرایی بصیر
\\
پس چو دانستی که قهرت می‌کند
&&
بر سرت دبوس محنت می‌زند
\\
پس بکن دفعش چو نمرودی به جنگ
&&
سوی او کش در هوا تیری خدنگ
\\
هم‌چو اسپاه مغل بر آسمان
&&
تیر می‌انداز دفع نزع جان
\\
یا گریز از وی اگر توانی برو
&&
چون روی چون در کف اویی گرو
\\
در عدم بودی نرستی از کفش
&&
از کف او چون رهی ای دست‌خوش
\\
آرزو جستن بود بگریختن
&&
پیش عدلش خون تقوی ریختن
\\
این جهان دامست و دانه‌آرزو
&&
در گریز از دامها روی آر زو
\\
چون چنین رفتی بدیدی صد گشاد
&&
چون شدی در ضد آن دیدی فساد
\\
پس پیمبر گفت استفتوا القلوب
&&
گر چه مفتیتان برون گوید خطوب
\\
آرزو بگذار تا رحم آیدش
&&
آزمودی که چنین می‌بایدش
\\
چون نتانی جست پس خدمت کنش
&&
تا روی از حبس او در گلشنش
\\
دم به دم چون تو مراقب می‌شوی
&&
داد می‌بینی و داور ای غوی
\\
ور ببندی چشم خود را ز احتجاب
&&
کار خود را کی گذارد آفتاب
\\
\end{longtable}
\end{center}
