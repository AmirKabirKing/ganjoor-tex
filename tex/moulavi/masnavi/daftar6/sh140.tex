\begin{center}
\section*{بخش ۱۴۰ - مثل}
\label{sec:sh140}
\addcontentsline{toc}{section}{\nameref{sec:sh140}}
\begin{longtable}{l p{0.5cm} r}
آنچنان که گفت مادر بچه را
&&
گر خیالی آیدت در شب فرا
\\
یا بگورستان و جای سهمگین
&&
تو خیالی بینی اسود پر ز کین
\\
دل قوی دار و بکن حمله برو
&&
او بگرداند ز تو در حال رو
\\
گفت کودک آن خیال دیووش
&&
گر بدو این گفته باشد مادرش
\\
حمله آرم افتد اندر گردنم
&&
ز امر مادر پس من آنگه چون کنم
\\
تو همی‌آموزیم که چست ایست
&&
آن خیال زشت را هم مادریست
\\
دیو و مردم را ملقن آن یکیست
&&
غالب از وی گردد ار خصم اندکیست
\\
تا کدامین سوی باشد آن یواش
&&
الله‌الله رو تو هم زان سوی باش
\\
گفت اگر از مکر ناید در کلام
&&
حیله را دانسته باشد آن همام
\\
سر او را چون شناسی راست گو
&&
گفت من خامش نشینم پیش او
\\
صبر را سلم کنم سوی درج
&&
تا بر آیم صبر مفتاح الفرج
\\
ور بجوشد در حضورش از دلم
&&
منطقی بیرون ازین شادی و غم
\\
من بدانم کو فرستاد آن بمن
&&
از ضمیر چون سهیل اندر یمن
\\
در دل من آن سخن زان میمنه‌ست
&&
زانک از دل جانب دل روزنه‌ست
\\
\end{longtable}
\end{center}
