\begin{center}
\section*{بخش ۲۶ - داستان آن شخص کی بر در سرایی نیم‌شب سحوری می‌زد همسایه او را گفت کی آخر نیم‌شبست سحر نیست و دیگر آنک درین سرا کسی نیست بهر کی می‌زنی و جواب گفتن مطرب او را}
\label{sec:sh026}
\addcontentsline{toc}{section}{\nameref{sec:sh026}}
\begin{longtable}{l p{0.5cm} r}
آن یکی می‌زد سحوری بر دری
&&
درگهی بود و رواق مهتری
\\
نیم‌شب می‌زد سحوری را به جد
&&
گفت او را قایلی کای مستمد
\\
اولا وقت سحر زن این سحور
&&
نیم‌شب نبود گه این شر و شور
\\
دیگر آنک فهم کن ای بوالهوس
&&
که درین خانه درون خود هست کس
\\
کس درینجا نیست جز دیو و پری
&&
روزگار خود چه یاوه می‌بری
\\
بهر گوشی می‌زنی دف گوش کو
&&
هوش باید تا بداند هوش کو
\\
گفت گفتی بشنو از چاکر جواب
&&
تا نمانی در تحیر و اضطراب
\\
گرچه هست این دم بر تو نیم‌شب
&&
نزد من نزدیک شد صبح طرب
\\
هر شکستی پیش من پیروز شد
&&
جمله شبها پیش چشمم روز شد
\\
پیش تو خونست آب رود نیل
&&
نزد من خون نیست آبست ای نبیل
\\
در حق تو آهنست آن و رخام
&&
پیش داود نبی مومست و رام
\\
پیش تو که بس گرانست و جماد
&&
مطربست او پیش داود اوستاد
\\
پیش تو آن سنگ‌ریزه ساکتست
&&
پیش احمد او فصیح و قانتست
\\
پیش تو استون مسجد مرده‌ایست
&&
پیش احمد عاشقی دل برده‌ایست
\\
جمله اجزای جهان پیش عوام
&&
مرده و پیش خدا دانا و رام
\\
آنچ گفتی کاندرین خانه و سرا
&&
نیست کس چون می‌زنی این طبل را
\\
بهر حق این خلق زرها می‌دهند
&&
صد اساس خیر و مسجد می‌نهند
\\
مال و تن در راه حج دوردست
&&
خوش همی‌بازند چون عشاق مست
\\
هیچ می‌گویند کان خانه تهیست
&&
بلک صاحب‌خانه جان مختبیست
\\
پر همی‌بیند سرای دوست را
&&
آنک از نور الهستش ضیا
\\
بس سرای پر ز جمع و انبهی
&&
پیش چشم عاقبت‌بینان تهی
\\
هر که را خواهی تو در کعبه بجو
&&
تا بروید در زمان او پیش رو
\\
صورتی کو فاخر و عالی بود
&&
او ز بیت الله کی خالی بود
\\
او بود حاضر منزه از رتاج
&&
باقی مردم برای احتیاج
\\
هیچ می‌گویند کین لبیکها
&&
بی‌ندایی می‌کنیم آخر چرا
\\
بلک توفیقی که لبیک آورد
&&
هست هر لحظه ندایی از احد
\\
من ببو دانم که این قصر و سرا
&&
بزم جان افتاد و خاکش کیمیا
\\
مس خود را بر طریق زیر و بم
&&
تا ابد بر کیمیااش می‌زنم
\\
تا بجوشد زین چنین ضرب سحور
&&
در درافشانی و بخشایش به حور
\\
خلق در صف قتال و کارزار
&&
جان همی‌بازند بهر کردگار
\\
آن یکی اندر بلا ایوب‌وار
&&
وان دگر در صابری یعقوب‌وار
\\
صد هزاران خلق تشنه و مستمند
&&
بهر حق از طمع جهدی می‌کنند
\\
من هم از بهر خداوند غفور
&&
می‌زنم بر در به اومیدش سحور
\\
مشتری خواهی که از وی زر بری
&&
به ز حق کی باشد ای دل مشتری
\\
می‌خرد از مالت انبانی نجس
&&
می‌دهد نور ضمیری مقتبس
\\
می‌ستاند این یخ جسم فنا
&&
می‌دهد ملکی برون از وهم ما
\\
می‌ستاند قطرهٔ چندی ز اشک
&&
می‌دهد کوثر که آرد قند رشک
\\
می‌ستاند آه پر سودا و دود
&&
می‌دهد هر آه را صد جاه سود
\\
باد آهی که ابر اشک چشم راند
&&
مر خلیلی را بدان اواه خواند
\\
هین درین بازار گرم بی‌نظیر
&&
کهنه‌ها بفروش و ملک نقد گیر
\\
ور ترا شکی و ریبی ره زند
&&
تاجران انبیا را کن سند
\\
بس که افزود آن شهنشه بختشان
&&
می‌نتاند که کشیدن رختشان
\\
\end{longtable}
\end{center}
