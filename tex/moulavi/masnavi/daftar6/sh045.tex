\begin{center}
\section*{بخش ۴۵ - قصهٔ سلطان محمود و غلام هندو}
\label{sec:sh045}
\addcontentsline{toc}{section}{\nameref{sec:sh045}}
\begin{longtable}{l p{0.5cm} r}
رحمة الله علیه گفته است
&&
ذکر شه محمود غازی سفته است
\\
کز غزای هند پیش آن همام
&&
در غنیمت اوفتادش یک غلام
\\
پس خلیفه‌ش کرد و بر تختش نشاند
&&
بر سپه بگزیدش و فرزند خواند
\\
طول و عرض و وصف قصه تو به تو
&&
در کلام آن بزرگ دین بجو
\\
حاصل آن کودک برین تخت نضار
&&
شسته پهلوی قباد شهریار
\\
گریه کردی اشک می‌راندی بسوز
&&
گفت شه او را کای پیروز روز
\\
از چه گریی دولتت شد ناگوار
&&
فوق املاکی قرین شهریار
\\
تو برین تخت و وزیران و سپاه
&&
پیش تختت صف زده چون نجم و ماه
\\
گفت کودک گریه‌ام زانست زار
&&
که مرا مادر در آن شهر و دیار
\\
از توم تهدید کردی هر زمان
&&
بینمت در دست محمود ارسلان
\\
پس پدر مر مادرم را در جواب
&&
جنگ کردی کین چه خشمست و عذاب
\\
می‌نیابی هیچ نفرینی دگر
&&
زین چنین نفرین مهلک سهلتر
\\
سخت بی‌رحمی و بس سنگین‌دلی
&&
که به صد شمشیر او را قاتلی
\\
من ز گفت هر دو حیران گشتمی
&&
در دل افتادی مرا بیم و غمی
\\
تا چه دوزخ‌خوست محمود ای عجب
&&
که مثل گشتست در ویل و کرب
\\
من همی‌لرزیدمی از بیم تو
&&
غافل از اکرام و از تعظیم تو
\\
مادرم کو تا ببیند این زمان
&&
مر مرا بر تخت ای شاه جهان
\\
فقر آن محمود تست ای بی‌سعت
&&
طبع ازو دایم همی ترساندت
\\
گر بدانی رحم این محمود راد
&&
خوش بگویی عاقبت محمود باد
\\
فقر آن محمود تست ای بیم‌دل
&&
کم شنو زین مادر طبع مضل
\\
چون شکار فقر کردی تو یقین
&&
هم‌چوکودک اشک باری یوم دین
\\
گرچه اندر پرورش تن مادرست
&&
لیک از صد دشمنت دشمن‌ترست
\\
تن چو شد بیمار داروجوت کرد
&&
ور قوی شد مر ترا طاغوت کرد
\\
چون زره دان این تن پر حیف را
&&
نی شتا را شاید و نه صیف را
\\
یار بد نیکوست بهر صبر را
&&
که گشاید صبر کردن صدر را
\\
صبر مه با شب منور داردش
&&
صبر گل با خار اذفر داردش
\\
صبر شیر اندر میان فرث و خون
&&
کرده او را ناعش ابن اللبون
\\
صبر جملهٔ انبیا با منکران
&&
کردشان خاص حق و صاحب‌قران
\\
هر که را بینی یکی جامه درست
&&
دانک او آن را به صبر و کسب جست
\\
هرکه را دیدی برهنه و بی‌نوا
&&
هست بر بی‌صبری او آن گوا
\\
هرکه مستوحش بود پر غصه جان
&&
کرده باشد با دغایی اقتران
\\
صبر اگر کردی و الف با وفا
&&
ار فراق او نخوردی این قفا
\\
خوی با حق نساختی چون انگبین
&&
با لبن که لا احب الافلین
\\
لاجرم تنها نماندی هم‌چنان
&&
که آتشی مانده به راه از کاروان
\\
چون ز بی‌صبری قرین غیر شد
&&
در فراقش پر غم و بی‌خیر شد
\\
صحبتت چون هست زر ده‌دهی
&&
پیش خاین چون امانت می‌نهی
\\
خوی با او کن که امانتهای تو
&&
آمن آید از افول و از عتو
\\
خوی با او کن که خو را آفرید
&&
خویهای انبیا را پرورید
\\
بره‌ای بدهی رمه بازت دهد
&&
پرورندهٔ هر صفت خود رب بود
\\
بره پیش گرگ امانت می‌نهی
&&
گرگ و یوسف را مفرما همرهی
\\
گرگ اگر با تو نماید روبهی
&&
هین مکن باور که ناید زو بهی
\\
جاهل ار با تو نماید هم‌دلی
&&
عاقبت زحمت زند از جاهلی
\\
او دو آلت دارد و خنثی بود
&&
فعل هر دو بی‌گمان پیدا شود
\\
او ذکر را از زنان پنهان کند
&&
تا که خود را خواهر ایشان کند
\\
شله از مردان به کف پنهان کند
&&
تا که خود را جنس آن مردان کند
\\
گفت یزدان زان کس مکتوم او
&&
شله‌ای سازیم بر خرطوم او
\\
تا که بینایان ما زان ذو دلال
&&
در نیایند از فن او در جوال
\\
حاصل آنک از هر ذکر ناید نری
&&
هین ز جاهل ترس اگر دانش‌وری
\\
دوستی جاهل شیرین‌سخن
&&
کم شنو کان هست چون سم کهن
\\
جان مادر چشم روشن گویدت
&&
جز غم و حسرت از آن نفزویدت
\\
مر پدر را گوید آن مادر جهار
&&
که ز مکتب بچه‌ام شد بس نزار
\\
از زن دیگر گرش آوردیی
&&
بر وی این جور و جفا کم کردیی
\\
از جز تو گر بدی این بچه‌ام
&&
این فشار آن زن بگفتی نیز هم
\\
هین بجه زن مادر و تیبای او
&&
سیلی بابا به از حلوای او
\\
هست مادر نفس و بابا عقل راد
&&
اولش تنگی و آخر صد گشاد
\\
ای دهندهٔ عقلها فریاد رس
&&
تا نخواهی تو نخواهد هیچ کس
\\
هم طلب از تست و هم آن نیکوی
&&
ما کییم اول توی آخر توی
\\
هم بگو تو هم تو بشنو هم تو باش
&&
ما همه لاشیم با چندین تراش
\\
زین حواله رغبت افزا در سجود
&&
کاهلی جبر مفرست و خمود
\\
جبر باشد پر و بال کاملان
&&
جبر هم زندان و بند کاهلان
\\
هم‌چو آب نیل دان این جبر را
&&
آب مؤمن را و خون مر گبر را
\\
بال بازان را سوی سلطان برد
&&
بال زاغان را به گورستان برد
\\
باز گرد اکنون تو در شرح عدم
&&
که چو پازهرست و پنداریش سم
\\
هم‌چو هندوبچه هین ای خواجه‌تاش
&&
رو ز محمود عدم ترسان مباش
\\
از وجودی ترس که اکنون در ویی
&&
آن خیالت لاشی و تو لا شیی
\\
لاشیی بر لاشیی عاشق شدست
&&
هیچ نی مر هیچ نی را ره زدست
\\
چون برون شد این خیالات از میان
&&
گشت نامعقول تو بر تو عیان
\\
\end{longtable}
\end{center}
