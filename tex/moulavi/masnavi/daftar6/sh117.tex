\begin{center}
\section*{بخش ۱۱۷ - بیان مجاهد کی دست از مجاهده باز ندارد اگر چه داند بسطت عطاء حق را کی آن مقصود از طرف دیگر و به سبب نوع عمل دیگر بدو رساند کی در وهم او نبوده باشد  او همه وهم و اومید درین طریق معین بسته باشد حلقهٔ همین در می‌زند بوک حق تعالی آن روزی را از در دیگر بدو رساند کی او آن تدبیر نکرده باشد و یرزقه من حیث لا یحتسب العبد یدبر والله یقدر و بود کی بنده را وهم بندگی بود کی مرا از غیر این در برساند اگر چه من حلقهٔ این در می‌زنم حق تعالی او را هم ازین در روزی رساند فی‌الجمله این  همه درهای یکی سرایست مع تقریره}
\label{sec:sh117}
\addcontentsline{toc}{section}{\nameref{sec:sh117}}
\begin{longtable}{l p{0.5cm} r}
یا درین ره آیدم آن کام من
&&
یا چو باز آیم ز ره سوی وطن
\\
بوک موقوفست کامم بر سفر
&&
چون سفر کردم بیابم در حضر
\\
یار را چندین بجویم جد و چست
&&
که بدانم که نمی‌بایست جست
\\
آن معیت کی رود در گوش من
&&
تا نگردم گرد دوران زمن
\\
کی کنم من از معیت فهم راز
&&
جز که از بعد سفرهای دراز
\\
حق معیت گفت و دل را مهر کرد
&&
تا که عکس آید به گوش دل نه طرد
\\
چون سفرها کرد و داد راه داد
&&
بعد از آن مهر از دل او بر گشاد
\\
چون خطایین آن حساب با صفا
&&
گرددش روشن ز بعد دو خطا
\\
بعد از آن گوید اگر دانستمی
&&
این معیت را کی او را جستمی
\\
دانش آن بود موقوف سفر
&&
ناید آن دانش به تیزی فکر
\\
آنچنان که وجه وام شیخ بود
&&
بسته و موقوف گریهٔ آن وجود
\\
کودک حلواییی بگریست زار
&&
توخته شد وام آن شیخ کبار
\\
گفته شد آن داستان معنوی
&&
پیش ازین اندر خلال مثنوی
\\
در دلت خوف افکند از موضعی
&&
تا نباشد غیر آنت مطمعی
\\
در طمع فایدهٔ دیگر نهد
&&
وآن مرادت از کسی دیگر دهد
\\
ای طمع در بسته در یک جای سخت
&&
که آیدم میوه از آن عالی‌درخت
\\
آن طمع زان جا نخواهد شد وفا
&&
بل ز جای دیگر آید آن عطا
\\
آن طمع را پس چرا در تو نهاد
&&
چون نخواستت زان طرف آن چیز داد
\\
از برای حکمتی و صنعتی
&&
نیز تا باشد دلت در حیرتی
\\
تا دلت حیران بود ای مستفید
&&
که مرادم از کجا خواهد رسد
\\
تا بدانی عجز خویش و جهل خویش
&&
تا شود ایقان تو در غیب بیش
\\
هم دلت حیران بود در منتجع
&&
که چه رویاند مصرف زین طمع
\\
طمع داری روزیی در درزیی
&&
تا ز خیاطی بی زر تا زیی
\\
رزق تو در زرگری آرد پدید
&&
که ز وهمت بود آن مکسب بعید
\\
پس طمع در درزیی بهر چه بود
&&
چون نخواست آن رزق زان جانب گشود
\\
بهر نادر حکمتی در علم حق
&&
که نبشت آن حکم را در ما سبق
\\
نیز تا حیران بود اندیشه‌ات
&&
تا که حیرانی بود کل پیشه‌ات
\\
یا وصال یار زین سعیم رسد
&&
یا ز راهی خارج از سعی جسد
\\
من نگویم زین طریق آید مراد
&&
می‌طپم تا از کجا خواهد گشاد
\\
سربریده مرغ هر سو می‌فتد
&&
تا کدامین سو رهد جان از جسد
\\
یا مراد من برآید زین خروج
&&
یا ز برجی دیگر از ذات البروج
\\
\end{longtable}
\end{center}
