\begin{center}
\section*{بخش ۱۳۱ - باز آمدن به شرح قصهٔ شاه‌زاده و ملازمت او در حضرت شاه}
\label{sec:sh131}
\addcontentsline{toc}{section}{\nameref{sec:sh131}}
\begin{longtable}{l p{0.5cm} r}
شاه‌زاده پیش شه حیران این
&&
هفت گردون دیده در یک مشت طین
\\
هیچ ممکن نه ببحثی لب گشود
&&
لیک جان با جان دمی خامش نبود
\\
آمده در خاطرش کین بس خفیست
&&
این همه معنیست پس صورت ز چیست
\\
صورتی از صورتت بیزار کن
&&
خفته‌ای هر خفته را بیدار کن
\\
آن کلامت می‌رهاند از کلام
&&
وان سقامت می‌جهاند از سقام
\\
پس سقام عشق جان صحتست
&&
رنجهااش حسرت هر راحتست
\\
ای تن اکنون دست خود زین جان بشو
&&
ور نمی‌شویی جز این جانی بجو
\\
حاصل آن شه نیک او را می‌نواخت
&&
او از آن خورشید چون مه می‌گداخت
\\
آن گداز عاشقان باشد نمو
&&
هم‌چو مه اندر گدازش تازه‌رو
\\
جمله رنجوران دوا دارند امید
&&
نالد این رنجور کم افزون کنید
\\
خوش‌تر از این سم ندیدم شربتی
&&
زین مرض خوش‌تر نباشد صحتی
\\
زین گنه بهتر نباشد طاعتی
&&
سالها نسبت بدین دم ساعتی
\\
مدتی بد پیش این شه زین نسق
&&
دل کباب و جان نهاده بر طبق
\\
گفت شه از هر کسی یک سر برید
&&
من ز شه هر لحظه قربانم جدید
\\
من فقیرم از زر از سر محتشم
&&
صد هزاران سر خلف دارد سرم
\\
با دو پا در عشق نتوان تاختن
&&
با یکی سر عشق نتوان باختن
\\
هر کسی را خود دو پا و یک‌سرست
&&
با هزاران پا و سر تن نادرست
\\
زین سبب هنگامه‌ها شد کل هدر
&&
هست این هنگامه هر دم گرم‌تر
\\
معدن گرمیست اندر لامکان
&&
هفت دوزخ از شرارش یک دخان
\\
\end{longtable}
\end{center}
