\begin{center}
\section*{بخش ۱۰۴ - بیان استمداد عارف از سرچشمهٔ حیات ابدی و مستغنی شدن او از استمداد و اجتذاب از چشمه‌های آبهای بی‌وفا کی علامة ذالک التجافی عن دار الغرور کی آدمی چون بر مددهای آن چشمه‌ها اعتماد کند در طلب چشمهٔ باقی دایم سست شود کاری ز درون جان تو می‌باید کز عاریه‌ها ترا دری نگشاید یک چشمهٔ آب از درون خانه به زان جویی که آن ز بیرون آید}
\label{sec:sh104}
\addcontentsline{toc}{section}{\nameref{sec:sh104}}
\begin{longtable}{l p{0.5cm} r}
حبذا کاریز اصل چیزها
&&
فارغت آرد ازین کاریزها
\\
تو ز صد ینبوع شربت می‌کشی
&&
هرچه زان صد کم شود کاهد خوشی
\\
چون بجوشید از درون چشمهٔ سنی
&&
ز استراق چشمه‌ها گردی غنی
\\
قرةالعینت چو ز آب و گل بود
&&
راتبهٔ این قره درد دل بود
\\
قلعه را چون آب آید از برون
&&
در زمان امن باشد بر فزون
\\
چونک دشمن گرد آن حلقه کند
&&
تا که اندر خونشان غرقه کند
\\
آب بیرون را ببرند آن سپاه
&&
تا نباشد قلعه را زانها پناه
\\
آن زمان یک چاه شوری از درون
&&
به ز صد جیحون شیرین از برون
\\
قاطع الاسباب و لشکرهای مرگ
&&
هم‌چو دی آید به قطع شاخ و برگ
\\
در جهان نبود مددشان از بهار
&&
جز مگر در جان بهار روی یار
\\
زان لقب شد خاک را دار الغرور
&&
کو کشد پا را سپس یوم العبور
\\
پیش از آن بر راست و بر چپ می‌دوید
&&
که بچینم درد تو چیزی نچید
\\
او بگفتی مر ترا وقت غمان
&&
دور از تو رنج و ده که در میان
\\
چون سپاه رنج آمد بست دم
&&
خود نمی‌گوید ترا من دیده‌ام
\\
حق پی شیطان بدین سان زد مثل
&&
که ترا در رزم آرد با حیل
\\
که ترا یاری دهم من با توم
&&
در خطرها پیش تو من می‌دوم
\\
اسپرت باشم گه تیر خدنگ
&&
مخلص تو باشم اندر وقت تنگ
\\
جان فدای تو کنم در انتعاش
&&
رستمی شیری هلا مردانه باش
\\
سوی کفرش آورد زین عشوه‌ها
&&
آن جوال خدعه و مکر و دها
\\
چون قدم بنهاد در خندق فتاد
&&
او به قاهاقاه خنده لب گشاد
\\
هی بیا من طمعها دارم ز تو
&&
گویدش رو رو که بیزارم ز تو
\\
تو نترسیدی ز عدل کردگار
&&
من همی‌ترسم دو دست از من بدار
\\
گفت حق خود او جدا شد از بهی
&&
تو بدین تزویرها هم کی رهی
\\
فاعل و مفعول در روز شمار
&&
روسیاهند و حریف سنگسار
\\
ره‌زده و ره‌زن یقین در حکم و داد
&&
در چه بعدند و در بئس المهاد
\\
گول را و غول را کو را فریفت
&&
از خلاص و فوز می‌باید شکیفت
\\
هم خر و خرگیر اینجا در گلند
&&
غافلند این‌جا و آن‌جا آفلند
\\
جز کسانی را که وا گردند از آن
&&
در بهار فضل آیند از خزان
\\
توبه آرند و خدا توبه‌پذیر
&&
امر او گیرند و او نعم الامیر
\\
چون بر آرند از پشیمانی حنین
&&
عرش لرزد از انین المذنبین
\\
آن‌چنان لرزد که مادر بر ولد
&&
دستشان گیرد به بالا می‌کشد
\\
کای خداتان وا خریده از غرور
&&
نک ریاض فضل و نک رب غفور
\\
بعد ازینتان برگ و رزق جاودان
&&
از هوای حق بود نه از ناودان
\\
چونک دریا بر وسایط رشک کرد
&&
تشنه چون ماهی به ترک مشک کرد
\\
\end{longtable}
\end{center}
