\begin{center}
\section*{بخش ۱۰۲ - گفتن خواجه در خواب به آن پای‌مرد وجوه وام آن دوست را کی آمده بود و نشان دادن جای دفن آن سیم و پیغام کردن به وارثان کی البته آن را بسیار نبینند وهیچ باز نگیرند و اگر چه او هیچ از آن قبول  نکند یا بعضی را قبول نکند هم آنجا بگذارند تا هر آنک خواهد برگیرد کی من با خدا نذرها کردم کی از آن سیم به من و به متعلقان من حبه‌ای باز نگردد الی آخره}
\label{sec:sh102}
\addcontentsline{toc}{section}{\nameref{sec:sh102}}
\begin{longtable}{l p{0.5cm} r}
بشنو اکنون داد مهمان جدید
&&
من همی دیدم که او خواهد رسید
\\
من شنوده بودم از وامش خبر
&&
بسته بهر او دو سه پاره گهر
\\
که وفای وام او هستند و بیش
&&
تا که ضیفم را نگردد سینه ریش
\\
وام دارد از ذهب او نه هزار
&&
وام را از بعض این گو بر گزار
\\
فضله ماند زین بسی گو خرج کن
&&
در دعایی گو مرا هم درج کن
\\
خواستم تا آن به دست خود دهم
&&
در فلان دفتر نوشتست این قسم
\\
خود اجل مهلت ندادم تا که من
&&
خفیه بسپارم بدو در عدن
\\
لعل و یاقوتست بهر وام او
&&
در خنوری و نبشته نام او
\\
در فلان طاقیش مدفون کرده‌ام
&&
من غم آن یار پیشین خورده‌ام
\\
قیمت آن را نداند جز ملوک
&&
فاجتهد بالبیع ان لا یخدعوک
\\
در بیوع آن کن تو از خوف غرار
&&
که رسول آموخت سه روز اختیار
\\
از کساد آن مترس و در میفت
&&
که رواج آن نخواهد هیچ خفت
\\
وارثانم را سلام من بگو
&&
وین وصیت را بگو هم مو به مو
\\
تا ز بسیاری آن زر نشکهند
&&
بی‌گرانی پیش آن مهمان نهند
\\
ور بگوید او نخواهم این فره
&&
گو بگیر و هر که را خواهی بده
\\
زانچ دادم باز نستانم نقیر
&&
سوی پستان باز ناید هیچ شیر
\\
گشته باشد هم‌چو سگ قی را اکول
&&
مسترد نحله بر قول رسول
\\
ور ببندد در نباید آن زرش
&&
تا بریزند آن عطا را بر درش
\\
هر که آنجا بگذرد زر می‌برد
&&
نیست هدیهٔ مخلصان را مسترد
\\
بهر او بنهاده‌ام آن از دو سال
&&
کرده‌ام من نذرها با ذوالجلال
\\
ور روا دارند چیزی زان ستد
&&
بیست چندان خو زیانشان اوفتد
\\
گر روانم را پژولانند زود
&&
صد در محنت بریشان بر گشود
\\
از خدا اومید دارم من لبق
&&
که رساند حق را در مستحق
\\
دو قضیهٔ دیگر او را شرح داد
&&
لب به ذکر آن نخواهم بر گشاد
\\
تا بماند دو قضیه سر و راز
&&
هم نگردد مثنوی چندین دراز
\\
برجهید از خواب انگشتک‌زنان
&&
گه غزل‌گویان و گه نوحه‌کنان
\\
گفت مهمان در چه سوداهاستی
&&
پای‌مردا مست و خوش بر خاستی
\\
تا چه دیدی خواب دوش ای بوالعلا
&&
که نمی‌گنجی تو در شهر و فلا
\\
خواب دیده پیل تو هندوستان
&&
که رمیدستی ز حلقهٔ دوستان
\\
گفت سوداناک خوابی دیده‌ام
&&
در دل خود آفتابی دیده‌ام
\\
خواب دیدم خواجهٔ بیدار را
&&
آن سپرده جان پی دیدار را
\\
خواب دیدم خواجهٔ معطی المنی
&&
واحد کالالف ان امر عنی
\\
مست و بی‌خود این چنین بر می‌شمرد
&&
تا که مستی عقل و هوشش را ببرد
\\
در میان خانه افتاد او دراز
&&
خلق انبه گرد او آمد فراز
\\
با خود آمد گفت ای بحر خوشی
&&
ای نهاده هوش‌ها در بیهشی
\\
خواب در بنهاده‌ای بیداریی
&&
بسته‌ای در بی‌دلی دلداریی
\\
توانگری پنهان کنی در ذل فقر
&&
طوق دولت بسته اندر غل فقر
\\
ضد اندر ضد پنهان مندرج
&&
آتش اندر آب سوزان مندرج
\\
روضه اندر آتش نمرود درج
&&
دخل‌ها رویان شده از بذل و خرج
\\
تا بگفته مصطفی شاه نجاح
&&
السماح یا اولی النعمی رباح
\\
ما نقص مال من الصدقات قط
&&
انما الخیرات نعم المرتبط
\\
جوشش و افزونی زر در زکات
&&
عصمت از فحشا و منکر در صلات
\\
آن زکاتت کیسه‌ات را پاسبان
&&
وآن صلاتت هم ز گرگانت شبان
\\
میوهٔ شیرین نهان در شاخ و برگ
&&
زندگی جاودان در زیر مرگ
\\
زبل گشته قوت خاک از شیوه‌ای
&&
زان غذا زاده زمین را میوه‌ای
\\
درعدم پنهان شده موجودیی
&&
در سرشت ساجدی مسجودیی
\\
آهن و سنگ از برونش مظلمی
&&
اندرون نوری و شمع عالمی
\\
درج در خوفی هزاران آمنی
&&
در سواد چشم چندان روشنی
\\
اندرون گاو تن شه‌زاده‌ای
&&
گنج در ویرانه‌ای بنهاده‌ای
\\
تا خری پیری گریزد زان نفیس
&&
گاو بیند شاه نی یعنی بلیس
\\
\end{longtable}
\end{center}
