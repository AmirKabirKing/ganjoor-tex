\begin{center}
\section*{بخش ۲۰ - امتحان کردن مصطفی علیه‌السلام عایشه را رضی الله عنها کی چه پنهان می‌شوی پنهان مشو که اعمی ترا نمی‌بیند تا پدید آید کی عایشه رضی الله عنها از ضمیر مصطفی علیه السلام واقف هست  یا خود مقلد گفت ظاهرست}
\label{sec:sh020}
\addcontentsline{toc}{section}{\nameref{sec:sh020}}
\begin{longtable}{l p{0.5cm} r}
گفت پیغامبر برای امتحان
&&
او نمی‌بیند ترا کم شو نهان
\\
کرد اشارت عایشه با دستها
&&
او نبیند من همی‌بینم ورا
\\
غیرت عقل است بر خوبی روح
&&
پر ز تشبیهات و تمثیل این نصوح
\\
با چنین پنهانیی کین روح راست
&&
عقل بر وی این چنین رشکین چراست
\\
از که پنهان می‌کنی ای رشک‌خو
&&
آنک پوشیدست نورش روی او
\\
می‌رود بی‌روی‌پوش این آفتاب
&&
فرط نور اوست رویش را نقاب
\\
از که پنهان می‌کنی ای رشک‌ور
&&
که آفتاب از وی نمی‌بیند اثر
\\
رشک از آن افزون‌ترست اندر تنم
&&
کز خودش خواهم که هم پنهان کنم
\\
ز آتش رشک گران آهنگ من
&&
با دو چشم و گوش خود در جنگ من
\\
چون چنین رشکیستت ای جان و دل
&&
پس دهان بر بند و گفتن را بهل
\\
ترسم ار خامش کنم آن آفتاب
&&
از سوی دیگر بدراند حجاب
\\
در خموشی گفت ما اظهر شود
&&
که ز منع آن میل افزون‌تر شود
\\
گر بغرد بحر غره‌ش کف شود
&&
جوش احببت بان اعرف شود
\\
حرف گفتن بستن آن روزنست
&&
عین اظهار سخن پوشیدنست
\\
بلبلانه نعره زن در روی گل
&&
تا کنی مشغولشان از بوی گل
\\
تا به قل مغشول گردد گوششان
&&
سوی روی گل نپرد هوششان
\\
پیش این خورشید کو بس روشنیست
&&
در حقیقت هر دلیلی ره‌زنیست
\\
\end{longtable}
\end{center}
