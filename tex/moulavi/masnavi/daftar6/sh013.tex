\begin{center}
\section*{بخش ۱۳ - حکایت آن شخص کی دزدان قوج او را بدزدیدند  و بر آن قناعت نکرد به حیله جامه‌هاش را هم دزدیدند}
\label{sec:sh013}
\addcontentsline{toc}{section}{\nameref{sec:sh013}}
\begin{longtable}{l p{0.5cm} r}
آن یکی قج داشت از پس می‌کشید
&&
دزد قج را برد حبلش را برید
\\
چونک آگه شد دوان شد چپ و راست
&&
تا بیابد کان قج برده کجاست
\\
بر سر چاهی بدید آن دزد را
&&
که فغان می‌کرد کای واویلتا
\\
گفت نالان از چئی ای اوستاد
&&
گفت همیان زرم در چه فتاد
\\
گر توانی در روی بیرون کشی
&&
خمس بدهم مر ترا با دلخوشی
\\
خمس صد دینار بستانی به دست
&&
گفت او خود این بهای ده قجست
\\
گر دری بر بسته شد ده در گشاد
&&
گر قجی شد حق عوض اشتر بداد
\\
جامه‌ها بر کند و اندر چاه رفت
&&
جامه‌ها را برد هم آن دزد تفت
\\
حازمی باید که ره تا ده برد
&&
حزم نبود طمع طاعون آورد
\\
او یکی دزدست فتنه‌سیرتی
&&
چون خیال او را بهر دم صورتی
\\
کس نداند مکر او الا خدا
&&
در خدا بگریز و وا ره زان دغا
\\
\end{longtable}
\end{center}
