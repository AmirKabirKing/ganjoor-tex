\begin{center}
\section*{بخش ۱۴ - مناظرهٔ مرغ با صیاد در ترهب و در معنی ترهبی کی مصطفی علیه‌السلام نهی کرد از آن امت خود را کی لا رهبانیة فی الاسلام}
\label{sec:sh014}
\addcontentsline{toc}{section}{\nameref{sec:sh014}}
\begin{longtable}{l p{0.5cm} r}
مرغ گفتش خواجه در خلوت مه‌ایست
&&
دین احمد را ترهب نیک نیست
\\
از ترهب نهی کردست آن رسول
&&
بدعتی چون در گرفتی ای فضول
\\
جمعه شرطست و جماعت در نماز
&&
امر معروف و ز منکر احتراز
\\
رنج بدخویان کشیدن زیر صبر
&&
منفعت دادن به خلقان هم‌چو ابر
\\
خیر ناس آن ینفع الناس ای پدر
&&
گر نه سنگی چه حریفی با مدر
\\
در میان امت مرحوم باش
&&
سنت احمد مهل محکوم باشد
\\
گفت عقل هر که را نبود رسوخ
&&
پیش عاقل او چو سنگست و کلوخ
\\
چون حمارست آنک نانش امنیتست
&&
صحبت او عین رهبانیتست
\\
زانک غیر حق همه گردد رفات
&&
کل آت بعد حین فهو آت
\\
حکم او هم حکم قبلهٔ او بود
&&
مرده‌اش خوان چونک مرده‌جو بود
\\
هر که با این قوم باشد راهبست
&&
که کلوخ و سنگ او را صاحبست
\\
خود کلوخ و سنگ کس را ره نزد
&&
زین کلوخان صد هزار آفت رسد
\\
گفت مرغش پس جهاد آنگه بود
&&
کین چنین ره‌زن میان ره بود
\\
از برای حفظ و یاری و نبرد
&&
بر ره ناآمن آید شیرمرد
\\
عرق مردی آنگهی پیدا شود
&&
که مسافر همره اعدا شود
\\
چون نبی سیف بودست آن رسول
&&
امت او صفدرانند و فحول
\\
مصلحت در دین ما جنگ و شکوه
&&
مصلحت در دین عیسی غار و کوه
\\
گفت آری گر بود یاری و زور
&&
تا به قوت بر زند بر شر و شور
\\
چون نباشد قوتی پرهیز به
&&
در فرار لا یطاق آسان بجه
\\
گفت صدق دل بباید کار را
&&
ورنه یاران کم نیاید یار را
\\
یار شو تا یار بینی بی‌عدد
&&
زانک بی‌یاران بمانی بی‌مدد
\\
دیو گرگست و تو هم‌چون یوسفی
&&
دامن یعقوب مگذار ای صفی
\\
گرگ اغلب آنگهی گیرا بود
&&
کز رمه شیشک به خود تنها رود
\\
آنک سنت یا جماعت ترک کرد
&&
در چنین مسبع نه خون خویش خورد
\\
هست سنت ره جماعت چون رفیق
&&
بی‌ره و بی‌یار افتی در مضیق
\\
همرهی نه کو بود خصم خرد
&&
فرصتی جوید که جامهٔ تو برد
\\
می‌رود با تو که یابد عقبه‌ای
&&
که تواند کردت آنجا نهبه‌ای
\\
یا بود اشتردلی چون دید ترس
&&
گوید او بهر رجوع از راه درس
\\
یار را ترسان کند ز اشتردلی
&&
این چنین همره عدو دان نه ولی
\\
راه جان‌بازیست و در هر غیشه‌ای
&&
آفتی در دفع هر جان‌شیشه‌ای
\\
راه دین زان رو پر از شور و شرست
&&
که نه راه هر مخنث گوهرست
\\
در ره این ترس امتحانهای نفوس
&&
هم‌چو پرویزن به تمییز سبوس
\\
راه چه بود پر نشان پایها
&&
یار چه بود نردبان رایها
\\
گیرم آن گرگت نیابد ز احتیاط
&&
بی ز جمعیت نیابی آن نشاط
\\
آنک تنها در رهی او خوش رود
&&
با رفیقان سیر او صدتو شود
\\
با غلیظی خر ز یاران ای فقیر
&&
در نشاط آید شود قوت‌پذیر
\\
هر خری کز کاروان تنها رود
&&
بر وی آن راه از تعب صدتو شود
\\
چند سیخ و چند چوب افزون خورد
&&
تا که تنها آن بیابان را برد
\\
مر ترا می‌گوید آن خر خوش شنو
&&
گر نه‌ای خر هم‌چنین تنها مرو
\\
آنک تنها خوش رود اندر رصد
&&
با رفیقان بی‌گمان خوشتر رود
\\
هر نبیی اندرین راه درست
&&
معجزه بنمود و همراهان بجست
\\
گر نباشد یاری دیوارها
&&
کی برآید خانه و انبارها
\\
هر یکی دیوار اگر باشد جدا
&&
سقف چون باشد معلق در هوا
\\
گر نباشد یاری حبر و قلم
&&
کی فتد بر روی کاغذها رقم
\\
این حصیری که کسی می‌گسترد
&&
گر نپیوندد به هم بادش برد
\\
حق ز هر جنسی چو زوجین آفرید
&&
پس نتایج شد ز جمعیت پدید
\\
او بگفت و او بگفت از اهتزاز
&&
بحثشان شد اندرین معنی دراز
\\
مثنوی را چابک و دلخواه کن
&&
ماجرا را موجز و کوتاه کن
\\
بعد از آن گفتش که گندم آن کیست
&&
گفت امانت از یتیم بی وصیست
\\
مال ایتام است امانت پیش من
&&
زانک پندارند ما را مؤتمن
\\
گفت من مضطرم و مجروح‌حال
&&
هست مردار این زمان بر من حلال
\\
هین به دستوری ازین گندم خورم
&&
ای امین و پارسا و محترم
\\
گفت مفتی ضرورت هم توی
&&
بی‌ضرورت گر خوری مجرم شوی
\\
ور ضرورت هست هم پرهیز به
&&
ور خوری باری ضمان آن بده
\\
مرغ پس در خود فرو رفت آن زمان
&&
توسنش سر بستد از جذب عنان
\\
چون بخورد آن گندم اندر فخ بماند
&&
چند او یاسین و الانعام خواند
\\
بعد در ماندن چه افسوس و چه آه
&&
پیش از آن بایست این دود سیاه
\\
آن زمان که حرص جنبید و هوس
&&
آن زمان می‌گو کای فریادرس
\\
کان زمان پیش از خرابی بصره است
&&
بوک بصره وا رهد هم زان شکست
\\
ابک لی یا باکیی یا ثاکلی
&&
قبل هدم البصرة و الموصل
\\
نح علی قبل موتی واغتفر
&&
لا تنح لی بعد موتی واصطبر
\\
ابک لی قبل ثبوری فی‌النوی
&&
بعد طوفان النوی خل البکا
\\
آن زمان که دیو می‌شد راه‌زن
&&
آن زمان بایست یاسین خواندن
\\
پیش از آنک اشکسته گردد کاروان
&&
آن زمان چوبک بزن ای پاسبان
\\
\end{longtable}
\end{center}
