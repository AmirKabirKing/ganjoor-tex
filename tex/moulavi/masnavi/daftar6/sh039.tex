\begin{center}
\section*{بخش ۳۹ - داستان آن درویش کی آن گیلانی را دعا کرد  کی خدا ترا به سلامت به خان و مان باز رساناد}
\label{sec:sh039}
\addcontentsline{toc}{section}{\nameref{sec:sh039}}
\begin{longtable}{l p{0.5cm} r}
گفت یک روزی به خواجهٔ گیلیی
&&
نان پرستی نر گدا زنبیلیی
\\
چون ستد زو نان بگفت ای مستعان
&&
خوش به خان و مان خود بازش رسان
\\
گفت خان ار آنست که من دیده‌ام
&&
حق ترا آنجا رساند ای دژم
\\
هر محدث را خسان باذل کنند
&&
حرفش ار عالی بود نازل کنند
\\
زانک قدر مستمع آید نبا
&&
بر قد خواجه برد درزی قبا
\\
\end{longtable}
\end{center}
