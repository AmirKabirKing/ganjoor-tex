\begin{center}
\section*{بخش ۱۲۱ - رسیدن آن شخص به مصر و شب بیرون آمدن به کوی از بهر شبکوکی و گدایی و گرفتن عسس او را و مراد اوحاصل شدن از عسس بعد از خوردن زخم بسیار و عسی ان تکرهوا شیا و هو خیر لکم و قوله  تعالی سیجعل الله بعد عسر یسرا و قوله علیه‌السلام اشتدی ازمة  تنفرجی و جمیع القرآن و الکتب المنزلة فی تقریر هذا}
\label{sec:sh121}
\addcontentsline{toc}{section}{\nameref{sec:sh121}}
\begin{longtable}{l p{0.5cm} r}
ناگهانی خود عسس او را گرفت
&&
مشت و چوبش زد ز صفرا تا شکفت
\\
اتفاقا اندر آن شب‌های تار
&&
دیده بد مردم ز شب‌دزدان ضرار
\\
بود شب‌های مخوف و منتحس
&&
پس به جد می‌جست دزدان را عسس
\\
تا خلیفه گفت که ببرید دست
&&
هر که شب گردد وگر خویش منست
\\
بر عسس کرده ملک تهدید و بیم
&&
که چرا باشید بر دزدان رحیم
\\
عشوه‌شان را از چه رو باور کنید
&&
یا چرا زیشان قبول زر کنید
\\
رحم بر دزدان و هر منحوس‌دست
&&
بر ضعیفان ضربت و بی‌رحمیست
\\
هین ز رنج خاص مسکل ز انتقام
&&
رنج او کم بین ببین تو رنج عام
\\
اصبع ملدوغ بر در دفع شر
&&
در تعدی و هلاک تن نگر
\\
اتفاقا اندر آن ایام دزد
&&
گشته بود انبوه پخته و خام دزد
\\
در چنین وقتش بدید و سخت زد
&&
چوب‌ها و زخمهای بی‌عدد
\\
نعره و فریاد زان درویش خاست
&&
که مزن تا من بگویم حال راست
\\
گفت اینک دادمت مهلت بگو
&&
تا به شب چون آمدی بیرون به کو
\\
تو نه‌ای زینجا غریب و منکری
&&
راستی گو تا بچه مکر اندری
\\
اهل دیوان بر عسس طعنه زدند
&&
که چرا دزدان کنون انبه شدند
\\
انبهی از تست و از امثال تست
&&
وا نما یاران زشتت را نخست
\\
ورنه کین جمله را از تو کشم
&&
تا شود آمن زر هر محتشم
\\
گفت او از بعد سوگندان پر
&&
که نیم من خانه‌سوز و کیسه‌بر
\\
من نه مرد دزدی و بیدادیم
&&
من غریب مصرم و بغدادیم
\\
\end{longtable}
\end{center}
