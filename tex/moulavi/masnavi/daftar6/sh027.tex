\begin{center}
\section*{بخش ۲۷ - قصهٔ احد احد گفتن بلال در حر حجاز از محبت مصطفی علیه‌السلام در آن چاشتگاهها کی خواجه‌اش از تعصب جهودی به شاخ خارش می‌زد پیش آفتاب حجاز و از زخم خون از تن بلال برمی‌جوشید ازو احد احد می‌جست بی‌قصد او چنانک از دردمندان دیگر ناله جهد بی‌قصد زیرا از درد عشق ممتلی بود اهتمام دفع درد خار را مدخل نبود هم‌چون سحرهٔ فرعون و جرجیس و غیر هم لایعد و لا یحصی}
\label{sec:sh027}
\addcontentsline{toc}{section}{\nameref{sec:sh027}}
\begin{longtable}{l p{0.5cm} r}
تن فدای خار می‌کرد آن بلال
&&
خواجه‌اش می‌زد برای گوشمال
\\
که چرا تو یاد احمد می‌کنی
&&
بندهٔ بد منکر دین منی
\\
می‌زد اندر آفتابش او به خار
&&
او احد می‌گفت بهر افتخار
\\
تا که صدیق آن طرف بر می‌گذشت
&&
آن احد گفتن به گوش او برفت
\\
چشم او پر آب شد دل پر عنا
&&
زان احد می‌یافت بوی آشنا
\\
بعد از آن خلوت بدیدش پند داد
&&
کز جهودان خفیه می‌دار اعتقاد
\\
عالم السرست پنهان دار کام
&&
گفت کردم توبه پیشت ای همام
\\
روز دیگر از پگه صدیق تفت
&&
آن طرف از بهر کاری می‌برفت
\\
باز احد بشنید و ضرب زخم خار
&&
برفروزید از دلش سوز و شرار
\\
باز پندش داد باز او توبه کرد
&&
عشق آمد توبهٔ او را بخورد
\\
توبه کردن زین نمط بسیار شد
&&
عاقبت از توبه او بیزار شد
\\
فاش کرد اسپرد تن را در بلا
&&
کای محمد ای عدو توبه‌ها
\\
ای تن من وی رگ من پر ز تو
&&
توبه را گنجا کجا باشد درو
\\
توبه را زین پس ز دل بیرون کنم
&&
از حیات خلد توبه چون کنم
\\
عشق قهارست و من مقهور عشق
&&
چون شکر شیرین شدم از شور عشق
\\
برگ کاهم پیش تو ای تند باد
&&
من چه دانم که کجا خواهم فتاد
\\
گر هلالم گر بلالم می‌دوم
&&
مقتدی آفتابت می‌شوم
\\
ماه را با زفتی و زاری چه کار
&&
در پی خورشید پوید سایه‌وار
\\
با قضا هر کو قراری می‌دهد
&&
ریش‌خند سبلت خود می‌کند
\\
کاه‌برگی پیش باد آنگه قرار
&&
رستخیزی وانگهانی عزم‌کار
\\
گربه در انبانم اندر دست عشق
&&
یک‌دمی بالا و یک‌دم پست عشق
\\
او همی‌گرداندم بر گرد سر
&&
نه به زیر آرام دارم نه زبر
\\
عاشقان در سیل تند افتاده‌اند
&&
بر قضای عشق دل بنهاده‌اند
\\
هم‌چو سنگ آسیا اندر مدار
&&
روز و شب گردان و نالان بی‌قرار
\\
گردشش بر جوی جویان شاهدست
&&
تا نگوید کس که آن جو راکدست
\\
گر نمی‌بینی تو جو را در کمین
&&
گردش دولاب گردونی ببین
\\
چون قراری نیست گردون را ازو
&&
ای دل اختروار آرامی مجو
\\
گر زنی در شاخ دستی کی هلد
&&
هر کجا پیوند سازی بسکلد
\\
گر نمی‌بینی تو تدویر قدر
&&
در عناصر جوشش و گردش نگر
\\
زانک گردشهای آن خاشاک و کف
&&
باشد از غلیان بحر با شرف
\\
باد سرگردان ببین اندر خروش
&&
پیش امرش موج دریا بین بجوش
\\
آفتاب و ماه دو گاو خراس
&&
گرد می‌گردند و می‌دارند پاس
\\
اختران هم خانه خانه می‌دوند
&&
مرکب هر سعد و نحسی می‌شوند
\\
اختران چرخ گر دورند هی
&&
وین حواست کاهل‌اند و سست‌پی
\\
اختران چشم و گوش و هوش ما
&&
شب کجااند و به بیداری کجا
\\
گاه در سعد و وصال و دلخوشی
&&
گاه در نحس فراق و بیهشی
\\
ماه گردون چون درین گردیدنست
&&
گاه تاریک و زمانی روشنست
\\
گه بهار و صیف هم‌چون شهد و شیر
&&
گه سیاستگاه برف و زمهریر
\\
چونک کلیات پیش او چو گوست
&&
سخره و سجده کن چوگان اوست
\\
تو که یک جزوی دلا زین صدهزار
&&
چون نباشی پیش حکمش بی‌قرار
\\
چون ستوری باش در حکم امیر
&&
گه در آخر حبس گاهی در مسیر
\\
چونک بر میخت ببندد بسته باش
&&
چونک بگشاید برو بر جسته باش
\\
آفتاب اندر فلک کژ می‌جهد
&&
در سیه‌روزی خسوفش می‌دهد
\\
کز ذنب پرهیز کن هین هوش‌دار
&&
تا نگردی تو سیه‌رو دیگ‌وار
\\
ابر را هم تازیانهٔ آتشین
&&
می‌زنندش کانچنان رو نه چنین
\\
بر فلان وادی ببار این سو مبار
&&
گوشمالش می‌دهد که گوش دار
\\
عقل تو از آفتابی بیش نیست
&&
اندر آن فکری که نهی آمد مه‌ایست
\\
کژ منه ای عقل تو هم گام خویش
&&
تا نیاید آن خسوف رو به پیش
\\
چون گنه کمتر بود نیم آفتاب
&&
منکسف بینی و نیمی نورتاب
\\
که به قدر جرم می‌گیرم ترا
&&
این بود تقریر در داد و جزا
\\
خواه نیک و خواه بد فاش و ستیر
&&
بر همه اشیا سمیعیم و بصیر
\\
زین گذر کن ای پدر نوروز شد
&&
خلق از خلاق خوش پدفوز شد
\\
باز آمد آب جان در جوی ما
&&
باز آمد شاه ما در کوی ما
\\
می‌خرامد بخت و دامن می‌کشد
&&
نوبت توبه شکستن می‌زند
\\
توبه را بار دگر سیلاب برد
&&
فرصت آمد پاسبان را خواب برد
\\
هر خماری مست گشت و باده خورد
&&
رخت را امشب گرو خواهیم کرد
\\
زان شراب لعل جان جان‌فزا
&&
لعل اندر لعل اندر لعل ما
\\
باز خرم گشت مجلس دلفروز
&&
خیز دفع چشم بد اسپند سوز
\\
نعرهٔ مستان خوش می‌آیدم
&&
تا ابد جانا چنین می‌بایدم
\\
نک هلالی با بلالی یار شد
&&
زخم خار او را گل و گلزار شد
\\
گر ز زخم خار تن غربال شد
&&
جان و جسمم گلشن اقبال شد
\\
تن به پیش زخم خار آن جهود
&&
جان من مست و خراب آن و دود
\\
بوی جانی سوی جانم می‌رسد
&&
بوی یار مهربانم می‌رسد
\\
از سوی معراج آمد مصطفی
&&
بر بلالش حبذا لی حبذا
\\
چونک صدیق از بلال دم‌درست
&&
این شنید از توبهٔ او دست شست
\\
\end{longtable}
\end{center}
