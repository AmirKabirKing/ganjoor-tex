\begin{center}
\section*{بخش ۵۳ - جواب قاضی سال صوفی را و قصهٔ ترک و درزی را مثل آوردن}
\label{sec:sh053}
\addcontentsline{toc}{section}{\nameref{sec:sh053}}
\begin{longtable}{l p{0.5cm} r}
گفت قاضی بس تهی‌رو صوفیی
&&
خالی از فطنت چو کاف کوفیی
\\
تو بنشنیدی که آن پر قند لب
&&
غدر خیاطان همی‌گفتی به شب
\\
خلق را در دزدی آن طایفه
&&
می‌نمود افسانه‌های سالفه
\\
قصهٔ پاره‌ربایی در برین
&&
می حکایت کرد او با آن و این
\\
در سمر می‌خواند دزدی‌نامه‌ای
&&
گرد او جمع آمده هنگامه‌ای
\\
مستمع چون یافت جاذب زان وفود
&&
جمله اجزااش حکایت گشته بود
\\
\end{longtable}
\end{center}
