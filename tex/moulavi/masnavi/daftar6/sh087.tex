\begin{center}
\section*{بخش ۸۷ - مبالغه کردن موش در لابه و زاری و وصلت جستن از چغز آبی}
\label{sec:sh087}
\addcontentsline{toc}{section}{\nameref{sec:sh087}}
\begin{longtable}{l p{0.5cm} r}
گفت کای یار عزیز مهرکار
&&
من ندارم بی‌رخت یک‌دم قرار
\\
روز نور و مکسب و تابم توی
&&
شب قرار و سلوت و خوابم توی
\\
از مروت باشد ار شادم کنی
&&
وقت و بی‌وقت از کرم یادم کنی
\\
در شبان‌روزی وظیفهٔ چاشتگاه
&&
راتبه کردی وصال ای نیک‌خواه
\\
من بدین یک‌بار قانع نیستم
&&
در هوایت طرفه انسانیستم
\\
پانصد استسقاستم اندر جگر
&&
با هر استسقا قرین جوع البقر
\\
بی‌نیازی از غم من ای امیر
&&
ده زکات جاه و بنگر در فقیر
\\
این فقیر بی‌ادب نا درخورست
&&
لیک لطف عام تو زان برترست
\\
می‌نجوید لطف عام تو سند
&&
آفتابی بر حدثها می‌زند
\\
نور او را زان زیانی نابده
&&
وان حدث از خشکیی هیزم شده
\\
تا حدث در گلخنی شد نور یافت
&&
در در و دیوار حمامی بتافت
\\
بود آلایش شد آرایش کنون
&&
چون برو بر خواند خورشید آن فسون
\\
شمس هم معدهٔ زمین را گرم کرد
&&
تا زمین باقی حدثها را بخورد
\\
جزو خاکی گشت و رست از وی نبات
&&
هکذا یمحو الاله السیئات
\\
با حدث که بترینست این کند
&&
کش نبات و نرگس و نسرین کند
\\
تا به نسرین مناسک در وفا
&&
حق چه بخشد در جزا و در عطا
\\
چون خبیثان را چنین خلعت دهد
&&
طیبین را تا چه بخشد در رصد
\\
آن دهد حقشان که لا عین رات
&&
که نگنجد در زبان و در لغت
\\
ما کییم این را بیا ای یار من
&&
روز من روشن کن از خلق حسن
\\
منگر اندر زشتی و مکروهیم
&&
که ز پر زهری چو مار کوهیم
\\
ای که من زشت و خصالم جمله زشت
&&
چون شوم گل چون مرا او خار کشت
\\
نوبهار حسن گل ده خار را
&&
زینت طاووس ده این مار را
\\
در کمال زشتیم من منتهی
&&
لطف تو در فضل و در فن منتهی
\\
حاجت این منتهی زان منتهی
&&
تو بر آر ای حسرت سرو سهی
\\
چون بمیرم فضل تو خواهد گریست
&&
از کرم گرچه ز حاجت او بریست
\\
بر سر گورم بسی خواهد نشست
&&
خواهد از چشم لطیفش اشک جست
\\
نوحه خواهد کرد بر محرومیم
&&
چشم خواهد بست از مظلومیم
\\
اندکی زان لطفها اکنون بکن
&&
حلقه‌ای در گوش من کن زان سخن
\\
آنک خواهی گفت تو با خاک من
&&
برفشان بر مدرک غمناک من
\\
\end{longtable}
\end{center}
