\begin{center}
\section*{بخش ۲ - سال سایل از مرغی کی بر سر ربض شهری نشسته باشد سر او فاضل‌ترست و عزیزتر و شریف‌تر و مکرم‌تر یا دم او و جواب دادن واعظ سایل را به قدر فهم او}
\label{sec:sh002}
\addcontentsline{toc}{section}{\nameref{sec:sh002}}
\begin{longtable}{l p{0.5cm} r}
واعظی را گفت روزی سایلی
&&
کای تو منبر را سنی‌تر قایلی
\\
یک سؤالستم بگو ای ذو لباب
&&
اندرین مجلس سؤالم را جواب
\\
بر سر بارو یکی مرغی نشست
&&
از سر و از دم کدامینش بهست
\\
گفت اگر رویش به شهر و دم به ده
&&
روی او از دم او می‌دان که به
\\
ور سوی شهرست دم رویش به ده
&&
خاک آن دم باش و از رویش بجه
\\
مرغ با پر می‌پرد تا آشیان
&&
پر مردم همتست ای مردمان
\\
عاشقی که آلوده شد در خیر و شر
&&
خیر و شر منگر تو در همت نگر
\\
باز اگر باشد سپید و بی‌نظیر
&&
چونک صیدش موش باشد شد حقیر
\\
ور بود چغدی و میل او به شاه
&&
او سر بازست منگر در کلاه
\\
آدمی بر قد یک طشت خمیر
&&
بر فزود از آسمان و از اثیر
\\
هیچ کرمنا شنید این آسمان
&&
که شنید این آدمی پر غمان
\\
بر زمین و چرخ عرضه کرد کس
&&
خوبی و عقل و عبارات و هوس
\\
جلوه کردی هیچ تو بر آسمان
&&
خوبی روی و اصابت در گمان
\\
پیش صورتهای حمام ای ولد
&&
عرضه کردی هیچ سیم‌اندام خود
\\
بگذری زان نقشهای هم‌چو حور
&&
جلوه آری با عجوز نیم‌کور
\\
در عجوزه چیست که ایشان را نبود
&&
که ترا زان نقشها با خود ربود
\\
تو نگویی من بگویم در بیان
&&
عقل و حس و درک و تدبیرست و جان
\\
در عجوزه جان آمیزش‌کنیست
&&
صورت گرمابه‌ها را روح نیست
\\
صورت گرمابه گر جنبش کند
&&
در زمان او از عجوزه بر کند
\\
جان چه باشد با خبر از خیر و شر
&&
شاد با احسان و گریان از ضرر
\\
چون سر و ماهیت جان مخبرست
&&
هر که او آگاه‌تر با جان‌ترست
\\
روح را تاثیر آگاهی بود
&&
هر که را این بیش اللهی بود
\\
چون خبرها هست بیرون زین نهاد
&&
باشد این جانها در آن میدان جماد
\\
جان اول مظهر درگاه شد
&&
جان جان خود مظهر الله شد
\\
آن ملایک جمله عقل و جان بدند
&&
جان نو آمد که جسم آن بدند
\\
از سعادت چون بر آن جان بر زدند
&&
هم‌چو تن آن روح را خادم شدند
\\
آن بلیس از جان از آن سر برده بود
&&
یک نشد با جان که عضو مرده بود
\\
چون نبودش آن فدای آن نشد
&&
دست بشکسته مطیع جان نشد
\\
جان نشد ناقص گر آن عضوش شکست
&&
کان بدست اوست تواند کرد هست
\\
سر دیگر هست کو گوش دگر
&&
طوطیی کو مستعد آن شکر
\\
طوطیان خاص را قندیست ژرف
&&
طوطیان عام از آن خور بسته طرف
\\
کی چشد درویش صورت زان زکات
&&
معنیست آن نه فعولن فاعلات
\\
از خر عیسی دریغش نیست قند
&&
لیک خر آمد به خلقت که پسند
\\
قند خر را گر طرب انگیختی
&&
پیش خر قنطار شکر ریختی
\\
معنی نختم علی افواههم
&&
این شناس اینست ره‌رو را مهم
\\
تا ز راه خاتم پیغامبران
&&
بوک بر خیزد ز لب ختم گران
\\
ختمهایی که انبیا بگذاشتند
&&
آن بدین احمدی برداشتند
\\
قفلهای ناگشاده مانده بود
&&
از کف انا فتحنا برگشود
\\
او شفیع است این جهان و آن جهان
&&
این جهان زی دین و آنجا زی جنان
\\
این جهان گوید که تو رهشان نما
&&
وآن جهان گوید که تو مهشان نما
\\
پیشه‌اش اندر ظهور و در کمون
&&
اهد قومی انهم لا یعلمون
\\
باز گشته از دم او هر دو باب
&&
در دو عالم دعوت او مستجاب
\\
بهر این خاتم شدست او که به جود
&&
مثل او نه بود و نه خواهند بود
\\
چونک در صنعت برد استاد دست
&&
نه تو گویی ختم صنعت بر توست
\\
در گشاد ختمها تو خاتمی
&&
در جهان روح‌بخشان حاتمی
\\
هست اشارات محمدالمراد
&&
کل گشاد اندر گشاد اندر گشاد
\\
صد هزاران آفرین بر جان او
&&
بر قدوم و دور فرزندان او
\\
آن خلیفه‌زادگان مقبلش
&&
زاده‌اند از عنصر جان و دلش
\\
گر ز بغداد و هری یا از ری‌اند
&&
بی‌مزاج آب و گل نسل وی‌اند
\\
شاخ گل هر جا که روید هم گلست
&&
خم مل هر جا که جوشد هم ملست
\\
گر ز مغرب بر زند خورشید سر
&&
عین خورشیدست نه چیز دگر
\\
عیب چینان را ازین دم کور دار
&&
هم بستاری خود ای کردگار
\\
گفت حق چشم خفاش بدخصال
&&
بسته‌ام من ز آفتاب بی‌مثال
\\
از نظرهای خفاش کم و کاست
&&
انجم آن شمس نیز اندر خفاست
\\
\end{longtable}
\end{center}
