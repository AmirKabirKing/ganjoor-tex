\begin{center}
\section*{بخش ۴۸ - طیره شدن قاضی از سیلی درویش و سرزنش کردن صوفی قاضی را}
\label{sec:sh048}
\addcontentsline{toc}{section}{\nameref{sec:sh048}}
\begin{longtable}{l p{0.5cm} r}
گشت قاضی طیره صوفی گفت هی
&&
حکم تو عدلست لاشپک نیست غی
\\
آنچ نپسندی به خود ای شیخ دین
&&
چون پسندی بر برادر ای امین
\\
این ندانی که می من چه کنی
&&
هم در آن چه عاقبت خود افکنی
\\
من حفر بئرا نخواندی از خبر
&&
آنچ خواندی کن عمل جان پدر
\\
این یکی حکمت چنین بد در قضا
&&
که ترا آورد سیلی بر قفا
\\
وای بر احکام دیگرهای تو
&&
تا چه آرد بر سر و بر پای تو
\\
ظالمی را رحم آری از کرم
&&
که برای نفقه بادت سه درم
\\
دست ظالم را ببر چه جای آن
&&
که بدست او نهی حکم و عنان
\\
تو بدان بز مانی ای مجهول‌داد
&&
که نژاد گرگ را او شیر داد
\\
\end{longtable}
\end{center}
