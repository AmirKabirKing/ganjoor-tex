\begin{center}
\section*{بخش ۱۲۵ - مکرر کردن برادران پند دادن بزرگین را و تاب ناآوردن او آن پند را و در رمیدن او ازیشان شیدا و بی‌خود رفتن و خود را در بارگاه پادشاه انداختن بی‌دستوری خواستن لیک از فرط عشق و محبت نه از گستاخی و لاابالی الی آخره}
\label{sec:sh125}
\addcontentsline{toc}{section}{\nameref{sec:sh125}}
\begin{longtable}{l p{0.5cm} r}
آن دو گفتندش که اندر جان ما
&&
هست پاسخ‌ها چو نجم اندر سما
\\
گر نگوییم آن نیاید راست نرد
&&
ور بگوییم آن دلت آید به درد
\\
هم‌چو چغزیم اندر آب از گفت الم
&&
وز خموشی اختناقست و سقم
\\
گر نگوییم آتشی را نور نیست
&&
ور بگوییم آن سخن دستور نیست
\\
در زمان برجست کای خویشان وداع
&&
انما الدنیا و ما فیها متاع
\\
پس برون جست او چو تیری از کمان
&&
که مجال گفت کم بود آن زمان
\\
اندر آمد مست پیش شاه چین
&&
زود مستانه ببوسید او زمین
\\
شاه را مکشوف یک یک حالشان
&&
اول و آخر غم و زلزالشان
\\
میش مشغولست در مرعای خویش
&&
لیک چوپان واقفست از حال میش
\\
کلکم راع بداند از رمه
&&
کی علف‌خوارست و کی در ملحمه
\\
گرچه در صورت از آن صف دور بود
&&
لیک چون دف در میان سور بود
\\
واقف از سوز و لهیب آن وفود
&&
مصلحت آن بد که خشک آورده بود
\\
در میان جانشان بود آن سمی
&&
لک قاصد کرده خود را اعجمی
\\
صورت آتش بود پایان دیگ
&&
معنی آتش بود در جان دیگ
\\
صورتش بیرون و معنیش اندرون
&&
معنی معشوق جان در رگ چو خون
\\
شاه‌زاده پیش شه زانو زده
&&
ده معرف شارح حالش شده
\\
گرچه شه عارف بد از کل پیش پیش
&&
لیک می‌کردی معرف کار خویش
\\
در درون یک ذره نور عارفی
&&
به بود از صد معرف ای صفی
\\
گوش را رهن معرف داشتن
&&
آیت محجوبیست و حزر و ظن
\\
آنک او را چشم دل شد دیدبان
&&
دید خواهد چشم او عین العیان
\\
با تواتر نیست قانع جان او
&&
بل ز چشم دل رسد ایقان او
\\
پس معرف پیش شاه منتجب
&&
در بیان حال او بگشود لب
\\
گفت شاها صید احسان توست
&&
پادشاهی کن که بی بیرون شوست
\\
دست در فتراک این دولت زدست
&&
بر سر سرمست او بر مال دست
\\
گفت شه هر منصبی و ملکتی
&&
که التماسش هست یابد این فتی
\\
بیست چندان ملک کو شد زان بری
&&
بخشمش اینجا و ما خود بر سری
\\
گفت تا شاهیت در وی عشق کاشت
&&
جز هوای تو هوایی کی گذاشت
\\
بندگی تش چنان درخورد شد
&&
که شهی اندر دل او سرد شد
\\
شاهی و شه‌زادگی در باختست
&&
از پی تو در غریبی ساختست
\\
صوفیست انداخت خرقه وجد در
&&
کی رود او بر سر خرقه دگر
\\
میل سوی خرقهٔ داده و ندم
&&
آنچنان باشد که من مغبون شدم
\\
باز ده آن خرقه این سو ای قرین
&&
که نمی‌ارزید آن یعنی بدین
\\
دور از عاشق که این فکر آیدش
&&
ور بیاید خاک بر سر بایدش
\\
عشق ارزد صد چو خرقه کالبد
&&
که حیاتی دارد و حس و خرد
\\
خاصه خرقهٔ ملک دنیا کابترست
&&
پنج دانگ مستیش درد سرست
\\
ملک دنیا تن‌پرستان را حلال
&&
ما غلام ملک عشق بی‌زوال
\\
عامل عشقست معزولش مکن
&&
جز به عشق خویش مشغولش مکن
\\
منصبی کانم ز رؤیت محجبست
&&
عین معزولیست و نامش منصبست
\\
موجب تاخیر اینجا آمدن
&&
فقد استعداد بود و ضعف فن
\\
بی ز استعداد در کانی روی
&&
بر یکی حبه نگردی محتوی
\\
هم‌چو عنینی که بکری را خرد
&&
گرچه سیمین‌بر بود کی بر خورد
\\
چون چراغی بی ز زیت و بی فتیل
&&
نه کثیرستش ز شمع و نه قلیل
\\
در گلستان اندر آید اخشمی
&&
کی شود مغزش ز ریحان خرمی
\\
هم‌چو خوبی دلبری مهمان غر
&&
بانگ چنگ و بربطی در پیش کر
\\
هم‌چو مرغ خاک که آید در بحار
&&
زان چه یابد جز هلاک و جز خسار
\\
هم‌چو بی‌گندم شده در آسیا
&&
جز سپیدی ریش و مو نبود عطا
\\
آسیای چرخ بر بی‌گندمان
&&
موسپیدی بخشد و ضعف میان
\\
لیک با باگندمان این آسیا
&&
ملک‌بخش آمد دهد کار و کیا
\\
اول استعداد جنت بایدت
&&
تا ز جنت زندگانی زایدت
\\
طفل نو را از شراب و از کباب
&&
چه حلاوت وز قصور و از قباب
\\
حد ندارد این مثل کم جو سخن
&&
تو برو تحصیل استعداد کن
\\
بهر استعداد تا اکنون نشست
&&
شوق از حد رفت و آن نامد به دست
\\
گفت استعداد هم از شه رسد
&&
بی ز جان کی مستعد گردد جسد
\\
لطف‌های شه غمش را در نوشت
&&
شد که صید شه کند او صید گشت
\\
هر که در اشکار چون تو صید شد
&&
صید را ناکرده قید او قید شد
\\
هرکه جویای امیری شد یقین
&&
پیش از آن او در اسیری شد رهین
\\
عکس می‌دان نقش دیباجهٔ جهان
&&
نام هر بندهٔ جهان خواجهٔ جهان
\\
ای تن کژ فکرت معکوس‌رو
&&
صد هزار آزاد را کرده گرو
\\
مدتی بگذار این حیلت پزی
&&
چند دم پیش از اجل آزاد زی
\\
ور در آزادیت چون خر راه نیست
&&
هم‌چو دلوت سیر جز در چاه نیست
\\
مدتی رو ترک جان من بگو
&&
رو حریف دیگری جز من بجو
\\
نوبت من شد مرا آزاد کن
&&
دیگری را غیر من داماد کن
\\
ای تن صدکاره ترک من بگو
&&
عمر من بردی کسی دیگر بجو
\\
\end{longtable}
\end{center}
