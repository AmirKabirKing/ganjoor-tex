\begin{center}
\section*{غزل ۴۷۳: دی به چمن برگذشت سرو سخنگوی من}
\label{sec:473}
\addcontentsline{toc}{section}{\nameref{sec:473}}
\begin{longtable}{l p{0.5cm} r}
دی به چمن برگذشت سرو سخنگوی من
&&
تا نکند گل غرور رنگ من و بوی من
\\
برگ گل لعل بود شاهد بزم بهار
&&
آب گلستان ببرد شاهد گلروی من
\\
شد سپر از دست عقل تا ز کمین عتاب
&&
تیغ جفا برکشید ترک زره موی من
\\
ساعد دل چون نداشت قوت بازوی صبر
&&
دست غمش درشکست پنجه نیروی من
\\
عشق به تاراج داد رخت صبوری دل
&&
می‌نکند بخت شور خیمه ز پهلوی من
\\
کرده‌ام از راه عشق چند گذر سوی او
&&
او به تفضل نکرد هیچ نگه سوی من
\\
جور کشم بنده وار ور کشدم حاکم است
&&
خیره کشی کار اوست بارکشی خوی من
\\
ای گل خوش بوی من یاد کنی بعد از این
&&
سعدی بیچاره بود بلبل خوشگوی من
\\
\end{longtable}
\end{center}
