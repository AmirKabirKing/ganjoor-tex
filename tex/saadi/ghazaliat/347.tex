\begin{center}
\section*{غزل ۳۴۷: جزای آن که نگفتیم شکر روز وصال}
\label{sec:347}
\addcontentsline{toc}{section}{\nameref{sec:347}}
\begin{longtable}{l p{0.5cm} r}
جزای آن که نگفتیم شکر روز وصال
&&
شب فراق نخفتیم لاجرم ز خیال
\\
بدار یک نفس ای قائد این زمام جمال
&&
که دیده سیر نمی‌گردد از نظر به جمال
\\
دگر به گوش فراموش عهد سنگین دل
&&
پیام ما که رساند مگر نسیم شمال
\\
به تیغ هندی دشمن قتال می‌نکند
&&
چنان که دوست به شمشیر غمزه قتال
\\
جماعتی که نظر را حرام می‌گویند
&&
نظر حرام بکردند و خون خلق حلال
\\
غزال اگر به کمند اوفتد عجب نبود
&&
عجب فتادن مرد است در کمند غزال
\\
تو بر کنار فراتی ندانی این معنی
&&
به راه بادیه دانند قدر آب زلال
\\
اگر مراد نصیحت کنان ما این است
&&
که ترک دوست بگویم تصوریست محال
\\
به خاک پای تو داند که تا سرم نرود
&&
ز سر به در نرود همچنان امید وصال
\\
حدیث عشق چه حاجت که بر زبان آری
&&
به آب دیده خونین نبشته صورت حال
\\
سخن دراز کشیدیم و همچنان باقیست
&&
که ذکر دوست نیارد به هیچ گونه ملال
\\
به ناله کار میسر نمی‌شود سعدی
&&
ولیک ناله بیچارگان خوش است بنال
\\
\end{longtable}
\end{center}
