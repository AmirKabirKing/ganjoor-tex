\begin{center}
\section*{غزل ۹۵: آن که دل من چو گوی در خم چوگان اوست}
\label{sec:095}
\addcontentsline{toc}{section}{\nameref{sec:095}}
\begin{longtable}{l p{0.5cm} r}
آن که دل من چو گوی در خم چوگان اوست
&&
موقف آزادگان بر سر میدان اوست
\\
ره به در از کوی دوست نیست که بیرون برند
&&
سلسله پای جمع زلف پریشان اوست
\\
چند نصیحت کنند بی‌خبرانم به صبر
&&
درد مرا ای حکیم صبر نه درمان اوست
\\
گر کند انعام او در من مسکین نگاه
&&
ور نکند حاکمست بنده به فرمان اوست
\\
گر بزند بی‌گناه عادت بخت منست
&&
ور بنوازد به لطف غایت احسان اوست
\\
میل ندارم به باغ انس نگیرم به سرو
&&
سروی اگر لایقست قد خرامان اوست
\\
چون بتواند نشست آن که دلش غایبست
&&
یا بتواند گریخت آن که به زندان اوست
\\
حیرت عشاق را عیب کند بی بصر
&&
بهره ندارد ز عیش هر که نه حیران اوست
\\
چون تو گلی کس ندید در چمن روزگار
&&
خاصه که مرغی چو من بلبل بستان اوست
\\
گر همه مرغی زنند سخت کمانان به تیر
&&
حیف بود بلبلی کاین همه دستان اوست
\\
سعدی اگر طالبی راه رو و رنج بر
&&
کعبه دیدار دوست صبر بیابان اوست
\\
\end{longtable}
\end{center}
