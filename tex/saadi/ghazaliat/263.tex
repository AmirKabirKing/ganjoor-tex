\begin{center}
\section*{غزل ۲۶۳: گفتمش سیر ببینم مگر از دل برود}
\label{sec:263}
\addcontentsline{toc}{section}{\nameref{sec:263}}
\begin{longtable}{l p{0.5cm} r}
گفتمش سیر ببینم مگر از دل برود
&&
وآن چنان پای گرفته‌ست که مشکل برود
\\
دلی از سنگ بباید به سر راه وداع
&&
تا تحمل کند آن روز که محمل برود
\\
چشم حسرت به سر اشک فرو می‌گیرم
&&
که اگر راه دهم قافله بر گل برود
\\
ره ندیدم چو برفت از نظرم صورت دوست
&&
همچو چشمی که چراغش ز مقابل برود
\\
موج از این بار چنان کشتی طاقت بشکست
&&
که عجب دارم اگر تخته به ساحل برود
\\
سهل بود آن که به شمشیر عتابم می‌کشت
&&
قتل صاحب نظر آن است که قاتل برود
\\
نه عجب گر برود قاعده صبر و شکیب
&&
پیش هر چشم که آن قد و شمایل برود
\\
کس ندانم که در این شهر گرفتار تو نیست
&&
مگر آن کس که به شهر آید و غافل برود
\\
گر همه عمر نداده‌ست کسی دل به خیال
&&
چون بیاید به سر راه تو بی‌دل برود
\\
روی بنمای که صبر از دل صوفی ببری
&&
پرده بردار که هوش از تن عاقل برود
\\
سعدی ار عشق نبازد چه کند ملک وجود
&&
حیف باشد که همه عمر به باطل برود
\\
قیمت وصل نداند مگر آزرده هجر
&&
مانده آسوده بخسبد چو به منزل برود
\\
\end{longtable}
\end{center}
