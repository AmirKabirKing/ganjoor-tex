\begin{center}
\section*{غزل ۶۲۵: شبست و شاهد و شمع و شراب و شیرینی}
\label{sec:625}
\addcontentsline{toc}{section}{\nameref{sec:625}}
\begin{longtable}{l p{0.5cm} r}
شب است و شاهد و شمع و شراب و شیرینی
&&
غنیمت است چنین شب که دوستان بینی
\\
به شرط آن که منت بنده وار در خدمت
&&
بایستم تو خداوندوار بنشینی
\\
میان ما و شما عهد در ازل رفته‌ست
&&
هزار سال برآید همان نخستینی
\\
چو صبرم از تو میسر نمی‌شود چه کنم
&&
به خشم رفتم و باز آمدم به مسکینی
\\
به حکم آن که مرا هیچ دوست چون تو به دست
&&
نیاید و تو به از من هزار بگزینی
\\
به رنگ و بوی بهار ای فقیر قانع باش
&&
چو باغبان نگذارد که سیب و گل چینی
\\
تفاوتی نکند گر ترش کنی ابرو
&&
هزار تلخ بگویی هنوز شیرینی
\\
لگام بر سر شیران کند صلابت عشق
&&
چنان کشد که شتر را مهار دربینی
\\
ز نیکبختی سعدیست پای بند غمت
&&
زهی کبوتر مقبل که صید شاهینی
\\
مرا شکیب نمی‌باشد ای مسلمانان
&&
ز روی خوب لکم دینکم ولی دینی
\\
\end{longtable}
\end{center}
