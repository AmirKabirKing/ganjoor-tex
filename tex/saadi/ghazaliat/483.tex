\begin{center}
\section*{غزل ۴۸۳: راستی گویم به سروی ماند این بالای تو}
\label{sec:483}
\addcontentsline{toc}{section}{\nameref{sec:483}}
\begin{longtable}{l p{0.5cm} r}
راستی گویم به سروی ماند این بالای تو
&&
در عبارت می‌نیاید چهره زیبای تو
\\
چون تو حاضر می‌شوی من غایب از خود می‌شوم
&&
بس که حیران می‌بماندم وهم در سیمای تو
\\
کاشکی صد چشم از این بی خوابتر بودی مرا
&&
تا نظر می‌کردمی در منظر زیبای تو
\\
ای که در دل جای داری بر سر چشمم نشین
&&
کاندر آن بیغوله ترسم تنگ باشد جای تو
\\
گر ملامت می‌کنندم ور قیامت می‌شود
&&
بنده سر خواهد نهاد آن گه ز سر سودای تو
\\
در ازل رفته‌ست ما را با تو پیوندی که هست
&&
افتقار ما نه امروز است و استغنای تو
\\
گر بخوانی پادشاهی ور برانی بنده‌ایم
&&
رای ما سودی ندارد تا نباشد رای تو
\\
ما قلم در سر کشیدیم اختیار خویش را
&&
نفس ما قربان توست و رخت ما یغمای تو
\\
ما سراپای تو را ای سروتن چون جان خویش
&&
دوست می‌داریم و گر سر می‌رود در پای تو
\\
وین قبای صنعت سعدی که در وی حشو نیست
&&
حد زیبایی ندارد خاصه بر بالای تو
\\
\end{longtable}
\end{center}
