\begin{center}
\section*{غزل ۲۸۷: نه چندان آرزومندم که وصفش در بیان آید}
\label{sec:287}
\addcontentsline{toc}{section}{\nameref{sec:287}}
\begin{longtable}{l p{0.5cm} r}
نه چندان آرزومندم که وصفش در بیان آید
&&
و گر صد نامه بنویسم حکایت بیش از آن آید
\\
مرا تو جان شیرینی به تلخی رفته از اعضا
&&
الا ای جان به تن بازآ و گر نه تن به جان آید
\\
ملامت‌ها که بر من رفت و سختی‌ها که پیش آمد
&&
گر از هر نوبتی فصلی بگویم داستان آید
\\
چه پروای سخن گفتن بود مشتاق خدمت را
&&
حدیث آن گه کند بلبل که گل با بوستان آید
\\
چه سود آب فرات آن گه که جان تشنه بیرون شد
&&
چو مجنون بر کنار افتاد لیلی با میان آید
\\
من ای گل دوست می‌دارم تو را کز بوی مشکینت
&&
چنان مستم که گویی بوی یار مهربان آید
\\
نسیم صبح را گفتم تو با او جانبی داری
&&
کز آن جانب که او باشد صبا عنبرفشان آید
\\
گناه توست اگر وقتی بنالد ناشکیبایی
&&
ندانستی که چون آتش دراندازی دخان آید
\\
خطا گفتم به نادانی که جوری می‌کند عذرا
&&
نمی‌باید که وامق را شکایت بر زبان آید
\\
قلم خاصیتی دارد که سر تا سینه بشکافی
&&
دگربارش بفرمایی به فرق سر دوان آید
\\
زمین باغ و بستان را به عشق باد نوروزی
&&
بباید ساخت با جوری که از باد خزان آید
\\
گرت خونابه گردد دل ز دست دوستان سعدی
&&
نه شرط دوستی باشد که از دل بر دهان آید
\\
\end{longtable}
\end{center}
