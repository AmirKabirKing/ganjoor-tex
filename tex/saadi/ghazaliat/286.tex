\begin{center}
\section*{غزل ۲۸۶: اگر آن عهدشکن با سر میثاق آید}
\label{sec:286}
\addcontentsline{toc}{section}{\nameref{sec:286}}
\begin{longtable}{l p{0.5cm} r}
اگر آن عهدشکن با سر میثاق آید
&&
جان رفته‌ست که با قالب مشتاق آید
\\
همه شب‌های جهان روز کند طلعت او
&&
گر چو صبحیش نظر بر همه آفاق آید
\\
هر غمی را فرجی هست ولیکن ترسم
&&
پیش از آنم بکشد زهر که تریاق آید
\\
بندگی هیچ نکردیم و طمع می‌داریم
&&
که خداوندی از آن سیرت و اخلاق آید
\\
گر همه صورت خوبان جهان جمع کنند
&&
روی زیبای تو دیباچه اوراق آید
\\
دیگری گر همه احسان کند از من بخل است
&&
وز تو مطبوع بود گر همه احراق آید
\\
سرو از آن پای گرفته‌ست به یک جای مقیم
&&
که اگر با تو رود شرمش از آن ساق آید
\\
بی تو گر باد صبا می‌زندم بر دل ریش
&&
همچنان است که آتش که به حراق آید
\\
گر فراقت نکشد جان به وصالت بدهم
&&
تو گرو بردی اگر جفت و اگر طاق آید
\\
سعدیا هر که ندارد سر جان افشانی
&&
مرد آن نیست که در حلقه عشاق آید
\\
\end{longtable}
\end{center}
