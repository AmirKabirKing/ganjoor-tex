\begin{center}
\section*{غزل ۳۷: مپندار از لب شیرین عبارت}
\label{sec:037}
\addcontentsline{toc}{section}{\nameref{sec:037}}
\begin{longtable}{l p{0.5cm} r}
مپندار از لب شیرین عبارت
&&
که کامی حاصل آید بی مرارت
\\
فراق افتد میان دوستداران
&&
زیان و سود باشد در تجارت
\\
یکی را چون ببینی کشته دوست
&&
به دیگر دوستانش ده بشارت
\\
ندانم هیچ کس در عهد حسنت
&&
که بادل باشد الا بی بصارت
\\
مرا آن گوشه چشم دلاویز
&&
به کشتن می‌کند گویی اشارت
\\
گر آن حلوا به دست صوفی افتد
&&
خداترسی نباشد روز غارت
\\
عجب دارم درون عاشقان را
&&
که پیراهن نمی‌سوزد حرارت
\\
جمال دوست چندان سایه انداخت
&&
که سعدی ناپدیدست از حقارت
\\
\end{longtable}
\end{center}
