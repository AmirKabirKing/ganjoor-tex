\begin{center}
\section*{غزل ۴۸۲: هر که به خویشتن رود ره نبرد به سوی او}
\label{sec:482}
\addcontentsline{toc}{section}{\nameref{sec:482}}
\begin{longtable}{l p{0.5cm} r}
هر که به خویشتن رود ره نبرد به سوی او
&&
بینش ما نیاورد طاقت حسن روی او
\\
باغ بنفشه و سمن بوی ندارد ای صبا
&&
غالیه‌ای بساز از آن طره مشکبوی او
\\
هر کس از او به قدر خویش آرزویی همی‌کنند
&&
همت ما نمی‌کند زو به جز آرزوی او
\\
من به کمند او درم او به مراد خویشتن
&&
گر نرود به طبع من من بروم به خوی او
\\
دفع زبان خصم را تا نشوند مطلع
&&
دیده به سوی دیگری دارم و دل به سوی او
\\
دامن من به دست او روز قیامت اوفتد
&&
عمر به نقد می‌رود در سر گفت و گوی او
\\
سعدی اگر برآیدت پای به سنگ دم مزن
&&
روز نخست گفتمت سر نبری ز کوی او
\\
\end{longtable}
\end{center}
