\begin{center}
\section*{غزل ۲۷۰: هر لحظه در برم دل از اندیشه خون شود}
\label{sec:270}
\addcontentsline{toc}{section}{\nameref{sec:270}}
\begin{longtable}{l p{0.5cm} r}
هر لحظه در برم دل از اندیشه خون شود
&&
تا منتهای کار من از عشق چون شود
\\
دل برقرار نیست که گویم نصیحتی
&&
از راه عقل و معرفتش رهنمون شود
\\
یار آن حریف نیست که از در درآیدم
&&
عشق آن حدیث نیست که از دل برون شود
\\
فرهادوارم از لب شیرین گزیر نیست
&&
ور کوه محنتم به مثل بیستون شود
\\
ساکن نمی‌شود نفسی آب چشم من
&&
سیماب طرفه نبود اگر بی سکون شود
\\
دم درکش از ملامتم ای دوست زینهار
&&
کاین درد عاشقی به ملامت فزون شود
\\
جز دیده هیچ دوست ندیدم که سعی کرد
&&
تا زعفران چهره من لاله گون شود
\\
دیوار دل به سنگ تعنت خراب گشت
&&
رخت سرای عقل به یغما کنون شود
\\
چون دور عارض تو برانداخت رسم عقل
&&
ترسم که عشق در سر سعدی جنون شود
\\
\end{longtable}
\end{center}
