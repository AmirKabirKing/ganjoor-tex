\begin{center}
\section*{غزل ۵۹۳: بسم از هوا گرفتن که پری نماند و بالی}
\label{sec:593}
\addcontentsline{toc}{section}{\nameref{sec:593}}
\begin{longtable}{l p{0.5cm} r}
بسم از هوا گرفتن که پری نماند و بالی
&&
به کجا روم ز دستت که نمی‌دهی مجالی
\\
نه ره گریز دارم نه طریق آشنایی
&&
چه غم اوفتاده‌ای را که تواند احتیالی
\\
همه عمر در فراقت بگذشت و سهل باشد
&&
اگر احتمال دارد به قیامت اتصالی
\\
چه خوش است در فراقی همه عمر صبر کردن
&&
به امید آن که روزی به کف اوفتد وصالی
\\
به تو حاصلی ندارد غم روزگار گفتن
&&
که شبی نخفته باشی به درازنای سالی
\\
غم حال دردمندان نه عجب گرت نباشد
&&
که چنین نرفته باشد همه عمر بر تو حالی
\\
سخنی بگوی با من که چنان اسیر عشقم
&&
که به خویشتن ندارم ز وجودت اشتغالی
\\
چه نشینی ای قیامت بنمای سرو قامت
&&
به خلاف سرو بستان که ندارد اعتدالی
\\
که نه امشب آن سماع است که دف خلاص یابد
&&
به طپانچه‌ای و بربط برهد به گوشمالی
\\
دگر آفتاب رویت منمای آسمان را
&&
که قمر ز شرمساری بشکست چون هلالی
\\
خط مشک بوی و خالت به مناسبت تو گویی
&&
قلم غبار می‌رفت و فرو چکید خالی
\\
تو هم این مگوی سعدی که نظر گناه باشد
&&
گنه است برگرفتن نظر از چنین جمالی
\\
\end{longtable}
\end{center}
