\begin{center}
\section*{غزل ۲۷۱: بخت این کند که رای تو با ما یکی شود}
\label{sec:271}
\addcontentsline{toc}{section}{\nameref{sec:271}}
\begin{longtable}{l p{0.5cm} r}
بخت این کند که رای تو با ما یکی شود
&&
تا بشنود حسود و بر او ناوکی شود
\\
خونم بریز و بر سر خاکم گذار کن
&&
کاین رنج و سختیم همه پیش اندکی شود
\\
آن را مسلم است تماشای نوبهار
&&
کز عشق بوستان گل و خارش یکی شود
\\
ای مفلس آنچه در سر توست از خیال گنج
&&
پایت ضرورت است که در مهلکی شود
\\
سعدی در این کمند به دیوانگی فتاد
&&
گر دیگرش خلاص بود زیرکی شود
\\
\end{longtable}
\end{center}
