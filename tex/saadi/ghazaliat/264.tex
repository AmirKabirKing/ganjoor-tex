\begin{center}
\section*{غزل ۲۶۴: هر که مجموع نباشد به تماشا نرود}
\label{sec:264}
\addcontentsline{toc}{section}{\nameref{sec:264}}
\begin{longtable}{l p{0.5cm} r}
هر که مجموع نباشد به تماشا نرود
&&
یار با یار سفرکرده به تنها نرود
\\
باد آسایش گیتی نزند بر دل ریش
&&
صبح صادق ندمد تا شب یلدا نرود
\\
بر دل آویختگان عرصه عالم تنگست
&&
کان که جایی به گل افتاد دگر جا نرود
\\
هرگز اندیشه یار از دل دیوانه عشق
&&
به تماشای گل و سبزه و صحرا نرود
\\
به سر خار مغیلان بروم با تو چنان
&&
به ارادت که یکی بر سر دیبا نرود
\\
با همه رفتن زیبای تذرو اندر باغ
&&
که به شوخی برود پیش تو زیبا نرود
\\
گر تو ای تخت سلیمان به سر ما زین دست
&&
رفت خواهی عجب ار مورچه در پا نرود
\\
باغبانان به شب از زحمت بلبل چونند
&&
که در ایام گل از باغچه غوغا نرود
\\
همه عالم سخنم رفت و به گوشت نرسید
&&
آری آنجا که تو باشی سخن ما نرود
\\
هر که ما را به نصیحت ز تو می‌پیچد روی
&&
گو به شمشیر که عاشق به مدارا نرود
\\
ماه رخسار بپوشی تو بت یغمایی
&&
تا دل خلقی از این شهر به یغما نرود
\\
گوهر قیمتی از کام نهنگان آرند
&&
هر که او را غم جانست به دریا نرود
\\
سعدیا بار کش و یار فراموش مکن
&&
مهر وامق به جفا کردن عذرا نرود
\\
\end{longtable}
\end{center}
