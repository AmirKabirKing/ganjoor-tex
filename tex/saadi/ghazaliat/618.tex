\begin{center}
\section*{غزل ۶۱۸: همه کس را تن و اندام و جمالست و جوانی}
\label{sec:618}
\addcontentsline{toc}{section}{\nameref{sec:618}}
\begin{longtable}{l p{0.5cm} r}
همه کس را تن و اندام و جمال است و جوانی
&&
وین همه لطف ندارد تو مگر سرو روانی
\\
نظر آوردم و بردم که وجودی به تو ماند
&&
همه اسمند و تو جسمی همه جسمند و تو جانی
\\
تو مگر پرده بپوشی و کست روی نبیند
&&
ور همین پرده زنی پرده خلقی بدرانی
\\
تو ندانی که چرا در تو کسی خیره بماند
&&
تا کسی همچو تو باشد که در او خیره بمانی
\\
نوک تیر مژه از جوشن جان می‌گذرانی
&&
من تنک پوست نگفتم تو چنین سخت کمانی
\\
هر چه در حسن تو گویند چنانی به حقیقت
&&
عیبت آن است که با ما به ارادت نه چنانی
\\
رمقی بیش نمانده‌ست گرفتار غمت را
&&
چند مجروح توان داشت بکش تا برهانی
\\
بیش از این صبر ندارم که تو هر دم بر قومی
&&
بنشینی و مرا بر سر آتش بنشانی
\\
گر بمیرد عجب ار شخص و دگر زنده نباشد
&&
که برانی ز در خویش و دگربار بخوانی
\\
سعدیا گر قدمت راه به پایان نرساند
&&
باری اندر طلبش عمر به پایان برسانی
\\
\end{longtable}
\end{center}
