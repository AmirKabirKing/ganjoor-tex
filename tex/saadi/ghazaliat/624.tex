\begin{center}
\section*{غزل ۶۲۴: روزی به زنخدانت گفتم به سیمینی}
\label{sec:624}
\addcontentsline{toc}{section}{\nameref{sec:624}}
\begin{longtable}{l p{0.5cm} r}
روزی به زنخدانت گفتم به سیمینی
&&
گفت ار نظری داری ما را به از این بینی
\\
خورشید و گلت خوانم هم ترک ادب باشد
&&
چرخ مه و خورشیدی باغ گل و نسرینی
\\
حاجت به نگاریدن نبود رخ زیبا را
&&
تو ماه پری پیکر زیبا و نگارینی
\\
بر بستر هجرانت شاید که نپرسندم
&&
کس سوخته خرمن را گوید به چه غمگینی
\\
بنشین که فغان از ما برخاست در ایامت
&&
بس فتنه که برخیزد هر جا که تو بنشینی
\\
گر بنده خود خوانی افتیم به سلطانی
&&
ور روی بگردانی رفتیم به مسکینی
\\
کس عیب نیارد گفت آن را که تو بپسندی
&&
کس رد نتواند کرد آن را که تو بگزینی
\\
عشق لب شیرینت روزی بکشد سعدی
&&
فرهاد چنین کشته‌ست آن شوخ به شیرینی
\\
\end{longtable}
\end{center}
