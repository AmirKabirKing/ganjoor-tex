\begin{center}
\section*{غزل ۱۶۱: کس این کند که ز یار و دیار برگردد}
\label{sec:161}
\addcontentsline{toc}{section}{\nameref{sec:161}}
\begin{longtable}{l p{0.5cm} r}
کس این کند که ز یار و دیار برگردد
&&
کند هرآینه چون روزگار برگردد
\\
تنکدلی که نیارد کشید زحمت گل
&&
ملامتش نکنند ار ز خار برگردد
\\
به جنگ خصم کسی کز حیل فروماند
&&
ضرورتست که بیچاره وار برگردد
\\
به آب تیغ اجل تشنه است مرغ دلم
&&
که نیم کشته به خون چند بار برگردد
\\
به زیر سنگ حوادث کسی چه چاره کند
&&
جز این قدر که به پهلو چو مار برگردد
\\
دلم نماند پس این خون چیست هر ساعت
&&
که در دو دیده یاقوت بار برگردد
\\
گر از دیار به وحشت ملول شد سعدی
&&
گمان مبر که به معنی ز یار برگردد
\\
\end{longtable}
\end{center}
