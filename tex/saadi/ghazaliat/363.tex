\begin{center}
\section*{غزل ۳۶۳: روزگاریست که سودازده روی توام}
\label{sec:363}
\addcontentsline{toc}{section}{\nameref{sec:363}}
\begin{longtable}{l p{0.5cm} r}
روزگاریست که سودازده روی توام
&&
خوابگه نیست مگر خاک سر کوی توام
\\
به دو چشم تو که شوریده‌تر از بخت من است
&&
که به روی تو من آشفته‌تر از موی توام
\\
نقد هر عقل که در کیسه پندارم بود
&&
کمتر از هیچ برآمد به ترازوی توام
\\
همدمی نیست که گوید سخنی پیش منت
&&
محرمی نیست که آرد خبری سوی توام
\\
چشم بر هم نزنم گر تو به تیرم بزنی
&&
لیک ترسم که بدوزد نظر از روی توام
\\
زین سبب خلق جهانند مرید سخنم
&&
که ریاضت کش محراب دو ابروی توام
\\
دست موتم نکند میخ سراپرده عمر
&&
گر سعادت بزند خیمه به پهلوی توام
\\
تو مپندار کز این در به ملامت بروم
&&
که گرم تیغ زنی بنده بازوی توام
\\
سعدی از پرده عشاق چه خوش می‌گوید
&&
ترک من پرده برانداز که هندوی توام
\\
\end{longtable}
\end{center}
