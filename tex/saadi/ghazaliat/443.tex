\begin{center}
\section*{غزل ۴۴۳: بکن چندان که خواهی جور بر من}
\label{sec:443}
\addcontentsline{toc}{section}{\nameref{sec:443}}
\begin{longtable}{l p{0.5cm} r}
بکن چندان که خواهی جور بر من
&&
که دستت بر نمی‌دارم ز دامن
\\
چنان مرغ دلم را صید کردی
&&
که بازش دل نمی‌خواهد نشیمن
\\
اگر دانی که در زنجیر زلفت
&&
گرفتار است در پایش میفکن
\\
به حسن قامتت سروی در آفاق
&&
نپندارم که باشد غالب الظن
\\
الا ای باغبان این سرو بنشان
&&
و گر صاحب دلی آن سرو برکن
\\
جهان روشن به ماه و آفتاب است
&&
جهان ما به دیدار تو روشن
\\
تو بی زیور محلایی و بی رخت
&&
مزکایی و بی زینت مزین
\\
شبی خواهم که مهمان من آیی
&&
به کام دوستان و رغم دشمن
\\
گروهی عام را کز دل خبر نیست
&&
عجب دارند از آه سینه من
\\
چو آتش در سرای افتاده باشد
&&
عجب داری که دود آید ز روزن
\\
تو را خود هر که بیند دوست دارد
&&
گناهی نیست بر سعدی معین
\\
\end{longtable}
\end{center}
