\begin{center}
\section*{غزل ۱۵: برخیز تا یک سو نهیم این دلق ازرق فام را}
\label{sec:015}
\addcontentsline{toc}{section}{\nameref{sec:015}}
\begin{longtable}{l p{0.5cm} r}
برخیز تا یک سو نهیم این دلق ازرق فام را
&&
بر باد قلاشی دهیم این شرک تقوا نام را
\\
هر ساعت از نو قبله‌ای با بت پرستی می‌رود
&&
توحید بر ما عرضه کن تا بشکنیم اصنام را
\\
می با جوانان خوردنم باری تمنا می‌کند
&&
تا کودکان در پی فتند این پیر دردآشام را
\\
از مایه بیچارگی قطمیر مردم می‌شود
&&
ماخولیای مهتری سگ می‌کند بلعام را
\\
زین تنگنای خلوتم خاطر به صحرا می‌کشد
&&
کز بوستان باد سحر خوش می‌دهد پیغام را
\\
غافل مباش ار عاقلی دریاب اگر صاحب دلی
&&
باشد که نتوان یافتن دیگر چنین ایام را
\\
جایی که سرو بوستان با پای چوبین می‌چمد
&&
ما نیز در رقص آوریم آن سرو سیم اندام را
\\
دلبندم آن پیمان گسل منظور چشم آرام دل
&&
نی نی دلارامش مخوان کز دل ببرد آرام را
\\
دنیا و دین و صبر و عقل از من برفت اندر غمش
&&
جایی که سلطان خیمه زد غوغا نماند عام را
\\
باران اشکم می‌رود وز ابرم آتش می‌جهد
&&
با پختگان گوی این سخن سوزش نباشد خام را
\\
سعدی ملامت نشنود ور جان در این سر می‌رود
&&
صوفی گران جانی ببر ساقی بیاور جام را
\\
\end{longtable}
\end{center}
