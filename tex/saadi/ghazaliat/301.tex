\begin{center}
\section*{غزل ۳۰۱: هر شب اندیشه دیگر کنم و رای دگر}
\label{sec:301}
\addcontentsline{toc}{section}{\nameref{sec:301}}
\begin{longtable}{l p{0.5cm} r}
هر شب اندیشه دیگر کنم و رای دگر
&&
که من از دست تو فردا بروم جای دگر
\\
بامدادان که برون می‌نهم از منزل پای
&&
حسن عهدم نگذارد که نهم پای دگر
\\
هر کسی را سر چیزی و تمنای کسیست
&&
ما به غیر از تو نداریم تمنای دگر
\\
زان که هرگز به جمال تو در آیینه وهم
&&
متصور نشود صورت و بالای دگر
\\
وامقی بود که دیوانه عذرایی بود
&&
منم امروز و تویی وامق و عذرای دگر
\\
وقت آنست که صحرا گل و سنبل گیرد
&&
خلق بیرون شده هر قوم به صحرای دگر
\\
بامدادان به تماشای چمن بیرون آی
&&
تا فراغ از تو نماند به تماشای دگر
\\
هر صباحی غمی از دور زمان پیش آید
&&
گویم این نیز نهم بر سر غم‌های دگر
\\
بازگویم نه که دوران حیات این همه نیست
&&
سعدی امروز تحمل کن و فردای دگر
\\
\end{longtable}
\end{center}
