\begin{center}
\section*{غزل ۶۱۲: جمعی که تو در میان ایشانی}
\label{sec:612}
\addcontentsline{toc}{section}{\nameref{sec:612}}
\begin{longtable}{l p{0.5cm} r}
جمعی که تو در میان ایشانی
&&
زآن جمع به در بود پریشانی
\\
ای ذات شریف و شخص روحانی
&&
آرام دلی و مرهم جانی
\\
خرم تن آن که با تو پیوندد
&&
وآن حلقه که در میان ایشانی
\\
من نیز به خدمتت کمر بندم
&&
باشد که غلام خویشتن خوانی
\\
بر خوان تو این شکر که می‌بینم
&&
بی فایده‌ای مگس که می‌رانی
\\
هر جا که تو بگذری بدین خوبی
&&
کس شک نکند که سرو بستانی
\\
هرک این سر دست و ساعدت بیند
&&
گر دل ندهد به پنجه بستانی
\\
من جسم چنین ندیده‌ام هرگز
&&
چندان که قیاس می‌کنم جانی
\\
بر دیده من برو که مخدومی
&&
پروانه به خون بده که سلطانی
\\
من سر ز خط تو بر نمی‌گیرم
&&
ور چون قلمم به سر بگردانی
\\
این گرد که بر رخ است می‌بینی
&&
وآن درد که در دل است می‌دانی
\\
دودی که بیاید از دل سعدی
&&
پیداست که آتشیست پنهانی
\\
می‌گوید و جان به رقص می‌آید
&&
خوش می‌رود این سماع روحانی
\\
\end{longtable}
\end{center}
