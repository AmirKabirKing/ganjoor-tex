\begin{center}
\section*{غزل ۲۵۸: مرا راحت از زندگی دوش بود}
\label{sec:258}
\addcontentsline{toc}{section}{\nameref{sec:258}}
\begin{longtable}{l p{0.5cm} r}
مرا راحت از زندگی دوش بود
&&
که آن ماهرویم در آغوش بود
\\
چنان مست دیدار و حیران عشق
&&
که دنیا و دینم فراموش بود
\\
نگویم می لعل شیرین گوار
&&
که زهر از کف دست او نوش بود
\\
ندانستم از غایت لطف و حسن
&&
که سیم و سمن یا بر و دوش بود
\\
به دیدار و گفتار جان پرورش
&&
سراپای من دیده و گوش بود
\\
نمی‌دانم این شب که چون روز شد
&&
کسی باز داند که با هوش بود
\\
مؤذن غلط کرد بانگ نماز
&&
مگر همچو من مست و مدهوش بود
\\
بگفتیم و دشمن بدانست و دوست
&&
نماند آن تحمل که سرپوش بود
\\
به خوابش مگر دیده‌ای سعدیا
&&
زبان در کش امروز کآن دوش بود
\\
مبادا که گنجی ببیند فقیر
&&
که نتواند از حرص خاموش بود
\\
\end{longtable}
\end{center}
