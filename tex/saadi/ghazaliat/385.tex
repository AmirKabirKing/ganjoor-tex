\begin{center}
\section*{غزل ۳۸۵: یک امشبی که در آغوش شاهد شکرم}
\label{sec:385}
\addcontentsline{toc}{section}{\nameref{sec:385}}
\begin{longtable}{l p{0.5cm} r}
یک امشبی که در آغوش شاهد شکرم
&&
گرم چو عود بر آتش نهند غم نخورم
\\
چو التماس برآمد هلاک باکی نیست
&&
کجاست تیر بلا گو بیا که من سپرم
\\
ببند یک نفس ای آسمان دریچه صبح
&&
بر آفتاب که امشب خوش است با قمرم
\\
ندانم این شب قدر است یا ستاره روز
&&
تویی برابر من یا خیال در نظرم
\\
خوشا هوای گلستان و خواب در بستان
&&
اگر نبودی تشویش بلبل سحرم
\\
بدین دو دیده که امشب تو را همی‌بینم
&&
دریغ باشد فردا که دیگری نگرم
\\
روان تشنه برآساید از وجود فرات
&&
مرا فرات ز سر برگذشت و تشنه‌ترم
\\
چو می‌ندیدمت از شوق بی‌خبر بودم
&&
کنون که با تو نشستم ز ذوق بی‌خبرم
\\
سخن بگوی که بیگانه پیش ما کس نیست
&&
به غیر شمع و همین ساعتش زبان ببرم
\\
میان ما به جز این پیرهن نخواهد بود
&&
و گر حجاب شود تا به دامنش بدرم
\\
مگوی سعدی از این درد جان نخواهد برد
&&
بگو کجا برم آن جان که از غمت ببرم
\\
\end{longtable}
\end{center}
