\begin{center}
\section*{غزل ۶۱۴: کبر یک سو نه اگر شاهد درویشانی}
\label{sec:614}
\addcontentsline{toc}{section}{\nameref{sec:614}}
\begin{longtable}{l p{0.5cm} r}
کبر یک سو نه اگر شاهد درویشانی
&&
دیو خوش طبع به از حور گره پیشانی
\\
آرزو می‌کندم با تو دمی در بستان
&&
یا به هر گوشه که باشد که تو خود بستانی
\\
با من کشته هجران نفسی خوش بنشین
&&
تا مگر زنده شوم زآن نفس روحانی
\\
گر در آفاق بگردی به جز آیینه تو را
&&
صورتی کس ننماید که بدو می‌مانی
\\
هیچ دورانی بی فتنه نگویند که بود
&&
تو بدین حسن مگر فتنه این دورانی
\\
مردم از ترس خدا سجده رویت نکنند
&&
بامدادت که ببینند و من از حیرانی
\\
گرم از پیش برانی و به شوخی نروم
&&
عفو فرمای که عجز است نه بی فرمانی
\\
نه گزیر است مرا از تو نه امکان گریز
&&
چاره صبر است که هم دردی و هم درمانی
\\
بندگان را نبود جز غم آزادی و من
&&
پادشاهی کنم ار بنده خویشم خوانی
\\
زین سخن‌های دلاویز که شرح غم توست
&&
خرمنی دارم و ترسم به جوی نستانی
\\
تو که یک روز پراکنده نبوده‌ست دلت
&&
صورت حال پراکنده دلان کی دانی
\\
نفسی بنده نوازی کن و بنشین ار چند
&&
آتشی نیست که او را به دمی بنشانی
\\
سخن زنده دلان گوش کن از کشته خویش
&&
چون دلم زنده نباشد که تو در وی جانی
\\
این توانی که نیایی ز در سعدی باز
&&
لیک بیرون روی از خاطر او نتوانی
\\
\end{longtable}
\end{center}
