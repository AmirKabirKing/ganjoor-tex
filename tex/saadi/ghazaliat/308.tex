\begin{center}
\section*{غزل ۳۰۸: فتنه‌ام بر زلف و بالای تو ای بدر منیر}
\label{sec:308}
\addcontentsline{toc}{section}{\nameref{sec:308}}
\begin{longtable}{l p{0.5cm} r}
فتنه‌ام بر زلف و بالای تو ای بدر منیر
&&
قامت است آن یا قیامت عنبر است آن یا عبیر
\\
گم شدم در راه سودا رهنمایا ره نمای
&&
شخصم از پای اندر آمد دستگیرا دست گیر
\\
گر ز پیش خود برانی چون سگ از مسجد مرا
&&
سر ز حکمت برندارم چون مرید از گفت پیر
\\
ناوک فریاد من هر ساعت از مجرای دل
&&
بگذرد از چرخ اطلس همچو سوزن از حریر
\\
چون کنم کز دل شکیبایم ز دلبر ناشکیب
&&
چون کنم کز جان گزیر است و ز جانان ناگزیر
\\
بی تو گر در جنتم ناخوش شراب سلسبیل
&&
با تو گر در دوزخم خرم هوای زمهریر
\\
گر بپرد مرغ وصلت در هوای بخت من
&&
وه که آن ساعت ز شادی چارپر گردم چو تیر
\\
تا روانم هست خواهم راند نامت بر زبان
&&
تا وجودم هست خواهم کند نقشت در ضمیر
\\
گر نبارد فضل باران عنایت بر سرم
&&
لابه بر گردون رسانم چون جهودان در فطیر
\\
بوالعجب شوریده‌ام سهوم به رحمت درگذار
&&
سهمگن درمانده‌ام جرمم به طاعت درپذیر
\\
آه دردآلود سعدی گر ز گردون بگذرد
&&
در تو کافردل نگیرد ای مسلمانان نفیر
\\
\end{longtable}
\end{center}
