\begin{center}
\section*{غزل ۸۸: با خردمندی و خوبی پارسا و نیک خوست}
\label{sec:088}
\addcontentsline{toc}{section}{\nameref{sec:088}}
\begin{longtable}{l p{0.5cm} r}
با خردمندی و خوبی پارسا و نیکخوست
&&
صورتی هرگز ندیدم کاین همه معنی در اوست
\\
گر خیال یاری اندیشند باری چون تو یار
&&
یا هوای دوستی ورزند باری چون تو دوست
\\
خاک پایش بوسه خواهم داد آبم گو ببر
&&
آبروی مهربانان پیش معشوق آب جوست
\\
شاهدش دیدار و گفتن فتنه اش ابرو و چشم
&&
نادرش بالا و رفتن دلپذیرش طبع و خوست
\\
تا به خود بازآیم آن گه وصف دیدارش کنم
&&
از که می‌پرسی در این میدان که سرگردان چو گوست
\\
عیب پیراهن دریدن می‌کنندم دوستان
&&
بی‌وفا یارم که پیراهن همی‌درم نه پوست
\\
خاک سبزآرنگ و باد گلفشان و آب خوش
&&
ابر مرواریدباران و هوای مشک بوست
\\
تیرباران بر سر و صوفی گرفتار نظر
&&
مدعی در گفت و گوی و عاشق اندر جست و جوست
\\
هر که را کنج اختیار آمد تو دست از وی بدار
&&
کان چنان شوریده سر پایش به گنجی در فروست
\\
چشم اگر با دوست داری گوش با دشمن مکن
&&
عاشقی و نیک نامی سعدیا سنگ و سبوست
\\
\end{longtable}
\end{center}
