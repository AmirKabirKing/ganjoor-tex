\begin{center}
\section*{غزل ۱۳۹: دلم از دست غمت دامن صحرا بگرفت}
\label{sec:139}
\addcontentsline{toc}{section}{\nameref{sec:139}}
\begin{longtable}{l p{0.5cm} r}
دلم از دست غمت دامن صحرا بگرفت
&&
غمت از سر ننهم گر دلت از ما بگرفت
\\
خال مشکین تو از بنده چرا در خط شد
&&
مگر از دود دلم روی تو سودا بگرفت
\\
دوش چون مشعله شوق تو بگرفت وجود
&&
سایه‌ای در دلم انداخت که صد جا بگرفت
\\
به دم سرد سحرگاهی من بازنشست
&&
هر چراغی که زمین از دل صهبا بگرفت
\\
الغیاث از من دل سوخته ای سنگین دل
&&
در تو نگرفت که خون در دل خارا بگرفت
\\
دل شوریده ما عالم اندیشه ماست
&&
عالم از شوق تو در تاب که غوغا بگرفت
\\
بربود انده تو صبرم و نیکو بربود
&&
بگرفت انده تو جانم و زیبا بگرفت
\\
دل سعدی همه ز ایام بلا پرهیزد
&&
سر زلف تو ندانم به چه یارا بگرفت
\\
\end{longtable}
\end{center}
