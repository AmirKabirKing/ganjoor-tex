\begin{center}
\section*{غزل ۴۱۴: اگر دستم رسد روزی که انصاف از تو بستانم}
\label{sec:414}
\addcontentsline{toc}{section}{\nameref{sec:414}}
\begin{longtable}{l p{0.5cm} r}
اگر دستم رسد روزی که انصاف از تو بستانم
&&
قضای عهد ماضی را شبی دستی برافشانم
\\
چنانت دوست می‌دارم که گر روزی فراق افتد
&&
تو صبر از من توانی کرد و من صبر از تو نتوانم
\\
دلم صد بار می‌گوید که چشم از فتنه بر هم نه
&&
دگر ره دیده می‌افتد بر آن بالای فتانم
\\
تو را در بوستان باید که پیش سرو بنشینی
&&
و گر نه باغبان گوید که دیگر سرو ننشانم
\\
رفیقانم سفر کردند هر یاری به اقصایی
&&
خلاف من که بگرفته است دامن در مغیلانم
\\
به دریایی درافتادم که پایانش نمی‌بینم
&&
کسی را پنجه افکندم که درمانش نمی‌دانم
\\
فراقم سخت می‌آید ولیکن صبر می‌باید
&&
که گر بگریزم از سختی رفیق سست پیمانم
\\
مپرسم دوش چون بودی به تاریکی و تنهایی
&&
شب هجرم چه می‌پرسی که روز وصل حیرانم
\\
شبان آهسته می‌نالم مگر دردم نهان ماند
&&
به گوش هر که در عالم رسید آواز پنهانم
\\
دمی با دوست در خلوت به از صد سال در عشرت
&&
من آزادی نمی‌خواهم که با یوسف به زندانم
\\
من آن مرغ سخندانم که در خاکم رود صورت
&&
هنوز آواز می‌آید به معنی از گلستانم
\\
\end{longtable}
\end{center}
