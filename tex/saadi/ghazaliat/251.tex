\begin{center}
\section*{غزل ۲۵۱: اگر تو برشکنی دوستان سلام کنند}
\label{sec:251}
\addcontentsline{toc}{section}{\nameref{sec:251}}
\begin{longtable}{l p{0.5cm} r}
اگر تو برشکنی دوستان سلام کنند
&&
که جور قاعده باشد که بر غلام کنند
\\
هزار زخم پیاپی گر اتفاق افتد
&&
ز دست دوست نشاید که انتقام کنند
\\
به تیغ اگر بزنی بی‌دریغ و برگردی
&&
چو روی باز کنی بازت احترام کنند
\\
مرا کمند میفکن که خود گرفتارم
&&
لویشه بر سر اسبان بدلگام کنند
\\
چو مرغ خانه به سنگم بزن که بازآیم
&&
نه وحشیم که مرا پای بند دام کنند
\\
یکی به گوشه چشم التفات کن ما را
&&
که پادشاهان گه گه نظر به عام کنند
\\
که گفت در رخ زیبا حلال نیست نظر
&&
حلال نیست که بر دوستان حرام کنند
\\
ز من بپرس که فتوی دهم به مذهب عشق
&&
نظر به روی تو شاید که بر دوام کنند
\\
دهان غنچه بدرد نسیم باد صبا
&&
لبان لعل تو وقتی که ابتسام کنند
\\
غریب مشرق و مغرب به آشنایی تو
&&
غریب نیست که در شهر ما مقام کنند
\\
من از تو روی نپیچم که شرط عشق آن است
&&
که روی در غرض و پشت بر ملام کنند
\\
به جان مضایقه با دوستان مکن سعدی
&&
که دوستی نبود هر چه ناتمام کنند
\\
\end{longtable}
\end{center}
