\begin{center}
\section*{غزل ۴۰۶: بار فراق دوستان بس که نشست بر دلم}
\label{sec:406}
\addcontentsline{toc}{section}{\nameref{sec:406}}
\begin{longtable}{l p{0.5cm} r}
بار فراق دوستان بس که نشست بر دلم
&&
می‌روم و نمی‌رود ناقه به زیر محملم
\\
بار بیفکند شتر چون برسد به منزلی
&&
بار دل است همچنان ور به هزار منزلم
\\
ای که مهار می‌کشی صبر کن و سبک مرو
&&
کز طرفی تو می‌کشی وز طرفی سلاسلم
\\
بارکشیده جفا پرده دریده هوا
&&
راه ز پیش و دل ز پس واقعه‌ایست مشکلم
\\
معرفت قدیم را بعد حجاب کی شود
&&
گر چه به شخص غایبی در نظری مقابلم
\\
آخر قصد من تویی غایت جهد و آرزو
&&
تا نرسم ز دامنت دست امید نگسلم
\\
ذکر تو از زبان من فکر تو از جنان من
&&
چون برود که رفته‌ای در رگ و در مفاصلم
\\
مشتغل توام چنان کز همه چیز غایبم
&&
مفتکر توام چنان کز همه خلق غافلم
\\
گر نظری کنی کند کشته صبر من ورق
&&
ور نکنی چه بر دهد بیخ امید باطلم
\\
سنت عشق سعدیا ترک نمی‌دهی بلی
&&
کی ز دلم به در رود خوی سرشته در گلم
\\
داروی درد شوق را با همه علم عاجزم
&&
چاره کار عشق را با همه عقل جاهلم
\\
\end{longtable}
\end{center}
