\begin{center}
\section*{غزل ۴۹۹: خرم آن روز که چون گل به چمن بازآیی}
\label{sec:499}
\addcontentsline{toc}{section}{\nameref{sec:499}}
\begin{longtable}{l p{0.5cm} r}
خرم آن روز که چون گل به چمن بازآیی
&&
یا به بستان به در حجره من بازآیی
\\
گلبن عیش من آن روز شکفتن گیرد
&&
که تو چون سرو خرامان به چمن بازآیی
\\
شمع من روز نیامد که شبم بفروزی
&&
جان من وقت نیامد که به تن بازآیی
\\
آب تلخ است مدامم چو صراحی در حلق
&&
تا تو یک روز چو ساغر به دهن بازآیی
\\
کی به دیدار من ای مهرگسل برخیزی
&&
کی به گفتار من ای عهدشکن بازآیی
\\
مرغ سیر آمده‌ای از قفس صحبت و من
&&
دام زاری بنهم بو که به من بازآیی
\\
من خود آن بخت ندارم که به تو پیوندم
&&
نه تو آن لطف نداری که به من بازآیی
\\
سعدی آن دیو نباشد که به افسون برود
&&
هیچت افتد که چو مردم به سخن بازآیی
\\
\end{longtable}
\end{center}
