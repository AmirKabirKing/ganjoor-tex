\begin{center}
\section*{غزل ۳۴: دل هر که صید کردی نکشد سر از کمندت}
\label{sec:034}
\addcontentsline{toc}{section}{\nameref{sec:034}}
\begin{longtable}{l p{0.5cm} r}
دل هر که صید کردی نکشد سر از کمندت
&&
نه دگر امید دارد که رها شود ز بندت
\\
به خدا که پرده از روی چو آتشت برافکن
&&
که به اتفاق بینی دل عالمی سپندت
\\
نه چمن شکوفه‌ای رست چو روی دلستانت
&&
نه صبا صنوبری یافت چو قامت بلندت
\\
گرت آرزوی آنست که خون خلق ریزی
&&
چه کند که شیر گردن ننهد چو گوسفندت
\\
تو امیر ملک حسنی به حقیقت ای دریغا
&&
اگر التفات بودی به فقیر مستمندت
\\
نه تو را بگفتم ای دل که سر وفا ندارد
&&
به طمع ز دست رفتی و به پای درفکندت
\\
تو نه مرد عشق بودی خود از این حساب سعدی
&&
که نه قوت گریزست و نه طاقت گزندت
\\
\end{longtable}
\end{center}
