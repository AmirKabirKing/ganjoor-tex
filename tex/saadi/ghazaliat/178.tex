\begin{center}
\section*{غزل ۱۷۸: کیست آن فتنه که با تیر و کمان می‌گذرد}
\label{sec:178}
\addcontentsline{toc}{section}{\nameref{sec:178}}
\begin{longtable}{l p{0.5cm} r}
کیست آن فتنه که با تیر و کمان می‌گذرد
&&
وان چه تیرست که در جوشن جان می‌گذرد
\\
آن نه شخصی که جهانیست پر از لطف و کمال
&&
عمر ضایع مکن ای دل که جهان می‌گذرد
\\
آشکارا نپسندد دگر آن روی چو ماه
&&
گر بداند که چه بر خلق نهان می‌گذرد
\\
آخر ای نادره دور زمان از سر لطف
&&
بر ما آی زمانی که زمان می‌گذرد
\\
صورت روی تو ای ماه دلارای چنانک
&&
صورت حال من از شرح و بیان می‌گذرد
\\
تا دگر باد صبایی به چمن بازآید
&&
عمر می‌بینم و چون برق یمان می‌گذرد
\\
آتشی در دل سعدی به محبت زده‌ای
&&
دود آنست که وقتی به زبان می‌گذرد
\\
\end{longtable}
\end{center}
