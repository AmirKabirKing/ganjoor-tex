\begin{center}
\section*{غزل ۲۵: اگر تو برفکنی در میان شهر نقاب}
\label{sec:025}
\addcontentsline{toc}{section}{\nameref{sec:025}}
\begin{longtable}{l p{0.5cm} r}
اگر تو برفکنی در میان شهر نقاب
&&
هزار مؤمن مخلص درافکنی به عقاب
\\
که را مجال نظر بر جمال میمونت
&&
بدین صفت که تو دل می‌بری ورای حجاب
\\
درون ما ز تو یک دم نمی‌شود خالی
&&
کنون که شهر گرفتی روا مدار خراب
\\
به موی تافته پای دلم فروبستی
&&
چو موی تافتی ای نیکبخت روی متاب
\\
تو را حکایت ما مختصر به گوش آید
&&
که حال تشنه نمی‌دانی ای گل سیراب
\\
اگر چراغ بمیرد صبا چه غم دارد
&&
و گر بریزد کتان چه غم خورد مهتاب
\\
دعات گفتم و دشنام اگر دهی سهل است
&&
که با شکردهنان خوش بود سؤال و جواب
\\
کجایی ای که تعنت کنی و طعنه زنی
&&
تو بر کناری و ما اوفتاده در غرقاب
\\
اسیر بند بلا را چه جای سرزنش است
&&
گرت معاونتی دست می‌دهد دریاب
\\
اگر چه صبر من از روی دوست ممکن نیست
&&
همی‌کنم به ضرورت چو صبر ماهی از آب
\\
تو باز دعوی پرهیز می‌کنی سعدی
&&
که دل به کس ندهم کل مدع کذاب
\\
\end{longtable}
\end{center}
