\begin{center}
\section*{غزل ۶۱: افسوس بر آن دیده که روی تو ندیدست}
\label{sec:061}
\addcontentsline{toc}{section}{\nameref{sec:061}}
\begin{longtable}{l p{0.5cm} r}
افسوس بر آن دیده که روی تو ندیده‌ست
&&
یا دیده و بعد از تو به رویی نگریده‌ست
\\
گر مدعیان نقش ببینند پری را
&&
دانند که دیوانه چرا جامه دریده‌ست
\\
آن کیست که پیرامن خورشید جمالش
&&
از مشک سیه دایرهٔ نیمه کشیده‌ست
\\
ای عاقل اگر پای به سنگیت برآید
&&
فرهاد بدانی که چرا سنگ بریده‌ست
\\
رحمت نکند بر دل بیچاره فرهاد
&&
آن کس که سخن گفتن شیرین نشنیده‌ست
\\
از دست کمان مهرهٔ ابروی تو در شهر
&&
دل نیست که در بر چو کبوتر نطپیده‌ست
\\
در وهم نیاید که چه مطبوع درختی
&&
پیداست که هرگز کس از این میوه نچیده‌ست
\\
سر قلم قدرت بی چون الهی
&&
در روی تو چون روی در آیینه پدید است
\\
ما از تو به غیر از تو نداریم تمنا
&&
حلوا به کسی ده که محبت نچشیده‌ست
\\
با این همه باران بلا بر سر سعدی
&&
نشگفت اگرش خانهٔ چشم آب چکیده‌ست
\\
\end{longtable}
\end{center}
