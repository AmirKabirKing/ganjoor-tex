\begin{center}
\section*{غزل ۳۹۱: من اگر نظر حرامست بسی گناه دارم}
\label{sec:391}
\addcontentsline{toc}{section}{\nameref{sec:391}}
\begin{longtable}{l p{0.5cm} r}
من اگر نظر حرام است بسی گناه دارم
&&
چه کنم نمی‌توانم که نظر نگاه دارم
\\
ستم از کسیست بر من که ضرورت است بردن
&&
نه قرار زخم خوردن نه مجال آه دارم
\\
نه فراغت نشستن نه شکیب رخت بستن
&&
نه مقام ایستادن نه گریزگاه دارم
\\
نه اگر همی‌نشینم نظری کند به رحمت
&&
نه اگر همی‌گریزم دگری پناه دارم
\\
بسم از قبول عامی و صلاح نیکنامی
&&
چو به ترک سر بگفتم چه غم از کلاه دارم
\\
تن من فدای جانت سر بنده وآستانت
&&
چه مرا به از گدایی چو تو پادشاه دارم
\\
چو تو را بدین شگرفی قدم صلاح باشد
&&
نه مروت است اگر من نظر تباه دارم
\\
چه شب است یا رب امشب که ستاره‌ای برآمد
&&
که دگر نه عشق خورشید و نه مهر ماه دارم
\\
مکنید دردمندان گله از شب جدایی
&&
که من این صباح روشن ز شب سیاه دارم
\\
که نه روی خوب دیدن گنه است پیش سعدی
&&
تو گمان نیک بردی که من این گناه دارم
\\
\end{longtable}
\end{center}
