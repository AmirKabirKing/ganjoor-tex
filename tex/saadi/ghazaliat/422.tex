\begin{center}
\section*{غزل ۴۲۲: آن کس که از او صبر محالست و سکونم}
\label{sec:422}
\addcontentsline{toc}{section}{\nameref{sec:422}}
\begin{longtable}{l p{0.5cm} r}
آن کس که از او صبر محال است و سکونم
&&
بگذشت ده انگشت فروبرده به خونم
\\
پرسید که چونی ز غم و درد جدایی
&&
گفتم نه چنانم که توان گفت که چونم
\\
زان گه که مرا روی تو محراب نظر شد
&&
از دست زبان‌ها به تحمل چو ستونم
\\
مشنو که همه عمر جفا برده‌ام از کس
&&
جز بر سر کوی تو که دیوار زبونم
\\
بیم است چو شرح غم عشق تو نویسم
&&
کآتش به قلم در فتد از سوز درونم
\\
آنان که شمردند مرا عاقل و هشیار
&&
کو تا بنویسند گواهی به جنونم
\\
شمشیر برآور که مرادم سر سعدیست
&&
ور سر ننهم در قدمت عاشق دونم
\\
\end{longtable}
\end{center}
