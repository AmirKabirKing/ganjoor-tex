\begin{center}
\section*{غزل ۱۳۵: آن را که میسر نشود صبر و قناعت}
\label{sec:135}
\addcontentsline{toc}{section}{\nameref{sec:135}}
\begin{longtable}{l p{0.5cm} r}
آن را که میسر نشود صبر و قناعت
&&
باید که ببندد کمر خدمت و طاعت
\\
چون دوست گرفتی چه غم از دشمن خونخوار؟
&&
گو بوق ملامت بزن و کوس شناعت
\\
گر خود همه بیداد کند هیچ مگویید
&&
تعذیب دلارام به از ذل شفاعت
\\
از هر چه تو گویی به قناعت بشکیبم
&&
امکان شکیب از تو محالست و قناعت
\\
گر نسخه روی تو به بازار برآرند
&&
نقاش ببندد در دکان صناعت
\\
جان بر کف دست آمده تا روی تو بیند
&&
خود شرم نمی‌آیدش از ننگ بضاعت
\\
دریاب دمی صحبت یاری که دگربار
&&
چون رفت نیاید به کمند آن دم و ساعت
\\
انصاف نباشد که من خسته رنجور
&&
پروانه او باشم و او شمع جماعت
\\
لیکن چه توان کرد که قوت نتوان کرد
&&
با گردش ایام به بازوی شجاعت
\\
دل در هوست خون شد و جان در طلبت سوخت
&&
با این همه سعدی خجل از ننگ بضاعت
\\
\end{longtable}
\end{center}
