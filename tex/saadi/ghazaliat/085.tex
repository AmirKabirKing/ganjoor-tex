\begin{center}
\section*{غزل ۸۵: با همه مهر و با منش کینست}
\label{sec:085}
\addcontentsline{toc}{section}{\nameref{sec:085}}
\begin{longtable}{l p{0.5cm} r}
با همه مهر و با منش کینست
&&
چه کنم حظ بخت من اینست
\\
شاید ای نفس تا دگر نکنی
&&
پنجه با ساعدی که سیمینست
\\
ننهد پای تا نبیند جای
&&
هر که را چشم مصلحت بینست
\\
مثل زیرکان و چنبر عشق
&&
طفل نادان و مار رنگینست
\\
دردمند فراق سر ننهد
&&
مگر آن شب که گور بالینست
\\
گریه گو بر هلاک من مکنید
&&
که نه این نوبت نخستینست
\\
لازمست احتمال چندین جور
&&
که محبت هزار چندینست
\\
گر هزارم جواب تلخ دهی
&&
اعتقاد من آن که شیرینست
\\
مرد اگر شیر در کمند آرد
&&
چون کمندش گرفت مسکینست
\\
سعدیا تن به نیستی درده
&&
چاره با سخت بازوان اینست
\\
\end{longtable}
\end{center}
