\begin{center}
\section*{غزل ۵۴: سرو چمن پیش اعتدال تو پستست}
\label{sec:054}
\addcontentsline{toc}{section}{\nameref{sec:054}}
\begin{longtable}{l p{0.5cm} r}
سرو چمن پیش اعتدال تو پست است
&&
روی تو بازار آفتاب شکسته‌ست
\\
شمع فلک با هزار مشعل انجم
&&
پیش وجودت چراغ بازنشسته‌ست
\\
توبه کند مردم از گناه به شعبان
&&
در رمضان نیز چشم‌های تو مست است
\\
این همه زورآوری و مردی و شیری
&&
مرد ندانم که از کمند تو جسته‌ست
\\
این یکی از دوستان به تیغ تو کشته‌ست
&&
وان دگر از عاشقان به تیر تو خسته‌ست
\\
دیده به دل می‌برد حکایت مجنون
&&
دیده ندارد که دل به مهر نبسته‌ست
\\
دست طلب داشتن ز دامن معشوق
&&
پیش کسی گو کش اختیار به دست است
\\
با چو تو روحانیی تعلق خاطر
&&
هر که ندارد دواب نفس‌پرست است
\\
منکر سعدی که ذوق عشق ندارد
&&
نیشکرش در دهان تلخ کبست است
\\
\end{longtable}
\end{center}
