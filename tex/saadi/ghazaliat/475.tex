\begin{center}
\section*{غزل ۴۷۵: بهست آن یا زنخ یا سیب سیمین}
\label{sec:475}
\addcontentsline{toc}{section}{\nameref{sec:475}}
\begin{longtable}{l p{0.5cm} r}
به است آن یا زنخ یا سیب سیمین
&&
لب است آن یا شکر یا جان شیرین
\\
بتی دارم که چین ابروانش
&&
حکایت می‌کند بتخانه چین
\\
از آن ساعت که دیدم گوشوارش
&&
ز چشمانم بیفتاده‌ست پروین
\\
هر آن وقتی که دیدارش نبینم
&&
جهانم تیره باشد بر جهان بین
\\
به خوابی آرزومندم ولیکن
&&
سر بی دوست چون باشد به بالین
\\
از آب و گل چنین صورت که دیده‌ست
&&
تعالی خالق الانسان من طین
\\
غرور نیکوان باشد نه چندان
&&
جفا بر عاشقان باشد نه چندین
\\
من از مهری که دارم برنگردم
&&
تو را گر خاطر مهر است و گر کین
\\
نگارینا به شمشیرت چه حاجت
&&
مرا خود می‌کشد دست نگارین
\\
به دست دوستان بر کشته بودن
&&
ز دنیا رفتنی باشد به تمکین
\\
بکش تا عیب گیرانم نگویند
&&
نمی‌آید ملخ در چشم شاهین
\\
نظر کردن به خوبان دین سعدیست
&&
مباد آن روز کاو برگردد از دین
\\
\end{longtable}
\end{center}
