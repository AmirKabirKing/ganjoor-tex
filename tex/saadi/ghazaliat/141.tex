\begin{center}
\section*{غزل ۱۴۱: هر که دلارام دید از دلش آرام رفت}
\label{sec:141}
\addcontentsline{toc}{section}{\nameref{sec:141}}
\begin{longtable}{l p{0.5cm} r}
هر که دلارام دید از دلش آرام رفت
&&
چشم ندارد خلاص هر که در این دام رفت
\\
یاد تو می‌رفت و ما عاشق و بیدل بدیم
&&
پرده برانداختی کار به اتمام رفت
\\
ماه نتابد به روز چیست که در خانه تافت
&&
سرو نروید به بام کیست که بر بام رفت
\\
مشعله‌ای برفروخت پرتو خورشید عشق
&&
خرمن خاصان بسوخت خانگه عام رفت
\\
عارف مجموع را در پس دیوار صبر
&&
طاقت صبرش نبود ننگ شد و نام رفت
\\
گر به همه عمر خویش با تو برآرم دمی
&&
حاصل عمر آن دمست باقی ایام رفت
\\
هر که هوایی نپخت یا به فراقی نسوخت
&&
آخر عمر از جهان چون برود خام رفت
\\
ما قدم از سر کنیم در طلب دوستان
&&
راه به جایی نبرد هر که به اقدام رفت
\\
همت سعدی به عشق میل نکردی ولی
&&
می چو فرو شد به کام عقل به ناکام رفت
\\
\end{longtable}
\end{center}
