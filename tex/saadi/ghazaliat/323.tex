\begin{center}
\section*{غزل ۳۲۳: هر که نازک بود تن یارش}
\label{sec:323}
\addcontentsline{toc}{section}{\nameref{sec:323}}
\begin{longtable}{l p{0.5cm} r}
هر که نازک بود تن یارش
&&
گو دل نازنین نگه دارش
\\
عاشق گل دروغ می‌گوید
&&
که تحمل نمی‌کند خارش
\\
نیکخواها در آتشم بگذار
&&
وین نصیحت مکن که بگذارش
\\
کاش با دل هزار جان بودی
&&
تا فدا کردمی به دیدارش
\\
عاشق صادق از ملامت دوست
&&
گر برنجد به دوست مشمارش
\\
کس به آرام جان ما نرسد
&&
که نه اول به جان رسد کارش
\\
خانه یار سنگدل این است
&&
هر که سر می‌زند به دیوارش
\\
خون ما خود محل آن دارد
&&
که بود پیش دوست مقدارش
\\
سعدیا گر به جان خطاب کند
&&
ترک جان گوی و دل به دست آرش
\\
\end{longtable}
\end{center}
