\begin{center}
\section*{غزل ۳۲۲: خجلست سرو بستان بر قامت بلندش}
\label{sec:322}
\addcontentsline{toc}{section}{\nameref{sec:322}}
\begin{longtable}{l p{0.5cm} r}
خجل است سرو بستان بر قامت بلندش
&&
همه صید عقل گیرد خم زلف چون کمندش
\\
چو درخت قامتش دید صبا به هم برآمد
&&
ز چمن نرست سروی که ز بیخ برنکندش
\\
اگر آفتاب با او زند از گزاف لافی
&&
مه نو چه زهره دارد که بود سم سمندش
\\
نه چنان ز دست رفته‌ست وجود ناتوانم
&&
که معالجت توان کرد به پند یا به بندش
\\
گرم آن قرار بودی که ز دوست برکنم دل
&&
نشنیدمی ز دشمن سخنان ناپسندش
\\
تو که پادشاه حسنی نظری به بندگان کن
&&
حذر از دعای درویش و کف نیازمندش
\\
شکرین حدیث سعدی بر او چه قدر دارد
&&
که چنو هزار طوطی مگس است پیش قندش
\\
\end{longtable}
\end{center}
