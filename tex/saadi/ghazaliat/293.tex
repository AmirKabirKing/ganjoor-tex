\begin{center}
\section*{غزل ۲۹۳: آفتابست آن پری رخ یا ملایک یا بشر}
\label{sec:293}
\addcontentsline{toc}{section}{\nameref{sec:293}}
\begin{longtable}{l p{0.5cm} r}
آفتابست آن پری رخ یا ملایک یا بشر
&&
قامتست آن یا قیامت یا الف یا نیشکر
\\
هد صبری ما تولی رد عقلی ما ثنا
&&
صاد قلبی ما تمشی زاد وجدی ما عبر
\\
گلبنست آن یا تن نازک نهادش یا حریر
&&
آهنست آن یا دل نامهربانش یا حجر
\\
تهت و المطلوب عندی کیف حالی ان نا
&&
حرت و المامول نحوی ما احتیالی ان هجر
\\
باغ فردوسست گلبرگش نخوانم یا بهار
&&
جان شیرینست خورشیدش نگویم یا قمر
\\
قل لمن یبغی فرارا منه هل لی سلوه
&&
ام علی التقدیر انی ابتغی این المفر
\\
بر فراز سرو سیمینش چو بخرامد به ناز
&&
چشم شورانگیز بین تا نجم بینی بر شجر
\\
یکره المحبوب وصلی انتهی عما نهی
&&
یرسم المنظور قتلی ارتضی فیما امر
\\
کاش اندک مایه نرمی در خطابش دیدمی
&&
ور مرا عشقش به سختی کشت سهلست این قدر
\\
قیل لی فی الحب اخطار و تحصیل المنی
&&
دوله القی بمن القی بروحی فی الخطر
\\
گوشه گیر ای یار یا جان در میان آور که عشق
&&
تیربارانست یا تسلیم باید یا حذر
\\
فالتنائی غصه ما ذاق الامن صبا
&&
و التدانی فرصه ما نال الا من صبر
\\
دختران طبع را یعنی سخن با این جمال
&&
آبرویی نیست پیش آن آن زیبا پسر
\\
لحظک القتال یغوی فی هلاکی لا تدع
&&
عطفک المیاس یسعی فی بلائی لا تذر
\\
آخر ای سرو روان بر ما گذر کن یک زمان
&&
آخر ای آرام جان در ما نظر کن یک نظر
\\
یا رخیم الجسم لو لا انت شخصی ما انحنی
&&
یا کحیل الطرف لو لا انت دمعی ما انحدر
\\
دوستی را گفتم اینک عمر شد گفت ای عجب
&&
طرفه می‌دارم که بی دلدار چون بردی به سر
\\
بعض خلانی اتانی سائلا عن قصتی
&&
قلت لا تسئل صفار الوجه یغنی عن خبر
\\
گفت سعدی صبر کن یا سیم و زر ده یا گریز
&&
عشق را یا مال باید یا صبوری یا سفر
\\
\end{longtable}
\end{center}
