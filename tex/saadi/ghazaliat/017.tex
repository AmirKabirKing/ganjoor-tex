\begin{center}
\section*{غزل ۱۷: چه کند بنده که گردن ننهد فرمان را}
\label{sec:017}
\addcontentsline{toc}{section}{\nameref{sec:017}}
\begin{longtable}{l p{0.5cm} r}
چه کند بنده که گردن ننهد فرمان را
&&
چه کند گوی که عاجز نشود چوگان را
\\
سروبالای کمان ابرو اگر تیر زند
&&
عاشق آنست که بر دیده نهد پیکان را
\\
دست من گیر که بیچارگی از حد بگذشت
&&
سر من دار که در پای تو ریزم جان را
\\
کاشکی پرده برافتادی از آن منظر حسن
&&
تا همه خلق ببینند نگارستان را
\\
همه را دیده در اوصاف تو حیران ماندی
&&
تا دگر عیب نگویند من حیران را
\\
لیکن آن نقش که در روی تو من می‌بینم
&&
همه را دیده نباشد که ببینند آن را
\\
چشم گریان مرا حال بگفتم به طبیب
&&
گفت یک بار ببوس آن دهن خندان را
\\
گفتم آیا که در این درد بخواهم مردن
&&
که محالست که حاصل کنم این درمان را
\\
پنجه با ساعد سیمین نه به عقل افکندم
&&
غایت جهل بود مشت زدن سندان را
\\
سعدی از سرزنش خلق نترسد هیهات
&&
غرقه در نیل چه اندیشه کند باران را
\\
سر بنه گر سر میدان ارادت داری
&&
ناگزیرست که گویی بود این میدان را
\\
\end{longtable}
\end{center}
