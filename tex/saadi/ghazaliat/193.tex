\begin{center}
\section*{غزل ۱۹۳: شورش بلبلان سحر باشد}
\label{sec:193}
\addcontentsline{toc}{section}{\nameref{sec:193}}
\begin{longtable}{l p{0.5cm} r}
شورش بلبلان سحر باشد
&&
خفته از صبح بی‌خبر باشد
\\
تیرباران عشق خوبان را
&&
دل شوریدگان سپر باشد
\\
عاشقان کشتگان معشوقند
&&
هر که زنده‌ست در خطر باشد
\\
همه عالم جمال طلعت اوست
&&
تا که را چشم این نظر باشد
\\
کس ندانم که دل بدو ندهد
&&
مگر آن کس که بی بصر باشد
\\
آدمی را که خارکی در پای
&&
نرود طرفه جانور باشد
\\
گو ترش روی باش و تلخ سخن
&&
زهر شیرین لبان شکر باشد
\\
عاقلان از بلا بپرهیزند
&&
مذهب عاشقان دگر باشد
\\
پای رفتن نماند سعدی را
&&
مرغ عاشق بریده پر باشد
\\
\end{longtable}
\end{center}
