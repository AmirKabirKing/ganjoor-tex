\begin{center}
\section*{غزل ۱۹۰: از این تعلق بیهوده تا به من چه رسد}
\label{sec:190}
\addcontentsline{toc}{section}{\nameref{sec:190}}
\begin{longtable}{l p{0.5cm} r}
از این تعلق بیهوده تا به من چه رسد
&&
وزان که خون دلم ریخت تا به تن چه رسد
\\
به گرد پای سمندش نمی‌رسد مشتاق
&&
که دستبوس کند تا بدان دهن چه رسد
\\
همه خطای منست این که می‌رود بر من
&&
ز دست خویشتنم تا به خویشتن چه رسد
\\
بیا که گر به گریبان جان رسد دستم
&&
ز شوق پاره کنم تا به پیرهن چه رسد
\\
که دید رنگ بهاری به رنگ رخسارت
&&
که آب گل ببرد تا به یاسمن چه رسد
\\
رقیب کیست که در ماجرای خلوت ما
&&
فرشته ره نبرد تا به اهرمن چه رسد
\\
ز هر نبات که حسنی و منظری دارد
&&
به سرو قامت آن نازنین بدن چه رسد
\\
چو خسرو از لب شیرین نمی‌برد مقصود
&&
قیاس کن که به فرهاد کوهکن چه رسد
\\
زکات لعل لبت را بسی طلبکارند
&&
میان این همه خواهندگان به من چه رسد
\\
رسید ناله سعدی به هر که در آفاق
&&
و گر عبیر نسوزد به انجمن چه رسد
\\
\end{longtable}
\end{center}
