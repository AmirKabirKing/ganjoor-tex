\begin{center}
\section*{غزل ۵۴۵: بخت آیینه ندارم که در او می‌نگری}
\label{sec:545}
\addcontentsline{toc}{section}{\nameref{sec:545}}
\begin{longtable}{l p{0.5cm} r}
بخت آیینه ندارم که در او می‌نگری
&&
خاک بازار نیرزم که بر او می‌گذری
\\
من چنان عاشق رویت که ز خود بی‌خبرم
&&
تو چنان فتنه خویشی که ز ما بی‌خبری
\\
به چه ماننده کنم در همه آفاق تو را
&&
کآنچه در وهم من آید تو از آن خوبتری
\\
برقع از پیش چنین روی نشاید برداشت
&&
که به هر گوشه چشمی دل خلقی ببری
\\
دیده‌ای را که به دیدار تو دل می‌نرود
&&
هیچ علت نتوان گفت به جز بی بصری
\\
گفتم از دست غمت سر به جهان در بنهم
&&
نتوانم که به هر جا بروم در نظری
\\
به فلک می‌رود آه سحر از سینه ما
&&
تو همی برنکنی دیده ز خواب سحری
\\
خفتگان را خبر از محنت بیداران نیست
&&
تا غمت پیش نیاید غم مردم نخوری
\\
هر چه در وصف تو گویند به نیکویی هست
&&
عیبت آن است که هر روز به طبعی دگری
\\
گر تو از پرده برون آیی و رخ بنمایی
&&
پرده بر کار همه پرده نشینان بدری
\\
عذر سعدی ننهد هر که تو را نشناسد
&&
حال دیوانه نداند که ندیده‌ست پری
\\
\end{longtable}
\end{center}
