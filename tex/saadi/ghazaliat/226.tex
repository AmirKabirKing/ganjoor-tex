\begin{center}
\section*{غزل ۲۲۶: درخت غنچه برآورد و بلبلان مستند}
\label{sec:226}
\addcontentsline{toc}{section}{\nameref{sec:226}}
\begin{longtable}{l p{0.5cm} r}
درخت غنچه برآورد و بلبلان مستند
&&
جهان جوان شد و یاران به عیش بنشستند
\\
حریف مجلس ما خود همیشه دل می‌برد
&&
علی الخصوص که پیرایه‌ای بر او بستند
\\
کسان که در رمضان چنگ می‌شکستندی
&&
نسیم گل بشنیدند و توبه بشکستند
\\
بساط سبزه لگدکوب شد به پای نشاط
&&
ز بس که عارف و عامی به رقص برجستند
\\
دو دوست قدر شناسند عهد صحبت را
&&
که مدتی ببریدند و بازپیوستند
\\
به در نمی‌رود از خانگه یکی هشیار
&&
که پیش شحنه بگوید که صوفیان مستند
\\
یکی درخت گل اندر فضای خلوت ماست
&&
که سروهای چمن پیش قامتش پستند
\\
اگر جهان همه دشمن شود به دولت دوست
&&
خبر ندارم از ایشان که در جهان هستند
\\
مثال راکب دریاست حال کشته عشق
&&
به ترک بار بگفتند و خویشتن رستند
\\
به سرو گفت کسی میوه‌ای نمی‌آری
&&
جواب داد که آزادگان تهی دستند
\\
به راه عقل برفتند سعدیا بسیار
&&
که ره به عالم دیوانگان ندانستند
\\
\end{longtable}
\end{center}
