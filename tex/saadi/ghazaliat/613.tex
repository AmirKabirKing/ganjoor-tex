\begin{center}
\section*{غزل ۶۱۳: ذوقی چنان ندارد بی دوست زندگانی}
\label{sec:613}
\addcontentsline{toc}{section}{\nameref{sec:613}}
\begin{longtable}{l p{0.5cm} r}
ذوقی چنان ندارد بی دوست زندگانی
&&
دودم به سر برآمد زین آتش نهانی
\\
شیراز در نبسته‌ست از کاروان ولیکن
&&
ما را نمی‌گشایند از قید مهربانی
\\
اشتر که اختیارش در دست خود نباشد
&&
می‌بایدش کشیدن باری به ناتوانی
\\
خون هزار وامق خوردی به دلفریبی
&&
دست از هزار عذرا بردی به دلستانی
\\
صورت نگار چینی بی خویشتن بماند
&&
گر صورتت ببیند سر تا به سر معانی
\\
ای بر در سرایت غوغای عشقبازان
&&
همچون بر آب شیرین آشوب کاروانی
\\
تو فارغی و عشقت بازیچه می‌نماید
&&
تا خرمنت نسوزد تشویش ما ندانی
\\
می‌گفتمت که جانی دیگر دریغم آید
&&
گر جوهری به از جان ممکن بود تو آنی
\\
سروی چو در سماعی بدری چو در حدیثی
&&
صبحی چو در کناری شمعی چو در میانی
\\
اول چنین نبودی باری حقیقتی شد
&&
دی حظ نفس بودی امروز قوت جانی
\\
شهر آن توست و شاهی فرمای هر چه خواهی
&&
گر بی عمل ببخشی ور بی‌گنه برانی
\\
روی امید سعدی بر خاک آستان است
&&
بعد از تو کس ندارد یا غایة الامانی
\\
\end{longtable}
\end{center}
