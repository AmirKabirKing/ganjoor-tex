\begin{center}
\section*{غزل ۵۱۱: هر کس به تماشایی رفتند به صحرایی}
\label{sec:511}
\addcontentsline{toc}{section}{\nameref{sec:511}}
\begin{longtable}{l p{0.5cm} r}
هر کس به تماشایی رفتند به صحرایی
&&
ما را که تو منظوری خاطر نرود جایی
\\
یا چشم نمی‌بیند یا راه نمی‌داند
&&
هر کاو به وجود خود دارد ز تو پروایی
\\
دیوانه عشقت را جایی نظر افتاده‌ست
&&
کآنجا نتواند رفت اندیشه دانایی
\\
امید تو بیرون برد از دل همه امیدی
&&
سودای تو خالی کرد از سر همه سودایی
\\
زیبا ننماید سرو اندر نظر عقلش
&&
آن کش نظری باشد با قامت زیبایی
\\
گویند رفیقانم در عشق چه سر داری
&&
گویم که سری دارم درباخته در پایی
\\
زنهار نمی‌خواهم کز کشتن امانم ده
&&
تا سیرترت بینم یک لحظه مدارایی
\\
در پارس که تا بوده‌ست از ولوله آسوده‌ست
&&
بیم است که برخیزد از حسن تو غوغایی
\\
من دست نخواهم برد الا به سر زلفت
&&
گر دسترسی باشد یک روز به یغمایی
\\
گویند تمنایی از دوست بکن سعدی
&&
جز دوست نخواهم کرد از دوست تمنایی
\\
\end{longtable}
\end{center}
