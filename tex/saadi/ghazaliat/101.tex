\begin{center}
\section*{غزل ۱۰۱: ای پیک پی خجسته که داری نشان دوست}
\label{sec:101}
\addcontentsline{toc}{section}{\nameref{sec:101}}
\begin{longtable}{l p{0.5cm} r}
ای پیک پی خجسته که داری نشان دوست
&&
با ما مگو به جز سخن دل نشان دوست
\\
حال از دهان دوست شنیدن چه خوش بود
&&
یا از دهان آن که شنید از دهان دوست
\\
ای یار آشنا علم کاروان کجاست
&&
تا سر نهیم بر قدم ساربان دوست
\\
گر زر فدای دوست کنند اهل روزگار
&&
ما سر فدای پای رسالت رسان دوست
\\
دردا و حسرتا که عنانم ز دست رفت
&&
دستم نمی‌رسد که بگیرم عنان دوست
\\
رنجور عشق دوست چنانم که هر که دید
&&
رحمت کند مگر دل نامهربان دوست
\\
گر دوست بنده را بکشد یا بپرورد
&&
تسلیم از آن بنده و فرمان از آن دوست
\\
گر آستین دوست بیفتد به دست من
&&
چندان که زنده‌ام سر من و آستان دوست
\\
بی حسرت از جهان نرود هیچ کس به در
&&
الا شهید عشق به تیر از کمان دوست
\\
بعد از تو هیچ در دل سعدی گذر نکرد
&&
وان کیست در جهان که بگیرد مکان دوست
\\
\end{longtable}
\end{center}
