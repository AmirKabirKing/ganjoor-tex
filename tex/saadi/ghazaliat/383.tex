\begin{center}
\section*{غزل ۳۸۳: می‌روم وز سر حسرت به قفا می‌نگرم}
\label{sec:383}
\addcontentsline{toc}{section}{\nameref{sec:383}}
\begin{longtable}{l p{0.5cm} r}
می‌روم وز سر حسرت به قفا می‌نگرم
&&
خبر از پای ندارم که زمین می‌سپرم
\\
می‌روم بی‌دل و بی یار و یقین می‌دانم
&&
که من بی‌دل بی یار نه مرد سفرم
\\
خاک من زنده به تأثیر هوای لب توست
&&
سازگاری نکند آب و هوای دگرم
\\
وه که گر بر سر کوی تو شبی روز کنم
&&
غلغل اندر ملکوت افتد از آه سحرم
\\
پای می‌پیچم و چون پای دلم می‌پیچد
&&
بار می‌بندم و از بار فروبسته‌ترم
\\
چه کنم دست ندارم به گریبان اجل
&&
تا به تن در ز غمت پیرهن جان بدرم
\\
آتش خشم تو برد آب من خاک آلود
&&
بعد از این باد به گوش تو رساند خبرم
\\
هر نوردی که ز طومار غمم باز کنی
&&
حرف‌ها بینی آلوده به خون جگرم
\\
نی مپندار که حرفی به زبان آرم اگر
&&
تا به سینه چو قلم بازشکافند سرم
\\
به هوای سر زلف تو درآویخته بود
&&
از سر شاخ زبان برگ سخن‌های ترم
\\
گر سخن گویم من بعد شکایت باشد
&&
ور شکایت کنم از دست تو پیش که برم
\\
خار سودای تو آویخته در دامن دل
&&
ننگم آید که به اطراف گلستان گذرم
\\
بصر روشنم از سرمه خاک در توست
&&
قیمت خاک تو من دانم کاهل بصرم
\\
گر چه در کلبه خلوت بودم نور حضور
&&
هم سفر به که نماندست مجال حضرم
\\
سرو بالای تو در باغ تصور برپای
&&
شرم دارم که به بالای صنوبر نگرم
\\
گر به تن بازکنم جای دگر باکی نیست
&&
که به دل غاشیه بر سر به رکاب تو درم
\\
گر به دوری سفر از تو جدا خواهم ماند
&&
شرم بادم که همان سعدی کوته نظرم
\\
به قدم رفتم و ناچار به سر بازآیم
&&
گر به دامن نرسد چنگ قضا و قدرم
\\
شوخ چشمی چو مگس کردم و برداشت عدو
&&
به مگسران ملامت ز کنار شکرم
\\
از قفا سیر نگشتم من بدبخت هنوز
&&
می‌روم وز سر حسرت به قفا می‌نگرم
\\
\end{longtable}
\end{center}
