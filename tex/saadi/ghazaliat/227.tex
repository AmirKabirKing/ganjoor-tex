\begin{center}
\section*{غزل ۲۲۷: آخر ای سنگ دل سیم زنخدان تا چند}
\label{sec:227}
\addcontentsline{toc}{section}{\nameref{sec:227}}
\begin{longtable}{l p{0.5cm} r}
آخر ای سنگدل سیم زنخدان تا چند
&&
تو ز ما فارغ و ما از تو پریشان تا چند
\\
خار در پای گل از دور به حسرت دیدن
&&
تشنه بازآمدن از چشمه حیوان تا چند
\\
گوش در گفتن شیرین تو واله تا کی
&&
چشم در منظر مطبوع تو حیران تا چند
\\
بیم آنست دمادم که برآرم فریاد
&&
صبر پیدا و جگر خوردن پنهان تا چند
\\
تو سر ناز برآری ز گریبان هر روز
&&
ما ز جورت سر فکرت به گریبان تا چند
\\
رنگ دستت نه به حناست که خون دل ماست
&&
خوردن خون دل خلق به دستان تا چند
\\
سعدی از دست تو از پای درآید روزی
&&
طاقت بار ستم تا کی و هجران تا چند
\\
\end{longtable}
\end{center}
