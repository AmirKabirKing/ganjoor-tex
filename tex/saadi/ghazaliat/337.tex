\begin{center}
\section*{غزل ۳۳۷: گر یکی از عشق برآرد خروش}
\label{sec:337}
\addcontentsline{toc}{section}{\nameref{sec:337}}
\begin{longtable}{l p{0.5cm} r}
گر یکی از عشق برآرد خروش
&&
بر سر آتش نه غریب است جوش
\\
پیرهنی گر بدرد ز اشتیاق
&&
دامن عفوش به گنه بربپوش
\\
بوی گل آورد نسیم صبا
&&
بلبل بیدل ننشیند خموش
\\
مطرب اگر پرده از این ره زند
&&
بازنیایند حریفان به هوش
\\
ساقی اگر باده از این خم دهد
&&
خرقه صوفی ببرد می فروش
\\
زهر بیاور که ز اجزای من
&&
بانگ برآید به ارادت که نوش
\\
از تو نپرسند درازای شب
&&
آن کس داند که نخفته‌ست دوش
\\
حیف بود مردن بی عاشقی
&&
تا نفسی داری و نفسی بکوش
\\
سر که نه در راه عزیزان رود
&&
بار گران است کشیدن به دوش
\\
سعدی اگر خاک شود همچنان
&&
ناله زاریدنش آید به گوش
\\
هر که دلی دارد از انفاس او
&&
می‌شنود تا به قیامت خروش
\\
\end{longtable}
\end{center}
