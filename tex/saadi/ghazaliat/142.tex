\begin{center}
\section*{غزل ۱۴۲: ای کسوت زیبایی بر قامت چالاکت}
\label{sec:142}
\addcontentsline{toc}{section}{\nameref{sec:142}}
\begin{longtable}{l p{0.5cm} r}
ای کسوت زیبایی بر قامت چالاکت
&&
زیبا نتواند دید الا نظر پاکت
\\
گر منزلتی دارم بر خاک درت میرم
&&
باشد که گذر باشد یک روز بر آن خاکت
\\
دانم که سرم روزی در پای تو خواهد شد
&&
هم در تو گریزندم دست من و فتراکت
\\
ای چشم خرد حیران در منظر مطبوعت
&&
وی دست نظر کوتاه از دامن ادراکت
\\
گفتم که نیاویزم با مار سر زلفت
&&
بیچاره فروماندم پیش لب ضحاکت
\\
مه روی بپوشاند خورشید خجل ماند
&&
گر پرتو روی افتد بر طارم افلاکت
\\
گر جمله ببخشایی فضلست بر اصحابت
&&
ور جمله بسوزانی حکمست بر املاکت
\\
خون همه کس ریزی از کس نبود بیمت
&&
جرم همه کس بخشی از کس نبود باکت
\\
چندان که جفا خواهی می‌کن که نمی‌گردد
&&
غم گرد دل سعدی با یاد طربناکت
\\
\end{longtable}
\end{center}
