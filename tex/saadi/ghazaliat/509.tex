\begin{center}
\section*{غزل ۵۰۹: من ندانستم از اول که تو بی مهر و وفایی}
\label{sec:509}
\addcontentsline{toc}{section}{\nameref{sec:509}}
\begin{longtable}{l p{0.5cm} r}
من ندانستم از اول که تو بی مهر و وفایی
&&
عهد نابستن از آن به که ببندی و نپایی
\\
دوستان عیب کنندم که چرا دل به تو دادم
&&
باید اول به تو گفتن که چنین خوب چرایی
\\
ای که گفتی مرو اندر پی خوبان زمانه
&&
ما کجاییم در این بحر تفکر تو کجایی
\\
آن نه خالست و زنخدان و سر زلف پریشان
&&
که دل اهل نظر برد که سریست خدایی
\\
پرده بردار که بیگانه خود این روی نبیند
&&
تو بزرگی و در آیینه کوچک ننمایی
\\
حلقه بر در نتوانم زدن از دست رقیبان
&&
این توانم که بیایم به محلت به گدایی
\\
عشق و درویشی و انگشت نمایی و ملامت
&&
همه سهلست تحمل نکنم بار جدایی
\\
روز صحرا و سماعست و لب جوی و تماشا
&&
در همه شهر دلی نیست که دیگر بربایی
\\
گفته بودم چو بیایی غم دل با تو بگویم
&&
چه بگویم که غم از دل برود چون تو بیایی
\\
شمع را باید از این خانه به دربردن و کشتن
&&
تا به همسایه نگوید که تو در خانه مایی
\\
سعدی آن نیست که هرگز ز کمندت بگریزد
&&
که بدانست که دربند تو خوشتر که رهایی
\\
خلق گویند برو دل به هوای دگری ده
&&
نکنم خاصه در ایام اتابک دو هوایی
\\
\end{longtable}
\end{center}
