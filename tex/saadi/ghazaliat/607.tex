\begin{center}
\section*{غزل ۶۰۷: من چرا دل به تو دادم که دلم می‌شکنی}
\label{sec:607}
\addcontentsline{toc}{section}{\nameref{sec:607}}
\begin{longtable}{l p{0.5cm} r}
من چرا دل به تو دادم که دلم می‌شکنی
&&
یا چه کردم که نگه باز به من می‌نکنی
\\
دل و جانم به تو مشغول و نظر در چپ و راست
&&
تا ندانند حریفان که تو منظور منی
\\
دیگران چون بروند از نظر از دل بروند
&&
تو چنان در دل من رفته که جان در بدنی
\\
تو همایی و من خسته بیچاره گدای
&&
پادشاهی کنم ار سایه به من برفکنی
\\
بنده وارت به سلام آیم و خدمت بکنم
&&
ور جوابم ندهی می‌رسدت کبر و منی
\\
مرد راضیست که در پای تو افتد چون گوی
&&
تا بدان ساعد سیمینش به چوگان بزنی
\\
مست بی خویشتن از خمر ظلوم است و جهول
&&
مستی از عشق نکو باشد و بی خویشتنی
\\
تو بدین نعت و صفت گر بخرامی در باغ
&&
باغبان بیند و گوید که تو سرو چمنی
\\
من بر از شاخ امیدت نتوانم خوردن
&&
غالب الظن و یقینم که تو بیخم بکنی
\\
خوان درویش به شیرینی و چربی بخورند
&&
سعدیا چرب زبانی کن و شیرین سخنی
\\
\end{longtable}
\end{center}
