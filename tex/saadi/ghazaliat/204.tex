\begin{center}
\section*{غزل ۲۰۴: گر گویمت که سروی سرو این چنین نباشد}
\label{sec:204}
\addcontentsline{toc}{section}{\nameref{sec:204}}
\begin{longtable}{l p{0.5cm} r}
گر گویمت که سروی سرو این چنین نباشد
&&
ور گویمت که ماهی مه بر زمین نباشد
\\
گر در جهان بگردی و آفاق درنوردی
&&
صورت بدین شگرفی در کفر و دین نباشد
\\
لعلست یا لبانت قندست یا دهانت
&&
تا در برت نگیرم نیکم یقین نباشد
\\
صورت کنند زیبا بر پرنیان و دیبا
&&
لیکن بر ابروانش سحر مبین نباشد
\\
زنبور اگر میانش باشد بدین لطیفی
&&
حقا که در دهانش این انگبین نباشد
\\
گر هر که در جهان را شاید که خون بریزی
&&
با یار مهربانت باید که کین نباشد
\\
گر جان نازنینش در پای ریزی ای دل
&&
در کار نازنینان جان نازنین نباشد
\\
ور زان که دیگری را بر ما همی‌گزیند
&&
گو برگزین که ما را بر تو گزین نباشد
\\
عشقش حرام بادا بر یار سروبالا
&&
تردامنی که جانش در آستین نباشد
\\
سعدی به هیچ علت روی از تو برنپیچد
&&
الا گرش برانی علت جز این نباشد
\\
\end{longtable}
\end{center}
