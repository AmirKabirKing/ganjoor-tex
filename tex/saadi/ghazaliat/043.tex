\begin{center}
\section*{غزل ۴۳: اگر مراد تو ای دوست بی مرادی ماست}
\label{sec:043}
\addcontentsline{toc}{section}{\nameref{sec:043}}
\begin{longtable}{l p{0.5cm} r}
اگر مراد تو ای دوست بی مرادی ماست
&&
مراد خویش دگرباره من نخواهم خواست
\\
اگر قبول کنی ور برانی از بر خویش
&&
خلاف رای تو کردن خلاف مذهب ماست
\\
میان عیب و هنر پیش دوستان کریم
&&
تفاوتی نکند چون نظر به عین رضاست
\\
عنایتی که تو را بود اگر مبدل شد
&&
خلل پذیر نباشد ارادتی که مراست
\\
مرا به هر چه کنی دل نخواهی آزردن
&&
که هر چه دوست پسندد به جای دوست رواست
\\
اگر عداوت و جنگ است در میان عرب
&&
میان لیلی و مجنون محبت است و صفاست
\\
هزار دشمنی افتد به قول بدگویان
&&
میان عاشق و معشوق دوستی برجاست
\\
غلام قامت آن لعبت قباپوشم
&&
که در محبت رویش هزار جامه قباست
\\
نمی‌توانم بی او نشست یک ساعت
&&
چرا که از سر جان بر نمی‌توانم خاست
\\
جمال در نظر و شوق همچنان باقی
&&
گدا اگر همه عالم بدو دهند گداست
\\
مرا به عشق تو اندیشه از ملامت نیست
&&
وگر کنند ملامت نه بر من تنهاست
\\
هر آدمی که چنین شخص دلستان بیند
&&
ضرورت است که گوید به سرو ماند راست
\\
به روی خوبان گفتی نظر خطا باشد
&&
خطا نباشد دیگر مگو چنین که خطاست
\\
خوش است با غم هجران دوست سعدی را
&&
که گر چه رنج به جان می‌رسد امید دواست
\\
بلا و زحمت امروز بر دل درویش
&&
از آن خوش است که امید رحمت فرداست
\\
\end{longtable}
\end{center}
