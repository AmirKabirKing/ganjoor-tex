\begin{center}
\section*{غزل ۴۴۸: خوشا و خرما وقت حبیبان}
\label{sec:448}
\addcontentsline{toc}{section}{\nameref{sec:448}}
\begin{longtable}{l p{0.5cm} r}
خوشا و خرما وقت حبیبان
&&
به بوی صبح و بانگ عندلیبان
\\
خوش آن ساعت نشیند دوست با دوست
&&
که ساکن گردد آشوب رقیبان
\\
دو تن در جامه‌ای چون پسته در پوست
&&
برآورده دو سر از یک گریبان
\\
سزای دشمنان این بس که بینند
&&
حبیبان روی در روی حبیبان
\\
نصیب از عمر دنیا نقد وقت است
&&
مباش ای هوشمند از بی نصیبان
\\
چو دانی کز تو چوپانی نیاید
&&
رها کن گوسفندان را به ذئبان
\\
من این رندان و مستان دوست دارم
&&
خلاف پارسایان و خطیبان
\\
بهل تا در حق من هر چه خواهند
&&
بگویند آشنایان و غریبان
\\
لب شیرین لبان را خصلتی هست
&&
که غارت می‌کند هوش لبیبان
\\
نشستم با جوانمردان اوباش
&&
بشستم هر چه خواندم بر ادیبان
\\
که می‌داند دوای درد سعدی
&&
که رنجورند از این علت طبیبان
\\
\end{longtable}
\end{center}
