\begin{center}
\section*{غزل ۳۵۳: رفیق مهربان و یار همدم}
\label{sec:353}
\addcontentsline{toc}{section}{\nameref{sec:353}}
\begin{longtable}{l p{0.5cm} r}
رفیق مهربان و یار همدم
&&
همه کس دوست می‌دارند و من هم
\\
نظر با نیکوان رسمیست معهود
&&
نه این بدعت من آوردم به عالم
\\
تو گر دعوی کنی پرهیزگاری
&&
مصدق دارمت والله اعلم
\\
و گر گویی که میل خاطرم نیست
&&
من این دعوی نمی‌دارم مسلم
\\
حدیث عشق اگر گویی گناه است
&&
گناه اول ز حوا بود و آدم
\\
گرفتار کمند ماه رویان
&&
نه از مدحش خبر باشد نه از ذم
\\
چو دست مهربان بر سینه ریش
&&
به گیتی در ندارم هیچ مرهم
\\
بگردان ساقیا جام لبالب
&&
بیاموز از فلک دور دمادم
\\
اگر دانی که دنیا غم نیرزد
&&
به روی دوستان خوش باش و خرم
\\
غنیمت دان اگر دانی که هر روز
&&
ز عمر مانده روزی می‌شود کم
\\
منه دل بر سرای عمر سعدی
&&
که بنیادش نه بنیادیست محکم
\\
برو شادی کن ای یار دل افروز
&&
چو خاکت می‌خورد چندین مخور غم
\\
\end{longtable}
\end{center}
