\begin{center}
\section*{غزل ۵۳۹: نگارا وقت آن آمد که دل با مهر پیوندی}
\label{sec:539}
\addcontentsline{toc}{section}{\nameref{sec:539}}
\begin{longtable}{l p{0.5cm} r}
نگارا وقت آن آمد که دل با مهر پیوندی
&&
که ما را بیش از این طاقت نمانده‌ست آرزومندی
\\
غریب از خوی مطبوعت که روی از بندگان پوشی
&&
بدیع از طبع موزونت که در بر دوستان بندی
\\
تو خرسند و شکیبایی چنینت در خیال آید
&&
که ما را همچنین باشد شکیبایی و خرسندی
\\
نگفتی بی‌وفا یارا که از ما نگسلی هرگز
&&
مگر در دل چنین بودت که خود با ما نپیوندی
\\
زهی آسایش و رحمت نظر را کش تو منظوری
&&
زهی بخشایش و دولت پدر را کش تو فرزندی
\\
شکار آن گه توان کشتن که محکم در کمند آید
&&
چو بیخ مهر بنشاندم درخت وصل برکندی
\\
نمودی چند بار از خود که حافظ عهد و پیمانم
&&
کنونت بازدانستم که ناقض عهد و سوگندی
\\
مرا زین پیش در خلوت فراغت بود و جمعیت
&&
تو در جمع آمدی ناگاه و مجموعان پراکندی
\\
گرت جان در قدم ریزم هنوزت عذر می‌خواهم
&&
که از من خدمتی ناید چنان لایق که بپسندی
\\
ترش بنشین و تیزی کن که ما را تلخ ننماید
&&
چه می‌گویی چنین شیرین که شوری در من افکندی
\\
شکایت گفتن سعدی مگر باد است نزدیکت
&&
که او چون رعد می‌نالد تو همچون برق می‌خندی
\\
\end{longtable}
\end{center}
