\begin{center}
\section*{غزل ۲۰۵: اگر سروی به بالای تو باشد}
\label{sec:205}
\addcontentsline{toc}{section}{\nameref{sec:205}}
\begin{longtable}{l p{0.5cm} r}
اگر سروی به بالای تو باشد
&&
نه چون بشن دلارای تو باشد
\\
و گر خورشید در مجلس نشیند
&&
نپندارم که همتای تو باشد
\\
و گر دوران ز سر گیرند هیهات
&&
که مولودی به سیمای تو باشد
\\
که دارد در همه لشکر کمانی
&&
که چون ابروی زیبای تو باشد
\\
مبادا ور بود غارت در اسلام
&&
همه شیراز یغمای تو باشد
\\
برای خود نشاید در تو پیوست
&&
همی‌سازیم تا رای تو باشد
\\
دو عالم را به یک بار از دل تنگ
&&
برون کردیم تا جای تو باشد
\\
یک امروزست ما را نقد ایام
&&
مرا کی صبر فردای تو باشد
\\
خوشست اندر سر دیوانه سودا
&&
به شرط آن که سودای تو باشد
\\
سر سعدی چو خواهد رفتن از دست
&&
همان بهتر که در پای تو باشد
\\
\end{longtable}
\end{center}
