\begin{center}
\section*{غزل ۴۲۳: ز دستم بر نمی‌خیزد که یک دم بی تو بنشینم}
\label{sec:423}
\addcontentsline{toc}{section}{\nameref{sec:423}}
\begin{longtable}{l p{0.5cm} r}
ز دستم بر نمی‌خیزد که یک دم بی تو بنشینم
&&
به جز رویت نمی‌خواهم که روی هیچ کس بینم
\\
من اول روز دانستم که با شیرین درافتادم
&&
که چون فرهاد باید شست دست از جان شیرینم
\\
تو را من دوست می‌دارم خلاف هر که در عالم
&&
اگر طعنه است در عقلم اگر رخنه است در دینم
\\
و گر شمشیر برگیری سپر پیشت بیندازم
&&
که بی شمشیر خود کشتی به ساعدهای سیمینم
\\
برآی ای صبح مشتاقان اگر نزدیک روز آمد
&&
که بگرفت این شب یلدا ملال از ماه و پروینم
\\
ز اول هستی آوردم قفای نیستی خوردم
&&
کنون امید بخشایش همی‌دارم که مسکینم
\\
دلی چون شمع می‌باید که بر جانم ببخشاید
&&
که جز وی کس نمی‌بینم که می‌سوزد به بالینم
\\
تو همچون گل ز خندیدن لبت با هم نمی‌آید
&&
روا داری که من بلبل چو بوتیمار بنشینم
\\
رقیب انگشت می‌خاید که سعدی چشم بر هم نه
&&
مترس ای باغبان از گل که می‌بینم نمی‌چینم
\\
\end{longtable}
\end{center}
