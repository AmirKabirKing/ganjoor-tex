\begin{center}
\section*{غزل ۲۸۸: که برگذشت که بوی عبیر می‌آید}
\label{sec:288}
\addcontentsline{toc}{section}{\nameref{sec:288}}
\begin{longtable}{l p{0.5cm} r}
که برگذشت که بوی عبیر می‌آید
&&
که می‌رود که چنین دلپذیر می‌آید
\\
نشان یوسف گم کرده می‌دهد یعقوب
&&
مگر ز مصر به کنعان بشیر می‌آید
\\
ز دست رفتم و بی دیدگان نمی‌دانند
&&
که زخم‌های نظر بر بصیر می‌آید
\\
همی‌خرامد و عقلم به طبع می‌گوید
&&
نظر بدوز که آن بی‌نظیر می‌آید
\\
جمال کعبه چنان می‌دواندم به نشاط
&&
که خارهای مغیلان حریر می‌آید
\\
نه آن چنان به تو مشغولم ای بهشتی رو
&&
که یاد خویشتنم در ضمیر می‌آید
\\
ز دیدنت نتوانم که دیده دربندم
&&
و گر مقابله بینم که تیر می‌آید
\\
هزار جامه معنی که من براندازم
&&
به قامتی که تو داری قصیر می‌آید
\\
به کشتن آمده بود آن که مدعی پنداشت
&&
که رحمتی مگرش بر اسیر می‌آید
\\
رسید ناله سعدی به هر که در آفاق
&&
هم آتشی زده‌ای تا نفیر می‌آید
\\
\end{longtable}
\end{center}
