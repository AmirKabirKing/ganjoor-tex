\begin{center}
\section*{غزل ۲۲۴: گلبنان پیرایه بر خود کرده‌اند}
\label{sec:224}
\addcontentsline{toc}{section}{\nameref{sec:224}}
\begin{longtable}{l p{0.5cm} r}
گلبنان پیرایه بر خود کرده‌اند
&&
بلبلان را در سماع آورده‌اند
\\
ساقیان لاابالی در طواف
&&
هوش میخواران مجلس برده‌اند
\\
جرعه‌ای خوردیم و کار از دست رفت
&&
تا چه بی هوشانه در می کرده‌اند
\\
ما به یک شربت چنین بیخود شدیم
&&
دیگران چندین قدح چون خورده‌اند
\\
آتش اندر پختگان افتاد و سوخت
&&
خام طبعان همچنان افسرده‌اند
\\
خیمه بیرون بر که فراشان باد
&&
فرش دیبا در چمن گسترده‌اند
\\
زندگانی چیست مردن پیش دوست
&&
کاین گروه زندگان دل مرده‌اند
\\
تا جهان بودست جماشان گل
&&
از سلحداران خار آزرده‌اند
\\
عاشقان را کشته می‌بینند خلق
&&
بشنو از سعدی که جان پرورده‌اند
\\
\end{longtable}
\end{center}
