\begin{center}
\section*{غزل ۷۹: این باد بهار بوستانست}
\label{sec:079}
\addcontentsline{toc}{section}{\nameref{sec:079}}
\begin{longtable}{l p{0.5cm} r}
این باد بهار بوستان است
&&
یا بوی وصال دوستان است
\\
دل می‌برد این خط نگارین
&&
گویی خط روی دلستان است
\\
ای مرغ به دام دل گرفتار
&&
بازآی که وقت آشیان است
\\
شب‌ها من و شمع می‌گدازیم
&&
این است که سوز من نهان است
\\
گوشم همه روز از انتظارت
&&
بر راه و نظر بر آستان است
\\
ور بانگ مؤذنی میاید
&&
گویم که درای کاروان است
\\
با آن همه دشمنی که کردی
&&
بازآی که دوستی همان است
\\
با قوت بازوان عشقت
&&
سرپنجهٔ صبر ناتوان است
\\
بیزاری دوستان دمساز
&&
تفریق میان جسم و جان است
\\
نالیدن دردناک سعدی
&&
بر دعوی دوستی بیان است
\\
آتش به نی قلم درانداخت
&&
وین حبر که می‌رود دخان است
\\
\end{longtable}
\end{center}
