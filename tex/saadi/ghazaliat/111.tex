\begin{center}
\section*{غزل ۱۱۱: مشنو ای دوست که غیر از تو مرا یاری هست}
\label{sec:111}
\addcontentsline{toc}{section}{\nameref{sec:111}}
\begin{longtable}{l p{0.5cm} r}
مشنو ای دوست که غیر از تو مرا یاری هست
&&
یا شب و روز به جز فکر توام کاری هست
\\
به کمند سر زلفت نه من افتادم و بس
&&
که به هر حلقه موییت گرفتاری هست
\\
گر بگویم که مرا با تو سر و کاری نیست
&&
در و دیوار گواهی بدهد کاری هست
\\
هر که عیبم کند از عشق و ملامت گوید
&&
تا ندیدست تو را بر منش انکاری هست
\\
صبر بر جور رقیبت چه کنم گر نکنم
&&
همه دانند که در صحبت گل خاری هست
\\
نه من خام طمع عشق تو می‌ورزم و بس
&&
که چو من سوخته در خیل تو بسیاری هست
\\
باد خاکی ز مقام تو بیاورد و ببرد
&&
آب هر طیب که در کلبه عطاری هست
\\
من چه در پای تو ریزم که پسند تو بود
&&
جان و سر را نتوان گفت که مقداری هست
\\
من از این دلق مرقع به درآیم روزی
&&
تا همه خلق بدانند که زناری هست
\\
همه را هست همین داغ محبت که مراست
&&
که نه مستم من و در دور تو هشیاری هست
\\
عشق سعدی نه حدیثیست که پنهان ماند
&&
داستانیست که بر هر سر بازاری هست
\\
\end{longtable}
\end{center}
