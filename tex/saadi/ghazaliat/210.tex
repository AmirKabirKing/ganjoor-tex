\begin{center}
\section*{غزل ۲۱۰: خواب خوش من ای پسر دستخوش خیال شد}
\label{sec:210}
\addcontentsline{toc}{section}{\nameref{sec:210}}
\begin{longtable}{l p{0.5cm} r}
خواب خوش من ای پسر دستخوش خیال شد
&&
نقد امید عمر من در طلب وصال شد
\\
گر نشد اشتیاق او غالب صبر و عقل من
&&
این به چه زیردست گشت آن به چه پایمال شد
\\
بر من اگر حرام شد وصل تو نیست بوالعجب
&&
بوالعجب آن که خون من بر تو چرا حلال شد
\\
پرتو آفتاب اگر بدر کند هلال را
&&
بدر وجود من چرا در نظرت هلال شد
\\
زیبد اگر طلب کند عزت ملک مصر دل
&&
آن که هزار یوسفش بنده جاه و مال شد
\\
طرفه مدار اگر ز دل نعره بیخودی زنم
&&
کآتش دل چو شعله زد صبر در او محال شد
\\
سعدی اگر نظر کند تا نه غلط گمان بری
&&
کاو نه به رسم دیگران بنده زلف و خال شد
\\
\end{longtable}
\end{center}
