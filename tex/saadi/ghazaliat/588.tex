\begin{center}
\section*{غزل ۵۸۸: عمرم به آخر آمد عشقم هنوز باقی}
\label{sec:588}
\addcontentsline{toc}{section}{\nameref{sec:588}}
\begin{longtable}{l p{0.5cm} r}
عمرم به آخر آمد عشقم هنوز باقی
&&
وز می چنان نه مستم کز عشق روی ساقی
\\
یا غایة الامانی قلبی لدیک فانی
&&
شخصی کما ترانی من غایة اشتیاقی
\\
ای دردمند مفتون بر خد و خال موزون
&&
قدر وصالش اکنون دانی که در فراقی
\\
یا سعد کیف صرنا فی بلدة هجرنا
&&
من بعد ما سهرنا و الایدی فی العناقی
\\
بعد از عراق جایی خوش نایدم هوایی
&&
مطرب بزن نوایی زان پرده عراقی
\\
خان الزمان عهدی حتی بقیت وحدی
&&
ردوا علی ودی بالله یا رفاقی
\\
در سرو و مه چه گویی ای مجمع نکویی
&&
تو ماه مشکبویی تو سرو سیم ساقی
\\
ان مت فی هواها دعنی امت فداها
&&
یا عاذلی نباها ذرنی و ما الاقی
\\
چند از حدیث آنان خیزید ای جوانان
&&
تا در هوای جانان بازیم عمر باقی
\\
قام الغیاث لما زم الجمال زما
&&
و اللیل مدلهما و الدمع فی المآقی
\\
تا در میان نیاری بیگانه‌ای نه یاری
&&
درباز هر چه داری گر مرد اتفاقی
\\
\end{longtable}
\end{center}
