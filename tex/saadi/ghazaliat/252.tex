\begin{center}
\section*{غزل ۲۵۲: نشاید که خوبان به صحرا روند}
\label{sec:252}
\addcontentsline{toc}{section}{\nameref{sec:252}}
\begin{longtable}{l p{0.5cm} r}
نشاید که خوبان به صحرا روند
&&
همه کس شناسند و هر جا روند
\\
حلالست رفتن به صحرا ولیک
&&
نه انصاف باشد که بی ما روند
\\
نباید دل از دست مردم ربود
&&
چو خواهند جایی که تنها روند
\\
که بپسندد از باغبانان گل
&&
که از بانگ بلبل به سودا روند
\\
برآرند فریاد عشق از ختا
&&
گر این شوخ چشمان به یغما روند
\\
همه سروها را بباید خمید
&&
که در پای آن سروبالا روند
\\
بسا هوشمندا که در کوی عشق
&&
چو من عاقل آیند و شیدا روند
\\
بسازیم بر آسمان سلمی
&&
اگر شاهدان بر ثریا روند
\\
نه سعدی در این گل فرورفت و بس
&&
که آنان که بر روی دریا روند
\\
\end{longtable}
\end{center}
