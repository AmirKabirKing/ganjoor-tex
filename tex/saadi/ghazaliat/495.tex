\begin{center}
\section*{غزل ۴۹۵: می‌برزند ز مشرق شمع فلک زبانه}
\label{sec:495}
\addcontentsline{toc}{section}{\nameref{sec:495}}
\begin{longtable}{l p{0.5cm} r}
می‌برزند ز مشرق شمع فلک زبانه
&&
ای ساقی صبوحی درده می شبانه
\\
عقلم بدزد لختی چند اختیار دانش
&&
هوشم ببر زمانی تا کی غم زمانه
\\
گر سنگ فتنه بارد فرق منش سپر کن
&&
ور تیر طعنه آید جان منش نشانه
\\
گر می به جان دهندت بستان که پیش دانا
&&
ز آب حیات بهتر خاک شرابخانه
\\
آن کوزه بر کفم نه کآب حیات دارد
&&
هم طعم نار دارد هم رنگ ناردانه
\\
صوفی چگونه گردد گرد شراب صافی
&&
گنجشک را نگنجد عنقا در آشیانه
\\
دیوانگان نترسند از صولت قیامت
&&
بشکیبد اسب چوبین از سیف و تازیانه
\\
صوفی و کنج خلوت سعدی و طرف صحرا
&&
صاحب هنر نگیرد بر بی هنر بهانه
\\
\end{longtable}
\end{center}
