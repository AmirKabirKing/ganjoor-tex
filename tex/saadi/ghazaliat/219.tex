\begin{center}
\section*{غزل ۲۱۹: کسی که روی تو دیدست حال من داند}
\label{sec:219}
\addcontentsline{toc}{section}{\nameref{sec:219}}
\begin{longtable}{l p{0.5cm} r}
کسی که روی تو دیده‌ست حال من داند
&&
که هر که دل به تو پرداخت صبر نتواند
\\
مگر تو روی بپوشی و گر نه ممکن نیست
&&
که آدمی که تو بیند نظر بپوشاند
\\
هر آفریده که چشمش بر آن جمال افتاد
&&
دلش ببخشد و بر جانت آفرین خواند
\\
اگر به دست کند باغبان چنین سروی
&&
چه جای چشمه که بر چشم‌هات بنشاند
\\
چه روزها به شب آورد جان منتظرم
&&
به بوی آن که شبی با تو روز گرداند
\\
به چند حیله شبی در فراق روز کنم
&&
و گر نبینمت آن روز هم به شب ماند
\\
جفا و سلطنتت می‌رسد ولی مپسند
&&
که گر سوار براند پیاده درماند
\\
به دست رحمتم از خاک آستان بردار
&&
که گر بیفکنیم کس به هیچ نستاند
\\
چه حاجت است به شمشیر قتل عاشق را
&&
حدیث دوست بگویش که جان برافشاند
\\
پیام اهل دل است این خبر که سعدی داد
&&
نه هر که گوش کند معنی سخن داند
\\
\end{longtable}
\end{center}
