\begin{center}
\section*{غزل ۳۷۵: چنان در قید مهرت پای بندم}
\label{sec:375}
\addcontentsline{toc}{section}{\nameref{sec:375}}
\begin{longtable}{l p{0.5cm} r}
چنان در قید مهرت پای بندم
&&
که گویی آهوی سر در کمندم
\\
گهی بر درد بی درمان بگریم
&&
گهی بر حال بی سامان بخندم
\\
مرا هوشی نماند از عشق و گوشی
&&
که پند هوشمندان کار بندم
\\
مجال صبر تنگ آمد به یک بار
&&
حدیث عشق بر صحرا فکندم
\\
نه مجنونم که دل بردارم از دوست
&&
مده گر عاقلی ای خواجه پندم
\\
چنین صورت نبندد هیچ نقاش
&&
معاذالله من این صورت نبندم
\\
چه جان‌ها در غمت فرسود و تن‌ها
&&
نه تنها من اسیر و مستمندم
\\
تو هم بازآمدی ناچار و ناکام
&&
اگر بازآمدی بخت بلندم
\\
گر آوازم دهی من خفته در گور
&&
برآساید روان دردمندم
\\
سری دارم فدای خاک پایت
&&
گر آسایش رسانی ور گزندم
\\
و گر در رنج سعدی راحت توست
&&
من این بیداد بر خود می‌پسندم
\\
\end{longtable}
\end{center}
