\begin{center}
\section*{غزل ۲۷۵: نگفتم روزه بسیاری نپاید}
\label{sec:275}
\addcontentsline{toc}{section}{\nameref{sec:275}}
\begin{longtable}{l p{0.5cm} r}
نگفتم روزه بسیاری نپاید
&&
ریاضت بگذرد سختی سر آید
\\
پس از دشواری آسانیست ناچار
&&
ولیکن آدمی را صبر باید
\\
رخ از ما تا به کی پنهان کند عید
&&
هلال آنک به ابرو می‌نماید
\\
سرابستان در این موسم چه بندی
&&
درش بگشای تا دل برگشاید
\\
غلامان را بگو تا عود سوزند
&&
کنیزک را بگو تا مشک ساید
\\
که پندارم نگار سروبالا
&&
در این دم تهنیت گویان درآید
\\
سواران حلقه بربودند و آن شوخ
&&
هنوز از حلقه‌ها دل می‌رباید
\\
چو یار اندر حدیث آید به مجلس
&&
مغنی را بگو تا کم سراید
\\
که شعر اندر چنین مجلس نگنجد
&&
بلی گر گفته سعدیست شاید
\\
\end{longtable}
\end{center}
