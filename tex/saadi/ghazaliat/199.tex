\begin{center}
\section*{غزل ۱۹۹: تا حال منت خبر نباشد}
\label{sec:199}
\addcontentsline{toc}{section}{\nameref{sec:199}}
\begin{longtable}{l p{0.5cm} r}
تا حال منت خبر نباشد
&&
در کار منت نظر نباشد
\\
تا قوت صبر بود کردیم
&&
دیگر چه کنیم اگر نباشد
\\
آیین وفا و مهربانی
&&
در شهر شما مگر نباشد
\\
گویند نظر چرا نبستی
&&
تا مشغله و خطر نباشد
\\
ای خواجه برو که جهد انسان
&&
با تیر قضا سپر نباشد
\\
این شور که در سر است ما را
&&
وقتی برود که سر نباشد
\\
بیچاره کجا رود گرفتار
&&
کز کوی تو ره به در نباشد
\\
چون روی تو دلفریب و دلبند
&&
در روی زمین دگر نباشد
\\
در پارس چنین نمک ندیدم
&&
در مصر چنین شکر نباشد
\\
گر حکم کنی به جان سعدی
&&
جان از تو عزیزتر نباشد
\\
\end{longtable}
\end{center}
