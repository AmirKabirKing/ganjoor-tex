\begin{center}
\section*{غزل ۴۹۶: ای صورتت ز گوهر معنی خزینه‌ای}
\label{sec:496}
\addcontentsline{toc}{section}{\nameref{sec:496}}
\begin{longtable}{l p{0.5cm} r}
ای صورتت ز گوهر معنی خزینه‌ای
&&
ما را ز داغ عشق تو در دل دفینه‌ای
\\
دانی که آه سوختگان را اثر بود
&&
مگذار ناله‌ای که برآید ز سینه‌ای
\\
زیور همان دو رشته مرجان کفایت است
&&
وز موی در کنار و برت عنبرینه‌ای
\\
سر در نیاورم به سلاطین روزگار
&&
گر من ز بندگان تو باشم کمینه‌ای
\\
چشمی که جز به روی تو بر می‌کنم خطاست
&&
وآن دم که بی تو می‌گذرانم غبینه‌ای
\\
تدبیر نیست جز سپر انداختن که خصم
&&
سنگی به دست دارد و ما آبگینه‌ای
\\
وآن را روا بود که زند لاف مهر دوست
&&
کز دل به در کند همه مهری و کینه‌ای
\\
سعدی به پاکبازی و رندی مثل نشد
&&
تنها در این مدینه که در هر مدینه‌ای
\\
شعرش چو آب در همه عالم چنان شده
&&
کز پارس می‌رود به خراسان سفینه‌ای
\\
\end{longtable}
\end{center}
