\begin{center}
\section*{غزل ۲۰۰: چه کسی که هیچ کس را به تو بر نظر نباشد}
\label{sec:200}
\addcontentsline{toc}{section}{\nameref{sec:200}}
\begin{longtable}{l p{0.5cm} r}
چه کسی که هیچ کس را به تو بر نظر نباشد
&&
که نه در تو بازماند مگرش بصر نباشد
\\
نه طریق دوستانست و نه شرط مهربانی
&&
که ز دوستی بمیریم و تو را خبر نباشد
\\
مکن ار چه می‌توانی که ز خدمتم برانی
&&
نزنند سائلی را که دری دگر نباشد
\\
به رهت نشسته بودم که نظر کنی به حالم
&&
نکنی که چشم مستت ز خمار بر نباشد
\\
همه شب در این حدیثم که خنک تنی که دارد
&&
مژه‌ای به خواب و بختی که به خواب در نباشد
\\
چه خوشست مرغ وحشی که جفای کس نبیند
&&
من و مرغ خانگی را بکشند و پر نباشد
\\
نه من آن گناه دارم که بترسم از عقوبت
&&
نظری که سر نبازی ز سر نظر نباشد
\\
قمری که دوست داری همه روز دل بر آن نه
&&
که شبیت خون بریزد که در او قمر نباشد
\\
چه وجود نقش دیوار و چه آدمی که با او
&&
سخنی ز عشق گویند و در او اثر نباشد
\\
شب و روز رفت باید قدم روندگان را
&&
چو به مأمنی رسیدی دگرت سفر نباشد
\\
عجبست پیش بعضی که تر است شعر سعدی
&&
ورق درخت طوبیست چگونه تر نباشد
\\
\end{longtable}
\end{center}
