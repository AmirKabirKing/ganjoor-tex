\begin{center}
\section*{غزل ۶۱۶: نگویم آب و گلست آن وجود روحانی}
\label{sec:616}
\addcontentsline{toc}{section}{\nameref{sec:616}}
\begin{longtable}{l p{0.5cm} r}
نگویم آب و گل است آن وجود روحانی
&&
بدین کمال نباشد جمال انسانی
\\
اگر تو آب و گلی همچنان که سایر خلق
&&
گل بهشت مخمر به آب حیوانی
\\
به هر چه خوبتر اندر جهان نظر کردم
&&
که گویمش به تو ماند تو خوبتر ز آنی
\\
وجود هر که نگه می‌کنم ز جان و جسد
&&
مرکب است و تو از فرق تا قدم جانی
\\
گرت در آینه سیمای خویش دل ببرد
&&
چو من شوی و به درمان خویش درمانی
\\
دلی که با سر زلفت تعلقی دارد
&&
چگونه جمع شود با چنان پریشانی
\\
مرا که پیش تو اقرار بندگی کردم
&&
رواست گر بنوازی و گر برنجانی
\\
ولی خلاف بزرگان که گفته‌اند مکن
&&
بکن هر آن چه بشاید نه هر چه بتوانی
\\
طمع مدار که از دامنت بدارم دست
&&
به آستین ملالی که بر من افشانی
\\
فدای جان تو گر من فدا شوم چه شود
&&
برای عید بود گوسفند قربانی
\\
روان روشن سعدی که شمع مجلس توست
&&
به هیچ کار نیاید گرش نسوزانی
\\
\end{longtable}
\end{center}
