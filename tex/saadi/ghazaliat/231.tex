\begin{center}
\section*{غزل ۲۳۱: تو آن نه‌ای که دل از صحبت تو برگیرند}
\label{sec:231}
\addcontentsline{toc}{section}{\nameref{sec:231}}
\begin{longtable}{l p{0.5cm} r}
تو آن نه‌ای که دل از صحبت تو برگیرند
&&
و گر ملول شوی صاحبی دگر گیرند
\\
و گر به خشم برانی طریق رفتن نیست
&&
کجا روند که یار از تو خوبتر گیرند
\\
به تیغ اگر بزنی بی‌دریغ و برگردی
&&
چو روی باز کنی دوستی ز سر گیرند
\\
هلاک نفس به نزدیک طالبان مراد
&&
اگر چه کار بزرگست مختصر گیرند
\\
روا بود همه خوبان آفرینش را
&&
که پیش صاحب ما دست بر کمر گیرند
\\
قمر مقابله با روی او نیارد کرد
&&
و گر کند همه کس عیب بر قمر گیرند
\\
به چند سال نشاید گرفت ملکی را
&&
که خسروان ملاحت به یک نظر گیرند
\\
خدنگ غمزه خوبان خطا نمی‌افتد
&&
اگر چه طایفه‌ای زهد را سپر گیرند
\\
کم از مطالعه‌ای بوستان سلطان را
&&
چو باغبان نگذارد کز او ثمر گیرند
\\
وصال کعبه میسر نمی‌شود سعدی
&&
مگر که راه بیابان پرخطر گیرند
\\
\end{longtable}
\end{center}
