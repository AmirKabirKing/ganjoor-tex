\begin{center}
\section*{غزل ۴۳۳: ما به روی دوستان از بوستان آسوده‌ایم}
\label{sec:433}
\addcontentsline{toc}{section}{\nameref{sec:433}}
\begin{longtable}{l p{0.5cm} r}
ما به روی دوستان از بوستان آسوده‌ایم
&&
گر بهار آید وگر باد خزان آسوده‌ایم
\\
سروبالایی که مقصود است اگر حاصل شود
&&
سرو اگر هرگز نباشد در جهان آسوده‌ایم
\\
گر به صحرا دیگران از بهر عشرت می‌روند
&&
ما به خلوت با تو ای آرام جان آسوده‌ایم
\\
هر چه در دنیا و عقبی راحت و آسایش است
&&
گر تو با ما خوش درآیی ما از آن آسوده‌ایم
\\
برق نوروزی گر آتش می‌زند در شاخسار
&&
ور گل افشان می‌کند در بوستان آسوده‌ایم
\\
باغبان را گو اگر در گلستان آلاله‌ایست
&&
دیگری را ده که ما با دلستان آسوده‌ایم
\\
گر سیاست می‌کند سلطان و قاضی حاکمند
&&
ور ملامت می‌کند پیر و جوان آسوده‌ایم
\\
موج اگر کشتی برآرد تا به اوج آفتاب
&&
یا به قعر اندر برد ما بر کران آسوده‌ایم
\\
رنج‌ها بردیم و آسایش نبود اندر جهان
&&
ترک آسایش گرفتیم این زمان آسوده‌ایم
\\
سعدیا سرمایه‌داران از خلل ترسند و ما
&&
گر بر آید بانگ دزد از کاروان آسوده‌ایم
\\
\end{longtable}
\end{center}
