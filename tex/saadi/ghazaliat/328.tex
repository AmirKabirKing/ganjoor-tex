\begin{center}
\section*{غزل ۳۲۸: رها نمی‌کند ایام در کنار منش}
\label{sec:328}
\addcontentsline{toc}{section}{\nameref{sec:328}}
\begin{longtable}{l p{0.5cm} r}
رها نمی‌کند ایام در کنار منش
&&
که داد خود بستانم به بوسه از دهنش
\\
همان کمند بگیرم که صید خاطر خلق
&&
بدان همی‌کند و درکشم به خویشتنش
\\
ولیک دست نیارم زدن در آن سر زلف
&&
که مبلغی دل خلق است زیر هر شکنش
\\
غلام قامت آن لعبتم که بر قد او
&&
بریده‌اند لطافت چو جامه بر بدنش
\\
ز رنگ و بوی تو ای سروقد سیم اندام
&&
برفت رونق نسرین باغ و نسترنش
\\
یکی به حکم نظر پای در گلستان نه
&&
که پایمال کنی ارغوان و یاسمنش
\\
خوشا تفرج نوروز خاصه در شیراز
&&
که برکند دل مرد مسافر از وطنش
\\
عزیز مصر چمن شد جمال یوسف گل
&&
صبا به شهر درآورد بوی پیرهنش
\\
شگفت نیست گر از غیرت تو بر گلزار
&&
بگرید ابر و بخندد شکوفه بر چمنش
\\
در این روش که تویی گر به مرده برگذری
&&
عجب نباشد اگر نعره آید از کفنش
\\
نماند فتنه در ایام شاه جز سعدی
&&
که بر جمال تو فتنه‌ست و خلق بر سخنش
\\
\end{longtable}
\end{center}
