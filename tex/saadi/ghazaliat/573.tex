\begin{center}
\section*{غزل ۵۷۳: تو در کمند نیفتاده‌ای و معذوری}
\label{sec:573}
\addcontentsline{toc}{section}{\nameref{sec:573}}
\begin{longtable}{l p{0.5cm} r}
تو در کمند نیفتاده‌ای و معذوری
&&
از آن به قوت بازوی خویش مغروری
\\
گر آن که خرمن من سوخت با تو پردازد
&&
میسرت نشود عاشقی و مستوری
\\
بهشت روی من آن لعبت پری رخسار
&&
که در بهشت نباشد به لطف او حوری
\\
به گریه گفتمش ای سرو قد سیم اندام
&&
اگر چه سرو نباشد بر او گل سوری
\\
درشتخویی و بدعهدی از تو نپسندند
&&
که خوب منظری و دلفریب منظوری
\\
تو در میان خلایق به چشم اهل نظر
&&
چنان که در شب تاریک پاره نوری
\\
اگر به حسن تو باشد طبیب در آفاق
&&
کس از خدای نخواهد شفای رنجوری
\\
ز کبر و ناز چنان می‌کنی به مردم چشم
&&
که بی شراب گمان می‌برد که مخموری
\\
من از تو دست نخواهم به بی‌وفایی داشت
&&
تو هر گناه که خواهی بکن که مغفوری
\\
ز چند گونه سخن رفت و در میان آمد
&&
حدیث عاشقی و مفلسی و مهجوری
\\
به خنده گفت که سعدی سخن دراز مکن
&&
میان تهی و فراوان سخن چو طنبوری
\\
چو سایه هیچ کس است آدمی که هیچش نیست
&&
مرا از این چه که چون آفتاب مشهوری
\\
\end{longtable}
\end{center}
