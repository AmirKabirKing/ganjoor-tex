\begin{center}
\section*{غزل ۱۷۶: آن کیست کاندر رفتنش صبر از دل ما می‌برد}
\label{sec:176}
\addcontentsline{toc}{section}{\nameref{sec:176}}
\begin{longtable}{l p{0.5cm} r}
آن کیست کاندر رفتنش صبر از دل ما می‌برد
&&
ترک از خراسان آمدست از پارس یغما می‌برد
\\
شیراز مشکین می‌کند چون ناف آهوی ختن
&&
گر باد نوروز از سرش بویی به صحرا می‌برد
\\
من پاس دارم تا به روز امشب به جای پاسبان
&&
کان چشم خواب آلوده خواب از دیده ما می‌برد
\\
برتاس در بر می‌کنم یک لحظه بی اندام او
&&
چون خارپشتم گوییا سوزن در اعضا می‌برد
\\
بسیار می‌گفتم که دل با کس نپیوندم ولی
&&
دیدار خوبان اختیار از دست دانا می‌برد
\\
دل برد و تن درداده‌ام ور می‌کشد استاده‌ام
&&
کآخر نداند بیش از این یا می‌کشد یا می‌برد
\\
چون حلقه در گوشم کند هر روز لطفش وعده‌ای
&&
دیگر چو شب نزدیک شد چون زلف در پا می‌برد
\\
حاجت به ترکی نیستش تا در کمند آرد دلی
&&
من خود به رغبت در کمند افتاده‌ام تا می‌برد
\\
هر کو نصیحت می‌کند در روزگار حسن او
&&
دیوانگان عشق را دیگر به سودا می‌برد
\\
وصفش نداند کرد کس دریای شیرینست و بس
&&
سعدی که شوخی می‌کند گوهر به دریا می‌برد
\\
\end{longtable}
\end{center}
