\begin{center}
\section*{غزل ۱۵۶: فرهاد را چو بر رخ شیرین نظر فتاد}
\label{sec:156}
\addcontentsline{toc}{section}{\nameref{sec:156}}
\begin{longtable}{l p{0.5cm} r}
فرهاد را چو بر رخ شیرین نظر فتاد
&&
دودش به سر درآمد و از پای درفتاد
\\
مجنون ز جام طلعت لیلی چو مست شد
&&
فارغ ز مادر و پدر و سیم و زر فتاد
\\
رامین چو اختیار غم عشق ویس کرد
&&
یک بارگی جدا ز کلاه و کمر فتاد
\\
وامق چو کارش از غم عذرا به جان رسید
&&
کارش مدام با غم و آه سحر فتاد
\\
زین گونه صد هزار کس از پیر و از جوان
&&
مست از شراب عشق چو من بی‌خبر فتاد
\\
بسیار کس شدند اسیر کمند عشق
&&
تنها نه از برای من این شور و شر فتاد
\\
روزی به دلبری نظری کرد چشم من
&&
زان یک نظر مرا دو جهان از نظر فتاد
\\
عشق آمد آن چنان به دلم در زد آتشی
&&
کز وی هزار سوز مرا در جگر فتاد
\\
بر من مگیر اگر شدم آشفته دل ز عشق
&&
مانند این بسی ز قضا و قدر فتاد
\\
سعدی ز خلق چند نهان راز دل کنی
&&
چون ماجرای عشق تو یک یک به درفتاد
\\
\end{longtable}
\end{center}
