\begin{center}
\section*{غزل ۳۰۰: یار آن بود که صبر کند بر جفای یار}
\label{sec:300}
\addcontentsline{toc}{section}{\nameref{sec:300}}
\begin{longtable}{l p{0.5cm} r}
یار آن بود که صبر کند بر جفای یار
&&
ترک رضای خویش کند در رضای یار
\\
گر بر وجود عاشق صادق نهند تیغ
&&
بیند خطای خویش و نبیند خطای یار
\\
یار از برای نفس گرفتن طریق نیست
&&
ما نفس خویشتن بکشیم از برای یار
\\
یاران شنیده‌ام که بیابان گرفته‌اند
&&
بی‌طاقت از ملامت خلق و جفای یار
\\
من ره نمی‌برم مگر آن جا که کوی دوست
&&
من سر نمی‌نهم مگر آن جا که پای یار
\\
گفتی هوای باغ در ایام گل خوشست
&&
ما را به در نمی‌رود از سر هوای یار
\\
بستان بی مشاهده دیدن مجاهده‌ست
&&
ور صد درخت گل بنشانی به جای یار
\\
ای باد اگر به گلشن روحانیان روی
&&
یار قدیم را برسانی دعای یار
\\
ما را ز درد عشق تو با کس حدیث نیست
&&
هم پیش یار گفته شود ماجرای یار
\\
هر کس میان جمعی و سعدی و گوشه‌ای
&&
بیگانه باشد از همه خلق آشنای یار
\\
\end{longtable}
\end{center}
