\begin{center}
\section*{غزل ۵۱۸: تو خون خلق بریزی و روی درتابی}
\label{sec:518}
\addcontentsline{toc}{section}{\nameref{sec:518}}
\begin{longtable}{l p{0.5cm} r}
تو خون خلق بریزی و روی درتابی
&&
ندانمت چه مکافات این گنه یابی
\\
تصد عنی فی الجور و النوی لکن
&&
الیک قلبی یا غایة المنی صاب
\\
چو عندلیب چه فریادها که می‌دارم
&&
تو از غرور جوانی همیشه در خوابی
\\
الی العداة وصلتم و تصحبونهمو
&&
و فی وداد کمو قد هجرت احبابی
\\
نه هر که صاحب حسن است جور پیشه کند
&&
تو را چه شد که خود اندر کمین اصحابی
\\
احبتی امرونی بترک ذکراه
&&
لقد اطعت ولکن حبه آبی
\\
غمت چگونه بپوشم که دیده بر رویت
&&
همی گواهی بر من دهد به کذابی
\\
مرا تو بر سر آتش نشانده‌ای عجب آنک
&&
منم در آتش و از حال من تو در تابی
\\
من از تو سیر نگردم که صاحب استسقا
&&
نه ممکن است که هرگز رسد به سیرابی
\\
\end{longtable}
\end{center}
