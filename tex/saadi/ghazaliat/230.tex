\begin{center}
\section*{غزل ۲۳۰: شاید این طلعت میمون که به فالش دارند}
\label{sec:230}
\addcontentsline{toc}{section}{\nameref{sec:230}}
\begin{longtable}{l p{0.5cm} r}
شاید این طلعت میمون که به فالش دارند
&&
در دل اندیشه و در دیده خیالش دارند
\\
که در آفاق چنین روی دگر نتوان دید
&&
یا مگر آینه در پیش جمالش دارند
\\
عجب از دام غمش گر بجهد مرغ دلی
&&
این همه میل که با دانه خالش دارند
\\
نازنینی که سر اندر قدمش باید باخت
&&
نه حریفی که توقع به وصالش دارند
\\
غالب آنست که مرغی چو به دامی افتاد
&&
تا به جایی نرود بی پر و بالش دارند
\\
عشق لیلی نه به اندازه هر مجنونیست
&&
مگر آنان که سر ناز و دلالش دارند
\\
دوستی با تو حرامست که چشمان کشت
&&
خون عشاق بریزند و حلالش دارند
\\
خرما دور وصالی و خوشا درد دلی
&&
که به معشوق توان گفت و مجالش دارند
\\
حال سعدی تو ندانی که تو را دردی نیست
&&
دردمندان خبر از صورت حالش دارند
\\
\end{longtable}
\end{center}
