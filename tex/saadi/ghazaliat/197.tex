\begin{center}
\section*{غزل ۱۹۷: نظر خدای بینان طلب هوا نباشد}
\label{sec:197}
\addcontentsline{toc}{section}{\nameref{sec:197}}
\begin{longtable}{l p{0.5cm} r}
نظر خدای بینان طلب هوا نباشد
&&
سفر نیازمندان قدم خطا نباشد
\\
همه وقت عارفان را نظرست و عامیان را
&&
نظری معاف دارند و دوم روا نباشد
\\
به نسیم صبح باید که نبات زنده باشی
&&
نه جماد مرده کان را خبر از صبا نباشد
\\
اگرت سعادتی هست که زنده دل بمیری
&&
به حیاتی اوفتادی که دگر فنا نباشد
\\
به کسی نگر که ظلمت بزداید از وجودت
&&
نه کسی نعوذبالله که در او صفا نباشد
\\
تو خود از کدام شهری که ز دوستان نپرسی
&&
مگر اندر آن ولایت که تویی وفا نباشد
\\
اگر اهل معرفت را چو نی استخوان بسنبی
&&
چو دفش به هیچ سختی خبر از قفا نباشد
\\
اگرم تو خون بریزی به قیامتت نگیرم
&&
که میان دوستان این همه ماجرا نباشد
\\
نه حریف مهربانست حریف سست پیمان
&&
که به روز تیرباران سپر بلا نباشد
\\
تو در آینه نگه کن که چه دلبری ولیکن
&&
تو که خویشتن ببینی نظرت به ما نباشد
\\
تو گمان مبر که سعدی ز جفا ملول گردد
&&
که گرش تو بی جنایت بکشی جفا نباشد
\\
دگری همین حکایت بکند که من ولیکن
&&
چو معاملت ندارد سخن آشنا نباشد
\\
\end{longtable}
\end{center}
