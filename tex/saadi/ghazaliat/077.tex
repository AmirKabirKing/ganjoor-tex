\begin{center}
\section*{غزل ۷۷: بر من که صبوحی زده‌ام خرقه حرامست}
\label{sec:077}
\addcontentsline{toc}{section}{\nameref{sec:077}}
\begin{longtable}{l p{0.5cm} r}
بر من که صبوحی زده‌ام خرقه حرام است
&&
ای مجلسیان راه خرابات کدام است
\\
هر کس به جهان خرمیی پیش گرفتند
&&
ما را غمت ای ماه پری چهره تمام است
\\
برخیز که در سایه سروی بنشینیم
&&
کان جا که تو بنشینی بر سرو قیام است
\\
دام دل صاحب نظرانت خم گیسوست
&&
وان خال بناگوش مگر دانه دام است
\\
با چون تو حریفی به چنین جای در این وقت
&&
گر باده خورم خمر بهشتی نه حرام است
\\
با محتسب شهر بگویید که زنهار
&&
در مجلس ما سنگ مینداز که جام است
\\
غیرت نگذارد که بگویم که مرا کشت
&&
تا خلق ندانند که معشوقه چه نام است
\\
دردا که بپختیم در این سوز نهانی
&&
وان را خبر از آتش ما نیست که خام است
\\
سعدی مبر اندیشه که در کام نهنگان
&&
چون در نظر دوست نشینی همه کام است
\\
\end{longtable}
\end{center}
