\begin{center}
\section*{غزل ۲۹۸: شرطست جفا کشیدن از یار}
\label{sec:298}
\addcontentsline{toc}{section}{\nameref{sec:298}}
\begin{longtable}{l p{0.5cm} r}
شرط است جفا کشیدن از یار
&&
خمر است و خمار و گلبن و خار
\\
من معتقدم که هر چه گویی
&&
شیرین بود از لب شکربار
\\
پیش دگری نمی‌توان رفت
&&
از تو به تو آمدم به زنهار
\\
عیبت نکنم اگر بخندی
&&
بر من چو بگریم از غمت زار
\\
شک نیست که بوستان بخندد
&&
هر گه که بگرید ابر آذار
\\
تو می‌روی و خبر نداری
&&
واندر عقبت قلوب و ابصار
\\
گر پیش تو نوبتی بمیرم
&&
هیچم نبود گزند و تیمار
\\
جز حسرت آن که زنده گردم
&&
تا پیش بمیرمت دگربار
\\
گفتم که به گوشه‌ای چو سنگی
&&
بنشینم و روی دل به دیوار
\\
دانم که میسرم نگردد
&&
تو سنگ درآوری به گفتار
\\
سعدی نرود به سختی از پیش
&&
با قید کجا رود گرفتار
\\
\end{longtable}
\end{center}
