\begin{center}
\section*{غزل ۱۸۴: دلم دل از هوس یار بر نمی‌گیرد}
\label{sec:184}
\addcontentsline{toc}{section}{\nameref{sec:184}}
\begin{longtable}{l p{0.5cm} r}
دلم دل از هوس یار بر نمی‌گیرد
&&
طریق مردم هشیار بر نمی‌گیرد
\\
بلای عشق خدایا ز جان ما برگیر
&&
که جان من دل از این کار بر نمی‌گیرد
\\
همی‌گدازم و می‌سازم و شکیباییست
&&
که پرده از سر اسرار بر نمی‌گیرد
\\
وجود خسته من زیر بار جور فلک
&&
جفای یار به سربار بر نمی‌گیرد
\\
رواست گر نکند یار دعوی یاری
&&
چو بار غم ز دل یار بر نمی‌گیرد
\\
چه باشد ار به وفا دست گیردم یک بار
&&
گرم ز دست به یک بار بر نمی‌گیرد
\\
بسوخت سعدی در دوزخ فراق و هنوز
&&
طمع ز وعده دیدار بر نمی‌گیرد
\\
\end{longtable}
\end{center}
