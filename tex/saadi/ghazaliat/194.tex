\begin{center}
\section*{غزل ۱۹۴: شب عاشقان بی‌دل چه شبی دراز باشد}
\label{sec:194}
\addcontentsline{toc}{section}{\nameref{sec:194}}
\begin{longtable}{l p{0.5cm} r}
شب عاشقان بی‌دل چه شبی دراز باشد
&&
تو بیا کز اول شب در صبح باز باشد
\\
عجبست اگر توانم که سفر کنم ز دستت
&&
به کجا رود کبوتر که اسیر باز باشد
\\
ز محبتت نخواهم که نظر کنم به رویت
&&
که محب صادق آنست که پاکباز باشد
\\
به کرشمه عنایت نگهی به سوی ما کن
&&
که دعای دردمندان ز سر نیاز باشد
\\
سخنی که نیست طاقت که ز خویشتن بپوشم
&&
به کدام دوست گویم که محل راز باشد
\\
چه نماز باشد آن را که تو در خیال باشی
&&
تو صنم نمی‌گذاری که مرا نماز باشد
\\
نه چنین حساب کردم چو تو دوست می‌گرفتم
&&
که ثنا و حمد گوییم و جفا و ناز باشد
\\
دگرش چو بازبینی غم دل مگوی سعدی
&&
که شب وصال کوتاه و سخن دراز باشد
\\
قدمی که برگرفتی به وفا و عهد یاران
&&
اگر از بلا بترسی قدم مجاز باشد
\\
\end{longtable}
\end{center}
