\begin{center}
\section*{غزل ۴۱۷: سخن عشق تو بی آن که برآید به زبانم}
\label{sec:417}
\addcontentsline{toc}{section}{\nameref{sec:417}}
\begin{longtable}{l p{0.5cm} r}
سخن عشق تو بی آن که برآید به زبانم
&&
رنگ رخساره خبر می‌دهد از حال نهانم
\\
گاه گویم که بنالم ز پریشانی حالم
&&
بازگویم که عیان است چه حاجت به بیانم
\\
هیچم از دنیی و عقبی نبرد گوشه خاطر
&&
که به دیدار تو شغل است و فراغ از دو جهانم
\\
گر چنان است که روی من مسکین گدا را
&&
به در غیر ببینی ز در خویش برانم
\\
من در اندیشه آنم که روان بر تو فشانم
&&
نه در اندیشه که خود را ز کمندت برهانم
\\
گر تو شیرین زمانی نظری نیز به من کن
&&
که به دیوانگی از عشق تو فرهاد زمانم
\\
نه مرا طاقت غربت نه تو را خاطر قربت
&&
دل نهادم به صبوری که جز این چاره ندانم
\\
من همان روز بگفتم که طریق تو گرفتم
&&
که به جانان نرسم تا نرسد کار به جانم
\\
درم از دیده چکان است به یاد لب لعلت
&&
نگهی باز به من کن که بسی در بچکانم
\\
سخن از نیمه بریدم که نگه کردم و دیدم
&&
که به پایان رسدم عمر و به پایان نرسانم
\\
\end{longtable}
\end{center}
