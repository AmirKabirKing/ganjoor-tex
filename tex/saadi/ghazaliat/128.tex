\begin{center}
\section*{غزل ۱۲۸: چو ترک دلبر من شاهدی به شنگی نیست}
\label{sec:128}
\addcontentsline{toc}{section}{\nameref{sec:128}}
\begin{longtable}{l p{0.5cm} r}
چو ترک دلبر من شاهدی به شنگی نیست
&&
چو زلف پرشکنش حلقه فرنگی نیست
\\
دهانش ار چه نبینی مگر به وقت سخن
&&
چو نیک درنگری چون دلم به تنگی نیست
\\
به تیغ غمزه خون خوار لشکری بزنی
&&
بزن که با تو در او هیچ مرد جنگی نیست
\\
قوی به چنگ من افتاده بود دامن وصل
&&
ولی دریغ که دولت به تیزچنگی نیست
\\
دوم به لطف ندارد عجب که چون سعدی
&&
غلام سعد ابوبکر سعد زنگی نیست
\\
\end{longtable}
\end{center}
