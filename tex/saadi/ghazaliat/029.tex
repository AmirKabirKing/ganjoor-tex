\begin{center}
\section*{غزل ۲۹: متناسبند و موزون حرکات دلفریبت}
\label{sec:029}
\addcontentsline{toc}{section}{\nameref{sec:029}}
\begin{longtable}{l p{0.5cm} r}
متناسبند و موزون حرکات دلفریبت
&&
متوجه است با ما سخنان بی حسیبت
\\
چو نمی‌توان صبوری ستمت کشم ضروری
&&
مگر آدمی نباشد که برنجد از عتیبت
\\
اگرم تو خصم باشی نروم ز پیش تیرت
&&
وگرم تو سیل باشی نگریزم از نشیبت
\\
به قیاس درنگنجی و به وصف درنیایی
&&
متحیرم در اوصاف جمال و روی و زیبت
\\
اگرم برآورد بخت به تخت پادشاهی
&&
نه چنان که بنده باشم همه عمر در رکیبت
\\
عجب از کسی در این شهر که پارسا بماند
&&
مگر او ندیده باشد رخ پارسافریبت
\\
تو برون خبر نداری که چه می‌رود ز عشقت
&&
به درآی اگر نه آتش بزنیم در حجیبت
\\
تو درخت خوب منظر همه میوه‌ای ولیکن
&&
چه کنم به دست کوته که نمی‌رسد به سیبت
\\
تو شبی در انتظاری ننشسته‌ای چه دانی
&&
که چه شب گذشت بر منتظران ناشکیبت
\\
تو خود ای شب جدایی چه شبی بدین درازی
&&
بگذر که جان سعدی بگداخت از نهیبت
\\
\end{longtable}
\end{center}
