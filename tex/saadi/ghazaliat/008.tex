\begin{center}
\section*{غزل ۸: ز اندازه بیرون تشنه‌ام ساقی بیار آن آب را}
\label{sec:008}
\addcontentsline{toc}{section}{\nameref{sec:008}}
\begin{longtable}{l p{0.5cm} r}
ز اندازه بیرون تشنه‌ام ساقی بیار آن آب را
&&
اول مرا سیراب کن وان گه بده اصحاب را
\\
من نیز چشم از خواب خوش بر می‌نکردم پیش از این
&&
روز فراق دوستان شب خوش بگفتم خواب را
\\
هر پارسا را کان صنم در پیش مسجد بگذرد
&&
چشمش بر ابرو افکند باطل کند محراب را
\\
من صید وحشی نیستم در بند جان خویشتن
&&
گر وی به تیرم می‌زند استاده‌ام نشاب را
\\
مقدار یار همنفس چون من نداند هیچ کس
&&
ماهی که بر خشک اوفتد قیمت بداند آب را
\\
وقتی در آبی تا میان دستی و پایی می‌زدم
&&
اکنون همان پنداشتم دریای بی پایاب را
\\
امروز حالی غرقه‌ام تا با کناری اوفتم
&&
آن گه حکایت گویمت درد دل غرقاب را
\\
گر بی‌وفایی کردمی یرغو به قاآن بردمی
&&
کان کافر اعدا می‌کشد وین سنگدل احباب را
\\
فریاد می‌دارد رقیب از دست مشتاقان او
&&
آواز مطرب در سرا زحمت بود بواب را
\\
«سعدی! چو جورش می‌بری نزدیک او دیگر مرو»
&&
ای بی‌بصر! من می‌روم؟ او می‌کشد قلاب را
\\
\end{longtable}
\end{center}
