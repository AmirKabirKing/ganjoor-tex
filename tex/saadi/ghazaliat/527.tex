\begin{center}
\section*{غزل ۵۲۷: ای باد که بر خاک در دوست گذشتی}
\label{sec:527}
\addcontentsline{toc}{section}{\nameref{sec:527}}
\begin{longtable}{l p{0.5cm} r}
ای باد که بر خاک در دوست گذشتی
&&
پندارمت از روضه بستان بهشتی
\\
دور از سببی نیست که شوریده سودا
&&
هر لحظه چو دیوانه دوان بر در و دشتی
\\
باری مگرت بر رخ جانان نظر افتاد
&&
سرگشته چو من در همه آفاق بگشتی
\\
از کف ندهم دامن معشوقه زیبا
&&
هل تا برود نام من ای یار به زشتی
\\
جز یاد تو بر خاطر من نگذرد ای جان
&&
با آن که به یک باره‌ام از یاد بهشتی
\\
با طبع ملولت چه کند دل که نسازد
&&
شرطه همه وقتی نبود لایق کشتی
\\
بسیار گذشتی که نکردی سوی ما چشم
&&
یک دم ننشستم که به خاطر نگذشتی
\\
شوخی شکرالفاظ و مهی لاله بناگوش
&&
سروی سمن اندام و بتی حورسرشتی
\\
قلاب تو در کس نفکندی که نبردی
&&
شمشیر تو بر کس نکشیدی که نکشتی
\\
سیلاب قضا نسترد از دفتر ایام
&&
این‌ها که تو بر خاطر سعدی بنوشتی
\\
\end{longtable}
\end{center}
