\begin{center}
\section*{غزل ۶۸: چشمت خوشست و بر اثر خواب خوشترست}
\label{sec:068}
\addcontentsline{toc}{section}{\nameref{sec:068}}
\begin{longtable}{l p{0.5cm} r}
چشمت خوشست و بر اثر خواب خوشترست
&&
طعم دهانت از شکر ناب خوشترست
\\
زنهار از آن تبسم شیرین که می‌کنی
&&
کز خنده شکوفه سیراب خوشترست
\\
شمعی به پیش روی تو گفتم که برکنم
&&
حاجت به شمع نیست که مهتاب خوشترست
\\
دوش آرزوی خواب خوشم بود یک زمان
&&
امشب نظر به روی تو از خواب خوشترست
\\
در خوابگاه عاشق سر بر کنار دوست
&&
کیمخت خارپشت ز سنجاب خوشترست
\\
زان سوی بحر آتش اگر خوانیم به لطف
&&
رفتن به روی آتشم از آب خوشترست
\\
ز آب روان و سبزه و صحرا و لاله زار
&&
با من مگو که چشم در احباب خوشترست
\\
زهرم مده به دست رقیبان تندخوی
&&
از دست خود بده که ز جلاب خوشترست
\\
سعدی دگر به گوشه وحدت نمی‌رود
&&
خلوت خوشست و صحبت اصحاب خوشترست
\\
هر باب از این کتاب نگارین که برکنی
&&
همچون بهشت گویی از آن باب خوشترست
\\
\end{longtable}
\end{center}
