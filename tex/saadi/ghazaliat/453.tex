\begin{center}
\section*{غزل ۴۵۳: سخت به ذوق می‌دهد باد ز بوستان نشان}
\label{sec:453}
\addcontentsline{toc}{section}{\nameref{sec:453}}
\begin{longtable}{l p{0.5cm} r}
سخت به ذوق می‌دهد باد ز بوستان نشان
&&
صبح دمید و روز شد خیز و چراغ وانشان
\\
گر همه خلق را چو من بی‌دل و مست می‌کنی
&&
روی به صالحان نما خمر به زاهدان چشان
\\
طایفه‌ای سماع را عیب کنند و عشق را
&&
زمزمه‌ای بیار خوش تا بروند ناخوشان
\\
خرقه بگیر و می بده باده بیار و غم ببر
&&
بی‌خبر است عاقل از لذت عیش بیهشان
\\
سوختگان عشق را دود به سقف می‌رود
&&
وقع ندارد این سخن پیش فسرده آتشان
\\
رقص حلال بایدت سنت اهل معرفت
&&
دنیا زیر پای نه دست به آخرت فشان
\\
تیغ به خفیه می‌خورم آه نهفته می‌کنم
&&
گوش کجا که بشنود ناله زار خامشان
\\
چند نصیحتم کنی کز پی نیکوان مرو
&&
چون نروم که بیخودم شوق همی‌برد کشان
\\
من نه به وقت خویشتن پیر و شکسته بوده‌ام
&&
موی سپید می‌کند چشم سیاه اکدشان
\\
بوی بهشت می‌دمد ما به عذاب در گرو
&&
آب حیات می‌رود ما تن خویشتن کشان
\\
باد بهار و بوی گل متفقند سعدیا
&&
چون تو فصیح بلبلی حیف بود ز خامشان
\\
\end{longtable}
\end{center}
