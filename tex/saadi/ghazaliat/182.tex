\begin{center}
\section*{غزل ۱۸۲: زنده شود هر که پیش دوست بمیرد}
\label{sec:182}
\addcontentsline{toc}{section}{\nameref{sec:182}}
\begin{longtable}{l p{0.5cm} r}
زنده شود هر که پیش دوست بمیرد
&&
مرده دلست آن که هیچ دوست نگیرد
\\
هر که ز ذوقش درون سینه صفاییست
&&
شمع دلش را ز شاهدی نگزیرد
\\
طالب عشقی دلی چو موم به دست آر
&&
سنگ سیه صورت نگین نپذیرد
\\
صورت سنگین دلی کشنده سعدیست
&&
هر که بدین صورتش کشند نمیرد
\\
\end{longtable}
\end{center}
