\begin{center}
\section*{غزل ۴۶: دیگر نشنیدیم چنین فتنه که برخاست}
\label{sec:046}
\addcontentsline{toc}{section}{\nameref{sec:046}}
\begin{longtable}{l p{0.5cm} r}
دیگر نشنیدیم چنین فتنه که برخاست
&&
از خانه برون آمد و بازار بیاراست
\\
در وهم نگنجد که چه دلبند و چه شیرین
&&
در وصف نیاید که چه مطبوع و چه زیباست
\\
صبر و دل و دین می‌رود و طاقت و آرام
&&
از زخم پدید است که بازوش تواناست
\\
از بهر خدا روی مپوش از زن و از مرد
&&
تا صنع خدا می‌نگرند از چپ و از راست
\\
چشمی که تو را بیند و در قدرت بی چون
&&
مدهوش نماند نتوان گفت که بیناست
\\
دنیا به چه کار آید و فردوس چه باشد
&&
از بارخدا به ز تو حاجت نتوان خواست
\\
فریاد من از دست غمت عیب نباشد
&&
کاین درد نپندارم از آن من تنهاست
\\
با جور و جفای تو نسازیم چه سازیم
&&
چون زهره و یارا نبود چاره مداراست
\\
از روی شما صبر نه صبر است که زهر است
&&
وز دست شما زهر نه زهر است که حلواست
\\
آن کام و دهان و لب و دندان که تو داری
&&
عیش است ولی تا ز برای که مهیاست
\\
گر خون من و جمله عالم تو بریزی
&&
اقرار بیاریم که جرم از طرف ماست
\\
تسلیم تو سعدی نتواند که نباشد
&&
گر سر بنهد ور ننهد دست تو بالاست
\\
\end{longtable}
\end{center}
