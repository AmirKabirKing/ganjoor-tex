\begin{center}
\section*{غزل ۱۱۷: ای که گفتی هیچ مشکل چون فراق یار نیست}
\label{sec:117}
\addcontentsline{toc}{section}{\nameref{sec:117}}
\begin{longtable}{l p{0.5cm} r}
ای که گفتی هیچ مشکل چون فراق یار نیست
&&
گر امید وصل باشد همچنان دشوار نیست
\\
خلق را بیدار باید بود از آب چشم من
&&
وین عجب کان وقت می‌گریم که کس بیدار نیست
\\
نوک مژگانم به سرخی بر بیاض روی زرد
&&
قصه دل می‌نویسد حاجت گفتار نیست
\\
بی‌دلان را عیب کردم لاجرم بی‌دل شدم
&&
آن گنه را این عقوبت همچنان بسیار نیست
\\
ای نسیم صبح اگر باز اتفاقی افتدت
&&
آفرین گویی بر آن حضرت که ما را بار نیست
\\
بارها روی از پریشانی به دیوار آورم
&&
ور غم دل با کسی گویم به از دیوار نیست
\\
ما زبان اندرکشیدیم از حدیث خلق و روی
&&
گر حدیثی هست با یارست و با اغیار نیست
\\
قادری بر هر چه می‌خواهی مگر آزار من
&&
زان که گر شمشیر بر فرقم نهی آزار نیست
\\
احتمال نیش کردن واجبست از بهر نوش
&&
حمل کوه بیستون بر یاد شیرین بار نیست
\\
سرو را مانی ولیکن سرو را رفتار نه
&&
ماه را مانی ولیکن ماه را گفتار نیست
\\
گر دلم در عشق تو دیوانه شد عیبش مکن
&&
بدر بی نقصان و زر بی عیب و گل بی خار نیست
\\
لوحش الله از قد و بالای آن سرو سهی
&&
زان که همتایش به زیر گنبد دوار نیست
\\
دوستان گویند سعدی خیمه بر گلزار زن
&&
من گلی را دوست می‌دارم که در گلزار نیست
\\
\end{longtable}
\end{center}
