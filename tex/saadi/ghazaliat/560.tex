\begin{center}
\section*{غزل ۵۶۰: خبر از عیش ندارد که ندارد یاری}
\label{sec:560}
\addcontentsline{toc}{section}{\nameref{sec:560}}
\begin{longtable}{l p{0.5cm} r}
خبر از عیش ندارد که ندارد یاری
&&
دل نخوانند که صیدش نکند دلداری
\\
جان به دیدار تو یک روز فدا خواهم کرد
&&
تا دگر برنکنم دیده به هر دیداری
\\
یعلم الله که من از دست غمت جان نبرم
&&
تو به از من بتر از من بکشی بسیاری
\\
غم عشق آمد و غم‌های دگر پاک ببرد
&&
سوزنی باید کز پای برآرد خاری
\\
می حرام است ولیکن تو بدین نرگس مست
&&
نگذاری که ز پیشت برود هشیاری
\\
می‌روی خرم و خندان و نگه می‌نکنی
&&
که نگه می‌کند از هر طرفت غمخواری
\\
خبرت هست که خلقی ز غمت بی‌خبرند
&&
حال افتاده نداند که نیفتد باری
\\
سرو آزاد به بالای تو می‌ماند راست
&&
لیکنش با تو میسر نشود رفتاری
\\
می‌نماید که سر عربده دارد چشمت
&&
مست خوابش نبرد تا نکند آزاری
\\
سعدیا دوست نبینی و به وصلش نرسی
&&
مگر آن وقت که خود را ننهی مقداری
\\
\end{longtable}
\end{center}
