\begin{center}
\section*{غزل ۱۱۲: زهی رفیق که با چون تو سروبالاییست}
\label{sec:112}
\addcontentsline{toc}{section}{\nameref{sec:112}}
\begin{longtable}{l p{0.5cm} r}
زهی رفیق که با چون تو سروبالاییست
&&
که از خدای بر او نعمتی و آلاییست
\\
هر آن که با تو دمی یافتست در همه عمر
&&
نیافتست اگرش بعد از آن تمناییست
\\
هر آن که رای تو معلوم کرد و دیگربار
&&
برای خود نفسی می‌زند نه بس راییست
\\
نه عاشقست که هر ساعتش نظر به کسی
&&
نه عارفست که هر روز خاطرش جاییست
\\
مرا و یاد تو بگذار و کنج تنهایی
&&
که هر که با تو به خلوت بود نه تنهاییست
\\
به اختیار شکیبایی از تو نتوان بود
&&
به اضطرار توان بود اگر شکیباییست
\\
نظر به روی تو هر بامداد نوروزیست
&&
شب فراق تو هر شب که هست یلداییست
\\
خلاص بخش خدایا همه اسیران را
&&
مگر کسی که اسیر کمند زیباییست
\\
حکیم بین که برآورد سر به شیدایی
&&
حکیم را که دل از دست رفت شیداییست
\\
ولیک عذر توان گفت پای سعدی را
&&
در این لجم چو فروشد نه اولین پاییست
\\
\end{longtable}
\end{center}
