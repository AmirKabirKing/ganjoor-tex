\begin{center}
\section*{غزل ۲۴۱: میل بین کان سروبالا می‌کند}
\label{sec:241}
\addcontentsline{toc}{section}{\nameref{sec:241}}
\begin{longtable}{l p{0.5cm} r}
میل بین کان سروبالا می‌کند
&&
سرو بین کاهنگ صحرا می‌کند
\\
میل از این خوشتر نداند کرد سرو
&&
ناخوش آن میلست کز ما می‌کند
\\
حاجت صحرا نبود آیینه هست
&&
گر نگارستان تماشا می‌کند
\\
غافلست از صورت زیبای او
&&
آن که صورت‌های دیبا می‌کند
\\
من هم اول روز دانستم که عشق
&&
خون مباح و خانه یغما می‌کند
\\
صبر هم سودی ندارد کآب چشم
&&
راز پنهان آشکارا می‌کند
\\
گر مراد ما نباشد گو مباش
&&
چون مراد اوست هل تا می‌کند
\\
یار زیبا گر بریزد خون یار
&&
زشت نتوان گفت زیبا می‌کند
\\
سعدیا بعد از تحمل چاره نیست
&&
هر ستم کان دوست با ما می‌کند
\\
تا مگس را جان شیرین در تنست
&&
گرد آن گردد که حلوا می‌کند
\\
\end{longtable}
\end{center}
