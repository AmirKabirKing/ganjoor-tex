\begin{center}
\section*{غزل ۵۰۶: دریچه‌ای ز بهشتش به روی بگشایی}
\label{sec:506}
\addcontentsline{toc}{section}{\nameref{sec:506}}
\begin{longtable}{l p{0.5cm} r}
دریچه‌ای ز بهشتش به روی بگشایی
&&
که بامداد پگاهش تو روی بنمایی
\\
جهان شب است و تو خورشید عالم آرایی
&&
صباح مقبل آن کز درش تو بازآیی
\\
به از تو مادر گیتی به عمر خود فرزند
&&
نیاورد که همین بود حد زیبایی
\\
هر آن که با تو وصالش دمی میسر شد
&&
میسرش نشود بعد از آن شکیبایی
\\
درون پیرهن از غایت لطافت جسم
&&
چو آب صافی در آبگینه پیدایی
\\
مرا مجال سخن بیش در بیان تو نیست
&&
کمال حسن ببندد زبان گویایی
\\
ز گفت و گوی عوام احتراز می‌کردم
&&
کز این سپس بنشینم به کنج تنهایی
\\
وفای صحبت جانان به گوش جانم گفت
&&
نه عاشقی که حذر می‌کنی ز رسوایی
\\
گذشت بر من از آسیب عشقت آن چه گذشت
&&
هنوز منتظرم تا چه حکم فرمایی
\\
دو روزه باقی عمرم فدای جان تو باد
&&
اگر بکاهی و در عمر خود بیفزایی
\\
گر او نظر نکند سعدیا به چشم نواخت
&&
به دست سعی تو باد است تا نپیمایی
\\
\end{longtable}
\end{center}
