\begin{center}
\section*{غزل ۲۴۷: با دوست باش گر همه آفاق دشمنند}
\label{sec:247}
\addcontentsline{toc}{section}{\nameref{sec:247}}
\begin{longtable}{l p{0.5cm} r}
با دوست باش گر همه آفاق دشمنند
&&
کاو مرهم است اگر دگران نیش می‌زنند
\\
ای صورتی که پیش تو خوبان روزگار
&&
همچون طلسم پای خجالت به دامنند
\\
یک بامداد اگر بخرامی به بوستان
&&
بینی که سرو را ز لب جوی برکنند
\\
تلخ است پیش طایفه‌ای جور خوبروی
&&
از معتقد شنو که شکر می‌پراکنند
\\
ای متقی گر اهل دلی دیده‌ها بدوز
&&
کاینان به دل ربودن مردم معینند
\\
یا پرده‌ای به چشم تأمل فروگذار
&&
یا دل بنه که پرده ز کارت برافکنند
\\
جانم دریغ نیست ولیکن دل ضعیف
&&
صندوق سر توست نخواهم که بشکنند
\\
حسن تو نادر است در این عهد و شعر من
&&
من چشم بر تو و همگان گوش بر منند
\\
گویی جمال دوست که بیند چنان که اوست
&&
الا به راه دیده سعدی نظر کنند
\\
\end{longtable}
\end{center}
