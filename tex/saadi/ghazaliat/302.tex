\begin{center}
\section*{غزل ۳۰۲: به فلک می‌رسد از روی چو خورشید تو نور}
\label{sec:302}
\addcontentsline{toc}{section}{\nameref{sec:302}}
\begin{longtable}{l p{0.5cm} r}
به فلک می‌رسد از روی چو خورشید تو نور
&&
قل هو الله احد چشم بد از روی تو دور
\\
آدمی چون تو در آفاق نشان نتوان داد
&&
بلکه در جنت فردوس نباشد چو تو حور
\\
حور فردا که چنین روی بهشتی بیند
&&
گرش انصاف بود معترف آید به قصور
\\
شب ما روز نباشد مگر آن گاه که تو
&&
از شبستان به درآیی چو صباح از دیجور
\\
زندگان را نه عجب گر به تو میلی باشد
&&
مردگان بازنشینند به عشقت ز قبور
\\
آن بهایم نتوان گفت که جانی دارد
&&
که ندارد نظری با چو تو زیبامنظور
\\
سحر چشمان تو باطل نکند چشم آویز
&&
مست چندان که بکوشند نباشد مستور
\\
این حلاوت که تو داری نه عجب کز دستت
&&
عسلی دوزد و زنار ببندد زنبور
\\
آن چه در غیبتت ای دوست به من می‌گذرد
&&
نتوانم که حکایت کنم الا به حضور
\\
منم امروز و تو انگشت نمای زن و مرد
&&
من به شیرین سخنی تو به نکویی مشهور
\\
سختم آید که به هر دیده تو را می‌نگرند
&&
سعدیا غیرتت آمد نه عجب سعد غیور
\\
\end{longtable}
\end{center}
