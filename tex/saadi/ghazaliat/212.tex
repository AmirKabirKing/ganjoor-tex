\begin{center}
\section*{غزل ۲۱۲: هر که شیرینی فروشد مشتری بر وی بجوشد}
\label{sec:212}
\addcontentsline{toc}{section}{\nameref{sec:212}}
\begin{longtable}{l p{0.5cm} r}
هر که شیرینی فروشد مشتری بر وی بجوشد
&&
یا مگس را پر ببندد یا عسل را سر بپوشد
\\
همچنان عاشق نباشد ور بود صادق نباشد
&&
هر که درمان می‌پذیرد یا نصیحت می‌نیوشد
\\
گر مطیع خدمتت را کفر فرمایی بگوید
&&
ور حریف مجلست را زهر فرمایی بنوشد
\\
شمع پیشت روشنایی نزد آتش می‌نماید
&&
گل به دستت خوبرویی پیش یوسف می‌فروشد
\\
سود بازرگان دریا بی‌خطر ممکن نگردد
&&
هر که مقصودش تو باشی تا نفس دارد بکوشد
\\
برگ چشمم می‌نخوشد در زمستان فراقت
&&
وین عجب کاندر زمستان برگ‌های تر بخوشد
\\
هر که معشوقی ندارد عمر ضایع می‌گذارد
&&
همچنان ناپخته باشد هر که بر آتش نجوشد
\\
تا غمی پنهان نباشد رقتی پیدا نگردد
&&
هم گلی دیدست سعدی تا چو بلبل می‌خروشد
\\
\end{longtable}
\end{center}
