\begin{center}
\section*{غزل ۱۴۷: جان و تنم ای دوست فدای تن و جانت}
\label{sec:147}
\addcontentsline{toc}{section}{\nameref{sec:147}}
\begin{longtable}{l p{0.5cm} r}
جان و تنم ای دوست فدای تن و جانت
&&
مویی نفروشم به همه ملک جهانت
\\
شیرینتر از این لب نشنیدم که سخن گفت
&&
تو خود شکری یا عسلست آب دهانت
\\
یک روز عنایت کن و تیری به من انداز
&&
باشد که تفرج بکنم دست و کمانت
\\
گر راه بگردانی و گر روی بپوشی
&&
من می‌نگرم گوشه چشم نگرانت
\\
بر سرو نباشد رخ چون ماه منیرت
&&
بر ماه نباشد قد چون سرو روانت
\\
آخر چه بلایی تو که در وصف نیایی
&&
بسیار بگفتیم و نکردیم بیانت
\\
هر کس که ملامت کند از عشق تو ما را
&&
معذور بدارند چو بینند عیانت
\\
حیفست چنین روی نگارین که بپوشی
&&
سودی به مساکین رسد آخر چه زیانت
\\
بازآی که در دیده بماندست خیالت
&&
بنشین که به خاطر بگرفتست نشانت
\\
بسیار نباشد دلی از دست بدادن
&&
از جان رمقی دارم و هم برخی جانت
\\
دشنام کرم کردی و گفتی و شنیدم
&&
خرم تن سعدی که برآمد به زبانت
\\
\end{longtable}
\end{center}
