\begin{center}
\section*{غزل ۱۷۱: مگر نسیم سحر بوی یار من دارد}
\label{sec:171}
\addcontentsline{toc}{section}{\nameref{sec:171}}
\begin{longtable}{l p{0.5cm} r}
مگر نسیم سحر بوی یار من دارد
&&
که راحت دل امیدوار من دارد
\\
به پای سرو درافتاده‌اند لاله و گل
&&
مگر شمایل قد نگار من دارد
\\
نشان راه سلامت ز من مپرس که عشق
&&
زمام خاطر بی‌اختیار من دارد
\\
گلا و تازه بهارا تویی که عارض تو
&&
طراوت گل و بوی بهار من دارد
\\
دگر سر من و بالین عافیت هیهات
&&
بدین هوس که سر خاکسار من دارد
\\
به هرزه در سر او روزگار کردم و او
&&
فراغت از من و از روزگار من دارد
\\
مگر به درد دلی بازمانده‌ام یا رب
&&
کدام دامن همت غبار من دارد
\\
به زیر بار تو سعدی چو خر به گل درماند
&&
دلت نسوزد که بیچاره بار من دارد
\\
\end{longtable}
\end{center}
