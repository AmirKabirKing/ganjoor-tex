\begin{center}
\section*{غزل ۵۳۳: ای باد بامدادی خوش می‌روی به شادی}
\label{sec:533}
\addcontentsline{toc}{section}{\nameref{sec:533}}
\begin{longtable}{l p{0.5cm} r}
ای باد بامدادی خوش می‌روی به شادی
&&
پیوند روح کردی پیغام دوست دادی
\\
بر بوستان گذشتی یا در بهشت بودی
&&
شاد آمدی و خرم فرخنده بخت بادی
\\
تا من در این سرایم این در ندیده بودم
&&
کامروز پیش چشمم در بوستان گشادی
\\
چون گل روند و آیند این دلبران و خوبان
&&
تو در برابر من چون سرو بایستادی
\\
ایدون که می‌نماید در روزگار حسنت
&&
بس فتنه‌ها بزاید تو فتنه از که زادی
\\
اول چراغ بودی آهسته شمع گشتی
&&
آسان فراگرفتم در خرمن اوفتادی
\\
خواهم که بامدادی بیرون روی به صحرا
&&
تا بوستان بریزد گل‌های بامدادی
\\
یاری که با قرینی الفت گرفته باشد
&&
هر وقت یادش آید تو دم به دم به یادی
\\
گر در غمت بمیرم شادی به روزگارت
&&
پیوسته نیکوان را غم خورده‌اند و شادی
\\
جایی که داغ گیرد دردش دوا پذیرد
&&
آن است داغ سعدی کاول نظر نهادی
\\
\end{longtable}
\end{center}
