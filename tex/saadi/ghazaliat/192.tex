\begin{center}
\section*{غزل ۱۹۲: گر آن مراد شبی در کنار ما باشد}
\label{sec:192}
\addcontentsline{toc}{section}{\nameref{sec:192}}
\begin{longtable}{l p{0.5cm} r}
گر آن مراد شبی در کنار ما باشد
&&
زهی سعادت و دولت که یار ما باشد
\\
اگر هزار غم است از جهانیان بر دل
&&
همین بس است که او غمگسار ما باشد
\\
به کنج غاری عزلت گزینم از همه خلق
&&
گر آن لطیف جهان یار غار ما باشد
\\
از آن طرف نپذیرد کمال او نقصان
&&
وزین جهت شرف روزگار ما باشد
\\
جفای پرده درانم تفاوتی نکند
&&
اگر عنایت او پرده دار ما باشد
\\
مراد خاطر ما مشکل است و مشکل نیست
&&
اگر مراد خداوندگار ما باشد
\\
به اختیار قضای زمان بباید ساخت
&&
که دایم آن نبود کاختیار ما باشد
\\
وگر به دست نگارین دوست کشته شویم
&&
میان عالمیان افتخار ما باشد
\\
به هیچ کار نیایم گرم تو نپسندی
&&
وگر قبول کنی کار کار ما باشد
\\
نگارخانه چینی که وصف می‌گویند
&&
نه ممکن است که مثل نگار ما باشد
\\
چنین غزال که وصفش همی‌رود سعدی
&&
گمان مبر که به تنها شکار ما باشد
\\
\end{longtable}
\end{center}
