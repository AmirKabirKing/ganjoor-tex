\begin{center}
\section*{غزل ۶۰۸: ای سرو حدیقه معانی}
\label{sec:608}
\addcontentsline{toc}{section}{\nameref{sec:608}}
\begin{longtable}{l p{0.5cm} r}
ای سرو حدیقه معانی
&&
جانی و لطیفه جهانی
\\
پیش تو به اتفاق مردن
&&
خوشتر که پس از تو زندگانی
\\
چشمان تو سحر اولین اند
&&
تو فتنه آخرالزمانی
\\
چون اسم تو در میان نباشد
&&
گویی که به جسم در میانی
\\
آن را که تو از سفر بیایی
&&
حاجت نبود به ارمغانی
\\
گر ز آمدنت خبر بیارند
&&
من جان بدهم به مژدگانی
\\
دفع غم دل نمی‌توان کرد
&&
الا به امید شادمانی
\\
گر صورت خویشتن ببینی
&&
حیران وجود خود بمانی
\\
گر صلح کنی لطیف باشد
&&
در وقت بهار و مهربانی
\\
سعدی خط سبز دوست دارد
&&
پیرامن خد ارغوانی
\\
این پیر نگر که همچنانش
&&
از یاد نمی‌رود جوانی
\\
\end{longtable}
\end{center}
