\begin{center}
\section*{غزل ۳۵۷: حکایت از لب شیرین دهان سیم اندام}
\label{sec:357}
\addcontentsline{toc}{section}{\nameref{sec:357}}
\begin{longtable}{l p{0.5cm} r}
حکایت از لب شیرین دهان سیم اندام
&&
تفاوتی نکند گر دعاست یا دشنام
\\
حریف دوست که از خویشتن خبر دارد
&&
شراب صرف محبت نخورده است تمام
\\
اگر ملول شوی یا ملامتم گویی
&&
اسیر عشق نیندیشد از ملال و ملام
\\
من آن نیم که به جور از مراد بگریزم
&&
به آستین نرود مرغ پای بسته به دام
\\
بسی نماند که پنجاه ساله عاقل را
&&
به پنج روز به دیوانگی برآید نام
\\
مرا که با توام از هر که هست باکی نیست
&&
حریف خاص نیندیشد از ملامت عام
\\
شب دراز نخفتم که دوستان گویند
&&
به سرزنش عجبا للمحب کیف ینام
\\
تو در کنار من آیی من این طمع نکنم
&&
که می‌نیایدت از حسن وصف در اوهام
\\
ضرورت است که روزی بسوزد این اوراق
&&
که تاب آتش سعدی نیاورد اقلام
\\
\end{longtable}
\end{center}
