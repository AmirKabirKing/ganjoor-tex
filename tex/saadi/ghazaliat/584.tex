\begin{center}
\section*{غزل ۵۸۴: هرگز آن دل بنمیرد که تو جانش باشی}
\label{sec:584}
\addcontentsline{toc}{section}{\nameref{sec:584}}
\begin{longtable}{l p{0.5cm} r}
هرگز آن دل بنمیرد که تو جانش باشی
&&
نیکبخت آن که تو در هر دو جهانش باشی
\\
غم و اندیشه در آن دایره هرگز نرود
&&
به حقیقت که تو چون نقطه میانش باشی
\\
هرگزش باد صبا برگ پریشان نکند
&&
بوستانی که چو تو سرو روانش باشی
\\
همه عالم نگران تا نظر بخت بلند
&&
بر که افتد که تو یک دم نگرانش باشی
\\
تشنگانت به لب ای چشمه حیوان مردند
&&
تشنه‌تر آن که تو نزدیک دهانش باشی
\\
گر توان بود که دور فلک از سر گیرند
&&
تو دگر نادره دور زمانش باشی
\\
وصفت آن نیست که در وهم سخندان گنجد
&&
ور کسی گفت مگر هم تو زبانش باشی
\\
چون تحمل نکند بار فراق تو کسی
&&
با همه درد دل آسایش جانش باشی
\\
ای که بی دوست به سر می‌نتوانی که بری
&&
شاید ار محتمل بار گرانش باشی
\\
سعدی آن روز که غوغای قیامت باشد
&&
چشم دارد که تو منظور نهانش باشی
\\
\end{longtable}
\end{center}
