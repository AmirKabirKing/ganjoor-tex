\begin{center}
\section*{غزل ۳۶۱: ماه چنین کس ندید خوش سخن و کش خرام}
\label{sec:361}
\addcontentsline{toc}{section}{\nameref{sec:361}}
\begin{longtable}{l p{0.5cm} r}
ماه چنین کس ندید خوش سخن و کش خرام
&&
ماه مبارک طلوع سرو قیامت قیام
\\
سرو درآید ز پای گر تو بجنبی ز جای
&&
ماه بیفتد به زیر گر تو برآیی به بام
\\
تا دل از آن تو شد دیده فرودوختم
&&
هر چه پسند شماست بر همه عالم حرام
\\
گوش دلم بر در است تا چه بیاید خبر
&&
چشم امیدم به راه تا که بیارد پیام
\\
دعوت بی شمع را هیچ نباشد فروغ
&&
مجلس بی دوست را هیچ نباشد نظام
\\
در همه عمرم شبی بی‌خبر از در درآی
&&
تا شب درویش را صبح برآید به شام
\\
بار غمت می‌کشم وز همه عالم خوشم
&&
گر نکند التفات یا نکند احترام
\\
رای خداوند راست حاکم و فرمانرواست
&&
گر بکشد بنده‌ایم ور بنوازد غلام
\\
ای که ملامت کنی عارف دیوانه را
&&
شاهد ما حاضر است گر تو ندانی کدام
\\
گو به سلام من آی با همه تندی و جور
&&
وز من بی‌دل ستان جان به جواب سلام
\\
سعدی اگر طالبی راه رو و رنج بر
&&
یا برسد جان به حلق یا برسد دل به کام
\\
\end{longtable}
\end{center}
