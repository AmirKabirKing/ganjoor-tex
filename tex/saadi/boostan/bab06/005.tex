\begin{center}
\section*{بخش ۵ - حکایت در مذلت بسیار خوردن: چه آوردم از بصره دانی عجب}
\label{sec:005}
\addcontentsline{toc}{section}{\nameref{sec:005}}
\begin{longtable}{l p{0.5cm} r}
چه آوردم از بصره دانی عجب
&&
حدیثی که شیرین تر است از رطب
\\
تنی چند در خرقه راستان
&&
گذشتیم بر طرف خرماستان
\\
یکی در میان معده انبار بود
&&
ز پر خواری خویش بس خوار بود
\\
میان بست مسکین و شد بر درخت
&&
وز آنجا به گردن در افتاد سخت
\\
نه هر بار خرما توان خورد و برد
&&
لت انبان بد عاقبت خورد و مرد
\\
رئیس ده آمد که این را که کشت؟
&&
بگفتم مزن بانگ بر ما درشت
\\
شکم دامن اندر کشیدش ز شاخ
&&
بود تنگدل رودگانی فراخ
\\
شکم بند دست است و زنجیر پای
&&
شکم بنده نادر پرستد خدای
\\
سراسر شکم شد ملخ لاجرم
&&
به پایش کشد مور کوچک شکم
\\
برو اندرونی به دست آر، پاک
&&
شکم پر نخواهد شد الّا به خاک
\\
\end{longtable}
\end{center}
