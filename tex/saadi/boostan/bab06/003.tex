\begin{center}
\section*{بخش ۳ - حکایت: یکی پر طمع پیش خوارزمشاه}
\label{sec:003}
\addcontentsline{toc}{section}{\nameref{sec:003}}
\begin{longtable}{l p{0.5cm} r}
یکی پر طمع پیش خوارزمشاه
&&
شنیدم که شد بامدادی پگاه
\\
چو دیدش به خدمت دوتا گشت و راست
&&
دگر روی بر خاک مالید و خاست
\\
پسر گفتش ای بابک نامجوی
&&
یکی مشکلت می‌بپرسم بگوی
\\
نگفتی که قبله‌ست سوی حجاز
&&
چرا کردی امروز از این سو نماز؟
\\
مبر طاعت نفس شهوت پرست
&&
که هر ساعتش قبلهٔ دیگر است
\\
مبر ای برادر به فرمانش دست
&&
که هر کس که فرمان نبردش برست
\\
قناعت سرافرازد ای مرد هوش
&&
سر پر طمع بر نیاید ز دوش
\\
طمع آبروی توقر بریخت
&&
برای دو جو دامنی در بریخت
\\
چو سیراب خواهی شدن ز آب جوی
&&
چرا ریزی از بهر برف آبروی؟
\\
مگر از تنعم شکیبا شوی
&&
وگرنه ضرورت به درها شوی
\\
برو خواجه کوتاه کن دست آز
&&
چه می‌بایدت ز آستین دراز؟
\\
کسی را که درج طمع در نوشت
&&
نباید به کس عبد و خادم نبشت
\\
توقع براند ز هر مجلست
&&
بران از خودش تا نراند کست
\\
\end{longtable}
\end{center}
