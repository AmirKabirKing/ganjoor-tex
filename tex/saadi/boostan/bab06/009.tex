\begin{center}
\section*{بخش ۹ - حکایت: یکی نان خورش جز پیازی نداشت}
\label{sec:009}
\addcontentsline{toc}{section}{\nameref{sec:009}}
\begin{longtable}{l p{0.5cm} r}
یکی نان خورش جز پیازی نداشت
&&
چو دیگر کسان برگ و سازی نداشت
\\
کسی گفتش ای سغبهٔ خاکسار
&&
برو طبخی از خوان یغما بیار
\\
بخواه و مدار ای پسر شرم و باک
&&
که مقطوع روزی بود شرمناک
\\
قبا بست و چابک نوردید دست
&&
قبایش دریدند و دستش شکست
\\
همی گفت و بر خویشتن می‌گریست
&&
که مر خویشتن کرده را چاره چیست؟
\\
بلا جوی باشد گرفتار آز
&&
من و خانه من بعد و نان و پیاز
\\
جوینی که از سعی بازو خورم
&&
به از میده بر خوان اهل کرم
\\
چه دلتنگ خفت آن فرومایه دوش
&&
که بر سفرهٔ دیگران داشت گوش
\\
\end{longtable}
\end{center}
