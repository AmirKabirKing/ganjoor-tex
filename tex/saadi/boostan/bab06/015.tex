\begin{center}
\section*{بخش ۱۵ - حکایت در معنی آسانی پس از دشواری: شنیدم ز پیران شیرین سخن}
\label{sec:015}
\addcontentsline{toc}{section}{\nameref{sec:015}}
\begin{longtable}{l p{0.5cm} r}
شنیدم ز پیران شیرین سخن
&&
که بود اندر این شهر پیری کهن
\\
بسی دیده شاهان و دوران و امر
&&
سرآورده عمری ز تاریخ عمرو
\\
درخت کهن میوه‌ای تازه داشت
&&
که شهر از نکویی پرآوازه داشت
\\
عجب در زنخدان آن دل فریب
&&
که هرگز نبوده‌ست بر سرو سیب
\\
ز شوخی و مردم خراشیدنش
&&
فرج دید در سر تراشیدنش
\\
به موسی، کهن عمر کوته امید
&&
سرش کرد چون دست موسی سپید
\\
ز سر تیزی آن آهنین دل که بود
&&
به عیب پری‌رخ زبان برگشود
\\
به مویی که کرد از نکوییش کم
&&
نهادند حالی سرش در شکم
\\
چو چنگ از خجالت سر خوبروی
&&
نگونسار و در پیشش افتاده موی
\\
یکی را که خاطر در او رفته بود
&&
چو چشمان دلبندش آشفته بود
\\
کسی گفت جور آزمودی و درد
&&
دگر گرد سودای باطل مگرد
\\
ز مهرش بگردان چو پروانه پشت
&&
که مقراض، شمع جمالش بکشت
\\
برآمد خروش از هوادار چست
&&
که تردامنان را بود عهد سست
\\
پسر خوش منش باید و خوبروی
&&
پدر گو به جهلش بینداز موی
\\
مرا جان به مهرش برآمیخته‌ست
&&
نه خاطر به مویی در آویخته‌ست
\\
چو روی نکو داری انده مخور
&&
که موی ار بیفتد بروید دگر
\\
نه پیوسته رز خوشهٔ تر دهد
&&
گهی برگ ریزد، گهی بر دهد
\\
بزرگان چو خور در حجاب اوفتند
&&
حسودان چو اخگر در آب اوفتند
\\
برون آید از زیر ابر آفتاب
&&
به تدریج و اخگر بمیرد در آب
\\
ز ظلمت مترس ای پسندیده دوست
&&
که ممکن بود کاب حیوان در اوست
\\
نه گیتی پس از جنبش آرام یافت؟
&&
نه سعدی سفر کرد تا کام یافت؟
\\
دل از بی مرادی به فکرت مسوز
&&
شب آبستن است ای برادر به روز
\\
\end{longtable}
\end{center}
