\begin{center}
\section*{بخش ۹ - حکایت در معنی تواضع نیکمردان: شنیدم که فرزانه‌ای حق پرست}
\label{sec:009}
\addcontentsline{toc}{section}{\nameref{sec:009}}
\begin{longtable}{l p{0.5cm} r}
شنیدم که فرزانه‌ای حق پرست
&&
گریبان گرفتش یکی رند مست
\\
از آن تیره دل مرد صافی درون
&&
قفا خورد و سر بر نکرد از سکون
\\
یکی گفتش آخر نه مردی تو نیز؟
&&
تحمل دریغ است از این بی تمیز
\\
شنید این سخن مرد پاکیزه خوی
&&
بدو گفت از این نوع با من مگوی
\\
درد مست نادان گریبان مرد
&&
که با شیر جنگی سگالد نبرد
\\
ز هشیار عاقل نزیبد که دست
&&
زند در گریبان نادان مست
\\
هنرور چنین زندگانی کند
&&
جفا بیند و مهربانی کند
\\
\end{longtable}
\end{center}
