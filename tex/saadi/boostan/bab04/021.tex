\begin{center}
\section*{بخش ۲۱ - حکایت لقمان حکیم: شنیدم که لقمان سیه‌فام بود}
\label{sec:021}
\addcontentsline{toc}{section}{\nameref{sec:021}}
\begin{longtable}{l p{0.5cm} r}
شنیدم که لقمان سیه‌فام بود
&&
نه تن‌پرور و نازک اندام بود
\\
یکی بندهٔ خویش پنداشتش
&&
زبون دید و در کار گل داشتش
\\
جفا دید و با جور و قهرش بساخت
&&
به سالی سرایی ز بهرش بساخت
\\
چو پیش آمدش بندهٔ رفته باز
&&
ز لقمانش آمد نهیبی فراز
\\
به پایش در افتاد و پوزش نمود
&&
بخندید لقمان که پوزش چه سود؟
\\
به سالی ز جورت جگر خون کنم
&&
به یک ساعت از دل به در چون کنم؟
\\
ولی هم ببخشایم ای نیکمرد
&&
که سود تو ما را زیانی نکرد
\\
تو آباد کردی شبستان خویش
&&
مرا حکمت و معرفت گشت بیش
\\
غلامی است در خیلم ای نیکبخت
&&
که فرمایمش وقتها کار سخت
\\
دگر ره نیازارمش سخت، دل
&&
چو یاد آیدم سختی کار گل
\\
هر آن کس که جور بزرگان نبرد
&&
نسوزد دلش بر ضعیفان خرد
\\
گر از حاکمان سختت آید سخن
&&
تو بر زیردستان درشتی مکن
\\
نکو گفت بهرام شه با وزیر
&&
که دشوار با زیردستان مگیر
\\
\end{longtable}
\end{center}
