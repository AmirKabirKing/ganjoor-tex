\begin{center}
\section*{بخش ۱۴ - حکایت: ملک صالح از پادشاهان شام}
\label{sec:014}
\addcontentsline{toc}{section}{\nameref{sec:014}}
\begin{longtable}{l p{0.5cm} r}
ملک صالح از پادشاهان شام
&&
برون آمدی صبحدم با غلام
\\
بگشتی در اطراف بازار و کوی
&&
به رسم عرب نیمه بر بسته روی
\\
که صاحب نظر بود و درویش دوست
&&
هر آن کاین دو دارد ملک صالح اوست
\\
دو درویش در مسجدی خفته یافت
&&
پریشان دل و خاطر آشفته یافت
\\
شب سردشان دیده نابرده خواب
&&
چو حربا تأمل کنان آفتاب
\\
یکی زآن دو می گفت با دیگری
&&
که هم روز محشر بود داوری
\\
گر این پادشاهان گردن فراز
&&
که در لهو و عیشند و با کام و ناز
\\
در آیند با عاجزان در بهشت
&&
من از گور سر بر نگیرم ز خشت
\\
بهشت برین ملک و مأوای ماست
&&
که بند غم امروز بر پای ماست
\\
همه عمر از اینان چه دیدی خوشی
&&
که در آخرت نیز زحمت کشی؟
\\
اگر صالح آنجا به دیوار باغ
&&
بر آید، به کفشش بدرم دماغ
\\
چو مرد این سخن گفت و صالح شنید
&&
دگر بودن آنجا مصالح ندید
\\
دمی رفت تا چشمهٔ آفتاب
&&
ز چشم خلایق فرو شست خواب
\\
دوان هر دو را کس فرستاد و خواند
&&
به هیبت نشست و به حرمت نشاند
\\
بر ایشان ببارید باران جود
&&
فرو شستشان گرد ذل از وجود
\\
پس از رنج سرما و باران و سیل
&&
نشستند با نامداران خیل
\\
گدایان بی جامه شب کرده روز
&&
معطر کنان جامه بر عود سوز
\\
یکی گفت از اینان ملک را نهان
&&
که ای حلقه در گوش حکمت جهان
\\
پسندیدگان در بزرگی رسند
&&
ز ما بندگانت چه آمد پسند؟
\\
شهنشه ز شادی چو گل بر شکفت
&&
بخندید در روی درویش و گفت
\\
من آن کس نیم کز غرور حشم
&&
ز بیچارگان روی در هم کشم
\\
تو هم با من از سر بنه خوی زشت
&&
که ناسازگاری کنی در بهشت
\\
من امروز کردم در صلح باز
&&
تو فردا مکن در به رویم فراز
\\
چنین راه اگر مقبلی پیش گیر
&&
شرف بایدت دست درویش گیر
\\
بر از شاخ طوبی کسی بر نداشت
&&
که امروز تخم ارادت نکاشت
\\
ارادت نداری سعادت مجوی
&&
به چوگان خدمت توان برد گوی
\\
تو را کی بود چون چراغ التهاب
&&
که از خود پری همچو قندیل از آب؟
\\
وجودی دهد روشنایی به جمع
&&
که سوزیش در سینه باشد چو شمع
\\
\end{longtable}
\end{center}
