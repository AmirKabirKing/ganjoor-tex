\begin{center}
\section*{بخش ۳ - حکایت در معنی نظر مردان در خود به حقارت: جوانی خردمند پاکیزه بوم}
\label{sec:003}
\addcontentsline{toc}{section}{\nameref{sec:003}}
\begin{longtable}{l p{0.5cm} r}
جوانی خردمند پاکیزه بوم
&&
ز دریا بر آمد به دربند روم
\\
در او فضل دیدند و فقر و تمیز
&&
نهادند رختش به جایی عزیز
\\
سر صالحان گفت روزی به مرد
&&
که خاشاک مسجد بیفشان و گرد
\\
همان کاین سخن مرد رهرو شنید
&&
برون رفت و بازش کس آنجا ندید
\\
بر آن حمل کردند یاران و پیر
&&
که پروای خدمت نبودش فقیر
\\
دگر روز خادم گرفتش به راه
&&
که ناخوب کردی به رأی تباه
\\
ندانستی ای کودک خودپسند
&&
که مردان ز خدمت به جایی رسند
\\
گرستن گرفت از سر صدق و سوز
&&
که ای یار جان پرور دلفروز
\\
نه گرد اندر آن بقعه دیدم نه خاک
&&
من آلوده بودم در آن جای پاک
\\
گرفتم قدم لاجرم باز پس
&&
که پاکیزه به مسجد از خاک و خس
\\
طریقت جز این نیست درویش را
&&
که افکنده دارد تن خویش را
\\
بلندیت باید تواضع گزین
&&
که آن بام را نیست سلم جز این
\\
\end{longtable}
\end{center}
