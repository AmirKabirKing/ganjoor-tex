\begin{center}
\section*{بخش ۱ - سرآغاز: به نام خداوند جان آفرین}
\label{sec:001}
\addcontentsline{toc}{section}{\nameref{sec:001}}
\begin{longtable}{l p{0.5cm} r}
به نام خداوند جان آفرین
&&
حکیم سخن در زبان آفرین
\\
خداوند بخشندهٔ دستگیر
&&
کریم خطا بخش پوزش پذیر
\\
عزیزی که هر کز درش سر بتافت
&&
به هر در که شد هیچ عزت نیافت
\\
سر پادشاهان گردن فراز
&&
به درگاه او بر زمین نیاز
\\
نه گردن کشان را بگیرد به فور
&&
نه عذرآوران را براند به جور
\\
وگر خشم گیرد ز کردار زشت
&&
چو بازآمدی ماجرا در نوشت
\\
اگر با پدر جنگ جوید کسی
&&
پدر بی گمان خشم گیرد بسی
\\
وگر خویش راضی نباشد ز خویش
&&
چو بیگانگانش براند ز پیش
\\
وگر بنده چابک نباشد به کار
&&
عزیزش ندارد خداوندگار
\\
وگر بر رفیقان نباشی شفیق
&&
به فرسنگ بگریزد از تو رفیق
\\
وگر ترک خدمت کند لشکری
&&
شود شاه لشکرکش از وی بری
\\
ولیکن خداوند بالا و پست
&&
به عصیان در رزق بر کس نبست
\\
دو کونش یکی قطره از بحر علم
&&
گنه بیند و پرده پوشد به حلم
\\
ادیم زمین، سفرهٔ عام اوست
&&
بر این خوان یغما چه دشمن چه دوست
\\
اگر بر جفا پیشه بشتافتی
&&
که از دست قهرش امان یافتی؟
\\
بری ذاتش از تهمت ضد و جنس
&&
غنی ملکش از طاعت جن و انس
\\
پرستار امرش همه چیز و کس
&&
بنی آدم و مرغ و مور و مگس
\\
چنان پهن خوان کرم گسترد
&&
که سیمرغ در قاف قسمت خورد
\\
لطیف کرم گستر کارساز
&&
که دارای خلق است و دانای راز
\\
مر او را رسد کبریا و منی
&&
که ملکش قدیم است و ذاتش غنی
\\
یکی را به سر برنهد تاج بخت
&&
یکی را به خاک اندر آرد ز تخت
\\
کلاه سعادت یکی بر سرش
&&
گلیم شقاوت یکی در برش
\\
گلستان کند آتشی بر خلیل
&&
گروهی بر آتش برد ز آب نیل
\\
گر آن است، منشور احسان اوست
&&
ور این است، توقیع فرمان اوست
\\
پس پرده بیند عملهای بد
&&
هم او پرده پوشد به آلای خود
\\
به تهدید اگر برکشد تیغ حکم
&&
بمانند کروبیان صم و بکم
\\
وگر در دهد یک صلای کرم
&&
عزازیل گوید نصیبی برم
\\
به درگاه لطف و بزرگیش بر
&&
بزرگان نهاده بزرگی ز سر
\\
فروماندگان را به رحمت قریب
&&
تضرع کنان را به دعوت مجیب
\\
بر احوال نابوده، علمش بصیر
&&
به اسرار ناگفته، لطفش خبیر
\\
به قدرت، نگهدار بالا و شیب
&&
خداوند دیوان روز حسیب
\\
نه مستغنی از طاعتش پشت کس
&&
نه بر حرف او جای انگشت کس
\\
قدیمی نکوکار نیکی پسند
&&
به کلک قضا در رحم نقش بند
\\
ز مشرق به مغرب مه و آفتاب
&&
روان کرد و بنهاد گیتی بر آب
\\
زمین از تب لرزه آمد ستوه
&&
فرو کوفت بر دامنش میخ کوه
\\
دهد نطفه را صورتی چون پری
&&
که کرده‌ست بر آب صورتگری؟
\\
نهد لعل و پیروزه در صلب سنگ
&&
گل و لعل در شاخ پیروزه رنگ
\\
ز ابر افکند قطره‌ای سوی یم
&&
ز صلب افکند نطفه‌ای در شکم
\\
از آن قطره لولوی لالا کند
&&
وز این، صورتی سرو بالا کند
\\
بر او علم یک ذره پوشیده نیست
&&
که پیدا و پنهان به نزدش یکیست
\\
مهیاکن روزی مار و مور
&&
اگر چند بی‌دست و پایند و زور
\\
به امرش وجود از عدم نقش بست
&&
که داند جز او کردن از نیست، هست؟
\\
دگر ره به کتم عدم در برد
&&
وز انجا به صحرای محشر برد
\\
جهان متفق بر الهیتش
&&
فرومانده از کنه ماهیتش
\\
بشر ماورای جلالش نیافت
&&
بصر منتهای جمالش نیافت
\\
نه بر اوج ذاتش پرد مرغ وهم
&&
نه در ذیل وصفش رسد دست فهم
\\
در این ورطه کشتی فروشد هزار
&&
که پیدا نشد تخته‌ای بر کنار
\\
چه شبها نشستم در این سیر، گم
&&
که دهشت گرفت آستینم که قم
\\
محیط است علم ملک بر بسیط
&&
قیاس تو بر وی نگردد محیط
\\
نه ادراک در کنه ذاتش رسید
&&
نه فکرت به غور صفاتش رسید
\\
توان در بلاغت به سحبان رسید
&&
نه در کنه بی چون سبحان رسید
\\
که خاصان در این ره فرس رانده‌اند
&&
به لااحصی از تک فرومانده‌اند
\\
نه هر جای مرکب توان تاختن
&&
که جاها سپر باید انداختن
\\
وگر سالکی محرم راز گشت
&&
ببندند بر وی در بازگشت
\\
کسی را در این بزم ساغر دهند
&&
که داروی بیهوشیش در دهند
\\
یکی باز را دیده بردوخته‌ست
&&
یکی دیدها باز و پر سوخته‌ست
\\
کسی ره سوی گنج قارون نبرد
&&
وگر برد، ره باز بیرون نبرد
\\
بمردم در این موج دریای خون
&&
کز او کس نبرده‌ست کشتی برون
\\
اگر طالبی کاین زمین طی کنی
&&
نخست اسب باز آمدن پی کنی
\\
تأمل در آیینهٔ دل کنی
&&
صفایی به تدریج حاصل کنی
\\
مگر بویی از عشق مستت کند
&&
طلبکار عهد الستت کند
\\
به پای طلب ره بدان جا بری
&&
وز آنجا به بال محبت پری
\\
بدرد یقین پرده‌های خیال
&&
نماند سراپرده الا جلال
\\
دگر مرکب عقل را پویه نیست
&&
عنانش بگیرد تحیر که بیست
\\
در این بحر جز مرد راعی نرفت
&&
گم آن شد که دنبال داعی نرفت
\\
کسانی کز این راه برگشته‌اند
&&
برفتند بسیار و سرگشته‌اند
\\
خلاف پیمبر کسی ره گزید
&&
که هرگز به منزل نخواهد رسید
\\
مپندار سعدی که راه صفا
&&
توان رفت جز بر پی مصطفی
\\
\end{longtable}
\end{center}
