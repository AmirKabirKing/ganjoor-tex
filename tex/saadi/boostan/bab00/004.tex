\begin{center}
\section*{بخش ۴ - مدح ابوبکر بن سعد بن زنگی: مرا طبع از این نوع خواهان نبود}
\label{sec:004}
\addcontentsline{toc}{section}{\nameref{sec:004}}
\begin{longtable}{l p{0.5cm} r}
مرا طبع از این نوع خواهان نبود
&&
سر مدحت پادشاهان نبود
\\
ولی نظم کردم به نام فلان
&&
مگر باز گویند صاحبدلان
\\
که سعدی که گوی بلاغت ربود
&&
در ایام بوبکر بن سعد بود
\\
سزد گر به دورش بنازم چنان
&&
که سید به دوران نوشیروان
\\
جهانبان دین پرور دادگر
&&
نیامد چو بوبکر بعد از عمر
\\
سر سرفرازان و تاج مهان
&&
به دوران عدلش بناز، ای جهان
\\
گر از فتنه آید کسی در پناه
&&
ندارد جز این کشور آرامگاه
\\
فطوبی لباب کبیت العتیق
&&
حوالیه من کل فج عمیق
\\
ندیدم چنین گنج و ملک و سریر
&&
که وقف است بر طفل و درویش و پیر
\\
نیامد برش دردناک غمی
&&
که ننهاد بر خاطرش مرهمی
\\
طلبکار خیر است امیدوار
&&
خدایا امیدی که دارد برآر
\\
کله گوشه بر آسمان برین
&&
هنوز از تواضع سرش بر زمین
\\
گدا گر تواضع کند خوی اوست
&&
ز گردن فرازان تواضع نکوست
\\
اگر زیردستی بیفتد چه خاست؟
&&
زبردست افتاده مرد خداست
\\
نه ذکر جمیلش نهان می‌رود
&&
که صیت کرم در جهان می‌رود
\\
چنویی خردمند فرخ نژاد
&&
ندارد جهان تا جهان است، یاد
\\
نبینی در ایام او رنجه‌ای
&&
که نالد ز بیداد سرپنجه‌ای
\\
کس این رسم و ترتیب و آیین ندید
&&
فریدون با آن شکوه، این ندید
\\
از آن پیش حق پایگاهش قوی است
&&
که دست ضعیفان به جاهش قوی است
\\
چنان سایه گسترده بر عالمی
&&
که زالی نیندیشد از رستمی
\\
همه وقت مردم ز جور زمان
&&
بنالند و از گردش آسمان
\\
در ایام عدل تو ای شهریار
&&
ندارد شکایت کس از روزگار
\\
به عهد تو می‌بینم آرام خلق
&&
پس از تو ندانم سرانجام خلق
\\
هم از بخت فرخنده فرجام توست
&&
که تاریخ سعدی در ایام توست
\\
که تا بر فلک ماه و خورشید هست
&&
در این دفترت ذکر جاوید هست
\\
ملوک ار نکو نامی اندوختند
&&
ز پیشینگان سیرت آموختند
\\
تو در سیرت پادشاهی خویش
&&
سبق بردی از پادشاهان پیش
\\
سکندر به دیوار رویین و سنگ
&&
بکرد از جهان راه یأجوج تنگ
\\
تو را سد یأجوج کفر از زر است
&&
نه رویین چو دیوار اسکندر است
\\
زبان آوری کاندر این امن و داد
&&
سپاست نگوید زبانش مباد
\\
زهی بحر بخشایش و کان جود
&&
که مستظهرند از وجودت وجود
\\
برون بینم اوصاف شاه از حساب
&&
نگنجد در این تنگ میدان کتاب
\\
گر آن جمله را سعدی انشا کند
&&
مگر دفتری دیگر املا کند
\\
فروماندم از شکر چندین کرم
&&
همان به که دست دعا گسترم
\\
جهانت به کام و فلک یار باد
&&
جهان آفرینت نگهدار باد
\\
بلند اخترت عالم افروخته
&&
زوال اختر دشمنت سوخته
\\
غم از گردش روزگارت مباد
&&
وز اندیشه بر دل غبارت مباد
\\
که بر خاطر پادشاهان غمی
&&
پریشان کند خاطر عالمی
\\
دل و کشورت جمع و معمور باد
&&
ز ملکت پراکندگی دور باد
\\
تنت باد پیوسته چون دین، درست
&&
بداندیش را دل چو تدبیر، سست
\\
درونت به تأیید حق شاد باد
&&
دل و دین و اقلیمت آباد باد
\\
جهان آفرین بر تو رحمت کناد
&&
دگر هرچه گویم فسانه‌ست و باد
\\
همینت بس از کردگار مجید
&&
که توفیق خیرت بود بر مزید
\\
نرفت از جهان سعد زنگی به درد
&&
که چون تو خلف نامبردار کرد
\\
عجب نیست این فرع از آن اصل پاک
&&
که جانش بر اوج است و جسمش به خاک
\\
خدایا بر آن تربت نامدار
&&
به فضلت که باران رحمت ببار
\\
گر از سعد زنگی مثل ماند یاد
&&
فلک یاور سعد بوبکر باد
\\
\end{longtable}
\end{center}
