\begin{center}
\section*{بخش ۲ - فی نعت سید المرسلین علیه الصلوة و السلام: کریم السجایا جمیل الشیم}
\label{sec:002}
\addcontentsline{toc}{section}{\nameref{sec:002}}
\begin{longtable}{l p{0.5cm} r}
کریم السجایا جمیل الشیم
&&
نبی البرایا شفیع الامم
\\
امام رسل، پیشوای سبیل
&&
امین خدا، مهبط جبرئیل
\\
شفیع الوری، خواجه بعث و نشر
&&
امام الهدی، صدر دیوان حشر
\\
کلیمی که چرخ فلک طور اوست
&&
همه نورها پرتو نور اوست
\\
شفیع مطاع نبی کریم
&&
قسیم جسیم نسیم وسیم
\\
یتیمی که ناکرده قرآن درست
&&
کتب خانهٔ چند ملت بشست
\\
چو عزمش برآهخت شمشیر بیم
&&
به معجز میان قمر زد دو نیم
\\
چو صیتش در افواه دنیا فتاد
&&
تزلزل در ایوان کسری فتاد
\\
به لا قامت لات بشکست خرد
&&
به اعزاز دین آب عزی ببرد
\\
نه از لات و عزی برآورد گرد
&&
که تورات و انجیل منسوخ کرد
\\
شبی بر نشست از فلک برگذشت
&&
به تمکین و جاه از ملک درگذشت
\\
چنان گرم در تیه قربت براند
&&
که بر سدره جبریل از او بازماند
\\
بدو گفت سالار بیت‌الحرام
&&
که ای حامل وحی برتر خرام
\\
چو در دوستی مخلصم یافتی
&&
عنانم ز صحبت چرا تافتی؟
\\
بگفتا فراتر مجالم نماند
&&
بماندم که نیروی بالم نماند
\\
اگر یک سر موی برتر پرم
&&
فروغ تجلی بسوزد پرم
\\
نماند به عصیان کسی در گرو
&&
که دارد چنین سیدی پیشرو
\\
چه نعت پسندیده گویم تو را؟
&&
علیک السلام ای نبی الوری
\\
درود ملک بر روان تو باد
&&
بر اصحاب و بر پیروان تو باد
\\
نخستین ابوبکر پیر مرید
&&
عمر، پنجه بر پیچ دیو مرید
\\
خردمند عثمان شب زنده‌دار
&&
چهارم علی، شاه دلدل سوار
\\
خدایا به حق بنی فاطمه
&&
که بر قولم ایمان کنم خاتمه
\\
اگر دعوتم رد کنی ور قبول
&&
من و دست و دامان آل رسول
\\
چه کم گردد ای صدر فرخنده پی
&&
ز قدر رفیعت به درگاه حی
\\
که باشند مشتی گدایان خیل
&&
به مهمان دارالسلامت طفیل
\\
خدایت ثنا گفت و تبجیل کرد
&&
زمین بوس قدر تو جبریل کرد
\\
بلند آسمان پیش قدرت خجل
&&
تو مخلوق و آدم هنوز آب و گل
\\
تو اصل وجود آمدی از نخست
&&
دگر هرچه موجود شد فرع توست
\\
ندانم کدامین سخن گویمت
&&
که والاتری زانچه من گویمت
\\
تو را عز لولاک تمکین بس است
&&
ثنای تو طه و یس بس است
\\
چه وصفت کند سعدی ناتمام؟
&&
علیک الصلوة ای نبی السلام
\\
\end{longtable}
\end{center}
