\begin{center}
\section*{بخش ۴ - حکایت در معنی تحمل محب صادق: شنیدم که وقتی گدازاده‌ای}
\label{sec:004}
\addcontentsline{toc}{section}{\nameref{sec:004}}
\begin{longtable}{l p{0.5cm} r}
شنیدم که وقتی گدازاده‌ای
&&
نظر داشت با پادشازاده‌ای
\\
همی‌رفت و می‌پخت سودای خام
&&
خیالش فرو برده دندان به کام
\\
ز میدانش خالی نبودی چو میل
&&
همه وقت پهلوی اسبش چو پیل
\\
دلش خون شد و راز در دل بماند
&&
ولی پایش از گریه در گل بماند
\\
رقیبان خبر یافتندش ز درد
&&
دگرباره گفتندش اینجا مگرد
\\
دمی رفت و یاد آمدش روی دوست
&&
دگر خیمه زد بر سر کوی دوست
\\
غلامی شکستش سر و دست و پای
&&
که باری نگفتیمت ایدر مپای
\\
دگر رفت و صبر و قرارش نبود
&&
شکیبایی از روی یارش نبود
\\
مگس وارش از پیش شکر به جور
&&
براندندی و بازگشتی بفور
\\
کسی گفتش ای شوخ دیوانه رنگ
&&
عجب صبر داری تو بر چوب و سنگ!
\\
بگفت این جفا بر من از دست اوست
&&
نه شرطیست نالیدن از دست دوست
\\
من اینک دم دوستی می‌زنم
&&
گر او دوست دارد وگر دشمنم
\\
ز من صبر بی او توقع مدار
&&
که با او هم امکان ندارد قرار
\\
نه نیروی صبرم نه جای ستیز
&&
نه امکان بودن نه پای گریز
\\
مگو زین در بارگه سر بتاب
&&
وگر سر چو میخم نهد در طناب
\\
نه پروانه جان داده در پای دوست
&&
به از زنده در کنج تاریک اوست؟
\\
بگفت ار خوری زخم چوگان اوی؟
&&
بگفتا به پایش در افتم چو گوی
\\
بگفتا سرت گر ببرد به تیغ؟
&&
بگفت این قدر نبود از وی دریغ
\\
مرا خود ز سر نیست چندان خبر
&&
که تاج است بر تارکم یا تبر
\\
مکن با من ناشکیبا عتیب
&&
که در عشق صورت نبندد شکیب
\\
چو یعقوبم ار دیده گردد سپید
&&
نبرم ز دیدار یوسف امید
\\
یکی را که سر خوش بود با یکی
&&
نیازارد از وی به هر اندکی
\\
رکابش ببوسید روزی جوان
&&
برآشفت و برتافت از وی عنان
\\
بخندید و گفتا عنان برمپیچ
&&
که سلطان عنان برنپیچد ز هیچ
\\
مرا با وجود تو هستی نماند
&&
به یاد توام خودپرستی نماند
\\
گرم جرم بینی مکن عیب من
&&
تویی سر بر آورده از جیب من
\\
بدان زهره دستت زدم در رکاب
&&
که خود را نیاوردم اندر حساب
\\
کشیدم قلم در سر نام خویش
&&
نهادم قدم بر سر کام خویش
\\
مرا خود کشد تیر آن چشم مست
&&
چه حاجت که آری به شمشیر دست؟
\\
تو آتش به نی در زن و در گذر
&&
که نه خشک در بیشه ماند نه تر
\\
\end{longtable}
\end{center}
