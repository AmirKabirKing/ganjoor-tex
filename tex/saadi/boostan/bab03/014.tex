\begin{center}
\section*{بخش ۱۴ - حکایت در معنی عزت محبوب در نظر محب: میان دو عم زاده وصلت فتاد}
\label{sec:014}
\addcontentsline{toc}{section}{\nameref{sec:014}}
\begin{longtable}{l p{0.5cm} r}
میان دو عم زاده وصلت فتاد
&&
دو خورشید سیمای مهتر نژاد
\\
یکی را به غایت خوش افتاده بود
&&
دگر نافر و سرکش افتاده بود
\\
یکی خلق و لطف پریوار داشت
&&
یکی روی در روی دیوار داشت
\\
یکی خویشتن را بیاراستی
&&
دگر مرگ خویش از خدا خواستی
\\
پسر را نشاندند پیران ده
&&
که مهرت بر او نیست مهرش بده
\\
بخندید و گفتا به صد گوسفند
&&
تغابن نباشد رهایی ز بند
\\
به ناخن پری چهره می‌کند پوست
&&
که هرگز بدین کی شکیبم ز دوست؟
\\
نه صد گوسفندم که سیصد هزار
&&
نباید به نادیدن روی یار
\\
تو را هر چه مشغول دارد ز دوست
&&
اگر راست خواهی دلارامت اوست
\\
یکی پیش شوریده حالی نبشت
&&
که دوزخ تمنا کنی یا بهشت؟
\\
بگفتا مپرس از من این ماجرا
&&
پسندیدم آنچ او پسندد مرا
\\
\end{longtable}
\end{center}
