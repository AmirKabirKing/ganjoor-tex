\begin{center}
\section*{بخش ۲۱ - حکایت صاحب نظر پارسا: یکی را چو من دل به دست کسی}
\label{sec:021}
\addcontentsline{toc}{section}{\nameref{sec:021}}
\begin{longtable}{l p{0.5cm} r}
یکی را چو من دل به دست کسی
&&
گرو بود و می‌برد خواری بسی
\\
پس از هوشمندی و فرزانگی
&&
به دف بر زدندش به دیوانگی
\\
ز دشمن جفا بردی از بهر دوست
&&
که تریاک اکبر بود زهر دوست
\\
قفا خوردی از دست یاران خویش
&&
چو مسمار پیشانی آورده پیش
\\
خیالش چنان بر سر آشوب کرد
&&
که بام دماغش لگدکوب کرد
\\
نبودش ز تشنیع یاران خبر
&&
که غرقه ندارد ز باران خبر
\\
کرا پای خاطر بر آمد به سنگ
&&
نیندیشد از شیشهٔ نام و ننگ
\\
شبی دیو خود را پریچهره ساخت
&&
در آغوش آن مرد و بر وی بتاخت
\\
سحرگه مجال نمازش نبود
&&
ز یاران کس آگه ز رازش نبود
\\
به آبی فرو رفت نزدیک بام
&&
بر او بسته سرما دری از رخام
\\
نصیحتگری لومش آغاز کرد
&&
که خود را بکشتی در این آب سرد
\\
ز برنای منصف بر آمد خروش
&&
که ای یار چند از ملامت؟ خموش
\\
مرا پنج روز این پسر دل فریفت
&&
ز مهرش چنانم که نتوان شکیفت
\\
نپرسید باری به خلق خوشم
&&
ببین تا چه بارش به جان می‌کشم
\\
پس آن را که شخصم ز خاک آفرید
&&
به قدرت در او جان پاک آفرید
\\
عجب داری ار بار امرش برم
&&
که دایم به احسان و فضلش درم؟
\\
\end{longtable}
\end{center}
