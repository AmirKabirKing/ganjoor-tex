\begin{center}
\section*{بخش ۲۵ - مخاطبه شمع و پروانه: شبی یاد دارم که چشمم نخفت}
\label{sec:025}
\addcontentsline{toc}{section}{\nameref{sec:025}}
\begin{longtable}{l p{0.5cm} r}
شبی یاد دارم که چشمم نخفت
&&
شنیدم که پروانه با شمع گفت
\\
که من عاشقم گر بسوزم رواست
&&
تو را گریه و سوز باری چراست؟
\\
بگفت ای هوادار مسکین من
&&
برفت انگبین یار شیرین من
\\
چو شیرینی از من به در می‌رود
&&
چو فرهادم آتش به سر می‌رود
\\
همی گفت و هر لحظه سیلاب درد
&&
فرو می‌دویدش به رخسار زرد
\\
که ای مدعی عشق کار تو نیست
&&
که نه صبر داری نه یارای ایست
\\
تو بگریزی از پیش یک شعله خام
&&
من استاده‌ام تا بسوزم تمام
\\
تو را آتش عشق اگر پر بسوخت
&&
مرا بین که از پای تا سر بسوخت
\\
همه شب در این گفت و گو بود شمع
&&
به دیدار او وقت اصحاب، جمع
\\
نرفته ز شب همچنان بهره‌ای
&&
که ناگه بکشتش پریچهره‌ای
\\
همی گفت و می‌رفت دودش به سر
&&
که این است پایان عشق، ای پسر
\\
اگر عاشقی خواهی آموختن
&&
به کشتن فرج یابی از سوختن
\\
مکن گریه بر گور مقتول دوست
&&
برو خرمی کن که مقبول اوست
\\
اگر عاشقی سر مشوی از مرض
&&
چو سعدی فرو شوی دست از غرض
\\
فدایی ندارد ز مقصود چنگ
&&
و گر بر سرش تیر بارند و سنگ
\\
به دریا مرو گفتمت زینهار
&&
وگر می‌روی تن به طوفان سپار
\\
\end{longtable}
\end{center}
