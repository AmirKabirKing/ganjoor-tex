\begin{center}
\section*{بخش ۹ - حکایت: شنیدم که پیری شبی زنده داشت}
\label{sec:009}
\addcontentsline{toc}{section}{\nameref{sec:009}}
\begin{longtable}{l p{0.5cm} r}
شنیدم که پیری شبی زنده داشت
&&
سحر دست حاجت به حق بر فراشت
\\
یکی هاتف انداخت در گوش پیر
&&
که بی حاصلی، رو سر خویش گیر
\\
بر این در دعای تو مقبول نیست
&&
به خواری برو یا به زاری بایست
\\
شب دیگر از ذکر و طاعت نخفت
&&
مریدی ز حالش خبر یافت، گفت
\\
چو دیدی کز آن روی بسته‌ست در
&&
به بی حاصلی سعی چندین مبر
\\
به دیباچه بر اشک یاقوت فام
&&
به حسرت ببارید و گفت ای غلام
\\
به نومیدی آنگه بگردیدمی
&&
از این ره، که راهی دگر دیدمی
\\
مپندار گر وی عنان بر شکست
&&
که من باز دارم ز فتراک دست
\\
چو خواهنده محروم گشت از دری
&&
چه غم گر شناسد در دیگری؟
\\
شنیدم که راهم در این کوی نیست
&&
ولی هیچ راه دگر روی نیست
\\
در این بود سر بر زمین فدا
&&
که گفتند در گوش جانش ندا
\\
قبول است اگر چه هنر نیستش
&&
که جز ما پناهی دگر نیستش
\\
\end{longtable}
\end{center}
