\begin{center}
\section*{بخش ۹ - حکایت عداوت در میان دو شخص: میان دو تن دشمنی بود و جنگ}
\label{sec:009}
\addcontentsline{toc}{section}{\nameref{sec:009}}
\begin{longtable}{l p{0.5cm} r}
میان دو تن دشمنی بود و جنگ
&&
سر از کبر بر یکدیگر چون پلنگ
\\
ز دیدار هم تا به حدی رمان
&&
که بر هر دو تنگ آمدی آسمان
\\
یکی را اجل در سر آورد جیش
&&
سرآمد بر او روزگاران عیش
\\
بداندیش او را درون شاد گشت
&&
به گورش پس از مدتی برگذشت
\\
شبستان گورش در اندوده دید
&&
که وقتی سرایش زر اندوده دید
\\
خرامان به بالینش آمد فراز
&&
همی گفت با خود لب از خنده باز
\\
خوشا وقت مجموع آن کس که اوست
&&
پس از مرگ دشمن در آغوش دوست
\\
پس از مرگ آن کس نباید گریست
&&
که روزی پس از مرگ دشمن بزیست
\\
ز روی عداوت به بازوی زور
&&
یکی تخته برکندش از روی گور
\\
سر تاجور دیدش اندر مغاک
&&
دو چشم جهان بینش آکنده خاک
\\
وجودش گرفتار زندان گور
&&
تنش طعمه کرم و تاراج مور
\\
چنان تنگش آکنده خاک استخوان
&&
که از عاج پر توتیا سرمه دان
\\
ز دور فلک بدر رویش هلال
&&
ز جور زمان سرو قدش خلال
\\
کف دست و سرپنجهٔ زورمند
&&
جدا کرده ایام بندش ز بند
\\
چنانش بر او رحمت آمد ز دل
&&
که بسرشت بر خاکش از گریه گل
\\
پشیمان شد از کرده و خوی زشت
&&
بفرمود بر سنگ گورش نبشت
\\
مکن شادمانی به مرگ کسی
&&
که دهرت نماند پس از وی بسی
\\
شنید این سخن عارفی هوشیار
&&
بنالید کای قادر کردگار
\\
عجب گر تو رحمت نیاری بر او
&&
که بگریست دشمن به زاری بر او
\\
تن ما شود نیز روزی چنان
&&
که بر وی بسوزد دل دشمنان
\\
مگر در دل دوست رحم آیدم
&&
چو بیند که دشمن ببخشایدم
\\
به جایی رسد کار سر دیر و زود
&&
که گویی در او دیده هرگز نبود
\\
زدم تیشه یک روز بر تل خاک
&&
به گوش آمدم ناله‌ای دردناک
\\
که زنهار اگر مردی آهسته‌تر
&&
که چشم و بناگوش و روی است و سر
\\
\end{longtable}
\end{center}
