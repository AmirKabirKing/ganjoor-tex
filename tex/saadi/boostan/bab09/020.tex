\begin{center}
\section*{بخش ۲۰ - حکایت سفر حبشه: غریب آمدم در سواد حبش}
\label{sec:020}
\addcontentsline{toc}{section}{\nameref{sec:020}}
\begin{longtable}{l p{0.5cm} r}
غریب آمدم در سواد حبش
&&
دل از دهر فارغ سر از عیش خوش
\\
به ره بر یکی دکه دیدم بلند
&&
تنی چند مسکین بر او پای بند
\\
بسیج سفر کردم اندر نفس
&&
بیابان گرفتم چو مرغ از قفس
\\
یکی گفت کاین بندیان شبروند
&&
نصیحت نگیرند و حق نشنوند
\\
چو بر کس نیامد ز دستت ستم
&&
تو را گر جهان شحنه گیرد چه غم؟
\\
نیاورده عامل غش اندر میان
&&
نیندیشد از رفع دیوانیان
\\
وگر عفتت را فریب است زیر
&&
زبان حسابت نگردد دلیر
\\
نکونام را کس نگیرد اسیر
&&
بترس از خدای و مترس از امیر
\\
چو خدمت پسندیده آرم به جای
&&
نیندیشم از دشمن تیره رای
\\
اگر بنده کوشش کند بنده‌وار
&&
عزیزش بدارد خداوندگار
\\
وگر کند رای است در بندگی
&&
ز جانداری افتد به خربندگی
\\
قدم پیش نه کز ملک بگذری
&&
که گر باز مانی ز دد کمتری
\\
\end{longtable}
\end{center}
