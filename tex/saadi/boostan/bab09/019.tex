\begin{center}
\section*{بخش ۱۹ - مثل: پلیدی کند گربه بر جای پاک}
\label{sec:019}
\addcontentsline{toc}{section}{\nameref{sec:019}}
\begin{longtable}{l p{0.5cm} r}
پلیدی کند گربه بر جای پاک
&&
چو زشتش نماید بپوشد به خاک
\\
تو آزادی از ناپسندیده‌ها
&&
نترسی که بر وی فتد دیده‌ها
\\
براندیش از آن بندهٔ پر گناه
&&
که از خواجه مخفی شود چند گاه
\\
اگر بر نگردد به صدق و نیاز
&&
به زنجیر و بندش بیارند باز
\\
به کین آوری با کسی بر ستیز
&&
که از وی گزیرت بود یا گریز
\\
کنون کرد باید عمل را حساب
&&
نه وقتی که منشور گردد کتاب
\\
کسی گر چه بد کرد هم بد نکرد
&&
که پیش از قیامت غم خود بخورد
\\
گر آیینه از آه گردد سیاه
&&
شود روشن آیینهٔ دل به آه
\\
بترس از گناهان خویش این نفس
&&
که روز قیامت نترسی ز کس
\\
\end{longtable}
\end{center}
