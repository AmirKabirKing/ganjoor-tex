\begin{center}
\section*{بخش ۱۲ - حکایت در عالم طفولیت: ز عهد پدر یادم آید همی}
\label{sec:012}
\addcontentsline{toc}{section}{\nameref{sec:012}}
\begin{longtable}{l p{0.5cm} r}
ز عهد پدر یادم آید همی
&&
که باران رحمت بر او هر دمی
\\
که در طفلیم لوح و دفتر خرید
&&
ز بهرم یکی خاتم زر خرید
\\
به در کرد ناگه یکی مشتری
&&
به خرمایی از دستم انگشتری
\\
چو نشناسد انگشتری طفل خرد
&&
به شیرینی از وی توانند برد
\\
تو هم قیمت عمر نشناختی
&&
که در عیش شیرین برانداختی
\\
قیامت که نیکان بر اعلی رسند
&&
ز قعر ثری بر ثریا رسند
\\
تو را خود بماند سر از ننگ پیش
&&
که گردت برآید عملهای خویش
\\
برادر، ز کار بدان شرم دار
&&
که در روی نیکان شوی شرمسار
\\
در آن روز کز فعل پرسند و قول
&&
اولوالعزم را تن بلرزد ز هول
\\
به جایی که دهشت خورند انبیا
&&
تو عذر گنه را چه داری؟ بیا
\\
زنانی که طاعت به رغبت برند
&&
ز مردان ناپارسا بگذرند
\\
تو را شرم ناید ز مردی خویش
&&
که باشد زنان را قبول از تو بیش؟
\\
زنان را به عذری معین که هست
&&
ز طاعت بدارند گه گاه دست
\\
تو بی عذر یک سو نشینی چو زن
&&
رو ای کم ز زن، لاف مردی مزن
\\
مرا خود چه باشد زبان آوری
&&
چنین گفت شاه سخن عنصری:
\\
«چو از راستی بگذری خم بود
&&
چه مردی بود کز زنی کم بود؟»
\\
به ناز و طرب نفس پروده گیر
&&
به ایام دشمن قوی کرده گیر
\\
یکی بچهٔ گرگ می‌پرورید
&&
چو پروده شد خواجه بر هم درید
\\
چو بر پهلوی جان سپردن بخفت
&&
زبان آوری در سرش رفت و گفت
\\
تو دشمن چنین نازنین پروری
&&
ندانی که ناچار زخمش خوری؟
\\
نه ابلیس در حق ما طعنه زد
&&
کز اینان نیاید به جز کار بد؟
\\
فغان از بدیها که در نفس ماست
&&
که ترسم شود ظن ابلیس راست
\\
چو ملعون پسند آمدش قهر ما
&&
خدایش بینداخت از بهر ما
\\
کجا سر برآریم از این عار و ننگ
&&
که با او به صلحیم و با حق به جنگ
\\
نظر دوست نادر کند سوی تو
&&
چو در روی دشمن بود روی تو
\\
گرت دوست باید کز او بر خوری
&&
نباید که فرمان دشمن بری
\\
روا دارد از دوست بیگانگی
&&
که دشمن گزیند به همخانگی
\\
ندانی که کمتر نهد دوست پای
&&
چو بیند که دشمن بود در سرای؟
\\
به سیم سیه تا چه خواهی خرید
&&
که خواهی دل از مهر یوسف برید؟
\\
تو از دوست گر عاقلی بر مگرد
&&
که دشمن نیارد نگه در تو کرد
\\
\end{longtable}
\end{center}
