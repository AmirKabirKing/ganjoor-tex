\begin{center}
\section*{بخش ۴ - گفتار اندر غنیمت شمردن جوانی پیش از پیری: جوانا ره طاعت امروز گیر}
\label{sec:004}
\addcontentsline{toc}{section}{\nameref{sec:004}}
\begin{longtable}{l p{0.5cm} r}
جوانا ره طاعت امروز گیر
&&
که فردا جوانی نیاید ز پیر
\\
فراغ دلت هست و نیروی تن
&&
چو میدان فراخ است گویی بزن
\\
قضا روزگاری ز من در ربود
&&
که هر روزی از وی شبی قدر بود
\\
من آن روز را قدر نشناختم
&&
بدانستم اکنون که در باختم
\\
چه کوشش کند پیر خر زیر بار؟
&&
تو می‌رو که بر بادپایی سوار
\\
شکسته قدح ور ببندند چست
&&
نیاورد خواهد بهای درست
\\
کنون کاوفتادت به غفلت ز دست
&&
طریقی ندارد مگر باز بست
\\
که گفتت به جیحون در انداز تن؟
&&
چو افتاد، هم دست و پایی بزن
\\
به غفلت بدادی ز دست آب پاک
&&
چه چاره کنون جز تیمم به خاک؟
\\
چو از چابکان در دویدن گرو
&&
نبردی، هم افتان و خیزان برو
\\
گر آن بادپایان برفتند تیز
&&
تو بی دست و پای از نشستن بخیز
\\
\end{longtable}
\end{center}
