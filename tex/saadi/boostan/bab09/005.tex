\begin{center}
\section*{بخش ۵ - حکایت در معنی ادراک پیش از فوت: شبی خوابم اندر بیابان فید}
\label{sec:005}
\addcontentsline{toc}{section}{\nameref{sec:005}}
\begin{longtable}{l p{0.5cm} r}
شبی خوابم اندر بیابان فید
&&
فرو بست پای دویدن به قید
\\
شتربانی آمد به هول و ستیز
&&
زمام شتر بر سرم زد که خیز
\\
مگر دل نهادی به مردن ز پس
&&
که بر می‌نخیزی به بانگ جرس؟
\\
مرا همچو تو خواب خوش در سر است
&&
ولیکن بیابان به پیش اندر است
\\
تو کز خواب نوشین به بانگ رحیل
&&
نخیزی، دگر کی رسی در سبیل
\\
فرو کوفت طبل شتر ساروان
&&
به منزل رسید اول کاروان
\\
خنک هوشیاران فرخنده بخت
&&
که پیش از دهلزن بسازند رخت
\\
به ره خفتگان تا بر آرند سر
&&
نبینند ره رفتگان را اثر
\\
سبق برد رهرو که برخاست زود
&&
پس از نقل بیدار بودن چه سود؟
\\
یکی در بهاران بیفشانده جو
&&
چه گندم ستاند به وقت درو؟
\\
کنون باید ای خفته بیدار بود
&&
چو مرگ اندر آرد ز خوابت، چه سود؟
\\
چو شیبت در آمد به روی شباب
&&
شبت روز شد دیده بر کن ز خواب
\\
من آن روز بر کندم از عمر امید
&&
که افتادم اندر سیاهی سپید
\\
دریغا که بگذشت عمر عزیز
&&
بخواهد گذشت این دمی چند نیز
\\
گذشت آنچه در ناصوابی گذشت
&&
ور این نیز هم در نیابی گذشت
\\
کنون وقت تخم است اگر پروری
&&
گر امید داری که خرمن بری
\\
به شهر قیامت مرو تنگدست
&&
که وجهی ندارد به حسرت نشست
\\
گرت چشم عقل است تدبیر گور
&&
کنون کن که چشمت نخورده‌ست مور
\\
به مایه توان ای پسر سود کرد
&&
چه سود افتد آن را که سرمایه خورد؟
\\
کنون کوش کآب از کمر در گذشت
&&
نه وقتی که سیلابت از سر گذشت
\\
کنونت که چشم است اشکی ببار
&&
زبان در دهان است عذری بیار
\\
نه پیوسته باشد روان در بدن
&&
نه همواره گردد زبان در دهن
\\
کنون بایدت عذر تقصیر گفت
&&
نه چون نفس ناطق ز گفتن بخفت
\\
ز دانندگان بشنو امروز قول
&&
که فردا نکیرت بپرسد به هول
\\
غنیمت شمار این گرامی نفس
&&
که بی مرغ قیمت ندارد قفس
\\
مکن عمر ضایع به افسوس و حیف
&&
که فرصت عزیز است و الوقت سیف
\\
\end{longtable}
\end{center}
