\begin{center}
\section*{بخش ۱۵ - حکایت: همی یادم آید ز عهد صغر}
\label{sec:015}
\addcontentsline{toc}{section}{\nameref{sec:015}}
\begin{longtable}{l p{0.5cm} r}
همی یادم آید ز عهد صغر
&&
که عیدی برون آمدم با پدر
\\
به بازیچه مشغول مردم شدم
&&
در آشوب خلق از پدر گم شدم
\\
برآوردم از هول و دهشت خروش
&&
پدر ناگهانم بمالید گوش
\\
که ای شوخ چشم آخرت چند بار
&&
بگفتم که دستم ز دامن مدار
\\
به تنها نداند شدن طفل خرد
&&
که مشکل توان راه نادیده برد
\\
تو هم طفل راهی به سعی ای فقیر
&&
برو دامن راه دانان بگیر
\\
مکن با فرومایه مردم نشست
&&
چو کردی، ز هیبت فرو شوی دست
\\
به فتراک پاکان درآویز چنگ
&&
که عارف ندارد ز دریوزه ننگ
\\
مریدان به قوت ز طفلان کمند
&&
مشایخ چو دیوار مستحکمند
\\
بیاموز رفتار از آن طفل خرد
&&
که چون استعانت به دیوار برد
\\
ز زنجیر ناپارسایان برست
&&
که در حلقهٔ پارسایان نشست
\\
اگر حاجتی داری این حلقه گیر
&&
که سلطان ندارد از این در گزیر
\\
برو خوشه چین باش سعدی صفت
&&
که گرد آوری خرمن معرفت
\\
الا ای مقیمان محراب انس
&&
که فردا نشینید بر خوان قدس
\\
متابید روی از گدایان خیل
&&
که صاحب مروت نراند طفیل
\\
کنون با خرد باید انباز گشت
&&
که فردا نماند ره بازگشت
\\
\end{longtable}
\end{center}
