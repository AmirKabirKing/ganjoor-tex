\begin{center}
\section*{بخش ۶ - گفتار اندر بخشایش بر ناتوانان و شکر  نعمت حق در توانایی: نداند کسی قدر روز خوشی}
\label{sec:006}
\addcontentsline{toc}{section}{\nameref{sec:006}}
\begin{longtable}{l p{0.5cm} r}
نداند کسی قدر روز خوشی
&&
مگر روزی افتد به سختی کشی
\\
زمستان درویش در تنگ سال
&&
چه سهل است پیش خداوند مال
\\
سلیمی که یک چند نالان نخفت
&&
خداوند را شکر صحت نگفت
\\
چو مردانه‌رو باشی و تیز پای
&&
به شکرانه با کندپایان بپای
\\
به پیر کهن بر ببخشد جوان
&&
توانا کند رحم بر ناتوان
\\
چه دانند جیحونیان قدر آب
&&
ز واماندگان پرس در آفتاب
\\
عرب را که در دجله باشد قعود
&&
چه غم دارد از تشنگان زرود
\\
کسی قیمت تندرستی شناخت
&&
که یک چند بیچاره در تب گداخت
\\
تو را تیره شب کی نماید دراز
&&
که غلطی ز پهلو به پهلوی ناز؟
\\
براندیش از افتان و خیزان تب
&&
که رنجور داند درازای شب
\\
به بانگ دهل خواجه بیدار گشت
&&
چه داند شب پاسبان چون گذشت؟
\\
\end{longtable}
\end{center}
