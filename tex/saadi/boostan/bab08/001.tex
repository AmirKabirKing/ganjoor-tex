\begin{center}
\section*{بخش ۱ - سر آغاز: نفس می‌نیارم زد از شکر دوست}
\label{sec:001}
\addcontentsline{toc}{section}{\nameref{sec:001}}
\begin{longtable}{l p{0.5cm} r}
نفس می‌نیارم زد از شکر دوست
&&
که شکری ندانم که در خورد اوست
\\
عطایی است هر موی از او بر تنم
&&
چگونه به هر موی شکری کنم؟
\\
ستایش خداوند بخشنده را
&&
که موجود کرد از عدم بنده را
\\
که را قوت وصف احسان اوست؟
&&
که اوصاف مستغرق شأن اوست
\\
بدیعی که شخص آفریند ز گل
&&
روان و خرد بخشد و هوش و دل
\\
ز پشت پدر تا به پایان شیب
&&
نگر تا چه تشریف دادت ز غیب
\\
چو پاک آفریدت بهُش باش و پاک
&&
که ننگ است ناپاک رفتن به خاک
\\
پیاپی بیفشان از آیینه گرد
&&
که مصقل نگیرد چو زنگار خورد
\\
نه در ابتدا بودی آب منی؟
&&
اگر مردی از سر به در کن منی
\\
چو روزی به سعی آوری سوی خویش
&&
مکن تکیه بر زور بازوی خویش
\\
چرا حق نمی‌بینی ای خودپرست
&&
که بازو به گردش درآورد و دست؟
\\
چو آید به کوشیدنت خیر پیش
&&
به توفیق حق دان نه از سعی خویش
\\
به سرپنجگی کس نبرده‌ست گوی
&&
سپاس خداوند توفیق گوی
\\
تو قائم به خود نیستی یک قدم
&&
ز غیبت مدد می‌رسد دم به دم
\\
نه طفل زبان بسته بودی ز لاف؟
&&
همی روزی آمد به جوفش ز ناف
\\
چو نافش بریدند و روزی گسست
&&
به پستان مادر در آویخت دست
\\
غریبی که رنج آردش دهر پیش
&&
به دارو دهند آبش از شهر خویش
\\
پس او در شکم پرورش یافته‌ست
&&
ز انبوب معده خورش یافته‌ست
\\
دو پستان که امروز دلخواه اوست
&&
دو چشمه هم از پرورشگاه اوست
\\
کنار و بر مادر دلپذیر
&&
بهشت است و پستان در او جوی شیر
\\
درختی است بالای جان پرورش
&&
ولد میوه نازنین بر برش
\\
نه رگهای پستان درون دل است؟
&&
پس ار بنگری شیر خون دل است
\\
به خونش فرو برده دندان چو نیش
&&
سرشته در او مهر خونخوار خویش
\\
چو بازو قوی کرد و دندان ستبر
&&
براندایدش دایه پستان به صبر
\\
چنان صبرش از شیر خامش کند
&&
که پستان شیرین فرامش کند
\\
تو نیز ای که در توبه‌ای طفل راه
&&
به صبرت فراموش گردد گناه
\\
\end{longtable}
\end{center}
