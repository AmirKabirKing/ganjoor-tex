\begin{center}
\section*{بخش ۴ - حکایت اندر معنی شکر منعم: ملک زاده‌ای ز اسب ادهم فتاد}
\label{sec:004}
\addcontentsline{toc}{section}{\nameref{sec:004}}
\begin{longtable}{l p{0.5cm} r}
ملک زاده‌ای ز اسب ادهم فتاد
&&
به گردن درش مهره بر هم فتاد
\\
چو پیلش فرو رفت گردن به تن
&&
نگشتی سرش تا نگشتی بدن
\\
پزشکان بماندند حیران در این
&&
مگر فیلسوفی ز یونان زمین
\\
سرش باز پیچید و رگ راست شد
&&
وگر وی نبودی زمن خواست شد
\\
دگر نوبت آمد به نزدیک شاه
&&
به عین عنایت نکردش نگاه
\\
خردمند را سر فرو شد به شرم
&&
شنیدم که می‌رفت و می‌گفت نرم
\\
اگر دی نپیچیدمی گردنش
&&
نپیچیدی امروز روی از منش
\\
فرستاد تخمی به دست رهی
&&
که باید که بر عودسوزش نهی
\\
ملک را یکی عطسه آمد ز دود
&&
سر و گردنش همچنان شد که بود
\\
به عذر از پی مرد بشتافتند
&&
بجستند بسیار و کم یافتند
\\
مکن، گردن از شکر منعم مپیچ
&&
که روز پسین سر بر آری به هیچ
\\
شنیدم که پیری پسر را به خشم
&&
ملامت همی کرد کای شوخ چشم
\\
تو را تیشه دادم که هیزم شکن
&&
نگفتم که دیوار مسجد بکن
\\
زبان آمد از بهر شکر و سپاس
&&
به غیبت نگرداندش حق شناس
\\
گذرگاه قرآن و پند است گوش
&&
به بهتان و باطل شنیدن مکوش
\\
دو چشم از پی صنع باری نکوست
&&
ز عیب برادر فرو گیر و دوست
\\
\end{longtable}
\end{center}
