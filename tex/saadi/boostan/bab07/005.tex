\begin{center}
\section*{بخش ۵ - حکایت: یکی ناسزا گفت در وقت جنگ}
\label{sec:005}
\addcontentsline{toc}{section}{\nameref{sec:005}}
\begin{longtable}{l p{0.5cm} r}
یکی ناسزا گفت در وقت جنگ
&&
گریبان دریدند وی را به چنگ
\\
قفا خورده عریان و گریان نشست
&&
جهاندیده‌ای گفتش ای خودپرست
\\
چو غنچه گرت بسته بودی دهن
&&
دریده ندیدی چو گل پیرهن
\\
سراسیمه گوید سخن بر گزاف
&&
چو طنبور بی مغز بسیار لاف
\\
نبینی که آتش زبان است و بس
&&
به آبی توان کشتنش در نفس؟
\\
اگر هست مرد از هنر بهره‌ور
&&
هنر خود بگوید نه صاحب هنر
\\
اگر مشک خالص نداری مگوی
&&
ورت هست خود فاش گردد به بوی
\\
به سوگند گفتن که زر مغربی است
&&
چه حاجت؟ محک خود بگوید که چیست
\\
بگویند از این حرف گیران هزار
&&
که سعدی نه اهل است و آمیزگار
\\
روا باشد ار پوستینم درند
&&
که طاقت ندارم که مغزم برند
\\
\end{longtable}
\end{center}
