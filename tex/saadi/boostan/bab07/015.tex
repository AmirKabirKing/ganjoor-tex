\begin{center}
\section*{بخش ۱۵ - حکایت روزه در حال طفولیت: به طفلی درم رغبت روزه خاست}
\label{sec:015}
\addcontentsline{toc}{section}{\nameref{sec:015}}
\begin{longtable}{l p{0.5cm} r}
به طفلی درم رغبت روزه خاست
&&
ندانستمی چپ کدام است و راست
\\
یکی عابد از پارسایان کوی
&&
همی شستن آموختم دست و روی
\\
که بسم الله اول به سنت بگوی
&&
دوم نیت آور، سوم کف بشوی
\\
پس آن گه دهن شوی و بینی سه بار
&&
مناخر به انگشت کوچک بخار
\\
به سبابه دندان پیشین بمال
&&
که نهی است در روزه بعد از زوال
\\
وز آن پس سه مشت آب بر روی زن
&&
ز رستنگه موی سر تا ذقن
\\
دگر دستها تا به مرفق بشوی
&&
ز تسبیح و ذکر آنچه دانی بگوی
\\
دگر مسح سر، بعد از آن غسل پای
&&
همین است و ختمش به نام خدای
\\
کس از من نداند در این شیوه به
&&
نبینی که فرتوت شد پیر ده؟
\\
بگفتند با دهخدای آنچه گفت
&&
فرستاد پیغامش اندر نهفت
\\
که ای زشت کردار زیبا سخن
&&
نخست آنچه گویی به مردم بکن
\\
نه مسواک در روزه گفتی خطاست
&&
بنی آدم مرده خوردن رواست؟
\\
دهن گو ز ناگفتنیها نخست
&&
بشوی آن که از خوردنیها بشست
\\
کسی را که نام آمد اندر میان
&&
به نیکوترین نام و نعتش بخوان
\\
چو همواره گویی که مردم خرند
&&
مبر ظن که نامت چو مردم برند
\\
چنان گوی سیرت به کوی اندرم
&&
که گفتن توانی به روی اندرم
\\
وگر شرمت از دیدهٔ ناظر است
&&
نه ای بی‌بصر، غیب دان حاضر است؟
\\
نیاید همی شرمت از خویشتن
&&
کز او فارغ و شرم داری ز من؟
\\
\end{longtable}
\end{center}
