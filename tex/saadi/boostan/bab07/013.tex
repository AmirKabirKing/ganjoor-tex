\begin{center}
\section*{بخش ۱۳ - حکایت: کسی گفت حجاج خون‌خواره‌ای است}
\label{sec:013}
\addcontentsline{toc}{section}{\nameref{sec:013}}
\begin{longtable}{l p{0.5cm} r}
کسی گفت حجاج خون‌خواره‌ای است
&&
دلش همچو سنگ سیه پاره‌ای است
\\
نترسد همی ز آه و فریاد خلق
&&
خدایا تو بستان از او داد خلق
\\
جهاندیده‌ای پیر دیرینه زاد
&&
جوان را یکی پند پیرانه داد
\\
کز او داد مظلوم مسکین او
&&
بخواهند و از دیگران کین او
\\
تو دست از وی و روزگارش بدار
&&
که خود زیر دستش کند روزگار
\\
نه بیداد از او بهره‌مند آیدم
&&
نه نیز از تو غیبت پسند آیدم
\\
به دوزخ برد مدبری را گناه
&&
که پیمانه پر کرد و دیوان سیاه
\\
دگر کس به غیبت پیش می‌دود
&&
مبادا که تنها به دوزخ رود
\\
\end{longtable}
\end{center}
