\begin{center}
\section*{بخش ۱ - سر آغاز: سخن در صلاح است و تدبیر وخوی}
\label{sec:001}
\addcontentsline{toc}{section}{\nameref{sec:001}}
\begin{longtable}{l p{0.5cm} r}
سخن در صلاح است و تدبیر و خوی
&&
نه در اسب و میدان و چوگان و گوی
\\
تو با دشمن نفس هم‌خانه‌ای
&&
چه در بند پیکار بیگانه‌ای؟
\\
عنان باز پیچان نفس از حرام
&&
به مردی ز رستم گذشتند و سام
\\
تو خود را چو کودک ادب کن به چوب
&&
به گرز گران مغز مردم مکوب
\\
وجود تو شهری است پر نیک و بد
&&
تو سلطان و دستور دانا خرد
\\
رضا و ورع: نیکنامان حر
&&
هوی و هوس: رهزن و کیسه بر
\\
چو سلطان عنایت کند با بدان
&&
کجا ماند آسایش بخردان؟
\\
تو را شهوت و حرص و کین و حسد
&&
چو خون در رگانند و جان در جسد
\\
هوی و هوس را نماند ستیز
&&
چو بینند سر پنجهٔ عقل تیز
\\
رئیسی که دشمن سیاست نکرد
&&
هم از دست دشمن ریاست نکرد
\\
نخواهم در این نوع گفتن بسی
&&
که حرفی بس ار کار بندد کسی
\\
\end{longtable}
\end{center}
