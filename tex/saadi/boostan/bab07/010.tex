\begin{center}
\section*{بخش ۱۰ - حکایت در خاصیت پرده پوشی و سلامت خاموشی: یکی پیش داود طائی نشست}
\label{sec:010}
\addcontentsline{toc}{section}{\nameref{sec:010}}
\begin{longtable}{l p{0.5cm} r}
یکی پیش داود طائی نشست
&&
که دیدم فلان صوفی افتاده مست
\\
قی آلوده دستار و پیراهنش
&&
گروهی سگان حلقه پیرامنش
\\
چو پیر از جوان این حکایت شنید
&&
به آزار از او روی در هم کشید
\\
زمانی بر آشفت و گفت ای رفیق
&&
به کار آید امروز یار شفیق
\\
برو زآن مقام شنیعش بیار
&&
که در شرع نهی است و در خرقه عار
\\
به پشتش در آور چو مردان که مست
&&
عنان سلامت ندارد به دست
\\
نیوشنده شد زین سخن تنگدل
&&
به فکرت فرو رفت چون خر به گل
\\
نه زهره که فرمان نگیرد به گوش
&&
نه یارا که مست اندر آرد به دوش
\\
زمانی بپیچید و درمان ندید
&&
ره سر کشیدن ز فرمان ندید
\\
میان بست و بی اختیارش به دوش
&&
در آورد و شهری بر او عام جوش
\\
یکی طعنه می‌زد که درویش بین
&&
زهی پارسایان پاکیزه دین!
\\
یکی صوفیان بین که می خورده‌اند
&&
مرقع به سیکی گرو کرده‌اند
\\
اشارت کنان این و آن را به دست
&&
که آن سر گران است و این نیم مست
\\
به گردن بر از جور دشمن حسام
&&
به از شنعت شهر و جوش عوام
\\
بلا دید و روزی به محنت گذاشت
&&
به ناکام بردش به جایی که داشت
\\
شب از فکرت و نامرادی نخفت
&&
دگر روز پیرش به تعلیم گفت
\\
مریز آبروی برادر به کوی
&&
که دهرت نریزد به شهر آبروی
\\
\end{longtable}
\end{center}
