\begin{center}
\section*{بخش ۲۷ - حکایت درویش صاحب نظر و بقراط حکیم: یکی صورتی دید صاحب جمال}
\label{sec:027}
\addcontentsline{toc}{section}{\nameref{sec:027}}
\begin{longtable}{l p{0.5cm} r}
یکی صورتی دید صاحب جمال
&&
بگردیدش از شورش عشق حال
\\
بر انداخت بیچاره چندان عرق
&&
که شبنم بر اردیبهشتی ورق
\\
گذر کرد بقراط بر وی سوار
&&
بپرسید کاین را چه افتاده کار؟
\\
کسی گفتش این عابدی پارساست
&&
که هرگز خطایی ز دستش نخاست
\\
رود روز و شب در بیابان و کوه
&&
ز صحبت گریزان، ز مردم ستوه
\\
ربوده‌ست خاطرفریبی دلش
&&
فرو رفته پای نظر در گلش
\\
چو آید ز خلقش ملامت به گوش
&&
بگرید که چند از ملامت؟ خموش
\\
مگوی ار بنالم که معذور نیست
&&
که فریادم از علتی دور نیست
\\
نه این نقش دل می‌رباید ز دست
&&
دل آن می‌رباید که این نقش بست
\\
شنید این سخن مرد کار آزمای
&&
کهنسال پروردهٔ پخته رای
\\
بگفت ار چه صیت نکویی رود
&&
نه با هر کسی هر چه گویی رود
\\
نگارنده را خود همین نقش بود
&&
که شوریده را دل به یغما ربود؟
\\
چرا طفل یک روزه هوشش نبرد؟
&&
که در صنع دیدن چه بالغ چه خرد
\\
محقق همان بیند اندر ابل
&&
که در خوبرویان چین و چگل
\\
نقابی است هر سطر من زین کتیب
&&
فرو هشته بر عارضی دل فریب
\\
معانی است در زیر حرف سیاه
&&
چو در پرده معشوق و در میغ ماه
\\
در اوراق سعدی نگنجد ملال
&&
که دارد پس پرده چندین جمال
\\
مرا کاین سخنهاست مجلس فروز
&&
چو آتش در او روشنایی و سوز
\\
نرنجم ز خصمان اگر بر تپند
&&
کز این آتش پارسی در تبند
\\
\end{longtable}
\end{center}
