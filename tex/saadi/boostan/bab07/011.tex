\begin{center}
\section*{بخش ۱۱ - گفتار اندر غیبت و خللهایی که از وی صادر شود: بد اندر حق مردم نیک و بد}
\label{sec:011}
\addcontentsline{toc}{section}{\nameref{sec:011}}
\begin{longtable}{l p{0.5cm} r}
بد اندر حق مردم نیک و بد
&&
مگوی ای جوانمرد صاحب خرد
\\
که بد مرد را خصم خود می‌کنی
&&
وگر نیکمردست بد می‌کنی
\\
تو را هر که گوید فلان کس بدست
&&
چنان دان که در پوستین خودست
\\
که فعل فلان را بباید بیان
&&
وز این فعل بد می‌برآید عیان
\\
به بد گفتن خلق چون دم زدی
&&
اگر راست گویی سخن هم بدی
\\
زبان کرد شخصی به غیبت دراز
&&
بدو گفت داننده‌ای سرفراز
\\
که یاد کسان پیش من بد مکن
&&
مرا بدگمان در حق خود مکن
\\
گرفتم ز تمکین او کم ببود
&&
نخواهد به جاه تو اندر فزود
\\
کسی گفت و پنداشتم طیبت است
&&
که دزدی بسامان تر از غیبت است
\\
بدو گفتم ای یار آشفته هوش
&&
شگفت آمد این داستانم به گوش
\\
به ناراستی در چه بینی بهی
&&
که بر غیبتش مرتبت می‌نهی؟
\\
بلی گفت دزدان تهور کنند
&&
به بازوی مردی شکم پر کنند
\\
ز غیبت چه می‌خواهد آن ساده مرد
&&
که دیوان سیه کرد و چیزی نخورد!
\\
\end{longtable}
\end{center}
