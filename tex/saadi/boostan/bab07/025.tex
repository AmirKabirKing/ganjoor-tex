\begin{center}
\section*{بخش ۲۵ - گفتار اندر پرهیز کردن از صحبت احداث: خرابت کند شاهد خانه کن}
\label{sec:025}
\addcontentsline{toc}{section}{\nameref{sec:025}}
\begin{longtable}{l p{0.5cm} r}
خرابت کند شاهد خانه کن
&&
برو خانه آباد گردان به زن
\\
نشاید هوس باختن با گلی
&&
که هر بامدادش بود بلبلی
\\
چو خود را به هر مجلسی شمع کرد
&&
تو دیگر چو پروانه گردش مگرد
\\
زن خوب خوش خوی آراسته
&&
چه ماند به نادان نو خاسته؟
\\
در او دم چو غنچه دمی از وفا
&&
که از خنده افتد چو گل در قفا
\\
نه چون کودک پیچ بر پیچ شنگ
&&
که چون مقل نتوان شکستن به سنگ
\\
مبین دلفریبش چو حور بهشت
&&
کز آن روی دیگر چو غول است زشت
\\
گرش پای بوسی نداردت پاس
&&
ورش خاک باشی نداند سپاس
\\
سر از مغز و دست از درم کن تهی
&&
چو خاطر به فرزند مردم نهی
\\
مکن بد به فرزند مردم نگاه
&&
که فرزند خویشت برآید تباه
\\
\end{longtable}
\end{center}
