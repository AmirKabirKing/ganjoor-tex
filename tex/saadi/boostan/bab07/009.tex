\begin{center}
\section*{بخش ۹ - حکایت در فضیلت خاموشی و آفت بسیار سخنی: چنین گفت پیری پسندیده دوش}
\label{sec:009}
\addcontentsline{toc}{section}{\nameref{sec:009}}
\begin{longtable}{l p{0.5cm} r}
چنین گفت پیری پسندیده هوش
&&
خوش آید سخنهای پیران به گوش
\\
که در هند رفتم به کنجی فراز
&&
چه دیدم؟ چو یلدا سیاهی دراز
\\
تو گفتی که عفریت بلقیس بود
&&
به زشتی نمودار ابلیس بود
\\
در آغوش وی دختری چون قمر
&&
فرو برده دندان به لبهاش در
\\
چنان تنگش آورده اندر کنار
&&
که پنداری اللیل یغشی النهار
\\
مرا امر معروف دامن گرفت
&&
فضول آتشی گشت و در من گرفت
\\
طلب کردم از پیش و پس چوب و سنگ
&&
که ای نا خدا ترس بی نام و ننگ
\\
به تشنیع و دشنام و آشوب و زجر
&&
سپید از سیه فرق کردم چو فجر
\\
شد آن ابر ناخوش ز بالای باغ
&&
پدید آمد آن بیضه از زیر زاغ
\\
ز لا حولم آن دیو هیکل بجست
&&
پری پیکر اندر من آویخت دست
\\
که ای زرق سجادهٔ دلق پوش
&&
سیه‌کار دنیاخر دین‌فروش
\\
مرا عمرها دل ز کف رفته بود
&&
بر این شخص و جان بر وی آشفته بود
\\
کنون پخته شد لقمه خام من
&&
که گرمش به در کردی از کام من
\\
تظلم برآورد و فریاد خواند
&&
که شفقت بر افتاد و رحمت نماند
\\
نماند از جوانان کسی دستگیر
&&
که بستاندم داد از این مرد پیر؟
\\
که شرمش نیاید ز پیری همی
&&
زدن دست در ستر نامحرمی
\\
همی کرد فریاد و دامن به چنگ
&&
مرا مانده سر در گریبان ز ننگ
\\
فرو گفت عقلم به گوش ضمیر
&&
که از جامه بیرون روم همچو سیر
\\
نه خصمی که با او برآیی به داو
&&
بگرداندت گرد گیتی به گاو
\\
برهنه دوان رفتم از پیش زن
&&
که در دست او جامه بهتر که من
\\
پس از مدتی کرد بر من گذار
&&
که می‌دانیم؟ گفتمش زینهار!
\\
که من توبه کردم به دست تو بر
&&
که گرد فضولی نگردم دگر
\\
کسی را نیاید چنین کار پیش
&&
که عاقل نشیند پس کار خویش
\\
از آن شنعت این پند برداشتم
&&
دگر دیده نادیده انگاشتم
\\
زبان در کش ار عقل داری و هوش
&&
چو سعدی سخن گوی ور نه خموش
\\
\end{longtable}
\end{center}
