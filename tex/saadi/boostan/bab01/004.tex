\begin{center}
\section*{بخش ۴ - در معنی شفقت بر حال رعیت: شنیدم که فرماندهی دادگر}
\label{sec:004}
\addcontentsline{toc}{section}{\nameref{sec:004}}
\begin{longtable}{l p{0.5cm} r}
شنیدم که فرماندهی دادگر
&&
قبا داشتی هر دو روی آستر
\\
یکی گفتش ای خسرو نیکروز
&&
ز دیبای چینی قبایی بدوز
\\
بگفت این قدر ستر و آسایش است
&&
وز این بگذری زیب و آرایش است
\\
نه از بهر آن می‌ستانم خراج
&&
که زینت کنم بر خود و تخت و تاج
\\
چو همچون زنان حله در تن کنم
&&
به مردی کجا دفع دشمن کنم؟
\\
مرا هم ز صد گونه آز و هواست
&&
ولیکن خزینه نه تنها مراست
\\
خزاین پر از بهر لشکر بود
&&
نه از بهر آذین و زیور بود
\\
سپاهی که خوشدل نباشد ز شاه
&&
ندارد حدود ولایت نگاه
\\
چو دشمن خر روستایی برد
&&
ملک باج و ده یک چرا می‌خورد؟
\\
مخالف خرش برد و سلطان خراج
&&
چه اقبال ماند در آن تخت و تاج؟
\\
رعیت درخت است اگر پروری
&&
به کام دل دوستان بر خوری
\\
به بی‌رحمی از بیخ و بارش مکن
&&
که نادان کند حیف بر خویشتن
\\
مروت نباشد بر افتاده زور
&&
برد مرغ دون دانه از پیش مور
\\
کسان بر خورند از جوانی و بخت
&&
که بر زیردستان نگیرند سخت
\\
اگر زیردستی در آید ز پای
&&
حذر کن ز نالیدنش بر خدای
\\
چو شاید گرفتن به نرمی دیار
&&
به پیکار خون از مشامی میار
\\
به مردی که ملک سراسر زمین
&&
نیرزد که خونی چکد بر زمین
\\
شنیدم که جمشید فرخ سرشت
&&
به سرچشمه‌ای بر به سنگی نوشت
\\
بر این چشمه چون ما بسی دم زدند
&&
برفتند چون چشم بر هم زدند
\\
گرفتیم عالم به مردی و زور
&&
ولیکن نبردیم با خود به گور
\\
چو بر دشمنی باشدت دسترس
&&
مرنجانش کاو را همین غصه بس
\\
عدو زنده سرگشته پیرامنت
&&
به از خون او کشته در گردنت
\\
\end{longtable}
\end{center}
