\begin{center}
\section*{بخش ۳۵ - گفتار اندر تقویت مردان کار آزموده: به پیکار دشمن دلیران فرست}
\label{sec:035}
\addcontentsline{toc}{section}{\nameref{sec:035}}
\begin{longtable}{l p{0.5cm} r}
به پیکار دشمن دلیران فرست
&&
هژبران به ناورد شیران فرست
\\
به رای جهاندیدگان کار کن
&&
که صید آزموده‌ست گرگ کهن
\\
مترس از جوانان شمشیر زن
&&
حذر کن ز پیران بسیار فن
\\
جوانان پیل افکن شیرگیر
&&
ندانند دستان روباه پیر
\\
خردمند باشد جهاندیده مرد
&&
که بسیار گرم آزموده‌ست و سرد
\\
جوانان شایستهٔ بخت ور
&&
ز گفتار پیران نپیچند سر
\\
گرت مملکت باید آراسته
&&
مده کار معظم به نوخاسته
\\
سپه را مکن پیشرو جز کسی
&&
که در جنگها بوده باشد بسی
\\
به خردان مفرمای کار درشت
&&
که سندان نشاید شکستن به مشت
\\
رعیت نوازی و سر لشکری
&&
نه کاری است بازیچه و سرسری
\\
نخواهی که ضایع شود روزگار
&&
به ناکاردیده مفرمای کار
\\
نتابد سگ صید روی از پلنگ
&&
ز روبه رمد شیر نادیده جنگ
\\
چو پرورده باشد پسر در شکار
&&
نترسد چو پیش آیدش کارزار
\\
به کشتی و نخجیر و آماج و گوی
&&
دلاور شود مرد پرخاشجوی
\\
به گرمابه پرورده و عیش و ناز
&&
برنجد چو بیند در جنگ باز
\\
دو مردش نشانند بر پشت زین
&&
بود کش زند کودکی بر زمین
\\
یکی را که دیدی تو در جنگ پشت
&&
بکش گر عدو در مصافش نکشت
\\
مخنث به از مرد شمشیر زن
&&
که روز وغا سر بتابد چو زن
\\
چه خوش گفت گرگین به فرزند خویش
&&
چو قربان پیکار بربست و کیش
\\
اگر چون زنان جست خواهی گریز
&&
مرو آب مردان جنگی مریز
\\
سواری که در جنگ بنمود پشت
&&
نه خود را که نام آوران را بکشت
\\
شجاعت نیاید مگر زآن دو یار
&&
که افتند در حلقهٔ کارزار
\\
دو همجنس همسفرهٔ همزبان
&&
بکوشند در قلب هیجا به جان
\\
که تنگ آیدش رفتن از پیش تیر
&&
برادر به چنگال دشمن اسیر
\\
چو بینی که یاران نباشند یار
&&
هزیمت ز میدان غنیمت شمار
\\
\end{longtable}
\end{center}
