\begin{center}
\section*{بخش ۱۳ - حکایت در معنی رحمت با ناتوانان در حال توانایی: چنان قحط سالی شد اندر دمشق}
\label{sec:013}
\addcontentsline{toc}{section}{\nameref{sec:013}}
\begin{longtable}{l p{0.5cm} r}
چنان قحط سالی شد اندر دمشق
&&
که یاران فراموش کردند عشق
\\
چنان آسمان بر زمین شد بخیل
&&
که لب تر نکردند زرع و نخیل
\\
بخوشید سرچشمه‌های قدیم
&&
نماند آب، جز آب چشم یتیم
\\
نبودی به جز آه بیوه زنی
&&
اگر برشدی دودی از روزنی
\\
چو درویش بی رنگ دیدم درخت
&&
قوی بازوان سست و درمانده سخت
\\
نه در کوه سبزی نه در باغ شخ
&&
ملخ بوستان خورده مردم ملخ
\\
در آن حال پیش آمدم دوستی
&&
از او مانده بر استخوان پوستی
\\
وگر چه به مکنت قوی حال بود
&&
خداوند جاه و زر و مال بود
\\
بدو گفتم: ای یار پاکیزه خوی
&&
چه درماندگی پیشت آمد؟ بگوی
\\
بغرید بر من که عقلت کجاست؟
&&
چو دانی و پرسی سؤالت خطاست
\\
نبینی که سختی به غایت رسید
&&
مشقت به حد نهایت رسید؟
\\
نه باران همی آید از آسمان
&&
نه بر می‌رود دود فریاد خوان
\\
بدو گفتم: آخر تو را باک نیست
&&
کشد زهر جایی که تریاک نیست
\\
گر از نیستی دیگری شد هلاک
&&
تو را هست، بط را ز طوفان چه باک؟
\\
نگه کرد رنجیده در من فقیه
&&
نگه کردن عالم اندر سفیه
\\
که مرد ار چه بر ساحل است، ای رفیق
&&
نیاساید و دوستانش غریق
\\
من از بینوایی نیم روی زرد
&&
غم بینوایان رخم زرد کرد
\\
نخواهد که بیند خردمند، ریش
&&
نه بر عضو مردم، نه بر عضو خویش
\\
یکی اول از تندرستان منم
&&
که ریشی ببینم بلرزد تنم
\\
منغص بود عیش آن تندرست
&&
که باشد به پهلوی بیمار سست
\\
چو بینم که درویش مسکین نخورد
&&
به کام اندرم لقمه زهر است و درد
\\
یکی را به زندان درش دوستان
&&
کجا ماندش عیش در بوستان؟
\\
\end{longtable}
\end{center}
