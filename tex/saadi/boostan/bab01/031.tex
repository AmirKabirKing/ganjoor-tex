\begin{center}
\section*{بخش ۳۱ - حکایت زورآزمای تنگدست: یکی مشت زن بخت و روزی نداشت}
\label{sec:031}
\addcontentsline{toc}{section}{\nameref{sec:031}}
\begin{longtable}{l p{0.5cm} r}
یکی مشت زن بخت و روزی نداشت
&&
نه اسباب شامش مهیا نه چاشت
\\
ز جور شکم گل کشیدی به پشت
&&
که روزی محال است خوردن به مشت
\\
مدام از پریشانی روزگار
&&
دلش حسرت آورد و تن سوگوار
\\
گهش جنگ با عالم خیره‌کش
&&
گه از بخت شوریده، رویش ترش
\\
گه از دیدن عیش شیرین خلق
&&
فرو می‌شدی آب تلخش به حلق
\\
گه از کار آشفته بگریستی
&&
که کس دید از این تلخ‌تر زیستی؟
\\
کسان شهد نوشند و مرغ و بره
&&
مرا روی نان می‌نبیند تره
\\
گر انصاف پرسی نه نیکوست این
&&
برهنه من و گربه را پوستین
\\
چه بودی که پایم در این کار گل
&&
به گنجی فرو رفتی از کام دل!
\\
مگر روزگاری هوس راندمی
&&
ز خود گرد محنت بیفشاندمی
\\
شنیدم که روزی زمین می‌شکافت
&&
عظام زنخدان پوسیده یافت
\\
به خاک اندرش عقد بگسیخته
&&
گهرهای دندان فرو ریخته
\\
دهان بی زبان پند می‌گفت و راز
&&
که ای خواجه با بینوایی بساز
\\
نه این است حال دهن زیر گل
&&
شکر خورده انگار یا خون دل
\\
غم از گردش روزگاران مدار
&&
که بی ما بگردد بسی روزگار
\\
همان لحظه کاین خاطرش روی داد
&&
غم از خاطرش رخت یک سو نهاد
\\
که ای نفس بی رای و تدبیر و هش
&&
بکش بار تیمار و خود را مکش
\\
اگر بنده‌ای بار بر سر برد
&&
وگر سر به اوج فلک بر برد
\\
در آن دم که حالش دگرگون شود
&&
به مرگ از سرش هر دو بیرون شود
\\
غم و شادمانی نماند ولیک
&&
جزای عمل ماند و نام نیک
\\
کرم پای دارد، نه دیهیم و تخت
&&
بده کز تو این ماند ای نیکبخت
\\
مکن تکیه بر ملک و جاه و حشم
&&
که پیش از تو بوده‌ست و بعد از تو هم
\\
خداوند دولت غم دین خورد
&&
که دنیا به هر حال می‌بگذرد
\\
نخواهی که ملکت برآید بهم
&&
غم ملک و دین هر دو باید بهم
\\
زرافشان، چو دنیا بخواهی گذاشت
&&
که سعدی در افشاند اگر زر نداشت
\\
\end{longtable}
\end{center}
