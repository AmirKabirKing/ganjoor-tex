\begin{center}
\section*{بخش ۲۸ - حکایت پادشاه غور با روستایی: شنیدم که از پادشاهان غور}
\label{sec:028}
\addcontentsline{toc}{section}{\nameref{sec:028}}
\begin{longtable}{l p{0.5cm} r}
شنیدم که از پادشاهان غور
&&
یکی پادشه خر گرفتی به زور
\\
خران زیر بار گران بی علف
&&
به روزی دو مسکین شدندی تلف
\\
چو منعم کند سفله را، روزگار
&&
نهد بر دل تنگ درویش، بار
\\
چو بام بلندش بود خودپرست
&&
کند بول و خاشاک بر بام پست
\\
شنیدم که باری به عزم شکار
&&
برون رفت بیدادگر شهریار
\\
تکاور به دنبال صیدی براند
&&
شبش در گرفت از حشم باز ماند
\\
به تنها ندانست روی و رهی
&&
بینداخت ناکام شب در دهی
\\
یکی پیرمرد اندر آن ده مقیم
&&
ز پیران مردم شناس قدیم
\\
پسر را همی‌گفت کای شادبهر
&&
خرت را مبر بامدادان به شهر
\\
که این ناجوانمرد برگشته بخت
&&
که تابوت بینمش بر جای تخت
\\
کمر بسته دارد به فرمان دیو
&&
به گردون بر از دست جورش غریو
\\
در این کشور آسایش و خرمی
&&
ندید و نبیند به چشم آدمی
\\
مگر کاین سیه نامهٔ بی‌صفا
&&
به دوزخ برد لعنت اندر قفا
\\
پسر گفت: راه دراز است و سخت
&&
پیاده نیارم شد ای نیکبخت
\\
طریقی بیندیش و رایی بزن
&&
که رای تو روشن‌تر از رای من
\\
پدر گفت: اگر پند من بشنوی
&&
یکی سنگ برداشت باید قوی
\\
زدن بر خر نامور چند بار
&&
سر و دست و پهلوش کردن فگار
\\
مگر کان فرومایهٔ زشت کیش
&&
به کارش نیاید خر پشت ریش
\\
چو خضر پیمبر که کشتی شکست
&&
وز او دست جبار ظالم ببست
\\
به سالی که در بحر کشتی گرفت
&&
بسی سالها نام زشتی گرفت
\\
تفو بر چنان ملک و دولت که راند
&&
که شنعت بر او تا قیامت بماند
\\
پسر چون شنید این حدیث از پدر
&&
سر از خط فرمان نبردش به در
\\
فرو کوفت بیچاره خر را به سنگ
&&
خر از دست عاجز شد از پای لنگ
\\
پدر گفتش اکنون سر خویش گیر
&&
هر آن ره که می‌بایدت پیش گیر
\\
پسر در پی کاروان اوفتاد
&&
ز دشنام چندان که دانست داد
\\
وز آن سو پدر روی در آستان
&&
که یارب به سجادهٔ راستان
\\
که چندان امانم ده از روزگار
&&
کز این نحس ظالم بر آید دمار
\\
اگر من نبینم مر او را هلاک
&&
شب گور چشمم نخسبد به خاک
\\
اگر مار زاید زن باردار
&&
به از آدمی زادهٔ دیوسار
\\
زن از مرد موذی به بسیار به
&&
سگ از مردم مردم‌آزار به
\\
مخنث که بیداد بر خود کند
&&
از آن به که با دیگری بد کند
\\
شه این جمله بشنید و چیزی نگفت
&&
ببست اسب و سر بر نمد زین بخفت
\\
همه شب به بیداری اختر شمرد
&&
ز سودا و اندیشه خوابش نبرد
\\
چو آواز مرغ سحر گوش کرد
&&
پریشانی شب فراموش کرد
\\
سواران همه شب همی تاختند
&&
سحرگه پی اسب بشناختند
\\
بر آن عرصه بر اسب دیدند شاه
&&
پیاده دویدند یکسر سپاه
\\
به خدمت نهادند سر بر زمین
&&
چو دریا شد از موج لشکر، زمین
\\
یکی گفتش از دوستان قدیم
&&
که شب حاجبش بود و روزش ندیم
\\
رعیت چه نزلت نهادند دوش؟
&&
که ما را نه چشم آرمید و نه گوش
\\
شهنشه نیارست کردن حدیث
&&
که بر وی چه آمد ز خبث خبیث
\\
هم آهسته سر برد پیش سرش
&&
فرو گفت پنهان به گوش اندرش
\\
کسم پای مرغی نیاورد پیش
&&
ولی دست خر رفت از اندازه بیش
\\
بزرگان نشستند و خوان خواستند
&&
بخوردند و مجلس بیاراستند
\\
چو شور و طرب در نهاد آمدش
&&
ز دهقان دوشینه یاد آمدش
\\
بفرمود و جستند و بستند سخت
&&
به خواری فکندند در پای تخت
\\
سیه دل برآهخت شمشیر تیز
&&
ندانست بیچاره راه گریز
\\
سر ناامیدی برآورد و گفت
&&
نشاید شب گور در خانه خفت
\\
نه تنها منت گفتم ای شهریار
&&
که برگشته بختی و بد روزگار
\\
چرا خشم بر من گرفتی و بس؟
&&
منت پیش گفتم، همه خلق پس
\\
چو بیداد کردی توقع مدار
&&
که نامت به نیکی رود در دیار
\\
ور ایدون که دشوارت آمد سخن
&&
دگر هر چه دشوارت آید مکن
\\
تو را چاره از ظلم برگشتن است
&&
نه بیچاره بی‌گنه کشتن است
\\
مرا پنج روز دگر مانده گیر
&&
دو روز دگر عیش خوش رانده گیر
\\
نماند ستمکار بد روزگار
&&
بماند بر او لعنت پایدار
\\
تو را نیک پند است اگر بشنوی
&&
وگر نشنوی خود پشیمان شوی
\\
بدان کی ستوده شود پادشاه
&&
که خلقش ستایند در بارگاه؟
\\
چه سود آفرین بر سر انجمن
&&
پس چرخه نفرین کنان پیرزن؟
\\
همی گفت و شمشیر بالای سر
&&
سپر کرده جان پیش تیر قدر
\\
نبینی که چون کارد بر سر بود
&&
قلم را زبانش روان تر بود
\\
شه از مستی غفلت آمد به هوش
&&
به گوشش فرو گفت فرخ سروش
\\
کز این پیر دست عقوبت بدار
&&
یکی کشته گیر از هزاران هزار
\\
زمانی سر اندر گریبان بماند
&&
پس آن گه به عفو آستین برفشاند
\\
به دستان خود بند از او برگرفت
&&
سرش را ببوسید و در بر گرفت
\\
بزرگیش بخشید و فرماندهی
&&
ز شاخ امیدش برآمد بهی
\\
به گیتی حکایت شد این داستان
&&
رود نیکبخت از پی راستان
\\
بیاموزی از عاقلان حسن خوی
&&
نه چندان که از غافل عیب جوی
\\
ز دشمن شنو سیرت خود که دوست
&&
هرآنچ از تو آید به چشمش نکوست
\\
وبال است دادن به رنجور قند
&&
که داروی تلخش بود سودمند
\\
ترش روی بهتر کند سرزنش
&&
که یاران خوش طبع شیرین منش
\\
از این به نصیحت نگوید کست
&&
اگر عاقلی یک اشارت بست
\\
\end{longtable}
\end{center}
