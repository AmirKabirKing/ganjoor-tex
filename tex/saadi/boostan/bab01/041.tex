\begin{center}
\section*{بخش ۴۱ - گفتار اندر پوشیدن راز خویش: به تدبیر جنگ بد اندیش کوش}
\label{sec:041}
\addcontentsline{toc}{section}{\nameref{sec:041}}
\begin{longtable}{l p{0.5cm} r}
به تدبیر جنگ بد اندیش کوش
&&
مصالح بیندیش و نیت بپوش
\\
منه در میان راز با هر کسی
&&
که جاسوس هم کاسه دیدم بسی
\\
سکندر که با شرقیان حرب داشت
&&
در خیمه گویند در غرب داشت
\\
چو بهمن به زاولستان خواست شد
&&
چپ آوازه افکند و از راست شد
\\
اگر جز تو داند که عزم تو چیست
&&
بر آن رای و دانش بباید گریست
\\
کرم کن، نه پرخاش و کین‌آوری
&&
که عالم به زیر نگین آوری
\\
چو کاری بر آید به لطف و خوشی
&&
چه حاجت به تندی و گردن کشی؟
\\
نخواهی که باشد دلت دردمند
&&
دل درمندان بر آور ز بند
\\
به بازو توانا نباشد سپاه
&&
برو همت از ناتوانان بخواه
\\
دعای ضعیفان امیدوار
&&
ز بازوی مردی به آید به کار
\\
هر آن کاستعانت به درویش برد
&&
اگر بر فریدون زد از پیش برد
\\
\end{longtable}
\end{center}
