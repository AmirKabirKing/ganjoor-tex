\begin{center}
\section*{بخش ۲۱ - حکایت حجاج یوسف: حکایت کنند از یکی نیکمرد}
\label{sec:021}
\addcontentsline{toc}{section}{\nameref{sec:021}}
\begin{longtable}{l p{0.5cm} r}
حکایت کنند از یکی نیکمرد
&&
که اکرام حجاج یوسف نکرد
\\
به سرهنگ دیوان نگه کرد تیز
&&
که نطعش بیانداز و خونش بریز
\\
چو حجت نماند جفا جوی را
&&
به پرخاش در هم کشد روی را
\\
بخندید و بگریست مرد خدای
&&
عجب داشت سنگین دل تیره رای
\\
چو دیدش که خندید و دیگر گریست
&&
بپرسید کاین خنده و گریه چیست؟
\\
بگفتا همی‌گریم از روزگار
&&
که طفلان بیچاره دارم چهار
\\
همی‌خندم از لطف یزدان پاک
&&
که مظلوم رفتم نه ظالم به خاک
\\
پسر گفتش: ای نامور شهریار
&&
یکی دست از این مرد صوفی بدار
\\
که خلقی بر او روی دارند و پشت
&&
نه رای است خلقی به یک بار کشت
\\
بزرگی و عفو و کرم پیشه کن
&&
ز خردان اطفالش اندیشه کن
\\
شنیدم که نشنید و خونش بریخت
&&
ز فرمان داور که داند گریخت؟
\\
بزرگی در آن فکرت آن شب بخفت
&&
به خواب اندرش دید و پرسید و گفت:
\\
دمی بیش بر من سیاست نراند
&&
عقوبت بر او تا قیامت بماند
\\
نخفته‌ست مظلوم از آهش بترس
&&
ز دود دل صبحگاهش بترس
\\
نترسی که پاک اندرونی شبی
&&
بر آرد ز سوز جگر یا ربی؟
\\
نه ابلیس بد کرد و نیکی ندید؟
&&
بر پاک ناید ز تخم پلید
\\
\end{longtable}
\end{center}
