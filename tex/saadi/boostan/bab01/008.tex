\begin{center}
\section*{بخش ۸ - حکایت در معنی شفقت: یکی از بزرگان اهل تمیز}
\label{sec:008}
\addcontentsline{toc}{section}{\nameref{sec:008}}
\begin{longtable}{l p{0.5cm} r}
یکی از بزرگان اهل تمیز
&&
حکایت کند ز ابن عبدالعزیز
\\
که بودش نگینی در انگشتری
&&
فرو مانده در قیمتش جوهری
\\
به شب گفتی از جرم گیتی فروز
&&
دری بود از روشنایی چو روز
\\
قضا را درآمد یکی خشک سال
&&
که شد بدر سیمای مردم هلال
\\
چو در مردم آرام و قوت ندید
&&
خود آسوده بودن مروت ندید
\\
چو بیند کسی زهر در کام خلق
&&
کیش بگذرد آب نوشین به حلق
\\
بفرمود و بفروختندش به سیم
&&
که رحم آمدش بر غریب و یتیم
\\
به یک هفته نقدش به تاراج داد
&&
به درویش و مسکین و محتاج داد
\\
فتادند در وی ملامت کنان
&&
که دیگر به دستت نیاید چنان
\\
شنیدم که می‌گفت و باران دمع
&&
فرو می‌دویدش به عارض چو شمع
\\
که زشت است پیرایه بر شهریار
&&
دل شهری از ناتوانی فگار
\\
مرا شاید انگشتری بی‌نگین
&&
نشاید دل خلقی اندوهگین
\\
خنک آن که آسایش مرد و زن
&&
گزیند بر آرایش خویشتن
\\
نکردند رغبت هنرپروران
&&
به شادی خویش از غم دیگران
\\
اگر خوش بخسبد ملک بر سریر
&&
نپندارم آسوده خسبد فقیر
\\
وگر زنده دارد شب دیر باز
&&
بخسبند مردم به آرام و ناز
\\
بحمدالله این سیرت و راه راست
&&
اتابک ابوبکر بن سعد راست
\\
کس از فتنه در پارس دیگر نشان
&&
نبیند مگر قامت مهوشان
\\
یکی پنج بیتم خوش آمد به گوش
&&
که در مجلسی می‌سرودند دوش
\\
مرا راحت از زندگی دوش بود
&&
که آن ماهرویم در آغوش بود
\\
مر او را چو دیدم سر از خواب مست
&&
بدو گفتم ای سرو پیش تو پست
\\
دمی نرگس از خواب نوشین بشوی
&&
چو گلبن بخند و چو بلبل بگوی
\\
چه می‌خسبی ای فتنه روزگار؟
&&
بیا و می لعل نوشین بیار
\\
نگه کرد شوریده از خواب و گفت
&&
مرا فتنه خوانی و گویی مخفت
\\
در ایام سلطان روشن نفس
&&
نبیند دگر فتنه بیدار کس
\\
\end{longtable}
\end{center}
