\begin{center}
\section*{بخش ۹ - حکایت اتابک تکله: در اخبار شاهان پیشینه هست}
\label{sec:009}
\addcontentsline{toc}{section}{\nameref{sec:009}}
\begin{longtable}{l p{0.5cm} r}
در اخبار شاهان پیشینه هست
&&
که چون تکله بر تخت زنگی نشست
\\
به دورانش از کس نیازرد کس
&&
سبق برد اگر خود همین بود و بس
\\
چنین گفت یک ره به صاحبدلی
&&
که عمرم به سر رفت بی حاصلی
\\
بخواهم به کنج عبادت نشست
&&
که دریابم این پنج روزی که هست
\\
چو می‌بگذرد جاه و ملک و سریر
&&
نبرد از جهان دولت الا فقیر
\\
چو بشنید دانای روشن نفس
&&
به تندی برآشفت کای تکله بس!
\\
طریقت به جز خدمت خلق نیست
&&
به تسبیح و سجاده و دلق نیست
\\
تو بر تخت سلطانی خویش باش
&&
به اخلاق پاکیزه درویش باش
\\
به صدق و ارادت میان بسته‌دار
&&
ز طامات و دعوی زبان بسته‌دار
\\
قدم باید اندر طریقت نه دم
&&
که اصلی ندارد دم بی‌قدم
\\
بزرگان که نقد صفا داشتند
&&
چنین خرقه زیر قبا داشتند
\\
\end{longtable}
\end{center}
