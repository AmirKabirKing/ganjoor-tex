\begin{center}
\section*{بخش ۳۸ - گفتار اندر دفع دشمن به رای و تدبیر: میان دو بد خواه کوتاه دست}
\label{sec:038}
\addcontentsline{toc}{section}{\nameref{sec:038}}
\begin{longtable}{l p{0.5cm} r}
میان دو بد خواه کوتاه دست
&&
نه فرزانگی باشد ایمن نشست
\\
که گر هر دو باهم سگالند راز
&&
شود دست کوتاه ایشان دراز
\\
یکی را به نیرنگ مشغول دار
&&
دگر را برآور ز هستی دمار
\\
اگر دشمنی پیش گیرد ستیز
&&
به شمشیر تدبیر خونش بریز
\\
برو دوستی گیر با دشمنش
&&
که زندان شود پیرهن بر تنش
\\
چو در لشکر دشمن افتد خلاف
&&
تو بگذار شمشیر خود در غلاف
\\
چو گرگان پسندند بر هم گزند
&&
بر آساید اندر میان گوسفند
\\
چو دشمن به دشمن بود مشتغل
&&
تو با دوست بنشین به آرام دل
\\
\end{longtable}
\end{center}
