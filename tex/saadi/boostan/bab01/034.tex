\begin{center}
\section*{بخش ۳۴ - گفتار اندر نواخت لشکریان در حالت امن: دلاور که باری تهور نمود}
\label{sec:034}
\addcontentsline{toc}{section}{\nameref{sec:034}}
\begin{longtable}{l p{0.5cm} r}
دلاور که باری تهور نمود
&&
بباید به مقدارش اندر فزود
\\
که بار دگر دل نهد بر هلاک
&&
ندارد ز پیکار یأجوج باک
\\
سپاهی در آسودگی خوش بدار
&&
که در حالت سختی آید به کار
\\
سپاهی که کارش نباشد به برگ
&&
چرا دل نهد روز هیجا به مرگ؟
\\
کنون دست مردان جنگی ببوس
&&
نه آنگه که دشمن فرو کوفت کوس
\\
نواحی ملک از کف بدسگال
&&
به لشکر نگه دار و لشکر به مال
\\
ملک را بود بر عدو دست، چیر
&&
چو لشکر دل آسوده باشند و سیر
\\
بهای سر خویشتن می‌خورد
&&
نه انصاف باشد که سختی برد
\\
چو دارند گنج از سپاهی دریغ
&&
دریغ آیدش دست بردن به تیغ
\\
چه مردی کند در صف کارزار
&&
که دستش تهی باشد و کار، زار
\\
\end{longtable}
\end{center}
