\begin{center}
\section*{بخش ۲۸ - حکایت: جوانی به دانگی کرم کرده بود}
\label{sec:028}
\addcontentsline{toc}{section}{\nameref{sec:028}}
\begin{longtable}{l p{0.5cm} r}
جوانی به دانگی کرم کرده بود
&&
تمنای پیری بر آورده بود
\\
به جرمی گرفت آسمان ناگهش
&&
فرستاد سلطان به کشتنگهش
\\
تکاپوی ترکان و غوغای عام
&&
تماشاکنان بر در و کوی و بام
\\
چو دید اندر آشوب، درویش پیر
&&
جوان را به دست خلایق اسیر
\\
دلش بر جوانمرد مسکین بخست
&&
که باری دل آورده بودش به دست
\\
برآورد زاری که سلطان بمرد
&&
جهان ماند و خوی پسندیده برد
\\
به هم بر همی‌سود دست دریغ
&&
شنیدند ترکان آهخته تیغ
\\
به فریاد از ایشان بر آمد خروش
&&
تپانچه زنان بر سر و روی و دوش
\\
پیاده به سر تا در بارگاه
&&
دویدند و بر تخت دیدند شاه
\\
جوان از میان رفت و بردند پیر
&&
به گردن بر تخت سلطان اسیر
\\
به ولش بپرسید و هیبت نمود
&&
که مرگ منت خواستن بر چه بود؟
\\
چو نیک است خوی من و راستی
&&
بد مردم آخر چرا خواستی؟
\\
برآورد پیر دلاور زبان
&&
که ای حلقه در گوش حکمت جهان
\\
به قول دروغی که سلطان بمرد
&&
نمردی و بیچاره‌ای جان ببرد
\\
ملک زین حکایت چنان بر شکفت
&&
که چیزش ببخشید و چیزی نگفت
\\
وز این جانب افتان و خیزان جوان
&&
همی رفت بیچاره هر سو دوان
\\
یکی گفتش از چار سوی قصاص
&&
چه کردی که آمد به جانت خلاص؟
\\
به گوشش فرو گفت کای هوشمند
&&
به جانی و دانگی رهیدم ز بند
\\
یکی تخم در خاک از آن می‌نهد
&&
که روز فرو ماندگی بر دهد
\\
جوی باز دارد بلایی درشت
&&
عصایی شنیدی که عوجی بکشت
\\
حدیث درست آخر از مصطفاست
&&
که بخشایش و خیر دفع بلاست
\\
عدو را نبینی در این بقعه پای
&&
که بوبکر سعد است کشور خدای
\\
بگیر ای جهانی به روی تو شاد
&&
جهانی، که شادی به روی تو باد
\\
کس از کس به دور تو باری نبرد
&&
گلی در چمن جور خاری نبرد
\\
تویی سایهٔ لطف حق بر زمین
&&
پیمبر صفت رحمة ‌العالمین
\\
تو را قدر اگر کس نداند چه غم؟
&&
شب قدر را می‌ندانند هم
\\
\end{longtable}
\end{center}
