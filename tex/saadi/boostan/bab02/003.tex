\begin{center}
\section*{بخش ۳ - حکایت ابراهیم علیه‌السلام: شنیدم که یک هفته ابن‌السبیل}
\label{sec:003}
\addcontentsline{toc}{section}{\nameref{sec:003}}
\begin{longtable}{l p{0.5cm} r}
شنیدم که یک هفته ابن‌السبیل
&&
نیامد به مهمانسرای خلیل
\\
ز فرخنده خویی نخوردی بگاه
&&
مگر بینوایی در آید ز راه
\\
برون رفت و هر جانبی بنگرید
&&
بر اطراف وادی نگه کرد و دید
\\
به تنها یکی در بیابان چو بید
&&
سر و مویش از گرد پیری سپید
\\
به دلداریش مرحبایی بگفت
&&
به رسم کریمان صلایی بگفت
\\
که ای چشمهای مرا مردمک
&&
یکی مردمی کن به نان و نمک
\\
نعم گفت و برجست و برداشت گام
&&
که دانست خلقش، علیه‌السلام
\\
رقیبان مهمانسرای خلیل
&&
به عزت نشاندند پیر ذلیل
\\
بفرمود و ترتیب کردند خوان
&&
نشستند بر هر طرف همگنان
\\
چو بسم الله آغاز کردند جمع
&&
نیامد ز پیرش حدیثی به سمع
\\
چنین گفتش: ای پیر دیرینه روز
&&
چو پیران نمی‌بینمت صدق و سوز
\\
نه شرط است وقتی که روزی خوری
&&
که نام خداوند روزی بری؟
\\
بگفتا نگیرم طریقی به دست
&&
که نشنیدم از پیر آذرپرست
\\
بدانست پیغمبر نیک فال
&&
که گبر است پیر تبه بوده حال
\\
به خواری براندش چو بیگانه دید
&&
که منکر بود پیش پاکان پلید
\\
سروش آمد از کردگار جلیل
&&
به هیبت ملامت کنان کای خلیل
\\
منش داده صد سال روزی و جان
&&
تو را نفرت آمد از او یک زمان
\\
گر او می‌برد پیش آتش سجود
&&
تو وا پس چرا می‌بری دست جود؟
\\
\end{longtable}
\end{center}
