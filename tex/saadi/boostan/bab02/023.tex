\begin{center}
\section*{بخش ۲۳ - حکایت: یکی را خری در گل افتاده بود}
\label{sec:023}
\addcontentsline{toc}{section}{\nameref{sec:023}}
\begin{longtable}{l p{0.5cm} r}
یکی را خری در گل افتاده بود
&&
ز سوداش خون در دل افتاده بود
\\
بیابان و باران و سرما و سیل
&&
فرو هشته ظلمت بر آفاق ذیل
\\
همه شب در این غصه تا بامداد
&&
سقط گفت و نفرین و دشنام داد
\\
نه دشمن برست از زبانش نه دوست
&&
نه سلطان که این بوم و بر زآن اوست
\\
قضا را خداوند آن پهن دشت
&&
در آن حال منکر بر او بر گذشت
\\
شنید این سخنهای دور از صواب
&&
نه صبر شنیدن، نه روی جواب
\\
ملک شرمگین در حشم بنگریست
&&
که سودای این بر من از بهر چیست؟
\\
یکی گفت شاها به تیغش بزن
&&
که نگذاشت کس را نه دختر نه زن
\\
نگه کرد سلطان عالی محل
&&
خودش در بلا دید و خر در وحل
\\
ببخشود بر حال مسکین مرد
&&
فرو خورد خشم سخنهای سرد
\\
زرش داد و اسب و قبا پوستین
&&
چه نیکو بود مهر در وقت کین
\\
یکی گفتش ای پیر بی عقل و هوش
&&
عجب رستی از قتل، گفتا خموش
\\
اگر من بنالیدم از درد خویش
&&
وی انعام فرمود در خورد خویش
\\
بدی را بدی سهل باشد جزا
&&
اگر مردی احسن الی من اسا
\\
\end{longtable}
\end{center}
