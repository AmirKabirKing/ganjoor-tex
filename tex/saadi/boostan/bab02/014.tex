\begin{center}
\section*{بخش ۱۴ - حکایت: یکی سیرت نیکمردان شنو}
\label{sec:014}
\addcontentsline{toc}{section}{\nameref{sec:014}}
\begin{longtable}{l p{0.5cm} r}
یکی سیرت نیکمردان شنو
&&
اگر نیکبختی و مردانه رو
\\
که شبلی ز حانوت گندم فروش
&&
به ده برد انبان گندم به دوش
\\
نگه کرد و موری در آن غله دید
&&
که سرگشته هر گوشه‌ای می‌دوید
\\
ز رحمت بر او شب نیارست خفت
&&
به مأوای خود بازش آورد و گفت
\\
مروت نباشد که این مور ریش
&&
پراکنده گردانم از جای خویش
\\
درون پراکندگان جمع دار
&&
که جمعیتت باشد از روزگار
\\
چه خوش گفت فردوسی پاک زاد
&&
که رحمت بر آن تربت پاک باد
\\
میازار موری که دانه‌کش است
&&
که جان دارد و جان شیرین خوش است
\\
سیاه اندرون باشد و سنگدل
&&
که خواهد که موری شود تنگدل
\\
مزن بر سر ناتوان دست زور
&&
که روزی به پایش در افتی چو مور
\\
درون فروماندگان شاد کن
&&
ز روز فروماندگی یاد کن
\\
نبخشود بر حال پروانه شمع
&&
نگه کن که چون سوخت در پیش جمع
\\
گرفتم ز تو ناتوان تر بسی است
&&
تواناتر از تو هم آخر کسی است
\\
\end{longtable}
\end{center}
