\begin{center}
\section*{بخش ۱۵ - گفتار اندر ثمره جوانمردی: ببخش ای پسر کآدمی زاده صید}
\label{sec:015}
\addcontentsline{toc}{section}{\nameref{sec:015}}
\begin{longtable}{l p{0.5cm} r}
ببخش ای پسر کآدمی زاده صید
&&
به احسان توان کرد و، وحشی به قید
\\
عدو را به الطاف گردن ببند
&&
که نتوان بریدن به تیغ این کمند
\\
چو دشمن کرم بیند و لطف و جود
&&
نیاید دگر خبث از او در وجود
\\
مکن بد که بد بینی از یار نیک
&&
نروید ز تخم بدی بار نیک
\\
چو با دوست دشخوار گیری و تنگ
&&
نخواهد که بیند تو را نقش و رنگ
\\
و گر خواجه با دشمنان نیکخوست
&&
بسی بر نیاید که گردند دوست
\\
\end{longtable}
\end{center}
