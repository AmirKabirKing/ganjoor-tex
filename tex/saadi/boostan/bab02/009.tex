\begin{center}
\section*{بخش ۹ - حکایت: به سرهنگ سلطان چنین گفت زن}
\label{sec:009}
\addcontentsline{toc}{section}{\nameref{sec:009}}
\begin{longtable}{l p{0.5cm} r}
به سرهنگ سلطان چنین گفت زن
&&
که خیز ای مبارک در رزق زن
\\
برو تا ز خوانت نصیبی دهند
&&
که فرزندکانت نظر بر رهند
\\
بگفتا بود مطبخ امروز سرد
&&
که سلطان به شب نیت روزه کرد
\\
زن از ناامیدی سر انداخت پیش
&&
همی گفت با خود دل از فاقه ریش
\\
که سلطان از این روزه گویی چه خواست؟
&&
که افطار او عید طفلان ماست
\\
خورنده که خیرش برآید ز دست
&&
به از صائم الدهر دنیا پرست
\\
مسلم کسی را بود روزه داشت
&&
که درمانده‌ای را دهد نان چاشت
\\
وگرنه چه لازم که سعیی بری
&&
ز خود بازگیری و هم خود خوری؟
\\
\end{longtable}
\end{center}
