\begin{center}
\section*{غزل شماره ۸۱: چنین که غمزهٔ تو خون خلق می‌ریزد}
\label{sec:081}
\addcontentsline{toc}{section}{\nameref{sec:081}}
\begin{longtable}{l p{0.5cm} r}
چنین که غمزهٔ تو خون خلق می‌ریزد
&&
عجب نباشد اگر رستخیز انگیزد
\\
فتور غمزهٔ تو صدهزار صف بشکست
&&
که در میانه یکی گرد برنمی‌خیزد
\\
ز چشم جادوی مردافگن شبه رنگت
&&
جهان، اگر بتواند، دو اسبه بگریزد
\\
فروغ عشق تو تا کی روان من سوزد
&&
فریب چشم تو تا چند خون من ریزد؟
\\
مرنج، اگر به سر زلف تو در آویزم
&&
که غرقه هرچه ببیند درو بیاویزد
\\
تو را، چنان که تویی، تا کسیت نشناسد
&&
رخ تو هر نفسی رنگ دیگر آمیزد
\\
اگر چه خون عراقی بریزی از دیده
&&
به خاکپای تو کز عشق تو نپرهیزد
\\
\end{longtable}
\end{center}
