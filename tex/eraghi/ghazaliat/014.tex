\begin{center}
\section*{غزل شماره ۱۴: از پرده برون آمد ساقی، قدحی در دست}
\label{sec:014}
\addcontentsline{toc}{section}{\nameref{sec:014}}
\begin{longtable}{l p{0.5cm} r}
از پرده برون آمد، ساقی، قدحی در دست
&&
هم پردهٔ ما بدرید، هم توبهٔ ما بشکست
\\
بنمود رخ زیبا، گشتیم همه شیدا
&&
چون هیچ نماند از ما آمد بر ما بنشست
\\
زلفش گرهی بگشاد بند از دل ما برخاست
&&
جان دل ز جهان برداشت وندر سر زلفش بست
\\
در دام سر زلفش ماندیم همه حیران
&&
وز جام می لعلش گشتیم همه سرمست
\\
از دست بشد چون دل در طرهٔ او زد چنگ
&&
غرقه زند از حیرت در هرچه بیابد دست
\\
چون سلسلهٔ زلفش بند دل حیران شد
&&
آزاد شد از عالم وز هستی ما وارست
\\
دل در سر زلفش شد، از طره طلب کردم
&&
گفتا که: لب او خوش اینک سرما پیوست
\\
با یار خوشی بنشست دل کز سر جان برخاست
&&
با جان و جهان پیوست دل کز دو جهان بگسست
\\
از غمزهٔ روی او گه مستم و گه هشیار
&&
وز طرهٔ لعل او گه نیستم و گه هست
\\
می‌خواستم از اسرار اظهار کنم حرفی
&&
ز اغیار نترسیدم گفتم سخن سر بست
\\
\end{longtable}
\end{center}
