\begin{center}
\section*{غزل شماره ۳۰۴: گر از زلف پریشانت صبا بر هم زند مویی}
\label{sec:304}
\addcontentsline{toc}{section}{\nameref{sec:304}}
\begin{longtable}{l p{0.5cm} r}
گر از زلف پریشانت صبا بر هم زند مویی
&&
برآید زان پریشانی هزار افغان ز هر سویی
\\
به بوی زلف تو هر دم حیات تازه می‌یابم
&&
وگر نه بی‌تو از عیشم نه رنگی ماند و نه بویی
\\
به یاد سرو بالایت روان در پای تو ریزم
&&
به بالای تو گر سروی ببینم بر لب جویی
\\
چو زلفت گر برآرم سر به سودایت، عجب نبود
&&
چه باشد با کمند شیرگیری صید آهویی؟
\\
ز کویت گر رسد گردی به استقبال برخیزد
&&
ز جان افشانی صاحبدلان گردی ز هر کویی
\\
چنان بنشست نقش دوست در آیینهٔ چشمم
&&
که چشمم عکس روی دوست می‌بیند ز هر سویی
\\
رقیبان دست گیریدم، که باز از نو در افتادم
&&
به دست بی‌وفایی، سست پیمانی، جفاجویی
\\
ملولی، زود سیری، نازنینی، ناز پروردی
&&
لطیفی همچو گل نازک ولی چون سرو خودرویی
\\
نیارد جستن از بند کمندش هیچ چالاکی
&&
ندارد طاقت دست و کمانش هیچ بازویی
\\
اگر چه هر سر مویم ازو دردی جدا دارم
&&
دل من کم نخواهد کرد از مهرش سر مویی
\\
ز سودا عاشقانش همچو این گردون چوگان قد
&&
به گرد کوی او سرگشته می‌گردند چون گویی
\\
نگیرد سوز مهر جان گدازش در دل هر کس
&&
مگر باشد چو شمع آتش زبانی، چرب پهلویی
\\
به سودای نکورویی اگر دل گرمیی داری
&&
تحمل بایدت کردن جواب سرد بدخویی
\\
\end{longtable}
\end{center}
