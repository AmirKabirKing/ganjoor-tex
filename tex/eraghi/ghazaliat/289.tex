\begin{center}
\section*{غزل شماره ۲۸۹: بیا، که بی‌تو به جان آمدم ز تنهایی}
\label{sec:289}
\addcontentsline{toc}{section}{\nameref{sec:289}}
\begin{longtable}{l p{0.5cm} r}
بیا، که بی‌تو به جان آمدم ز تنهایی
&&
نمانده صبر و مرا بیش ازین شکیبایی
\\
بیا، که جان مرا بی‌تو نیست برگ حیات
&&
بیا، که چشم مرا بی‌تو نیست بینایی
\\
بیا، که بی‌تو دلم راحتی نمی‌یابد
&&
بیا، که بی‌تو ندارد دو دیده بینایی
\\
اگر جهان همه زیر و زبر شود ز غمت
&&
تو را چه غم؟ که تو خو کرده‌ای به تنهایی
\\
حجاب روی تو هم روی توست در همه حال
&&
نهانی از همه عالم ز بسکه پیدایی
\\
عروس حسن تو را هیچ درنمی‌یابد
&&
به گاه جلوه، مگر دیدهٔ تماشایی
\\
ز بس که بر سر کوی تو ناله‌ها کردم
&&
بسوخت بر من مسکین دل تماشایی
\\
ندیده روی تو، از عشق عالمی مرده
&&
یکی نماند، اگر خود جمال بنمایی
\\
ز چهره پرده برانداز، تا سر اندازی
&&
روان فشاند بر روی تو ز شیدایی
\\
به پرده در چه نشینی؟ چه باشد ار نفسی
&&
به پرسش دل بیچاره‌ای برون آیی!
\\
نظر کنی به دل خستهٔ شکسته دلی
&&
مگر که رحمتت آید، برو ببخشایی
\\
دل عراقی بیچاره آرزومند است
&&
امید بسته که: تا کی نقاب بگشایی؟
\\
\end{longtable}
\end{center}
