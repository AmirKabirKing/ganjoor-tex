\begin{center}
\section*{غزل شماره ۲: ای مرا یک بارگی از خویشتن کرده جدا}
\label{sec:002}
\addcontentsline{toc}{section}{\nameref{sec:002}}
\begin{longtable}{l p{0.5cm} r}
ای مرا یک بارگی از خویشتن کرده جدا
&&
گر بدآن شادی که دور از تو بمیرم مرحبا
\\
دل ز غم رنجور و تو فارغ ازو وز حال ما
&&
بازپرس آخر که: چون شد حال آن بیمار ما؟
\\
شب خیالت گفت با جانم که: چون شد حال دل؟
&&
نعره زد جانم که: ای مسکین، بقا بادا تو را
\\
دوستان را زار کشتی ز آرزوی روی خود
&&
در طریق دوستی آخر کجا باشد روا؟
\\
بود دل را با تو آخر آشنایی پیش ازین
&&
این کند هرگز؟ که کرد این آشنا با آشنا؟
\\
هم چنان در خاک و خون غلتانش باید جان سپرد
&&
خسته‌ای کامید دارد از نکورویان وفا
\\
روز و شب خونابه‌اش باید فشاندن بر درت
&&
دیده‌ای کز خاک درگاه تو جوید توتیا
\\
دل برفت از دست وز تیمار تو خون شد جگر
&&
نیم جانی ماند و آن هم ناتوانی، گو بر آ
\\
از عراقی دوش پرسیدم که: چون است حال تو؟
&&
گفت: چون باشد کسی کز دوستان باشد جدا؟
\\
\end{longtable}
\end{center}
