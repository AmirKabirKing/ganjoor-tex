\begin{center}
\section*{غزل شماره ۲۵۳: چه بد کردم؟ چه شد؟ از من چه دیدی}
\label{sec:253}
\addcontentsline{toc}{section}{\nameref{sec:253}}
\begin{longtable}{l p{0.5cm} r}
چه بد کردم؟ چه شد؟ از من چه دیدی؟
&&
که ناگه دامن از من درکشیدی
\\
چه افتادت که از من برشکستی؟
&&
چرا یکبارگی از من رمیدی؟
\\
به هر تردامنی رخ می‌نمایی
&&
چرا از دیدهٔ من ناپدیدی؟
\\
تو را گفتم که: مشنو گفت بد گوی
&&
علی‌رغم من مسکین شنیدی
\\
مرا گفتی: رسم روزیت فریاد
&&
عفا الله نیک فریادم رسیدی!
\\
دمی از پرده بیرون آی، باری
&&
که کلی پردهٔ صبرم دریدی
\\
هم از لطف تو بگشاید مرا کار
&&
که جمله بستگی‌ها را کلیدی
\\
نخستم برگزیدی از دو عالم
&&
چو طفلی در برم می‌پروریدی
\\
لب خود بر لب من می‌نهادی
&&
حیات تازه در من می‌دمیدی
\\
خوشا آن دم که با من شاد و خرم
&&
میان انجمن خوش می‌چمیدی
\\
ز بیم دشمنان با من نهانی
&&
لب زیرین به دندان می‌گزیدی
\\
چو عنقا، تا به چنگ آری مرا باز
&&
ورای هر دو عالم می‌پریدی
\\
مرا چون صید خود کردی، به آخر
&&
شدی با آشیان و آرمیدی
\\
تو با من آن زمان پیوستی، ای جان
&&
که بر قدم لباس خود بریدی
\\
از آن دم بازگشتی عاشق من
&&
که در من روی خوب خود بدیدی
\\
من ار چه از تو می‌آیم پدیدار
&&
تو نیز اندر جهان از من پدیدی
\\
مراد تو منم، آری، ولیکن
&&
چو وابینی تو خود خود را مریدی
\\
گزیدی هر کسی را بهر کاری
&&
عراقی را برای خود گزیدی
\\
\end{longtable}
\end{center}
