\begin{center}
\section*{غزل شماره ۱۵: دو اسبه پیک نظر می‌دوانم از چپ و راست}
\label{sec:015}
\addcontentsline{toc}{section}{\nameref{sec:015}}
\begin{longtable}{l p{0.5cm} r}
دو اسبه پیک نظر می‌دوانم از چپ و راست
&&
به جست و جوی نگاری، که نور دیدهٔ ماست
\\
مرا، که جز رخ او در نظر نمی‌آید
&&
دو دیده از هوس روی او پر آب چراست؟
\\
چو غرق آب حیاتم چه آب می‌جویم؟
&&
چو با من است نگارم چه می‌دوم چپ و راست؟
\\
نگاه کردم و در خود همه تو را دیدم
&&
نظر چنین نکند آن که او به خود بیناست
\\
به نور طلعت تو یافتم وجود تو را
&&
به آفتاب توان دید کآفتاب کجاست؟
\\
ز روی روشن هر ذره شد مرا روشن
&&
که آفتاب رخت در همه جهان پیداست
\\
به قامت خوش خوبان نگاه می‌کردم
&&
لباس حسن تو دیدم به قد هریک راست
\\
شمایل تو بدیدم ز قامت شمشاد
&&
ازین سپس کشش من همه سوی بالاست
\\
شگفت نیست که در بند زلف توست دلم
&&
که هرکجا که دلی هست اندر آن سوداست
\\
به غمزه گر نربودی دل همه عالم
&&
ز عشق تو دل جمله جهان چرا شیداست؟
\\
وگر جمال تو با عاشقان کرشمه نکرد
&&
ز بهر چه شر و آشوب از جهان برخاست؟
\\
ور از جهان سخن سر تو برون افتاد
&&
سزد، که راز نگه داشتن نه کار صداست
\\
ندید چشم عراقی تو را، چنان که تویی
&&
از آن که در نظرش جمله کاینات هباست
\\
\end{longtable}
\end{center}
