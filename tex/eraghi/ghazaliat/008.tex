\begin{center}
\section*{غزل شماره ۸: مست خراب یابد هر لحظه در خرابات}
\label{sec:008}
\addcontentsline{toc}{section}{\nameref{sec:008}}
\begin{longtable}{l p{0.5cm} r}
مست خراب یابد هر لحظه در خرابات
&&
گنجی که آن نیابد صد پیر در مناجات
\\
خواهی که راه یابی بی‌رنج بر سر گنج
&&
می‌بیز هر سحرگاه خاک در خرابات
\\
یک ذره گرد از آن خاک در چشم جانت افتد
&&
با صدهزار خورشید افتد تو را ملاقات
\\
ور عکس جام باده ناگاه بر تو تابد
&&
نز خویش گردی آگه، نز جام، نز شعاعات
\\
در بیخودی و مستی جایی رسی، که آنجا
&&
در هم شود عبادات، پی گم کند اشارات
\\
تا گم نگردی از خود گنجی چنین نیابی
&&
حالی چنین نیابد گم گشته از ملاقات
\\
تا کی کنی به عادت در صومعه عبادت؟
&&
کفر است زهد و طاعت تا نگذری ز میقات
\\
تا تو ز خودپرستی وز جست وجو نرستی
&&
می‌دان که می‌پرستی در دیر عزی و لات
\\
در صومعه تو دانی می‌کوش تا توانی
&&
در میکده رها کن از سر فضول و طامات
\\
جان باز در خرابات، تا جرعه‌ای بیابی
&&
مفروش زهد، کانجا کمتر خرند طامات
\\
لب تشنه چند باشی، در ساحل تمنی؟
&&
انداز خویشتن را در بحر بی‌نهایات
\\
تا گم شود نشانت در پای بی‌نشانی
&&
تا در کشد به کامت یک ره نهنگ حالات
\\
چون غرقه شد عراقی یابد حیات باقی
&&
اسرار غیب بیند در عالم شهادات
\\
\end{longtable}
\end{center}
