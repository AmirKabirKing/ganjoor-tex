\begin{center}
\section*{غزل شماره ۲۶۶: چه خوش باشد! که دلدارم تو باشی}
\label{sec:266}
\addcontentsline{toc}{section}{\nameref{sec:266}}
\begin{longtable}{l p{0.5cm} r}
چه خوش باشد! که دلدارم تو باشی
&&
ندیم و مونس و یارم تو باشی
\\
دل پر درد را درمان تو سازی
&&
شفای جان بیمارم تو باشی
\\
ز شادی در همه عالم نگنجم
&&
اگر یک لحظه غم خوارم تو باشی
\\
ندارم مونسی در غار گیتی
&&
بیا، تا مونس غارم تو باشی
\\
اگر چه سخت دشوار است کارم
&&
شود آسان، چو در کارم تو باشی
\\
اگر جمله جهانم خصم گردند
&&
نترسم، چون نگهدارم تو باشی
\\
همی نالم چو بلبل در سحرگاه
&&
به بوی آنکه گلزارم تو باشی
\\
چو گویم وصف حسن ماهرویی
&&
غرض زان زلف و رخسارم تو باشی
\\
اگر نام تو گویم ور نگویم
&&
مراد جمله گفتارم تو باشی
\\
از آن دل در تو بندم، چون عراقی
&&
که می‌خواهم که دلدارم تو باشی
\\
\end{longtable}
\end{center}
