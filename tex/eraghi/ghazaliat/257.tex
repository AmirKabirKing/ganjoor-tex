\begin{center}
\section*{غزل شماره ۲۵۷: تا چند عشق بازیم بر روی هر نگاری}
\label{sec:257}
\addcontentsline{toc}{section}{\nameref{sec:257}}
\begin{longtable}{l p{0.5cm} r}
تا چند عشق بازیم بر روی هر نگاری؟
&&
چون می‌شویم عاشق بر چهرهٔ تو باری
\\
از گلبن جمالت خاری است حسن خوبان
&&
مسکین کسی کزان گل قانع شود به خاری!
\\
خواهی که همچو زلفت عالم بهم بر آید؟ژ
&&
بنمای عاشقان را از طرهٔ تو تاری
\\
آن خوشدلی کجا شد؟ وان دور کو که ما را
&&
دیدار می‌نمودی، هر روز یک دو باری؟
\\
ما را ز هم جدا کرد ایام ورنه ما را
&&
با دولت وصالت خوش بود روزگاری
\\
در پرده چند باشی؟ برگیر برقع از روی
&&
تا روی تو ببیند یک دم امیدواری
\\
در انتظار وصلت جانم رسید بر لب
&&
از وصل تو چه حاصل، ما را جز انتظاری؟
\\
جام جهان نمایت بنمای، تا عراقی
&&
اندر رخت ببیند رخسار هر نگاری
\\
\end{longtable}
\end{center}
