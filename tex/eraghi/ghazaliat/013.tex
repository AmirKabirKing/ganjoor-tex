\begin{center}
\section*{غزل شماره ۱۳: ساقی قدحی شراب در دست}
\label{sec:013}
\addcontentsline{toc}{section}{\nameref{sec:013}}
\begin{longtable}{l p{0.5cm} r}
ساقی قدحی شراب در دست
&&
آمد ز شراب خانه سرمست
\\
آن توبهٔ نادرست ما را
&&
همچون سر زلف خویش بشکست
\\
از مجلسیان خروش برخاست
&&
کان فتنهٔ روزگار بنشست
\\
ماییم کنون و نیم جانی
&&
و آن نیز نهاده بر کف دست
\\
آن دل، که ازو خبر نداریم
&&
هم در سر زلف اوست گر هست
\\
دیوانهٔ روی اوست دایم
&&
آشفتهٔ موی اوست پیوست
\\
در سایهٔ زلف او بیاسود
&&
وز نیک و بد زمانه وارست
\\
چون دید شعاع روی خوبش
&&
در حال ز سایه رخت بربست
\\
در سایه مجو دل عراقی
&&
کان ذره به آفتاب پیوست
\\
\end{longtable}
\end{center}
