\begin{center}
\section*{غزل شماره ۲۵۱: چه کرده‌ام که دلم از فراق خون کردی}
\label{sec:251}
\addcontentsline{toc}{section}{\nameref{sec:251}}
\begin{longtable}{l p{0.5cm} r}
چه کرده‌ام که دلم از فراق خون کردی؟
&&
چه اوفتاد که درد دلم فزون کردی؟
\\
چرا ز غم دل پر حسرتم بیازردی؟
&&
چه شد که جان حزینم ز غصه خون کردی؟
\\
نخست ار چه به صد زاریم درون خواندی
&&
به آخر از چه به صد خواریم برون کردی؟
\\
همه حدیث وفا و وصال می‌گفتی
&&
چو عاشق تو شدم قصه واژگون کردی
\\
ز اشتیاق تو جانم به لب رسید، بیا
&&
نظر به حال دلم کن، ببین که: چون کردی؟
\\
لوای عشق برافراختی چنان در دل
&&
که در زمان، علم صبر سرنگون کردی
\\
کنون که با تو شدم راست چون الف یکتا
&&
ز بار محنت، پشتم دو تا چو نون کردی
\\
نگفته بودی، بیداد کم کنم روزی؟
&&
چو کم نکردی باری چرا فزون کردی؟
\\
هزار بار بگفتی نکو کنم کارت
&&
نکو نکردی و از بد بتر کنون کردی
\\
به دشمنی نکند هیچ کس به جان کسی
&&
که تو به دوستی آن با من زبون کردی
\\
بسوختی دل و جانم، گداختی جگرم
&&
به آتش غمت از بسکه آزمون کردی
\\
کجا به درگه وصل تو ره توانم یافت؟
&&
چو تو مرا به در هجر رهنمون کردی
\\
سیاهروی دو عالم شدم، که در خم فقر
&&
گلیم بخت عراقی سیاه گون کردی
\\
\end{longtable}
\end{center}
