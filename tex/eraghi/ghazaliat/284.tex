\begin{center}
\section*{غزل شماره ۲۸۴: نگویی باز: کای غم خوار چونی}
\label{sec:284}
\addcontentsline{toc}{section}{\nameref{sec:284}}
\begin{longtable}{l p{0.5cm} r}
نگویی باز: کای غم خوار چونی؟
&&
همیشه با غم و تیمار چونی؟
\\
کجایی؟ با فراقم در چه کاری؟
&&
جدا افتاده از دلدار چونی؟
\\
مرا دانی که بیمارم ز تیمار
&&
نپرسی هیچ: کای بیمار چونی؟
\\
نیاری یاد از من: کای ز غم زار
&&
درین رنج و غم بسیار چونی؟
\\
مرا گر چه ز غم جان بر لب آمد
&&
نخواهی گفت: کای غم خوار چونی؟
\\
تو گر چه بینیم غلتان به خون در
&&
نگویی آخر: ای افگار چونی؟
\\
سحرگه با خیالت دیده می‌گفت:
&&
که هر شب با من بیدار چونی؟
\\
خیالت گفت: کری نیک زارم
&&
ز بهر تو، که هر شب زار چونی؟
\\
سگ کویت عراقی را نگوید
&&
شبی: کای یار من، بی یار چونی؟
\\
\end{longtable}
\end{center}
