\begin{center}
\section*{غزل شماره ۱۳۰: به دست غم گرفتارم، بیا ای یار، دستم گیر}
\label{sec:130}
\addcontentsline{toc}{section}{\nameref{sec:130}}
\begin{longtable}{l p{0.5cm} r}
به دست غم گرفتارم، بیا ای یار، دستم گیر
&&
به رنج دل سزاوارم، مرا مگذار، دستم گیر
\\
یکی دل داشتم پر خون، شد آن هم از کفم بیرون
&&
چو کار از دست شد بیرون، بیا ای یار، دستم گیر
\\
ز وصلت تا جدا ماندم همیشه در عنا ماندم
&&
از آن دم کز تو واماندم شدم بیمار، دستم گیر
\\
کنون در حال من بنگر: که عاجز گشتم و مضطر
&&
مرا مگذار و خود مگذر، درین تیمار دستم گیر
\\
به جان آمد دلم، ای جان، ز دست هجر بی‌پایان
&&
ندارم طاقت هجران، به جان، زنهار، دستم گیر
\\
همیشه گرد کوی تو همی گردم به بوی تو
&&
ندیدم رنگ روی تو، از آنم زار، دستم گیر
\\
چو کردی حلقه در گوشم، مکن آزاد و مفروشم
&&
مکن جانا فراموشم، ز من یاد آر، دستم گیر
\\
شنیدی آه و فریادم، ندادی از کرم دادم
&&
کنون کز پا درافتادم، مرا بردار، دستم گیر
\\
نیابم در جهان یاری، نبینم غیر غم‌خواری
&&
ندارم هیچ دلداری، تویی دلدار، دستم گیر
\\
عراقی، چون نه‌ای خرم، گرفتاری به دست غم
&&
فغان کن بر درش هر دم، که ای غمخوار، دستم گیر
\\
\end{longtable}
\end{center}
