\begin{center}
\section*{غزل شماره ۱۸۶: بر در یار من سحر مست و خراب می‌روم}
\label{sec:186}
\addcontentsline{toc}{section}{\nameref{sec:186}}
\begin{longtable}{l p{0.5cm} r}
بر در یار من سحر مست و خراب می‌روم
&&
جام طرب کشیده‌ام، زآن به شتاب می‌روم
\\
ساغری از می لبش دوش سؤال کرده‌ام
&&
وقت سحر به کوی او بهر جواب می‌روم
\\
از می ناب جزع او گرچه خراب گشته‌ام
&&
تا دهد از کرشمه‌ام باز شراب، می‌روم
\\
بر سر خوان درد او درد بسی کشیده‌ام
&&
تا کشم از دو لعل او بادهٔ ناب می‌روم
\\
جذبهٔ حسن دلکشش می‌کشدم به سوی خود
&&
از پی آن کشش دگر، همچو ذباب می‌روم
\\
برقع تن ز شوق او پیش رخش گشادمی
&&
لیک ز شرم روی او بسته نقاب می‌روم
\\
در سر باده می‌کنم هستی خویش هر زمان
&&
خاک رهم، رواست گر بر سر آب می‌روم
\\
شحنهٔ عشق هر شبی بر کندم ز خواب خوش
&&
در هوس خیال او باز به خواب می‌روم
\\
شاید اگر هوای او می‌کشدم، که در رهش
&&
بر سر آب چشم خود همچو حباب می‌روم
\\
بیخود اگر ز صومعه بر در میکده روم
&&
گر تو خطا گمان بری راه صواب می‌روم
\\
نیست مرا ز خود خبر، بیش ازین که: در جهان
&&
مست و خراب آمدم، مست و خراب می‌روم
\\
\end{longtable}
\end{center}
