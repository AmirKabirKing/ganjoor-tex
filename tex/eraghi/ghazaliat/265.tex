\begin{center}
\section*{غزل شماره ۲۶۵: خوشا دردی!که درمانش تو باشی}
\label{sec:265}
\addcontentsline{toc}{section}{\nameref{sec:265}}
\begin{longtable}{l p{0.5cm} r}
خوشا دردی!که درمانش تو باشی
&&
خوشا راهی! که پایانش تو باشی
\\
خوشا چشمی!که رخسار تو بیند
&&
خوشا ملکی! که سلطانش تو باشی
\\
خوشا آن دل! که دلدارش تو گردی
&&
خوشا جانی! که جانانش تو باشی
\\
خوشی و خرمی و کامرانی
&&
کسی دارد که خواهانش تو باشی
\\
چه خوش باشد دل امیدواری
&&
که امید دل و جانش تو باشی!
\\
همه شادی و عشرت باشد، ای دوست
&&
در آن خانه که مهمانش تو باشی
\\
گل و گلزار خوش آید کسی را
&&
که گلزار و گلستانش تو باشی
\\
چه باک آید ز کس؟ آن را که او را
&&
نگهدار و نگهبانش تو باشی
\\
مپرس از کفر و ایمان بی‌دلی را
&&
که هم کفر و هم ایمانش تو باشی
\\
مشو پنهان از آن عاشق که پیوست
&&
همه پیدا و پنهانش تو باشی
\\
برای آن به ترک جان بگوید
&&
دل بیچاره، تا جانش تو باشی
\\
عراقی طالب درد است دایم
&&
به بوی آنکه درمانش تو باشی
\\
\end{longtable}
\end{center}
