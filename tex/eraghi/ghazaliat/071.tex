\begin{center}
\section*{غزل شماره ۷۱: بیا، که عمر من خاکسار می‌گذرد}
\label{sec:071}
\addcontentsline{toc}{section}{\nameref{sec:071}}
\begin{longtable}{l p{0.5cm} r}
بیا، که عمر من خاکسار می‌گذرد
&&
مدار منتظرم، روزگار می‌گذرد
\\
بیا، که جان من از آرزوی دیدارت
&&
به لب رسید و غم دل فگار می‌گذرد
\\
بیا، به لطف ز جان به لب رسیده بپرس
&&
که از جهان ز غمت زار زار می‌گذرد
\\
بر آن شکسته دلی رحم کن ز روی کرم
&&
که ناامید ز درگاه یار می‌گذرد
\\
چه باشد ار بگذاری که بگذرم ز درت؟
&&
که بر درت ز سگان صدهزار می‌گذرد
\\
مکش کمان جفا بر دلم، که تیر غمت
&&
خود از نشانهٔ جان بی‌شمار می‌گذرد
\\
من ار چه دورم از درگهت دلم هر دم
&&
بر آستان درت چندبار می‌گذرد
\\
ز دل که می‌گذرد بر درت بپرس آخر:
&&
که آن شکسته برین در چه کار می‌گذرد
\\
مکش چو دشمنم، ای دوست ز انتظار، بیا
&&
که این نفس ز جهان دوستدار می‌گذرد
\\
به انتظار مکش بیش ازین عراقی را
&&
که عمر او همه در انتظار می‌گذرد
\\
\end{longtable}
\end{center}
