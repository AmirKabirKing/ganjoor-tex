\begin{center}
\section*{غزل شماره ۲۵۹: نگارا، از وصال خود مرا تا کی جدا داری}
\label{sec:259}
\addcontentsline{toc}{section}{\nameref{sec:259}}
\begin{longtable}{l p{0.5cm} r}
نگارا، از وصال خود مرا تا کی جدا داری؟
&&
چو شادم می‌توانی داشت، غمگینم چرا داری؟
\\
چه دلداری؟ که هر لحظه دلم از غم به جان آری
&&
چه غم خواری؟ که هر ساعت تنم را در بلا داری
\\
به کام دشمنم داری و گویی: دوست می‌دارم
&&
چگونه دوستی باشد، که جانم در عنا داری؟
\\
چه دانم؟ تا چه اجر آرم من مسکین بجای تو
&&
که گر گردم هلاک از غم من مسکین، روا داری
\\
بکن رحمی که مسکینم، ببخشایم که غمگینم
&&
بمیرم گر چنین، دانم مرا از خود جدا داری
\\
مرا گویی: مشو غمگین، که خوش دارم تو را روزی
&&
چو می‌گردم هلاک از غم تو آنگه خوش مرا داری!
\\
عراقی کیست تا لافد ز عشق تو؟ که در هر کو
&&
میان خاک و خون غلتان چو او صد مبتلا داری
\\
\end{longtable}
\end{center}
