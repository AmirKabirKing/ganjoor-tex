\begin{center}
\section*{غزل شماره ۸۲: اگر یکبار زلف یار از رخسار برخیزد}
\label{sec:082}
\addcontentsline{toc}{section}{\nameref{sec:082}}
\begin{longtable}{l p{0.5cm} r}
اگر یکبار زلف یار از رخسار برخیزد
&&
هزاران آه مشتاقان ز هر سو زار برخیزد
\\
وگر غمزه‌اش کمین سازد دل از جان دست بفشاند
&&
وگر زلفش برآشوبد ز جان زنهار برخیزد
\\
چو رویش پرده بگشاید که و صحرا به رقص آید
&&
چو عشقش روی بنماید خرد ناچار برخیزد
\\
صبا گر از سر زلفش به گورستان برد بویی
&&
ز هر گوری دو صد بی‌دل ز بوی یار برخیزد
\\
نسیم زلفش ار ناگه به ترکستان گذر سازد
&&
هزاران عاشق از سقسین و از بلغار برخیزد
\\
نوای مطرب عشقش اگر در گوش جان آید
&&
ز کویش دست بفشاند قلندروار برخیزد
\\
چو یاد او شود مونس ز جان اندوه بنشیند
&&
چو اندوهش شود غم خور ز دل تیمار برخیزد
\\
دلا بی‌عشق او منشین ز جان برخیز و سر در باز
&&
چو عیاران مکن کاری که گرد از کار برخیزد
\\
درین دریا فگن خود را مگر دری به دست آری
&&
کزین دریای بی‌پایان گهر بسیار برخیزد
\\
وگر موجیت برباید، زهی دولت، تو را آن به
&&
که عالم پیش قدر تو چو خدمتکار برخیزد
\\
حجاب ره تویی برخیز و در فتراک عشق آویز
&&
که بی‌عشق آن حجاب تو ز ره دشوار برخیزد
\\
عراقی، هر سحرگاهی بر آر از سوز دل آهی
&&
ز خواب این دیدهٔ بختت مگر یکبار برخیزد
\\
\end{longtable}
\end{center}
