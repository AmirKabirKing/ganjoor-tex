\begin{center}
\section*{غزل شماره ۷۸: از در یار گذر نتوان کرد}
\label{sec:078}
\addcontentsline{toc}{section}{\nameref{sec:078}}
\begin{longtable}{l p{0.5cm} r}
از در یار گذر نتوان کرد
&&
رخ سوی یار دگر نتوان کرد
\\
ناگذشته ز سر هر دو جهان
&&
بر سر کوش گذر نتوان کرد
\\
زان چنان رخ، که تمنای دل است
&&
صبر ازین بیش مگر نتوان کرد
\\
با چنین دیده، که پرخوناب است
&&
به چنان روی نظر نتوان کرد
\\
چون حدیث لب شیرینش رود
&&
یاد حلوا و شکر نتوان کرد
\\
سخن زلف مشوش بگذار
&&
دل ازین شیفته‌تر نتوان کرد
\\
قصهٔ درد دل خود چه کنم؟
&&
راز خود جمله سمر نتوان کرد
\\
غم او مایهٔ عیش و طرب است
&&
از طرب بیش حذر نتوان کرد
\\
گرچه دل خون شود از تیمارش
&&
غمش از سینه به در نتوان کرد
\\
ابتلایی است درین راه مرا
&&
که از آن هیچ خبر نتوان کرد
\\
گفتم: ای دل، بگذر زین منزل
&&
محنت آباد مقر نتوان کرد
\\
گفت: جایی که عراقی باشد
&&
زود از آنجای سفر نتوان کرد
\\
\end{longtable}
\end{center}
