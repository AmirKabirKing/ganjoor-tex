\begin{center}
\section*{غزل شماره ۲۹۸: کشید کار ز تنهاییم به شیدایی}
\label{sec:298}
\addcontentsline{toc}{section}{\nameref{sec:298}}
\begin{longtable}{l p{0.5cm} r}
کشید کار ز تنهاییم به شیدایی
&&
ندانم این همه غم چون کشم به تنهایی؟
\\
ز بس که داد قلم شرح سرنوشت فراق
&&
ز سرنوشت قلم نامه گشت سودایی
\\
مرا تو عمر عزیزی و رفته‌ای ز برم
&&
چو خوش بود اگر، ای عمر رفته بازآیی
\\
زبان گشاده، کمر بسته‌ایم، تا چو قلم
&&
به سر کنیم هر آن خدمتی که فرمایی
\\
به احتیاط گذر بر سواد دیدهٔ من
&&
چنان که گوشهٔ دامن به خون نیالایی
\\
نه مرد عشق تو بودم ازین طریق، که عقل
&&
درآمده است به سر، با وجود دانایی
\\
درم گشای، که امید بسته‌ام در تو
&&
در امید که بگشاید؟ ار تو نگشایی
\\
به آفتاب خطاب تو خواستم کردن
&&
دلم نداد، که هست آفتاب هر جایی
\\
سعادت دو جهان است دیدن رویت
&&
زهی! سعادت، اگر زان چه روی بنمایی!
\\
\end{longtable}
\end{center}
