\begin{center}
\section*{غزل شماره ۱۲۴: مرا از هر چه می‌بینم رخ دلدار اولی‌تر}
\label{sec:124}
\addcontentsline{toc}{section}{\nameref{sec:124}}
\begin{longtable}{l p{0.5cm} r}
مرا از هر چه می‌بینم رخ دلدار اولی‌تر
&&
نظر چون می‌کنم باری بدان رخسار اولی‌تر
\\
تماشای رخ خوبان خوش است، آری، ولی ما را
&&
تماشای رخ دلدار از آن بسیار اولی‌تر
\\
بیا، ای چشم من، جان و جمال روی جانان بین
&&
چو عاشق می‌شوم باری، بدان رخسار اولی‌تر
\\
ز رویش هرچه بگشایم نقاب روی او اولی
&&
ز زلفش هر چه بر بندم، مرا زنار اولی‌تر
\\
کسی کاهل مناجات است او را کنج مسجد به
&&
مرا، کاهل خراباتم، در خمار اولی‌تر
\\
فریب غمزهٔ ساقی چو بستاند مرا از من
&&
لبش با جان من در کار و من بی‌کار اولی‌تر
\\
چو زان می درکشم جامی، جهان را جرعه‌ای بخشم
&&
جهان از جرعهٔ من مست و من هشیار اولی‌تر
\\
به یک ساغر در آشامم همه دریای مستی را
&&
چو ساغر می‌کشم، باری، قلندروار اولی‌تر
\\
خرد گفتا: به پیران سر چه گردی گرد میخانه؟
&&
ازین رندی و قلاشی شوی بیزار اولی‌تر
\\
نهان از چشم خود ساقی مرا گفتا: فلان، می خور
&&
که عاشق در همه حالی چو من می‌خوار اولی‌تر
\\
عراقی را به خود بگذار و بی‌خود در خرابات آی
&&
که این جا یک خراباتی ز صد دین‌دار اولی‌تر
\\
\end{longtable}
\end{center}
