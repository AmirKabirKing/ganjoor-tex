\begin{center}
\section*{غزل شماره ۷۰: بیا بیا، که نسیم بهار می‌گذرد}
\label{sec:070}
\addcontentsline{toc}{section}{\nameref{sec:070}}
\begin{longtable}{l p{0.5cm} r}
بیا بیا، که نسیم بهار می‌گذرد
&&
بیا، که گل ز رخت شرمسار می‌گذرد
\\
بیا، که وقت بهار است و موسم شادی
&&
مدار منتظرم، وقت کار می‌گذرد
\\
ز راه لطف به صحرا خرام یک نفسی
&&
که عیش تازه کنم، چون بهار می‌گذرد
\\
نسیم لطف تو از کوی می‌برد هر دم
&&
غمی که بر دل این جان فگار می‌گذرد
\\
ز جام وصل تو ناخورده جرعه‌ای دل من
&&
ز بزم عیش تو در سر خمار می‌گذرد
\\
سحرگهی که به کوی دلم گذر کردی
&&
به دیده گفت دلم: کان شکار می‌گذرد
\\
چو دیده کرد نظر صدهزار عاشق دید
&&
که نعره می‌زد هر یک که: یار می‌گذرد
\\
به گوش جان عراقی رسید آن زاری
&&
از آن ز کوی تو زار و نزار می‌گذرد
\\
\end{longtable}
\end{center}
