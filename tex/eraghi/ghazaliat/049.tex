\begin{center}
\section*{غزل شماره ۴۹: بی‌رخت جان در میان نتوان نهاد}
\label{sec:049}
\addcontentsline{toc}{section}{\nameref{sec:049}}
\begin{longtable}{l p{0.5cm} r}
بی‌رخت جان در میان نتوان نهاد
&&
بی‌یقین پا بر گمان نتوان نهاد
\\
جان بباید داد و بستد بوسه‌ای
&&
بی‌کنارت در میان نتوان نهاد
\\
نیم‌جانی دارم از تو یادگار
&&
بر لبت لب رایگان نتوان نهاد
\\
در جهان چشمت خرابی می‌کند
&&
جرم بر دور زمان نتوان نهاد
\\
خون ما ز ابرو و مژگان ریختی
&&
تیر به زین در کمان نتوان نهاد
\\
حال من زلفت پریشان می‌کند
&&
پس گنه بر دیگران نتوان نهاد
\\
در جهان چون هرچه خواهی می‌کنی
&&
جرم بر هر ناتوان نتوان نهاد
\\
هر چه هست اندر همه عالم تویی
&&
نام هستی بر جهان نتوان نهاد
\\
چون تو را، جز تو، نمی‌بیند کسی
&&
منتی بر عاشقان نتوان نهاد
\\
بر در وصلت چو کس می‌گذرد
&&
تهمتی بر انس و جان نتوان نهاد
\\
عاشق تو هم تو بس، پس نام عشق
&&
گه برین و گه بر آن نتوان نهاد
\\
تا نگیرد دست من دامان تو
&&
پای دل بر فرق جان نتوان نهاد
\\
چون عراقی آستین ما گرفت
&&
رخت او بر آسمان نتوان نهاد
\\
\end{longtable}
\end{center}
