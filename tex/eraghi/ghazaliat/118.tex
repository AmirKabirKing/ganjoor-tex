\begin{center}
\section*{غزل شماره ۱۱۸: ای باد صبا، به کوی آن یار}
\label{sec:118}
\addcontentsline{toc}{section}{\nameref{sec:118}}
\begin{longtable}{l p{0.5cm} r}
ای باد صبا، به کوی آن یار
&&
گر بر گذری ز بنده یاد آر
\\
ور هیچ مجال گفت یابی
&&
پیغام من شکسته بگزار
\\
با یار بگوی کان شکسته
&&
این خسته جگر، غریب و غم‌خوار
\\
چون از تو ندید چارهٔ خویش
&&
بیچاره بماند بی‌تو ناچار
\\
خورشید رخت ندید روزی
&&
بی‌نور بماند در شب تار
\\
نی این شب تیره دید روشن
&&
نی خفته عدو، نه بخت بیدار
\\
می‌کرد شبی به روز کاخر
&&
روزی بشود که به شود کار
\\
کارش چو به جان رسید می‌گفت:
&&
کای کرده به تیغ هجرم افگار
\\
ای کرده به کام دشمنانم
&&
با یار چنین، چنین کند یار؟
\\
آخر نظری به حال من کن
&&
بنگر که: چگونه بی‌توام زار؟
\\
یک بارگیم مکن فراموش
&&
یاد آر ز من شکسته، یاد آر
\\
مزار ز من، که هیچ هیچم
&&
از هیچ، کسی نگیرد آزار
\\
من نیک بدم، تو نیکویی کن
&&
ای نیک، بدم، به نیک بردار
\\
بگذار که بگذرم به کویت
&&
یکدم ز سگان کویم انگار
\\
بگذاشتم این حدیث، کز من
&&
دارند سگان کوی تو عار
\\
پندار که مشت خاک باشم
&&
زیر قدم سگ درت خوار
\\
القصه به جانم از عراقی
&&
مگذار، کزو نماند آثار
\\
بالجمله تو باشی و تو گویی
&&
او کم کند از میانه گفتار
\\
\end{longtable}
\end{center}
