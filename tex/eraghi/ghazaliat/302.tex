\begin{center}
\section*{غزل شماره ۳۰۲: درین ره گر بترک خود بگویی}
\label{sec:302}
\addcontentsline{toc}{section}{\nameref{sec:302}}
\begin{longtable}{l p{0.5cm} r}
درین ره گر بترک خود بگویی
&&
یقین گردد تو را کو تو، تو اویی
\\
سر مویی ز تو، تا با تو باقی است
&&
درین ره در نگنجی، گر چه مویی
\\
کم خود گیر، تا جمله تو باشی
&&
روان شو سوی دریا، زانکه جویی
\\
چو با دریا گرفتی آشنایی
&&
مجرد شو، ز سر برکش دو تویی
\\
درین دریا گلیمت شسته گردد
&&
اگر یک بار دست از خود بشویی
\\
ز بهر آبرو یک رویه کن کار
&&
که آنجا آبرو ریزد دورویی
\\
چو با توست آنچه می‌جویی به هرجا
&&
به هرزه گرد عالم چند پویی؟
\\
نخستین گم کنند آنگاه جویند
&&
تو چون چیزی نکردی ؟ گم؟ چه جویی؟
\\
تو را تا در درون صد خار خار است
&&
ازین بستان گلی هرگز نبویی
\\
پس در همچو جادویی که پیوست
&&
میان در بسته بهر رفت و رویی
\\
تو را رنگی ندادند از خم عشق
&&
از آن در آرزوی رنگ و بویی
\\
بهش نه پا درین وادی خون خوار
&&
که ره پر سنگلاخ و تو سبویی
\\
درین میدان همی خور زخم، چون تو
&&
فتاده در خم چوگان چو گویی
\\
نیابی از خم چوگان رهایی
&&
عراقی، تا به ترک خود نگویی
\\
\end{longtable}
\end{center}
