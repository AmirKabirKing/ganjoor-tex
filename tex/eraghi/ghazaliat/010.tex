\begin{center}
\section*{غزل شماره ۱۰: به یک گره که دو چشمت بر ابروان انداخت}
\label{sec:010}
\addcontentsline{toc}{section}{\nameref{sec:010}}
\begin{longtable}{l p{0.5cm} r}
به یک گره که دو چشمت بر ابروان انداخت
&&
هزار فتنه و آشوب در جهان انداخت
\\
فریب زلف تو با عاشقان چه شعبده ساخت؟
&&
که هر که جان و دلی داشت در میان انداخت
\\
دلم، که در سر زلف تو شد، توان گه گه
&&
ز آفتاب رخت سایه‌ای بر آن انداخت
\\
رخ تو در خور چشم من است، لیک چه سود
&&
که پرده از رخ تو برنمی‌توان انداخت
\\
حلاوت لب تو، دوش، یاد می‌کردم
&&
بسا شکر که در آن لحظه در دهان انداخت
\\
من از وصال تو دل برگرفته بودم، لیک
&&
زبان لطف توام باز در گمان انداخت
\\
قبول تو دگران را به صدر وصل نشاند
&&
دل شکستهٔ ما را بر آستان انداخت
\\
چه قدر دارد، جانا، دلی؟ توان هردم
&&
بر آستان درت صدهزار جان انداخت
\\
عراقی از دل و جان آن زمان امید برید
&&
که چشم جادوی تو چین در ابروان انداخت
\\
\end{longtable}
\end{center}
