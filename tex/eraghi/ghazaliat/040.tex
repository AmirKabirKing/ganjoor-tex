\begin{center}
\section*{غزل شماره ۴۰: آه، به یک‌بارگی یار کم ما گرفت}
\label{sec:040}
\addcontentsline{toc}{section}{\nameref{sec:040}}
\begin{longtable}{l p{0.5cm} r}
آه، به یک‌بارگی یار کم ما گرفت!
&&
چون دل ما تنگ دید خانه دگر جا گرفت
\\
بر دل ما گه گهی، داشت خیالی گذر
&&
نیز خیالش کنون ترک دل ما گرفت
\\
دل به غمش بود شاد، رفت غمش هم ز دل
&&
غم چه کند در دلی کان همه سودا گرفت؟
\\
دیدهٔ گریان مگر بر جگر آبی زند؟
&&
کاتش سودای او در دل شیدا گرفت
\\
خوش سخنی داشتم، با دل پردرد خویش
&&
لشکر هجران بتاخت در سر من تا گرفت
\\
دین و دل و هوش من هر سه به تاراج برد
&&
جان و تن و هرچه بود جمله به یغما گرفت
\\
هجر مگر در جهان هیچ کسی را نیافت
&&
کز همه وامانده‌ای، هیچکسی را گرفت
\\
هیچ کسی در جهان یار عراقی نشد
&&
لاجرمش عشق یار، بی‌کس و تنها گرفت
\\
\end{longtable}
\end{center}
