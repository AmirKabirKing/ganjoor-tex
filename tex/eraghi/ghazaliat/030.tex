\begin{center}
\section*{غزل شماره ۳۰: رخ نگار مرا هر زمان دگر رنگ است}
\label{sec:030}
\addcontentsline{toc}{section}{\nameref{sec:030}}
\begin{longtable}{l p{0.5cm} r}
رخ نگار مرا هر زمان دگر رنگ است
&&
به زیر هر خم زلفش هزار نیرنگ است
\\
کرشمه‌ای بکند، صدهزار دل ببرد
&&
ازین سبب دل عشاق در جهان تنگ است
\\
اگر برفت دل از دست، گو: برو، که مرا
&&
بجای دل سر زلف نگار در چنگ است
\\
از آن گهی که خراباتیی دلم بربود
&&
مرا هوای خرابات و باده و چنگ است
\\
بدین صفت که منم، از شراب عشق خراب
&&
مرا چه جای کرامات و نام یا ننگ است؟
\\
بیار ساقی، از آن می، که ساغر او را
&&
ز عکس چهرهٔ تو هر زمان دگر رنگ است
\\
بریز خون عراقی و آشتی وا کن
&&
که آشتی بهمه حال بهتر از جنگ است
\\
\end{longtable}
\end{center}
