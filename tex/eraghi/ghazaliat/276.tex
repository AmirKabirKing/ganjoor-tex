\begin{center}
\section*{غزل شماره ۲۷۶: گر به رخسار تو، ای دوست، نظر داشتمی}
\label{sec:276}
\addcontentsline{toc}{section}{\nameref{sec:276}}
\begin{longtable}{l p{0.5cm} r}
گر به رخسار تو، ای دوست، نظر داشتمی
&&
نظر از روی خوشت بهر چه برداشتمی؟
\\
چون من بی‌خبر از دوست دهندم خبری
&&
باری، از بی‌خبری کاش خبر داشتمی؟
\\
در میان آمدمی چون سر زلفت با تو
&&
از سر زلف تو گر هیچ کمر داشتمی؟
\\
گر ندادی جگرم وعدهٔ وصلت هر دم
&&
کی دل و دیده پر از خون جگر داشتمی؟
\\
گفتیم: صبر کن، از صبر برآید کارت
&&
کردمی صبر ز روی تو، اگر داشتمی
\\
خود کجا آمدی اندر نظرم آب روان؟
&&
گر ز خاک در تو کحل بصر داشتمی
\\
دل گم گشتهٔ خود بار دگر یافتمی
&&
بر سر کوی تو گر هیچ گذر داشتمی
\\
گر ز روی و لب تو هیچ نصیبم بودی
&&
بهر بیماری دل گل بشکر داشتمی
\\
کردمی بر سر کویت گهرافشانی‌ها
&&
بجز از اشک اگر هیچ گهر داشتمی
\\
گر عراقی نشدی پردهٔ روی نظرم
&&
به رخ خوب تو هر لحظه نظر داشتمی
\\
\end{longtable}
\end{center}
