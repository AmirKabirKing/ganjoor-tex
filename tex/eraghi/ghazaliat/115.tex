\begin{center}
\section*{غزل شماره ۱۱۵: صبا وقت سحر، گویی، ز کوی یار می‌آید}
\label{sec:115}
\addcontentsline{toc}{section}{\nameref{sec:115}}
\begin{longtable}{l p{0.5cm} r}
صبا وقت سحر، گویی، ز کوی یار می‌آید
&&
که بوی او شفای جان هر بیمار می‌آید
\\
نسیم او مگر در باغ جلوه می‌دهد گل را
&&
که آواز خوش بلبل ز هر سو زار می‌آید
\\
مگر از زلف دلدارم صبا بویی به باغ آورد
&&
که از باغ و گل و گلزار بوی یار می‌آید
\\
از آن چون بلبل بی‌دل ز رنگ و بوی گل شادم
&&
که از گلزار در چشمم رخ دلدار می‌آید
\\
گر آید در نظر کس را به جز رخسار او رویی
&&
مرا باری نظر دایم بر آن رخسار می‌آید
\\
مرا از هرچه در عالم به چشم اندر نیامد هیچ
&&
مگر آبی که در چشمم دمی صد بار می‌آید
\\
چو اندر آب عکس یار خوشتر می‌شود پیدا
&&
از آنروز آب در چشمم مگر بسیار می‌آید
\\
جهان آب است و من در وی جمال یار می‌بینم
&&
ازینجا خواب در چشمم مگر بسیار می‌آید
\\
عراقی در چنین خوابی همی بیند چنان رویی
&&
از آن در خاطرش هر دم هزاران کار می‌آید
\\
\end{longtable}
\end{center}
