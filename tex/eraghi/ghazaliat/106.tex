\begin{center}
\section*{غزل شماره ۱۰۶: تا کی از ما یار ما پنهان بود}
\label{sec:106}
\addcontentsline{toc}{section}{\nameref{sec:106}}
\begin{longtable}{l p{0.5cm} r}
تا کی از ما یار ما پنهان بود؟
&&
چشم ما تا کی چنین گریان بود؟
\\
تا کی از وصلش نصیب بخت ما
&&
محنت و درد دل و هجران بود؟
\\
این چنین کز یار دور افتاده‌ام
&&
گر بگرید دیده، جای آن بود
\\
چون دل ما خون شد از هجران او
&&
چشم ما شاید که خون افشان بود
\\
از فراقش دل ز جان آمد به جان
&&
خود گرانی یار مرگ جان بود
\\
بر امیدی زنده‌ام، ورنه که را
&&
طاقت آن هجر بی‌پایان بود؟
\\
پیچ بر پیچ است بی او کار ما
&&
کار ما تا کی چنین پیچان بود؟
\\
محنت آباد دل پر درد ما
&&
تا کی از هجران او ویران بود؟
\\
درد ما را نیست درمان در جهان
&&
درد ما را روی او درمان بود
\\
چون دل ما از سر جان برنخاست
&&
لاجرم پیوسته سرگردان بود
\\
چون عراقی هر که دور از یار ماند
&&
چشم او گریان، دلش بریان بود
\\
\end{longtable}
\end{center}
