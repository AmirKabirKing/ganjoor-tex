\begin{center}
\section*{غزل شماره ۱۴۹: گر آفتاب رخت سایه افکند بر خاک}
\label{sec:149}
\addcontentsline{toc}{section}{\nameref{sec:149}}
\begin{longtable}{l p{0.5cm} r}
گر آفتاب رخت سایه افکند بر خاک
&&
زمینیان همه دامن کشند بر افلاک
\\
به من نگر، که به من ظاهر است حسن رخت
&&
شعاع خور ننماید، اگر نباشد خاک
\\
دل من آینهٔ توست، پاک می‌دارش
&&
که روی پاک نماید، بود چو آینه پاک
\\
لبت تو بر لب من نه، ببار و بوسه بده
&&
چو جان من به لب آمد چه می‌کنم تریاک؟
\\
به تیر غمزه مرا می‌زنی و می‌ترسم
&&
که بر تو آید تیری که می زنی بی‌باک
\\
برای صورت خود سوی من نگاه کنی
&&
برای آنکه به من حسن خود کنی ادراک
\\
مرا به زیور هستی خود بیارایی
&&
و گرنه سوی عدم نظر کنی؟حاشاک
\\
اگر نبودی بر من لباس هستی تو
&&
ز بی‌نیازی تو کردمی گریبان چاک
\\
مده ز دست به یک بارگی عراقی را
&&
کف تو نیست محیطی که رد کند خاشاک
\\
\end{longtable}
\end{center}
