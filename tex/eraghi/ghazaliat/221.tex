\begin{center}
\section*{غزل شماره ۲۲۱: ای رخ جان فزای تو گشته خجسته فال من}
\label{sec:221}
\addcontentsline{toc}{section}{\nameref{sec:221}}
\begin{longtable}{l p{0.5cm} r}
ای رخ جان فزای تو گشته خجسته فال من
&&
باز نمای رخ، که شد بی تو تباه حال من
\\
ناز مکن، که می‌کند جان من آرزوی تو
&&
عشوه مده، که می‌دهد هجر تو گوشمال من
\\
رفت دل و نمی‌رود آرزوی تو از دلم
&&
عمر شد و نمی‌شود نقش تو از خیال من
\\
باز نگر که: می‌کشد بی تو مرا فراق تو
&&
چارهٔ من بکن، مجو بی سببی زوال من
\\
ز آرزوی جمال تو، نیست مرا ز خود خبر
&&
طعنه مزن، که: نیستی شیفتهٔ جمال من
\\
بر سر کوی وصل تو مرغ صفت پریدمی
&&
آه! اگر نسوختی آتش هجر بال من
\\
آمدمی به درگهت هر نفسی هزار بار
&&
گر نه عراقی آمدی سد ره وصال من
\\
\end{longtable}
\end{center}
