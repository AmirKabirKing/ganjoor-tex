\begin{center}
\section*{غزل شماره ۱۴۳: کردم گذری به میکده دوش}
\label{sec:143}
\addcontentsline{toc}{section}{\nameref{sec:143}}
\begin{longtable}{l p{0.5cm} r}
کردم گذری به میکده دوش
&&
سبحه به کف و سجاده بر دوش
\\
پیری به در آمد از خرابات
&&
کین جا نخرند زرق، مفروش
\\
تسبیح بده، پیاله بستان
&&
خرقه بنه و پلاس درپوش
\\
در صومعه بیهده چه باشی؟
&&
در میکده رو، شراب می‌نوش
\\
گر یاد کنی جمال ساقی
&&
جان و دل و دین کنی فراموش
\\
ور بینی عکس روش در جام
&&
بی‌باده شوی خراب و مدهوش
\\
خواهی که بیابی این چنین کام
&&
در ترک مراد خویشتن کوش
\\
چون ترک مراد خویش گیری
&&
گیری همه آرزو در آغوش
\\
گر ساقی عشق‌از خم درد
&&
دردی دهدت، مخواه سر جوش
\\
تو کار بدو گذار و خوش باش
&&
گر زهر تو را دهد بکن نوش
\\
چون راست نمی‌شود، عراقی،
&&
این کار به گفت و گوی، خاموش!
\\
\end{longtable}
\end{center}
