\begin{center}
\section*{غزل شماره ۱۰۷: ای خوشا دل کاندر او از عشق تو جانی بود}
\label{sec:107}
\addcontentsline{toc}{section}{\nameref{sec:107}}
\begin{longtable}{l p{0.5cm} r}
ای خوشا دل کاندر او از عشق تو جانی بود
&&
شادمانی جانی که او را چون تو جانانی بود
\\
خرم آن خانه که باشد چون تو مهمانی در او
&&
مقبل آن کشور که او را چون تو سلطانی بود
\\
زنده چو نباشد دلی کز عشق تو بویی نیافت؟
&&
کی بمیرد عاشقی کو را چو تو جانی بود؟
\\
هر که رویت دید و دل را در سر زلفت نبست
&&
در حقیقت آدمی نبود که حیوانی بود
\\
در همه عمر ار برآرم بی غم تو یک نفس
&&
زان نفس بر جان من هر لحظه تاوانی بود
\\
آفتاب روی تو گر بر جهان تابد دمی
&&
در جهان هر ذره‌ای خورشید تابانی بود
\\
در همه عالم ندیدم جز جمال روی تو
&&
گر کسی دعوی کند کو دید، بهتانی بود
\\
گنج حسنی و نپندارم که گنجی در جهان
&&
و آنچنان گنجی عجب در کنج ویرانی بود
\\
آتش رخسار خوبت گر بسوزاند مرا
&&
اندر آن آتش مرا هر سو گلستانی بود
\\
روزی آخر از وصال تو به کام دل رسم
&&
این شب هجر تو را گر هیچ پایانی بود
\\
عاشقان را جز سر زلف تو دست‌آویز نیست
&&
چه خلاص آن را که دست‌آویز ثعبانی بود؟
\\
چون عراقی در غزل یاد لب تو می‌کند
&&
هر نفس کز جان برآرد شکر افشانی بود
\\
\end{longtable}
\end{center}
