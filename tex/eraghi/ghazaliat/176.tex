\begin{center}
\section*{غزل شماره ۱۷۶: هر زمان جوری ز خوبان می‌کشم}
\label{sec:176}
\addcontentsline{toc}{section}{\nameref{sec:176}}
\begin{longtable}{l p{0.5cm} r}
هر زمان جوری ز خوبان می‌کشم
&&
هر نفس دردی ز دوران می‌کشم
\\
خون دل هر دم دگرگون می‌خورم
&&
جام غم هر شب دگرسان می‌کشم
\\
باز دست غم گریبانم گرفت
&&
گرچه بر افلاک دامان می‌کشم
\\
جور دلدار و جفای روزگار
&&
گرچه دشوار است، آسان می‌کشم
\\
از پی عشق پری رخساره‌ای
&&
زحمتی هر دم ز دیوان می‌کشم
\\
جور بین، کز دست دوران دم به دم
&&
ساغر پر زهر هجران می‌کشم
\\
چون ننالم از جفای ناکسان؟
&&
کین همه بیداد ازیشان می‌کشم
\\
تا نباید دیدنم روی رقیب
&&
هر نفس سر در گریبان می‌کشم
\\
با خیال دوست همدم می‌شوم
&&
وز لب او آب حیوان می‌کشم
\\
تن چو سوزن کرده‌ام، تا روز و شب
&&
مهر او در رشتهٔ جان می‌کشم
\\
نازنینا، ناز کن بر جان من
&&
ناز تو چندان که بتوان می‌کشم
\\
از تو چیزی دیده‌ام ناگفتنی
&&
وین همه محنت پی آن می‌کشم
\\
\end{longtable}
\end{center}
