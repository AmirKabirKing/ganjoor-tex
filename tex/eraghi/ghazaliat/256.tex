\begin{center}
\section*{غزل شماره ۲۵۶: ای دل، بنشین چو سوکواری}
\label{sec:256}
\addcontentsline{toc}{section}{\nameref{sec:256}}
\begin{longtable}{l p{0.5cm} r}
ای دل، بنشین چو سوکواری
&&
کان رفت که آید از تو کاری
\\
وی دیده ببار اشک خونین
&&
بی کار چه مانده‌ای تو، باری؟
\\
وی جان، بشتاب بر در دوست
&&
چون نیست جز اوت هیچ یاری
\\
گو: آمده‌ام به درگه تو
&&
تا در نگری به دوستداری
\\
گر بپذیرم: اینت دولت
&&
ور رد کنی، اینت خاکساری
\\
نومید چگونه باز گردد
&&
از درگه تو امیدواری؟
\\
یاد آر ز من، که بودم آخر
&&
در بندگی تو روزگاری
\\
چون از تو جدا فکندم ایام
&&
ناکام شدم به هر دیاری
\\
بی‌روی تو هر گلی که دیدم
&&
در دیدهٔ من خلید خاری
\\
بی‌بوی خوشت نیایدم خوش
&&
بوی خوش هیچ نوبهاری
\\
بی دوست، که را خوش آید آخر
&&
بوی گل و رنگ لاله زاری؟
\\
و اکنون که ز جمله ناامیدم
&&
بی روی تو نیستم قراری
\\
دریاب، که مانده‌ام به ره در
&&
در گردن من فتاده باری
\\
بشتاب، که بر درت گدایی است
&&
مانا که عراقی است، آری
\\
\end{longtable}
\end{center}
