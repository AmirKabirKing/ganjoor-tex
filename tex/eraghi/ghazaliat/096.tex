\begin{center}
\section*{غزل شماره ۹۶: در من نگرد یار دگربار که داند}
\label{sec:096}
\addcontentsline{toc}{section}{\nameref{sec:096}}
\begin{longtable}{l p{0.5cm} r}
در من نگرد یار دگربار که داند
&&
زین پس دهدم بر در خود بار که داند؟
\\
از یاد خودم کرد فراموش به یکبار
&&
یادآورد از من دگر آن یار که داند؟
\\
خون شد جگرم از غم و اندیشهٔ آن دوست
&&
خشنود شود از من غمخوار که داند؟
\\
بیمار دلم، خسته جگر از غم عشقش
&&
آید به عیادت بر بیمار که داند؟
\\
ای دشمن بدخواه، چه باشی به غمم شاد؟
&&
باشد که شود دوست دگربار که داند؟
\\
در بند امید، ای دل، بگشای دو دیده
&&
باشد که ببینی رخ دلدار که داند؟
\\
روشن شود این تیره شب بخت عراقی
&&
از صبح رخ یار وفادار که داند؟
\\
\end{longtable}
\end{center}
