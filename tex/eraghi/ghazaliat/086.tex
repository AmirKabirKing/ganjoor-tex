\begin{center}
\section*{غزل شماره ۸۶: من مست می عشقم هشیار نخواهم شد}
\label{sec:086}
\addcontentsline{toc}{section}{\nameref{sec:086}}
\begin{longtable}{l p{0.5cm} r}
من مست می عشقم هشیار نخواهم شد
&&
وز خواب خوش مستی بیدار نخواهم شد
\\
امروز چنان مستم از بادهٔ دوشینه
&&
تا روز قیامت هم هشیار نخواهم شد
\\
تا هست ز نیک و بد در کیسهٔ من نقدی
&&
در کوی جوانمردان عیار نخواهم شد
\\
آن رفت که می‌رفتم در صومعه هر باری
&&
جز بر در میخانه این بار نخواهم شد
\\
از توبه و قرایی بیزار شدم، لیکن
&&
از رندی و قلاشی بیزار نخواهم شد
\\
از دوست به هر خشمی آزرده نخواهم گشت
&&
وز یار به هر زخمی افگار نخواهم شد
\\
چون یار من او باشد، بی‌یار نخواهم ماند
&&
چون غم خورم او باشد غم‌خوار نخواهم شد
\\
تا دلبرم او باشد دل بر دگری ننهم
&&
تا غم خورم او باشد غمخوار نخواهم شد
\\
چون ساختهٔ دردم در حلقه نیارامم
&&
چون سوختهٔ عشقم در نار نخواهم شد
\\
تا هست عراقی را در درگه او باری
&&
بر درگه این و آن بسیار نخواهم شد
\\
\end{longtable}
\end{center}
