\begin{center}
\section*{غزل شماره ۹۸: نخستین باده کاندر جام کردند}
\label{sec:098}
\addcontentsline{toc}{section}{\nameref{sec:098}}
\begin{longtable}{l p{0.5cm} r}
نخستین باده کاندر جام کردند
&&
ز چشم مست ساقی وام کردند
\\
چو باخود یافتند اهل طرب را
&&
شراب بیخودی در جام کردند
\\
لب میگون جانان جام در داد
&&
شراب عاشقانش نام کردند
\\
ز بهر صید دل‌های جهانی
&&
کمند زلف خوبان دام کردند
\\
به گیتی هر کجا درد دلی بود
&&
به هم کردند و عشقش نام کردند
\\
سر زلف بتان آرام نگرفت
&&
ز بس دل‌ها که بی‌آرام کردند
\\
چو گوی حسن در میدان فکندند
&&
به یک جولان دو عالم رام کردند
\\
ز بهر نقل مستان از لب و چشم
&&
مهیا پسته و بادام کردند
\\
از آن لب، کز در صد آفرین است
&&
نصیب بی‌دلان دشنام کردند
\\
به مجلس نیک و بد را جای دادند
&&
به جامی کار خاص و عام کردند
\\
به غمزه صد سخن با جان بگفتند
&&
به دل ز ابرو دو صد پیغام کردند
\\
جمال خویشتن را جلوه دادند
&&
به یک جلوه دو عالم رام کردند
\\
دلی را تا به دست آرند، هر دم
&&
سر زلفین خود را دام کردند
\\
نهان با محرمی رازی بگفتند
&&
جهانی را از آن اعلام کردند
\\
چو خود کردند راز خویشتن فاش
&&
عراقی را چرا بدنام کردند؟
\\
\end{longtable}
\end{center}
