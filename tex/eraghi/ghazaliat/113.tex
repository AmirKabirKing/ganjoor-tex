\begin{center}
\section*{غزل شماره ۱۱۳: آخر این تیره شب هجر به پایان آید}
\label{sec:113}
\addcontentsline{toc}{section}{\nameref{sec:113}}
\begin{longtable}{l p{0.5cm} r}
آخر این تیره شب هجر به پایان آید
&&
آخر این درد مرا نوبت درمان آید
\\
چند گردم چو فلک گرد جهان سرگردان؟
&&
آخر این گردش ما نیز به پایان آید
\\
آخر این بخت من از خواب درآید سحری
&&
روز آخر نظرم بر رخ جانان آید
\\
یافتم صحبت آن یار، مگر روزی چند
&&
این همه سنگ محن بر سر ما زان آید
\\
تا بود گوی دلم در خم چوگان هوس
&&
کی مرا گوی غرض در خم چوگان آید؟
\\
یوسف گم شده را گرچه نیابم به جهان
&&
لاجرم سینهٔ من کلبهٔ احزان آید
\\
بلبل‌آسا همه شب تا به سحر ناله زنم
&&
بو که بویی به مشامم ز گلستان آید
\\
او چه خواهد؟ که همی با وطن آید، لیکن
&&
تا خود از درگه تقدیر چه فرمان آید
\\
به عراق ار نرسد باز عراقی چه عجب!
&&
که نه هر خار و خسی لایق بستان آید
\\
\end{longtable}
\end{center}
