\begin{center}
\section*{غزل شماره ۴۸: بر من، ای دل، بند جان نتوان نهاد}
\label{sec:048}
\addcontentsline{toc}{section}{\nameref{sec:048}}
\begin{longtable}{l p{0.5cm} r}
بر من، ای دل، بند جان نتوان نهاد
&&
شور در دیوانگان نتوان نهاد
\\
های و هویی در فلک نتوان فکند
&&
شر و شوری در جهان نتوان نهاد
\\
چون پریشانی سر زلفت کند
&&
سلسله بر پای جان نتوان نهاد
\\
چون خرابی چشم مستت می‌کند
&&
جرم بر دور زمان نتوان نهاد
\\
عشق تو مهمان و ما را هیچ نه
&&
هیچ پیش میهمان نتوان نهاد
\\
نیم جانی پیش او نتوان کشید
&&
پیش سیمرغ استخوان نتوان نهاد
\\
گرچه گه‌گه وعدهٔ وصلم دهد
&&
غمزهٔ تو، دل بر آن نتوان نهاد
\\
گویمت: بوسی به جانی، گوییم:
&&
بر لبم لب رایگان نتوان نهاد
\\
بر سر خوان لبت، خود بی‌جگر
&&
لقمه‌ای خوش در دهان نتوان نهاد
\\
بر دلم بار غمت چندین منه
&&
برکهی کوه گران نتوان نهاد
\\
شب در دل می‌زدم، مهر تو گفت:
&&
زود پابر آسمان نتوان نهاد
\\
تا تو را در دل هوای جان بود
&&
پای بر آب روان نتوان نهاد
\\
تات وجهی روشن است، این هفت‌خوان
&&
پیش تو بس، هشت خوان نتوان نهاد
\\
ور عراقی محرم این حرف نیست
&&
راز با او در میان نتوان نهاد
\\
\end{longtable}
\end{center}
