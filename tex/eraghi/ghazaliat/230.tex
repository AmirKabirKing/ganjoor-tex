\begin{center}
\section*{غزل شماره ۲۳۰: ترک من، ای من غلام روی تو}
\label{sec:230}
\addcontentsline{toc}{section}{\nameref{sec:230}}
\begin{longtable}{l p{0.5cm} r}
ترک من، ای من غلام روی تو
&&
جمله ترکان جهان هندوی تو
\\
لعل تو شیرین‌تر از آب حیات
&&
زان بگو خوشتر چه باشد؟ روی تو
\\
خرم آن عاشق، که بیند آشکار
&&
بامدادان طلعت نیکوی تو
\\
فرخ آن بی‌دل، که یابد هر سحر
&&
از گل گلزار عالم بوی تو
\\
حیف نبود ما چنین تشنه جگر؟
&&
و آب حیوان رایگان در جوی تو
\\
دل گرفتار کمند زلف تو
&&
جان شکار غمزهٔ جادوی تو
\\
غمزهٔ خونخوار تو کرد آنچه کرد
&&
تا چه خواهد کرد با ما خوی تو؟
\\
من چو سر در پای تو انداختم
&&
بر سر آیم عاقبت چون موی تو
\\
چون دل من در سر زلف تو شد
&&
هم شود گه گاه همزانوی تو
\\
هم ببیند جان جمال تو عیان
&&
چون نهان شد در خم گیسوی تو
\\
هم زمان جایی دگر سازی مقام
&&
تا نیابد کس نشان و بوی تو
\\
هر نفس جایی دگر پی گم کنی
&&
تا عراقی ره نیابد سوی تو
\\
\end{longtable}
\end{center}
