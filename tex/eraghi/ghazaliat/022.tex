\begin{center}
\section*{غزل شماره ۲۲: چنین که حال من زار در خرابات است}
\label{sec:022}
\addcontentsline{toc}{section}{\nameref{sec:022}}
\begin{longtable}{l p{0.5cm} r}
چنین که حال من زار در خرابات است
&&
می مغانه مرا بهتر از مناجات است
\\
مرا چو می‌نرهاند ز دست خویشتنم
&&
به میکده شدنم بهترین طاعات است
\\
درون کعبه عبادت چه سود؟ چون دل من
&&
میان میکده مولای عزی و لات است
\\
مرا که بتکده و مصطبه مقام بود
&&
چه جای صومعه و زهد و وجد و حالات است؟
\\
مرا که قبله خم ابروی بتان باشد
&&
چه جای مسجد و محراب و زهد و طاعات است
\\
ملامتم مکنید، ار به دیر درد کشم
&&
که حال بی‌خبران بهترین حالات است
\\
ز ذوق با خبری آنکه را خبر باشد
&&
به نزد او سخن ناقصان خرافات است
\\
خراب کوی خرابات را از آن چه خبر
&&
که اهل صومعه را بهترین مقامات است
\\
اگر چه اهل خرابات را ز من ننگی است
&&
مرا نصیحت ایشان بسی مباهات است
\\
کسی که حالت دیوانگان میکده یافت
&&
مقام اهل خرد نزدش از خرافات است
\\
گلیم بخت کسی را که بافتند سیاه
&&
سفید کردن آن نوعی از محالات است
\\
کجاست می؟ که به جان آمدم ز خسته دلی
&&
که پر ز شیوه و سالوس و زرق و طامات است
\\
مقام دردکشانی که در خراباتند
&&
یقین بدان که ورای همه مقامات است
\\
کنون مقام عراقی مجوی در مسجد
&&
که او حریف بتان است و در خرابات است
\\
\end{longtable}
\end{center}
