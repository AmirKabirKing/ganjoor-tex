\begin{center}
\section*{غزل شماره ۹: دیدی چو من خرابی افتاده در خرابات}
\label{sec:009}
\addcontentsline{toc}{section}{\nameref{sec:009}}
\begin{longtable}{l p{0.5cm} r}
دیدی چو من خرابی افتاده در خرابات
&&
فارغ شده ز مسجد وز لذت مباحات
\\
از خانقاه رفته، در میکده نشسته
&&
صد سجده کرده هر دم در پیش عزی ولات
\\
در باخته دل و دین، مفلس بمانده مسکین
&&
افتاده خوار و غمگین در گوشهٔ خرابات
\\
نی همدمی که با او یک دم دمی برآرد
&&
نی محرمی که یابد با وی دمی مراعات
\\
نی هیچ گبری او را دستی گرفت روزی
&&
نی کرده پایمردی با او دمی مدارات
\\
دردش ندید درمان، زخمش نجست مرهم
&&
در ساخته به ناکام با درد بی‌مداوات
\\
خوش بود روزگاری بر بوی وصل یاری
&&
هم خوشدلیش رفته، هم روزگار، هیهات!
\\
با این همه، عراقی، امیدوار می‌باش
&&
باشد که به شود حال، گردنده است حالات
\\
\end{longtable}
\end{center}
