\begin{center}
\section*{غزل شماره ۶۳: خرم تن آن کس که دل ریش ندارد}
\label{sec:063}
\addcontentsline{toc}{section}{\nameref{sec:063}}
\begin{longtable}{l p{0.5cm} r}
خرم تن آن کس که دل ریش ندارد
&&
و اندیشهٔ یار ستم‌اندیش ندارد
\\
گویند رقیبان که ندارد سر تو یار
&&
سلطان چه عجب گر سر درویش ندارد؟
\\
او را چه خبر از من و از حال دل من
&&
کو دیدهٔ پر خون و دل ریش ندارد
\\
این طرفه که او من شد و من او وز من یار
&&
بیگانه چنان شد که سر خویش ندارد
\\
هان، ای دل خونخوار، سر محنت خود گیر
&&
کان یار سر صحبت ما بیش ندارد
\\
معشوق چو شمشیر جفا بر کشد، از خشم
&&
عاشق چه کند گر سر خود پیش ندارد؟
\\
بیچاره دل ریش عراقی که همیشه
&&
از نوش لبان، بهره به جز نیش ندارد
\\
\end{longtable}
\end{center}
