\begin{center}
\section*{غزل شماره ۱۲۷: آب حیوان است، آن لب، یا شکر}
\label{sec:127}
\addcontentsline{toc}{section}{\nameref{sec:127}}
\begin{longtable}{l p{0.5cm} r}
آب حیوان است، آن لب، یا شکر؟
&&
یا سرشته آب حیوان با شکر؟
\\
نی خطا گفتم: کجا لذت دهد
&&
آب حیوان پیش آن لب یا شکر؟
\\
کس نگوید نوش جان‌ها را نبات
&&
کس نخواند جان شیرین را شکر
\\
لعل تو شکر توان گفت، ار بود
&&
کوثر و تسنیم جان افزا شکر
\\
قوت جان است و حیات جاودان
&&
نیست یار لعل تو تنها شکر
\\
ای به رشک از لعل تو آب حیات
&&
وی خجل زان لعل شکرخا شکر
\\
وامق ار دیدی لب شیرین تو
&&
خود نجستی از لب عذرا شکر
\\
نام تو تا بر زبان ما گذشت
&&
می‌گدازد در دهان ما شکر
\\
از لب و دندان تو در حیرتم
&&
تا گهر چون می‌کند پیدا شکر؟
\\
تا دهانت شکرستان گشت و لب
&&
در جهان تنگ است چون دلها شکر
\\
من چرا سودایی لعلت شدم
&&
از مزاج ار می‌برد سودا شکر؟
\\
گرد لعل تو همی گردد نبات
&&
نی، طمع دارد از آن لبها شکر
\\
گرد بر گرد لب شیرین تو
&&
طوطیان بین جمله سر تا پا شکر
\\
لعل و گفتار تو با هم در خور است
&&
باشد آری نایب حلوا شکر
\\
طبع من شیرین شد از یاد لبت
&&
ای عجب، چون می‌شود دریا شکر؟
\\
لفظ شیرین عراقی چون لبت
&&
می‌فشاند در سخن هر جا شکر
\\
\end{longtable}
\end{center}
