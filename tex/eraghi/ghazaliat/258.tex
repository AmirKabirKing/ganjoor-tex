\begin{center}
\section*{غزل شماره ۲۵۸: نگارا، کی بود کامیدواری}
\label{sec:258}
\addcontentsline{toc}{section}{\nameref{sec:258}}
\begin{longtable}{l p{0.5cm} r}
نگارا، کی بود کامیدواری
&&
بیابد بر در وصل تا باری؟
\\
چه خوش باشد که بعد از ناامیدی
&&
به کام دل رسد امیدواری؟
\\
بده کام دلم، مگذار، جانا
&&
که دشمن کام گردد دوستداری
\\
دلی دارم گرفتار غم تو
&&
ندارد جز غم تو غمگساری
\\
چنان خو کرد با دل غم، که گویی
&&
بجز غم خوردن او را نیست کاری
\\
بیا، ای یار و دل را یاریی کن
&&
که بیچاره ندارد جز تو یاری
\\
به غم شادم ازان، کاندر فراقت
&&
ندارم از تو جز غم یادگاری
\\
چه خوش باشد که جان من برآید
&&
ز محنت وارهم یک باره، باری!
\\
عراقی را ز غم جان بر لب آمد
&&
چه می‌خواهد غمت از دل فگاری؟
\\
\end{longtable}
\end{center}
