\begin{center}
\section*{غزل شماره ۱۵۵: اکئوس تلالات بمدام}
\label{sec:155}
\addcontentsline{toc}{section}{\nameref{sec:155}}
\begin{longtable}{l p{0.5cm} r}
اکئوس تلالات بمدام
&&
ام شموس تهللت بغمام
\\
از صفای می و لطافت جام
&&
در هم آمیخت رنگ جام، مدام
\\
همه‌جا مست و نیست گویی می
&&
یا مدام است و نیست گویی جام
\\
چون هوا رنگ آفتاب گرفت
&&
رخت برگیرد از میانه ظلام
\\
چون شب و روز در هم آمیزند
&&
رنگ و بوی سحر دهند به شام
\\
جام را رنگ و بوی می دادند
&&
تا ز ساقی و می دهد اعلام
\\
رنگ جام ارچه گشت گوناگون
&&
از چه افتاد بر وی این همه نام؟
\\
از دو رنگی ماست این همه رنگ
&&
ورنه یک رنگ بیش نیست مدام
\\
مجلس آراستند صبح دمی
&&
تا صبوحی کنند خاصه و عام
\\
خاص را باده خاصگی دادند
&&
عام را دردیی به رسم عوام
\\
عامه از بوی باده مست شدند
&&
خاص خود مست ساقیند مدام
\\
مست ساقی به رنگ و بو چه کند؟
&&
حاضران را چه کار با پیغام؟
\\
باده‌نوشان، که کار آب کنند،
&&
خاک را تیزتر کنند مسام
\\
جرعه‌ای کان ز خاک نیست دریغ
&&
بر چو من خاکیی چراست حرام؟
\\
ساقی، ار صاف نیست، دردی ده
&&
باش، گو، هر چه هست، پخته و خام
\\
چه شود گر کنی درین مجلس
&&
ناقصی را به نیم جرعه تمام ؟
\\
در دو عالم نگنجم از شادی
&&
گر مرا بوی تو رسد به مشام
\\
سر این جام و باده کشف کنم
&&
نزند تا غلط ره اوهام
\\
باز گویم که: این چه رنگ و چه بوست
&&
می کدام است و جام باده کدام؟
\\
بوی وجد است و رنگ نور صفات
&&
می تجلی ذات و جام کلام
\\
\end{longtable}
\end{center}
