\begin{center}
\section*{غزل شماره ۸۷: گر نظر کردم به روی ماه رخساری چه شد}
\label{sec:087}
\addcontentsline{toc}{section}{\nameref{sec:087}}
\begin{longtable}{l p{0.5cm} r}
گر نظر کردم به روی ماه رخساری چه شد؟
&&
ور شدم مست از شراب عشق یکباری چه شد؟
\\
روی او دیدم سر زلفش چرا آشفته گشت ؟
&&
گر نبیند بلبل شوریده، گلزاری چه شد؟
\\
چشم او با جان من گر گفته رازی، گو، بگوی
&&
حال بیماری اگر پرسید بیماری چه شد؟
\\
دشمنم با دوستان گوید: فلانی عاشق است
&&
عاشقم بر روی خوبان، عاشقم، آری چه شد؟
\\
در سر سودای عشق خوبرویان شد دلم
&&
وز چنان زلف ار ببستم نیز زناری چه شد؟
\\
گر گذشتم بر در میخانه ناگاهی چه باک؟
&&
گر به پیران سر شکستم توبه یکباری چه شد؟
\\
چون شدم مست از شراب عشق، عقلم گو: برو
&&
گر فرو شست آب حیوان نقش دیواری چه شد؟
\\
گر میان عاشق و معشوق جرمی رفت رفت
&&
تو نه معشوقی نه عاشق، مر تو را باری چه شد؟
\\
زاهدی را کز می و معشوق رنگی نیست نیست
&&
گر کند بر عاشقان هر لحظه انکاری چه شد؟
\\
های و هوی عاشقان شد از زمین بر آسمان
&&
نعرهٔ مستان اگر نشنید هشیاری چه شد؟
\\
از خمستان نعرهٔ مستان به گوش من رسید
&&
رفتم آنجا تا ببینم حال میخواری چه شد؟
\\
دیدم اندر کنج میخانه عراقی را خراب
&&
گفتم: ای مسکین، نگویی تا تو را باری چه شد؟
\\
\end{longtable}
\end{center}
