\begin{center}
\section*{غزل شماره ۱۰۴: به خرابات شدم دوش مرا بار نبود}
\label{sec:104}
\addcontentsline{toc}{section}{\nameref{sec:104}}
\begin{longtable}{l p{0.5cm} r}
به خرابات شدم دوش مرا بار نبود
&&
می‌زدم نعره و فریاد ز من کس نشنود
\\
یا نبد هیچ کس از باده‌فروشان بیدار
&&
یا خود از هیچ کسی هیچ کسم در نگشود
\\
چون که یک نیم ز شب یا کم یا بیش برفت
&&
رندی از غرفه برون کرد سر و رخ بنمود
\\
گفت: خیر است، درین وقت تو دیوانه شدی
&&
نغز پرداختی آخر تو نگویی که چه بود؟
\\
گفتمش: در بگشا، گفت: برو، هرزه مگوی
&&
تا درین وقت ز بهر چو تویی در که گشود؟
\\
این نه مسجد که به هر لحظه درش، بگشایند
&&
تا تو اندر دوی، اندر صف پیش آیی زود
\\
این خرابات مغان است و درو زنده‌دلان
&&
شاهد و شمع و شراب و غزل و رود و سرود
\\
زر و سر را نبود هیچ درین بقعه محل
&&
سودشان جمله زیان است و زیانشان همه سود
\\
سر کوشان عرفات است و سراشان کعبه
&&
عاشقان همچو خلیلند و رقیبان نمرود
\\
ای عراقی، چه زنی حلقه برین در شب و روز؟
&&
زین همه آتش خود هیچ نبینی جز دود
\\
\end{longtable}
\end{center}
