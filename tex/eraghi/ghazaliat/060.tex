\begin{center}
\section*{غزل شماره ۶۰: با پرتو جمالت برهان چه کار دارد}
\label{sec:060}
\addcontentsline{toc}{section}{\nameref{sec:060}}
\begin{longtable}{l p{0.5cm} r}
با پرتو جمالت برهان چه کار دارد؟
&&
با عشق زلف و خالت ایمان چه کار دارد؟
\\
با عشق دلگشایت عاشق کجا برآید؟
&&
با وصل جانفزایت هجران چه کار دارد؟
\\
در بارگاه دردت درمان چه راه یابد؟
&&
با جلوه‌گاه وصلت هجران چه کار دارد؟
\\
با سوز بی‌دلانت مالک چه طاقت آرد؟
&&
با عیش عاشقانت رضوان چه کار دارد؟
\\
گرنه گریخت جانم از پرتو جمالت
&&
در سایهٔ دو زلفت پنهان چه کار دارد؟
\\
چون در پناه وصلت افتاد جان نگویی:
&&
هجری بدین درازی با جان چه کار دارد؟
\\
گر در خورت نیابم، شاید، که بر سماطت
&&
پوسیده استخوانی بر خوان چه کار دارد؟
\\
آری عجب نباشد گر در دلم نیابی
&&
در کلبهٔ گدایان سلطان چه کار دارد؟
\\
من نیز اگر نگنجم در حضرتت، عجب نیست
&&
آنجا که آن کمال است نقصان چه کار دارد؟
\\
در تنگنای وحدت کثرت چگونه گنجد
&&
در عالم حقیقت بطلان چه کار دارد؟
\\
گویند نیکوان را نظارگی نباید
&&
کانجا که درد نبود درمان چه کار دارد؟
\\
آری، ولی چو عاشق پوشید رنگ معشوق
&&
آن دم میان ایشان دربان چه کار دارد؟
\\
جایی که در میانه معشوق هم نگنجد
&&
مالک چه زحمت آرد؟ رضوان چه کار دارد ؟
\\
هان! خسته دل عراقی، با درد یار خو کن
&&
کانجا که دردش آمد درمان چه کار دارد؟
\\
\end{longtable}
\end{center}
