\begin{center}
\section*{غزل شماره ۶۶: راحت سر مردمی ندارد}
\label{sec:066}
\addcontentsline{toc}{section}{\nameref{sec:066}}
\begin{longtable}{l p{0.5cm} r}
راحت سر مردمی ندارد
&&
دولت دل همدمی ندارد
\\
ز احسان زمانه دیده بردوز
&&
کو دیدهٔ مردمی ندارد
\\
از خوان فلک نواله کم پیچ
&&
کو گردهٔ گندمی ندارد
\\
با درد بساز، از آنکه درمان
&&
با جان تو محرمی ندارد
\\
در تار حیات دل چه بندی؟
&&
چون پود تو محکمی ندارد
\\
دردا! که درین سرای پر غم
&&
کس دولت بی‌غمی ندارد
\\
دارد همه چیز آدمی زاد
&&
افسوس که خرمی ندارد
\\
گر خوشدلیی درین جهان هست
&&
باری دل آدمی ندارد
\\
بنمای به من دلی فراهم
&&
کو محنت درهمی ندارد
\\
کم خور غم این جهان، عراقی،
&&
زیرا که غمش کمی ندارد
\\
\end{longtable}
\end{center}
