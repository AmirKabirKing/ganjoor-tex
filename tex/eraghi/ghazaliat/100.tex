\begin{center}
\section*{غزل شماره ۱۰۰: اگر شکسته دلانت هزار جان دارند}
\label{sec:100}
\addcontentsline{toc}{section}{\nameref{sec:100}}
\begin{longtable}{l p{0.5cm} r}
اگر شکسته دلانت هزار جان دارند
&&
به خدمت تو کمر بسته بر میان دارند
\\
شدند حلقه به گوش تو را چو حلقه به گوش
&&
چه خوش دلند که مثل تو دلستان دارند
\\
کسان که وصل تو یک دم به نقد یافته‌اند
&&
از ین طلب طرب و عیش جاودان دارند
\\
چو بگذری به تعجب تو ماهروی به راه
&&
چو ماه ماهرخان دست بر دهان دارند
\\
خرد از آن ز ره زلف تو پناه گرفت
&&
که چشم و ابروی تو تیر در کمان دارند
\\
مجاهدان رهت تا عنایت تو بود
&&
چه بیم و باک به عالم ازین و آن دارند؟
\\
ز آب دیده و تاب دل است غمازی
&&
وگرنه راز تو بیچارگان نهان دارند
\\
غلام غمزهٔ بیمارتم که از هوسش
&&
چه تندرستان خود را ناتوان دارند؟
\\
اگر کسی به شکایت بود ز دلبر خویش
&&
ز تو عراقی و دل شکر بی‌کران دارند
\\
\end{longtable}
\end{center}
