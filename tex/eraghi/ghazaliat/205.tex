\begin{center}
\section*{غزل شماره ۲۰۵: تا کی همه مدح خویش گوییم}
\label{sec:205}
\addcontentsline{toc}{section}{\nameref{sec:205}}
\begin{longtable}{l p{0.5cm} r}
تا کی همه مدح خویش گوییم؟
&&
تا چند مراد خویش جوییم؟
\\
بر خیره قصیده چند خوانیم؟
&&
بیهوده فسانه چند گوییم؟
\\
ای دیده بیا، که خون بگرییم
&&
وی بخت، بیا، که خوش بموییم
\\
ما را چو به کام دشمنان کرد
&&
آن یار که دوستدار اوییم
\\
نگذاشت که با سگان کویش
&&
گرد سر کوی او بپوییم
\\
دانم که روا ندارد آن خود
&&
کز باغ رخش گلی ببوییم
\\
زین به نبود، کز آب دیده
&&
خیزیم و گلیم خود بشوییم؟
\\
گردی است به راه در، عراقی
&&
آن گرد ز راه خود بروبیم
\\
\end{longtable}
\end{center}
