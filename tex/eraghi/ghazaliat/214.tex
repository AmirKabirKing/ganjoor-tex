\begin{center}
\section*{غزل شماره ۲۱۴: نگار از سر کویت گذر کردن توان؟ نتوان}
\label{sec:214}
\addcontentsline{toc}{section}{\nameref{sec:214}}
\begin{longtable}{l p{0.5cm} r}
نگارا از سر کویت گذر کردن توان؟ نتوان
&&
به خوبی در همه عالم نظر کردن توان؟ نتوان
\\
چو آمد در دل و دیده خیالت آشنا بنشست
&&
ز ملک خویش سلطان را بدر کردن توان؟ نتوان
\\
مرا این دوستی با تو قضای آسمانی بود
&&
قضای آسمانی را دگر کردن توان؟ نتوان
\\
چو با ابروی تو چشمم به پنهانی سخن گوید
&&
از آن معنی رقیبان را خبر کردن توان؟ نتوان
\\
چو چشم مست خونریزت ز مژگان ناوک اندازد
&&
بجز جان پیش تیر تو سپر کردن توان؟ نتوان
\\
گرفتم خود که بگریزم ز دام زلف دلگیرت
&&
ز تیر غمزهٔ مستت حذر کردن توان؟ نتوان
\\
نگویی چشم مستت را، که خون من همی ریزد
&&
ز خون بی‌گناه او را حذر کردن توان؟ نتوان
\\
بگو با غمزهٔ شوخت، که رسوای جهانم کرد:
&&
به پیران سر عراقی را سمر کردن توان؟ نتوان
\\
\end{longtable}
\end{center}
