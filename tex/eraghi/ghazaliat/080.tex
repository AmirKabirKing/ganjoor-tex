\begin{center}
\section*{غزل شماره ۸۰: بتم از غمزه و ابرو، همه تیر و کمان سازد}
\label{sec:080}
\addcontentsline{toc}{section}{\nameref{sec:080}}
\begin{longtable}{l p{0.5cm} r}
بتم از غمزه و ابرو، همه تیر و کمان سازد
&&
به غمزه خون دل ریزد به ابرو کار جان سازد
\\
چو در دام سر زلفش همه عالم گرفتار است
&&
چرا مژگان کند ناوک چرا ابرو کمان سازد؟
\\
خرابی ها کند چشمش که نتوان کرد در عالم
&&
چه شاید گفت با مستی که خود را ناتوان سازد؟
\\
دل و جان همه عالم فدای لعل نوشینش
&&
که چون جام طرب نوشد دو عالم جرعه‌دان سازد
\\
غلام آن نگارینم که از رخ مجلس افروزد
&&
لب او از شکر خنده شراب عاشقان سازد
\\
بتی کز حسن در عالم نمی‌گنجد عجب دارم
&&
که دایم در دل تنگم چگونه خان و مان سازد؟
\\
عراقی، بگذر از غوغا، دلی فارغ به دست آور
&&
که سیمرغ وصال او در آنجا آشیان سازد
\\
\end{longtable}
\end{center}
