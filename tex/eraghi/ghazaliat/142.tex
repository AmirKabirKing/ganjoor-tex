\begin{center}
\section*{غزل شماره ۱۴۲: صلای عشق، که ساقی ز لعل خندانش}
\label{sec:142}
\addcontentsline{toc}{section}{\nameref{sec:142}}
\begin{longtable}{l p{0.5cm} r}
صلای عشق، که ساقی ز لعل خندانش
&&
شراب و نقل فرو ریخته به مستانش
\\
بیا، که بزم طرب ساخت، خوان عشق نهاد
&&
برای ما لب نوشین شکر افشانش
\\
تبسم لب ساقی خوش است و خوشتر از آن
&&
خرابیی که کند باز چشم فتانش
\\
به یک کرشمه چنان مست کرد جان مرا
&&
که در بهشت نیارد به هوش رضوانش
\\
خوشا شراب و خوشا ساقی و خوشا بزمی
&&
که غمزهٔ خوش ساقی بود خمستانش!
\\
ازین شراب که یک قطره بیش نیست که تو
&&
گهی حیات جهان خوانی و گهی جانش
\\
ز عکس ساغر آن پرتوی است این که تو باز
&&
همیشه نام نهی آفتاب تابانش
\\
ازین شراب اگر خضر یافتی قدحی
&&
خود التفات نبودی به آب حیوانش
\\
نگشت مست به جز غمزهٔ خوش ساقی
&&
ازان شراب که در داد لعل خندانش
\\
نبود نیز به جز عکس روی او در جام
&&
نظارگی، که بود همنشین و همخوانش
\\
نظارگی به من و هم به من هویدا شد
&&
کمال او، که به من ظاهر است برهانش
\\
عجب مدار که: چشمش به من نگاه کند
&&
برای آنکه منم در وجود انسانش
\\
نگاه کرد به من، دید صورت خود را
&&
شد آشکار ز آیینه راز پنهانش
\\
عجب، چرا به عراقی سپرد امانت را؟
&&
نبود در همه عالم کسی نگهبانش
\\
مگر که راز جهان خواست آشکارا کرد
&&
بدو سپرد امانت، که دید تاوانش
\\
\end{longtable}
\end{center}
