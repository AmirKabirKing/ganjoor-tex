\begin{center}
\section*{غزل شماره ۳۷: ساقی، ار جام می، دمادم نیست}
\label{sec:037}
\addcontentsline{toc}{section}{\nameref{sec:037}}
\begin{longtable}{l p{0.5cm} r}
ساقی، ار جام می، دمادم نیست
&&
جان فدای تو، دردیی کم نیست
\\
من که در میکده کم از خاکم
&&
جرعه‌ای هم مرا مسلم نیست
\\
جرعه‌ای ده، مرا ز غم برهان
&&
که دلم بی‌شراب خرم نیست
\\
از خودی خودم خلاصی ده
&&
کز خودم زخم هست مرهم نیست
\\
چون حجاب من است هستی من
&&
گر نباشد، مباش، گو: غم نیست
\\
ز آرزوی دمی دلم خون شد
&&
که شوم یک نفس درین دم نیست
\\
بهر دل درهم و پریشانم
&&
چه کنم؟ کار دل فراهم نیست
\\
خوشدلی در جهان نمی‌یابم
&&
خود خوشی در نهاد عالم نیست
\\
در جهان گر خوشی کم است مرا
&&
خوش از آنم که ناخوشی هم نیست
\\
کشت امید را، که خشک بماند
&&
بهتر از آب چشم من نم نیست
\\
ساقیا، یک دمم حریفی کن
&&
کین دمم جز تو هیچ همدم نیست
\\
ساغری ده، مرا ز من برهان
&&
که عراقی حریف و محرم نیست
\\
\end{longtable}
\end{center}
