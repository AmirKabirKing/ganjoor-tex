\begin{center}
\section*{غزل شماره ۲۲۴: چو دل ز دایرهٔ عقل بی تو شد بیرون}
\label{sec:224}
\addcontentsline{toc}{section}{\nameref{sec:224}}
\begin{longtable}{l p{0.5cm} r}
چو دل ز دایرهٔ عقل بی تو شد بیرون
&&
مپرس از دلم آخر که: چون شد آن مجنون؟
\\
دلم، که از سر سودا به هر دری می‌شد
&&
چو حلقه بین که بمانده است بر در تو کنون
\\
کسی که خاک درت دوست‌تر ز جان دارد
&&
چگونه جای دگر باشدش قرار و سکون؟
\\
دلم، که حلقه به گوش در تو شد مفروش
&&
که هیچ قدر ندارد بهای قطرهٔ خون
\\
چو رایگان است آب حیات در جویت
&&
چرا بود دل مسکین چو ریگ در جیحون؟
\\
دل عراقی اگر چه هزار گونه بگشت
&&
ولی ز مهر تو هرگز نگشت دیگر گون
\\
\end{longtable}
\end{center}
