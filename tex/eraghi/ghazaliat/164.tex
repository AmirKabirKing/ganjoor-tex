\begin{center}
\section*{غزل شماره ۱۶۴: کجایی؟ای ز جان خوشتر ، شبت خوش باد ، من رفتم}
\label{sec:164}
\addcontentsline{toc}{section}{\nameref{sec:164}}
\begin{longtable}{l p{0.5cm} r}
کجایی؟ای ز جان خوشتر ، شبت خوش باد ، من رفتم
&&
بیا در من خوشی بنگر، شبت خوش باد من رفتم
\\
نگارا، بر سر کویت دلم را هیچ اگر بینی
&&
ز من دلخسته یاد آور، شبت خوش باد من رفتم
\\
ز من چون مهر بگسستی، خوشی در خانه بنشستی
&&
مرا بگذاشتی بر در، شبت خوش باد من رفتم
\\
تو با عیش و طرب خوش باش، من با ناله و زاری
&&
مرا کان نیست این بهتر، شبت خوش باد من رفتم
\\
مرا چون روزگار بد ز وصل تو جدا افکند
&&
بماندم عاجز و مضطر، شبت خوش باد من رفتم
\\
بماندم واله و حیران میان خاک و خون غلتان
&&
دو لب خشک و دو دیده تر ، شبت خوش باد من رفتم
\\
منم امروز بیچاره، ز خان و مانم آواره
&&
نه دل در دست و نه دلبر، شبت خوش باد من رفتم
\\
مرا گویی که: ای عاشق، نه ای وصل مرا لایق
&&
تو را چون نیستم در خور، شبت خوش باد من رفتم
\\
همی گفتم که: ناگاهی، بمیرم در غم عشقت
&&
نکردی گفت من باور، شبت خوش باد من رفتم
\\
عراقی می‌سپارد جان و می‌گوید ز درد دل:
&&
کجایی؟ای ز جان خوشتر ، شبت خوش باد من رفتم
\\
\end{longtable}
\end{center}
