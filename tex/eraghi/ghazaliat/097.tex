\begin{center}
\section*{غزل شماره ۹۷: ای دل، چو در خانهٔ خمار گشادند}
\label{sec:097}
\addcontentsline{toc}{section}{\nameref{sec:097}}
\begin{longtable}{l p{0.5cm} r}
ای دل، چو در خانهٔ خمار گشادند
&&
می‌نوش، که از می گره کار گشادند
\\
در خود منگر، نرگس مخمور بتان بین
&&
در کعبه مرو، چون در خمار گشادند
\\
از خود بدرآ، در رخ خوبان نظری کن
&&
در خان منشین چون در گلزار گشادند
\\
بنگر که: دو صد مهر به یک ذره نمودند
&&
از یک سر مویی که ز رخسار گشادند
\\
تا باز گشادند سر زلف ز رخسار
&&
از روی جهان زلف شب تار گشادند
\\
تا مهر گیاهی ز گل تیره برآید
&&
بر روی زمین چشمهٔ انوار گشادند
\\
تا لاله رخی در چمن آید به تماشا
&&
از چهرهٔ گل پردهٔ زنگار گشادند
\\
از پرتو مل پردهٔ خورشید دریدند
&&
وز خندهٔ گل مبسم اشجار گشادند
\\
تا کرد نسیم سحر آفاق معطر
&&
در هر چمنی طبلهٔ عطار گشادند
\\
مانا که صبا کرد پریشان سر زلفین
&&
کز بوی خوشش نافهٔ تاتار گشادند
\\
در گوش دلم گفت صبا دوش: عراقی
&&
در بند در خود، که در یار گشادند
\\
چشم سر اغیار ببستند ز غیرت
&&
آنگاه در مخزن اسرار گشادند
\\
\end{longtable}
\end{center}
