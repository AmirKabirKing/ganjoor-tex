\begin{center}
\section*{غزل شماره ۲۳۹: در صومعه نگنجد رند شرابخانه}
\label{sec:239}
\addcontentsline{toc}{section}{\nameref{sec:239}}
\begin{longtable}{l p{0.5cm} r}
در صومعه نگنجد رند شرابخانه
&&
ساقی، بده مغی را، درد می مغانه
\\
ره ده قلندری را، در بزم دردنوشان
&&
بنما مقامری را، راه قمارخانه
\\
تا بشکند چو توبه، هر بت که می‌پرستید
&&
تا جان نهد چو جرعه، شکرانه در میانه
\\
بیرون شود، چو عنقا، از خانه سوی صحرا
&&
پرواز گیرد از خود، بگذارد آشیانه
\\
فارغ شود ز هستی وز خویشتن پرستی
&&
بر هم زند ز مستی نیک و بد زمانه
\\
در خلوتی چنین خوش چه خوش بود صبوحی!
&&
با محرمی موافق، با همدمی یگانه
\\
آورده روی در روی با شاهدی شکر لب
&&
در کف می صبوحی، در سر می شبانه
\\
ساقی شراب داده هر لحظه از دگر جام
&&
مطرب سرود گفته هر دم دگر ترانه
\\
باده حدیث جانان، باقی همه حکایت
&&
نغمه خروش مستان دیگر همه فسانه
\\
نظاره روی ساقی، نظارگی عراقی
&&
خم خانه عشق باقی، باقی همه بهانه
\\
\end{longtable}
\end{center}
