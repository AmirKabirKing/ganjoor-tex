\begin{center}
\section*{غزل شماره ۲۹۵: ز دو دیده خون فشانم، ز غمت شب جدایی}
\label{sec:295}
\addcontentsline{toc}{section}{\nameref{sec:295}}
\begin{longtable}{l p{0.5cm} r}
ز دو دیده خون فشانم، ز غمت شب جدایی
&&
چه کنم؟ که هست اینها گل خیر آشنایی
\\
همه شب نهاده‌ام سر، چو سگان، بر آستانت
&&
که رقیب در نیاید به بهانهٔ گدایی
\\
مژه‌ها و چشم یارم به نظر چنان نماید
&&
که میان سنبلستان چرد آهوی ختایی
\\
در گلستان چشمم ز چه رو همیشه باز است؟
&&
به امید آنکه شاید تو به چشم من درآیی
\\
سر برگ گل ندارم، به چه رو روم به گلشن؟
&&
که شنیده‌ام ز گلها همه بوی بی‌وفایی
\\
به کدام مذهب است این؟ به کدام ملت است این؟
&&
که کشند عاشقی را، که تو عاشقم چرایی؟
\\
به طواف کعبه رفتم به حرم رهم ندادند
&&
که برون در چه کردی؟ که درون خانه آیی؟
\\
به قمار خانه رفتم، همه پاکباز دیدم
&&
چو به صومعه رسیدم همه زاهد ریایی
\\
در دیر می‌زدم من، که یکی ز در در آمد
&&
که : درآ، درآ، عراقی، که تو خاص از آن مایی
\\
\end{longtable}
\end{center}
