\begin{center}
\section*{غزل شماره ۷۲: پشت بر روزگار باید کرد}
\label{sec:072}
\addcontentsline{toc}{section}{\nameref{sec:072}}
\begin{longtable}{l p{0.5cm} r}
پشت بر روزگار باید کرد
&&
روی در روی یار باید کرد
\\
چون ز رخسار پرده برگیرد
&&
در دمش جان نثار باید کرد
\\
پیش شمع رخش چو پروانه
&&
سوختن اختیار باید کرد
\\
از پی یک نظاره بر در او
&&
سال‌ها انتظار باید کرد
\\
تا کند یار روی در رویت
&&
دلت آیینه‌وار باید کرد
\\
تات در بوته‌زار بگدازد
&&
قلب خود را عیار باید کرد
\\
تا نهد بر سرت عزیزی پای
&&
خویش، چون خاک خوار باید کرد
\\
ور تو خود را ز خاک به دانی
&&
خود تو را سنگسار باید کرد
\\
تا دهی بوسه بر کف پایش
&&
خویشتن را غبار باید کرد
\\
دشمنی کت ز دوست وا دارد
&&
زودت از وی فرار باید کرد
\\
ور ز چشمت نهان بود دشمن
&&
پس دو چشمت چهار باید کرد
\\
دشمن خود تویی، چو در نگری
&&
با خودت کارزار باید کرد
\\
چون عراقی ز دست خود فریاد
&&
هر دمت صدهزار باید کرد
\\
\end{longtable}
\end{center}
