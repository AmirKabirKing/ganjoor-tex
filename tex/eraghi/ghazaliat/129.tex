\begin{center}
\section*{غزل شماره ۱۲۹: بر درت افتاده‌ام خوار و حقیر}
\label{sec:129}
\addcontentsline{toc}{section}{\nameref{sec:129}}
\begin{longtable}{l p{0.5cm} r}
بر درت افتاده‌ام خوار و حقیر
&&
از کرم، افتاده‌ای را دست گیر
\\
دردمندم، بر من مسکین نگر
&&
تا شود درد دلم درمان پذیر
\\
از تو نگریزد دل من یک زمان
&&
کالبد را کی بود از جان گزیر؟
\\
دایهٔ لطفت مرا در بر گرفت
&&
داد جای مادرم صد گونه شیر
\\
چون نیابم بوی مهرت یک نفس
&&
از دل و جانم برآید صد نفیر
\\
دل، که با وصلت چنان خو کرده بود
&&
در کف هجرت کنون مانده است اسیر
\\
باز هجرت قصد جانم می‌کند
&&
کشته‌ای را بار دیگر کشته گیر
\\
\end{longtable}
\end{center}
