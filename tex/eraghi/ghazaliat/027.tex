\begin{center}
\section*{غزل شماره ۲۷: در کوی خرابات، کسی را که نیاز است}
\label{sec:027}
\addcontentsline{toc}{section}{\nameref{sec:027}}
\begin{longtable}{l p{0.5cm} r}
در کوی خرابات، کسی را که نیاز است
&&
هشیاری و مستیش همه عین نماز است
\\
آنجا نپذیرند صلاح و ورع امروز
&&
آنچ از تو پذیرند در آن کوی نیاز است
\\
اسرار خرابات به جز مست نداند
&&
هشیار چه داند که درین کوی چه راز است؟
\\
تا مستی رندان خرابات بدیدم
&&
دیدم به حقیقت که جزین کار مجاز است
\\
خواهی که درون حرم عشق خرامی؟
&&
در میکده بنشین که ره کعبه دراز است
\\
هان! تا ننهی پای درین راه ببازی
&&
زیرا که درین راه بسی شیب و فراز است
\\
از میکده‌ها نالهٔ دلسوز برآمد
&&
در زمزمهٔ عشق ندانم که چه ساز است؟
\\
در زلف بتان تا چه فریب است؟که پیوست
&&
محمود پریشان سر زلف ایاز است
\\
زان شعله که از روی بتان حسن تو افروخت
&&
جان همه مشتاقان در سوز و گداز است
\\
چون بر در میخانه مرا بار ندادند
&&
رفتم به در صومعه، دیدم که فراز است
\\
آواز ز میخانه برآمد که: عراقی
&&
در باز تو خود را که در میکده باز است
\\
\end{longtable}
\end{center}
