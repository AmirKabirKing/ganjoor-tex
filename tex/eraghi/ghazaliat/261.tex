\begin{center}
\section*{غزل شماره ۲۶۱: چه خوش باشد دلا کز عشق یار مهربان میری}
\label{sec:261}
\addcontentsline{toc}{section}{\nameref{sec:261}}
\begin{longtable}{l p{0.5cm} r}
چه خوش باشد دلا کز عشق یار مهربان میری
&&
شراب شوق او در کام و نامش در زبان میری
\\
چو با تو شاد بنشیند ز هر چت هست برخیزی
&&
جو از رخ پرده برگیرد به پیشش شادمان میری
\\
چو عمر جاودان خواهی به روی او بر افشان جان
&&
بقای سرمدی یابی چو پیشش جان فشان میری
\\
به معنی زیستن باشد که نزد دوست جان بازی
&&
حقیقت مردن آن باشد که دور از دوستان میری
\\
در آن لحظه که بنماید جمال خود عجب نبود
&&
که از حسرت سرانگشت تعجب در دهان میری
\\
ببینی عاشقانش راکه چون در خاک و خون خسبند؟
&&
تو نیز از عاشقی باید که اندر خون چنان میری
\\
اگر تو زندگی خواهی دل از جان و جهان بگسل
&&
نیابی زندگی تا تو ز بهر این و آن میری
\\
مقام تو ورای عرش و از دون همتی خواهی
&&
که چون دونان درین عالم ز بهر یک دو نان میری
\\
به نوعی زندگانی کن که راحت یابی از مردن
&&
ببین چون می‌زیی امروز، فردا آن چنان میری
\\
اگر مشتاق جانانی چو مردی زیستی جاوید
&&
و گر عشقی دگر داری ندانم تا چسان میری؟
\\
بدو گر زنده‌ای، یابی ز مرگ آسایش کلی
&&
و گر زنده به جانی تو، ضرورت جان کنان میری
\\
عراقی، گفتنت سهل است ولیکن فعل می‌باید
&&
و گر تو هم از آنان به مردن هم چنان میری
\\
\end{longtable}
\end{center}
