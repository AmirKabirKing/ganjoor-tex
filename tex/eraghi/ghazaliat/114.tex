\begin{center}
\section*{غزل شماره ۱۱۴: صبا وقت سحر گویی ز کوی یار می‌آید}
\label{sec:114}
\addcontentsline{toc}{section}{\nameref{sec:114}}
\begin{longtable}{l p{0.5cm} r}
صبا وقت سحر گویی ز کوی یار می‌آید
&&
که بوی او شفای جان هر بیمار می‌آید
\\
نسیم خوش مگر از باغ جلوه می‌دهد گل را
&&
که آواز خوش از هر سو ز خلقی زار می‌آید
\\
بیا در گلشن ای بی‌دل، به بوی گل برافشان جان
&&
که از گلزار و گل امروز بوی یار می‌آید
\\
گل از شادی همی خندد، من از غم زار می‌گریم
&&
که از گلشن مرا یاد از رخ دلدار می‌آید
\\
ز بستان هیچ در چشمم نمی‌آید، مگر آبی
&&
که در چشمم ز یاد او دمی صدبار می‌آید
\\
اگر گلزار می‌آید کسی را خوش، مرا باری
&&
نسیم کوی او خوشتر ز صد گلزار می‌آید
\\
مرا چه از گل و گلزار؟ کاندر دست امیدم
&&
ز گلزار وصال یار زخم خار می‌آید
\\
عراقی خسته دل هردم ز سویی می‌خورد زخمی
&&
همه زخم بلا گویی برین افگار می‌آید
\\
\end{longtable}
\end{center}
