\begin{center}
\section*{غزل شماره ۲۶۳: از کرم در من بیچاره نظر کن نفسی}
\label{sec:263}
\addcontentsline{toc}{section}{\nameref{sec:263}}
\begin{longtable}{l p{0.5cm} r}
از کرم در من بیچاره نظر کن نفسی
&&
که ندارم به جز از لطف تو فریادرسی
\\
روی بنمای، که تا پیش رخت جان بدهم
&&
چه زیان دارد اگر سود کند از تو کسی؟
\\
در سرم نیست به جز دیدن تو سودایی
&&
در دلم نیست، به جز پیش تو مردن هوسی
\\
پیش از آن کز تو مرا جان به لب آید ناگاه
&&
نظری کن تو، مرا عمر نمانده است بسی
\\
تو خود انصاف بده، بلبل جان مشتاق
&&
بی‌گلستان رخت چند تپد در قفسی؟
\\
آتش هجر تو پنهان جگرم می‌سوزد
&&
لیکن از بیم نیارم که برآرم نفسی
\\
مکن از خاک سر کوی عراقی را دور
&&
باش، گو: کم نشود قیمت گوهر ز خسی
\\
\end{longtable}
\end{center}
