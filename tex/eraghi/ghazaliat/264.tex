\begin{center}
\section*{غزل شماره ۲۶۴: نگارا، وقت آن آمد که یکدم ز آن من باشی}
\label{sec:264}
\addcontentsline{toc}{section}{\nameref{sec:264}}
\begin{longtable}{l p{0.5cm} r}
نگارا، وقت آن آمد که یکدم ز آن من باشی
&&
دلم بی‌تو به جان آمد، بیا، تا جان من باشی
\\
دلم آنگاه خوش گردد که تو دلدار من باشی
&&
مرا جان آن زمان باشد که تو جانان من باشی
\\
به غم زان شاد می‌گردم که تو غم خوار من گردی
&&
از آن با درد می‌سازم که تو درمان من باشی
\\
بسا خون جگر، جانا، که بر خوان غمت خوردم
&&
به بوی آنکه یک باری تو هم مهمان من باشی
\\
منم دایم تو را خواهان، تو و خواهان خود دایم
&&
مرا آن بخت کی باشد که تو خواهان من باشی؟
\\
همه زان خودی، جانا، از آن با کس نپردازی
&&
چه باشد، ای ز جان خوشتر ، که یک دم آن من باشی؟
\\
اگر تو آن من باشی، ازین و آن نیندیشم
&&
ز کفر آخر چرا ترسم، چو تو ایمان من باشی؟
\\
ز دوزخ آنگهی ترسم که جز تو مالکی یابم
&&
بهشت آنگاه خوش باشد که تو رضوان من باشی
\\
فلک پیشم زمین بوسد، چو من خاک درت بوسم
&&
ملک پیشم کمر بندد، چو تو سلطان من باشی
\\
عراقی، بس عجب نبود که اندر من بود حیران
&&
چو خود را بنگری در من، تو هم حیران من باشی
\\
\end{longtable}
\end{center}
