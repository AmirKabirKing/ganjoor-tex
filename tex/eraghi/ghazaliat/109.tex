\begin{center}
\section*{غزل شماره ۱۰۹: اندرین ره هر که او یکتا شود}
\label{sec:109}
\addcontentsline{toc}{section}{\nameref{sec:109}}
\begin{longtable}{l p{0.5cm} r}
اندرین ره هر که او یکتا شود
&&
گنج معنی در دلش پیدا شود
\\
جز جمال خود نبیند در جهان
&&
اندرین ره هر که او بینا شود
\\
قطره کز دریا برون آید همی
&&
چون سوی دریا شود دریا شود
\\
گر صفات خود کند یکباره محو
&&
در مقامات بقا یکتا شود
\\
هر که دل بر نیستی خود نهاد
&&
در حریم هستی، او تنها شود
\\
از مسما هر که یابد بهره‌ای
&&
فارغ و آسوده از اسما شود
\\
ور کند گم صورت هستی خویش
&&
صورت او جملگی معنی شود
\\
ور نهنگ لاخورش زو طعمه ساخت
&&
زندهٔ جاوید در الا شود
\\
صورتت چون شد حجاب راه تو
&&
محو کن، تا سیرتت زیبا شود
\\
گر از این منزل برون رفتی، یقین
&&
دانکه منزلگاهت او ادنی شود
\\
ما به جانان زنده‌ایم، از جان بری
&&
تا ابد هرگز کسی چون ما شود؟
\\
هر که آنجا مقصد و مقصود یافت
&&
در دو عالم والی والا شود
\\
هر که را دل رازدار عشق شد
&&
کی دلش مایل سوی صحرا شود؟
\\
هم به بالا در رسد بی‌عقل و دین
&&
گر عراقی محو اندر لا شود
\\
\end{longtable}
\end{center}
