\begin{center}
\section*{غزل شماره ۱۶۹: در حسن رخ خوبان پیدا همه او دیدم}
\label{sec:169}
\addcontentsline{toc}{section}{\nameref{sec:169}}
\begin{longtable}{l p{0.5cm} r}
در حسن رخ خوبان پیدا همه او دیدم
&&
در چشم نکورویان زیبا همه او دیدم
\\
در دیدهٔ هر عاشق او بود همه لایق
&&
وندر نظر وامق عذرا همه او دیدم
\\
دلدار دل افگاران غم‌خوار جگرخواران
&&
یاری ده بی‌یاران، هرجا همه او دیدم
\\
مطلوب دل در هم او یافتم از عالم
&&
مقصود من پر غم ز اشیا همه او دیدم
\\
دیدم همه پیش و پس، جز دوست ندیدم کس
&&
او بود، همه او، بس، تنها همه او دیدم
\\
آرام دل غمگین جز دوست کسی مگزین
&&
فی‌الجمله همه او بین، زیرا همه او دیدم
\\
دیدم گل بستان ها ، صحرا و بیابان ها
&&
او بود گلستان ها ، صحرا همه او دیدم
\\
هان! ای دل دیوانه، بخرام به میخانه
&&
کاندر خم و پیمانه پیدا همه او دیدم
\\
در میکده و گلشن، می‌نوش می روشن
&&
میبوی گل و سوسن، کاینها همه او دیدم
\\
در میکده ساقی شو، می در کش و باقی شو
&&
جویای عراقی شو، کو را همه او دیدم
\\
\end{longtable}
\end{center}
