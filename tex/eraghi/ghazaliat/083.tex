\begin{center}
\section*{غزل شماره ۸۳: آن را که چو تو نگار باشد}
\label{sec:083}
\addcontentsline{toc}{section}{\nameref{sec:083}}
\begin{longtable}{l p{0.5cm} r}
آن را که چو تو نگار باشد
&&
با خویشتنش چه کار باشد؟
\\
ناخوش نبود کسی که او را
&&
یاری چو تو در کنار باشد
\\
ناخوش چو منی بود که پیوست
&&
دل خسته و جان فگار باشد
\\
مزار ز من، اگر بنالم
&&
ماتم‌زده سوکوار باشد
\\
وان دیده که او ندید رویت
&&
شاید اگر آشکار باشد
\\
آن کس که جدا فتاد از تو
&&
دور از تو همیشه زار باشد
\\
بیچاره کسی که در دو عالم
&&
جز تو دگریش یار باشد
\\
خرم دل آن کسی که او را
&&
اندوه تو غمگسار باشد
\\
تا کی دلم، ای عزیز چون جان
&&
بر خاک در تو خوار باشد؟
\\
نامد گه آن که خسته‌ای را
&&
بر درگه وصل بار باشد؟
\\
تا چند دل عراقی آخر
&&
در زحمت انتظار باشد؟
\\
\end{longtable}
\end{center}
