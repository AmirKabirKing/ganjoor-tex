\begin{center}
\section*{غزل شماره ۸۹: ناگه بت من مست به بازار برآمد}
\label{sec:089}
\addcontentsline{toc}{section}{\nameref{sec:089}}
\begin{longtable}{l p{0.5cm} r}
ناگه بت من مست به بازار برآمد
&&
شور از سر بازار به یکبار برآمد
\\
مانا به کرشمه سوی او باز نظر کرد
&&
کین شور و شغب از سر بازار برآمد
\\
با اهل خرابات ندانم چه سخن گفت؟
&&
کاشوب و غریو از در خمار برآمد
\\
در صومعه ناگاه رخش پرده برانداخت
&&
فریاد و فغان از دل ابرار برآمد
\\
آورد چو در کار لب و غمزه و رخسار
&&
جان و دل و چشم همه از کار برآمد
\\
تا جز رخ او هیچ کسی هیچ نبیند
&&
در جمله صور آن بت عیار برآمد
\\
هر بار به رنگی بت من روی نمودی
&&
آن بار به رنگ همه اطوار برآمد
\\
و آن شیفته کز زلف و قدش دار و رسن یافت
&&
بگرفت رسن، خوش به سر دار برآمد
\\
فی‌الجمله برآورد سر از جیب بزودی
&&
هر دم به لباسی دگر آن یار برآمد
\\
و آن سوخته کاتش همه تاب رخ او دید
&&
زو دعوی «النار ولاالعار» برآمد
\\
المنةلله که پس از منت بسیار
&&
مقصود و مرادم ز لب یار برآمد
\\
دور از لب و دندان عراقی همه کامم
&&
زان دو لب شیرین شکر بار برآمد
\\
\end{longtable}
\end{center}
