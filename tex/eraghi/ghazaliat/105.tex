\begin{center}
\section*{غزل شماره ۱۰۵: هر که در بند زلف یار بود}
\label{sec:105}
\addcontentsline{toc}{section}{\nameref{sec:105}}
\begin{longtable}{l p{0.5cm} r}
هر که در بند زلف یار بود
&&
در جهانش کجا قرار بود؟
\\
وانکه چیند گلی ز باغ رخش
&&
در دلش بس که خار خار بود
\\
وانکه یاد لبش کند روزی
&&
تا قیامت در آن خمار بود
\\
کارهایی که چشم یار کند
&&
نه زیاری روزگار بود
\\
فتنه‌هایی که زلفش انگیزد
&&
همه خود نقش آن نگار بود
\\
از فلک آنکه هر شبی شنوی
&&
نالهٔ بیدلان زار بود
\\
نفس عاقشان او باشد
&&
آن کزو چرخ را مدار بود
\\
یک شبی با خیال او گفتم:
&&
چند مسکین در انتظار بود؟
\\
روی بنما، که جان نثار کنم
&&
گفت: جان را چه اعتبار بود؟
\\
تا تو در بند خویشتن مانی
&&
کی تو را نزد دوست بار بود؟
\\
نبود عاشق آنکه جوید کام
&&
عشق را با غرض چه کار بود؟
\\
عاشق آن است کو نخواهد هیچ
&&
ور همه خود وصال یار بود
\\
ای عراقی، تو اختیار مکن
&&
کانکه به بود اختیار بود
\\
\end{longtable}
\end{center}
