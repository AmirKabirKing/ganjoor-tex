\begin{center}
\section*{غزل شماره ۱۸۰: دلی یا دلبری، یا جان و یا جانان، نمی‌دانم}
\label{sec:180}
\addcontentsline{toc}{section}{\nameref{sec:180}}
\begin{longtable}{l p{0.5cm} r}
دلی یا دلبری، یا جان و یا جانان، نمی‌دانم
&&
همه هستی تویی، فی‌الجمله، این و آن نمی‌دانم
\\
بجز تو در همه عالم دگر دلبر نمی‌بینم
&&
بجز تو در همه گیتی دگر جانان نمی‌دانم
\\
بجز غوغای عشق تو درون دل نمی‌یابم
&&
بجز سودای وصل تو میان جان نمی‌دانم
\\
چه آرم بر در وصلت؟ که دل لایق نمی‌افتد
&&
چه بازم در ره عشقت؟ که جان شایان نمی‌دانم
\\
یکی دل داشتم پر خون شد آن هم از کفم بیرون
&&
کجا افتاد آن مجنون، درین دوران؟ نمی‌دانم
\\
دلم سرگشته می‌دارد سر زلف پریشانت
&&
چه می‌خواهد ازین مسکین سرگردان؟ نمی‌دانم
\\
دل و جان مرا هر لحظه بی جرمی بیزاری
&&
چه می خواهی ازین مسکین سرگردان ؟ نمی دانم
\\
اگر مقصود تو جان است، رخ بنما و جان بستان
&&
و گر قصد دگر داری، من این و آن نمی‌دانم
\\
مرا با توست پیمانی، تو با من کرده‌ای عهدی
&&
شکستی عهد، یا هستی بر آن پیمان؟ نمی‌دانم
\\
تو را یک ذره سوی خود هواخواهی نمی‌بینم
&&
مرا یک موی بر تن نیست کت خواهان نمی‌دانم
\\
چه بی‌روزی کسم، یارب، که از وصل تو محرومم!
&&
چرا شد قسمت بختم ز تو حرمان؟ نمی‌دانم
\\
چو اندر چشم هر ذره، چو خورشید آشکارایی
&&
چرایی از من حیران چنین پنهان؟ نمی‌دانم
\\
به امید وصال تو دلم را شاد می‌دارم
&&
چرا درد دل خود را دگر درمان نمی‌دانم؟
\\
نمی‌یابم تو را در دل، نه در عالم، نه در گیتی
&&
کجا جویم تو را آخر من حیران؟ نمی‌دانم
\\
عجب‌تر آنکه می‌بینم جمال تو عیان، لیکن
&&
نمی‌دانم چه می‌بینم من نادان؟ نمی‌دانم
\\
همی‌دانم که روزوشب جهان روشن به روی توست
&&
ولیکن آفتابی یا مه تابان؟ نمی‌دانم
\\
به زندان فراقت در، عراقی پایبندم شد
&&
رها خواهم شدن یا نی، ازین زندان؟ نمی‌دانم
\\
\end{longtable}
\end{center}
