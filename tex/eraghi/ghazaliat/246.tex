\begin{center}
\section*{غزل شماره ۲۴۶: نگارا، گر چه از ما برشکستی}
\label{sec:246}
\addcontentsline{toc}{section}{\nameref{sec:246}}
\begin{longtable}{l p{0.5cm} r}
نگارا، گر چه از ما برشکستی
&&
ز جانت بنده‌ام، هر جا که هستی
\\
ربودی دل ز من، چون رخ نمودی
&&
شکستی پشت من، چون برشکستی
\\
چرا پیوستی، ای جان، با دل من؟
&&
چو آخر دست، از من می‌گسستی
\\
ز نوش لب چو مرهم می‌ندادی
&&
ز نیش لب چرا جانم بخستی؟
\\
ز بهر کشتنم صد حیله کردی
&&
چو خونم ریختی فارغ نشستی
\\
اگر چه یافتی از کشتنم رنج
&&
ز محنت‌های من، باری، برستی
\\
مرا کشتی، به طنز آنگاه گویی:
&&
عراقی، از کف من نیک جستی!
\\
\end{longtable}
\end{center}
