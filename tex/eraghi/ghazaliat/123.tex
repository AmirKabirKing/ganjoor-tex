\begin{center}
\section*{غزل شماره ۱۲۳: غلام روی توام، ای غلام، باده بیار}
\label{sec:123}
\addcontentsline{toc}{section}{\nameref{sec:123}}
\begin{longtable}{l p{0.5cm} r}
غلام روی توام، ای غلام، باده بیار
&&
که فارغ آمدم از ننگ و نام، باده بیار
\\
کرشمه‌های خوش تو شراب ناب من است
&&
درآ به مجلس و پیش از طعام باده بیار
\\
به غمزه‌ای چو مرا مست می‌توانی کرد
&&
چه حاجت است صراحی و جام؟ باده بیار
\\
به مستی از لب تو وام کرده‌ام بوسی
&&
گر آمدی به تقاضای وام، باده بیار
\\
مگر که مرغ طرب درفتد به دام مرا
&&
شده است تن همه دیده چو دام، باده بیار
\\
کجاست دانهٔ مرغان؟ که طوطی روحم
&&
فتاد از پی دانه به دام، باده بیار
\\
نظام بزم طرب از می است، مجلس ما
&&
چو می نگیرد بی می نظام، باده بیار
\\
عنان ربود ز من توسن طرب، ساقی
&&
مگر زبون شود این بدلگام، باده بیار
\\
ز انتظار چو ساغر دلم پر از خون شد
&&
مدار منتظرم بر دوام، باده بیار
\\
اگر چه روز فروشد، صبوح فوت مکن
&&
که آفتاب برآید ز جام، باده بیار
\\
درین مقام که خونم حلال می‌داری
&&
مدار خون صراحی حرام، باده بیار
\\
به وقت شام، بیا تا قضای صبح کنیم
&&
اگر چه صبح خوش آید، به شام باده بیار
\\
نمی‌پزد تف غم آرزوی خام مرا
&&
برای پختن سودای خام باده بیار
\\
منم کنون و یکی نیم جان رسیده به لب
&&
همی دهم به تو، بستان تمام، باده بیار
\\
به مستی از لب تو می‌توان ستد بوسی
&&
مگر رسم ز لب تو به کام، باده بیار
\\
مرا ز دست عراقی خلاص ده نفسی
&&
غلام روی توام، ای غلام، باده بیار
\\
\end{longtable}
\end{center}
