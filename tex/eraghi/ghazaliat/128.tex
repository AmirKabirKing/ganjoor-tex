\begin{center}
\section*{غزل شماره ۱۲۸: ای امید جان، عنایت از عراقی وامگیر}
\label{sec:128}
\addcontentsline{toc}{section}{\nameref{sec:128}}
\begin{longtable}{l p{0.5cm} r}
ای امید جان، عنایت از عراقی وامگیر
&&
چاره ساز آن را که از تو نیستش یک دم گزیر
\\
مانده در تیه فراقم، رهنمایا، ره نمای
&&
غرقهٔ دریای هجرم، دستگیرا، دست گیر
\\
در دل زارم نظر کن، کز غمت آمد به جان
&&
چاره کن، جانا، که شد در دست هجرانت اسیر
\\
سوی من بنگر، که عمری بر امید یک نظر
&&
مانده‌ام چون خاک بر خاک درت خوار و حقیر
\\
از تو بو نایافته، نه راحتی دیده ز عمر
&&
ساخته با درد بی‌درمان تو، مسکین فقیر
\\
دل که سودای تو می‌پخت آرزویش خام ماند
&&
کو تنور آرزو تا اندر او بندم فطیر؟
\\
دایهٔ مهرت به شیر لطف پرورده است جان
&&
شیرخواره چون زید، کش باز گیرد دایه شیر؟
\\
ز آفتاب مهر بر دل سایه افگن، تا شود
&&
در هوای مهر روی تو چو ذره مستنیر
\\
گر فتد بر خاک تیره پرتو عکس رخت
&&
گردد اندر حال هر ذره چو خورشید منیر
\\
وز نسیم لطف تو بر آتش دوزخ وزد
&&
خوشتر از خلد برین گردد درک‌های سعیر
\\
\end{longtable}
\end{center}
