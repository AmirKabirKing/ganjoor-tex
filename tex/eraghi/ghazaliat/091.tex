\begin{center}
\section*{غزل شماره ۹۱: بیا، که بی‌رخ زیبات دل به جان آمد}
\label{sec:091}
\addcontentsline{toc}{section}{\nameref{sec:091}}
\begin{longtable}{l p{0.5cm} r}
بیا، که بی‌رخ زیبات دل به جان آمد
&&
بیا، که بی‌تو همه سود من زیان آمد
\\
بیا، که بهر تو جان از جهان کرانه گرفت
&&
بیا، که بی‌تو دلم جمله در میان آمد
\\
بیا، که خانهٔ دل گرچه تنگ و تاریک است
&&
دمی برای دل ما درون توان آمد
\\
بیا، که غیر تو در چشم من نیامد هیچ
&&
جز آب دیده که بر چشم من روان آمد
\\
نگر هر آنچه که بر هیچکس نیامده بود
&&
برین شکسته دلم از غم تو آن آمد
\\
دل شکسته‌ام آن لحظه دل ز جان برداشت
&&
که رسم جور و جفای تو در جهان آمد
\\
ز جور یار چه نالم؟ که طالع دل من
&&
چنان که بخت عراقی است همچنان آمد
\\
\end{longtable}
\end{center}
