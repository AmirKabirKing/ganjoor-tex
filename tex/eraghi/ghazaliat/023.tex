\begin{center}
\section*{غزل شماره ۲۳: ندیده‌ام رخ خوب تو، روزکی چند است}
\label{sec:023}
\addcontentsline{toc}{section}{\nameref{sec:023}}
\begin{longtable}{l p{0.5cm} r}
ندیده‌ام رخ خوب تو، روزکی چند است
&&
بیا، که دیده به دیدارت آرزومند است
\\
به یک نظاره به روی تو دیده خشنود است
&&
به یک کرشمه دل از غمزهٔ تو خرسند است
\\
فتور غمزهٔ تو خون من بخواهد ریخت
&&
بدین صفت که در ابرو گره درافکند است
\\
یکی گره بگشای از دو زلف و رخ بنمای
&&
که صدهزار چو من دلشده در آن بند است
\\
مبر ز من، که رگ جان من بریده شود
&&
بیا، که با تو مرا صدهزار پیوند است
\\
مرا چو از لب شیرین تو نصیبی نیست
&&
از آن چه سود که لعل تو سر به سرقند است؟
\\
کسی که همچو عراقی اسیر عشق تو نیست
&&
شب فراق چه داند که تا سحر چند است؟
\\
\end{longtable}
\end{center}
