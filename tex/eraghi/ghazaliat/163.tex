\begin{center}
\section*{غزل شماره ۱۶۳: هیهات! کزین دیار رفتم}
\label{sec:163}
\addcontentsline{toc}{section}{\nameref{sec:163}}
\begin{longtable}{l p{0.5cm} r}
هیهات! کزین دیار رفتم
&&
ناکرده وداع یار رفتم
\\
چه سود قرار وصل جانان؟
&&
اکنون که من از قرار رفتم
\\
چون خاک در تو بوسه دادم
&&
با دیدهٔ اشکبار رفتم
\\
بگذاشتم، ای عزیز چون جان،
&&
دل نزد تو یادگار رفتم
\\
زنهار! دل مرا نگه‌دار
&&
چون من ز میان کار رفتم
\\
بردند به اضطرارم، ای دوست،
&&
زین جا نه به اختیار رفتم
\\
غم خواره و مونسم تو بودی
&&
بی‌مونس و غمگسار رفتم
\\
از خلق کریم تو ندیدم
&&
یک عهد چو استوار، رفتم
\\
چون از لب تو نیافتم کام
&&
ناکام به هر دیار رفتم
\\
نایافته مرهمی ز لطفت
&&
دل خسته و جان فگار رفتم
\\
شکرانه بده، که از در تو
&&
چون محنت روزگار رفتم
\\
تو خرم و شاد و کامران باش
&&
کز شهر تو سوکوار رفتم
\\
در قصهٔ درد من نگه کن
&&
بنگر که چگونه زار رفتم
\\
\end{longtable}
\end{center}
