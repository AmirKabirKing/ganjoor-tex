\begin{center}
\section*{غزل شماره ۲۳۶: ای در میان جانم گنجی نهان نهاده}
\label{sec:236}
\addcontentsline{toc}{section}{\nameref{sec:236}}
\begin{longtable}{l p{0.5cm} r}
ای در میان جانم گنجی نهان نهاده
&&
بس نکته‌های معنی اندر زبان نهاده
\\
سر حکیم ما را در شوق لایزالی
&&
در من یزید عشقش پیش دکان نهاده
\\
در جلوه‌گاه معنی معشوق رخ نموده
&&
در بارگاه صورت تختش عیان نهاده
\\
از نیست هست کرده، از بهر جلوهٔ خود
&&
وانگه نشان هستی بر بی‌نشان نهاده
\\
روحی بدین لطیفی در چاه تن فگنده
&&
سری بدین عزیزی در قعر جان نهاده
\\
خود کرده رهنمایی آدم به سوی گندم
&&
ابلیس بهر تادیب اندر میان نهاده
\\
خود کرده آنچه کرده، وانگه بدین بهانه
&&
هر لحظه جرم و عصیان بر این و آن نهاده
\\
بعضی برای دوزخ، بعضی برای انسان
&&
اندر بهشت باقی امن و امان نهاده
\\
کس را درین میانه چون و چرا نزیبد
&&
هر کس نصیب او را هم غیب‌دان نهاده
\\
عمری درین تفکر، از غایت تحیر
&&
گوش دل عراقی بر آستان نهاده
\\
\end{longtable}
\end{center}
