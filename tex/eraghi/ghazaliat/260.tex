\begin{center}
\section*{غزل شماره ۲۶۰: نمی‌دانم چه بد کردم، که نیکم زار می‌داری}
\label{sec:260}
\addcontentsline{toc}{section}{\nameref{sec:260}}
\begin{longtable}{l p{0.5cm} r}
نمی‌دانم چه بد کردم، که نیکم زار می‌داری؟
&&
تنم رنجور می‌خواهی، دلم بیمار می‌داری
\\
ز درد من خبر داری، ازینم دیر می‌پرسی
&&
به زاری کردنم شادی، از آنم زار می‌داری
\\
دلم را خسته می‌داری ز تیر غم، روا باشد
&&
به دست هجر جانم را چرا افگار می‌داری؟
\\
چه آزاری ز من خود را؟ به آزاری نمی‌ارزم
&&
که باشم؟ خود کیم؟ کز من چنین آزار می‌داری؟
\\
مرا دشمن چه می‌داری؟ که نیکت دوست‌می‌دارم
&&
مرا چون یار می‌دانی چرا اغیار می‌داری؟
\\
مرا گویی: مشو غمگین، که غم خوارت شوم روزی
&&
ندانم آن، کنون باری، مرا غم خوار می‌داری
\\
نهی بر جان من منت که: خواهم داشت تیمارت
&&
دلم خون شد ز تیمارت، نکو تیمار می‌داری!
\\
دریغا! آنکه گه گاهی به دردم یاد می‌کردی
&&
عزیزم داشتی اول، به آخر خوار می‌داری
\\
به دردی قانعم از تو، به دشنامی شدم راضی
&&
درین هم یاریم ندهی، چگونه یار می‌داری؟
\\
درین هم یاریم ندهی، به دشنامی عزیزم کن
&&
به دردی قانعم از تو، چگونه یار می‌داری؟
\\
به هر رویی که بتوانم من از تو رو نگردانم
&&
اگر بر تخت بنشانی وگر بر دار می‌داری
\\
به تو هر کس که فخر آرد، نداری عاز ازو، دانم
&&
عراقی نیک بدنام است، از آن رو عار می‌داری
\\
\end{longtable}
\end{center}
