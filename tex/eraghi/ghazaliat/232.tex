\begin{center}
\section*{غزل شماره ۲۳۲: ساقی، قدحی می مغان کو}
\label{sec:232}
\addcontentsline{toc}{section}{\nameref{sec:232}}
\begin{longtable}{l p{0.5cm} r}
ساقی، قدحی می مغان کو؟
&&
مطرب غزل تر روان کو؟
\\
آن مونس دل کجاست آخر؟
&&
و آن راحت جان ناتوان کو؟
\\
آیینهٔ سینه زنگ غم خورد
&&
آن صیقل غمزدای جان کو؟
\\
از زهد و صلاح توبه کردم
&&
مخمور میم، می مغان کو؟
\\
اسباب طرب همه مهیاست
&&
آن زاهد خشک جان فشان کو؟
\\
گر زهد تو نیست جمله تزویر
&&
ترک بد و نیک و سوزیان کو؟
\\
ور از دو جهان کران گرفتی
&&
جان و دل و دیده در میان کو؟
\\
با شاهد و شمع در خرابات
&&
عیش خوش و عمر جاودان کو؟
\\
در صومعه چند زهد ورزیم؟
&&
صحرا و گل و می مغان کو؟
\\
چون بلبل بی‌نوا چه باشیم؟
&&
بوی خوش باغ و بوستان کو؟
\\
ما را چه ز باغ و بوی گلزار؟
&&
بوی سر زلف دلستان کو؟
\\
با دل گفتم: مرا نگویی
&&
کان یار لطیف مهربان کو؟
\\
ور یافته‌ای ازو نشانی
&&
خونابهٔ چشم خون فشان کو؟
\\
با هم بودیم روزکی چند
&&
آن عیش کجا و آن زمان کو؟
\\
دل گفت: هر آنچه او ندانست
&&
از وی چه نشان دهیم: آن کو؟
\\
با این همه جهد می کنم هم
&&
باشد که دمی شود چنان کو
\\
خواهد که فدا کند عراقی
&&
جان در ره او، ولیک جان کو؟
\\
\end{longtable}
\end{center}
