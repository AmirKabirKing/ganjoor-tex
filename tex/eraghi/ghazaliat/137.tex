\begin{center}
\section*{غزل شماره ۱۳۷: ساقی، ز شکر خنده شراب طرب انگیز}
\label{sec:137}
\addcontentsline{toc}{section}{\nameref{sec:137}}
\begin{longtable}{l p{0.5cm} r}
ساقی، ز شکر خنده شراب طرب انگیز
&&
در ده، که به جان آمدم از توبه و پرهیز
\\
در بزم ز رخسار دو صد شمع برافروز
&&
وز لعل شکربار می و نقل فرو ریز
\\
هر ساعتی از غمزه فریبی دگر آغاز
&&
هر دم ز کرشمه شر و شوری دگر انگیز
\\
آن دل که به رخسار تو دزدیده نظر کرد
&&
او را به سر زلف نگونسار درآویز
\\
و آن جام که به دام سر زلف تو درافتاد
&&
قیدش کن و بسپار بدان غمزهٔ خونریز
\\
در شهر ز عشق تو بسی فتنه و غوغاست
&&
از خانه برون آ، بنشان شور شغب خیز
\\
چون طینت من از می مهر تو سرشتند
&&
کی توبه کنم از می ناب طرب انگیز؟
\\
ای فتنه، که آموخت تو را کز رخ چون ماه
&&
بفریب دل اهل جهان ناگه و بگریز؟
\\
خواهی که بیابی دل گم کرده، عراقی؟
&&
خاک در میخانه به غربال فرو بیز
\\
\end{longtable}
\end{center}
