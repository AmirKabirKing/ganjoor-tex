\begin{center}
\section*{غزل شماره ۱۸: باز مرا در غمت واقعه جانی است}
\label{sec:018}
\addcontentsline{toc}{section}{\nameref{sec:018}}
\begin{longtable}{l p{0.5cm} r}
باز مرا در غمت واقعه جانی است
&&
در دل زارم نگر، تا به چه حیرانی است
\\
دل که ز جان سیر گشت خون جگر می‌خورد
&&
بر سر خوان غمت باز به مهمانی است
\\
چون دل تنگم نشد شاد به تو یک زمان
&&
باز گذارش به غم، کوبه غم ارزانی است
\\
تا سر زلفین تو کرد پریشان دلم
&&
هیچ نگویی بدو کین چه پریشانی است؟
\\
از دل من خون چکید بر جگرم نم نماند
&&
تا ز غمت دیده‌ام در گهر افشانی است
\\
آه! که در طالعم باز پراکندگی است
&&
بخت بد آخر بگو کین چه پریشانی است
\\
رفت که بودی مرا کار به سامان، دریغ!
&&
نوبت کارم کنون بی سر و سامانی است
\\
صبح وصالم بماند در پس کوه فراق
&&
روز امیدم چو شب تیره و ظلمانی است
\\
وصل چو تو پادشه کی به گدایی رسد؟
&&
جستن وصلت مرا مایهٔ نادانی است
\\
خیز، دلا، وصل جو، ترک عراقی بگو
&&
دوست مدارش، که او دشمن پنهانی است
\\
\end{longtable}
\end{center}
