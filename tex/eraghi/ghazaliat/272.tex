\begin{center}
\section*{غزل شماره ۲۷۲: جانا، ز منت ملال تا کی}
\label{sec:272}
\addcontentsline{toc}{section}{\nameref{sec:272}}
\begin{longtable}{l p{0.5cm} r}
جانا، ز منت ملال تا کی؟
&&
مولای توام، دلال تا کی؟
\\
از حسن تو بازمانده تا چند؟
&&
بر صبر من احتمال تا کی؟
\\
بردار ز رخ نقاب یکبار
&&
در پرده چنان جمال تا کی؟
\\
از پرتو آفتاب رویت
&&
چون سایه مرا زوال تا کی؟
\\
یکباره ز من ملول گشتی
&&
از عاشق خود ملال تا کی؟
\\
بی وصل تو در هوای مهرت
&&
چون ذره مرا مجال تا کی؟
\\
خورشید رخا، به من نظر کن
&&
از ذره نهان جمال تا کی؟
\\
در لعل تو آب زندگانی
&&
من تشنهٔ آن زلال تا کی؟
\\
وصل خوش تو حرام تا چند ؟
&&
خون دل من حلال تا کی ؟
\\
فریاد من از تو چند باشد؟
&&
بیداد تو ماه و سال تا کی؟
\\
از دست تو پایمال گشتم
&&
آخر ز تو گوشمال تا کی؟
\\
ای دوست، به کام دشمنان باز
&&
کام دل بدسگال تا کی؟
\\
دل خون شده، جان به لب رسیده
&&
از حسرت آن جمال تا کی؟
\\
با دل به عتاب دوش گفتم:
&&
کایدل، پی هر خیال تا کی؟
\\
اندیشهٔ وصل یار بگذار
&&
سرگشته پی محال تا کی؟
\\
در پرتو آفتاب حسنش
&&
ای ذره تو را مجال تا کی؟
\\
آشفتهٔ روی خوب تا چند؟
&&
دیوانهٔ زلف و خال تا کی؟
\\
از مهر رخ جهان فروزش
&&
ای سایه، تو را زوال تا کی؟
\\
از حلقهٔ زلف هر نگاری
&&
بر پای دلت عقال تا کی؟
\\
در عشق خیال هر جمالی
&&
پیوسته اسیر خال تا کی؟
\\
بر بوی وصال عمر بگذشت
&&
آخر طلب محال تا کی؟
\\
در وصل تو را چو نیست طالع
&&
از دفتر هجر فال تا کی؟
\\
نادیده رخش به خواب یکشب
&&
ای خفته، درین خیال تا کی؟
\\
هر شب منم و خیال جانان
&&
من دانم و او و قال تا کی؟
\\
دل گفت که: حال من چه پرسی؟
&&
از شیفتگان سال تا کی؟
\\
من دانم و عشق، چند گویی؟
&&
با بی‌خبران جدال تا کی؟
\\
دم در کش و خون گری، عراقی
&&
فریاد چه؟ قیل و قال تا کی؟
\\
\end{longtable}
\end{center}
