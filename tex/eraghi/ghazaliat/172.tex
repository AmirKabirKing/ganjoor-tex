\begin{center}
\section*{غزل شماره ۱۷۲: چه خوش بودی، دریغا، روزگارم}
\label{sec:172}
\addcontentsline{toc}{section}{\nameref{sec:172}}
\begin{longtable}{l p{0.5cm} r}
چه خوش بودی، دریغا، روزگارم؟
&&
اگر با من خوشستی غمگسارم
\\
به آب دیده دست از خود بشویم
&&
کنون کز دست بیرون شد نگارم
\\
نگارا، بر تو نگزینم کسی را
&&
تویی از جمله خوبان اختیارم
\\
مرا جانی، که می‌دارم تو را دوست
&&
عجب نبود که جان را دوست دارم
\\
مرا تا کار با زلف تو باشد
&&
پریشان‌تر ز زلف توست کارم
\\
مرا کرامگه زلف تو باشد
&&
ببین چون باشد آرام و قرارم؟
\\
به بوی آنکه دامان تو گیرم
&&
نشسته بر سر ره چون غبارم
\\
در آویزم به دامان تو یک شب
&&
مگر روزی سر از جیبت برآرم
\\
عراقی، دامن او گیر و خوش باش
&&
که من با تو درین اندیشه یارم
\\
\end{longtable}
\end{center}
