\begin{center}
\section*{غزل شماره ۱۹۱: گر چه ز جهان جوی نداریم}
\label{sec:191}
\addcontentsline{toc}{section}{\nameref{sec:191}}
\begin{longtable}{l p{0.5cm} r}
گر چه ز جهان جوی نداریم
&&
هم سر به جهان فرو نیاریم
\\
زان جا که حساب همت ماست
&&
عالم همه حبه‌ای شماریم
\\
خود با دو جهان چکار ما را؟
&&
ما شیفتهٔ یکی نگاریم
\\
کی صید جهان شویم؟ چون ما
&&
در بند کمند زلف یاریم
\\
در دل همه مهر او نویسیم
&&
بر جان همه عشق او نگاریم
\\
این خود همه هست، بر در او
&&
از خاک بتر هزار باریم
\\
ما خود خجلیم از رخ یار
&&
با آنکه ز عشق زار زاریم
\\
از کردهٔ خود سیاه‌روییم
&&
وز گفتهٔ خویش شرمساریم
\\
رویش به کدام چشم بینیم؟
&&
وصلش به چه روی چشم داریم؟
\\
ما در خور او نه‌ایم، لیکن
&&
با این همه هم امیدواریم
\\
ای دوست، گناه ما همین است
&&
کز دیده و جانت دوست داریم
\\
باری، به نظاره‌ای برون آی
&&
بنگر که: چگونه جان سپاریم
\\
بر بوی نظارهٔ جمالت
&&
دیری است که ما در انتظاریم
\\
یک ره بنگر سوی عراقی
&&
بنگر که: چگونه جان سپاریم
\\
\end{longtable}
\end{center}
