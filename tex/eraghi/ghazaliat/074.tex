\begin{center}
\section*{غزل شماره ۷۴: می روان کن ساقیا، کین دم روان خواهیم کرد}
\label{sec:074}
\addcontentsline{toc}{section}{\nameref{sec:074}}
\begin{longtable}{l p{0.5cm} r}
می روان کن ساقیا، کین دم روان خواهیم کرد
&&
بهر یک جرعه میت این دم روان خواهیم کرد
\\
دردیی در ده، کزین جا دردسر خواهیم برد
&&
ساغری پر کن، که عزم آن جهان خواهیم کرد
\\
کاروان عمر ازین منزل روان شد ناگهی
&&
چون روان شد کاروان، ما هم روان خواهیم کرد
\\
چون فشاندیم آستین بی‌نیازی بر جهان
&&
دامن ناز اندر آن عالم کشان خواهیم کرد
\\
از کف ساقی همت ساغری خواهیم خورد
&&
جرعه‌دان بزم خود هفت آسمان خواهیم کرد
\\
تا فتد در ساغر ما عکس روی دلبری
&&
ساغر از باده لبالب هر زمان خواهیم کرد
\\
درچنین مجلس که می‌عشق است‌و ساغربیخودی
&&
نالهٔ مستانه نقل دوستان خواهیم کرد
\\
تا درین عالم نگردد آشکارا راز ما
&&
ناگهی رخ را ازین عالم نهان خواهیم کرد
\\
نزد زلف دلربایش تحفه، دل خواهیم برد
&&
پیش روی جانفزایش جان فشان خواهیم کرد
\\
چون بگردانیم رو، زین عالم بی‌آبرو
&&
روی در روی نگار مهربان خواهیم کرد
\\
بر سر بازار وصلش جان ندارد قیمتی
&&
تا نظر در روی خوبش رایگان خواهیم کرد
\\
سالها در جستجویش دست و پایی می‌زدیم
&&
چون نشان دیدیم، خود را بی‌نشان خواهیم کرد
\\
هر چه ما خواهیم کردن او بخواهد غیر آن
&&
آنچه آن دلبر کند ما خود همان خواهیم کرد
\\
عراقی هیچ خواهد گفت: اناالحق، این زمان
&&
بر سر دارش ز غیرت ناگهان خواهیم کرد
\\
\end{longtable}
\end{center}
