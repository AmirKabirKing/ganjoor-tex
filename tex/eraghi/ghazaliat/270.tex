\begin{center}
\section*{غزل شماره ۲۷۰: لقد فاح الربیع و دار ساقی}
\label{sec:270}
\addcontentsline{toc}{section}{\nameref{sec:270}}
\begin{longtable}{l p{0.5cm} r}
لقد فاح الربیع و دار ساقی
&&
وهب نسیم روضات العراق
\\
صبا بوی عراق آورد گویی
&&
که خوش گشت از نسیم او عراقی
\\
الا یا حبذا! نفحات ارض
&&
جوی المشتاق یشفی باشتیاق
\\
دریغا! روزگار نوش بگذشت
&&
ندیمم بخت بود و یار ساقی
\\
بلیت ان صبحی بالبلایا
&&
الاق مرور ایام التلاقی
\\
ز جور روزگار ناموافق
&&
جدا گشتم ز یاران وفاقی
\\
ادر، یا ایها الساقی، ارحنی
&&
زمانا من خمار الافتراق
\\
دلم را شاد کن، ساقی، که نگذاشت
&&
جدایی بر من از غم هیچ باقی
\\
و عل لعل لطیفی نار قلبی
&&
و قلبی من تراکم فی احتراق
\\
بده جامی، که اندر وی ببینم
&&
جمال دوستان هم وثاقی
\\
جرعت من التفرق کل یوم
&&
و اجریت الدموع من الماقی
\\
بنال، ایدل، ز درد و غم که پیوست
&&
گرفتار غم و درد فراقی
\\
الا یا اهل العراق، تحذ قلبی
&&
الیکم و اشتمل من اشتیاقی
\\
عراقی، خوش بموی و زار بگری
&&
که در هندوستان از جفت طاقی
\\
\end{longtable}
\end{center}
