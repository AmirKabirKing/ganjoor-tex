\begin{center}
\section*{غزل شماره ۲۲۹: ای همه میل دل من سوی تو}
\label{sec:229}
\addcontentsline{toc}{section}{\nameref{sec:229}}
\begin{longtable}{l p{0.5cm} r}
ای همه میل دل من سوی تو
&&
قبلهٔ جان چشم تو و ابروی تو
\\
نرگس مستت ربوده عقل من
&&
برده خوابم نرگس جادوی تو
\\
بر سر میدان جانبازی دلم
&&
در خم چوگان ز زلف و گوی تو
\\
آمدم در کوی امید تو باز
&&
تا مگر بینم رخ نیکوی تو
\\
من جگر تفتیده بر خاک درت
&&
آب حیوان رایگان در جوی تو
\\
ای امید من، روا داری مگر؟
&&
باز گردم ناامید از کوی تو
\\
لطف کن، دست جفا بر من مدار
&&
من ندارم طاقت بازوی تو
\\
روزگاری بوده‌ام بر درگهت
&&
چشم امیدم بمانده سوی تو
\\
تا مگر بینم دمی رنگ رخت
&&
تا مگر یابم زمانی بوی تو
\\
چون ندیدم رنگ رویت، لاجرم
&&
مانده‌ام در درد بی‌داروی تو
\\
بر من مسکین عاجز رحم کن
&&
چون فروماندم ز جست و جوی تو
\\
در غم تو روزگارم شد دریغ!
&&
ناشده یک لحظه همزانوی تو
\\
هم مشام جانم آخر خوش شود
&&
از نسیم جان فزای موی تو
\\
خود عراقی جان شیرین کی دهد؟
&&
تا به کام دل نبیند روی تو
\\
\end{longtable}
\end{center}
