\begin{center}
\section*{غزل شماره ۲۰۰: تا کی از دست فراق تو ستم‌ها بینیم}
\label{sec:200}
\addcontentsline{toc}{section}{\nameref{sec:200}}
\begin{longtable}{l p{0.5cm} r}
تا کی از دست فراق تو ستم‌ها بینیم؟
&&
هیچ باشد که تو را بار دگر وابینیم
\\
دل دهیم، از سر زلف تو چو بویی یابیم
&&
جان فشانیم، اگر آن رخ زیبا بینیم
\\
روی خوب تو که هر دم دگران می‌بینند
&&
چه شود گر بگذاری تو دمی ما بینیم؟
\\
ما که دور از تو ز هجرانت به جان آمده‌ایم
&&
از فراق تو بگو: چند بلاها بینیم؟
\\
خورد زنگار غمت آینهٔ دل به فسوس
&&
نیست ممکن که جمال تو در آنجا بینیم
\\
گم شد آخر دل ما، بر در تو آمده‌ایم
&&
تا بود کان دل گم‌کردهٔ خود وابینیم
\\
گر بیابیم دلی، بر سر کویت یابیم
&&
ور ببینیم رخی، در دل بینا بینیم
\\
روی بنمای، که امروز ندیدیم رخت
&&
ای بسا حسرت و اندوه که فردا بینیم!
\\
روی زیبای تو، ای دوست، به کام دل خویش
&&
تا عراقی بنمیرد نه همانا بینیم
\\
\end{longtable}
\end{center}
