\begin{center}
\section*{غزل شماره ۲۹۰: پسرا، ره قلندر سزد ار به من نمایی}
\label{sec:290}
\addcontentsline{toc}{section}{\nameref{sec:290}}
\begin{longtable}{l p{0.5cm} r}
پسرا، ره قلندر سزد ار به من نمایی
&&
که دراز و دور دیدم ره زهد و پارسایی
\\
پسرا، می مغانه دهی ار حریف مایی
&&
که نماند بیش ما را سر زهد و پارسایی
\\
قدحی می مغانه به من آر، تا بنوشم
&&
که دگر نماند ما را سر توبهٔ ریایی
\\
می صاف اگر نباشد، به من آر درد تیره
&&
که ز درد تیره یابد دل و دیده روشنایی
\\
کم خانقه گرفتم، سر مصلحی ندارم
&&
قدح شراب پر کن، به من آر، چند پایی؟
\\
نه ره و نه رسم دارم، نه دل و نه دین، نه دنیی
&&
منم و حریف و کنجی و نوای بی‌نوایی
\\
نیم اهل زهد و توبه به من آر ساغر می
&&
که به صدق توبه کردم ز عبادت ریایی
\\
تو مرا شراب در ده، که ز زهد تو به کردم
&&
ز صلاح چون ندیدم جز لاف و خودنمایی
\\
ز غم زمانه ما را برهان ز می زمانی
&&
که نیافت جز به می کس ز غم زمان رهایی
\\
چو ز باده مست گشتم، چه کلیسیا، چه کعبه؟
&&
چو به ترک خود بگفتم، چه وصال و چه جدایی؟
\\
به قمارخانه رفتم همه پاکباز دیدم
&&
چو به صومعه رسیدم همه یافتم دغایی
\\
چو شکست توبهٔ من، مشکن تو عهد، باری
&&
به من شکسته دل گو که: چگونه‌ای؟ کجایی؟
\\
به طواف کعبه رفتم به حرم رهم ندادند
&&
که برون در چه کردی، که درون خانه آیی؟
\\
در دیر می‌زدم من، ز درون صدا بر آمد
&&
که: درآی، ای عراقی، که تو خود حریف مایی
\\
\end{longtable}
\end{center}
