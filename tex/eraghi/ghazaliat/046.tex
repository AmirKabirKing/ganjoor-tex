\begin{center}
\section*{غزل شماره ۴۶: عشق، شوری در نهاد ما نهاد}
\label{sec:046}
\addcontentsline{toc}{section}{\nameref{sec:046}}
\begin{longtable}{l p{0.5cm} r}
عشق، شوری در نهاد ما نهاد
&&
جان ما در بوتهٔ سودا نهاد
\\
گفتگویی در زبان ما فکند
&&
جستجویی در درون ما نهاد
\\
داستان دلبران آغاز کرد
&&
آرزویی در دل شیدا نهاد
\\
رمزی از اسرار باده کشف کرد
&&
راز مستان جمله بر صحرا نهاد
\\
قصهٔ خوبان به نوعی باز گفت
&&
کاتشی در پیر و در برنا نهاد
\\
از خمستان جرعه‌ای بر خاک ریخت
&&
جنبشی در آدم و حوا نهاد
\\
عقل مجنون در کف لیلی سپرد
&&
جان وامق در لب عذرا نهاد
\\
دم به دم در هر لباسی رخ نمود
&&
لحظه لحظه جای دیگر پا نهاد
\\
چون نبود او را معین خانه‌ای
&&
هر کجا جا دید، رخت آنجا نهاد
\\
بر مثال خویشتن حرفی نوشت
&&
نام آن حرف آدم و حوا نهاد
\\
حسن را بر دیدهٔ خود جلوه داد
&&
منتی بر عاشق شیدا نهاد
\\
هم به چشم خود جمال خود بدید
&&
تهمتی بر چشم نابینا نهاد
\\
یک کرشمه کرد با خود، آنچنانک:
&&
فتنه‌ای در پیر و در برنا نهاد
\\
کام فرهاد و مراد ما همه
&&
در لب شیرین شکرخا نهاد
\\
بهر آشوب دل سوداییان
&&
خال فتنه بر رخ زیبا نهاد
\\
وز پی برک و نوای بلبلان
&&
رنگ و بویی در گل رعنا نهاد
\\
تا تماشای وصال خود کند
&&
نور خود در دیدهٔ بینا نهاد
\\
تا کمال علم او ظاهر شود
&&
این همه اسرار بر صحرا نهاد
\\
شور و غوغایی برآمد از جهان
&&
حسن او چون دست در یغما نهاد
\\
چون در آن غوغا عراقی را بدید
&&
نام او سر دفتر غوغا نهاد
\\
\end{longtable}
\end{center}
