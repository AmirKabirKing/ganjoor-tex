\begin{center}
\section*{غزل شماره ۲۱۲: ماهرخان، که داد عشق، عارض لاله رنگشان}
\label{sec:212}
\addcontentsline{toc}{section}{\nameref{sec:212}}
\begin{longtable}{l p{0.5cm} r}
ماهرخان، که داد عشق، عارض لاله رنگشان
&&
هان! به حذر شوید از غمزهٔ شوخ و شنگشان
\\
نالهٔ زار عاشقان، اشک چو خون بی‌دلان
&&
هیچ اثر نمی‌کند در دل همچو سنگشان
\\
با دل ریش عاشقان، وه! که چها نمی‌کنند؟
&&
ابرو چون کمانشان، غمزهٔ چون خدنگشان
\\
از لب و زلف و خال و خط دانه و دام کرده‌اند
&&
تا که برین صفت بود، دل که برد ز چنگشان؟
\\
ما چو شکر گداخته، ز آب غم و عجبتر آنک:
&&
در دل ماست چو شکر غصهٔ چون شرنگشان
\\
بیش مپرس حال من، زآنکه به شرح می‌دهد
&&
از دل و دست ما نشان چشم و دهان تنگشان
\\
غم مخور، ای دل، ار بود یک دو دمی چو دور گل
&&
دولت بی‌ثباتشان، خوبی بی‌درنگشان
\\
ابر صفت مریز اشک، از پی هجر و وصلشان
&&
زان که چو برق بگذرد مدت صلح و جنگشان
\\
جان عراقی از جهان گشت ملول و بس حزین
&&
کاهوی او رمید از آن عادت چون پلنگشان
\\
\end{longtable}
\end{center}
