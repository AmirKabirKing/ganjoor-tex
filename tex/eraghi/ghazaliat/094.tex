\begin{center}
\section*{غزل شماره ۹۴: آن را که غمت ز در براند}
\label{sec:094}
\addcontentsline{toc}{section}{\nameref{sec:094}}
\begin{longtable}{l p{0.5cm} r}
آن را که غمت ز در براند
&&
بختش همه دربدر دواند
\\
وآن را که عنایت تو ره داد
&&
جز بر در تو رهی نداند
\\
وآن را که قبول عشقت افتاد
&&
جان را بدهد، غمت ستاند
\\
عاشق که گذر کند به کویت
&&
جان پیش سگ درت فشاند
\\
با وصل بگو که: عاشقان را
&&
از دست فراق وارهاند
\\
بیچاره دلم که کشتهٔ توست
&&
دور از رخ تو نمی‌تواند
\\
بویی به نسیم کوی خود ده
&&
تا صبحدمی به دل رساند
\\
کین مرده به بوت زنده گردد
&&
وز عشق رخت کفن دراند
\\
مگذار که خسته دل عراقی
&&
بی‌عشق تو عمر بگذراند
\\
\end{longtable}
\end{center}
