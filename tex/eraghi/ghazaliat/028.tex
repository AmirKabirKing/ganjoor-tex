\begin{center}
\section*{غزل شماره ۲۸: طرهٔ یار پریشان چه خوش است}
\label{sec:028}
\addcontentsline{toc}{section}{\nameref{sec:028}}
\begin{longtable}{l p{0.5cm} r}
طرهٔ یار پریشان چه خوش است
&&
قامت دوست خرامان چه خوش است
\\
خط خوش بر لب جانان چه نکوست
&&
سبزه و چشمهٔ حیوان چه خوش است
\\
از می عشق دلی مست و خراب
&&
همچو چشم خوش جانان چه خوش است
\\
در خرابات خراب افتاده
&&
عاشق بی سر و سامان چه خوش است
\\
آن دل شیفتهٔ ما بنگر
&&
در خم زلف پریشان چه خوش است
\\
یوسف گم شدهٔ ما را بین
&&
کاندر آن چاه زنخدان چه خوش است
\\
لذت عشق بتم از من پرس
&&
تو از آن بی‌خبری کان چه خوش است
\\
تو چه دانی که شکر خندهٔ او
&&
از دهان شکرستان چه خوش است؟
\\
چه شناسی که می و نقل بهم
&&
از لب آن بت خندان چه خوش است
\\
گر ببینی که به وقت مستی
&&
لب من بر لب جانان چه خوش است
\\
یار ساقی و عراقی باقی
&&
وه که این عیش بدینسان چه خوش است
\\
\end{longtable}
\end{center}
