\begin{center}
\section*{غزل شماره ۲۴: جانا، نظری، که دل فگار است}
\label{sec:024}
\addcontentsline{toc}{section}{\nameref{sec:024}}
\begin{longtable}{l p{0.5cm} r}
جانا، نظری، که دل فگار است
&&
بخشای، که خسته نیک زار است
\\
بشتاب، که جان به لب رسید است
&&
دریاب کنون، که وقت کار است
\\
رحم آر، که بی‌تو زندگانی
&&
از مرگ بتر هزار بار است
\\
دیری است که بر در قبول است
&&
بیچاره دلم ، که نیک خوار است
\\
نومید چگونه باز گردد؟
&&
از درگهت، آن کامیدوار است
\\
ناخورده دلم شراب وصلت
&&
از دردی هجر در خمار است
\\
مگذار به کام دشمن ، ای دوست
&&
بیچاره مرا ، که دوستدار است
\\
رسواش مکن به کام دشمن
&&
کو خود ز رخ تو شرمسار است
\\
خرم دل آن کسی، که او را
&&
اندوه و غم تو غمگسار است
\\
یادیش ازین و آن نیاید
&&
آن را که، چو تو نگار، یار است
\\
کار آن دارد، که بر در تو
&&
هر لحظه و هر دمیش بار است
\\
نی آنکه همیشه چون عراقی
&&
بر خاک درت چو خاک خوار است
\\
\end{longtable}
\end{center}
