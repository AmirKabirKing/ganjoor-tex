\begin{center}
\section*{غزل شماره ۲۲۵: ای حسن تو بی‌پایان، آخر چه جمال است این}
\label{sec:225}
\addcontentsline{toc}{section}{\nameref{sec:225}}
\begin{longtable}{l p{0.5cm} r}
ای حسن تو بی‌پایان، آخر چه جمال است این؟
&&
در وصف توام حیران، آخر چه کمال است این؟
\\
رویت چو شود پیدا ابدال شود شیدا
&&
ای حسن رخت زیبا، آخر چه جمال است این؟
\\
حسنت چو برون تازد، عالم سپر اندازد
&&
هستی همه در بازد، آخر چه جلال است این؟
\\
عشقت سپه انگیزد، خون دل ما ریزد
&&
زین قطره چه برخیزد؟ آخر چه قتال است این؟
\\
در دل چو کنی منزل، هم جان ببری هم دل
&&
از تو چه مرا حاصل؟ آخر چه وصال است این؟
\\
وصلت بتر از هجران، درد تو مرا درمان
&&
منع تو به از احسان، آخر چه نوال است این؟
\\
میدان دل ما تنگ، قدر تو فراخ آهنگ
&&
ای با دو جهان در جنگ، آخر چه محال است این؟
\\
از عکس رخ روشن، آیینه کنی گلشن
&&
ای مردم چشم من، آخر چه مثال است این؟
\\
عقل ار همه بنگارد، نقشت به خیال آرد،
&&
کی تاب رخت دارد؟ آخر چه خیال است این؟
\\
جان ار چه بسی کوشد، وز عشق تو بخروشد
&&
کی جام لبت نوشد؟ آخر چه محال است این؟
\\
زلف تو کمند افکند، و افکند دلم در بند
&&
در سلسله شد پابند، آخر چه عقال است این؟
\\
آن دل، که به کوی تو، می‌بود به بوی تو
&&
خون گشت ز خوی تو، آخر چه خصال است این؟
\\
با جان من مسکین، چه ناز کنی چندین؟
&&
حال دل من می‌بین، آخر چه دلال است این؟
\\
\end{longtable}
\end{center}
