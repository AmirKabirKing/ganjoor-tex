\begin{center}
\section*{غزل شماره ۲۷۱: آن جام طرب فزای ساقی}
\label{sec:271}
\addcontentsline{toc}{section}{\nameref{sec:271}}
\begin{longtable}{l p{0.5cm} r}
آن جام طرب فزای ساقی
&&
بنمود مرا لقای ساقی
\\
در حال چو جام سجده بر دم
&&
پیش رخ جان فزای ساقی
\\
ننهاده هنوز چون پیاله
&&
لب بر لب دلگشای ساقی
\\
ترسم که کند خرابیی باز
&&
چشم خوش دلربای ساقی
\\
پیوسته چو جام در دل آتش
&&
در سر هوس و هوای ساقی
\\
با چشم پر آب چون قنینه
&&
جان می‌دهم از برای ساقی
\\
باشد چو پیاله غرقه در خون
&&
چشمی که شد آشنای ساقی
\\
عمری است که می‌زنم در دل
&&
یعنی که در سرای ساقی
\\
باشد که رسد به گوش جانم
&&
از میکده مرحبای ساقی
\\
آیینهٔ سینه زنگ غم خورد
&&
کو صیقل غم زدای ساقی؟
\\
تا بستاند مرا ز من باز
&&
این است خود اقتضای ساقی
\\
باشد که شود دل عراقی
&&
چون جام جهان نمای ساقی
\\
\end{longtable}
\end{center}
