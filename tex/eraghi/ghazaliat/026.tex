\begin{center}
\section*{غزل شماره ۲۶: ساز طرب عشق که داند که چه ساز است}
\label{sec:026}
\addcontentsline{toc}{section}{\nameref{sec:026}}
\begin{longtable}{l p{0.5cm} r}
ساز طرب عشق که داند که چه ساز است؟
&&
کز زخمهٔ آن نه فلک اندر تک و تاز است
\\
آورد به یک زخمه، جهان را همه، در رقص
&&
خود جان و جهان نغمهٔ آن پرده‌نواز است
\\
عالم چو صدایی است ازین پرده، که داند
&&
کین راه چه پرده است و درین پرده چه راز است؟
\\
رازی است درین پرده، گر آن را بشناسی
&&
دانی که حقیقت ز چه دربند مجاز است؟
\\
معلوم کنی کز چه سبب خاطر محمود
&&
پیوسته پریشان سر زلف ایاز است؟
\\
محتاج نیاز دل عشاق چرا شد
&&
حسن رخ خوبان، که همه مایهٔ ناز است؟
\\
عشق است که هر دم به دگر رنگ برآید
&&
ناز است بجایی و یه یک جای نیاز است
\\
در صورت عاشق چو درآید همه سوزاست
&&
در کسوت معشوق چو آید همه ساز است
\\
زان شعله که از روی بتان حسن برافروخت
&&
قسم دل عشاق همه سوز و گداز است
\\
راهی است ره عشق، به غایت خوش و نزدیک
&&
هر ره که جزین است همه دور و دراز است
\\
مستی، که خراب ره عشق است، درین ره
&&
خواب خوش مستیش همه عین نماز است
\\
در صومعه چون راه ندادند مرا دوش
&&
رفتم به در میکده، دیدم که فراز است
\\
از میکده آواز برآمد که: عراقی
&&
در باز تو خود را، که در میکده باز است
\\
\end{longtable}
\end{center}
