\begin{center}
\section*{غزل شماره ۱۶: شوری ز شراب خانه برخاست}
\label{sec:016}
\addcontentsline{toc}{section}{\nameref{sec:016}}
\begin{longtable}{l p{0.5cm} r}
شوری ز شراب خانه برخاست
&&
برخاست غریوی از چپ و راست
\\
تا چشم بتم چه فتنه انگیخت؟
&&
کز هر طرفی هزار غوغاست
\\
تا جام لبش کدام می داد؟
&&
کز جرعه‌اش هر که هست شیداست
\\
ساقی، قدحی، که مست عشقم
&&
و آن باده هنوز در سر ماست
\\
آن نعرهٔ شور هم‌چنان هست
&&
وآن شیفتگی هنوز برجاست
\\
کارم، که چو زلف توست در هم
&&
بی‌قامت تو نمی‌شود راست
\\
مقصود تویی مرا ز هستی
&&
کز جام، غرض می مصفاست
\\
آیینهٔ روی توست جانم
&&
عکس رخ تو درو هویداست
\\
گل رنگ رخ تو دارد ، ارنه
&&
رنگ رخش از پی چه زیباست ؟
\\
ور سرو نه قامت تو دیده است
&&
او را کشش از چه سوی بالاست؟
\\
باغی است جهان، ز عکس رویت
&&
خرم دل آن که در تماشاست
\\
در باغ همه رخ تو بیند
&&
از هر ورق گل، آن که بیناست
\\
از عکس رخت دل عراقی
&&
گلزار و بهار و باغ و صحراست
\\
\end{longtable}
\end{center}
