\begin{center}
\section*{غزل شماره ۵۶: جانا، حدیث شوقت در داستان نگنجد}
\label{sec:056}
\addcontentsline{toc}{section}{\nameref{sec:056}}
\begin{longtable}{l p{0.5cm} r}
جانا، حدیث شوقت در داستان نگنجد
&&
رمزی ز راز عشقت در صد بیان نگنجد
\\
جولانگه جلالت در کوی دل نباشد
&&
خلوتگه جمالت در جسم و جان نگنجد
\\
سودای زلف و خالت جز در خیال ناید
&&
اندیشهٔ وصالت جز در گمان نگنجد
\\
در دل چو عشقت آید، سودای جان نماند
&&
در جان چو مهرت افتد، عشق روان نگنجد
\\
دل کز تو بوی یابد، در گلستان نپوید
&&
جان کز تو رنگ بیند، اندر جهان نگنجد
\\
پیغام خستگانت در کوی تو که آرد؟
&&
کانجا ز عاشقانت باد وزان نگنجد
\\
آن دم که عاشقان را نزد تو بار باشد
&&
مسکین کسی که آنجا در آستان نگنجد
\\
بخشای بر غریبی کز عشق تو بمیرد
&&
وآنگه در آستانت خود یک زمان نگنجد
\\
جان داد دل که روزی کوی تو جای یابد
&&
نشناخت او که آخر جایی چنان نگنجد
\\
آن دم که با خیالت دل راز عشق گوید
&&
گر جان شود عراقی، اندر میان نگنجد
\\
\end{longtable}
\end{center}
