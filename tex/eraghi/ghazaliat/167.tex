\begin{center}
\section*{غزل شماره ۱۶۷: اگر فرصت دهد، جانا، فراقت روزکی چندم}
\label{sec:167}
\addcontentsline{toc}{section}{\nameref{sec:167}}
\begin{longtable}{l p{0.5cm} r}
اگر فرصت دهد، جانا، فراقت روزکی چندم
&&
زمانی با تو بنشینم، دمی در روی تو خندم
\\
درآ شاد از درم خندان که در پایت فشانم جان
&&
مدارم بیش ازین گریان، بیا، کت آرزومندم
\\
چو با خود خوش نمی‌باشم، بیا ، تا با تو خوش باشم
&&
چو مهر از خویش ببریدم، بیا، تا با تو پیوندم
\\
نیابی نزد مهجوران، نپرسی حال رنجوران
&&
بیا، زان پیش کز عالم بکلی رخت بربندم
\\
بیا کز عشق روی تو شبی خون جگر خوردم
&&
میزار از من بی‌دل، که سر در پایت افکندم
\\
مرا خوش دار، چون خود را به فتراک تو بر بستم
&&
بیا، کز آرزوی تو دمی صد بار جان کندم
\\
ز لفظ دلربای تو به یک گفتار خوشنودم
&&
ز وصل جان‌فزای تو به یک دیدار خرسندم
\\
وصالت، ای ز جان خوشتر، بیابم عاقبت روزی
&&
ولی ار زنده بگذارد فراقت روزکی چندم
\\
وطن گاه دل خود را به جز روی تو نگزینم
&&
تماشاگاه جسم و جان به جز روی تو نپسندم
\\
ز هستی عراقی هست بر پای دلم بندی
&&
جمال خوب خود بنما، گشادی ده ازین بندم
\\
\end{longtable}
\end{center}
