\begin{center}
\section*{غزل شماره ۱۵۴: خوشتر از خلد برین آراستند ایوان دل}
\label{sec:154}
\addcontentsline{toc}{section}{\nameref{sec:154}}
\begin{longtable}{l p{0.5cm} r}
خوشتر از خلد برین آراستند ایوان دل
&&
تا به شادی مجلس آراید درو سلطان دل
\\
هم ز حسن خود پدید آرد بهشت آباد جان
&&
هم به روی خود برآراید نگارستان دل
\\
در سرای دل چو سلطان حقیقت بار داد
&&
صف زدند ارواح عالم گرد شادروان دل
\\
جسم چبود؟ پرده‌ای پرنقش بر درگاه جان
&&
جان چه باشد؟ پرده‌داری بر در جانان دل
\\
عقل هر دم نامه‌ای دیگر نویسد نزد جان
&&
تا بود فرمان نویسی در بر دیوان دل
\\
مرغ همت برتر از فردوس اعلی زان پرد
&&
تا مگر یابد نسیم روضهٔ رضوان دل
\\
حسن بی‌پایان دل گرد جهان ظاهر شود
&&
هر که را چشمی بود باشد چو جان حیران دل
\\
خضر جان گرد سرابستان دل گردد مدام
&&
تا خورد آب حیات از چشمهٔ حیوان دل
\\
سر بر آر از جیب وحدت، تا ببینی آشکار
&&
صدرهٔ نه توی عالم کوته از دامان دل
\\
ظاهر و باطن نگه کن، اول و آخر ببین
&&
تا تو را روشن شود کز چیست چار ارکان دل
\\
طاق ایوانش خم ابروی جانان من است
&&
قبلهٔ جان من آمد زین قبل ایوان دل
\\
تا به رنگ خود برآرد هر که یابد در جهان
&&
شعله‌ای هر دم برافروزد رخ تابان دل
\\
چون نگار من به هر رنگی بر آید هر زمان
&&
لاجرم هر دم دگرگون می‌شود الوان دل
\\
خود دو عالم در محیط دل کم از یک شبنم است
&&
کی پدید آید نمی در بحر بی‌پایان دل؟
\\
از بهشت و زینت او در جهان رنگی بود
&&
کان بهشت آراستند، اعنی سرابستان دل
\\
بر بساط دل سماط عیش گستردند، لیک
&&
در جهان صاحبدلی کو تا شود مهمان دل؟
\\
حیف نبود در جهان خوانی چنین آراسته
&&
وانگهی ما بیخبر از حسن و از احسان دل؟
\\
از ثنای دل عراقی عاجز آمد بهر آنک
&&
هر کمالی کان بیندیشد بود نقصان دل
\\
\end{longtable}
\end{center}
