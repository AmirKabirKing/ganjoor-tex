\begin{center}
\section*{غزل شماره ۱۲۵: نیم چون یک نفس بی غم دلم خون خوار اولی‌تر}
\label{sec:125}
\addcontentsline{toc}{section}{\nameref{sec:125}}
\begin{longtable}{l p{0.5cm} r}
نیم چون یک نفس بی غم دلم خون خوار اولی‌تر
&&
ندارم چون دلی خرم، تنی بیمار اولی‌تر
\\
نیابد هر که دلداری، چو من زار و حزین اولی
&&
نبیند هر که غمخواری، چو من غمخوار اولی‌تر
\\
دلی کز یار خود بویی نیابد تن دهد بر باد
&&
چنین دل در کف هجران اسیر و زار اولی‌تر
\\
وصال او نمی‌یابم، تن اندر هجر او دارم
&&
به شادی چون نیم لایق، مرا تیمار اولی‌تر
\\
چو درد او بود درمان، تن من ناتوان خوشتر
&&
چو زخم او شود مرهم، دلم افگار اولی‌تر
\\
چو روزی من از وصلش همه تیمار و غم باشد
&&
به هر حالی مرا درد و غم بسیار اولی‌تر
\\
دلا، چون عاشق یاری، به درد او گرفتاری
&&
همی کن ناله و زاری، که عاشق زار اولی‌تر
\\
هر آنچه آرزو داری برو از درگه او خواه
&&
ز هر در، کان زند مفلس، در دلدار اولی‌تر
\\
عراقی، در رخ خوبان جمال یار خود می‌بین
&&
نظر چون می‌کنی باری به روی یار اولی‌تر
\\
\end{longtable}
\end{center}
