\begin{center}
\section*{غزل شماره ۴: کشیدم رنج بسیاری دریغا}
\label{sec:004}
\addcontentsline{toc}{section}{\nameref{sec:004}}
\begin{longtable}{l p{0.5cm} r}
کشیدم رنج بسیاری دریغا
&&
به کام من نشد کاری دریغا
\\
به عالم، در که دیدم باز کردم
&&
ندیدم روی دلداری دریغا
\\
شدم نومید کاندر چشم امید
&&
نیامد خوب رخساری دریغا
\\
ندیدم هیچ گلزاری به عالم
&&
که در چشمم نزد خاری دریغا
\\
مرا یاری است کز من یاد نارد
&&
که دارد این چنین یاری؟ دریغا
\\
دل بیمار من بیند نپرسد
&&
که چون شد حال بیماری؟ دریغا
\\
شدم صدبار بر درگاه وصلش
&&
ندادم بار یک باری دریغا
\\
ز اندوه فراقش بر دل من
&&
رسد هر لحظه تیماری دریغا
\\
به سر شد روزگارم بی‌رخ تو
&&
نماند از عمر بسیاری دریغا
\\
نپرسد از عراقی، تا بمیرد
&&
جهان گوید که: مرد، آری دریغا
\\
\end{longtable}
\end{center}
