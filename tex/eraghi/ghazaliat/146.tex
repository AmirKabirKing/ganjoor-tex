\begin{center}
\section*{غزل شماره ۱۴۶: بیا، که خانهٔ دل پاک کردم از خاشاک}
\label{sec:146}
\addcontentsline{toc}{section}{\nameref{sec:146}}
\begin{longtable}{l p{0.5cm} r}
بیا، که خانهٔ دل پاک کردم از خاشاک
&&
درین خرابه تو خود کی قدم نهی؟ حاشاک
\\
به لطف صید کنی صدهزار دل هر دم
&&
ولی نگاه نداری تو خود دل غمناک
\\
کدام دل که به خون در نمی‌کشد دامن؟
&&
کدام جان که نکرد از غمت گریبان چاک؟
\\
دل مرا، که به هر حال صید لاغر توست
&&
چو می کشیش، میفگن، ببند بر فتراک
\\
کنون اگر نرسی، کی رسی به فریادم؟
&&
مرا که جان به لب آمد کجا برم تریاک؟
\\
دلم که آینه‌ای شد، چرا نمی‌تابد
&&
درو رخ تو؟ همانا که نیست آینه پاک
\\
چو آفتاب بهر ذره می‌نماید رخ
&&
ولیک چشم عراقی نمی‌کند ادراک
\\
\end{longtable}
\end{center}
