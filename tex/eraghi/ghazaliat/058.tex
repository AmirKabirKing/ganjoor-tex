\begin{center}
\section*{غزل شماره ۵۸: امروز مرا در دل جز یار نمی‌گنجد}
\label{sec:058}
\addcontentsline{toc}{section}{\nameref{sec:058}}
\begin{longtable}{l p{0.5cm} r}
امروز مرا در دل جز یار نمی‌گنجد
&&
تنگ است، از آن در وی اغیار نمی‌گنجد
\\
در دیدهٔ پر آبم جز یار نمی‌آید
&&
وندر دلم از مستی جز یار نمی‌گنجد
\\
با این همه هم شادم کاندر دل تنگ من
&&
غم چاره نمی‌یابد، تیمار نمی‌گنجد
\\
جان در تنم ار بی‌دوست هربار نمی‌گنجد
&&
از غایت تنگ آمد کین بار نمی‌گنجد
\\
کو جام می عشقش؟ تا مست شوم زیراک:
&&
در بزم وصال او هشیار نمی‌گنجد
\\
کو دام سر زلفش؟ تا صید کند دل را
&&
کاندر خم زلف او دلدار نمی‌گنجد
\\
چون طره برافشاند این روی بپوشاند
&&
جایی که یقین آید پندار نمی‌گنجد
\\
عشقش چو درون تازد جان حجره بپردازد
&&
آنجا که وطن سازد دیار نمی‌گنجد
\\
این قطرهٔ خون تا یافت از خاک درش بویی
&&
از شادی آن در پوست چون نار نمی‌گنجد
\\
غم گرچه خورد جانم، هم غم نخورم زیراک:
&&
اندر حرم جانان غمخوار نمی‌گنجد
\\
تحفه بر دل بردم جان و تن و دین و هوش
&&
دل گفت: برو، کانجا هر چار نمی‌گنجد
\\
خواهی که درآیی تو، بگذار عراقی را
&&
کاندر حرم جانان جز یار نمی‌گنجد
\\
\end{longtable}
\end{center}
