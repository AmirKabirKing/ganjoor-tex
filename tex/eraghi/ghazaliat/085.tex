\begin{center}
\section*{غزل شماره ۸۵: دیدهٔ بختم، دریغا کور شد}
\label{sec:085}
\addcontentsline{toc}{section}{\nameref{sec:085}}
\begin{longtable}{l p{0.5cm} r}
دیدهٔ بختم، دریغا کور شد
&&
دل نمرده، زنده اندر گور شد
\\
دست گیر ای دوست این بخت مرا
&&
تا نبیند دشمنم کو کور شد
\\
بارگاه دل، که بودی جای تو
&&
بنگر اکنون جای مار و مور شد
\\
بی‌لب شیرینت عمرم تلخ گشت
&&
شوربختی بین که: عیشم شور شد
\\
دل قوی بودم به امید تو، لیک
&&
دل ندادی، خسته زان بی‌نور شد
\\
شور عشقت تا فتاد اندر جهان
&&
چون دل من عالمی پر شور شد
\\
عارت آمد از عراقی، لاجرم
&&
بی‌تو، مسکین، بی‌نوا و عور شد
\\
\end{longtable}
\end{center}
