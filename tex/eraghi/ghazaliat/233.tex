\begin{center}
\section*{غزل شماره ۲۳۳: مانا دمید بوی گلستان صبح گاه}
\label{sec:233}
\addcontentsline{toc}{section}{\nameref{sec:233}}
\begin{longtable}{l p{0.5cm} r}
مانا دمید بوی گلستان صبح گاه
&&
کاواز داد مرغ خوش‌الحان صبحگاه
\\
خوش نغمه‌ای است نغمهٔ مرغان صبح دم
&&
خوش نعره‌ای است نعرهٔ مستان صبحگاه
\\
وقتی خوش است و مرغ دل ار نغمه‌ای زند
&&
زیبد، که باز شد در بستان صبحگاه
\\
از صد نسیم گلشن فردوس خوشتر است
&&
بادی که می‌وزد ز گلستان صبحگاه
\\
در خلد هرچه نسیه تو را وعده داده‌اند
&&
نقد است این دم آنهمه بر خوان صبحگاه
\\
خوش مجلسی است: درد ندیم و دریغ یار
&&
غم میزبان و ما همه مهمان صبحگاه
\\
جانا، بخور ساز درین بزم، تا مگر
&&
خوشبو نشد نسیم گلستان صبحگاه
\\
تا ز آتش فراق دل عاشقی نسوخت
&&
خوشبو کند بخور تو ایوان صبحگاه
\\
خواهی چو صبح سر ز گریبان برآوری
&&
کوته مکن دو دست ز دامان صبحگاه
\\
باشد که قلب ناسرهٔ تو سره شود
&&
می‌سنج نقد خویش به میزان صبحگاه
\\
دامان صبح گیر، مگر سر برآورد
&&
صبح امید تو ز گریبان صبحگاه
\\
چون دانه‌ای، دل تو که چون جوز غم شده است
&&
انداز پیش مرغ خوش الحان صبحگاه
\\
شب خفته ماند بخت عراقی، از آن سبب
&&
محروم شد ز روح فراوان صبحگاه
\\
\end{longtable}
\end{center}
