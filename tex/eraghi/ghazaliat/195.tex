\begin{center}
\section*{غزل شماره ۱۹۵: گر چه دل خون کنی از خاک درت نگریزیم}
\label{sec:195}
\addcontentsline{toc}{section}{\nameref{sec:195}}
\begin{longtable}{l p{0.5cm} r}
گر چه دل خون کنی از خاک درت نگریزیم
&&
جز تو فریادرسی کو که درو آویزیم؟
\\
گذری کن، که مگر با تو دمی بنشینیم
&&
نظری کن که خوشی از سر و جان برخیزیم
\\
مشت خاکیم به خون جگر آغشته همه
&&
از چنین خاک درین راه چه گرد انگیزیم؟
\\
هم بسوزیم ز تاب رخ تو ناگاهی
&&
همچو پروانه ز شمع ارچه بسی پرهیزیم
\\
بیم آن است که در خون جگر غرق شویم
&&
بسکه بر خاک درت خون جگر می‌ریزیم
\\
تا دل گمشده را بر سر کویت یابیم
&&
همه شب تا به سحر خاک درت می‌بیزیم
\\
نیک و بد زان توایم، با دگریمان مگذار
&&
با تو آمیخته‌ایم، با دگری نامیزیم
\\
راه ده باز، که نزد تو پناه آوردیم
&&
بو که از دست عراقی نفسی بگریزیم
\\
\end{longtable}
\end{center}
