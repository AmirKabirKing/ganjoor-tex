\begin{center}
\section*{غزل شماره ۳۶۱: آن که پامال جفا کرد چو خاک راهم}
\label{sec:sh361}
\addcontentsline{toc}{section}{\nameref{sec:sh361}}
\begin{longtable}{l p{0.5cm} r}
آن که پامال جفا کرد چو خاک راهم
&&
خاک می‌بوسم و عذر قدمش می‌خواهم
\\
من نه آنم که ز جور تو بنالم حاشا
&&
بنده معتقد و چاکر دولتخواهم
\\
بسته‌ام در خم گیسوی تو امید دراز
&&
آن مبادا که کند دست طلب کوتاهم
\\
ذره خاکم و در کوی توام جای خوش است
&&
ترسم ای دوست که بادی ببرد ناگاهم
\\
پیر میخانه سحر جام جهان بینم داد
&&
و اندر آن آینه از حسن تو کرد آگاهم
\\
صوفی صومعه عالم قدسم لیکن
&&
حالیا دیر مغان است حوالتگاهم
\\
با من راه نشین خیز و سوی میکده آی
&&
تا در آن حلقه ببینی که چه صاحب جاهم
\\
مست بگذشتی و از حافظت اندیشه نبود
&&
آه اگر دامن حسن تو بگیرد آهم
\\
خوشم آمد که سحر خسرو خاور می‌گفت
&&
با همه پادشهی بنده تورانشاهم
\\
\end{longtable}
\end{center}
