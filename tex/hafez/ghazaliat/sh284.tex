\begin{center}
\section*{غزل شماره ۲۸۴: هاتفی از گوشه میخانه دوش}
\label{sec:sh284}
\addcontentsline{toc}{section}{\nameref{sec:sh284}}
\begin{longtable}{l p{0.5cm} r}
هاتفی از گوشه میخانه دوش
&&
گفت ببخشند گنه می بنوش
\\
لطف الهی بکند کار خویش
&&
مژده رحمت برساند سروش
\\
این خرد خام به میخانه بر
&&
تا می لعل آوردش خون به جوش
\\
گر چه وصالش نه به کوشش دهند
&&
هر قدر ای دل که توانی بکوش
\\
لطف خدا بیشتر از جرم ماست
&&
نکته سربسته چه دانی خموش
\\
گوش من و حلقه گیسوی یار
&&
روی من و خاک در می فروش
\\
رندی حافظ نه گناهیست صعب
&&
با کرم پادشه عیب پوش
\\
داور دین شاه شجاع آن که کرد
&&
روح قدس حلقه امرش به گوش
\\
ای ملک العرش مرادش بده
&&
و از خطر چشم بدش دار گوش
\\
\end{longtable}
\end{center}
