\begin{center}
\section*{غزل شماره ۴۶۳: سلام الله ما کر اللیالی}
\label{sec:sh463}
\addcontentsline{toc}{section}{\nameref{sec:sh463}}
\begin{longtable}{l p{0.5cm} r}
سلام الله ما کر اللیالی
&&
و جاوبت المثانی و المثالی
\\
علی وادی الاراک و من علیها
&&
و دار باللوی فوق الرمال
\\
دعاگوی غریبان جهانم
&&
و ادعو بالتواتر و التوالی
\\
به هر منزل که رو آرد خدا را
&&
نگه دارش به لطف لایزالی
\\
منال ای دل که در زنجیر زلفش
&&
همه جمعیت است آشفته حالی
\\
ز خطت صد جمال دیگر افزود
&&
که عمرت باد صد سال جلالی
\\
تو می‌باید که باشی ور نه سهل است
&&
زیان مایه جاهی و مالی
\\
بر آن نقاش قدرت آفرین باد
&&
که گرد مه کشد خط هلالی
\\
فحبک راحتی فی کل حین
&&
و ذکرک مونسی فی کل حال
\\
سویدای دل من تا قیامت
&&
مباد از شوق و سودای تو خالی
\\
کجا یابم وصال چون تو شاهی
&&
من بدنام رند لاابالی
\\
خدا داند که حافظ را غرض چیست
&&
و علم الله حسبی من سؤالی
\\
\end{longtable}
\end{center}
