\begin{center}
\section*{غزل شماره ۸۷: حسنت به اتفاق ملاحت جهان گرفت}
\label{sec:sh087}
\addcontentsline{toc}{section}{\nameref{sec:sh087}}
\begin{longtable}{l p{0.5cm} r}
حسنت به اتفاق ملاحت جهان گرفت
&&
آری به اتفاق جهان می‌توان گرفت
\\
افشای راز خلوتیان خواست کرد شمع
&&
شکر خدا که سر دلش در زبان گرفت
\\
زین آتش نهفته که در سینه من است
&&
خورشید شعله‌ایست که در آسمان گرفت
\\
می‌خواست گل که دم زند از رنگ و بوی دوست
&&
از غیرت صبا نفسش در دهان گرفت
\\
آسوده بر کنار چو پرگار می‌شدم
&&
دوران چو نقطه عاقبتم در میان گرفت
\\
آن روز شوق ساغر می خرمنم بسوخت
&&
کاتش ز عکس عارض ساقی در آن گرفت
\\
خواهم شدن به کوی مغان آستین فشان
&&
زین فتنه‌ها که دامن آخرزمان گرفت
\\
می خور که هر که آخر کار جهان بدید
&&
از غم سبک برآمد و رطل گران گرفت
\\
بر برگ گل به خون شقایق نوشته‌اند
&&
کان کس که پخته شد می چون ارغوان گرفت
\\
حافظ چو آب لطف ز نظم تو می‌چکد
&&
حاسد چگونه نکته تواند بر آن گرفت
\\
\end{longtable}
\end{center}
