\begin{center}
\section*{غزل شماره ۲۱۳: گوهر مخزن اسرار همان است که بود}
\label{sec:sh213}
\addcontentsline{toc}{section}{\nameref{sec:sh213}}
\begin{longtable}{l p{0.5cm} r}
گوهر مخزن اسرار همان است که بود
&&
حقه مهر بدان مهر و نشان است که بود
\\
عاشقان زمره ارباب امانت باشند
&&
لاجرم چشم گهربار همان است که بود
\\
از صبا پرس که ما را همه شب تا دم صبح
&&
بوی زلف تو همان مونس جان است که بود
\\
طالب لعل و گهر نیست وگرنه خورشید
&&
همچنان در عمل معدن و کان است که بود
\\
کشته غمزه خود را به زیارت دریاب
&&
زان که بیچاره همان دل‌نگران است که بود
\\
رنگ خون دل ما را که نهان می‌داری
&&
همچنان در لب لعل تو عیان است که بود
\\
زلف هندوی تو گفتم که دگر ره نزند
&&
سال‌ها رفت و بدان سیرت و سان است که بود
\\
حافظا بازنما قصه خونابه چشم
&&
که بر این چشمه همان آب روان است که بود
\\
\end{longtable}
\end{center}
