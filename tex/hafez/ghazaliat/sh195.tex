\begin{center}
\section*{غزل شماره ۱۹۵: غلام نرگس مست تو تاجدارانند}
\label{sec:sh195}
\addcontentsline{toc}{section}{\nameref{sec:sh195}}
\begin{longtable}{l p{0.5cm} r}
غلام نرگس مست تو تاجدارانند
&&
خراب باده لعل تو هوشیارانند
\\
تو را صبا و مرا آب دیده شد غماز
&&
و گر نه عاشق و معشوق رازدارانند
\\
ز زیر زلف دوتا چون گذر کنی بنگر
&&
که از یمین و یسارت چه سوگوارانند
\\
گذار کن چو صبا بر بنفشه زار و ببین
&&
که از تطاول زلفت چه بی‌قرارانند
\\
نصیب ماست بهشت ای خداشناس برو
&&
که مستحق کرامت گناهکارانند
\\
نه من بر آن گل عارض غزل سرایم و بس
&&
که عندلیب تو از هر طرف هزارانند
\\
تو دستگیر شو ای خضر پی خجسته که من
&&
پیاده می‌روم و همرهان سوارانند
\\
بیا به میکده و چهره ارغوانی کن
&&
مرو به صومعه کان جا سیاه کارانند
\\
خلاص حافظ از آن زلف تابدار مباد
&&
که بستگان کمند تو رستگارانند
\\
\end{longtable}
\end{center}
