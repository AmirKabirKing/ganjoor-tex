\begin{center}
\section*{غزل شماره ۴۵۵: عمر بگذشت به بی‌حاصلی و بوالهوسی}
\label{sec:sh455}
\addcontentsline{toc}{section}{\nameref{sec:sh455}}
\begin{longtable}{l p{0.5cm} r}
عمر بگذشت به بی‌حاصلی و بوالهوسی
&&
ای پسر جام می‌ام ده که به پیری برسی
\\
چه شکرهاست در این شهر که قانع شده‌اند
&&
شاهبازان طریقت به مقام مگسی
\\
دوش در خیل غلامان درش می‌رفتم
&&
گفت ای عاشق بیچاره تو باری چه کسی
\\
با دل خون شده چون نافه خوشش باید بود
&&
هر که مشهور جهان گشت به مشکین نفسی
\\
لمع البرق من الطور و آنست به
&&
فلعلی لک آت بشهاب قبس
\\
کاروان رفت و تو در خواب و بیابان در پیش
&&
وه که بس بی‌خبر از غلغل چندین جرسی
\\
بال بگشا و صفیر از شجر طوبی زن
&&
حیف باشد چو تو مرغی که اسیر قفسی
\\
تا چو مجمر نفسی دامن جانان گیرم
&&
جان نهادیم بر آتش ز پی خوش نفسی
\\
چند پوید به هوای تو ز هر سو حافظ
&&
یسر الله طریقا بک یا ملتمسی
\\
\end{longtable}
\end{center}
