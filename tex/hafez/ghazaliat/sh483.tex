\begin{center}
\section*{غزل شماره ۴۸۳: سحرگه ره روی در سرزمینی}
\label{sec:sh483}
\addcontentsline{toc}{section}{\nameref{sec:sh483}}
\begin{longtable}{l p{0.5cm} r}
سحرگه ره روی در سرزمینی
&&
همی‌گفت این معما با قرینی
\\
که ای صوفی شراب آن گه شود صاف
&&
که در شیشه برآرد  اربعینی
\\
خدا زان خرقه بیزار است صد بار
&&
که صد بت باشدش در آستینی
\\
مروت گر چه نامی بی‌نشان است
&&
نیازی عرضه کن بر نازنینی
\\
ثوابت باشد ای دارای خرمن
&&
اگر رحمی کنی بر خوشه چینی
\\
نمی‌بینم نشاط عیش در کس
&&
نه درمان دلی نه درد دینی
\\
درون‌ها تیره شد باشد که از غیب
&&
چراغی برکند خلوت نشینی
\\
گر انگشت سلیمانی نباشد
&&
چه خاصیت دهد نقش نگینی
\\
اگر چه رسم خوبان تندخوییست
&&
چه باشد گر بسازد با غمینی
\\
ره میخانه بنما تا بپرسم
&&
مآل خویش را از پیش بینی
\\
نه حافظ را حضور درس خلوت
&&
نه دانشمند را علم الیقینی
\\
\end{longtable}
\end{center}
