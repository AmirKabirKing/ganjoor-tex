\begin{center}
\section*{غزل شماره ۳۷۶: دوستان وقت گل آن به که به عشرت کوشیم}
\label{sec:sh376}
\addcontentsline{toc}{section}{\nameref{sec:sh376}}
\begin{longtable}{l p{0.5cm} r}
دوستان وقت گل آن به که به عشرت کوشیم
&&
سخن اهل دل است این و به جان بنیوشیم
\\
نیست در کس کرم و وقت طرب می‌گذرد
&&
چاره آن است که سجاده به می بفروشیم
\\
خوش هواییست فرح بخش خدایا بفرست
&&
نازنینی که به رویش می گلگون نوشیم
\\
ارغنون ساز فلک رهزن اهل هنر است
&&
چون از این غصه ننالیم و چرا نخروشیم
\\
گل به جوش آمد و از می نزدیمش آبی
&&
لاجرم ز آتش حرمان و هوس می‌جوشیم
\\
می‌کشیم از قدح لاله شرابی موهوم
&&
چشم بد دور که بی مطرب و می مدهوشیم
\\
حافظ این حال عجب با که توان گفت که ما
&&
بلبلانیم که در موسم گل خاموشیم
\\
\end{longtable}
\end{center}
