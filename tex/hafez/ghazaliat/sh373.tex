\begin{center}
\section*{غزل شماره ۳۷۳: خیز تا خرقه صوفی به خرابات بریم}
\label{sec:sh373}
\addcontentsline{toc}{section}{\nameref{sec:sh373}}
\begin{longtable}{l p{0.5cm} r}
خیز تا خرقه صوفی به خرابات بریم
&&
شطح و طامات به بازار خرافات بریم
\\
سوی رندان قلندر به ره آورد سفر
&&
دلق بسطامی و سجاده طامات بریم
\\
تا همه خلوتیان جام صبوحی گیرند
&&
چنگ صبحی به در پیر مناجات بریم
\\
با تو آن عهد که در وادی ایمن بستیم
&&
همچو موسی ارنی گوی به میقات بریم
\\
کوس ناموس تو بر کنگره عرش زنیم
&&
علم عشق تو بر بام سماوات بریم
\\
خاک کوی تو به صحرای قیامت فردا
&&
همه بر فرق سر از بهر مباهات بریم
\\
ور نهد در ره ما خار ملامت زاهد
&&
از گلستانش به زندان مکافات بریم
\\
شرممان باد ز پشمینه آلوده خویش
&&
گر بدین فضل و هنر نام کرامات بریم
\\
قدر وقت ار نشناسد دل و کاری نکند
&&
بس خجالت که از این حاصل اوقات بریم
\\
فتنه می‌بارد از این سقف مقرنس برخیز
&&
تا به میخانه پناه از همه آفات بریم
\\
در بیابان فنا گم شدن آخر تا کی
&&
ره بپرسیم مگر پی به مهمات بریم
\\
حافظ آب رخ خود بر در هر سفله مریز
&&
حاجت آن به که بر قاضی حاجات بریم
\\
\end{longtable}
\end{center}
