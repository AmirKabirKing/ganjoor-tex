\begin{center}
\section*{غزل شماره ۱۲۴: آن که از سنبل او غالیه تابی دارد}
\label{sec:sh124}
\addcontentsline{toc}{section}{\nameref{sec:sh124}}
\begin{longtable}{l p{0.5cm} r}
آن که از سنبل او غالیه تابی دارد
&&
باز با دلشدگان ناز و عتابی دارد
\\
از سر کشته خود می‌گذری همچون باد
&&
چه توان کرد که عمر است و شتابی دارد
\\
ماه خورشید نمایش ز پس پرده زلف
&&
آفتابیست که در پیش سحابی دارد
\\
چشم من کرد به هر گوشه روان سیل سرشک
&&
تا سهی سرو تو را تازه‌تر آبی دارد
\\
غمزه شوخ تو خونم به خطا می‌ریزد
&&
فرصتش باد که خوش فکر صوابی دارد
\\
آب حیوان اگر این است که دارد لب دوست
&&
روشن است این که خضر بهره سرابی دارد
\\
چشم مخمور تو دارد ز دلم قصد جگر
&&
ترک مست است مگر میل کبابی دارد
\\
جان بیمار مرا نیست ز تو روی سؤال
&&
ای خوش آن خسته که از دوست جوابی دارد
\\
کی کند سوی دل خسته حافظ نظری
&&
چشم مستش که به هر گوشه خرابی دارد
\\
\end{longtable}
\end{center}
