\begin{center}
\section*{غزل شماره ۱۴۵: چه مستیست ندانم که رو به ما آورد}
\label{sec:sh145}
\addcontentsline{toc}{section}{\nameref{sec:sh145}}
\begin{longtable}{l p{0.5cm} r}
چه مستیست ندانم که رو به ما آورد
&&
که بود ساقی و این باده از کجا آورد
\\
تو نیز باده به چنگ آر و راه صحرا گیر
&&
که مرغ نغمه سرا ساز خوش نوا آورد
\\
دلا چو غنچه شکایت ز کار بسته مکن
&&
که باد صبح نسیم گره گشا آورد
\\
رسیدن گل و نسرین به خیر و خوبی باد
&&
بنفشه شاد و کش آمد سمن صفا آورد
\\
صبا به خوش خبری هدهد سلیمان است
&&
که مژده طرب از گلشن سبا آورد
\\
علاج ضعف دل ما کرشمه ساقیست
&&
برآر سر که طبیب آمد و دوا آورد
\\
مرید پیر مغانم ز من مرنج ای شیخ
&&
چرا که وعده تو کردی و او به جا آورد
\\
به تنگ چشمی آن ترک لشکری نازم
&&
که حمله بر من درویش یک قبا آورد
\\
فلک غلامی حافظ کنون به طوع کند
&&
که التجا به در دولت شما آورد
\\
\end{longtable}
\end{center}
