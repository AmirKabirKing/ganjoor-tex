\begin{center}
\section*{غزل شماره ۱۹۶: آنان که خاک را به نظر کیمیا کنند}
\label{sec:sh196}
\addcontentsline{toc}{section}{\nameref{sec:sh196}}
\begin{longtable}{l p{0.5cm} r}
آنان که خاک را به نظر کیمیا کنند
&&
آیا بود که گوشه چشمی به ما کنند
\\
دردم نهفته به ز طبیبان مدعی
&&
باشد که از خزانه غیبم دوا کنند
\\
معشوق چون نقاب ز رخ در نمی‌کشد
&&
هر کس حکایتی به تصور چرا کنند
\\
چون حسن عاقبت نه به رندی و زاهدیست
&&
آن به که کار خود به عنایت رها کنند
\\
بی معرفت مباش که در من یزید عشق
&&
اهل نظر معامله با آشنا کنند
\\
حالی درون پرده بسی فتنه می‌رود
&&
تا آن زمان که پرده برافتد چه‌ها کنند
\\
گر سنگ از این حدیث بنالد عجب مدار
&&
صاحب دلان حکایت دل خوش ادا کنند
\\
می خور که صد گناه ز اغیار در حجاب
&&
بهتر ز طاعتی که به روی و ریا کنند
\\
پیراهنی که آید از او بوی یوسفم
&&
ترسم برادران غیورش قبا کنند
\\
بگذر به کوی میکده تا زمره حضور
&&
اوقات خود ز بهر تو صرف دعا کنند
\\
پنهان ز حاسدان به خودم خوان که منعمان
&&
خیر نهان برای رضای خدا کنند
\\
حافظ دوام وصل میسر نمی‌شود
&&
شاهان کم التفات به حال گدا کنند
\\
\end{longtable}
\end{center}
