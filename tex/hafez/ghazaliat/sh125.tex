\begin{center}
\section*{غزل شماره ۱۲۵: شاهد آن نیست که مویی و میانی دارد}
\label{sec:sh125}
\addcontentsline{toc}{section}{\nameref{sec:sh125}}
\begin{longtable}{l p{0.5cm} r}
شاهد آن نیست که مویی و میانی دارد
&&
بنده طلعت آن باش که آنی دارد
\\
شیوه حور و پری گر چه لطیف است ولی
&&
خوبی آن است و لطافت که فلانی دارد
\\
چشمه چشم مرا ای گل خندان دریاب
&&
که به امید تو خوش آب روانی دارد
\\
گوی خوبی که برد از تو که خورشید آن جا
&&
نه سواریست که در دست عنانی دارد
\\
دل نشان شد سخنم تا تو قبولش کردی
&&
آری آری سخن عشق نشانی دارد
\\
خم ابروی تو در صنعت تیراندازی
&&
برده از دست هر آن کس که کمانی دارد
\\
در ره عشق نشد کس به یقین محرم راز
&&
هر کسی بر حسب فکر گمانی دارد
\\
با خرابات نشینان ز کرامات ملاف
&&
هر سخن وقتی و هر نکته مکانی دارد
\\
مرغ زیرک نزند در چمنش پرده سرای
&&
هر بهاری که به دنباله خزانی دارد
\\
مدعی گو لغز و نکته به حافظ مفروش
&&
کلک ما نیز زبانی و بیانی دارد
\\
\end{longtable}
\end{center}
