\begin{center}
\section*{غزل شماره ۴۱۹: وصال او ز عمر جاودان به}
\label{sec:sh419}
\addcontentsline{toc}{section}{\nameref{sec:sh419}}
\begin{longtable}{l p{0.5cm} r}
وصال او ز عمر جاودان به
&&
خداوندا مرا آن ده که آن به
\\
به شمشیرم زد و با کس نگفتم
&&
که راز دوست از دشمن نهان به
\\
به داغ بندگی مردن بر این در
&&
به جان او که از ملک جهان به
\\
خدا را از طبیب من بپرسید
&&
که آخر کی شود این ناتوان به
\\
گلی کان پایمال سرو ما گشت
&&
بود خاکش ز خون ارغوان به
\\
به خلدم دعوت ای زاهد مفرما
&&
که این سیب زنخ زان بوستان به
\\
دلا دایم گدای کوی او باش
&&
به حکم آن که دولت جاودان به
\\
جوانا سر متاب از پند پیران
&&
که رای پیر از بخت جوان به
\\
شبی می‌گفت چشم کس ندیده‌ست
&&
ز مروارید گوشم در جهان به
\\
اگر چه زنده رود آب حیات است
&&
ولی شیراز ما از اصفهان به
\\
سخن اندر دهان دوست شکر
&&
ولیکن گفته حافظ از آن به
\\
\end{longtable}
\end{center}
