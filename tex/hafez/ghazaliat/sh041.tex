\begin{center}
\section*{غزل شماره ۴۱: اگر چه باده فرح بخش و باد گل‌بیز است}
\label{sec:sh041}
\addcontentsline{toc}{section}{\nameref{sec:sh041}}
\begin{longtable}{l p{0.5cm} r}
اگر چه باده فرح بخش و باد گل‌بیز است
&&
به بانگ چنگ مخور می که محتسب تیز است
\\
صراحی ای و حریفی گرت به چنگ افتد
&&
به عقل نوش که ایام فتنه انگیز است
\\
در آستین مرقع پیاله پنهان کن
&&
که همچو چشم صراحی زمانه خون‌ریز است
\\
به آب دیده بشوییم خرقه‌ها از می
&&
که موسم ورع و روزگار پرهیز است
\\
مجوی عیش خوش از دور باژگون سپهر
&&
که صاف این سر خم جمله دردی آمیز است
\\
سپهر برشده پرویزنیست خون افشان
&&
که ریزه‌اش سر کسری و تاج پرویز است
\\
عراق و فارس گرفتی به شعر خوش حافظ
&&
بیا که نوبت بغداد و وقت تبریز است
\\
\end{longtable}
\end{center}
