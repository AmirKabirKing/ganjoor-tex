\begin{center}
\section*{غزل شماره ۱۳۳: صوفی نهاد دام و سر حقه باز کرد}
\label{sec:sh133}
\addcontentsline{toc}{section}{\nameref{sec:sh133}}
\begin{longtable}{l p{0.5cm} r}
صوفی نهاد دام و سر حقه باز کرد
&&
بنیاد مکر با فلک حقه باز کرد
\\
بازی چرخ بشکندش بیضه در کلاه
&&
زیرا که عرض شعبده با اهل راز کرد
\\
ساقی بیا که شاهد رعنای صوفیان
&&
دیگر به جلوه آمد و آغاز ناز کرد
\\
این مطرب از کجاست که ساز عراق ساخت
&&
و آهنگ بازگشت به راه حجاز کرد
\\
ای دل بیا که ما به پناه خدا رویم
&&
زآنچ آستین کوته و دست دراز کرد
\\
صنعت مکن که هر که محبت نه راست باخت
&&
عشقش به روی دل در معنی فراز کرد
\\
فردا که پیشگاه حقیقت شود پدید
&&
شرمنده ره روی که عمل بر مجاز کرد
\\
ای کبک خوش خرام کجا می‌روی بایست
&&
غره مشو که گربه زاهد نماز کرد
\\
حافظ مکن ملامت رندان که در ازل
&&
ما را خدا ز زهد ریا بی‌نیاز کرد
\\
\end{longtable}
\end{center}
