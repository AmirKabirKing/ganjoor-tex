\begin{center}
\section*{غزل شماره ۲۱۴: دیدم به خواب خوش که به دستم پیاله بود}
\label{sec:sh214}
\addcontentsline{toc}{section}{\nameref{sec:sh214}}
\begin{longtable}{l p{0.5cm} r}
دیدم به خواب خوش که به دستم پیاله بود
&&
تعبیر رفت و کار به دولت حواله بود
\\
چل سال رنج و غصه کشیدیم و عاقبت
&&
تدبیر ما به دست شراب دوساله بود
\\
آن نافه مراد که می‌خواستم ز بخت
&&
در چین زلف آن بت مشکین کلاله بود
\\
از دست برده بود خمار غمم سحر
&&
دولت مساعد آمد و می در پیاله بود
\\
بر آستان میکده خون می‌خورم مدام
&&
روزی ما ز خوان قدر این نواله بود
\\
هر کو نکاشت مهر و ز خوبی گلی نچید
&&
در رهگذار باد نگهبان لاله بود
\\
بر طرف گلشنم گذر افتاد وقت صبح
&&
آن دم که کار مرغ سحر آه و ناله بود
\\
دیدیم شعر دلکش حافظ به مدح شاه
&&
یک بیت از این قصیده به از صد رساله بود
\\
آن شاه تندحمله که خورشید شیرگیر
&&
پیشش به روز معرکه کمتر غزاله بود
\\
\end{longtable}
\end{center}
