\begin{center}
\section*{غزل شماره ۶۲: مرحبا ای پیک مشتاقان بده پیغام دوست}
\label{sec:sh062}
\addcontentsline{toc}{section}{\nameref{sec:sh062}}
\begin{longtable}{l p{0.5cm} r}
مرحبا ای پیک مشتاقان بده پیغام دوست
&&
تا کنم جان از سر رغبت فدای نام دوست
\\
واله و شیداست دایم همچو بلبل در قفس
&&
طوطی طبعم ز عشق شکر و بادام دوست
\\
زلف او دام است و خالش دانه آن دام و من
&&
بر امید دانه‌ای افتاده‌ام در دام دوست
\\
سر ز مستی برنگیرد تا به صبح روز حشر
&&
هر که چون من در ازل یک جرعه خورد از جام دوست
\\
بس نگویم شمه‌ای از شرح شوق خود از آنک
&&
دردسر باشد نمودن بیش از این ابرام دوست
\\
گر دهد دستم کشم در دیده همچون توتیا
&&
خاک راهی کان مشرف گردد از اقدام دوست
\\
میل من سوی وصال و قصد او سوی فراق
&&
ترک کام خود گرفتم تا برآید کام دوست
\\
حافظ اندر درد او می‌سوز و بی‌درمان بساز
&&
زان که درمانی ندارد درد بی‌آرام دوست
\\
\end{longtable}
\end{center}
