\begin{center}
\section*{غزل شماره ۲۵۹: منم که دیده به دیدار دوست کردم باز}
\label{sec:sh259}
\addcontentsline{toc}{section}{\nameref{sec:sh259}}
\begin{longtable}{l p{0.5cm} r}
منم که دیده به دیدار دوست کردم باز
&&
چه شکر گویمت ای کارساز بنده نواز
\\
نیازمند بلا گو رخ از غبار مشوی
&&
که کیمیای مراد است خاک کوی نیاز
\\
ز مشکلات طریقت عنان متاب ای دل
&&
که مرد راه نیندیشد از نشیب و فراز
\\
طهارت ار نه به خون جگر کند عاشق
&&
به قول مفتی عشقش درست نیست نماز
\\
در این مقام مجازی به جز پیاله مگیر
&&
در این سراچه بازیچه غیر عشق مباز
\\
به نیم بوسه دعایی بخر ز اهل دلی
&&
که کید دشمنت از جان و جسم دارد باز
\\
فکند زمزمه عشق در حجاز و عراق
&&
نوای بانگ غزلهای حافظ از شیراز
\\
\end{longtable}
\end{center}
