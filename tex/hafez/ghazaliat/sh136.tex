\begin{center}
\section*{غزل شماره ۱۳۶: دست در حلقه آن زلف دوتا نتوان کرد}
\label{sec:sh136}
\addcontentsline{toc}{section}{\nameref{sec:sh136}}
\begin{longtable}{l p{0.5cm} r}
دست در حلقه آن زلف دوتا نتوان کرد
&&
تکیه بر عهد تو و باد صبا نتوان کرد
\\
آن چه سعی است من اندر طلبت بنمایم
&&
این قدر هست که تغییر قضا نتوان کرد
\\
دامن دوست به صد خون دل افتاد به دست
&&
به فسوسی که کند خصم رها نتوان کرد
\\
عارضش را به مثل ماه فلک نتوان گفت
&&
نسبت دوست به هر بی سر و پا نتوان کرد
\\
سروبالای من آن گه که درآید به سماع
&&
چه محل جامه جان را که قبا نتوان کرد
\\
نظر پاک تواند رخ جانان دیدن
&&
که در آیینه نظر جز به صفا نتوان کرد
\\
مشکل عشق نه در حوصله دانش ماست
&&
حل این نکته بدین فکر خطا نتوان کرد
\\
غیرتم کشت که محبوب جهانی لیکن
&&
روز و شب عربده با خلق خدا نتوان کرد
\\
من چه گویم که تو را نازکی طبع لطیف
&&
تا به حدیست که آهسته دعا نتوان کرد
\\
بجز ابروی تو محراب دل حافظ نیست
&&
طاعت غیر تو در مذهب ما نتوان کرد
\\
\end{longtable}
\end{center}
