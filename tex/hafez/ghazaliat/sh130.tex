\begin{center}
\section*{غزل شماره ۱۳۰: سحر بلبل حکایت با صبا کرد}
\label{sec:sh130}
\addcontentsline{toc}{section}{\nameref{sec:sh130}}
\begin{longtable}{l p{0.5cm} r}
سحر بلبل حکایت با صبا کرد
&&
که عشق روی گل با ما چه‌ها کرد
\\
از آن رنگ رخم خون در دل افتاد
&&
وز آن گلشن به خارم مبتلا کرد
\\
غلام همت آن نازنینم
&&
که کار خیر بی روی و ریا کرد
\\
من از بیگانگان دیگر ننالم
&&
که با من هر چه کرد آن آشنا کرد
\\
گر از سلطان طمع کردم خطا بود
&&
ور از دلبر وفا جستم جفا کرد
\\
خوشش باد آن نسیم صبحگاهی
&&
که درد شب نشینان را دوا کرد
\\
نقاب گل کشید و زلف سنبل
&&
گره بند قبای غنچه وا کرد
\\
به هر سو بلبل عاشق در افغان
&&
تنعم از میان باد صبا کرد
\\
بشارت بر به کوی می فروشان
&&
که حافظ توبه از زهد ریا کرد
\\
وفا از خواجگان شهر با من
&&
کمال دولت و دین بوالوفا کرد
\\
\end{longtable}
\end{center}
