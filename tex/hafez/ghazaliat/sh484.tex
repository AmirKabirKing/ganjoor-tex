\begin{center}
\section*{غزل شماره ۴۸۴: تو مگر بر لب آبی به هوس بنشینی}
\label{sec:sh484}
\addcontentsline{toc}{section}{\nameref{sec:sh484}}
\begin{longtable}{l p{0.5cm} r}
تو مگر بر لب آبی به هوس بنشینی
&&
ور نه هر فتنه که بینی همه از خود بینی
\\
به خدایی که تویی بنده بگزیده او
&&
که بر این چاکر دیرینه کسی نگزینی
\\
گر امانت به سلامت ببرم باکی نیست
&&
بی دلی سهل بود گر نبود بی‌دینی
\\
ادب و شرم تو را خسرو مه رویان کرد
&&
آفرین بر تو که شایسته صد چندینی
\\
عجب از لطف تو ای گل که نشستی با خار
&&
ظاهرا مصلحت وقت در آن می‌بینی
\\
صبر بر جور رقیبت چه کنم گر نکنم
&&
عاشقان را نبود چاره به جز مسکینی
\\
باد صبحی به هوایت ز گلستان برخاست
&&
که تو خوشتر ز گل و تازه‌تر از نسرینی
\\
شیشه بازی سرشکم نگری از چپ و راست
&&
گر بر این منظر بینش نفسی بنشینی
\\
سخنی بی‌غرض از بنده مخلص بشنو
&&
ای که منظور بزرگان حقیقت بینی
\\
نازنینی چو تو پاکیزه دل و پاک نهاد
&&
بهتر آن است که با مردم بد ننشینی
\\
سیل این اشک روان صبر و دل حافظ برد
&&
بلغ الطاقه یا مقله عینی بینی
\\
تو بدین نازکی و سرکشی ای شمع چگل
&&
لایق بندگی خواجه جلال الدینی
\\
\end{longtable}
\end{center}
