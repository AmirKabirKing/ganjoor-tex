\begin{center}
\section*{غزل شماره ۴۴۲: به جان او که گرم دسترس به جان بودی}
\label{sec:sh442}
\addcontentsline{toc}{section}{\nameref{sec:sh442}}
\begin{longtable}{l p{0.5cm} r}
به جان او که گرم دسترس به جان بودی
&&
کمینه پیشکش بندگانش آن بودی
\\
بگفتمی که بها چیست خاک پایش را
&&
اگر حیات گران مایه جاودان بودی
\\
به بندگی قدش سرو معترف گشتی
&&
گرش چو سوسن آزاده ده زبان بودی
\\
به خواب نیز نمی‌بینمش چه جای وصال
&&
چو این نبود و ندیدیم باری آن بودی
\\
اگر دلم نشدی پایبند طره او
&&
کی اش قرار در این تیره خاکدان بودی
\\
به رخ چو مهر فلک بی‌نظیر آفاق است
&&
به دل دریغ که یک ذره مهربان بودی
\\
درآمدی ز درم کاشکی چو لمعه نور
&&
که بر دو دیده ما حکم او روان بودی
\\
ز پرده ناله حافظ برون کی افتادی
&&
اگر نه همدم مرغان صبح خوان بودی
\\
\end{longtable}
\end{center}
