\begin{center}
\section*{غزل شماره ۳۰۷: هر نکته‌ای که گفتم در وصف آن شمایل}
\label{sec:sh307}
\addcontentsline{toc}{section}{\nameref{sec:sh307}}
\begin{longtable}{l p{0.5cm} r}
هر نکته‌ای که گفتم در وصف آن شمایل
&&
هر کو شنید گفتا للهِ دَرُّ قائل
\\
تحصیل عشق و رندی آسان نمود اول
&&
آخر بسوخت جانم در کسب این فضایل
\\
حلاج بر سر دار این نکته خوش سراید
&&
از شافعی نپرسند امثال این مسائل
\\
گفتم که کی ببخشی بر جان ناتوانم
&&
گفت آن زمان که نبود جان در میانه حائل
\\
دل داده‌ام به یاری شوخی کشی نگاری
&&
مرضیّةُ السجایا محمودةُ الخصائل
\\
در عین گوشه‌گیری بودم چو چشم مستت
&&
و اکنون شدم به مستان چون ابروی تو مایل
\\
از آب دیده صد ره طوفان نوح دیدم
&&
وز لوح سینه نقشت هرگز نگشت زایل
\\
ای دوست دست حافظ تعویذ چشم زخم است
&&
یا رب ببینم آن را در گردنت حمایل
\\
\end{longtable}
\end{center}
