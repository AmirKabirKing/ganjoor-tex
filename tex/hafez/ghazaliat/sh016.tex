\begin{center}
\section*{غزل شماره ۱۶: خمی که ابروی شوخ تو در کمان انداخت}
\label{sec:sh016}
\addcontentsline{toc}{section}{\nameref{sec:sh016}}
\begin{longtable}{l p{0.5cm} r}
خمی که ابروی شوخ تو در کمان انداخت
&&
به قصد جان من زار ناتوان انداخت
\\
نبود نقش دو عالم که رنگ الفت بود
&&
زمانه طرح محبت نه این زمان انداخت
\\
به یک کرشمه که نرگس به خودفروشی کرد
&&
فریب چشم تو صد فتنه در جهان انداخت
\\
شراب خورده و خوی کرده می‌روی به چمن
&&
که آب روی تو آتش در ارغوان انداخت
\\
به بزمگاه چمن دوش مست بگذشتم
&&
چو از دهان توام غنچه در گمان انداخت
\\
بنفشه طره مفتول خود گره می‌زد
&&
صبا حکایت زلف تو در میان انداخت
\\
ز شرم آن که به روی تو نسبتش کردم
&&
سمن به دست صبا خاک در دهان انداخت
\\
من از ورع می و مطرب ندیدمی زین پیش
&&
هوای مغبچگانم در این و آن انداخت
\\
کنون به آب می لعل خرقه می‌شویم
&&
نصیبه ازل از خود نمی‌توان انداخت
\\
مگر گشایش حافظ در این خرابی بود
&&
که بخشش ازلش در می مغان انداخت
\\
جهان به کام من اکنون شود که دور زمان
&&
مرا به بندگی خواجه جهان انداخت
\\
\end{longtable}
\end{center}
