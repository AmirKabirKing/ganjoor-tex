\begin{center}
\section*{غزل شماره ۵۰: به دام زلف تو دل مبتلای خویشتن است}
\label{sec:sh050}
\addcontentsline{toc}{section}{\nameref{sec:sh050}}
\begin{longtable}{l p{0.5cm} r}
به دام زلف تو دل مبتلای خویشتن است
&&
بکش به غمزه که اینش سزای خویشتن است
\\
گرت ز دست برآید مراد خاطر ما
&&
به دست باش که خیری به جای خویشتن است
\\
به جانت ای بت شیرین دهن که همچون شمع
&&
شبان تیره مرادم فنای خویشتن است
\\
چو رای عشق زدی با تو گفتم ای بلبل
&&
مکن که آن گل خندان برای خویشتن است
\\
به مشک چین و چگل نیست بوی گل محتاج
&&
که نافه‌هاش ز بند قبای خویشتن است
\\
مرو به خانه ارباب بی‌مروت دهر
&&
که گنج عافیتت در سرای خویشتن است
\\
بسوخت حافظ و در شرط عشقبازی او
&&
هنوز بر سر عهد و وفای خویشتن است
\\
\end{longtable}
\end{center}
