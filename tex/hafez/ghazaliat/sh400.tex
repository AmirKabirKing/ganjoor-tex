\begin{center}
\section*{غزل شماره ۴۰۰: بالابلند عشوه گر نقش باز من}
\label{sec:sh400}
\addcontentsline{toc}{section}{\nameref{sec:sh400}}
\begin{longtable}{l p{0.5cm} r}
بالابلند عشوه گر نقش باز من
&&
کوتاه کرد قصه زهد دراز من
\\
دیدی دلا که آخر پیری و زهد و علم
&&
با من چه کرد دیده معشوقه باز من
\\
می‌ترسم از خرابی ایمان که می‌برد
&&
محراب ابروی تو حضور نماز من
\\
گفتم به دلق زرق بپوشم نشان عشق
&&
غماز بود اشک و عیان کرد راز من
\\
مست است یار و یاد حریفان نمی‌کند
&&
ذکرش به خیر ساقی مسکین نواز من
\\
یا رب کی آن صبا بوزد کز نسیم آن
&&
گردد شمامه کرمش کارساز من
\\
نقشی بر آب می‌زنم از گریه حالیا
&&
تا کی شود قرین حقیقت مجاز من
\\
بر خود چو شمع خنده زنان گریه می‌کنم
&&
تا با تو سنگ دل چه کند سوز و ساز من
\\
زاهد چو از نماز تو کاری نمی‌رود
&&
هم مستی شبانه و راز و نیاز من
\\
حافظ ز گریه سوخت بگو حالش ای صبا
&&
با شاه دوست پرور دشمن گداز من
\\
\end{longtable}
\end{center}
