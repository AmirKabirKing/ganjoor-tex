\begin{center}
\section*{غزل شماره ۱۶۹: یاری اندر کس نمی‌بینیم یاران را چه شد}
\label{sec:sh169}
\addcontentsline{toc}{section}{\nameref{sec:sh169}}
\begin{longtable}{l p{0.5cm} r}
یاری اندر کس نمی‌بینیم یاران را چه شد
&&
دوستی کی آخر آمد دوستداران را چه شد
\\
آب حیوان تیره گون شد خضر فرخ پی کجاست
&&
خون چکید از شاخ گل باد بهاران را چه شد
\\
کس نمی‌گوید که یاری داشت حق دوستی
&&
حق شناسان را چه حال افتاد یاران را چه شد
\\
لعلی از کان مروت برنیامد سال‌هاست
&&
تابش خورشید و سعی باد و باران را چه شد
\\
شهر یاران بود و خاک مهربانان این دیار
&&
مهربانی کی سر آمد شهریاران را چه شد
\\
گوی توفیق و کرامت در میان افکنده‌اند
&&
کس به میدان در نمی‌آید سواران را چه شد
\\
صدهزاران گل شکفت و بانگ مرغی برنخاست
&&
عندلیبان را چه پیش آمد هزاران را چه شد
\\
زهره سازی خوش نمی‌سازد مگر عودش بسوخت
&&
کس ندارد ذوق مستی میگساران را چه شد
\\
حافظ اسرار الهی کس نمی‌داند خموش
&&
از که می‌پرسی که دور روزگاران را چه شد
\\
\end{longtable}
\end{center}
