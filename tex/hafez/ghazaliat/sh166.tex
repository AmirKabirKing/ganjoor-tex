\begin{center}
\section*{غزل شماره ۱۶۶: روز هجران و شب فرقت یار آخر شد}
\label{sec:sh166}
\addcontentsline{toc}{section}{\nameref{sec:sh166}}
\begin{longtable}{l p{0.5cm} r}
روز هجران و شب فرقت یار آخر شد
&&
زدم این فال و گذشت اختر و کار آخر شد
\\
آن همه ناز و تنعم که خزان می‌فرمود
&&
عاقبت در قدم باد بهار آخر شد
\\
شکر ایزد که به اقبال کله گوشه گل
&&
نخوت باد دی و شوکت خار آخر شد
\\
صبح امید که بد معتکف پرده غیب
&&
گو برون آی که کار شب تار آخر شد
\\
آن پریشانی شب‌های دراز و غم دل
&&
همه در سایه گیسوی نگار آخر شد
\\
باورم نیست ز بدعهدی ایام هنوز
&&
قصه غصه که در دولت یار آخر شد
\\
ساقیا لطف نمودی قدحت پرمی باد
&&
که به تدبیر تو تشویش خمار آخر شد
\\
در شمار ار چه نیاورد کسی حافظ را
&&
شکر کان محنت بی‌حد و شمار آخر شد
\\
\end{longtable}
\end{center}
