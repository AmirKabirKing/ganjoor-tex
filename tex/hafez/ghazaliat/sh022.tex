\begin{center}
\section*{غزل شماره ۲۲: چو بشنوی سخن اهل دل مگو که خطاست}
\label{sec:sh022}
\addcontentsline{toc}{section}{\nameref{sec:sh022}}
\begin{longtable}{l p{0.5cm} r}
چو بشنوی سخن اهل دل مگو که خطاست
&&
سخن شناس نه‌ای جان من خطا این جاست
\\
سرم به دنیی و عقبی فرو نمی‌آید
&&
تبارک الله از این فتنه‌ها که در سر ماست
\\
در اندرون من خسته دل ندانم کیست
&&
که من خموشم و او در فغان و در غوغاست
\\
دلم ز پرده برون شد کجایی ای مطرب
&&
بنال هان که از این پرده کار ما به نواست
\\
مرا به کار جهان هرگز التفات نبود
&&
رخ تو در نظر من چنین خوشش آراست
\\
نخفته‌ام ز خیالی که می‌پزد دل من
&&
خمار صدشبه دارم شرابخانه کجاست
\\
چنین که صومعه آلوده شد ز خون دلم
&&
گرم به باده بشویید حق به دست شماست
\\
از آن به دیر مغانم عزیز می‌دارند
&&
که آتشی که نمیرد همیشه در دل ماست
\\
چه ساز بود که در پرده می‌زد آن مطرب
&&
که رفت عمر و هنوزم دماغ پر ز هواست
\\
ندای عشق تو دیشب در اندرون دادند
&&
فضای سینه حافظ هنوز پر ز صداست
\\
\end{longtable}
\end{center}
