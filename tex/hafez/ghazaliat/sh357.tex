\begin{center}
\section*{غزل شماره ۳۵۷: در خرابات مغان نور خدا می‌بینم}
\label{sec:sh357}
\addcontentsline{toc}{section}{\nameref{sec:sh357}}
\begin{longtable}{l p{0.5cm} r}
در خرابات مغان نور خدا می‌بینم
&&
این عجب بین که چه نوری ز کجا می‌بینم
\\
جلوه بر من مفروش ای ملک الحاج که تو
&&
خانه می‌بینی و من خانه خدا می‌بینم
\\
خواهم از زلف بتان نافه گشایی کردن
&&
فکر دور است همانا که خطا می‌بینم
\\
سوز دل اشک روان آه سحر ناله شب
&&
این همه از نظر لطف شما می‌بینم
\\
هر دم از روی تو نقشی زندم راه خیال
&&
با که گویم که در این پرده چه‌ها می‌بینم
\\
کس ندیده‌ست ز مشک ختن و نافه چین
&&
آن چه من هر سحر از باد صبا می‌بینم
\\
دوستان عیب نظربازی حافظ مکنید
&&
که من او را ز محبان شما می‌بینم
\\
\end{longtable}
\end{center}
