\begin{center}
\section*{غزل شماره ۴۴۵: تو را که هر چه مراد است در جهان داری}
\label{sec:sh445}
\addcontentsline{toc}{section}{\nameref{sec:sh445}}
\begin{longtable}{l p{0.5cm} r}
تو را که هر چه مراد است در جهان داری
&&
چه غم ز حال ضعیفان ناتوان داری
\\
بخواه جان و دل از بنده و روان بستان
&&
که حکم بر سر آزادگان روان داری
\\
میان نداری و دارم عجب که هر ساعت
&&
میان مجمع خوبان کنی میانداری
\\
بیاض روی تو را نیست نقش درخور از آنک
&&
سوادی از خط مشکین بر ارغوان داری
\\
بنوش می که سبکروحی و لطیف مدام
&&
علی الخصوص در آن دم که سر گران داری
\\
مکن عتاب از این بیش و جور بر دل ما
&&
مکن هر آن چه توانی که جای آن داری
\\
به اختیارت اگر صد هزار تیر جفاست
&&
به قصد جان من خسته در کمان داری
\\
بکش جفای رقیبان مدام و جور حسود
&&
که سهل باشد اگر یار مهربان داری
\\
به وصل دوست گرت دست می‌دهد یک دم
&&
برو که هر چه مراد است در جهان داری
\\
چو گل به دامن از این باغ می‌بری حافظ
&&
چه غم ز ناله و فریاد باغبان داری
\\
\end{longtable}
\end{center}
