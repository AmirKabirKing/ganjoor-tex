\begin{center}
\section*{غزل شماره ۷۰: مردم دیده ما جز به رخت ناظر نیست}
\label{sec:sh070}
\addcontentsline{toc}{section}{\nameref{sec:sh070}}
\begin{longtable}{l p{0.5cm} r}
مردم دیده ما جز به رخت ناظر نیست
&&
دل سرگشته ما غیر تو را ذاکر نیست
\\
اشکم احرام طواف حرمت می‌بندد
&&
گر چه از خون دل ریش دمی طاهر نیست
\\
بسته دام و قفس باد چو مرغ وحشی
&&
طایر سدره اگر در طلبت طایر نیست
\\
عاشق مفلس اگر قلب دلش کرد نثار
&&
مکنش عیب که بر نقد روان قادر نیست
\\
عاقبت دست بدان سرو بلندش برسد
&&
هر که را در طلبت همت او قاصر نیست
\\
از روان بخشی عیسی نزنم دم هرگز
&&
زان که در روح فزایی چو لبت ماهر نیست
\\
من که در آتش سودای تو آهی نزنم
&&
کی توان گفت که بر داغ دلم صابر نیست
\\
روز اول که سر زلف تو دیدم گفتم
&&
که پریشانی این سلسله را آخر نیست
\\
سر پیوند تو تنها نه دل حافظ راست
&&
کیست آن کش سر پیوند تو در خاطر نیست
\\
\end{longtable}
\end{center}
