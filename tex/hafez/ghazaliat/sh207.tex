\begin{center}
\section*{غزل شماره ۲۰۷: یاد باد آن که سر کوی توام منزل بود}
\label{sec:sh207}
\addcontentsline{toc}{section}{\nameref{sec:sh207}}
\begin{longtable}{l p{0.5cm} r}
یاد باد آن که سر کوی توام منزل بود
&&
دیده را روشنی از خاک درت حاصل بود
\\
راست چون سوسن و گل از اثر صحبت پاک
&&
بر زبان بود مرا آن چه تو را در دل بود
\\
دل چو از پیر خرد نقل معانی می‌کرد
&&
عشق می‌گفت به شرح آن چه بر او مشکل بود
\\
آه از آن جور و تطاول که در این دامگه است
&&
آه از آن سوز و نیازی که در آن محفل بود
\\
در دلم بود که بی دوست نباشم هرگز
&&
چه توان کرد که سعی من و دل باطل بود
\\
دوش بر یاد حریفان به خرابات شدم
&&
خم می دیدم خون در دل و پا در گل بود
\\
بس بگشتم که بپرسم سبب درد فراق
&&
مفتی عقل در این مسئله لایعقل بود
\\
راستی خاتم فیروزه بواسحاقی
&&
خوش درخشید ولی دولت مستعجل بود
\\
دیدی آن قهقهه کبک خرامان حافظ
&&
که ز سرپنجه شاهین قضا غافل بود
\\
\end{longtable}
\end{center}
