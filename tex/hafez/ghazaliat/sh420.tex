\begin{center}
\section*{غزل شماره ۴۲۰: ناگهان پرده برانداخته‌ای یعنی چه}
\label{sec:sh420}
\addcontentsline{toc}{section}{\nameref{sec:sh420}}
\begin{longtable}{l p{0.5cm} r}
ناگهان پرده برانداخته‌ای یعنی چه
&&
مست از خانه برون تاخته‌ای یعنی چه
\\
زلف در دست صبا گوش به فرمان رقیب
&&
این چنین با همه درساخته‌ای یعنی چه
\\
شاه خوبانی و منظور گدایان شده‌ای
&&
قدر این مرتبه نشناخته‌ای یعنی چه
\\
نه سر زلف خود اول تو به دستم دادی
&&
بازم از پای درانداخته‌ای یعنی چه
\\
سخنت رمز دهان گفت و کمر سر میان
&&
و از میان تیغ به ما آخته‌ای یعنی چه
\\
هر کس از مهره مهر تو به نقشی مشغول
&&
عاقبت با همه کج باخته‌ای یعنی چه
\\
حافظا در دل تنگت چو فرود آمد یار
&&
خانه از غیر نپرداخته‌ای یعنی چه
\\
\end{longtable}
\end{center}
