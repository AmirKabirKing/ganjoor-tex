\begin{center}
\section*{غزل شماره ۴۳۰: به صوت بلبل و قمری اگر ننوشی می}
\label{sec:sh430}
\addcontentsline{toc}{section}{\nameref{sec:sh430}}
\begin{longtable}{l p{0.5cm} r}
به صوت بلبل و قمری اگر ننوشی می
&&
علاج کی کنمت آخرالدواء الکی
\\
ذخیره‌ای بنه از رنگ و بوی فصل بهار
&&
که می‌رسند ز پی رهزنان بهمن و دی
\\
چو گل نقاب برافکند و مرغ زد هوهو
&&
منه ز دست پیاله چه می‌کنی هی هی
\\
شکوه سلطنت و حسن کی ثباتی داد
&&
ز تخت جم سخنی مانده است و افسر کی
\\
خزینه داری میراث خوارگان کفر است
&&
به قول مطرب و ساقی به فتوی دف و نی
\\
زمانه هیچ نبخشد که بازنستاند
&&
مجو ز سفله مروت که شیئه لا شی
\\
نوشته‌اند بر ایوان جنه الماوی
&&
که هر که عشوه دنیی خرید وای به وی
\\
سخا نماند سخن طی کنم شراب کجاست
&&
بده به شادی روح و روان حاتم طی
\\
بخیل بوی خدا نشنود بیا حافظ
&&
پیاله گیر و کرم ورز و الضمان علی
\\
\end{longtable}
\end{center}
