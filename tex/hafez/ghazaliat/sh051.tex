\begin{center}
\section*{غزل شماره ۵۱: لعل سیراب به خون تشنه لب یار من است}
\label{sec:sh051}
\addcontentsline{toc}{section}{\nameref{sec:sh051}}
\begin{longtable}{l p{0.5cm} r}
لعل سیراب به خون تشنه لب یار من است
&&
وز پی دیدن او دادن جان کار من است
\\
شرم از آن چشم سیه بادش و مژگان دراز
&&
هر که دل بردن او دید و در انکار من است
\\
ساروان رخت به دروازه مبر کان سر کو
&&
شاهراهیست که منزلگه دلدار من است
\\
بنده طالع خویشم که در این قحط وفا
&&
عشق آن لولی سرمست خریدار من است
\\
طبله عطر گل و زلف عبیرافشانش
&&
فیض یک شمه ز بوی خوش عطار من است
\\
باغبان همچو نسیمم ز در خویش مران
&&
کآب گلزار تو از اشک چو گلنار من است
\\
شربت قند و گلاب از لب یارم فرمود
&&
نرگس او که طبیب دل بیمار من است
\\
آن که در طرز غزل نکته به حافظ آموخت
&&
یار شیرین سخن نادره گفتار من است
\\
\end{longtable}
\end{center}
