\begin{center}
\section*{غزل شماره ۳۸۲: فاتحه‌ای چو آمدی بر سر خسته‌ای بخوان}
\label{sec:sh382}
\addcontentsline{toc}{section}{\nameref{sec:sh382}}
\begin{longtable}{l p{0.5cm} r}
فاتحه‌ای چو آمدی بر سر خسته‌ای بخوان
&&
لب بگشا که می‌دهد لعل لبت به مرده جان
\\
آن که به پرسش آمد و فاتحه خواند و می‌رود
&&
گو نفسی که روح را می‌کنم از پی اش روان
\\
ای که طبیب خسته‌ای روی زبان من ببین
&&
کاین دم و دود سینه‌ام بار دل است بر زبان
\\
گر چه تب استخوان من کرد ز مهر گرم و رفت
&&
همچو تبم نمی‌رود آتش مهر از استخوان
\\
حال دلم ز خال تو هست در آتشش وطن
&&
چشمم از آن دو چشم تو خسته شده‌ست و ناتوان
\\
بازنشان حرارتم ز آب دو دیده و ببین
&&
نبض مرا که می‌دهد هیچ ز زندگی نشان
\\
آن که مدام شیشه‌ام از پی عیش داده است
&&
شیشه‌ام از چه می‌برد پیش طبیب هر زمان
\\
حافظ از آب زندگی شعر تو داد شربتم
&&
ترک طبیب کن بیا نسخه شربتم بخوان
\\
\end{longtable}
\end{center}
