\begin{center}
\section*{غزل شماره ۲۸۸: کنار آب و پای بید و طبع شعر و یاری خوش}
\label{sec:sh288}
\addcontentsline{toc}{section}{\nameref{sec:sh288}}
\begin{longtable}{l p{0.5cm} r}
کنار آب و پای بید و طبع شعر و یاری خوش
&&
معاشر دلبری شیرین و ساقی گلعذاری خوش
\\
الا ای دولتی طالع که قدر وقت می‌دانی
&&
گوارا بادت این عشرت که داری روزگاری خوش
\\
هر آن کس را که در خاطر ز عشق دلبری باریست
&&
سپندی گو بر آتش نه که دارد کار و باری خوش
\\
عروس طبع را زیور ز فکر بکر می‌بندم
&&
بود کز دست ایامم به دست افتد نگاری خوش
\\
شب صحبت غنیمت دان و داد خوشدلی بستان
&&
که مهتابی دل افروز است و طرف لاله زاری خوش
\\
میی در کاسه چشم است ساقی را بنامیزد
&&
که مستی می‌کند با عقل و می‌بخشد خماری خوش
\\
به غفلت عمر شد حافظ بیا با ما به میخانه
&&
که شنگولان خوش باشت بیاموزند کاری خوش
\\
\end{longtable}
\end{center}
