\begin{center}
\section*{غزل شماره ۶۴: اگر چه عرض هنر پیش یار بی‌ادبیست}
\label{sec:sh064}
\addcontentsline{toc}{section}{\nameref{sec:sh064}}
\begin{longtable}{l p{0.5cm} r}
اگر چه عرض هنر پیش یار بی‌ادبیست
&&
زبان خموش ولیکن دهان پر از عربیست
\\
پری نهفته رخ و دیو در کرشمه حسن
&&
بسوخت دیده ز حیرت که این چه بوالعجبیست
\\
در این چمن گل بی خار کس نچید آری
&&
چراغ مصطفوی با شرار بولهبیست
\\
سبب مپرس که چرخ از چه سفله پرور شد
&&
که کام بخشی او را بهانه بی سببیست
\\
به نیم جو نخرم طاق خانقاه و رباط
&&
مرا که مصطبه ایوان و پای خم طنبیست
\\
جمال دختر رز نور چشم ماست مگر
&&
که در نقاب زجاجی و پرده عنبیست
\\
هزار عقل و ادب داشتم من ای خواجه
&&
کنون که مست خرابم صلاح بی‌ادبیست
\\
بیار می که چو حافظ هزارم استظهار
&&
به گریه سحری و نیاز نیم شبیست
\\
\end{longtable}
\end{center}
