\begin{center}
\section*{غزل شماره ۲۵۷: روی بنما و مرا گو که ز جان دل برگیر}
\label{sec:sh257}
\addcontentsline{toc}{section}{\nameref{sec:sh257}}
\begin{longtable}{l p{0.5cm} r}
روی بنما و مرا گو که ز جان دل برگیر
&&
پیش شمع آتش پروانه به جان گو درگیر
\\
در لب تشنه ما بین و مدار آب دریغ
&&
بر سر کشته خویش آی و ز خاکش برگیر
\\
ترک درویش مگیر ار نبود سیم و زرش
&&
در غمت سیم شمار اشک و رخش را زر گیر
\\
چنگ بنواز و بساز ار نبود عود چه باک
&&
آتشم عشق و دلم عود و تنم مجمر گیر
\\
در سماع آی و ز سر خرقه برانداز و برقص
&&
ور نه با گوشه رو و خرقه ما در سر گیر
\\
صوف برکش ز سر و باده صافی درکش
&&
سیم در باز و به زر سیمبری در بر گیر
\\
دوست گو یار شو و هر دو جهان دشمن باش
&&
بخت گو پشت مکن روی زمین لشکر گیر
\\
میل رفتن مکن ای دوست دمی با ما باش
&&
بر لب جوی طرب جوی و به کف ساغر گیر
\\
رفته گیر از برم و زآتش و آب دل و چشم
&&
گونه‌ام زرد و لبم خشک و کنارم تر گیر
\\
حافظ آراسته کن بزم و بگو واعظ را
&&
که ببین مجلسم و ترک سر منبر گیر
\\
\end{longtable}
\end{center}
