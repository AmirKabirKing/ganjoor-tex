\begin{center}
\section*{غزل شماره ۲۹۴: در وفای عشق تو مشهور خوبانم چو شمع}
\label{sec:sh294}
\addcontentsline{toc}{section}{\nameref{sec:sh294}}
\begin{longtable}{l p{0.5cm} r}
در وفای عشق تو مشهور خوبانم چو شمع
&&
شب نشین کوی سربازان و رندانم چو شمع
\\
روز و شب خوابم نمی‌آید به چشم غم پرست
&&
بس که در بیماری هجر تو گریانم چو شمع
\\
رشته صبرم به مقراض غمت ببریده شد
&&
همچنان در آتش مهر تو سوزانم چو شمع
\\
گر کمیت اشک گلگونم نبودی گرم رو
&&
کی شدی روشن به گیتی راز پنهانم چو شمع
\\
در میان آب و آتش همچنان سرگرم توست
&&
این دل زار نزار اشک بارانم چو شمع
\\
در شب هجران مرا پروانه وصلی فرست
&&
ور نه از دردت جهانی را بسوزانم چو شمع
\\
بی جمال عالم آرای تو روزم چون شب است
&&
با کمال عشق تو در عین نقصانم چو شمع
\\
کوه صبرم نرم شد چون موم در دست غمت
&&
تا در آب و آتش عشقت گدازانم چو شمع
\\
همچو صبحم یک نفس باقیست با دیدار تو
&&
چهره بنما دلبرا تا جان برافشانم چو شمع
\\
سرفرازم کن شبی از وصل خود ای نازنین
&&
تا منور گردد از دیدارت ایوانم چو شمع
\\
آتش مهر تو را حافظ عجب در سر گرفت
&&
آتش دل کی به آب دیده بنشانم چو شمع
\\
\end{longtable}
\end{center}
