\begin{center}
\section*{غزل شماره ۱۵۴: راهی بزن که آهی بر ساز آن توان زد}
\label{sec:sh154}
\addcontentsline{toc}{section}{\nameref{sec:sh154}}
\begin{longtable}{l p{0.5cm} r}
راهی بزن که آهی بر ساز آن توان زد
&&
شعری بخوان که با او رطل گران توان زد
\\
بر آستان جانان گر سر توان نهادن
&&
گلبانگ سربلندی بر آسمان توان زد
\\
قد خمیده ما سهلت نماید اما
&&
بر چشم دشمنان تیر از این کمان توان زد
\\
در خانقه نگنجد اسرار عشقبازی
&&
جام می مغانه هم با مغان توان زد
\\
درویش را نباشد برگ سرای سلطان
&&
ماییم و کهنه دلقی کآتش در آن توان زد
\\
اهل نظر دو عالم در یک نظر ببازند
&&
عشق است و داو اول بر نقد جان توان زد
\\
گر دولت وصالت خواهد دری گشودن
&&
سرها بدین تخیل بر آستان توان زد
\\
عشق و شباب و رندی مجموعه مراد است
&&
چون جمع شد معانی گوی بیان توان زد
\\
شد رهزن سلامت زلف تو وین عجب نیست
&&
گر راه زن تو باشی صد کاروان توان زد
\\
حافظ به حق قرآن کز شید و زرق بازآی
&&
باشد که گوی عیشی در این جهان توان زد
\\
\end{longtable}
\end{center}
