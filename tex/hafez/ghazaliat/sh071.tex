\begin{center}
\section*{غزل شماره ۷۱: زاهد ظاهرپرست از حال ما آگاه نیست}
\label{sec:sh071}
\addcontentsline{toc}{section}{\nameref{sec:sh071}}
\begin{longtable}{l p{0.5cm} r}
زاهد ظاهرپرست از حال ما آگاه نیست
&&
در حق ما هر چه گوید جای هیچ اکراه نیست
\\
در طریقت هر چه پیش سالک آید خیر اوست
&&
در صراط مستقیم ای دل کسی گمراه نیست
\\
تا چه بازی رخ نماید بیدقی خواهیم راند
&&
عرصه شطرنج رندان را مجال شاه نیست
\\
چیست این سقف بلند ساده بسیارنقش
&&
زین معما هیچ دانا در جهان آگاه نیست
\\
این چه استغناست یا رب وین چه قادر حکمت است
&&
کاین همه زخم نهان هست و مجال آه نیست
\\
صاحب دیوان ما گویی نمی‌داند حساب
&&
کاندر این طغرا نشان حسبة لله نیست
\\
هر که خواهد گو بیا و هر چه خواهد گو بگو
&&
کبر و ناز و حاجب و دربان بدین درگاه نیست
\\
بر در میخانه رفتن کار یکرنگان بود
&&
خودفروشان را به کوی می فروشان راه نیست
\\
هر چه هست از قامت ناساز بی اندام ماست
&&
ور نه تشریف تو بر بالای کس کوتاه نیست
\\
بنده پیر خراباتم که لطفش دایم است
&&
ور نه لطف شیخ و زاهد گاه هست و گاه نیست
\\
حافظ ار بر صدر ننشیند ز عالی مشربیست
&&
عاشق دردی کش اندر بند مال و جاه نیست
\\
\end{longtable}
\end{center}
