\begin{center}
\section*{غزل شماره ۱۵۸: من و انکار شراب این چه حکایت باشد}
\label{sec:sh158}
\addcontentsline{toc}{section}{\nameref{sec:sh158}}
\begin{longtable}{l p{0.5cm} r}
من و انکار شراب این چه حکایت باشد
&&
غالبا این قدرم عقل و کفایت باشد
\\
تا به غایت ره میخانه نمی‌دانستم
&&
ور نه مستوری ما تا به چه غایت باشد
\\
زاهد و عجب و نماز و من و مستی و نیاز
&&
تا تو را خود ز میان با که عنایت باشد
\\
زاهد ار راه به رندی نبرد معذور است
&&
عشق کاریست که موقوف هدایت باشد
\\
من که شب‌ها ره تقوا زده‌ام با دف و چنگ
&&
این زمان سر به ره آرم چه حکایت باشد
\\
بنده پیر مغانم که ز جهلم برهاند
&&
پیر ما هر چه کند عین عنایت باشد
\\
دوش از این غصه نخفتم که رفیقی می‌گفت
&&
حافظ ار مست بود جای شکایت باشد
\\
\end{longtable}
\end{center}
