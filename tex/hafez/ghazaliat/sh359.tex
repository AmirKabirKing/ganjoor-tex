\begin{center}
\section*{غزل شماره ۳۵۹: خرم آن روز کز این منزل ویران بروم}
\label{sec:sh359}
\addcontentsline{toc}{section}{\nameref{sec:sh359}}
\begin{longtable}{l p{0.5cm} r}
خرم آن روز کز این منزل ویران بروم
&&
راحت جان طلبم و از پی جانان بروم
\\
گر چه دانم که به جایی نبرد راه غریب
&&
من به بوی سر آن زلف پریشان بروم
\\
دلم از وحشت زندان سکندر بگرفت
&&
رخت بربندم و تا ملک سلیمان بروم
\\
چون صبا با تن بیمار و دل بی‌طاقت
&&
به هواداری آن سرو خرامان بروم
\\
در ره او چو قلم گر به سرم باید رفت
&&
با دل زخم کش و دیده گریان بروم
\\
نذر کردم گر از این غم به درآیم روزی
&&
تا در میکده شادان و غزل خوان بروم
\\
به هواداری او ذره صفت رقص کنان
&&
تا لب چشمه خورشید درخشان بروم
\\
تازیان را غم احوال گران باران نیست
&&
پارسایان مددی تا خوش و آسان بروم
\\
ور چو حافظ ز بیابان نبرم ره بیرون
&&
همره کوکبه آصف دوران بروم
\\
\end{longtable}
\end{center}
