\begin{center}
\section*{غزل شماره ۲۹۱: ما آزموده‌ایم در این شهر بخت خویش}
\label{sec:sh291}
\addcontentsline{toc}{section}{\nameref{sec:sh291}}
\begin{longtable}{l p{0.5cm} r}
ما آزموده‌ایم در این شهر بخت خویش
&&
بیرون کشید باید از این ورطه رخت خویش
\\
از بس که دست می‌گزم و آه می‌کشم
&&
آتش زدم چو گل به تن لخت لخت خویش
\\
دوشم ز بلبلی چه خوش آمد که می‌سرود
&&
گل گوش پهن کرده ز شاخ درخت خویش
\\
کای دل تو شاد باش که آن یار تندخو
&&
بسیار تندروی نشیند ز بخت خویش
\\
خواهی که سخت و سست جهان بر تو بگذرد
&&
بگذر ز عهد سست و سخن‌های سخت خویش
\\
وقت است کز فراق تو وز سوز اندرون
&&
آتش درافکنم به همه رخت و پخت خویش
\\
ای حافظ ار مراد میسر شدی مدام
&&
جمشید نیز دور نماندی ز تخت خویش
\\
\end{longtable}
\end{center}
