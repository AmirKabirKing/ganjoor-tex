\begin{center}
\section*{غزل شماره ۳۳۲: مزن بر دل ز نوک غمزه تیرم}
\label{sec:sh332}
\addcontentsline{toc}{section}{\nameref{sec:sh332}}
\begin{longtable}{l p{0.5cm} r}
مزن بر دل ز نوک غمزه تیرم
&&
که پیش چشم بیمارت بمیرم
\\
نصاب حسن در حد کمال است
&&
زکاتم ده که مسکین و فقیرم
\\
چو طفلان تا کی ای زاهد فریبی
&&
به سیب بوستان و شهد و شیرم
\\
چنان پر شد فضای سینه از دوست
&&
که فکر خویش گم شد از ضمیرم
\\
قدح پر کن که من در دولت عشق
&&
جوان بخت جهانم گر چه پیرم
\\
قراری بسته‌ام با می فروشان
&&
که روز غم به جز ساغر نگیرم
\\
مبادا جز حساب مطرب و می
&&
اگر نقشی کشد کلک دبیرم
\\
در این غوغا که کس کس را نپرسد
&&
من از پیر مغان منت پذیرم
\\
خوشا آن دم کز استغنای مستی
&&
فراغت باشد از شاه و وزیرم
\\
من آن مرغم که هر شام و سحرگاه
&&
ز بام عرش می‌آید صفیرم
\\
چو حافظ گنج او در سینه دارم
&&
اگر چه مدعی بیند حقیرم
\\
\end{longtable}
\end{center}
