\begin{center}
\section*{غزل شماره ۹۹: دل من در هوای روی فرخ}
\label{sec:sh099}
\addcontentsline{toc}{section}{\nameref{sec:sh099}}
\begin{longtable}{l p{0.5cm} r}
دل من در هوای روی فرخ
&&
بود آشفته همچون موی فرخ
\\
بجز هندوی زلفش هیچ کس نیست
&&
که برخوردار شد از روی فرخ
\\
سیاهی نیکبخت است آن که دایم
&&
بود همراز و هم زانوی فرخ
\\
شود چون بید لرزان سرو آزاد
&&
اگر بیند قد دلجوی فرخ
\\
بده ساقی شراب ارغوانی
&&
به یاد نرگس جادوی فرخ
\\
دوتا شد قامتم همچون کمانی
&&
ز غم پیوسته چون ابروی فرخ
\\
نسیم مشک تاتاری خجل کرد
&&
شمیم زلف عنبربوی فرخ
\\
اگر میل دل هر کس به جایست
&&
بود میل دل من سوی فرخ
\\
غلام همت آنم که باشد
&&
چو حافظ بنده و هندوی فرخ
\\
\end{longtable}
\end{center}
