\begin{center}
\section*{غزل شماره ۴۶۱: کتبت قصه شوقی و مدمعی باکی}
\label{sec:sh461}
\addcontentsline{toc}{section}{\nameref{sec:sh461}}
\begin{longtable}{l p{0.5cm} r}
کتبت قصة شوقی و مدمعی باکی
&&
بیا که بی تو به جان آمدم ز غمناکی
\\
بسا که گفته‌ام از شوق با دو دیده خود
&&
ایا منازل سلمی فاین سلماک
\\
عجیب واقعه‌ای و غریب حادثه‌ای
&&
انا اصطبرت قتیلا و قاتلی شاکی
\\
که را رسد که کند عیب دامن پاکت
&&
که همچو قطره که بر برگ گل چکد پاکی
\\
ز خاک پای تو داد آب روی لاله و گل
&&
چو کلک صنع رقم زد به آبی و خاکی
\\
صبا عبیرفشان گشت ساقیا برخیز
&&
و هات شمسة کرم مطیب زاکی
\\
دع التکاسل تغنم فقد جری مثل
&&
که زاد راهروان چستی است و چالاکی
\\
اثر نماند ز من بی شمایلت آری
&&
اری مآثر محیای من محیاک
\\
ز وصف حسن تو حافظ چگونه نطق زند
&&
که همچو صنع خدایی ورای ادراکی
\\
\end{longtable}
\end{center}
