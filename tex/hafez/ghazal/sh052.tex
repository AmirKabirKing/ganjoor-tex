\begin{center}
\section*{غزل شماره ۵۲}
\label{sec:sh052}
\addcontentsline{toc}{section}{\nameref{sec:sh052}}
\begin{longtable}{l p{0.5cm} r}
روزگاریست که سودای بتان دین من است
&&
غم این کار نشاط دل غمگین من است
\\
دیدن روی تو را دیده جان بین باید
&&
وین کجا مرتبه چشم جهان بین من است
\\
یار من باش که زیب فلک و زینت دهر
&&
از مه روی تو و اشک چو پروین من است
\\
تا مرا عشق تو تعلیم سخن گفتن کرد
&&
خلق را ورد زبان مدحت و تحسین من است
\\
دولت فقر خدایا به من ارزانی دار
&&
کاین کرامت سبب حشمت و تمکین من است
\\
واعظ شحنه شناس این عظمت گو مفروش
&&
زان که منزلگه سلطان دل مسکین من است
\\
یا رب این کعبه مقصود تماشاگه کیست
&&
که مغیلان طریقش گل و نسرین من است
\\
حافظ از حشمت پرویز دگر قصه مخوان
&&
که لبش جرعه کش خسرو شیرین من است
\\
\end{longtable}
\end{center}
