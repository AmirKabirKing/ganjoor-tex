\begin{center}
\section*{غزل شماره ۲۰۰: دانی که چنگ و عود چه تقریر می‌کنند}
\label{sec:sh200}
\addcontentsline{toc}{section}{\nameref{sec:sh200}}
\begin{longtable}{l p{0.5cm} r}
دانی که چنگ و عود چه تقریر می‌کنند
&&
پنهان خورید باده که تعزیر می‌کنند
\\
ناموس عشق و رونق عشاق می‌برند
&&
عیب جوان و سرزنش پیر می‌کنند
\\
جز قلب تیره هیچ نشد حاصل و هنوز
&&
باطل در این خیال که اکسیر می‌کنند
\\
گویند رمز عشق مگویید و مشنوید
&&
مشکل حکایتیست که تقریر می‌کنند
\\
ما از برون در شده مغرور صد فریب
&&
تا خود درون پرده چه تدبیر می‌کنند
\\
تشویش وقت پیر مغان می‌دهند باز
&&
این سالکان نگر که چه با پیر می‌کنند
\\
صد ملک دل به نیم نظر می‌توان خرید
&&
خوبان در این معامله تقصیر می‌کنند
\\
قومی به جد و جهد نهادند وصل دوست
&&
قومی دگر حواله به تقدیر می‌کنند
\\
فی الجمله اعتماد مکن بر ثبات دهر
&&
کاین کارخانه‌ایست که تغییر می‌کنند
\\
می خور که شیخ و حافظ و مفتی و محتسب
&&
چون نیک بنگری همه تزویر می‌کنند
\\
\end{longtable}
\end{center}
