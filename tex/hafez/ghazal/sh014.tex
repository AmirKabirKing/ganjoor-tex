\begin{center}
\section*{غزل شماره ۱۴}
\label{sec:sh014}
\addcontentsline{toc}{section}{\nameref{sec:sh014}}
\begin{longtable}{l p{0.5cm} r}
گفتم ای سلطان خوبان رحم کن بر این غریب
&&
گفت در دنبال دل ره گم کند مسکین غریب
\\
گفتمش مگذر زمانی گفت معذورم بدار
&&
خانه پروردی چه تاب آرد غم چندین غریب
\\
خفته بر سنجاب شاهی نازنینی را چه غم
&&
گر ز خار و خاره سازد بستر و بالین غریب
\\
ای که در زنجیر زلفت جای چندین آشناست
&&
خوش فتاد آن خال مشکین بر رخ رنگین غریب
\\
می‌نماید عکس می در رنگ روی مه وشت
&&
همچو برگ ارغوان بر صفحه نسرین غریب
\\
بس غریب افتاده است آن مور خط گرد رخت
&&
گر چه نبود در نگارستان خط مشکین غریب
\\
گفتم ای شام غریبان طره شبرنگ تو
&&
در سحرگاهان حذر کن چون بنالد این غریب
\\
گفت حافظ آشنایان در مقام حیرتند
&&
دور نبود گر نشیند خسته و مسکین غریب
\\
\end{longtable}
\end{center}
