\begin{center}
\section*{غزل شماره ۵۹}
\label{sec:sh059}
\addcontentsline{toc}{section}{\nameref{sec:sh059}}
\begin{longtable}{l p{0.5cm} r}
دارم امید عاطفتی از جناب دوست
&&
کردم جنایتی و امیدم به عفو اوست
\\
دانم که بگذرد ز سر جرم من که او
&&
گر چه پریوش است ولیکن فرشته خوست
\\
چندان گریستیم که هر کس که برگذشت
&&
در اشک ما چو دید روان گفت کاین چه جوست
\\
هیچ است آن دهان و نبینم از او نشان
&&
موی است آن میان و ندانم که آن چه موست
\\
دارم عجب ز نقش خیالش که چون نرفت
&&
از دیده‌ام که دم به دمش کار شست و شوست
\\
بی گفت و گوی زلف تو دل را همی‌کشد
&&
با زلف دلکش تو که را روی گفت و گوست
\\
عمریست تا ز زلف تو بویی شنیده‌ام
&&
زان بوی در مشام دل من هنوز بوست
\\
حافظ بد است حال پریشان تو ولی
&&
بر بوی زلف یار پریشانیت نکوست
\\
\end{longtable}
\end{center}
