\begin{center}
\section*{غزل شماره ۴۴۸: ای که در کوی خرابات مقامی داری}
\label{sec:sh448}
\addcontentsline{toc}{section}{\nameref{sec:sh448}}
\begin{longtable}{l p{0.5cm} r}
ای که در کوی خرابات مقامی داری
&&
جم وقت خودی ار دست به جامی داری
\\
ای که با زلف و رخ یار گذاری شب و روز
&&
فرصتت باد که خوش صبحی و شامی داری
\\
ای صبا سوختگان بر سر ره منتظرند
&&
گر از آن یار سفرکرده پیامی داری
\\
خال سرسبز تو خوش دانه عیشیست ولی
&&
بر کنار چمنش وه که چه دامی داری
\\
بوی جان از لب خندان قدح می‌شنوم
&&
بشنو ای خواجه اگر زان که مشامی داری
\\
چون به هنگام وفا هیچ ثباتیت نبود
&&
می‌کنم شکر که بر جور دوامی داری
\\
نام نیک ار طلبد از تو غریبی چه شود
&&
تویی امروز در این شهر که نامی داری
\\
بس دعای سحرت مونس جان خواهد بود
&&
تو که چون حافظ شبخیز غلامی داری
\\
\end{longtable}
\end{center}
