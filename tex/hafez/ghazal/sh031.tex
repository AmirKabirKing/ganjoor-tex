\begin{center}
\section*{غزل شماره ۳۱}
\label{sec:sh031}
\addcontentsline{toc}{section}{\nameref{sec:sh031}}
\begin{longtable}{l p{0.5cm} r}
آن شب قدری که گویند اهل خلوت امشب است
&&
یا رب این تأثیر دولت در کدامین کوکب است
\\
تا به گیسوی تو دست ناسزایان کم رسد
&&
هر دلی از حلقه‌ای در ذکر یارب یارب است
\\
کشته چاه زنخدان توام کز هر طرف
&&
صد هزارش گردن جان زیر طوق غبغب است
\\
شهسوار من که مه آیینه دار روی اوست
&&
تاج خورشید بلندش خاک نعل مرکب است
\\
عکس خوی بر عارضش بین کآفتاب گرم رو
&&
در هوای آن عرق تا هست هر روزش تب است
\\
من نخواهم کرد ترک لعل یار و جام می
&&
زاهدان معذور داریدم که اینم مذهب است
\\
اندر آن ساعت که بر پشت صبا بندند زین
&&
با سلیمان چون برانم من که مورم مرکب است
\\
آن که ناوک بر دل من زیر چشمی می‌زند
&&
قوت جان حافظش در خنده زیر لب است
\\
آب حیوانش ز منقار بلاغت می‌چکد
&&
زاغ کلک من به نام ایزد چه عالی مشرب است
\\
\end{longtable}
\end{center}
