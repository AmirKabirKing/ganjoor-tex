\begin{center}
\section*{غزل شماره ۴۵۴: ز کوی یار می‌آید نسیم باد نوروزی}
\label{sec:sh454}
\addcontentsline{toc}{section}{\nameref{sec:sh454}}
\begin{longtable}{l p{0.5cm} r}
ز کوی یار می‌آید نسیم باد نوروزی
&&
از این باد ار مدد خواهی چراغ دل برافروزی
\\
چو گل گر خرده‌ای داری خدا را صرف عشرت کن
&&
که قارون را غلط‌ها داد سودای زراندوزی
\\
ز جام گل دگر بلبل چنان مست می لعل است
&&
که زد بر چرخ فیروزه صفیر تخت فیروزی
\\
به صحرا رو که از دامن غبار غم بیفشانی
&&
به گلزار آی کز بلبل غزل گفتن بیاموزی
\\
چو امکان خلود ای دل در این فیروزه ایوان نیست
&&
مجال عیش فرصت دان به فیروزی و بهروزی
\\
طریق کام بخشی چیست ترک کام خود کردن
&&
کلاه سروری آن است کز این ترک بردوزی
\\
سخن در پرده می‌گویم چو گل از غنچه بیرون آی
&&
که بیش از پنج روزی نیست حکم میر نوروزی
\\
ندانم نوحه قمری به طرف جویباران چیست
&&
مگر او نیز همچون من غمی دارد شبانروزی
\\
می‌ای دارم چو جان صافی و صوفی می‌کند عیبش
&&
خدایا هیچ عاقل را مبادا بخت بد روزی
\\
جدا شد یار شیرینت کنون تنها نشین ای شمع
&&
که حکم آسمان این است اگر سازی و گر سوزی
\\
به عجب علم نتوان شد ز اسباب طرب محروم
&&
بیا ساقی که جاهل را هنیتر می‌رسد روزی
\\
می اندر مجلس آصف به نوروز جلالی نوش
&&
که بخشد جرعه جامت جهان را ساز نوروزی
\\
نه حافظ می‌کند تنها دعای خواجه تورانشاه
&&
ز مدح آصفی خواهد جهان عیدی و نوروزی
\\
جنابش پارسایان راست محراب دل و دیده
&&
جبینش صبح خیزان راست روز فتح و فیروزی
\\
\end{longtable}
\end{center}
