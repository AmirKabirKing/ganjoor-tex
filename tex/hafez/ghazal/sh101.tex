\begin{center}
\section*{غزل شماره ۱۰۱}
\label{sec:sh101}
\addcontentsline{toc}{section}{\nameref{sec:sh101}}
\begin{longtable}{l p{0.5cm} r}
شراب و عیش نهان چیست کار بی‌بنیاد
&&
زدیم بر صف رندان و هر چه بادا باد
\\
گره ز دل بگشا و از سپهر یاد مکن
&&
که فکر هیچ مهندس چنین گره نگشاد
\\
ز انقلاب زمانه عجب مدار که چرخ
&&
از این فسانه هزاران هزار دارد یاد
\\
قدح به شرط ادب گیر زان که ترکیبش
&&
ز کاسه سر جمشید و بهمن است و قباد
\\
که آگه است که کاووس و کی کجا رفتند
&&
که واقف است که چون رفت تخت جم بر باد
\\
ز حسرت لب شیرین هنوز می‌بینم
&&
که لاله می‌دمد از خون دیده فرهاد
\\
مگر که لاله بدانست بی‌وفایی دهر
&&
که تا بزاد و بشد جام می ز کف ننهاد
\\
بیا بیا که زمانی ز می خراب شویم
&&
مگر رسیم به گنجی در این خراب آباد
\\
نمی‌دهند اجازت مرا به سیر سفر
&&
نسیم باد مصلا و آب رکن آباد
\\
قدح مگیر چو حافظ مگر به ناله چنگ
&&
که بسته‌اند بر ابریشم طرب دل شاد
\\
\end{longtable}
\end{center}
