\begin{center}
\section*{غزل شماره ۱۹۳: در نظربازی ما بی‌خبران حیرانند}
\label{sec:sh193}
\addcontentsline{toc}{section}{\nameref{sec:sh193}}
\begin{longtable}{l p{0.5cm} r}
در نظربازی ما بی‌خبران حیرانند
&&
من چنینم که نمودم دگر ایشان دانند
\\
عاقلان نقطه پرگار وجودند ولی
&&
عشق داند که در این دایره سرگردانند
\\
جلوه گاه رخ او دیده من تنها نیست
&&
ماه و خورشید همین آینه می‌گردانند
\\
عهد ما با لب شیرین دهنان بست خدا
&&
ما همه بنده و این قوم خداوندانند
\\
مفلسانیم و هوای می و مطرب داریم
&&
آه اگر خرقه پشمین به گرو نستانند
\\
وصل خورشید به شبپره اعمی نرسد
&&
که در آن آینه صاحب نظران حیرانند
\\
لاف عشق و گله از یار زهی لاف دروغ
&&
عشقبازان چنین مستحق هجرانند
\\
مگرم چشم سیاه تو بیاموزد کار
&&
ور نه مستوری و مستی همه کس نتوانند
\\
گر به نزهتگه ارواح برد بوی تو باد
&&
عقل و جان گوهر هستی به نثار افشانند
\\
زاهد ار رندی حافظ نکند فهم چه شد
&&
دیو بگریزد از آن قوم که قرآن خوانند
\\
گر شوند آگه از اندیشه ما مغبچگان
&&
بعد از این خرقه صوفی به گرو نستانند
\\
\end{longtable}
\end{center}
