\begin{center}
\section*{غزل شماره ۴۸۶: بلبل ز شاخ سرو به گلبانگ پهلوی}
\label{sec:sh486}
\addcontentsline{toc}{section}{\nameref{sec:sh486}}
\begin{longtable}{l p{0.5cm} r}
بلبل ز شاخ سرو به گلبانگ پهلوی
&&
می‌خواند دوش درس مقامات معنوی
\\
یعنی بیا که آتش موسی نمود گل
&&
تا از درخت نکته توحید بشنوی
\\
مرغان باغ قافیه سنجند و بذله گوی
&&
تا خواجه می خورد به غزل‌های پهلوی
\\
جمشید جز حکایت جام از جهان نبرد
&&
زنهار دل مبند بر اسباب دنیوی
\\
این قصه عجب شنو از بخت واژگون
&&
ما را بکشت یار به انفاس عیسوی
\\
خوش وقت بوریا و گدایی و خواب امن
&&
کاین عیش نیست درخور اورنگ خسروی
\\
چشمت به غمزه خانه مردم خراب کرد
&&
مخموریت مباد که خوش مست می‌روی
\\
دهقان سالخورده چه خوش گفت با پسر
&&
کای نور چشم من به جز از کشته ندروی
\\
ساقی مگر وظیفه حافظ زیاده داد
&&
کاشفته گشت طره دستار مولوی
\\
\end{longtable}
\end{center}
