\begin{center}
\section*{غزل شماره ۲۰۱: شراب بی‌غش و ساقی خوش دو دام رهند}
\label{sec:sh201}
\addcontentsline{toc}{section}{\nameref{sec:sh201}}
\begin{longtable}{l p{0.5cm} r}
شراب بی‌غش و ساقی خوش دو دام رهند
&&
که زیرکان جهان از کمندشان نرهند
\\
من ار چه عاشقم و رند و مست و نامه سیاه
&&
هزار شکر که یاران شهر بی‌گنهند
\\
جفا نه پیشه درویشیست و راهروی
&&
بیار باده که این سالکان نه مرد رهند
\\
مبین حقیر گدایان عشق را کاین قوم
&&
شهان بی کمر و خسروان بی کلهند
\\
به هوش باش که هنگام باد استغنا
&&
هزار خرمن طاعت به نیم جو ننهند
\\
مکن که کوکبه دلبری شکسته شود
&&
چو بندگان بگریزند و چاکران بجهند
\\
غلام همت دردی کشان یک رنگم
&&
نه آن گروه که ازرق لباس و دل سیهند
\\
قدم منه به خرابات جز به شرط ادب
&&
که سالکان درش محرمان پادشهند
\\
جناب عشق بلند است همتی حافظ
&&
که عاشقان ره بی‌همتان به خود ندهند
\\
\end{longtable}
\end{center}
