\begin{center}
\section*{غزل شماره ۳۷۸}
\label{sec:sh378}
\addcontentsline{toc}{section}{\nameref{sec:sh378}}
\begin{longtable}{l p{0.5cm} r}
ما نگوییم بد و میل به ناحق نکنیم
&&
جامه کس سیه و دلق خود ازرق نکنیم
\\
عیب درویش و توانگر به کم و بیش بد است
&&
کار بد مصلحت آن است که مطلق نکنیم
\\
رقم مغلطه بر دفتر دانش نزنیم
&&
سر حق بر ورق شعبده ملحق نکنیم
\\
شاه اگر جرعه رندان نه به حرمت نوشد
&&
التفاتش به می صاف مروق نکنیم
\\
خوش برانیم جهان در نظر راهروان
&&
فکر اسب سیه و زین مغرق نکنیم
\\
آسمان کشتی ارباب هنر می‌شکند
&&
تکیه آن به که بر این بحر معلق نکنیم
\\
گر بدی گفت حسودی و رفیقی رنجید
&&
گو تو خوش باش که ما گوش به احمق نکنیم
\\
حافظ ار خصم خطا گفت نگیریم بر او
&&
ور به حق گفت جدل با سخن حق نکنیم
\\
\end{longtable}
\end{center}
