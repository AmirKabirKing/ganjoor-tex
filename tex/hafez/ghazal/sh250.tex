\begin{center}
\section*{غزل شماره ۲۵۰}
\label{sec:sh250}
\addcontentsline{toc}{section}{\nameref{sec:sh250}}
\begin{longtable}{l p{0.5cm} r}
روی بنمای و وجود خودم از یاد ببر
&&
خرمن سوختگان را همه گو باد ببر
\\
ما چو دادیم دل و دیده به طوفان بلا
&&
گو بیا سیل غم و خانه ز بنیاد ببر
\\
زلف چون عنبر خامش که ببوید هیهات
&&
ای دل خام طمع این سخن از یاد ببر
\\
سینه گو شعله آتشکده فارس بکش
&&
دیده گو آب رخ دجله بغداد ببر
\\
دولت پیر مغان باد که باقی سهل است
&&
دیگری گو برو و نام من از یاد ببر
\\
سعی نابرده در این راه به جایی نرسی
&&
مزد اگر می‌طلبی طاعت استاد ببر
\\
روز مرگم نفسی وعده دیدار بده
&&
وان گهم تا به لحد فارغ و آزاد ببر
\\
دوش می‌گفت به مژگان درازت بکشم
&&
یا رب از خاطرش اندیشه بیداد ببر
\\
حافظ اندیشه کن از نازکی خاطر یار
&&
برو از درگهش این ناله و فریاد ببر
\\
\end{longtable}
\end{center}
