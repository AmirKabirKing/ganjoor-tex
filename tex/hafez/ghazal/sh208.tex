\begin{center}
\section*{غزل شماره ۲۰۸: خستگان را چو طلب باشد و قوت نبود}
\label{sec:sh208}
\addcontentsline{toc}{section}{\nameref{sec:sh208}}
\begin{longtable}{l p{0.5cm} r}
خستگان را چو طلب باشد و قوت نبود
&&
گر تو بیداد کنی شرط مروت نبود
\\
ما جفا از تو ندیدیم و تو خود نپسندی
&&
آنچه در مذهب ارباب طریقت نبود
\\
خیره آن دیده که آبش نبرد گریه عشق
&&
تیره آن دل که در او شمع محبت نبود
\\
دولت از مرغ همایون طلب و سایه او
&&
زان که با زاغ و زغن شهپر دولت نبود
\\
گر مدد خواستم از پیر مغان عیب مکن
&&
شیخ ما گفت که در صومعه همت نبود
\\
چون طهارت نبود کعبه و بتخانه یکیست
&&
نبود خیر در آن خانه که عصمت نبود
\\
حافظا علم و ادب ورز که در مجلس شاه
&&
هر که را نیست ادب لایق صحبت نبود
\\
\end{longtable}
\end{center}
