\begin{center}
\section*{غزل شماره ۴۶۲: یا مبسما یحاکی درجا من اللالی}
\label{sec:sh462}
\addcontentsline{toc}{section}{\nameref{sec:sh462}}
\begin{longtable}{l p{0.5cm} r}
یا مبسما یحاکی درجا من اللالی
&&
یا رب چه درخور آمد گردش خط هلالی
\\
حالی خیال وصلت خوش می‌دهد فریبم
&&
تا خود چه نقش بازد این صورت خیالی
\\
می ده که گر چه گشتم نامه سیاه عالم
&&
نومید کی توان بود از لطف لایزالی
\\
ساقی بیار جامی و از خلوتم برون کش
&&
تا در به در بگردم قلاش و لاابالی
\\
از چار چیز مگذر گر عاقلی و زیرک
&&
امن و شراب بی‌غش معشوق و جای خالی
\\
چون نیست نقش دوران در هیچ حال ثابت
&&
حافظ مکن شکایت تا می خوریم حالی
\\
صافیست جام خاطر در دور آصف عهد
&&
قم فاسقنی رحیقا اصفی من الزلال
\\
الملک قد تباهی من جده و جده
&&
یا رب که جاودان باد این قدر و این معالی
\\
مسندفروز دولت کان شکوه و شوکت
&&
برهان ملک و ملت بونصر بوالمعالی
\\
\end{longtable}
\end{center}
