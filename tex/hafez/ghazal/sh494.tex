\begin{center}
\section*{غزل شماره ۴۹۴: ای دل گر از آن چاه زنخدان به درآیی}
\label{sec:sh494}
\addcontentsline{toc}{section}{\nameref{sec:sh494}}
\begin{longtable}{l p{0.5cm} r}
ای دل گر از آن چاه زنخدان به درآیی
&&
هر جا که روی زود پشیمان به درآیی
\\
هش دار که گر وسوسه عقل کنی گوش
&&
آدم صفت از روضه رضوان به درآیی
\\
شاید که به آبی فلکت دست نگیرد
&&
گر تشنه لب از چشمه حیوان به درآیی
\\
جان می‌دهم از حسرت دیدار تو چون صبح
&&
باشد که چو خورشید درخشان به درآیی
\\
چندان چو صبا بر تو گمارم دم همت
&&
کز غنچه چو گل خرم و خندان به درآیی
\\
در تیره شب هجر تو جانم به لب آمد
&&
وقت است که همچون مه تابان به درآیی
\\
بر رهگذرت بسته‌ام از دیده دو صد جوی
&&
تا بو که تو چون سرو خرامان به درآیی
\\
حافظ مکن اندیشه که آن یوسف مه رو
&&
بازآید و از کلبه احزان به درآیی
\\
\end{longtable}
\end{center}
