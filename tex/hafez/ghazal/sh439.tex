\begin{center}
\section*{غزل شماره ۴۳۹: دیدم به خواب دوش که ماهی برآمدی}
\label{sec:sh439}
\addcontentsline{toc}{section}{\nameref{sec:sh439}}
\begin{longtable}{l p{0.5cm} r}
دیدم به خواب دوش که ماهی برآمدی
&&
کز عکس روی او شب هجران سر آمدی
\\
تعبیر رفت یار سفرکرده می‌رسد
&&
ای کاج هر چه زودتر از در درآمدی
\\
ذکرش به خیر ساقی فرخنده فال من
&&
کز در مدام با قدح و ساغر آمدی
\\
خوش بودی ار به خواب بدیدی دیار خویش
&&
تا یاد صحبتش سوی ما رهبر آمدی
\\
فیض ازل به زور و زر ار آمدی به دست
&&
آب خضر نصیبه اسکندر آمدی
\\
آن عهد یاد باد که از بام و در مرا
&&
هر دم پیام یار و خط دلبر آمدی
\\
کی یافتی رقیب تو چندین مجال ظلم
&&
مظلومی ار شبی به در داور آمدی
\\
خامان ره نرفته چه دانند ذوق عشق
&&
دریادلی بجوی دلیری سرآمدی
\\
آن کو تو را به سنگ دلی کرد رهنمون
&&
ای کاشکی که پاش به سنگی برآمدی
\\
گر دیگری به شیوه حافظ زدی رقم
&&
مقبول طبع شاه هنرپرور آمدی
\\
\end{longtable}
\end{center}
