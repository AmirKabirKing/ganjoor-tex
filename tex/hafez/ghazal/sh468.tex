\begin{center}
\section*{غزل شماره ۴۶۸}
\label{sec:sh468}
\addcontentsline{toc}{section}{\nameref{sec:sh468}}
\begin{longtable}{l p{0.5cm} r}
که برد به نزد شاهان ز من گدا پیامی
&&
که به کوی می فروشان دو هزار جم به جامی
\\
شده‌ام خراب و بدنام و هنوز امیدوارم
&&
که به همت عزیزان برسم به نیک نامی
\\
تو که کیمیافروشی نظری به قلب ما کن
&&
که بضاعتی نداریم و فکنده‌ایم دامی
\\
عجب از وفای جانان که عنایتی نفرمود
&&
نه به نامه‌ای پیامی نه به خامه‌ای سلامی
\\
اگر این شراب خام است اگر آن حریف پخته
&&
به هزار بار بهتر ز هزار پخته خامی
\\
ز رهم میفکن ای شیخ به دانه‌های تسبیح
&&
که چو مرغ زیرک افتد نفتد به هیچ دامی
\\
سر خدمت تو دارم بخرم به لطف و مفروش
&&
که چو بنده کمتر افتد به مبارکی غلامی
\\
به کجا برم شکایت به که گویم این حکایت
&&
که لبت حیات ما بود و نداشتی دوامی
\\
بگشای تیر مژگان و بریز خون حافظ
&&
که چنان کشنده‌ای را نکند کس انتقامی
\\
\end{longtable}
\end{center}
