\begin{center}
\section*{غزل شماره ۲۲۱: چو دست بر سر زلفش زنم به تاب رود}
\label{sec:sh221}
\addcontentsline{toc}{section}{\nameref{sec:sh221}}
\begin{longtable}{l p{0.5cm} r}
چو دست بر سر زلفش زنم به تاب رود
&&
ور آشتی طلبم با سر عتاب رود
\\
چو ماه نو ره بیچارگان نظاره
&&
زند به گوشه ابرو و در نقاب رود
\\
شب شراب خرابم کند به بیداری
&&
وگر به روز شکایت کنم به خواب رود
\\
طریق عشق پرآشوب و فتنه است ای دل
&&
بیفتد آن که در این راه با شتاب رود
\\
گدایی در جانان به سلطنت مفروش
&&
کسی ز سایه این در به آفتاب رود
\\
سواد نامه موی سیاه چون طی شد
&&
بیاض کم نشود گر صد انتخاب رود
\\
حباب را چو فتد باد نخوت اندر سر
&&
کلاه داریش اندر سر شراب رود
\\
حجاب راه تویی حافظ از میان برخیز
&&
خوشا کسی که در این راه بی‌حجاب رود
\\
\end{longtable}
\end{center}
