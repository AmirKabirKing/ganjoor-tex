\begin{center}
\section*{غزل شماره ۳۱۲}
\label{sec:sh312}
\addcontentsline{toc}{section}{\nameref{sec:sh312}}
\begin{longtable}{l p{0.5cm} r}
بُشری اِذِ السّلامةُ حَلَّت بِذی سَلَم
&&
للهِ حمدُ مُعتَرِفٍ غایةَ النِّعَم
\\
آن خوش خبر کجاست که این فتح مژده داد
&&
تا جان فشانمش چو زر و سیم در قدم
\\
از بازگشت شاه در این طرفه منزل است
&&
آهنگ خصم او به سراپردهٔ عدم
\\
پیمان شکن هرآینه گردد شکسته حال
&&
انَّ العُهودَ عِندَ مَلیکِ النُّهی ذِمَم
\\
می‌جست از سحاب امل رحمتی ولی
&&
جز دیده‌اش معاینه بیرون نداد نم
\\
در نیل غم فتاد سپهرش به طنز گفت
&&
الآنَ قَد نَدِمتَ و ما یَنفَعُ النَّدَم
\\
ساقی چو یار مه رخ و از اهل راز بود
&&
حافظ بخورد باده و شیخ و فقیه هم
\\
\end{longtable}
\end{center}
