\begin{center}
\section*{غزل شماره ۴۸۸: سحرم هاتف میخانه به دولتخواهی}
\label{sec:sh488}
\addcontentsline{toc}{section}{\nameref{sec:sh488}}
\begin{longtable}{l p{0.5cm} r}
سحرم هاتف میخانه به دولتخواهی
&&
گفت بازآی که دیرینه این درگاهی
\\
همچو جم جرعه ما کش که ز سر دو جهان
&&
پرتو جام جهان بین دهدت آگاهی
\\
بر در میکده رندان قلندر باشند
&&
که ستانند و دهند افسر شاهنشاهی
\\
خشت زیر سر و بر تارک هفت اختر پای
&&
دست قدرت نگر و منصب صاحب جاهی
\\
سر ما و در میخانه که طرف بامش
&&
به فلک بر شد و دیوار بدین کوتاهی
\\
قطع این مرحله بی همرهی خضر مکن
&&
ظلمات است بترس از خطر گمراهی
\\
اگرت سلطنت فقر ببخشند ای دل
&&
کمترین ملک تو از ماه بود تا ماهی
\\
تو دم فقر ندانی زدن از دست مده
&&
مسند خواجگی و مجلس تورانشاهی
\\
حافظ خام طمع شرمی از این قصه بدار
&&
عملت چیست که فردوس برین می‌خواهی
\\
\end{longtable}
\end{center}
