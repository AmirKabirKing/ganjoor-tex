\begin{center}
\section*{غزل شماره ۲۲۵: ساقی حدیث سرو و گل و لاله می‌رود}
\label{sec:sh225}
\addcontentsline{toc}{section}{\nameref{sec:sh225}}
\begin{longtable}{l p{0.5cm} r}
ساقی حدیث سرو و گل و لاله می‌رود
&&
وین بحث با ثلاثه غساله می‌رود
\\
می ده که نوعروس چمن حد حسن یافت
&&
کار این زمان ز صنعت دلاله می‌رود
\\
شکرشکن شوند همه طوطیان هند
&&
زین قند پارسی که به بنگاله می‌رود
\\
طی مکان ببین و زمان در سلوک شعر
&&
کاین طفل یک شبه ره یک ساله می‌رود
\\
آن چشم جادوانه عابدفریب بین
&&
کش کاروان سحر ز دنباله می‌رود
\\
از ره مرو به عشوه دنیا که این عجوز
&&
مکاره می‌نشیند و محتاله می‌رود
\\
باد بهار می‌وزد از گلستان شاه
&&
وز ژاله باده در قدح لاله می‌رود
\\
حافظ ز شوق مجلس سلطان غیاث دین
&&
غافل مشو که کار تو از ناله می‌رود
\\
\end{longtable}
\end{center}
