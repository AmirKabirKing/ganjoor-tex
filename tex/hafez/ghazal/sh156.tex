\begin{center}
\section*{غزل شماره ۱۵۶: به حسن و خلق و وفا کس به یار ما نرسد}
\label{sec:sh156}
\addcontentsline{toc}{section}{\nameref{sec:sh156}}
\begin{longtable}{l p{0.5cm} r}
به حسن و خلق و وفا کس به یار ما نرسد
&&
تو را در این سخن انکار کار ما نرسد
\\
اگر چه حسن فروشان به جلوه آمده‌اند
&&
کسی به حسن و ملاحت به یار ما نرسد
\\
به حق صحبت دیرین که هیچ محرم راز
&&
به یار یک جهت حق گزار ما نرسد
\\
هزار نقش برآید ز کلک صنع و یکی
&&
به دلپذیری نقش نگار ما نرسد
\\
هزار نقد به بازار کائنات آرند
&&
یکی به سکه صاحب عیار ما نرسد
\\
دریغ قافله عمر کان چنان رفتند
&&
که گردشان به هوای دیار ما نرسد
\\
دلا ز رنج حسودان مرنج و واثق باش
&&
که بد به خاطر امیدوار ما نرسد
\\
چنان بزی که اگر خاک ره شوی کس را
&&
غبار خاطری از رهگذار ما نرسد
\\
بسوخت حافظ و ترسم که شرح قصه او
&&
به سمع پادشه کامگار ما نرسد
\\
\end{longtable}
\end{center}
