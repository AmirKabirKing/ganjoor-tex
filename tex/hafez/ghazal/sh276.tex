\begin{center}
\section*{غزل شماره ۲۷۶}
\label{sec:sh276}
\addcontentsline{toc}{section}{\nameref{sec:sh276}}
\begin{longtable}{l p{0.5cm} r}
باغبان گر پنج روزی صحبت گل بایدش
&&
بر جفای خار هجران صبر بلبل بایدش
\\
ای دل اندر بند زلفش از پریشانی منال
&&
مرغ زیرک چون به دام افتد تحمل بایدش
\\
رند عالم سوز را با مصلحت بینی چه کار
&&
کار ملک است آن که تدبیر و تأمل بایدش
\\
تکیه بر تقوی و دانش در طریقت کافریست
&&
راهرو گر صد هنر دارد توکل بایدش
\\
با چنین زلف و رخش بادا نظربازی حرام
&&
هر که روی یاسمین و جعد سنبل بایدش
\\
نازها زان نرگس مستانه‌اش باید کشید
&&
این دل شوریده تا آن جعد و کاکل بایدش
\\
ساقیا در گردش ساغر تعلل تا به چند
&&
دور چون با عاشقان افتد تسلسل بایدش
\\
کیست حافظ تا ننوشد باده بی آواز رود
&&
عاشق مسکین چرا چندین تجمل بایدش
\\
\end{longtable}
\end{center}
