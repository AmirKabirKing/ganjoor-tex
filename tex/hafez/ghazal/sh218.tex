\begin{center}
\section*{غزل شماره ۲۱۸: در ازل هر کو به فیض دولت ارزانی بود}
\label{sec:sh218}
\addcontentsline{toc}{section}{\nameref{sec:sh218}}
\begin{longtable}{l p{0.5cm} r}
در ازل هر کو به فیض دولت ارزانی بود
&&
تا ابد جام مرادش همدم جانی بود
\\
من همان ساعت که از می خواستم شد توبه کار
&&
گفتم این شاخ ار دهد باری پشیمانی بود
\\
خود گرفتم کافکنم سجاده چون سوسن به دوش
&&
همچو گل بر خرقه رنگ می مسلمانی بود
\\
بی چراغ جام در خلوت نمی‌یارم نشست
&&
زان که کنج اهل دل باید که نورانی بود
\\
همت عالی طلب جام مرصع گو مباش
&&
رند را آب عنب یاقوت رمانی بود
\\
گر چه بی‌سامان نماید کار ما سهلش مبین
&&
کاندر این کشور گدایی رشک سلطانی بود
\\
نیک نامی خواهی ای دل با بدان صحبت مدار
&&
خودپسندی جان من برهان نادانی بود
\\
مجلس انس و بهار و بحث شعر اندر میان
&&
نستدن جام می از جانان گران جانی بود
\\
دی عزیزی گفت حافظ می‌خورد پنهان شراب
&&
ای عزیز من نه عیب آن به که پنهانی بود
\\
\end{longtable}
\end{center}
