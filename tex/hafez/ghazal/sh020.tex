\begin{center}
\section*{غزل شماره ۲۰: روزه یک سو شد و عید آمد و دل‌ها برخاست}
\label{sec:sh020}
\addcontentsline{toc}{section}{\nameref{sec:sh020}}
\begin{longtable}{l p{0.5cm} r}
روزه یک سو شد و عید آمد و دل‌ها برخاست
&&
می ز خمخانه به جوش آمد و می باید خواست
\\
نوبه زهدفروشان گران جان بگذشت
&&
وقت رندی و طرب کردن رندان پیداست
\\
چه ملامت بود آن را که چنین باده خورد
&&
این چه عیب است بدین بی‌خردی وین چه خطاست
\\
باده نوشی که در او روی و ریایی نبود
&&
بهتر از زهدفروشی که در او روی و ریاست
\\
ما نه رندان ریاییم و حریفان نفاق
&&
آن که او عالم سر است بدین حال گواست
\\
فرض ایزد بگزاریم و به کس بد نکنیم
&&
وان چه گویند روا نیست نگوییم رواست
\\
چه شود گر من و تو چند قدح باده خوریم
&&
باده از خون رزان است نه از خون شماست
\\
این چه عیب است کز آن عیب خلل خواهد بود
&&
ور بود نیز چه شد مردم بی‌عیب کجاست
\\
\end{longtable}
\end{center}
