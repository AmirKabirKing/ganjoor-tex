\begin{center}
\section*{غزل شماره ۴۳۴}
\label{sec:sh434}
\addcontentsline{toc}{section}{\nameref{sec:sh434}}
\begin{longtable}{l p{0.5cm} r}
ای دل مباش یک دم خالی ز عشق و مستی
&&
وان گه برو که رستی از نیستی و هستی
\\
گر جان به تن ببینی مشغول کار او شو
&&
هر قبله‌ای که بینی بهتر ز خودپرستی
\\
با ضعف و ناتوانی همچون نسیم خوش باش
&&
بیماری اندر این ره بهتر ز تندرستی
\\
در مذهب طریقت خامی نشان کفر است
&&
آری طریق دولت چالاکی است و چستی
\\
تا فضل و عقل بینی بی‌معرفت نشینی
&&
یک نکته‌ات بگویم خود را مبین که رستی
\\
در آستان جانان از آسمان میندیش
&&
کز اوج سربلندی افتی به خاک پستی
\\
خار ار چه جان بکاهد گل عذر آن بخواهد
&&
سهل است تلخی می در جنب ذوق مستی
\\
صوفی پیاله پیما حافظ قرابه پرهیز
&&
ای کوته آستینان تا کی درازدستی
\\
\end{longtable}
\end{center}
