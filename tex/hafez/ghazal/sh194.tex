\begin{center}
\section*{غزل شماره ۱۹۴: سمن بویان غبار غم چو بنشینند بنشانند}
\label{sec:sh194}
\addcontentsline{toc}{section}{\nameref{sec:sh194}}
\begin{longtable}{l p{0.5cm} r}
سمن بویان غبار غم چو بنشینند بنشانند
&&
پری رویان قرار از دل چو بستیزند بستانند
\\
به فتراک جفا دل‌ها چو بربندند بربندند
&&
ز زلف عنبرین جان‌ها چو بگشایند بفشانند
\\
به عمری یک نفس با ما چو بنشینند برخیزند
&&
نهال شوق در خاطر چو برخیزند بنشانند
\\
سرشک گوشه گیران را چو دریابند در یابند
&&
رخ مهر از سحرخیزان نگردانند اگر دانند
\\
ز چشمم لعل رمانی چو می‌خندند می‌بارند
&&
ز رویم راز پنهانی چو می‌بینند می‌خوانند
\\
دوای درد عاشق را کسی کو سهل پندارد
&&
ز فکر آنان که در تدبیر درمانند در مانند
\\
چو منصور از مراد آنان که بردارند بر دارند
&&
بدین درگاه حافظ را چو می‌خوانند می‌رانند
\\
در این حضرت چو مشتاقان نیاز آرند ناز آرند
&&
که با این درد اگر دربند درمانند درمانند
\\
\end{longtable}
\end{center}
