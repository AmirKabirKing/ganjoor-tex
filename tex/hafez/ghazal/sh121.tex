\begin{center}
\section*{غزل شماره ۱۲۱}
\label{sec:sh121}
\addcontentsline{toc}{section}{\nameref{sec:sh121}}
\begin{longtable}{l p{0.5cm} r}
هر آن کو خاطر مجموع و یار نازنین دارد
&&
سعادت همدم او گشت و دولت همنشین دارد
\\
حریم عشق را درگه بسی بالاتر از عقل است
&&
کسی آن آستان بوسد که جان در آستین دارد
\\
دهان تنگ شیرینش مگر ملک سلیمان است
&&
که نقش خاتم لعلش جهان زیر نگین دارد
\\
لب لعل و خط مشکین چو آنش هست و اینش هست
&&
بنازم دلبر خود را که حسنش آن و این دارد
\\
به خواری منگر ای منعم ضعیفان و نحیفان را
&&
که صدر مجلس عشرت گدای رهنشین دارد
\\
چو بر روی زمین باشی توانایی غنیمت دان
&&
که دوران ناتوانی‌ها بسی زیر زمین دارد
\\
بلاگردان جان و تن دعای مستمندان است
&&
که بیند خیر از آن خرمن که ننگ از خوشه چین دارد
\\
صبا از عشق من رمزی بگو با آن شه خوبان
&&
که صد جمشید و کیخسرو غلام کمترین دارد
\\
و گر گوید نمی‌خواهم چو حافظ عاشق مفلس
&&
بگوییدش که سلطانی گدایی همنشین دارد
\\
\end{longtable}
\end{center}
