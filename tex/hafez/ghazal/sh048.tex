\begin{center}
\section*{غزل شماره ۴۸: صوفی از پرتو می راز نهانی دانست}
\label{sec:sh048}
\addcontentsline{toc}{section}{\nameref{sec:sh048}}
\begin{longtable}{l p{0.5cm} r}
صوفی از پرتو می راز نهانی دانست
&&
گوهر هر کس از این لعل توانی دانست
\\
قدر مجموعه گل مرغ سحر داند و بس
&&
که نه هر کو ورقی خواند معانی دانست
\\
عرضه کردم دو جهان بر دل کارافتاده
&&
بجز از عشق تو باقی همه فانی دانست
\\
آن شد اکنون که ز ابنای عوام اندیشم
&&
محتسب نیز در این عیش نهانی دانست
\\
دلبر آسایش ما مصلحت وقت ندید
&&
ور نه از جانب ما دل نگرانی دانست
\\
سنگ و گل را کند از یمن نظر لعل و عقیق
&&
هر که قدر نفس باد یمانی دانست
\\
ای که از دفتر عقل آیت عشق آموزی
&&
ترسم این نکته به تحقیق ندانی دانست
\\
می بیاور که ننازد به گل باغ جهان
&&
هر که غارتگری باد خزانی دانست
\\
حافظ این گوهر منظوم که از طبع انگیخت
&&
ز اثر تربیت آصف ثانی دانست
\\
\end{longtable}
\end{center}
