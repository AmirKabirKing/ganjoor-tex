\begin{center}
\section*{غزل شماره ۴۵۸: ای دل آن دم که خراب از می گلگون باشی}
\label{sec:sh458}
\addcontentsline{toc}{section}{\nameref{sec:sh458}}
\begin{longtable}{l p{0.5cm} r}
ای دل آن دم که خراب از می گلگون باشی
&&
بی زر و گنج به صد حشمت قارون باشی
\\
در مقامی که صدارت به فقیران بخشند
&&
چشم دارم که به جاه از همه افزون باشی
\\
در ره منزل لیلی که خطرهاست در آن
&&
شرط اول قدم آن است که مجنون باشی
\\
نقطه عشق نمودم به تو هان سهو مکن
&&
ور نه چون بنگری از دایره بیرون باشی
\\
کاروان رفت و تو در خواب و بیابان در پیش
&&
کی روی ره ز که پرسی چه کنی چون باشی
\\
تاج شاهی طلبی گوهر ذاتی بنمای
&&
ور خود از تخمه جمشید و فریدون باشی
\\
ساغری نوش کن و جرعه بر افلاک فشان
&&
چند و چند از غم ایام جگرخون باشی
\\
حافظ از فقر مکن ناله که گر شعر این است
&&
هیچ خوشدل نپسندد که تو محزون باشی
\\
\end{longtable}
\end{center}
