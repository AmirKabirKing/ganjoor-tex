\begin{center}
\section*{غزل شماره ۴۹۲}
\label{sec:sh492}
\addcontentsline{toc}{section}{\nameref{sec:sh492}}
\begin{longtable}{l p{0.5cm} r}
سلامی چو بوی خوش آشنایی
&&
بدان مردم دیده روشنایی
\\
درودی چو نور دل پارسایان
&&
بدان شمع خلوتگه پارسایی
\\
نمی‌بینم از همدمان هیچ بر جای
&&
دلم خون شد از غصه ساقی کجایی
\\
ز کوی مغان رخ مگردان که آن جا
&&
فروشند مفتاح مشکل گشایی
\\
عروس جهان گر چه در حد حسن است
&&
ز حد می‌برد شیوه بی‌وفایی
\\
دل خسته من گرش همتی هست
&&
نخواهد ز سنگین دلان مومیایی
\\
می صوفی افکن کجا می‌فروشند
&&
که در تابم از دست زهد ریایی
\\
رفیقان چنان عهد صحبت شکستند
&&
که گویی نبوده‌ست خود آشنایی
\\
مرا گر تو بگذاری ای نفس طامع
&&
بسی پادشایی کنم در گدایی
\\
بیاموزمت کیمیای سعادت
&&
ز همصحبت بد جدایی جدایی
\\
مکن حافظ از جور دوران شکایت
&&
چه دانی تو ای بنده کار خدایی
\\
\end{longtable}
\end{center}
