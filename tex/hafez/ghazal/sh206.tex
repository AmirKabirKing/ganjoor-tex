\begin{center}
\section*{غزل شماره ۲۰۶: پیش از اینت بیش از این اندیشه عشاق بود}
\label{sec:sh206}
\addcontentsline{toc}{section}{\nameref{sec:sh206}}
\begin{longtable}{l p{0.5cm} r}
پیش از اینت بیش از این اندیشه عشاق بود
&&
مهرورزی تو با ما شهره آفاق بود
\\
یاد باد آن صحبت شب‌ها که با نوشین لبان
&&
بحث سر عشق و ذکر حلقه عشاق بود
\\
پیش از این کاین سقف سبز و طاق مینا برکشند
&&
منظر چشم مرا ابروی جانان طاق بود
\\
از دم صبح ازل تا آخر شام ابد
&&
دوستی و مهر بر یک عهد و یک میثاق بود
\\
سایه معشوق اگر افتاد بر عاشق چه شد
&&
ما به او محتاج بودیم او به ما مشتاق بود
\\
حسن مه رویان مجلس گر چه دل می‌برد و دین
&&
بحث ما در لطف طبع و خوبی اخلاق بود
\\
بر در شاهم گدایی نکته‌ای در کار کرد
&&
گفت بر هر خوان که بنشستم خدا رزاق بود
\\
رشته تسبیح اگر بگسست معذورم بدار
&&
دستم اندر دامن ساقی سیمین ساق بود
\\
در شب قدر ار صبوحی کرده‌ام عیبم مکن
&&
سرخوش آمد یار و جامی بر کنار طاق بود
\\
شعر حافظ در زمان آدم اندر باغ خلد
&&
دفتر نسرین و گل را زینت اوراق بود
\\
\end{longtable}
\end{center}
