\begin{center}
\section*{غزل شماره ۴۶۹: انت روائح رند الحمی و زاد غرامی}
\label{sec:sh469}
\addcontentsline{toc}{section}{\nameref{sec:sh469}}
\begin{longtable}{l p{0.5cm} r}
اتت روائح رند الحمی و زاد غرامی
&&
فدای خاک در دوست باد جان گرامی
\\
پیام دوست شنیدن سعادت است و سلامت
&&
من المبلغ عنی الی سعاد سلامی
\\
بیا به شام غریبان و آب دیده من بین
&&
به سان باده صافی در آبگینه شامی
\\
اذا تغرد عن ذی الاراک طائر خیر
&&
فلا تفرد عن روضها انین حمامی
\\
بسی نماند که روز فراق یار سر آید
&&
رایت من هضبات الحمی قباب خیام
\\
خوشا دمی که درآیی و گویمت به سلامت
&&
قدمت خیر قدوم نزلت خیر مقام
\\
بعدت منک و قد صرت ذائبا کهلال
&&
اگر چه روی چو ماهت ندیده‌ام به تمامی
\\
و ان دعیت بخلد و صرت ناقض عهد
&&
فما تطیب نفسی و ما استطاب منامی
\\
امید هست که زودت به بخت نیک ببینم
&&
تو شاد گشته به فرماندهی و من به غلامی
\\
چو سلک در خوشاب است شعر نغز تو حافظ
&&
که گاه لطف سبق می‌برد ز نظم نظامی
\\
\end{longtable}
\end{center}
