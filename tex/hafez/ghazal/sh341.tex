\begin{center}
\section*{غزل شماره ۳۴۱: گر من از سرزنش مدعیان اندیشم}
\label{sec:sh341}
\addcontentsline{toc}{section}{\nameref{sec:sh341}}
\begin{longtable}{l p{0.5cm} r}
گر من از سرزنش مدعیان اندیشم
&&
شیوه مستی و رندی نرود از پیشم
\\
زهد رندان نوآموخته راهی بدهیست
&&
من که بدنام جهانم چه صلاح اندیشم
\\
شاه شوریده سران خوان من بی‌سامان را
&&
زان که در کم خردی از همه عالم بیشم
\\
بر جبین نقش کن از خون دل من خالی
&&
تا بدانند که قربان تو کافرکیشم
\\
اعتقادی بنما و بگذر بهر خدا
&&
تا در این خرقه ندانی که چه نادرویشم
\\
شعر خونبار من ای باد بدان یار رسان
&&
که ز مژگان سیه بر رگ جان زد نیشم
\\
من اگر باده خورم ور نه چه کارم با کس
&&
حافظ راز خود و عارف وقت خویشم
\\
\end{longtable}
\end{center}
