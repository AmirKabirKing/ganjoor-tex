\begin{center}
\section*{غزل شماره ۴۹}
\label{sec:sh049}
\addcontentsline{toc}{section}{\nameref{sec:sh049}}
\begin{longtable}{l p{0.5cm} r}
روضه خلد برین خلوت درویشان است
&&
مایه محتشمی خدمت درویشان است
\\
گنج عزلت که طلسمات عجایب دارد
&&
فتح آن در نظر رحمت درویشان است
\\
قصر فردوس که رضوانش به دربانی رفت
&&
منظری از چمن نزهت درویشان است
\\
آن چه زر می‌شود از پرتو آن قلب سیاه
&&
کیمیاییست که در صحبت درویشان است
\\
آن که پیشش بنهد تاج تکبر خورشید
&&
کبریاییست که در حشمت درویشان است
\\
دولتی را که نباشد غم از آسیب زوال
&&
بی تکلف بشنو دولت درویشان است
\\
خسروان قبله حاجات جهانند ولی
&&
سببش بندگی حضرت درویشان است
\\
روی مقصود که شاهان به دعا می‌طلبند
&&
مظهرش آینه طلعت درویشان است
\\
از کران تا به کران لشکر ظلم است ولی
&&
از ازل تا به ابد فرصت درویشان است
\\
ای توانگر مفروش این همه نخوت که تو را
&&
سر و زر در کنف همت درویشان است
\\
گنج قارون که فرو می‌شود از قهر هنوز
&&
خوانده باشی که هم از غیرت درویشان است
\\
حافظ ار آب حیات ازلی می‌خواهی
&&
منبعش خاک در خلوت درویشان است
\\
من غلام نظر آصف عهدم کو را
&&
صورت خواجگی و سیرت درویشان است
\\
\end{longtable}
\end{center}
