\begin{center}
\section*{غزل شماره ۱۴۳: سال‌ها دل طلب جام جم از ما می‌کرد}
\label{sec:sh143}
\addcontentsline{toc}{section}{\nameref{sec:sh143}}
\begin{longtable}{l p{0.5cm} r}
سال‌ها دل طلب جام جم از ما می‌کرد
&&
وان چه خود داشت ز بیگانه تمنا می‌کرد
\\
گوهری کز صدف کون و مکان بیرون است
&&
طلب از گمشدگان لب دریا می‌کرد
\\
مشکل خویش بر پیر مغان بردم دوش
&&
کو به تأیید نظر حل معما می‌کرد
\\
دیدمش خرم و خندان قدح باده به دست
&&
و اندر آن آینه صد گونه تماشا می‌کرد
\\
گفتم این جام جهان بین به تو کی داد حکیم
&&
گفت آن روز که این گنبد مینا می‌کرد
\\
بی دلی در همه احوال خدا با او بود
&&
او نمی‌دیدش و از دور خدا را می‌کرد
\\
این همه شعبده خویش که می‌کرد این جا
&&
سامری پیش عصا و ید بیضا می‌کرد
\\
گفت آن یار کز او گشت سر دار بلند
&&
جرمش این بود که اسرار هویدا می‌کرد
\\
فیض روح القدس ار باز مدد فرماید
&&
دیگران هم بکنند آن چه مسیحا می‌کرد
\\
گفتمش سلسله زلف بتان از پی چیست
&&
گفت حافظ گله‌ای از دل شیدا می‌کرد
\\
\end{longtable}
\end{center}
