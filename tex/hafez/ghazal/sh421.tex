\begin{center}
\section*{غزل شماره ۴۲۱: در سرای مغان رفته بود و آب زده}
\label{sec:sh421}
\addcontentsline{toc}{section}{\nameref{sec:sh421}}
\begin{longtable}{l p{0.5cm} r}
در سرای مغان رفته بود و آب زده
&&
نشسته پیر و صلایی به شیخ و شاب زده
\\
سبوکشان همه در بندگیش بسته کمر
&&
ولی ز ترک کله چتر بر سحاب زده
\\
شعاع جام و قدح نور ماه پوشیده
&&
عذار مغبچگان راه آفتاب زده
\\
عروس بخت در آن حجله با هزاران ناز
&&
شکسته کسمه و بر برگ گل گلاب زده
\\
گرفته ساغر عشرت فرشته رحمت
&&
ز جرعه بر رخ حور و پری گلاب زده
\\
ز شور و عربده شاهدان شیرین کار
&&
شکر شکسته سمن ریخته رباب زده
\\
سلام کردم و با من به روی خندان گفت
&&
که ای خمارکش مفلس شراب زده
\\
که این کند که تو کردی به ضعف همت و رای
&&
ز گنج خانه شده خیمه بر خراب زده
\\
وصال دولت بیدار ترسمت ندهند
&&
که خفته‌ای تو در آغوش بخت خواب زده
\\
بیا به میکده حافظ که بر تو عرضه کنم
&&
هزار صف ز دعاهای مستجاب زده
\\
فلک جنیبه کش شاه نصره الدین است
&&
بیا ببین ملکش دست در رکاب زده
\\
خرد که ملهم غیب است بهر کسب شرف
&&
ز بام عرش صدش بوسه بر جناب زده
\\
\end{longtable}
\end{center}
