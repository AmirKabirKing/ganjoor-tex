\begin{center}
\section*{غزل شماره ۲۶۲: حال خونین دلان که گوید باز}
\label{sec:sh262}
\addcontentsline{toc}{section}{\nameref{sec:sh262}}
\begin{longtable}{l p{0.5cm} r}
حال خونین دلان که گوید باز
&&
وز فلک خون خم که جوید باز
\\
شرمش از چشم می پرستان باد
&&
نرگس مست اگر بروید باز
\\
جز فلاطون خم نشین شراب
&&
سر حکمت به ما که گوید باز
\\
هر که چون لاله کاسه گردان شد
&&
زین جفا رخ به خون بشوید باز
\\
نگشاید دلم چو غنچه اگر
&&
ساغری از لبش نبوید باز
\\
بس که در پرده چنگ گفت سخن
&&
ببرش موی تا نموید باز
\\
گرد بیت الحرام خم حافظ
&&
گر نمیرد به سر بپوید باز
\\
\end{longtable}
\end{center}
