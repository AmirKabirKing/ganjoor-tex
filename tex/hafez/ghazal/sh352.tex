\begin{center}
\section*{غزل شماره ۳۵۲: روزگاری شد که در میخانه خدمت می‌کنم}
\label{sec:sh352}
\addcontentsline{toc}{section}{\nameref{sec:sh352}}
\begin{longtable}{l p{0.5cm} r}
روزگاری شد که در میخانه خدمت می‌کنم
&&
در لباس فقر کار اهل دولت می‌کنم
\\
تا کی اندر دام وصل آرم تذروی خوش خرام
&&
در کمینم و انتظار وقت فرصت می‌کنم
\\
واعظ ما بوی حق نشنید بشنو کاین سخن
&&
در حضورش نیز می‌گویم نه غیبت می‌کنم
\\
با صبا افتان و خیزان می‌روم تا کوی دوست
&&
و از رفیقان ره استمداد همت می‌کنم
\\
خاک کویت زحمت ما برنتابد بیش از این
&&
لطف‌ها کردی بتا تخفیف زحمت می‌کنم
\\
زلف دلبر دام راه و غمزه‌اش تیر بلاست
&&
یاد دار ای دل که چندینت نصیحت می‌کنم
\\
دیده بدبین بپوشان ای کریم عیب پوش
&&
زین دلیری‌ها که من در کنج خلوت می‌کنم
\\
حافظم در مجلسی دردی کشم در محفلی
&&
بنگر این شوخی که چون با خلق صنعت می‌کنم
\\
\end{longtable}
\end{center}
