\begin{center}
\section*{غزل شماره ۴۷: به کوی میکده هر سالکی که ره دانست}
\label{sec:sh047}
\addcontentsline{toc}{section}{\nameref{sec:sh047}}
\begin{longtable}{l p{0.5cm} r}
به کوی میکده هر سالکی که ره دانست
&&
دری دگر زدن اندیشه تبه دانست
\\
زمانه افسر رندی نداد جز به کسی
&&
که سرفرازی عالم در این کله دانست
\\
بر آستانه میخانه هر که یافت رهی
&&
ز فیض جام می اسرار خانقه دانست
\\
هر آن که راز دو عالم ز خط ساغر خواند
&&
رموز جام جم از نقش خاک ره دانست
\\
ورای طاعت دیوانگان ز ما مطلب
&&
که شیخ مذهب ما عاقلی گنه دانست
\\
دلم ز نرگس ساقی امان نخواست به جان
&&
چرا که شیوه آن ترک دل سیه دانست
\\
ز جور کوکب طالع سحرگهان چشمم
&&
چنان گریست که ناهید دید و مه دانست
\\
حدیث حافظ و ساغر که می‌زند پنهان
&&
چه جای محتسب و شحنه پادشه دانست
\\
بلندمرتبه شاهی که نه رواق سپهر
&&
نمونه‌ای ز خم طاق بارگه دانست
\\
\end{longtable}
\end{center}
