\begin{center}
\section*{غزل شماره ۴۵۹}
\label{sec:sh459}
\addcontentsline{toc}{section}{\nameref{sec:sh459}}
\begin{longtable}{l p{0.5cm} r}
زین خوش رقم که بر گل رخسار می‌کشی
&&
خط بر صحیفه گل و گلزار می‌کشی
\\
اشک حرم نشین نهانخانه مرا
&&
زان سوی هفت پرده به بازار می‌کشی
\\
کاهل روی چو باد صبا را به بوی زلف
&&
هر دم به قید سلسله در کار می‌کشی
\\
هر دم به یاد آن لب میگون و چشم مست
&&
از خلوتم به خانه خمار می‌کشی
\\
گفتی سر تو بسته فتراک ما شود
&&
سهل است اگر تو زحمت این بار می‌کشی
\\
با چشم و ابروی تو چه تدبیر دل کنم
&&
وه زین کمان که بر من بیمار می‌کشی
\\
بازآ که چشم بد ز رخت دفع می‌کند
&&
ای تازه گل که دامن از این خار می‌کشی
\\
حافظ دگر چه می‌طلبی از نعیم دهر
&&
می می‌خوری و طره دلدار می‌کشی
\\
\end{longtable}
\end{center}
