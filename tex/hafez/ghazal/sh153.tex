\begin{center}
\section*{غزل شماره ۱۵۳: سحر چون خسرو خاور علم بر کوهساران زد}
\label{sec:sh153}
\addcontentsline{toc}{section}{\nameref{sec:sh153}}
\begin{longtable}{l p{0.5cm} r}
سحر چون خسرو خاور علم بر کوهساران زد
&&
به دست مرحمت یارم در امیدواران زد
\\
چو پیش صبح روشن شد که حال مهر گردون چیست
&&
برآمد خنده‌ای خوش بر غرور کامگاران زد
\\
نگارم دوش در مجلس به عزم رقص چون برخاست
&&
گره بگشود از ابرو و بر دل‌های یاران زد
\\
من از رنگ صلاح آن دم به خون دل بشستم دست
&&
که چشم باده پیمایش صلا بر هوشیاران زد
\\
کدام آهن دلش آموخت این آیین عیاری
&&
کز اول چون برون آمد ره شب زنده داران زد
\\
خیال شهسواری پخت و شد ناگه دل مسکین
&&
خداوندا نگه دارش که بر قلب سواران زد
\\
در آب و رنگ رخسارش چه جان دادیم و خون خوردیم
&&
چو نقشش دست داد اول رقم بر جان سپاران زد
\\
منش با خرقه پشمین کجا اندر کمند آرم
&&
زره مویی که مژگانش ره خنجرگزاران زد
\\
نظر بر قرعه توفیق و یمن دولت شاه است
&&
بده کام دل حافظ که فال بختیاران زد
\\
شهنشاه مظفر فر شجاع ملک و دین منصور
&&
که جود بی‌دریغش خنده بر ابر بهاران زد
\\
از آن ساعت که جام می به دست او مشرف شد
&&
زمانه ساغر شادی به یاد میگساران زد
\\
ز شمشیر سرافشانش ظفر آن روز بدرخشید
&&
که چون خورشید انجم سوز تنها بر هزاران زد
\\
دوام عمر و ملک او بخواه از لطف حق ای دل
&&
که چرخ این سکه دولت به دور روزگاران زد
\\
\end{longtable}
\end{center}
