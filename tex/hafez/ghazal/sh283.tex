\begin{center}
\section*{غزل شماره ۲۸۳: سحر ز هاتف غیبم رسید مژده به گوش}
\label{sec:sh283}
\addcontentsline{toc}{section}{\nameref{sec:sh283}}
\begin{longtable}{l p{0.5cm} r}
سحر ز هاتف غیبم رسید مژده به گوش
&&
که دور شاه شجاع است می دلیر بنوش
\\
شد آن که اهل نظر بر کناره می‌رفتند
&&
هزار گونه سخن در دهان و لب خاموش
\\
به صوت چنگ بگوییم آن حکایت‌ها
&&
که از نهفتن آن دیگ سینه می‌زد جوش
\\
شراب خانگی ترس محتسب خورده
&&
به روی یار بنوشیم و بانگ نوشانوش
\\
ز کوی میکده دوشش به دوش می‌بردند
&&
امام شهر که سجاده می‌کشید به دوش
\\
دلا دلالت خیرت کنم به راه نجات
&&
مکن به فسق مباهات و زهد هم مفروش
\\
محل نور تجلیست رای انور شاه
&&
چو قرب او طلبی در صفای نیت کوش
\\
به جز ثنای جلالش مساز ورد ضمیر
&&
که هست گوش دلش محرم پیام سروش
\\
رموز مصلحت ملک خسروان دانند
&&
گدای گوشه نشینی تو حافظا مخروش
\\
\end{longtable}
\end{center}
