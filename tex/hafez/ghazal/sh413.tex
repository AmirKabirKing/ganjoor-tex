\begin{center}
\section*{غزل شماره ۴۱۳: خط عذار یار که بگرفت ماه از او}
\label{sec:sh413}
\addcontentsline{toc}{section}{\nameref{sec:sh413}}
\begin{longtable}{l p{0.5cm} r}
خط عذار یار که بگرفت ماه از او
&&
خوش حلقه‌ایست لیک به در نیست راه از او
\\
ابروی دوست گوشه محراب دولت است
&&
آن جا بمال چهره و حاجت بخواه از او
\\
ای جرعه نوش مجلس جم سینه پاک دار
&&
کآیینه‌ایست جام جهان بین که آه از او
\\
کردار اهل صومعه‌ام کرد می پرست
&&
این دود بین که نامه من شد سیاه از او
\\
سلطان غم هر آن چه تواند بگو بکن
&&
من برده‌ام به باده فروشان پناه از او
\\
ساقی چراغ می به ره آفتاب دار
&&
گو برفروز مشعله صبحگاه از او
\\
آبی به روزنامه اعمال ما فشان
&&
باشد توان سترد حروف گناه از او
\\
حافظ که ساز مطرب عشاق ساز کرد
&&
خالی مباد عرصه این بزمگاه از او
\\
آیا در این خیال که دارد گدای شهر
&&
روزی بود که یاد کند پادشاه از او
\\
\end{longtable}
\end{center}
