\begin{center}
\section*{غزل شماره ۱۷۶}
\label{sec:sh176}
\addcontentsline{toc}{section}{\nameref{sec:sh176}}
\begin{longtable}{l p{0.5cm} r}
سحرم دولت بیدار به بالین آمد
&&
گفت برخیز که آن خسرو شیرین آمد
\\
قدحی درکش و سرخوش به تماشا بخرام
&&
تا ببینی که نگارت به چه آیین آمد
\\
مژدگانی بده ای خلوتی نافه گشای
&&
که ز صحرای ختن آهوی مشکین آمد
\\
گریه آبی به رخ سوختگان بازآورد
&&
ناله فریادرس عاشق مسکین آمد
\\
مرغ دل باز هوادار کمان ابروییست
&&
ای کبوتر نگران باش که شاهین آمد
\\
ساقیا می بده و غم مخور از دشمن و دوست
&&
که به کام دل ما آن بشد و این آمد
\\
رسم بدعهدی ایام چو دید ابر بهار
&&
گریه‌اش بر سمن و سنبل و نسرین آمد
\\
چون صبا گفته حافظ بشنید از بلبل
&&
عنبرافشان به تماشای ریاحین آمد
\\
\end{longtable}
\end{center}
