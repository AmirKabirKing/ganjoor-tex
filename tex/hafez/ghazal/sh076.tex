\begin{center}
\section*{غزل شماره ۷۶: جز آستان توام در جهان پناهی نیست}
\label{sec:sh076}
\addcontentsline{toc}{section}{\nameref{sec:sh076}}
\begin{longtable}{l p{0.5cm} r}
جز آستان توام در جهان پناهی نیست
&&
سر مرا به جز این در حواله گاهی نیست
\\
عدو چو تیغ کشد من سپر بیندازم
&&
که تیغ ما به جز از ناله‌ای و آهی نیست
\\
چرا ز کوی خرابات روی برتابم
&&
کز این بهم به جهان هیچ رسم و راهی نیست
\\
زمانه گر بزند آتشم به خرمن عمر
&&
بگو بسوز که بر من به برگ کاهی نیست
\\
غلام نرگس جماش آن سهی سروم
&&
که از شراب غرورش به کس نگاهی نیست
\\
مباش در پی آزار و هر چه خواهی کن
&&
که در شریعت ما غیر از این گناهی نیست
\\
عنان کشیده رو ای پادشاه کشور حسن
&&
که نیست بر سر راهی که دادخواهی نیست
\\
چنین که از همه سو دام راه می‌بینم
&&
به از حمایت زلفش مرا پناهی نیست
\\
خزینه دل حافظ به زلف و خال مده
&&
که کارهای چنین حد هر سیاهی نیست
\\
\end{longtable}
\end{center}
