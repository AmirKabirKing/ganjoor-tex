\begin{center}
\section*{غزل شماره ۴۷۰}
\label{sec:sh470}
\addcontentsline{toc}{section}{\nameref{sec:sh470}}
\begin{longtable}{l p{0.5cm} r}
سینه مالامال درد است ای دریغا مرهمی
&&
دل ز تنهایی به جان آمد خدا را همدمی
\\
چشم آسایش که دارد از سپهر تیزرو
&&
ساقیا جامی به من ده تا بیاسایم دمی
\\
زیرکی را گفتم این احوال بین خندید و گفت
&&
صعب روزی بوالعجب کاری پریشان عالمی
\\
سوختم در چاه صبر از بهر آن شمع چگل
&&
شاه ترکان فارغ است از حال ما کو رستمی
\\
در طریق عشقبازی امن و آسایش بلاست
&&
ریش باد آن دل که با درد تو خواهد مرهمی
\\
اهل کام و ناز را در کوی رندی راه نیست
&&
رهروی باید جهان سوزی نه خامی بی‌غمی
\\
آدمی در عالم خاکی نمی‌آید به دست
&&
عالمی دیگر بباید ساخت و از نو آدمی
\\
خیز تا خاطر بدان ترک سمرقندی دهیم
&&
کز نسیمش بوی جوی مولیان آید همی
\\
گریه حافظ چه سنجد پیش استغنای عشق
&&
کاندر این دریا نماید هفت دریا شبنمی
\\
\end{longtable}
\end{center}
