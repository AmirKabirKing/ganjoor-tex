\begin{center}
\section*{غزل شماره ۱۹۰: کلک مشکین تو روزی که ز ما یاد کند}
\label{sec:sh190}
\addcontentsline{toc}{section}{\nameref{sec:sh190}}
\begin{longtable}{l p{0.5cm} r}
کلک مشکین تو روزی که ز ما یاد کند
&&
ببرد اجر دو صد بنده که آزاد کند
\\
قاصد منزل سلمی که سلامت بادش
&&
چه شود گر به سلامی دل ما شاد کند
\\
امتحان کن که بسی گنج مرادت بدهند
&&
گر خرابی چو مرا لطف تو آباد کند
\\
یا رب اندر دل آن خسرو شیرین انداز
&&
که به رحمت گذری بر سر فرهاد کند
\\
شاه را به بود از طاعت صدساله و زهد
&&
قدر یک ساعته عمری که در او داد کند
\\
حالیا عشوه ناز تو ز بنیادم برد
&&
تا دگرباره حکیمانه چه بنیاد کند
\\
گوهر پاک تو از مدحت ما مستغنیست
&&
فکر مشاطه چه با حسن خداداد کند
\\
ره نبردیم به مقصود خود اندر شیراز
&&
خرم آن روز که حافظ ره بغداد کند
\\
\end{longtable}
\end{center}
