\begin{center}
\section*{غزل شماره ۱۸۸: مرا به رندی و عشق آن فضول عیب کند}
\label{sec:sh188}
\addcontentsline{toc}{section}{\nameref{sec:sh188}}
\begin{longtable}{l p{0.5cm} r}
مرا به رندی و عشق آن فضول عیب کند
&&
که اعتراض بر اسرار علم غیب کند
\\
کمال سر محبت ببین نه نقص گناه
&&
که هر که بی‌هنر افتد نظر به عیب کند
\\
ز عطر حور بهشت آن نفس برآید بوی
&&
که خاک میکده ما عبیر جیب کند
\\
چنان زند ره اسلام غمزه ساقی
&&
که اجتناب ز صهبا مگر صهیب کند
\\
کلید گنج سعادت قبول اهل دل است
&&
مباد آن که در این نکته شک و ریب کند
\\
شبان وادی ایمن گهی رسد به مراد
&&
که چند سال به جان خدمت شعیب کند
\\
ز دیده خون بچکاند فسانه حافظ
&&
چو یاد وقت زمان شباب و شیب کند
\\
\end{longtable}
\end{center}
