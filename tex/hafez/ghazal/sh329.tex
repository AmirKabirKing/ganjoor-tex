\begin{center}
\section*{غزل شماره ۳۲۹}
\label{sec:sh329}
\addcontentsline{toc}{section}{\nameref{sec:sh329}}
\begin{longtable}{l p{0.5cm} r}
جوزا سحر نهاد حمایل برابرم
&&
یعنی غلام شاهم و سوگند می‌خورم
\\
ساقی بیا که از مدد بخت کارساز
&&
کامی که خواستم ز خدا شد میسرم
\\
جامی بده که باز به شادی روی شاه
&&
پیرانه سر هوای جوانیست در سرم
\\
راهم مزن به وصف زلال خضر که من
&&
از جام شاه جرعه کش حوض کوثرم
\\
شاها اگر به عرش رسانم سریر فضل
&&
مملوک این جنابم و مسکین این درم
\\
من جرعه نوش بزم تو بودم هزار سال
&&
کی ترک آبخورد کند طبع خوگرم
\\
ور باورت نمی‌کند از بنده این حدیث
&&
از گفته کمال دلیلی بیاورم
\\
گر برکنم دل از تو و بردارم از تو مهر
&&
آن مهر بر که افکنم آن دل کجا برم
\\
منصور بن مظفر غازیست حرز من
&&
و از این خجسته نام بر اعدا مظفرم
\\
عهد الست من همه با عشق شاه بود
&&
و از شاهراه عمر بدین عهد بگذرم
\\
گردون چو کرد نظم ثریا به نام شاه
&&
من نظم در چرا نکنم از که کمترم
\\
شاهین صفت چو طعمه چشیدم ز دست شاه
&&
کی باشد التفات به صید کبوترم
\\
ای شاه شیرگیر چه کم گردد ار شود
&&
در سایه تو ملک فراغت میسرم
\\
شعرم به یمن مدح تو صد ملک دل گشاد
&&
گویی که تیغ توست زبان سخنورم
\\
بر گلشنی اگر بگذشتم چو باد صبح
&&
نی عشق سرو بود و نه شوق صنوبرم
\\
بوی تو می‌شنیدم و بر یاد روی تو
&&
دادند ساقیان طرب یک دو ساغرم
\\
مستی به آب یک دو عنب وضع بنده نیست
&&
من سالخورده پیر خرابات پرورم
\\
با سیر اختر فلکم داوری بسیست
&&
انصاف شاه باد در این قصه یاورم
\\
شکر خدا که باز در این اوج بارگاه
&&
طاووس عرش می‌شنود صیت شهپرم
\\
نامم ز کارخانه عشاق محو باد
&&
گر جز محبت تو بود شغل دیگرم
\\
شبل الاسد به صید دلم حمله کرد و من
&&
گر لاغرم وگرنه شکار غضنفرم
\\
ای عاشقان روی تو از ذره بیشتر
&&
من کی رسم به وصل تو کز ذره کمترم
\\
بنما به من که منکر حسن رخ تو کیست
&&
تا دیده‌اش به گزلک غیرت برآورم
\\
بر من فتاد سایه خورشید سلطنت
&&
و اکنون فراغت است ز خورشید خاورم
\\
مقصود از این معامله بازارتیزی است
&&
نی جلوه می‌فروشم و نی عشوه می‌خرم
\\
\end{longtable}
\end{center}
