\begin{center}
\section*{غزل شماره ۴۲۳}
\label{sec:sh423}
\addcontentsline{toc}{section}{\nameref{sec:sh423}}
\begin{longtable}{l p{0.5cm} r}
دوش رفتم به در میکده خواب آلوده
&&
خرقه تردامن و سجاده شراب آلوده
\\
آمد افسوس کنان مغبچه باده فروش
&&
گفت بیدار شو ای ره رو خواب آلوده
\\
شست و شویی کن و آن گه به خرابات خرام
&&
تا نگردد ز تو این دیر خراب آلوده
\\
به هوای لب شیرین پسران چند کنی
&&
جوهر روح به یاقوت مذاب آلوده
\\
به طهارت گذران منزل پیری و مکن
&&
خلعت شیب چو تشریف شباب آلوده
\\
پاک و صافی شو و از چاه طبیعت به درآی
&&
که صفایی ندهد آب تراب آلوده
\\
گفتم ای جان جهان دفتر گل عیبی نیست
&&
که شود فصل بهار از می ناب آلوده
\\
آشنایان ره عشق در این بحر عمیق
&&
غرقه گشتند و نگشتند به آب آلوده
\\
گفت حافظ لغز و نکته به یاران مفروش
&&
آه از این لطف به انواع عتاب آلوده
\\
\end{longtable}
\end{center}
