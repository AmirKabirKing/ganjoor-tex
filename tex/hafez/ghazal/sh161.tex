\begin{center}
\section*{غزل شماره ۱۶۱: کی شعر تر انگیزد خاطر که حزین باشد}
\label{sec:sh161}
\addcontentsline{toc}{section}{\nameref{sec:sh161}}
\begin{longtable}{l p{0.5cm} r}
کی شعر تر انگیزد خاطر که حزین باشد
&&
یک نکته از این معنی گفتیم و همین باشد
\\
از لعل تو گر یابم انگشتری زنهار
&&
صد ملک سلیمانم در زیر نگین باشد
\\
غمناک نباید بود از طعن حسود ای دل
&&
شاید که چو وابینی خیر تو در این باشد
\\
هر کو نکند فهمی زین کلک خیال انگیز
&&
نقشش به حرام ار خود صورتگر چین باشد
\\
جام می و خون دل هر یک به کسی دادند
&&
در دایره قسمت اوضاع چنین باشد
\\
در کار گلاب و گل حکم ازلی این بود
&&
کاین شاهد بازاری وان پرده نشین باشد
\\
آن نیست که حافظ را رندی بشد از خاطر
&&
کاین سابقه پیشین تا روز پسین باشد
\\
\end{longtable}
\end{center}
