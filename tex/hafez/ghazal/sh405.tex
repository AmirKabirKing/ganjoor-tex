\begin{center}
\section*{غزل شماره ۴۰۵}
\label{sec:sh405}
\addcontentsline{toc}{section}{\nameref{sec:sh405}}
\begin{longtable}{l p{0.5cm} r}
به جان پیر خرابات و حق صحبت او
&&
که نیست در سر من جز هوای خدمت او
\\
بهشت اگر چه نه جای گناهکاران است
&&
بیار باده که مستظهرم به همت او
\\
چراغ صاعقه آن سحاب روشن باد
&&
که زد به خرمن ما آتش محبت او
\\
بر آستانه میخانه گر سری بینی
&&
مزن به پای که معلوم نیست نیت او
\\
بیا که دوش به مستی سروش عالم غیب
&&
نوید داد که عام است فیض رحمت او
\\
مکن به چشم حقارت نگاه در من مست
&&
که نیست معصیت و زهد بی مشیت او
\\
نمی‌کند دل من میل زهد و توبه ولی
&&
به نام خواجه بکوشیم و فر دولت او
\\
مدام خرقه حافظ به باده در گرو است
&&
مگر ز خاک خرابات بود فطرت او
\\
\end{longtable}
\end{center}
