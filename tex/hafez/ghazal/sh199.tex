\begin{center}
\section*{غزل شماره ۱۹۹: واعظان کاین جلوه در محراب و منبر می‌کنند}
\label{sec:sh199}
\addcontentsline{toc}{section}{\nameref{sec:sh199}}
\begin{longtable}{l p{0.5cm} r}
واعظان کاین جلوه در محراب و منبر می‌کنند
&&
چون به خلوت می‌روند آن کار دیگر می‌کنند
\\
مشکلی دارم ز دانشمند مجلس بازپرس
&&
توبه فرمایان چرا خود توبه کمتر می‌کنند
\\
گوییا باور نمی‌دارند روز داوری
&&
کاین همه قلب و دغل در کار داور می‌کنند
\\
یا رب این نودولتان را با خر خودشان نشان
&&
کاین همه ناز از غلام ترک و استر می‌کنند
\\
ای گدای خانقه برجه که در دیر مغان
&&
می‌دهند آبی که دل‌ها را توانگر می‌کنند
\\
حسن بی‌پایان او چندان که عاشق می‌کشد
&&
زمره دیگر به عشق از غیب سر بر می‌کنند
\\
بر در میخانه عشق ای ملک تسبیح گوی
&&
کاندر آن جا طینت آدم مخمر می‌کنند
\\
صبحدم از عرش می‌آمد خروشی عقل گفت
&&
قدسیان گویی که شعر حافظ از بر می‌کنند
\\
\end{longtable}
\end{center}
