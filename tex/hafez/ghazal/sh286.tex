\begin{center}
\section*{غزل شماره ۲۸۶}
\label{sec:sh286}
\addcontentsline{toc}{section}{\nameref{sec:sh286}}
\begin{longtable}{l p{0.5cm} r}
دوش با من گفت پنهان کاردانی تیزهوش
&&
وز شما پنهان نشاید کرد سر می فروش
\\
گفت آسان گیر بر خود کارها کز روی طبع
&&
سخت می‌گردد جهان بر مردمان سخت‌کوش
\\
وان گهم در داد جامی کز فروغش بر فلک
&&
زهره در رقص آمد و بربط زنان می‌گفت نوش
\\
با دل خونین لب خندان بیاور همچو جام
&&
نی گرت زخمی رسد آیی چو چنگ اندر خروش
\\
تا نگردی آشنا زین پرده رمزی نشنوی
&&
گوش نامحرم نباشد جای پیغام سروش
\\
گوش کن پند ای پسر وز بهر دنیا غم مخور
&&
گفتمت چون در حدیثی گر توانی داشت هوش
\\
در حریم عشق نتوان زد دم از گفت و شنید
&&
زان که آنجا جمله اعضا چشم باید بود و گوش
\\
بر بساط نکته دانان خودفروشی شرط نیست
&&
یا سخن دانسته گو ای مرد عاقل یا خموش
\\
ساقیا می ده که رندی‌های حافظ فهم کرد
&&
آصف صاحب قران جرم بخش عیب پوش
\\
\end{longtable}
\end{center}
