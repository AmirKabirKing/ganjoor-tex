\begin{center}
\section*{غزل شماره ۲۴۶}
\label{sec:sh246}
\addcontentsline{toc}{section}{\nameref{sec:sh246}}
\begin{longtable}{l p{0.5cm} r}
عید است و آخر گل و یاران در انتظار
&&
ساقی به روی شاه ببین ماه و می بیار
\\
دل برگرفته بودم از ایام گل ولی
&&
کاری بکرد همت پاکان روزه دار
\\
دل در جهان مبند و به مستی سؤال کن
&&
از فیض جام و قصه جمشید کامگار
\\
جز نقد جان به دست ندارم شراب کو
&&
کان نیز بر کرشمه ساقی کنم نثار
\\
خوش دولتیست خرم و خوش خسروی کریم
&&
یا رب ز چشم زخم زمانش نگاه دار
\\
می خور به شعر بنده که زیبی دگر دهد
&&
جام مرصع تو بدین در شاهوار
\\
گر فوت شد سحور چه نقصان صبوح هست
&&
از می کنند روزه گشا طالبان یار
\\
زانجا که پرده پوشی عفو کریم توست
&&
بر قلب ما ببخش که نقدیست کم عیار
\\
ترسم که روز حشر عنان بر عنان رود
&&
تسبیح شیخ و خرقه رند شرابخوار
\\
حافظ چو رفت روزه و گل نیز می‌رود
&&
ناچار باده نوش که از دست رفت کار
\\
\end{longtable}
\end{center}
