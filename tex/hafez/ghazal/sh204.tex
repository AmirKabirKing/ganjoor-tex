\begin{center}
\section*{غزل شماره ۲۰۴}
\label{sec:sh204}
\addcontentsline{toc}{section}{\nameref{sec:sh204}}
\begin{longtable}{l p{0.5cm} r}
یاد باد آن که نهانت نظری با ما بود
&&
رقم مهر تو بر چهره ما پیدا بود
\\
یاد باد آن که چو چشمت به عتابم می‌کشت
&&
معجز عیسویت در لب شکرخا بود
\\
یاد باد آن که صبوحی زده در مجلس انس
&&
جز من و یار نبودیم و خدا با ما بود
\\
یاد باد آن که رخت شمع طرب می‌افروخت
&&
وین دل سوخته پروانه ناپروا بود
\\
یاد باد آن که در آن بزمگه خلق و ادب
&&
آن که او خنده مستانه زدی صهبا بود
\\
یاد باد آن که چو یاقوت قدح خنده زدی
&&
در میان من و لعل تو حکایت‌ها بود
\\
یاد باد آن که نگارم چو کمر بربستی
&&
در رکابش مه نو پیک جهان پیما بود
\\
یاد باد آن که خرابات نشین بودم و مست
&&
وآنچه در مسجدم امروز کم است آنجا بود
\\
یاد باد آن که به اصلاح شما می‌شد راست
&&
نظم هر گوهر ناسفته که حافظ را بود
\\
\end{longtable}
\end{center}
