\begin{center}
\section*{غزل شماره ۴۶۷: زان می عشق کز او پخته شود هر خامی}
\label{sec:sh467}
\addcontentsline{toc}{section}{\nameref{sec:sh467}}
\begin{longtable}{l p{0.5cm} r}
زان می عشق کز او پخته شود هر خامی
&&
گر چه ماه رمضان است بیاور جامی
\\
روزها رفت که دست من مسکین نگرفت
&&
زلف شمشادقدی ساعد سیم اندامی
\\
روزه هر چند که مهمان عزیز است ای دل
&&
صحبتش موهبتی دان و شدن انعامی
\\
مرغ زیرک به در خانقه اکنون نپرد
&&
که نهاده‌ست به هر مجلس وعظی دامی
\\
گله از زاهد بدخو نکنم رسم این است
&&
که چو صبحی بدمد در پی اش افتد شامی
\\
یار من چون بخرامد به تماشای چمن
&&
برسانش ز من ای پیک صبا پیغامی
\\
آن حریفی که شب و روز می صاف کشد
&&
بود آیا که کند یاد ز دردآشامی
\\
حافظا گر ندهد داد دلت آصف عهد
&&
کام دشوار به دست آوری از خودکامی
\\
\end{longtable}
\end{center}
