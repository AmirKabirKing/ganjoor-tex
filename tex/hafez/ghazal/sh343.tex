\begin{center}
\section*{غزل شماره ۳۴۳: چل سال بیش رفت که من لاف می‌زنم}
\label{sec:sh343}
\addcontentsline{toc}{section}{\nameref{sec:sh343}}
\begin{longtable}{l p{0.5cm} r}
چل سال بیش رفت که من لاف می‌زنم
&&
کز چاکران پیر مغان کمترین منم
\\
هرگز به یمن عاطفت پیر می فروش
&&
ساغر تهی نشد ز می صاف روشنم
\\
از جاه عشق و دولت رندان پاکباز
&&
پیوسته صدر مصطبه‌ها بود مسکنم
\\
در شان من به دردکشی ظن بد مبر
&&
کآلوده گشت جامه ولی پاکدامنم
\\
شهباز دست پادشهم این چه حالت است
&&
کز یاد برده‌اند هوای نشیمنم
\\
حیف است بلبلی چو من اکنون در این قفس
&&
با این لسان عذب که خامش چو سوسنم
\\
آب و هوای فارس عجب سفله پرور است
&&
کو همرهی که خیمه از این خاک برکنم
\\
حافظ به زیر خرقه قدح تا به کی کشی
&&
در بزم خواجه پرده ز کارت برافکنم
\\
تورانشه خجسته که در من یزید فضل
&&
شد منت مواهب او طوق گردنم
\\
\end{longtable}
\end{center}
