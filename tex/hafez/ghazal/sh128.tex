\begin{center}
\section*{غزل شماره ۱۲۸: نیست در شهر نگاری که دل ما ببرد}
\label{sec:sh128}
\addcontentsline{toc}{section}{\nameref{sec:sh128}}
\begin{longtable}{l p{0.5cm} r}
نیست در شهر نگاری که دل ما ببرد
&&
بختم ار یار شود رختم از این جا ببرد
\\
کو حریفی کش سرمست که پیش کرمش
&&
عاشق سوخته دل نام تمنا ببرد
\\
باغبانا ز خزان بی‌خبرت می‌بینم
&&
آه از آن روز که بادت گل رعنا ببرد
\\
رهزن دهر نخفته‌ست مشو ایمن از او
&&
اگر امروز نبرده‌ست که فردا ببرد
\\
در خیال این همه لعبت به هوس می‌بازم
&&
بو که صاحب نظری نام تماشا ببرد
\\
علم و فضلی که به چل سال دلم جمع آورد
&&
ترسم آن نرگس مستانه به یغما ببرد
\\
بانگ گاوی چه صدا بازدهد عشوه مخر
&&
سامری کیست که دست از ید بیضا ببرد
\\
جام مینایی می سد ره تنگ دلیست
&&
منه از دست که سیل غمت از جا ببرد
\\
راه عشق ار چه کمینگاه کمانداران است
&&
هر که دانسته رود صرفه ز اعدا ببرد
\\
حافظ ار جان طلبد غمزه مستانه یار
&&
خانه از غیر بپرداز و بهل تا ببرد
\\
\end{longtable}
\end{center}
