\begin{center}
\section*{غزل شماره ۳۷۴: بیا تا گل برافشانیم و می در ساغر اندازیم}
\label{sec:sh374}
\addcontentsline{toc}{section}{\nameref{sec:sh374}}
\begin{longtable}{l p{0.5cm} r}
بیا تا گل برافشانیم و می در ساغر اندازیم
&&
فلک را سقف بشکافیم و طرحی نو دراندازیم
\\
اگر غم لشکر انگیزد که خون عاشقان ریزد
&&
من و ساقی به هم تازیم و بنیادش براندازیم
\\
شراب ارغوانی را گلاب اندر قدح ریزیم
&&
نسیم عطرگردان را شِکَر در مجمر اندازیم
\\
چو در دست است رودی خوش بزن مطرب سرودی خوش
&&
که دست افشان غزل خوانیم و پاکوبان سر اندازیم
\\
صبا خاک وجود ما بدان عالی جناب انداز
&&
بود کان شاه خوبان را نظر بر منظر اندازیم
\\
یکی از عقل می‌لافد یکی طامات می‌بافد
&&
بیا کاین داوری‌ها را به پیش داور اندازیم
\\
بهشت عدن اگر خواهی بیا با ما به میخانه
&&
که از پای خمت روزی به حوض کوثر اندازیم
\\
سخندانیّ و خوشخوانی نمی‌ورزند در شیراز
&&
بیا حافظ که تا خود را به ملکی دیگر اندازیم
\\
\end{longtable}
\end{center}
