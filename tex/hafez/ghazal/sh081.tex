\begin{center}
\section*{غزل شماره ۸۱: صبحدم مرغ چمن با گل نوخاسته گفت}
\label{sec:sh081}
\addcontentsline{toc}{section}{\nameref{sec:sh081}}
\begin{longtable}{l p{0.5cm} r}
صبحدم مرغ چمن با گل نوخاسته گفت
&&
ناز کم کن که در این باغ بسی چون تو شکفت
\\
گل بخندید که از راست نرنجیم ولی
&&
هیچ عاشق سخن سخت به معشوق نگفت
\\
گر طمع داری از آن جام مرصع می لعل
&&
ای بسا در که به نوک مژه‌ات باید سفت
\\
تا ابد بوی محبت به مشامش نرسد
&&
هر که خاک در میخانه به رخساره نرفت
\\
در گلستان ارم دوش چو از لطف هوا
&&
زلف سنبل به نسیم سحری می‌آشفت
\\
گفتم ای مسند جم جام جهان بینت کو
&&
گفت افسوس که آن دولت بیدار بخفت
\\
سخن عشق نه آن است که آید به زبان
&&
ساقیا می ده و کوتاه کن این گفت و شنفت
\\
اشک حافظ خرد و صبر به دریا انداخت
&&
چه کند سوز غم عشق نیارست نهفت
\\
\end{longtable}
\end{center}
