\begin{center}
\section*{غزل شماره ۷۷}
\label{sec:sh077}
\addcontentsline{toc}{section}{\nameref{sec:sh077}}
\begin{longtable}{l p{0.5cm} r}
بلبلی برگ گلی خوش رنگ در منقار داشت
&&
و اندر آن برگ و نوا خوش ناله‌های زار داشت
\\
گفتمش در عین وصل این ناله و فریاد چیست
&&
گفت ما را جلوه معشوق در این کار داشت
\\
یار اگر ننشست با ما نیست جای اعتراض
&&
پادشاهی کامران بود از گدایی عار داشت
\\
در نمی‌گیرد نیاز و ناز ما با حسن دوست
&&
خرم آن کز نازنینان بخت برخوردار داشت
\\
خیز تا بر کلک آن نقاش جان افشان کنیم
&&
کاین همه نقش عجب در گردش پرگار داشت
\\
گر مرید راه عشقی فکر بدنامی مکن
&&
شیخ صنعان خرقه رهن خانه خمار داشت
\\
وقت آن شیرین قلندر خوش که در اطوار سیر
&&
ذکر تسبیح ملک در حلقه زنار داشت
\\
چشم حافظ زیر بام قصر آن حوری سرشت
&&
شیوه جنات تجری تحتها الانهار داشت
\\
\end{longtable}
\end{center}
