\begin{center}
\section*{غزل شماره ۲۱۲: یک دو جامم دی سحرگه اتفاق افتاده بود}
\label{sec:sh212}
\addcontentsline{toc}{section}{\nameref{sec:sh212}}
\begin{longtable}{l p{0.5cm} r}
یک دو جامم دی سحرگه اتفاق افتاده بود
&&
وز لب ساقی شرابم در مذاق افتاده بود
\\
از سر مستی دگر با شاهد عهد شباب
&&
رجعتی می‌خواستم لیکن طلاق افتاده بود
\\
در مقامات طریقت هر کجا کردیم سیر
&&
عافیت را با نظربازی فراق افتاده بود
\\
ساقیا جام دمادم ده که در سیر طریق
&&
هر که عاشق وش نیامد در نفاق افتاده بود
\\
ای معبر مژده‌ای فرما که دوشم آفتاب
&&
در شکرخواب صبوحی هم وثاق افتاده بود
\\
نقش می‌بستم که گیرم گوشه‌ای زان چشم مست
&&
طاقت و صبر از خم ابروش طاق افتاده بود
\\
گر نکردی نصرت دین شاه یحیی از کرم
&&
کار ملک و دین ز نظم و اتساق افتاده بود
\\
حافظ آن ساعت که این نظم پریشان می‌نوشت
&&
طایر فکرش به دام اشتیاق افتاده بود
\\
\end{longtable}
\end{center}
