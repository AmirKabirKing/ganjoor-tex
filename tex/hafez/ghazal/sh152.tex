\begin{center}
\section*{غزل شماره ۱۵۲: در ازل پرتو حسنت ز تجلی دم زد}
\label{sec:sh152}
\addcontentsline{toc}{section}{\nameref{sec:sh152}}
\begin{longtable}{l p{0.5cm} r}
در ازل پرتو حسنت ز تجلی دم زد
&&
عشق پیدا شد و آتش به همه عالم زد
\\
جلوه‌ای کرد رخت دید ملک عشق نداشت
&&
عین آتش شد از این غیرت و بر آدم زد
\\
عقل می‌خواست کز آن شعله چراغ افروزد
&&
برق غیرت بدرخشید و جهان برهم زد
\\
مدعی خواست که آید به تماشاگه راز
&&
دست غیب آمد و بر سینه نامحرم زد
\\
دیگران قرعه قسمت همه بر عیش زدند
&&
دل غمدیده ما بود که هم بر غم زد
\\
جان علوی هوس چاه زنخدان تو داشت
&&
دست در حلقه آن زلف خم اندر خم زد
\\
حافظ آن روز طربنامه عشق تو نوشت
&&
که قلم بر سر اسباب دل خرم زد
\\
\end{longtable}
\end{center}
