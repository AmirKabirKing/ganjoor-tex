\begin{center}
\section*{غزل شماره ۴۸۹}
\label{sec:sh489}
\addcontentsline{toc}{section}{\nameref{sec:sh489}}
\begin{longtable}{l p{0.5cm} r}
ای در رخ تو پیدا انوار پادشاهی
&&
در فکرت تو پنهان صد حکمت الهی
\\
کلک تو بارک الله بر ملک و دین گشاده
&&
صد چشمه آب حیوان از قطره سیاهی
\\
بر اهرمن نتابد انوار اسم اعظم
&&
ملک آن توست و خاتم فرمای هر چه خواهی
\\
در حکمت سلیمان هر کس که شک نماید
&&
بر عقل و دانش او خندند مرغ و ماهی
\\
باز ار چه گاه گاهی بر سر نهد کلاهی
&&
مرغان قاف دانند آیین پادشاهی
\\
تیغی که آسمانش از فیض خود دهد آب
&&
تنها جهان بگیرد بی منت سپاهی
\\
کلک تو خوش نویسد در شان یار و اغیار
&&
تعویذ جان فزایی افسون عمر کاهی
\\
ای عنصر تو مخلوق از کیمیای عزت
&&
و ای دولت تو ایمن از وصمت تباهی
\\
ساقی بیار آبی از چشمه خرابات
&&
تا خرقه‌ها بشوییم از عجب خانقاهی
\\
عمریست پادشاها کز می تهیست جامم
&&
اینک ز بنده دعوی و از محتسب گواهی
\\
گر پرتوی ز تیغت بر کان و معدن افتد
&&
یاقوت سرخ رو را بخشند رنگ کاهی
\\
دانم دلت ببخشد بر عجز شب نشینان
&&
گر حال بنده پرسی از باد صبحگاهی
\\
جایی که برق عصیان بر آدم صفی زد
&&
ما را چگونه زیبد دعوی بی‌گناهی
\\
حافظ چو پادشاهت گه گاه می‌برد نام
&&
رنجش ز بخت منما بازآ به عذرخواهی
\\
\end{longtable}
\end{center}
