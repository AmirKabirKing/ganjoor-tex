\begin{center}
\section*{غزل شماره ۲۰۳: سال‌ها دفتر ما در گرو صهبا بود}
\label{sec:sh203}
\addcontentsline{toc}{section}{\nameref{sec:sh203}}
\begin{longtable}{l p{0.5cm} r}
سال‌ها دفتر ما در گرو صهبا بود
&&
رونق میکده از درس و دعای ما بود
\\
نیکی پیر مغان بین که چو ما بدمستان
&&
هر چه کردیم به چشم کرمش زیبا بود
\\
دفتر دانش ما جمله بشویید به می
&&
که فلک دیدم و در قصد دل دانا بود
\\
از بتان آن طلب ار حسن شناسی ای دل
&&
کاین کسی گفت که در علم نظر بینا بود
\\
دل چو پرگار به هر سو دورانی می‌کرد
&&
و اندر آن دایره سرگشته پابرجا بود
\\
مطرب از درد محبت عملی می‌پرداخت
&&
که حکیمان جهان را مژه خون پالا بود
\\
می‌شکفتم ز طرب زان که چو گل بر لب جوی
&&
بر سرم سایه آن سرو سهی بالا بود
\\
پیر گلرنگ من اندر حق ازرق پوشان
&&
رخصت خبث نداد ار نه حکایت‌ها بود
\\
قلب اندوده حافظ بر او خرج نشد
&&
کاین معامل به همه عیب نهان بینا بود
\\
\end{longtable}
\end{center}
