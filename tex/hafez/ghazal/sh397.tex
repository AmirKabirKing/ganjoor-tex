\begin{center}
\section*{غزل شماره ۳۹۷}
\label{sec:sh397}
\addcontentsline{toc}{section}{\nameref{sec:sh397}}
\begin{longtable}{l p{0.5cm} r}
ز در درآ و شبستان ما منور کن
&&
هوای مجلس روحانیان معطر کن
\\
اگر فقیه نصیحت کند که عشق مباز
&&
پیاله‌ای بدهش گو دماغ را تر کن
\\
به چشم و ابروی جانان سپرده‌ام دل و جان
&&
بیا بیا و تماشای طاق و منظر کن
\\
ستاره شب هجران نمی‌فشاند نور
&&
به بام قصر برآ و چراغ مه برکن
\\
بگو به خازن جنت که خاک این مجلس
&&
به تحفه بر سوی فردوس و عود مجمر کن
\\
از این مزوجه و خرقه نیک در تنگم
&&
به یک کرشمه صوفی وشم قلندر کن
\\
چو شاهدان چمن زیردست حسن تواند
&&
کرشمه بر سمن و جلوه بر صنوبر کن
\\
فضول نفس حکایت بسی کند ساقی
&&
تو کار خود مده از دست و می به ساغر کن
\\
حجاب دیده ادراک شد شعاع جمال
&&
بیا و خرگه خورشید را منور کن
\\
طمع به قند وصال تو حد ما نبود
&&
حوالتم به لب لعل همچو شکر کن
\\
لب پیاله ببوس آنگهی به مستان ده
&&
بدین دقیقه دماغ معاشران تر کن
\\
پس از ملازمت عیش و عشق مه رویان
&&
ز کارها که کنی شعر حافظ از بر کن
\\
\end{longtable}
\end{center}
