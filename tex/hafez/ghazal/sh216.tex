\begin{center}
\section*{غزل شماره ۲۱۶: آن یار کز او خانه ما جای پری بود}
\label{sec:sh216}
\addcontentsline{toc}{section}{\nameref{sec:sh216}}
\begin{longtable}{l p{0.5cm} r}
آن یار کز او خانه ما جای پری بود
&&
سر تا قدمش چون پری از عیب بری بود
\\
دل گفت فروکش کنم این شهر به بویش
&&
بیچاره ندانست که یارش سفری بود
\\
تنها نه ز راز دل من پرده برافتاد
&&
تا بود فلک شیوه او پرده دری بود
\\
منظور خردمند من آن ماه که او را
&&
با حسن ادب شیوه صاحب نظری بود
\\
از چنگ منش اختر بدمهر به در برد
&&
آری چه کنم دولت دور قمری بود
\\
عذری بنه ای دل که تو درویشی و او را
&&
در مملکت حسن سر تاجوری بود
\\
اوقات خوش آن بود که با دوست به سر رفت
&&
باقی همه بی‌حاصلی و بی‌خبری بود
\\
خوش بود لب آب و گل و سبزه و نسرین
&&
افسوس که آن گنج روان رهگذری بود
\\
خود را بکش ای بلبل از این رشک که گل را
&&
با باد صبا وقت سحر جلوه گری بود
\\
هر گنج سعادت که خدا داد به حافظ
&&
از یمن دعای شب و ورد سحری بود
\\
\end{longtable}
\end{center}
