\begin{center}
\section*{غزل شماره ۴۷۶}
\label{sec:sh476}
\addcontentsline{toc}{section}{\nameref{sec:sh476}}
\begin{longtable}{l p{0.5cm} r}
نسیم صبح سعادت بدان نشان که تو دانی
&&
گذر به کوی فلان کن در آن زمان که تو دانی
\\
تو پیک خلوت رازی و دیده بر سر راهت
&&
به مردمی نه به فرمان چنان بران که تو دانی
\\
بگو که جان عزیزم ز دست رفت خدا را
&&
ز لعل روح فزایش ببخش آن که تو دانی
\\
من این حروف نوشتم چنان که غیر ندانست
&&
تو هم ز روی کرامت چنان بخوان که تو دانی
\\
خیال تیغ تو با ما حدیث تشنه و آب است
&&
اسیر خویش گرفتی بکش چنان که تو دانی
\\
امید در کمر زرکشت چگونه ببندم
&&
دقیقه‌ایست نگارا در آن میان که تو دانی
\\
یکیست ترکی و تازی در این معامله حافظ
&&
حدیث عشق بیان کن بدان زبان که تو دانی
\\
\end{longtable}
\end{center}
