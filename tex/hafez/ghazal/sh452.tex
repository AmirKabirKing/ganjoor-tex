\begin{center}
\section*{غزل شماره ۴۵۲}
\label{sec:sh452}
\addcontentsline{toc}{section}{\nameref{sec:sh452}}
\begin{longtable}{l p{0.5cm} r}
طفیل هستی عشقند آدمی و پری
&&
ارادتی بنما تا سعادتی ببری
\\
بکوش خواجه و از عشق بی‌نصیب مباش
&&
که بنده را نخرد کس به عیب بی‌هنری
\\
می صبوح و شکرخواب صبحدم تا چند
&&
به عذر نیم شبی کوش و گریه سحری
\\
تو خود چه لعبتی ای شهسوار شیرین کار
&&
که در برابر چشمی و غایب از نظری
\\
هزار جان مقدس بسوخت زین غیرت
&&
که هر صباح و مسا شمع مجلس دگری
\\
ز من به حضرت آصف که می‌برد پیغام
&&
که یاد گیر دو مصرع ز من به نظم دری
\\
بیا که وضع جهان را چنان که من دیدم
&&
گر امتحان بکنی می خوری و غم نخوری
\\
کلاه سروریت کج مباد بر سر حسن
&&
که زیب بخت و سزاوار ملک و تاج سری
\\
به بوی زلف و رخت می‌روند و می‌آیند
&&
صبا به غالیه سایی و گل به جلوه گری
\\
چو مستعد نظر نیستی وصال مجوی
&&
که جام جم نکند سود وقت بی‌بصری
\\
دعای گوشه نشینان بلا بگرداند
&&
چرا به گوشه چشمی به ما نمی‌نگری
\\
بیا و سلطنت از ما بخر به مایه حسن
&&
و از این معامله غافل مشو که حیف خوری
\\
طریق عشق طریقی عجب خطرناک است
&&
نعوذبالله اگر ره به مقصدی نبری
\\
به یمن همت حافظ امید هست که باز
&&
اری اسامر لیلای لیله القمر
\\
\end{longtable}
\end{center}
