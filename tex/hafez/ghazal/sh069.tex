\begin{center}
\section*{غزل شماره ۶۹}
\label{sec:sh069}
\addcontentsline{toc}{section}{\nameref{sec:sh069}}
\begin{longtable}{l p{0.5cm} r}
کس نیست که افتاده آن زلف دوتا نیست
&&
در رهگذر کیست که دامی ز بلا نیست
\\
چون چشم تو دل می‌برد از گوشه نشینان
&&
همراه تو بودن گنه از جانب ما نیست
\\
روی تو مگر آینه لطف الهیست
&&
حقا که چنین است و در این روی و ریا نیست
\\
نرگس طلبد شیوه چشم تو زهی چشم
&&
مسکین خبرش از سر و در دیده حیا نیست
\\
از بهر خدا زلف مپیرای که ما را
&&
شب نیست که صد عربده با باد صبا نیست
\\
بازآی که بی روی تو ای شمع دل افروز
&&
در بزم حریفان اثر نور و صفا نیست
\\
تیمار غریبان اثر ذکر جمیل است
&&
جانا مگر این قاعده در شهر شما نیست
\\
دی می‌شد و گفتم صنما عهد به جای آر
&&
گفتا غلطی خواجه در این عهد وفا نیست
\\
گر پیر مغان مرشد من شد چه تفاوت
&&
در هیچ سری نیست که سری ز خدا نیست
\\
عاشق چه کند گر نکشد بار ملامت
&&
با هیچ دلاور سپر تیر قضا نیست
\\
در صومعه زاهد و در خلوت صوفی
&&
جز گوشه ابروی تو محراب دعا نیست
\\
ای چنگ فروبرده به خون دل حافظ
&&
فکرت مگر از غیرت قرآن و خدا نیست
\\
\end{longtable}
\end{center}
