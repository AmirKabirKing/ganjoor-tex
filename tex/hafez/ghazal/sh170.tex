\begin{center}
\section*{غزل شماره ۱۷۰: زاهد خلوت نشین دوش به میخانه شد}
\label{sec:sh170}
\addcontentsline{toc}{section}{\nameref{sec:sh170}}
\begin{longtable}{l p{0.5cm} r}
زاهد خلوت نشین دوش به میخانه شد
&&
از سر پیمان برفت با سر پیمانه شد
\\
صوفی مجلس که دی جام و قدح می‌شکست
&&
باز به یک جرعه می عاقل و فرزانه شد
\\
شاهد عهد شباب آمده بودش به خواب
&&
باز به پیرانه سر عاشق و دیوانه شد
\\
مغبچه‌ای می‌گذشت راهزن دین و دل
&&
در پی آن آشنا از همه بیگانه شد
\\
آتش رخسار گل خرمن بلبل بسوخت
&&
چهره خندان شمع آفت پروانه شد
\\
گریه شام و سحر شکر که ضایع نگشت
&&
قطره باران ما گوهر یک دانه شد
\\
نرگس ساقی بخواند آیت افسونگری
&&
حلقه اوراد ما مجلس افسانه شد
\\
منزل حافظ کنون بارگه پادشاست
&&
دل بر دلدار رفت جان بر جانانه شد
\\
\end{longtable}
\end{center}
