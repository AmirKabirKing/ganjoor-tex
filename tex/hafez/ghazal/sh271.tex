\begin{center}
\section*{غزل شماره ۲۷۱: دارم از زلف سیاهش گله چندان که مپرس}
\label{sec:sh271}
\addcontentsline{toc}{section}{\nameref{sec:sh271}}
\begin{longtable}{l p{0.5cm} r}
دارم از زلف سیاهش گله چندان که مپرس
&&
که چنان ز او شده‌ام بی سر و سامان که مپرس
\\
کس به امید وفا ترک دل و دین مکناد
&&
که چنانم من از این کرده پشیمان که مپرس
\\
به یکی جرعه که آزار کسش در پی نیست
&&
زحمتی می‌کشم از مردم نادان که مپرس
\\
زاهد از ما به سلامت بگذر کاین می لعل
&&
دل و دین می‌برد از دست بدان سان که مپرس
\\
گفت‌وگوهاست در این راه که جان بگدازد
&&
هر کسی عربده‌ای این که مبین آن که مپرس
\\
پارسایی و سلامت هوسم بود ولی
&&
شیوه‌ای می‌کند آن نرگس فتان که مپرس
\\
گفتم از گوی فلک صورت حالی پرسم
&&
گفت آن می‌کشم اندر خم چوگان که مپرس
\\
گفتمش زلف به خون که شکستی گفتا
&&
حافظ این قصه دراز است به قرآن که مپرس
\\
\end{longtable}
\end{center}
