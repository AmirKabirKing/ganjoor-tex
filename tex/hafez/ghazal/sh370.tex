\begin{center}
\section*{غزل شماره ۳۷۰: صلاح از ما چه می‌جویی که مستان را صلا گفتیم}
\label{sec:sh370}
\addcontentsline{toc}{section}{\nameref{sec:sh370}}
\begin{longtable}{l p{0.5cm} r}
صلاح از ما چه می‌جویی که مستان را صلا گفتیم
&&
به دور نرگس مستت سلامت را دعا گفتیم
\\
در میخانه‌ام بگشا که هیچ از خانقه نگشود
&&
گرت باور بود ور نه سخن این بود و ما گفتیم
\\
من از چشم تو ای ساقی خراب افتاده‌ام لیکن
&&
بلایی کز حبیب آید هزارش مرحبا گفتیم
\\
اگر بر من نبخشایی پشیمانی خوری آخر
&&
به خاطر دار این معنی که در خدمت کجا گفتیم
\\
قدت گفتم که شمشاد است بس خجلت به بار آورد
&&
که این نسبت چرا کردیم و این بهتان چرا گفتیم
\\
جگر چون نافه‌ام خون گشت کم زینم نمی‌باید
&&
جزای آن که با زلفت سخن از چین خطا گفتیم
\\
تو آتش گشتی ای حافظ ولی با یار درنگرفت
&&
ز بدعهدی گل گویی حکایت با صبا گفتیم
\\
\end{longtable}
\end{center}
