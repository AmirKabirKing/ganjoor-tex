\begin{center}
\section*{غزل شماره ۴۷۵: گفتند خلایق که تویی یوسف ثانی}
\label{sec:sh475}
\addcontentsline{toc}{section}{\nameref{sec:sh475}}
\begin{longtable}{l p{0.5cm} r}
گفتند خلایق که تویی یوسف ثانی
&&
چون نیک بدیدم به حقیقت به از آنی
\\
شیرینتر از آنی به شکرخنده که گویم
&&
ای خسرو خوبان که تو شیرین زمانی
\\
تشبیه دهانت نتوان کرد به غنچه
&&
هرگز نبود غنچه بدین تنگ دهانی
\\
صد بار بگفتی که دهم زان دهنت کام
&&
چون سوسن آزاده چرا جمله زبانی
\\
گویی بدهم کامت و جانت بستانم
&&
ترسم ندهی کامم و جانم بستانی
\\
چشم تو خدنگ از سپر جان گذراند
&&
بیمار که دیده‌ست بدین سخت کمانی
\\
چون اشک بیندازیش از دیده مردم
&&
آن را که دمی از نظر خویش برانی
\\
\end{longtable}
\end{center}
