\begin{center}
\section*{غزل شماره ۱۹۲}
\label{sec:sh192}
\addcontentsline{toc}{section}{\nameref{sec:sh192}}
\begin{longtable}{l p{0.5cm} r}
سرو چمان من چرا میل چمن نمی‌کند
&&
همدم گل نمی‌شود یاد سمن نمی‌کند
\\
دی گله‌ای ز طره‌اش کردم و از سر فسوس
&&
گفت که این سیاه کج گوش به من نمی‌کند
\\
تا دل هرزه گرد من رفت به چین زلف او
&&
زان سفر دراز خود عزم وطن نمی‌کند
\\
پیش کمان ابرویش لابه همی‌کنم ولی
&&
گوش کشیده است از آن گوش به من نمی‌کند
\\
با همه عطف 
&&
چون ز نسیم می‌شود زلف بنفشه پرشکن
\\
وه که دلم چه یاد از آن عهدشکن نمی‌کند
&&
دل به امید روی او همدم جان نمی‌شود
\\
جان به هوای کوی او خدمت تن نمی‌کند
&&
ساقی سیم ساق من گر همه درد می‌دهد
\\
کیست که تن چو جام می جمله دهن نمی‌کند
&&
دستخوش جفا مکن آب رخم که فیض ابر
\\
بی مدد سرشک من در عدن نمی‌کند
&&
کشته غمزه تو شد حافظ ناشنیده پند
\\
تیغ سزاست هر که را درد سخن نمی‌کند
&&
\end{longtable}
\end{center}
