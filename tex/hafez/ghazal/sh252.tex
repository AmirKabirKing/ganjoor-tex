\begin{center}
\section*{غزل شماره ۲۵۲: گر بود عمر به میخانه رسم بار دگر}
\label{sec:sh252}
\addcontentsline{toc}{section}{\nameref{sec:sh252}}
\begin{longtable}{l p{0.5cm} r}
گر بود عمر به میخانه رسم بار دگر
&&
بجز از خدمت رندان نکنم کار دگر
\\
خرم آن روز که با دیده گریان بروم
&&
تا زنم آب در میکده یک بار دگر
\\
معرفت نیست در این قوم خدا را سببی
&&
تا برم گوهر خود را به خریدار دگر
\\
یار اگر رفت و حق صحبت دیرین نشناخت
&&
حاش لله که روم من ز پی یار دگر
\\
گر مساعد شودم دایره چرخ کبود
&&
هم به دست آورمش باز به پرگار دگر
\\
عافیت می‌طلبد خاطرم ار بگذارند
&&
غمزه شوخش و آن طرهٔ طرار دگر
\\
راز سربسته ما بین که به دستان گفتند
&&
هر زمان با دف و نی بر سر بازار دگر
\\
هر دم از درد بنالم که فلک هر ساعت
&&
کندم قصد دل ریش به آزار دگر
\\
بازگویم نه در این واقعه حافظ تنهاست
&&
غرقه گشتند در این بادیه بسیار دگر
\\
\end{longtable}
\end{center}
