\begin{center}
\section*{غزل شماره ۴۵۰: روزگاریست که ما را نگران می‌داری}
\label{sec:sh450}
\addcontentsline{toc}{section}{\nameref{sec:sh450}}
\begin{longtable}{l p{0.5cm} r}
روزگاریست که ما را نگران می‌داری
&&
مخلصان را نه به وضع دگران می‌داری
\\
گوشه چشم رضایی به منت باز نشد
&&
این چنین عزت صاحب نظران می‌داری
\\
ساعد آن به که بپوشی تو چو از بهر نگار
&&
دست در خون دل پرهنران می‌داری
\\
نه گل از دست غمت رست و نه بلبل در باغ
&&
همه را نعره زنان جامه دران می‌داری
\\
ای که در دلق ملمع طلبی نقد حضور
&&
چشم سری عجب از بی‌خبران می‌داری
\\
چون تویی نرگس باغ نظر ای چشم و چراغ
&&
سر چرا بر من دلخسته گران می‌داری
\\
گوهر جام جم از کان جهانی دگر است
&&
تو تمنا ز گل کوزه گران می‌داری
\\
پدر تجربه ای دل تویی آخر ز چه روی
&&
طمع مهر و وفا زین پسران می‌داری
\\
کیسه سیم و زرت پاک بباید پرداخت
&&
این طمع‌ها که تو از سیمبران می‌داری
\\
گر چه رندی و خرابی گنه ماست ولی
&&
عاشقی گفت که تو بنده بر آن می‌داری
\\
مگذران روز سلامت به ملامت حافظ
&&
چه توقع ز جهان گذران می‌داری
\\
\end{longtable}
\end{center}
