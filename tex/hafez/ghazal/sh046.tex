\begin{center}
\section*{غزل شماره ۴۶}
\label{sec:sh046}
\addcontentsline{toc}{section}{\nameref{sec:sh046}}
\begin{longtable}{l p{0.5cm} r}
گل در بر و می در کف و معشوق به کام است
&&
سلطان جهانم به چنین روز غلام است
\\
گو شمع میارید در این جمع که امشب
&&
در مجلس ما ماه رخ دوست تمام است
\\
در مذهب ما باده حلال است ولیکن
&&
بی روی تو ای سرو گل اندام حرام است
\\
گوشم همه بر قول نی و نغمه چنگ است
&&
چشمم همه بر لعل لب و گردش جام است
\\
در مجلس ما عطر میامیز که ما را
&&
هر لحظه ز گیسوی تو خوش بوی مشام است
\\
از چاشنی قند مگو هیچ و ز شکر
&&
زان رو که مرا از لب شیرین تو کام است
\\
تا گنج غمت در دل ویرانه مقیم است
&&
همواره مرا کوی خرابات مقام است
\\
از ننگ چه گویی که مرا نام ز ننگ است
&&
وز نام چه پرسی که مرا ننگ ز نام است
\\
میخواره و سرگشته و رندیم و نظرباز
&&
وان کس که چو ما نیست در این شهر کدام است
\\
با محتسبم عیب مگویید که او نیز
&&
پیوسته چو ما در طلب عیش مدام است
\\
حافظ منشین بی می و معشوق زمانی
&&
کایام گل و یاسمن و عید صیام است
\\
\end{longtable}
\end{center}
