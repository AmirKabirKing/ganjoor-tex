\begin{center}
\section*{غزل شماره ۳۴}
\label{sec:sh034}
\addcontentsline{toc}{section}{\nameref{sec:sh034}}
\begin{longtable}{l p{0.5cm} r}
رواق منظر چشم من آشیانه توست
&&
کرم نما و فرود آ که خانه خانه توست
\\
به لطف خال و خط از عارفان ربودی دل
&&
لطیفه‌های عجب زیر دام و دانه توست
\\
دلت به وصل گل ای بلبل صبا خوش باد
&&
که در چمن همه گلبانگ عاشقانه توست
\\
علاج ضعف دل ما به لب حوالت کن
&&
که این مفرح یاقوت در خزانه توست
\\
به تن مقصرم از دولت ملازمتت
&&
ولی خلاصه جان خاک آستانه توست
\\
من آن نیم که دهم نقد دل به هر شوخی
&&
در خزانه به مهر تو و نشانه توست
\\
تو خود چه لعبتی ای شهسوار شیرین کار
&&
که توسنی چو فلک رام تازیانه توست
\\
چه جای من که بلغزد سپهر شعبده باز
&&
از این حیل که در انبانه بهانه توست
\\
سرود مجلست اکنون فلک به رقص آرد
&&
که شعر حافظ شیرین سخن ترانه توست
\\
\end{longtable}
\end{center}
